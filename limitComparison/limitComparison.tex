\documentclass{ximera}

%% You can put user macros here
%% However, you cannot make new environments



\newcommand{\ffrac}[2]{\frac{\text{\footnotesize $#1$}}{\text{\footnotesize $#2$}}}
\newcommand{\vasymptote}[2][]{
    \draw [densely dashed,#1] ({rel axis cs:0,0} -| {axis cs:#2,0}) -- ({rel axis cs:0,1} -| {axis cs:#2,0});
}


\graphicspath{{./}{firstExample/}}

\usepackage{amsmath}
\usepackage{amssymb}
\usepackage{array}
\usepackage[makeroom]{cancel} %% for strike outs
\usepackage{pgffor} %% required for integral for loops
\usepackage{tikz}
\usepackage{tikz-cd}
\usepackage{tkz-euclide}
\usetikzlibrary{shapes.multipart}


\usetkzobj{all}
\tikzstyle geometryDiagrams=[ultra thick,color=blue!50!black]


\usetikzlibrary{arrows}
\tikzset{>=stealth,commutative diagrams/.cd,
  arrow style=tikz,diagrams={>=stealth}} %% cool arrow head
\tikzset{shorten <>/.style={ shorten >=#1, shorten <=#1 } } %% allows shorter vectors

\usetikzlibrary{backgrounds} %% for boxes around graphs
\usetikzlibrary{shapes,positioning}  %% Clouds and stars
\usetikzlibrary{matrix} %% for matrix
\usepgfplotslibrary{polar} %% for polar plots
\usepgfplotslibrary{fillbetween} %% to shade area between curves in TikZ



%\usepackage[width=4.375in, height=7.0in, top=1.0in, papersize={5.5in,8.5in}]{geometry}
%\usepackage[pdftex]{graphicx}
%\usepackage{tipa}
%\usepackage{txfonts}
%\usepackage{textcomp}
%\usepackage{amsthm}
%\usepackage{xy}
%\usepackage{fancyhdr}
%\usepackage{xcolor}
%\usepackage{mathtools} %% for pretty underbrace % Breaks Ximera
%\usepackage{multicol}



\newcommand{\RR}{\mathbb R}
\newcommand{\R}{\mathbb R}
\newcommand{\C}{\mathbb C}
\newcommand{\N}{\mathbb N}
\newcommand{\Z}{\mathbb Z}
\newcommand{\dis}{\displaystyle}
%\renewcommand{\d}{\,d\!}
\renewcommand{\d}{\mathop{}\!d}
\newcommand{\dd}[2][]{\frac{\d #1}{\d #2}}
\newcommand{\pp}[2][]{\frac{\partial #1}{\partial #2}}
\renewcommand{\l}{\ell}
\newcommand{\ddx}{\frac{d}{\d x}}

\newcommand{\zeroOverZero}{\ensuremath{\boldsymbol{\tfrac{0}{0}}}}
\newcommand{\inftyOverInfty}{\ensuremath{\boldsymbol{\tfrac{\infty}{\infty}}}}
\newcommand{\zeroOverInfty}{\ensuremath{\boldsymbol{\tfrac{0}{\infty}}}}
\newcommand{\zeroTimesInfty}{\ensuremath{\small\boldsymbol{0\cdot \infty}}}
\newcommand{\inftyMinusInfty}{\ensuremath{\small\boldsymbol{\infty - \infty}}}
\newcommand{\oneToInfty}{\ensuremath{\boldsymbol{1^\infty}}}
\newcommand{\zeroToZero}{\ensuremath{\boldsymbol{0^0}}}
\newcommand{\inftyToZero}{\ensuremath{\boldsymbol{\infty^0}}}


\newcommand{\numOverZero}{\ensuremath{\boldsymbol{\tfrac{\#}{0}}}}
\newcommand{\dfn}{\textbf}
%\newcommand{\unit}{\,\mathrm}
\newcommand{\unit}{\mathop{}\!\mathrm}
%\newcommand{\eval}[1]{\bigg[ #1 \bigg]}
\newcommand{\eval}[1]{ #1 \bigg|}
\newcommand{\seq}[1]{\left( #1 \right)}
\renewcommand{\epsilon}{\varepsilon}
\renewcommand{\iff}{\Leftrightarrow}

\DeclareMathOperator{\arccot}{arccot}
\DeclareMathOperator{\arcsec}{arcsec}
\DeclareMathOperator{\arccsc}{arccsc}
\DeclareMathOperator{\si}{Si}
\DeclareMathOperator{\proj}{proj}
\DeclareMathOperator{\scal}{scal}
\DeclareMathOperator{\cis}{cis}
\DeclareMathOperator{\Arg}{Arg}
%\DeclareMathOperator{\arg}{arg}
\DeclareMathOperator{\Rep}{Re}
\DeclareMathOperator{\Imp}{Im}
\DeclareMathOperator{\sech}{sech}
\DeclareMathOperator{\csch}{csch}
\DeclareMathOperator{\Log}{Log}

\newcommand{\tightoverset}[2]{% for arrow vec
  \mathop{#2}\limits^{\vbox to -.5ex{\kern-0.75ex\hbox{$#1$}\vss}}}
\newcommand{\arrowvec}{\overrightarrow}
\renewcommand{\vec}{\mathbf}
\newcommand{\veci}{{\boldsymbol{\hat{\imath}}}}
\newcommand{\vecj}{{\boldsymbol{\hat{\jmath}}}}
\newcommand{\veck}{{\boldsymbol{\hat{k}}}}
\newcommand{\vecl}{\boldsymbol{\l}}
\newcommand{\utan}{\vec{\hat{t}}}
\newcommand{\unormal}{\vec{\hat{n}}}
\newcommand{\ubinormal}{\vec{\hat{b}}}

\newcommand{\dotp}{\bullet}
\newcommand{\cross}{\boldsymbol\times}
\newcommand{\grad}{\boldsymbol\nabla}
\newcommand{\divergence}{\grad\dotp}
\newcommand{\curl}{\grad\cross}
%% Simple horiz vectors
\renewcommand{\vector}[1]{\left\langle #1\right\rangle}


\outcome{Use the LCT to determine series behavior}

\title{Limit Comparison Test}

\begin{document}

\begin{abstract}
We will use the LCT to determine if a series converges or diverges.
\end{abstract}

\maketitle

\section{Limit Comparison Test}
We have seen that the Direct Comparison Test can be inconclusive if the comparison goes in the wrong direction. 
In these cases, the Limit Comparison Test (LCT) can be used instead.
As its name suggests, the LCT involves computing a limit. More precisely, it involves computing the limit of the ratio of a given series with 
another series whose behavior is known.  The idea is that if the limit of the ratio of these two series is a positive number, $L$, then the two series
will have the same behavior, as one of them is essentially a multiple of the other. We state this in the following theorem.

\begin{theorem}[Limit Comparison Test]
Suppose $a_n \geq 0$ and $b_n \geq 0$ for all values of $n$.
If 
\[
\lim_{n \to \infty} \frac{a_n}{b_n} = L
\]
where
\[
0<L<\infty,
\]
then the two infinite series 
\[
\sum_{n=0}^\infty a_n \;\; \text{ and } \;\; \sum_{n=0}^\infty b_n
\]
have the same behavior, i.e., they either both converge or both diverge.
\end{theorem}


\begin{remark}
In practical applications of the LCT, the given series is $\displaystyle{\sum_{n=0}^\infty a_n}$
and the series we choose to compare it with is $\displaystyle{\sum_{n=0}^\infty b_n}$.
\end{remark}


\begin{remark}
If $L = 0$ and $\displaystyle{\sum_{n=0}^\infty b_n}$ converges, then so does $\displaystyle{\sum_{n=0}^\infty a_n}$
and if $L = \infty$ and $\displaystyle{\sum_{n=0}^\infty b_n}$ diverges, then so does $\displaystyle{\sum_{n=0}^\infty a_n}$.
\end{remark}


\begin{example}
Determine if the series 
\[
\sum_{n=1}^\infty \frac{n+2}{n^3}
\]
converges or diverges.(We saw this series in the section on the DCT and were not able to make a conclusion).\\
We will use the LCT with the convergent p-series $\displaystyle{\sum_{n=1}^\infty \frac{1}{n^2}}$. We have:
\[
\lim_{n \to \infty} \frac{\left(\frac{n+2}{n^3}\right)}{\left(\frac{1}{n^2}\right)} = \lim_{n \to \infty} \frac{(n+2)n^2}{n^3}
\]
\[
= \lim_{n \to \infty} \frac{n^3 +2n^2}{n^3} = 1,
\]
using the ratio of the lead coefficients. We now make our conclusion using the value $L = 1$. Since $0 < 1 < \infty$ and the series $\displaystyle{\sum_{n=1}^\infty \frac{1}{n^2}}$
converges, the series $\displaystyle{\sum_{n=1}^\infty \frac{n+2}{n^3}}$ also converges by the LCT.
\end{example}




\begin{problem}
Consider the series $\displaystyle{\sum_{n=1}^\infty \frac{5n^2 + 1}{3n^4 - 2}}$.\\
Which series should we compare this to?

\begin{multipleChoice}
\choice{$\displaystyle{\sum_{n=1}^\infty \frac{1}{n}}$}
\choice[correct]{$\displaystyle{\sum_{n=1}^\infty \frac{1}{n^2}}$}
\choice{$\displaystyle{\sum_{n=1}^\infty \frac{1}{n^3}}$}
\end{multipleChoice}

What is the value of the limit in the LCT?
\begin{multipleChoice}
\choice{$L = 0$}
\choice{$L = 2$}
\choice[correct]{$L = 5/3$}
\end{multipleChoice}

Describe the behavior of the series $\displaystyle{\sum_{n=1}^\infty \frac{5n^2 + 1}{3n^4 - 2}:}$
\begin{multipleChoice}
\choice[correct]{Converges by LCT}
\choice{Diverges by LCT}
\choice{No Conclusion from LCT}
\end{multipleChoice}

\end{problem}



\begin{example}
Determine if the series 
\[
\sum_{n=0}^\infty \frac{2^n + 4}{3^n + 1}
\]
converges or diverges.\\
We will use the LCT with the convergent geometric series $\displaystyle{\sum_{n=1}^\infty \left(\frac23\right)^n}$. We have:
\begin{align*}
\lim_{n \to \infty} \frac{\left(\frac{2^n + 4}{3^n + 1}\right)}{\left(\frac23\right)^n} &= \lim_{n \to \infty} \frac{\left(\frac{2^n + 4}{3^n + 1}\right)}{\left(\frac{2^n}{3^n}\right)}\\
&=\lim_{n \to \infty} \frac{(2^n + 4)3^n}{(3^n + 1)2^n}\\
&= \lim_{n \to \infty} \frac{2^n + 4}{2^n} \cdot \lim_{n \to \infty}\frac{3^n}{3^n + 1}\\
&= 1 \cdot 1 \\
&= 1
\end{align*}

using L'Hopital's rule on each of the last two limits. We now make our conclusion using the value $L = 1$. Since $0 < 1 < \infty$ and the series $\displaystyle{\sum_{n=0}^\infty \left(\frac23\right)^n}$
converges, the series $\displaystyle{\sum_{n=0}^\infty \frac{2^n + 4}{3^n + 1}}$ also converges by the LCT.
\end{example}




\begin{problem}
Consider the series $\displaystyle{\sum_{n=0}^\infty \frac{5^n + 1}{2^n + 3}}$.
Which series should we compare this to?

\begin{multipleChoice}
\choice{$\displaystyle{\sum_{n=0}^\infty \left(\frac13\right)^n}$}
\choice{$\displaystyle{\sum_{n=0}^\infty \left(\frac53\right)^n}$}
\choice[correct]{$\displaystyle{\sum_{n=0}^\infty \left(\frac52\right)^n}$}
\end{multipleChoice}

What is the value of the limit in the LCT?
\begin{multipleChoice}
\choice{$L = 0$}
\choice[correct]{$L = 1$}
\choice{$L = 2$}
\end{multipleChoice}

Describe the behavior of the series $\displaystyle{\sum_{n=0}^\infty \frac{5^n + 1}{2^n + 3}:}$
\begin{multipleChoice}
\choice{Converges by LCT}
\choice[correct]{Diverges by LCT}
\choice{No Conclusion from LCT}
\end{multipleChoice}

\end{problem}





\begin{example}
Determine if the series 
\[
\sum_{n=1}^\infty \frac{n^2 + 5n + 12}{n^3 + 6n^2 + 2n}
\]
converges or diverges.\\
We will use the LCT with the divergent harmonic series $\displaystyle{\sum_{n=1}^\infty \frac{1}{n}}$. We have:
\[
\lim_{n \to \infty} \frac{\left(\frac{n^2 + 5n + 12}{n^3 + 6n^2 + 2n}\right)}{\left(\frac{1}{n}\right)} = \lim_{n \to \infty} \frac{(n^2 + 5n + 12)n}{n^3 + 6n^2 + 2n}
\]
\[
= \lim_{n \to \infty} \frac{n^3 + 5n^2 + 12n}{n^3 + 6n^2 + 2n} = 1,
\]
using the ratio of the lead coefficients. We now make our conclusion using the value $L = 1$. Since $0 < 1 < \infty$ and the series $\displaystyle{\sum_{n=1}^\infty \frac{1}{n}}$
diverges, the series $\displaystyle{\sum_{n=1}^\infty \frac{n^2 + 5n + 12}{n^3 + 6n^2 + 2n}}$ also diverges by the LCT.
\end{example}




\begin{problem}
Consider the series $\displaystyle{\sum_{n=1}^\infty \frac{4n^3 -3}{5n^4 + n}}$.
Which series should we compare this to?

\begin{multipleChoice}
\choice[correct]{$\displaystyle{\sum_{n=1}^\infty \frac{1}{n}}$}
\choice{$\displaystyle{\sum_{n=1}^\infty \frac{1}{n^2}}$}
\choice{$\displaystyle{\sum_{n=1}^\infty \frac{1}{n^3}}$}
\end{multipleChoice}

What is the value of the limit in the LCT?
\begin{multipleChoice}
\choice[correct]{$L = 4/5$}
\choice{$L = 5/3$}
\choice{$L = 1$}
\end{multipleChoice}

Describe the behavior of the series $\displaystyle{\sum_{n=1}^\infty \frac{4n^3 -3}{5n^4 +n}:}$
\begin{multipleChoice}
\choice{Converges by LCT}
\choice[correct]{Diverges by LCT}
\choice{No Conclusion from LCT}
\end{multipleChoice}

\end{problem}





\begin{example}
Determine if the series 
\[
\sum_{n=1}^\infty \frac{1}{\sqrt{n^2 + 1}}
\]
converges or diverges.\\
We will use the LCT with the divergent harmonic series $\displaystyle{\sum_{n=1}^\infty \frac{1}{n}}$. We have:
\[
\lim_{n \to \infty} \frac{\left(\frac{1}{\sqrt{n^2 + 1}}\right)}{\left(\frac{1}{n}\right)} = \lim_{n \to \infty} \frac{n}{\sqrt{n^2 + 1}}
\]
\[
= \lim_{n \to \infty} \frac{n}{n\sqrt{1 + \frac{1}{n^2}}} = \lim_{n \to \infty} \frac{1}{\sqrt{1 + \frac{1}{n^2}}} = 1.
\]
We now make our conclusion using the value $L = 1$. Since $0 < 1 < \infty$ and the series $\displaystyle{\sum_{n=1}^\infty \frac{1}{n}}$
diverges, the series $\displaystyle{\sum_{n=1}^\infty \frac{1}{\sqrt{n^2 + 1}}}$ also diverges by the LCT.
\end{example}




\begin{problem}
Consider the series $\displaystyle{\sum_{n=1}^\infty \frac{1}{\sqrt{2n^6 - 1}}}$.
Which series should we compare this to?

\begin{multipleChoice}
\choice{$\displaystyle{\sum_{n=1}^\infty \frac{1}{n^2}}$}
\choice[correct]{$\displaystyle{\sum_{n=1}^\infty \frac{1}{n^3}}$}
\choice{$\displaystyle{\sum_{n=1}^\infty \frac{1}{n^6}}$}
\end{multipleChoice}

What is the value of the limit in the LCT?
\begin{multipleChoice}
\choice{$L = 1$}
\choice{$L = 1/2$}
\choice[correct]{$L = 1/\sqrt2$}
\end{multipleChoice}

Describe the behavior of the series $\sum_{n=1}^\infty \frac{4n^3 -3}{5n^4 +n}:$
\begin{multipleChoice}
\choice[correct]{Converges by LCT}
\choice{Diverges by LCT}
\choice{No Conclusion from LCT}
\end{multipleChoice}

\end{problem}


\begin{remark}
Since the limits
\[
\lim_{n \to \infty} \sin(n) \;\; \text{ and } \;\; \lim_{n \to \infty} \cos(n)
\]
do not exist (due to oscillation), the LCT is generally not effective in problems involving $\sin(n)$ and $\cos(n)$. Try the DCT instead.
\end{remark}

\section{Video Lesson}

\begin{center}
\begin{foldable}
\unfoldable{Here is a detailed, lecture style video on the Limit Comparison Test:}
\youtube{d2ei8EG-hyo}
\end{foldable}
\end{center}


\section{More Problems}


\begin{problem}
Consider the series $\displaystyle{\sum_{n=0}^\infty \frac{6^n + 5}{3^n + 2}}$.
Which series should we compare this to?

\begin{multipleChoice}
\choice[correct]{$\displaystyle{\sum_{n=0}^\infty 2^n}$}
\choice{$\displaystyle{\sum_{n=0}^\infty 3^n}$}
\choice{$\displaystyle{\sum_{n=0}^\infty \left(\frac52\right)^n}$}
\end{multipleChoice}

What is the value of the limit in the LCT?
\begin{multipleChoice}
\choice{$L = 0$}
\choice{$L = 2/5$}
\choice[correct]{$L = 1$}
\end{multipleChoice}

Describe the behavior of the series $\displaystyle{\sum_{n=0}^\infty \frac{6^n + 5}{3^n + 2}:}$
\begin{multipleChoice}
\choice{Converges by LCT}
\choice[correct]{Diverges by LCT}
\choice{No Conclusion from LCT}
\end{multipleChoice}

\end{problem}




\begin{problem}
Consider the series $\displaystyle{\sum_{n=1}^\infty \frac{3n^5 + 4}{2n^7 - 1}}$.
Which series should we compare this to?

\begin{multipleChoice}
\choice{$\displaystyle{\sum_{n=1}^\infty \frac{1}{n}}$}
\choice[correct]{$\displaystyle{\sum_{n=1}^\infty \frac{1}{n^2}}$}
\choice{$\displaystyle{\sum_{n=1}^\infty \frac{1}{n^7}}$}
\end{multipleChoice}

What is the value of the limit in the LCT?
\begin{multipleChoice}
\choice{$L = 1$}
\choice[correct]{$L = 3/2$}
\choice{$L = 4$}
\end{multipleChoice}

Describe the behavior of the series $\displaystyle{\sum_{n=1}^\infty \frac{3n^5 + 4}{2n^7 - 1}:}$
\begin{multipleChoice}
\choice[correct]{Converges by LCT}
\choice{Diverges by LCT}
\choice{No Conclusion from LCT}
\end{multipleChoice}

\end{problem}




\begin{problem}
Consider the series $\displaystyle{\sum_{n=1}^\infty \frac{3n^4 -2}{4n^5 + 3n^4}}$.
Which series should we compare this to?

\begin{multipleChoice}
\choice[correct]{$\displaystyle{\sum_{n=1}^\infty \frac{1}{n}}$}
\choice{$\displaystyle{\sum_{n=1}^\infty \frac{1}{n^4}}$}
\choice{$\displaystyle{\sum_{n=1}^\infty \frac{1}{n^5}}$}
\end{multipleChoice}

What is the value of the limit in the LCT?
\begin{multipleChoice}
\choice[correct]{$L = 1/2$}
\choice{$L = 3/4$}
\choice{$L = 1$}
\end{multipleChoice}

Describe the behavior of the series $\displaystyle{\sum_{n=1}^\infty \frac{3n^4 -2}{4n^5 + 3n^4}:}$
\begin{multipleChoice}
\choice{Converges by LCT}
\choice[correct]{Diverges by LCT}
\choice{No Conclusion from LCT}
\end{multipleChoice}

\end{problem}





\begin{problem}
Consider the series $\displaystyle{\sum_{n=1}^\infty \frac{n^2}{\sqrt{4n^8 - 3}}}$.
Which series should we compare this to?

\begin{multipleChoice}
\choice[correct]{$\displaystyle{\sum_{n=1}^\infty \frac{1}{n^2}}$}
\choice{$\displaystyle{\sum_{n=1}^\infty \frac{1}{n^4}}$}
\choice{$\displaystyle{\sum_{n=1}^\infty \frac{1}{n^8}}$}
\end{multipleChoice}

What is the value of the limit in the LCT?
\begin{multipleChoice}
\choice{$L = 1/\sqrt3$}
\choice{$L = 1/4$}
\choice[correct]{$L = 1/2$}
\end{multipleChoice}

Describe the behavior of the series $\displaystyle{\sum_{n=1}^\infty \frac{4n^3 -3}{5n^4 +n}:}$
\begin{multipleChoice}
\choice[correct]{Converges by LCT}
\choice{Diverges by LCT}
\choice{No Conclusion from LCT}
\end{multipleChoice}

\end{problem}




\end{document}





















































\begin{example} %example #15
Find $h'(x)$ if $h(x) = x^{\sin(x)}$.\\
We will use the fact that the exponential and logarithm functions are inverses,
\[e^{\ln(x)} = x,\]
and the exponent property of logarithms, 
\[\ln(x^n) = n\ln(x),\]
to rewrite $h(x)$.  We have 
\[h(x) = x^{\sin(x)} = e^{\ln(x^{\sin(x)})} = e^{\sin(x)\ln(x)}\]
and we can now compute $h'(x)$ using a combination of the chain rule and product rule.
We can write $h(x)$ as a composition, $f(g(x))$ with 
\[g(x) = \sin(x)\ln(x) \quad \text{and} \quad f(x) = e^x.\]
Then to find $g'(x)$ we us the product rule and we get $g'(x) = \frac{\sin(x)}{x} + \cos(x)\ln(x)$.
Next $f'(x) = e^x$ and 
hence $f'(g(x)) = e^{g(x)} = e^{\sin(x)\ln(x)} = x^{\sin(x)}$.
We can then conclude $h'(x) = f'(g(x))g'(x) = x^{\sin(x)} \left[ \frac{\sin(x)}{x} + \cos(x)\ln(x)\right]$.
\end{example}

%more question formats below













%\begin{verbatim}
\begin{question}
What is your favorite color?
\begin{multipleChoice}
\choice[correct]{Rainbow}
\choice{Blue}
\choice{Green}
\choice{Red}
\end{multipleChoice}
\begin{freeResponse}
Hello
\end{freeResponse}
\end{question}
%\end{verbatim}





\begin{question}
  Which one will you choose?
  \begin{multipleChoice}
    \choice[correct]{I'm correct.}
    \choice{I'm wrong.}
    \choice{I'm wrong too.}
  \end{multipleChoice}
\end{question}


\begin{question}
  Which one will you choose?
  \begin{selectAll}
    \choice[correct]{I'm correct.}
    \choice{I'm wrong.}
    \choice[correct]{I'm also correct.}
    \choice{I'm wrong too.}
  \end{selectAll}
\end{question}


\begin{freeResponse}
What is the chain rule used for?
\end{freeResponse}
