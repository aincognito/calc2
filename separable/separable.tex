\documentclass{ximera}

%% You can put user macros here
%% However, you cannot make new environments



\newcommand{\ffrac}[2]{\frac{\text{\footnotesize $#1$}}{\text{\footnotesize $#2$}}}
\newcommand{\vasymptote}[2][]{
    \draw [densely dashed,#1] ({rel axis cs:0,0} -| {axis cs:#2,0}) -- ({rel axis cs:0,1} -| {axis cs:#2,0});
}


\graphicspath{{./}{firstExample/}}

\usepackage{amsmath}
\usepackage{amssymb}
\usepackage{array}
\usepackage[makeroom]{cancel} %% for strike outs
\usepackage{pgffor} %% required for integral for loops
\usepackage{tikz}
\usepackage{tikz-cd}
\usepackage{tkz-euclide}
\usetikzlibrary{shapes.multipart}


\usetkzobj{all}
\tikzstyle geometryDiagrams=[ultra thick,color=blue!50!black]


\usetikzlibrary{arrows}
\tikzset{>=stealth,commutative diagrams/.cd,
  arrow style=tikz,diagrams={>=stealth}} %% cool arrow head
\tikzset{shorten <>/.style={ shorten >=#1, shorten <=#1 } } %% allows shorter vectors

\usetikzlibrary{backgrounds} %% for boxes around graphs
\usetikzlibrary{shapes,positioning}  %% Clouds and stars
\usetikzlibrary{matrix} %% for matrix
\usepgfplotslibrary{polar} %% for polar plots
\usepgfplotslibrary{fillbetween} %% to shade area between curves in TikZ



%\usepackage[width=4.375in, height=7.0in, top=1.0in, papersize={5.5in,8.5in}]{geometry}
%\usepackage[pdftex]{graphicx}
%\usepackage{tipa}
%\usepackage{txfonts}
%\usepackage{textcomp}
%\usepackage{amsthm}
%\usepackage{xy}
%\usepackage{fancyhdr}
%\usepackage{xcolor}
%\usepackage{mathtools} %% for pretty underbrace % Breaks Ximera
%\usepackage{multicol}



\newcommand{\RR}{\mathbb R}
\newcommand{\R}{\mathbb R}
\newcommand{\C}{\mathbb C}
\newcommand{\N}{\mathbb N}
\newcommand{\Z}{\mathbb Z}
\newcommand{\dis}{\displaystyle}
%\renewcommand{\d}{\,d\!}
\renewcommand{\d}{\mathop{}\!d}
\newcommand{\dd}[2][]{\frac{\d #1}{\d #2}}
\newcommand{\pp}[2][]{\frac{\partial #1}{\partial #2}}
\renewcommand{\l}{\ell}
\newcommand{\ddx}{\frac{d}{\d x}}

\newcommand{\zeroOverZero}{\ensuremath{\boldsymbol{\tfrac{0}{0}}}}
\newcommand{\inftyOverInfty}{\ensuremath{\boldsymbol{\tfrac{\infty}{\infty}}}}
\newcommand{\zeroOverInfty}{\ensuremath{\boldsymbol{\tfrac{0}{\infty}}}}
\newcommand{\zeroTimesInfty}{\ensuremath{\small\boldsymbol{0\cdot \infty}}}
\newcommand{\inftyMinusInfty}{\ensuremath{\small\boldsymbol{\infty - \infty}}}
\newcommand{\oneToInfty}{\ensuremath{\boldsymbol{1^\infty}}}
\newcommand{\zeroToZero}{\ensuremath{\boldsymbol{0^0}}}
\newcommand{\inftyToZero}{\ensuremath{\boldsymbol{\infty^0}}}


\newcommand{\numOverZero}{\ensuremath{\boldsymbol{\tfrac{\#}{0}}}}
\newcommand{\dfn}{\textbf}
%\newcommand{\unit}{\,\mathrm}
\newcommand{\unit}{\mathop{}\!\mathrm}
%\newcommand{\eval}[1]{\bigg[ #1 \bigg]}
\newcommand{\eval}[1]{ #1 \bigg|}
\newcommand{\seq}[1]{\left( #1 \right)}
\renewcommand{\epsilon}{\varepsilon}
\renewcommand{\iff}{\Leftrightarrow}

\DeclareMathOperator{\arccot}{arccot}
\DeclareMathOperator{\arcsec}{arcsec}
\DeclareMathOperator{\arccsc}{arccsc}
\DeclareMathOperator{\si}{Si}
\DeclareMathOperator{\proj}{proj}
\DeclareMathOperator{\scal}{scal}
\DeclareMathOperator{\cis}{cis}
\DeclareMathOperator{\Arg}{Arg}
%\DeclareMathOperator{\arg}{arg}
\DeclareMathOperator{\Rep}{Re}
\DeclareMathOperator{\Imp}{Im}
\DeclareMathOperator{\sech}{sech}
\DeclareMathOperator{\csch}{csch}
\DeclareMathOperator{\Log}{Log}

\newcommand{\tightoverset}[2]{% for arrow vec
  \mathop{#2}\limits^{\vbox to -.5ex{\kern-0.75ex\hbox{$#1$}\vss}}}
\newcommand{\arrowvec}{\overrightarrow}
\renewcommand{\vec}{\mathbf}
\newcommand{\veci}{{\boldsymbol{\hat{\imath}}}}
\newcommand{\vecj}{{\boldsymbol{\hat{\jmath}}}}
\newcommand{\veck}{{\boldsymbol{\hat{k}}}}
\newcommand{\vecl}{\boldsymbol{\l}}
\newcommand{\utan}{\vec{\hat{t}}}
\newcommand{\unormal}{\vec{\hat{n}}}
\newcommand{\ubinormal}{\vec{\hat{b}}}

\newcommand{\dotp}{\bullet}
\newcommand{\cross}{\boldsymbol\times}
\newcommand{\grad}{\boldsymbol\nabla}
\newcommand{\divergence}{\grad\dotp}
\newcommand{\curl}{\grad\cross}
%% Simple horiz vectors
\renewcommand{\vector}[1]{\left\langle #1\right\rangle}


\outcome{Solve separable differential equations}

\title{1.6 Separable Differential Equations}

\begin{document}

\begin{abstract}
We solve separable differential equations and initial value problems.
\end{abstract}

\maketitle

\section{Differential Equations}

A differential equation is an equation that involves one or more derivatives of an unknown function. 
Solving a differential equation entails determining the unknown function.

\begin{example}[example 1]
Solve the differential equation $y' = y$.\\
We seek $y = f(x)$ such that $f'(x) = f(x)$. By inspection $y = e^x$ is a solution of this equation.
However, $y = 2e^x$ is also a solution. In fact, there are infinitely many solutions: $y = Ce^x$,
where $C$ is any constant (including 0). This family of solutions to the differential equation is called the 
\textbf{general solution}.
\end{example}



\begin{problem}(problem 1)
Find all solutions of the differential equation $y' = 2y$.\\
\begin{hint}
Try $y = e^{kx}$ for some $k$
\end{hint}
\begin{hint}
There are infinitely many solutions
\end{hint}
\begin{hint}
A multiple of a solution is also a solution
\end{hint}
The solutions are: \; $y = \answer{Ce^{2x}}$ \; for any constant $C$.
\end{problem}



An initial value problem is a differential equation along with other information about the solution, usually the value of the function at a point.
The purpose of the initial value is to determine one specific solution of the differential equation, in the event that there was more than one solution.

\begin{example}[example 2]
Solve the initial value problem $y' = y, y(0) = 4$.\\
We saw in the last example that the differential equation $y' = y$ has infinitely many solutions, $y = Ce^x$, where $C$ is any constant.
The initial value $y(0) = 4$ can be used to determine the value of $C$ so that we will have a unique solution. Since $y(0) = Ce^0 = 4$, we can see that $C = 4$.
Thus the initial value problem $y' = y, y(0) = 4$ has the unique solution $y = 4e^x$.
\end{example}



\begin{problem}(problem 2)
Solve the initial value problem $y' = 3y, \; y(0) = 2$.\\

The general solution of the differential equation is

\begin{multipleChoice}
\choice{$y = 3Ce^x$}
\choice[correct]{$y = Ce^{3x}$}
\choice{$y = e^{3x}$}
\end{multipleChoice}

The solution to the initial value problem is\\
$y = \answer{2e^{3x}}$
\end{problem}




\section{Separable Differential Equations}

A \textbf{separable differential equation} is a differential equation that can be put in the form
$y' = f(x)g(y)$. To solve such an equation, we separate the variables by moving the $y$'s to one side and the $x$'s to the other, then integrate both sides with respect to $x$
and solve for $y$. In general, the process goes as follows:
\[
y' = f(x)g(y) \implies \frac{1}{g(y)} y'= f(x) .
\]
Let $h(y) = 1/g(y)$ for convenience and we have
\[
h(y) y'= f(x).
\]

Integrating both sides with respect to $x$ using a the substitution $u = y, \; du = y' \, dx$, we have
\[
\int h(y) y' \, dx = \int f(x) \, dx
\]
\[
\implies \int h(u) \, du = \int f(x) \, dx
\]
\[
\implies H(u) = F(x) + C,
\]
and so
\[
H(y) = F(x) + C.
\]
Now we solve for $y$ algebraically to get the final answer.  To simplify the computations, we will use $\frac{dy}{dx}$ instead of $y'$ and then 
make a slight abuse of notation to get to the same end result. From the beginning:
\[
y' = f(x) g(y) \implies \frac{1}{g(y)} \frac{dy}{dx} = f(x) \implies h(y) \frac{dy}{dx} = f(x),
\]
again using $h(y) = 1/g(y)$ for convenience. Now we write this last equation in differential form and put an integral sign on both sides(!):
\[
h(y)\, dy = f(x) \, dx
\]
\[
\int h(y) \, dy = \int f(x) \, dx,
\]
yielding
\[
H(y) = F(x) + C,
\]
as before.

\begin{example}[example 3]
Solve the differential equation $y' = xy^2$.\\
Rewrite the equation as 
\[
\frac{dy}{dx} = xy^2.
\]
Move the $y^2$ to the left and the $dx$ to the right to get
\[
\frac{1}{y^2} \, dy = x \,  dx.
\]
Integrate both sides:
\[
\int \frac{1}{y^2}\,  dy  = \int x \, dx.
\]
This gives
\[
-\frac{1}{y} = \frac{x^2}{2} + C.
\]
Solve for $y$:
\[
\frac{1}{y} = -\frac{x^2}{2} + C
\]
Note that $-C$ has been replaced by $+C$, as this makes no difference. Now take reciprocals:
\[
y = \frac{1}{-\frac{x^2}{2} + C} = - \frac{2}{x^2 + C}.
\]
Note that $-2C$ was replaced by $+C$ as this makes no difference.
We can check the differential equation by calculating $y'$ and observing that $y = xy^2$.
From the quotient rule, we have
\[
y' = \frac{(x^2 + C)\cdot 0 - (-2)(2x)}{(x^2 + C)^2} = \frac{4x}{(x^2 + C)^2} 
\]
\[
= x \cdot \frac{4}{(x^2 + C)^2} = x \cdot \left[\frac{2}{x^2 + C}\right]^2 = xy^2.
\]



\end{example}


\begin{problem}(problem 3a)
Find the general solution of the differential equation
\[
\frac{dy}{dx} = \frac{3x^2}{2y^3}
\]

Separating the variables gives

\begin{multipleChoice}
\choice[correct]{$\displaystyle{2y^3 \, dy = 3x^2 \, dx}$}
\choice{$\displaystyle{\frac{1}{2y^3} \, dy = \frac{1}{3x^2} \, dx}$}
\choice{$\displaystyle{\frac{1}{2y^3} \, dy = 3x^2 \, dx}$}
\end{multipleChoice}

Integrating gives

\begin{multipleChoice}
\choice{$\displaystyle{\frac{y^4}{2} = \frac{x^3}{3} + C}$}
\choice[correct]{$\displaystyle{\frac{y^4}{2} = x^3 + C}$}
\choice{$\displaystyle{\frac{y^4}{4} = x^3 + C}$}
\end{multipleChoice}

The general solution is

\begin{multipleChoice}
\choice[correct]{$\displaystyle{y = \sqrt[4]{2x^3 + C}}$}
\choice{$\displaystyle{y = \sqrt[4]{4x^3 + C}}$}
\choice{$\displaystyle{y = \sqrt{2x^3 + C}}$}
\end{multipleChoice}

\end{problem}






\begin{problem}(problem 3b)
Solve the initial value problem
\[
\frac{dy}{dx} = y^2e^{3x}, \; y(0) = 4
\]

Separating the variables gives

\begin{multipleChoice}
\choice{$\displaystyle{y^2 \, dy = e^{3x} \, dx}$}
\choice{$\displaystyle{\frac{e}{y^2} \, dy = 3x \, dx}$}
\choice[correct]{$\displaystyle{\frac{1}{y^2} \, dy = e^{3x} \, dx}$}
\end{multipleChoice}

Integrating gives

\begin{multipleChoice}
\choice{$\displaystyle{\ln|y^2| = \frac13 e^{3x} + C}$}
\choice[correct]{$\displaystyle{-\frac{1}{y} = \frac13 e^{3x} + C}$}
\choice{$\displaystyle{\frac{1}{y} = \frac13 e^{3x} + C}$}
\end{multipleChoice}

The general solution is

\begin{multipleChoice}
\choice[correct]{$\displaystyle{y = -\frac{3}{e^{3x} + C}}$}
\choice{$\displaystyle{y = \frac{3}{e^{3x} + C}}$}
\choice{$\displaystyle{y = -\frac{1}{e^{3x} + C}}$}
\end{multipleChoice}

The solution to the initial value problem is

\begin{multipleChoice}
\choice{$\displaystyle{y = \frac{3}{7 - 4e^{3x}}}$}
\choice{$\displaystyle{y = \frac{12}{4e^{3x} - 7}}$}
\choice[correct]{$\displaystyle{y = \frac{12}{7 - 4e^{3x}}}$}
\end{multipleChoice}

\end{problem}






\begin{example}[example 4]
Solve the differential equation $y' = x^2y$.\\
Rewrite the equation as 
\[
\frac{dy}{dx} = x^2y.
\]
Move the $y$ to the left and the $dx$ to the right to get
\[
\frac{1}{y} \, dy = x^2 \,  dx.
\]
Integrate both sides:
\[
\int \frac{1}{y}\,  dy  = \int x^2 \, dx.
\]
This gives
\[
\ln|y| = \frac{x^3}{3} + C.
\]
Solve for $y$:
\[
|y| = e^{\frac{x^3}{3} + C} = e^C \cdot e^{x^3/3}.
\]
Note that $e^C$ is a positive constant. Remove the absolute values:
\[
y = \pm e^C \cdot e^{x^3/3} = Ae^{x^3/3},
\]
where $A$ is any non-zero constant. Looking back at the original differential equation, we see that $y = 0$ is a solution, so 
we can say that $A$ is actually any constant.
We can check the differential equation by calculating $y'$ and observing that $y = x^2y$.
From the chain rule, we have
\[
y' = Ae^{x^3/3}\cdot \frac{d}{dx} \left(\frac{x^3}{3}\right) = Ae^{x^3/3} \cdot x^2 = x^2 y.
\]

\end{example}



\begin{problem}(problem 4a)
Find the general solution of the differential equation
\[
ty + \frac{dy}{dt} = 0
\]

Separating the variables gives

\begin{multipleChoice}
\choice{$\displaystyle{y \, dy = t \, dt}$}
\choice{$\displaystyle{\frac{1}{y} \, dy = t \, dt}$}
\choice[correct]{$\displaystyle{\frac{1}{y} \, dy = -t \, dt}$}
\end{multipleChoice}

Integrating gives

\begin{multipleChoice}
\choice{$\displaystyle{\ln(y) = -\frac{t^2}{2} + C}$}
\choice[correct]{$\displaystyle{\ln|y| = -\frac{t^2}{2} + C}$}
\choice{$\displaystyle{\ln|y| = \frac{t^2}{2} + C}$}
\end{multipleChoice}

The general solution is

\begin{multipleChoice}
\choice[correct]{$\displaystyle{y = Ce^{-t^2/2}}$}
\choice{$\displaystyle{y = e^{-t^2/2} + C}$}
\choice{$\displaystyle{y = Ce^{-t^2}}$}
\end{multipleChoice}

\end{problem}



\begin{problem}(problem 4b)
Solve the initial value problem
\[
(1+t^2)\frac{dy}{dt} = 2ty, \; y(1) = 8
\]

Separating the variables gives

\begin{multipleChoice}
\choice{$\displaystyle{y \, dy = 2t(1+t^2) \, dt}$}
\choice{$\displaystyle{\frac{1}{y} \, dy = 2t(1+t^2) \, dt}$}
\choice[correct]{$\displaystyle{\frac{1}{y} \, dy = \frac{2t}{1+t^2} \, dt}$}
\end{multipleChoice}

Integrating gives

\begin{multipleChoice}
\choice{$\displaystyle{\ln|y| = \frac12\ln(1+t^2) + C}$}
\choice[correct]{$\displaystyle{\ln|y| = \ln(1+t^2) + C}$}
\choice{$\displaystyle{\ln|y| = \tan^{-1}(1+t^2) + C}$}
\end{multipleChoice}

The general solution is

\begin{multipleChoice}
\choice[correct]{$\displaystyle{y = C(1+t^2)}$}
\choice{$\displaystyle{y = t^2 + C}$}
\choice{$\displaystyle{y = Ce^{1+t^2}}$}
\end{multipleChoice}

The solution to the initial value problem is

\begin{multipleChoice}
\choice{$\displaystyle{y = t^2 + 8}$}
\choice{$\displaystyle{y = 8+8t^2}$}
\choice[correct]{$\displaystyle{y = 4+4t^2}$}
\end{multipleChoice}

\end{problem}



\begin{example}[example 5]
A population of 2 million bacteria doubles in 1 hour. If we assume that the population is growing at a rate which is proportional to the number of bacteria present,
write a differential equation that models the population at time $t$ and find the population after 90 minutes.\\
Let $P = P(t)$ denote the population of the bacteria (in millions) after $t$ hours.
Since the rate of growth is proportional to the number present, we have
\[
\frac{dP}{dt} = k P,
\]
for some constant $k$.
Moving $P$ to the left and $dt$ to the right and integrating gives
\[
\int \frac{1}{P}\, dP = \int k \, dt
\]
Anti-differentiating gives:
\[
\ln|P| = kt + C
\]
Solving for $P$:
\[
|P| = e^{kt+C} = e^C \cdot e^{kt},
\]
Removing the absolute value bars, we have
\[
P = \pm e^C \cdot e^{kt} = Ae^{kt},
\]
where $A$ is a nonzero constant.
To determine the values of $A$ and $k$ we will use the given information about the size of the population at time $t = 0$ and $t = 1$.
The population was initially 2 million, so $P(0) = 2$ and in one hour it doubled, so $P(1) = 4$. the first of these gives,
\[
P(0) = Ae^{0} = 2 \implies A = 2.
\]
Using the second condition gives:
\[
P(1) = 2e^k = 4 \implies e^k = 2 \implies k = \ln(2).
\]
Thus $P(t) = 2e^{\ln(2)t}$ is the formula for the population at time $t$. Finally, after 90 minutes (or 1.5 hours), the population is 
$P(1.5) = 2e^{1.5\ln(2)} \approx 5.66$ million bacteria.

\end{example}



\begin{problem}(problem 5a)
In the year 2000, the human population on earth was approximately 6 billion people and in the year 2010 it was approximately 6.5 billion. 
If we assume that the population is growing at a rate which is proportional to the number of people present,
write a differential equation that models the population at time $t$ and find the population in the year 2050.\\


Let $P = P(t)$ denote the population  (in billions)  $t$ years after the year 2000.\\
The differential equation which models human population is\\

$\frac{dP}{dt} = \answer{k P},$ for some constant $k$.

The general solution to this differential equation is 

\begin{multipleChoice}
\choice{$P =  e^{kt} + C$}
\choice[correct]{$P =  Ce^{kt}$}
\choice{$P =  e^{Ckt}$}
\end{multipleChoice}

The value of $C$ is $\answer{6}$.\\
The value of $k$ to 3 decimal places is $\answer{0.008}$.\\
The population in the year 2050 is projected to be 

\begin{multipleChoice}
\choice{8.5 billion}
\choice[correct]{9 billion}
\choice{9.5 billion}
\end{multipleChoice}

\end{problem}



\begin{problem}(problem 5b)
With continuous compounding of interest on money in a bank account, the rate of growth is proportional to the amount present.
How long will it take for money in an account to double if the account earns a $7\%$ yearly interest rate with continuous compounding?



Let $A= A(t)$ denote the amount of money in the account at time  $t$  in years.\\
The differential equation which models the amount in the account is\\

$\frac{dA}{dt} = \answer{0.07A}$

The general solution to this differential equation is 

\begin{multipleChoice}
\choice[correct]{$A =  Ce^{0.07t}$}
\choice{$A =  Ce^{7t}$}
\choice{$A =  e^{0.07t} + C$}
\end{multipleChoice}

To determine the doubling time, we should let $A = \answer{2C}$\\

The doubling time is approximately

\begin{multipleChoice}
\choice{9 years}
\choice{9.5 years}
\choice[correct]{10 years}
\end{multipleChoice}

\end{problem}


\begin{example}[example 6]
A water tank contains 100 gallons of pure water. Salt water with a concentration of 0.2 lbs/gal is added to the tank at a rate of 3 gallons per minute.
At the same time, water is drained from the tank at the same rate of 3 gallons per minute.
Find an expression for the amount of salt in the tank at time $t$.\\
Let $A = A(t)$ represent the amount of salt in the tank at time $t$. Salt is entering the tank at a rate of 0.6 lb/min and 
salt is leaving the tank at a rate of 0.03A lbs/min. This leads to the differential equation
\[
A' = 0.6 - 0.03A \text{  or  } \frac{dA}{dt} = 0.6 - 0.03A.
\]
Separating variables gives
\[
\frac{1}{0.6 -0.03A} \, dA = 1\, dt.
\]
Integrating gives
\[
\int \frac{1}{0.6 -0.03A} \, dA = \int 1 \, dt \implies -\frac{1}{0.03}\ln|0.6 - 0.03A|  = t + C.
\]
Solving for $A$, we have
\[
\ln|0.6 - 0.03A|  = -0.03t + C \implies |0.6-0.03A| = e^{-0.03t + C} = e^C \cdot e^{-0.03t}.
\]
Removing the absolute value bars, we have
\[
0.6 - 0.03A = \pm e^C \cdot e^{-0.03t} = Ke^{-0.03t},
\]
where K is a nonzero constant.
subtracting $0.6$ and dividing by $-0.03$ we have 
\[
-0.03A = -0.6 + Ke^{-0.03t} \implies A = \frac{0.6}{0.03} - \frac{1}{0.03}Ke^{-0.03t}.
\]
We can find the value of $K$ since we know that at time $t = 0$ there was no salt in the tank, so that $A(0) = 0$:
\[
A(0) = \frac{0.6}{0.03} - \frac{1}{0.03}Ke^0 = 0 \implies  K = 0.6.
\]
Since $0.6/0.03 = 20$ we can write the amount of salt in the tank as
\[
A= A(t) = 20 - 20e^{-0.03t}.
\]
Note that as $t\to \infty$, $A(t) \to 20 - 0 = 20$ pounds of salt. 
This makes sense since we are adding salt at a concentration of 0.2 lbs/gal to a 100 gallon tank and 0.2 lbs/gal $\cdot$ 100 gal = 20 lbs.
\end{example}



\begin{problem}(problem 6a)
A water tanks contains 60 gallons of salt water with 1 pound of salt mixed in. 
Salt water with a concentration of 0.1 lbs/gal is added to the tank at a rate of 1.5 gallons per minute.
At the same time, water is drained from the tank at the same rate of 1.5 gallons per minute.
How long will it take for the tank to contain 2 pounds of salt?\\


Let $A = A(t)$ represent the amount of salt in the tank at time $t$ minutes. \\


Salt is entering the tank at a rate of $\answer{0.15}$ \; lb/min \\
Salt is leaving the tank at a rate of $\answer{0.025A}$ \; lbs/min \\

This leads to the differential equation
\[
\frac{dA}{dt} = \answer{0.15 - 0.025A}.
\]
Separating variables gives

\begin{multipleChoice}
\choice{$\displaystyle{\frac{1}{0.025A} \, dA = 0.15\, dt}$}
\choice{$\displaystyle{0.025A + dA \, dA = 0.15\, dt}$}
\choice[correct]{$\displaystyle{\frac{1}{0.15 -0.025A} \, dA = 1\, dt}$}
\end{multipleChoice}


Integrating gives
\begin{multipleChoice}
\choice[correct]{$\displaystyle{-\frac{1}{0.025}\ln|0.15 - 0.025A|  = t + C}$}
\choice{$\displaystyle{\frac{1}{0.025}\ln|0.15 - 0.025A|  = t + C}$}
\choice{$\displaystyle{0.025\ln|0.15 - 0.025A|  = t + C}$}
\end{multipleChoice}


Solving for $A$, we have

\begin{multipleChoice}
\choice{$\displaystyle{A = 60 - Ce^{-0.025t}}$}
\choice[correct]{$\displaystyle{A = 6 - Ce^{-0.025t}}$}
\choice{$\displaystyle{A = 6 - Ce^{-40t}}$}
\end{multipleChoice}

The value of $C$ is $\answer{5}$

The time it takes for the tank to contain 2 pounds of salt  is approximately

\begin{multipleChoice}
\choice[correct]{9 minutes}
\choice{9.5 minutes}
\choice{10 minutes}
\end{multipleChoice}

\end{problem}



\begin{problem}(problem 6b)
Newtons Law of Cooling states that the rate of cooling of an object at temperature $T = T(t)$ is proportional to the difference between $T$
and the ambient temperature. A loaf of banana nut bread (with chocolate chips) is removed from an oven at $400^\circ F$ and placed on a cooling rack
in a kitchen whose temperature is $70^{\circ} F$.  
Two minutes later, the temperature of the bread is measured to be $250^\circ F$.
How much longer will take to be at the desired eating temperature of $150^\circ F$ \\

The differential equation that models Newton's Law of Cooling is
\[
\frac{dT}{dt} = \answer{k(T-70)}, \; \text{ where $k$ is the constant of proportionality.}
\]

Separating variables gives

\begin{multipleChoice}
\choice[correct]{$\displaystyle{\frac{1}{T-70 } \, dT = k\, dt}$}
\choice{$\displaystyle{\frac{T}{70} \, dT = k\, dt}$}
\choice{$\displaystyle{\frac{1}{T} \, dT = 70k\, dt}$}
\end{multipleChoice}


Integrating gives
\begin{multipleChoice}
\choice{$\displaystyle{\ln|T-70|  = \frac{k^2}{2} + C}$}
\choice{$\displaystyle{-\ln|T-70|  = kt + C}$}
\choice[correct]{$\displaystyle{\ln|T-70|  = kt + C}$}
\end{multipleChoice}


Solving for $T$, we have

\begin{multipleChoice}
\choice{$\displaystyle{T = 70e^{kt + C}}$}
\choice[correct]{$\displaystyle{T = 70 + Ce^{kt}}$}
\choice{$\displaystyle{T = 70 +  e^{kt+C}}$}
\end{multipleChoice}

The value of $C$ is $\answer{330}$.

The value of the constant of proportionality, $k$ is

\begin{multipleChoice}
\choice[correct]{-0.303}
\choice{0.303}
\choice{-0.606}
\end{multipleChoice}



The additional time until the bread is at $150^\circ$ is 

\begin{multipleChoice}
\choice[correct]{2.7 minutes}
\choice{4.7 minutes}
\choice{6.7 minutes}
\end{multipleChoice}


\end{problem}



\end{document}



%more question formats below













%\begin{verbatim}
\begin{question}
What is your favorite color?
\begin{multipleChoice}
\choice[correct]{Rainbow}
\choice{Blue}
\choice{Green}
\choice{Red}
\end{multipleChoice}
\begin{freeResponse}
Hello
\end{freeResponse}
\end{question}
%\end{verbatim}





\begin{question}
  Which one will you choose?
  \begin{multipleChoice}
    \choice[correct]{I'm correct.}
    \choice{I'm wrong.}
    \choice{I'm wrong too.}
  \end{multipleChoice}
\end{question}


\begin{question}
  Which one will you choose?
  \begin{selectAll}
    \choice[correct]{I'm correct.}
    \choice{I'm wrong.}
    \choice[correct]{I'm also correct.}
    \choice{I'm wrong too.}
  \end{selectAll}
\end{question}


\begin{freeResponse}
What is the chain rule used for?
\end{freeResponse}
