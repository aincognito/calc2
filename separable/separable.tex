\documentclass{ximera}

%% You can put user macros here
%% However, you cannot make new environments



\newcommand{\ffrac}[2]{\frac{\text{\footnotesize $#1$}}{\text{\footnotesize $#2$}}}
\newcommand{\vasymptote}[2][]{
    \draw [densely dashed,#1] ({rel axis cs:0,0} -| {axis cs:#2,0}) -- ({rel axis cs:0,1} -| {axis cs:#2,0});
}


\graphicspath{{./}{firstExample/}}

\usepackage{amsmath}
\usepackage{amssymb}
\usepackage{array}
\usepackage[makeroom]{cancel} %% for strike outs
\usepackage{pgffor} %% required for integral for loops
\usepackage{tikz}
\usepackage{tikz-cd}
\usepackage{tkz-euclide}
\usetikzlibrary{shapes.multipart}


\usetkzobj{all}
\tikzstyle geometryDiagrams=[ultra thick,color=blue!50!black]


\usetikzlibrary{arrows}
\tikzset{>=stealth,commutative diagrams/.cd,
  arrow style=tikz,diagrams={>=stealth}} %% cool arrow head
\tikzset{shorten <>/.style={ shorten >=#1, shorten <=#1 } } %% allows shorter vectors

\usetikzlibrary{backgrounds} %% for boxes around graphs
\usetikzlibrary{shapes,positioning}  %% Clouds and stars
\usetikzlibrary{matrix} %% for matrix
\usepgfplotslibrary{polar} %% for polar plots
\usepgfplotslibrary{fillbetween} %% to shade area between curves in TikZ



%\usepackage[width=4.375in, height=7.0in, top=1.0in, papersize={5.5in,8.5in}]{geometry}
%\usepackage[pdftex]{graphicx}
%\usepackage{tipa}
%\usepackage{txfonts}
%\usepackage{textcomp}
%\usepackage{amsthm}
%\usepackage{xy}
%\usepackage{fancyhdr}
%\usepackage{xcolor}
%\usepackage{mathtools} %% for pretty underbrace % Breaks Ximera
%\usepackage{multicol}



\newcommand{\RR}{\mathbb R}
\newcommand{\R}{\mathbb R}
\newcommand{\C}{\mathbb C}
\newcommand{\N}{\mathbb N}
\newcommand{\Z}{\mathbb Z}
\newcommand{\dis}{\displaystyle}
%\renewcommand{\d}{\,d\!}
\renewcommand{\d}{\mathop{}\!d}
\newcommand{\dd}[2][]{\frac{\d #1}{\d #2}}
\newcommand{\pp}[2][]{\frac{\partial #1}{\partial #2}}
\renewcommand{\l}{\ell}
\newcommand{\ddx}{\frac{d}{\d x}}

\newcommand{\zeroOverZero}{\ensuremath{\boldsymbol{\tfrac{0}{0}}}}
\newcommand{\inftyOverInfty}{\ensuremath{\boldsymbol{\tfrac{\infty}{\infty}}}}
\newcommand{\zeroOverInfty}{\ensuremath{\boldsymbol{\tfrac{0}{\infty}}}}
\newcommand{\zeroTimesInfty}{\ensuremath{\small\boldsymbol{0\cdot \infty}}}
\newcommand{\inftyMinusInfty}{\ensuremath{\small\boldsymbol{\infty - \infty}}}
\newcommand{\oneToInfty}{\ensuremath{\boldsymbol{1^\infty}}}
\newcommand{\zeroToZero}{\ensuremath{\boldsymbol{0^0}}}
\newcommand{\inftyToZero}{\ensuremath{\boldsymbol{\infty^0}}}


\newcommand{\numOverZero}{\ensuremath{\boldsymbol{\tfrac{\#}{0}}}}
\newcommand{\dfn}{\textbf}
%\newcommand{\unit}{\,\mathrm}
\newcommand{\unit}{\mathop{}\!\mathrm}
%\newcommand{\eval}[1]{\bigg[ #1 \bigg]}
\newcommand{\eval}[1]{ #1 \bigg|}
\newcommand{\seq}[1]{\left( #1 \right)}
\renewcommand{\epsilon}{\varepsilon}
\renewcommand{\iff}{\Leftrightarrow}

\DeclareMathOperator{\arccot}{arccot}
\DeclareMathOperator{\arcsec}{arcsec}
\DeclareMathOperator{\arccsc}{arccsc}
\DeclareMathOperator{\si}{Si}
\DeclareMathOperator{\proj}{proj}
\DeclareMathOperator{\scal}{scal}
\DeclareMathOperator{\cis}{cis}
\DeclareMathOperator{\Arg}{Arg}
%\DeclareMathOperator{\arg}{arg}
\DeclareMathOperator{\Rep}{Re}
\DeclareMathOperator{\Imp}{Im}
\DeclareMathOperator{\sech}{sech}
\DeclareMathOperator{\csch}{csch}
\DeclareMathOperator{\Log}{Log}

\newcommand{\tightoverset}[2]{% for arrow vec
  \mathop{#2}\limits^{\vbox to -.5ex{\kern-0.75ex\hbox{$#1$}\vss}}}
\newcommand{\arrowvec}{\overrightarrow}
\renewcommand{\vec}{\mathbf}
\newcommand{\veci}{{\boldsymbol{\hat{\imath}}}}
\newcommand{\vecj}{{\boldsymbol{\hat{\jmath}}}}
\newcommand{\veck}{{\boldsymbol{\hat{k}}}}
\newcommand{\vecl}{\boldsymbol{\l}}
\newcommand{\utan}{\vec{\hat{t}}}
\newcommand{\unormal}{\vec{\hat{n}}}
\newcommand{\ubinormal}{\vec{\hat{b}}}

\newcommand{\dotp}{\bullet}
\newcommand{\cross}{\boldsymbol\times}
\newcommand{\grad}{\boldsymbol\nabla}
\newcommand{\divergence}{\grad\dotp}
\newcommand{\curl}{\grad\cross}
%% Simple horiz vectors
\renewcommand{\vector}[1]{\left\langle #1\right\rangle}


\outcome{Solve separable differential equations}

\title{1.6 Separable Diff Eq's}

\begin{document}

\begin{abstract}
We solve separable differential equations and initial value problems.
\end{abstract}

\maketitle

\section{Differential Equations}

A differential equation is an equation that invloves one or more derivatives of an unknown function. 
Solving a differential equation entails determining the unknown function.

\begin{example}[example 1]
Solve the differential equation $y' = y$.\\
We seek $y = f(x)$ such that $f'(x) = f(x)$. By inspection $y = e^x$ is a solution of this equation.
However, $y = 2e^x$ is also a solution. In fact, there are infinitely many solutions: $y = Ce^x$,
where $C$ is any constant (including 0).
\end{example}

An initial value problem is a differential equation along with other information about the solution, usually the value of the function at a point.
The purpose of the initial value is to determine one specific solution of the differential equation, in the event that there was more than one solution.

\begin{example}[example 2]
Solve the initial value problem $y' = y, y(0) = 4$.\\
We saw in the last example that the differential equation $y' = y$ has infinitely many solutions, $y = Ce^x$, where $C$ is any constant.
The initial value $y(0) = 4$ can be used to determine the value of $C$ so that we will have a unique solution. Since $y(0) = Ce^0 = 4$, we can see that $C = 4$.
Thus the initial value problem $y' = y, y(0) = 4$ has the unique solution $y = 4e^x$.
\end{example}



\section{Separable Differential Equations}

A \textbf{separable differential equation} is a differential equation that can be put in the form
$y' = f(y)g(x)$. To solve such an equation, we rewrite $y'$ as $\frac{dy}{dx}$ and we move the $y's$ and the $dy$ to one 
side of the equation and the $x's$ and the $dx$ to the other. Then we integrate both sides and solve for $y$.

\begin{example}[example 3]
Solve the differential equation $y' = xy^2$.\\
Rewrite the equation as 
\[
\frac{dy}{dx} = xy^2.
\]
Move the $y^2$ to the left and the $dx$ to the right to get
\[
\frac{1}{y^2} \, dy = x \,  dx.
\]
Integrate both sides:
\[
\int \frac{1}{y^2}\,  dy  = \int x \, dx.
\]
This gives
\[
-\frac{1}{y} = \frac{x^2}{2} + C.
\]
Solve for $y$:
\[
\frac{1}{y} = -\frac{x^2}{2} + C
\]
Note that $-C$ has been replaced by $+C$, as this makes no difference. Now take reciprocals:
\[
y = \frac{1}{-\frac{x^2}{2} + C} = - \frac{2}{x^2 + C}.
\]
Note that $-2C$ was replaced by $+C$ as this makes no difference.
We can check the differential equation by calculuating $y'$ and observing that $y = xy^2$.
From the quotient rule, we have
\[
y' = \frac{(x^2 + C)\cdot 0 - (-2)(2x)}{(x^2 + C)^2} = \frac{4x}{(x^2 + C)^2} 
\]
\[
= x \cdot \frac{4}{(x^2 + C)^2} = x \cdot \left[\frac{2}{x^2 + C}\right]^2 = xy^2.
\]



\end{example}




\begin{example}[example 4]
Solve the differential equation $y' = x^2y$.\\
Rewrite the equation as 
\[
\frac{dy}{dx} = x^2y.
\]
Move the $y$ to the left and the $dx$ to the right to get
\[
\frac{1}{y} \, dy = x^2 \,  dx.
\]
Integrate both sides:
\[
\int \frac{1}{y}\,  dy  = \int x^2 \, dx.
\]
This gives
\[
\ln|y| = \frac{x^3}{3} + C.
\]
Solve for $y$:
\[
|y| = e^{\frac{x^3}{3} + C} = e^C \cdot e^{x^3/3}.
\]
Note that $e^C$ is a positive constant. Remove the absolute values:
\[
y = \pm e^C \cdot e^{x^3/3} = Ae^{x^3/3},
\]
where $A$ is any non-zero constant. Looking back at the original differential equation, we see that $y = 0$ is a solution, so 
we can say that $A$ is actually any constant.
We can check the differential equation by calculuating $y'$ and observing that $y = x^2y$.
From the chain rule, we have
\[
y' = Ae^{x^3/3}\cdot \frac{d}{dx} \frac{x^3}{3} = Ae^{x^3/3} \cdot x^2 = x^2 y.
\]




\end{example}











\end{document}



%more question formats below













%\begin{verbatim}
\begin{question}
What is your favorite color?
\begin{multipleChoice}
\choice[correct]{Rainbow}
\choice{Blue}
\choice{Green}
\choice{Red}
\end{multipleChoice}
\begin{freeResponse}
Hello
\end{freeResponse}
\end{question}
%\end{verbatim}





\begin{question}
  Which one will you choose?
  \begin{multipleChoice}
    \choice[correct]{I'm correct.}
    \choice{I'm wrong.}
    \choice{I'm wrong too.}
  \end{multipleChoice}
\end{question}


\begin{question}
  Which one will you choose?
  \begin{selectAll}
    \choice[correct]{I'm correct.}
    \choice{I'm wrong.}
    \choice[correct]{I'm also correct.}
    \choice{I'm wrong too.}
  \end{selectAll}
\end{question}


\begin{freeResponse}
What is the chain rule used for?
\end{freeResponse}
