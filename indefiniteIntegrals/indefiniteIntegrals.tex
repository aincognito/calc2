\documentclass{ximera}


%% You can put user macros here
%% However, you cannot make new environments



\newcommand{\ffrac}[2]{\frac{\text{\footnotesize $#1$}}{\text{\footnotesize $#2$}}}
\newcommand{\vasymptote}[2][]{
    \draw [densely dashed,#1] ({rel axis cs:0,0} -| {axis cs:#2,0}) -- ({rel axis cs:0,1} -| {axis cs:#2,0});
}


\graphicspath{{./}{firstExample/}}

\usepackage{amsmath}
\usepackage{amssymb}
\usepackage{array}
\usepackage[makeroom]{cancel} %% for strike outs
\usepackage{pgffor} %% required for integral for loops
\usepackage{tikz}
\usepackage{tikz-cd}
\usepackage{tkz-euclide}
\usetikzlibrary{shapes.multipart}


\usetkzobj{all}
\tikzstyle geometryDiagrams=[ultra thick,color=blue!50!black]


\usetikzlibrary{arrows}
\tikzset{>=stealth,commutative diagrams/.cd,
  arrow style=tikz,diagrams={>=stealth}} %% cool arrow head
\tikzset{shorten <>/.style={ shorten >=#1, shorten <=#1 } } %% allows shorter vectors

\usetikzlibrary{backgrounds} %% for boxes around graphs
\usetikzlibrary{shapes,positioning}  %% Clouds and stars
\usetikzlibrary{matrix} %% for matrix
\usepgfplotslibrary{polar} %% for polar plots
\usepgfplotslibrary{fillbetween} %% to shade area between curves in TikZ



%\usepackage[width=4.375in, height=7.0in, top=1.0in, papersize={5.5in,8.5in}]{geometry}
%\usepackage[pdftex]{graphicx}
%\usepackage{tipa}
%\usepackage{txfonts}
%\usepackage{textcomp}
%\usepackage{amsthm}
%\usepackage{xy}
%\usepackage{fancyhdr}
%\usepackage{xcolor}
%\usepackage{mathtools} %% for pretty underbrace % Breaks Ximera
%\usepackage{multicol}



\newcommand{\RR}{\mathbb R}
\newcommand{\R}{\mathbb R}
\newcommand{\C}{\mathbb C}
\newcommand{\N}{\mathbb N}
\newcommand{\Z}{\mathbb Z}
\newcommand{\dis}{\displaystyle}
%\renewcommand{\d}{\,d\!}
\renewcommand{\d}{\mathop{}\!d}
\newcommand{\dd}[2][]{\frac{\d #1}{\d #2}}
\newcommand{\pp}[2][]{\frac{\partial #1}{\partial #2}}
\renewcommand{\l}{\ell}
\newcommand{\ddx}{\frac{d}{\d x}}

\newcommand{\zeroOverZero}{\ensuremath{\boldsymbol{\tfrac{0}{0}}}}
\newcommand{\inftyOverInfty}{\ensuremath{\boldsymbol{\tfrac{\infty}{\infty}}}}
\newcommand{\zeroOverInfty}{\ensuremath{\boldsymbol{\tfrac{0}{\infty}}}}
\newcommand{\zeroTimesInfty}{\ensuremath{\small\boldsymbol{0\cdot \infty}}}
\newcommand{\inftyMinusInfty}{\ensuremath{\small\boldsymbol{\infty - \infty}}}
\newcommand{\oneToInfty}{\ensuremath{\boldsymbol{1^\infty}}}
\newcommand{\zeroToZero}{\ensuremath{\boldsymbol{0^0}}}
\newcommand{\inftyToZero}{\ensuremath{\boldsymbol{\infty^0}}}


\newcommand{\numOverZero}{\ensuremath{\boldsymbol{\tfrac{\#}{0}}}}
\newcommand{\dfn}{\textbf}
%\newcommand{\unit}{\,\mathrm}
\newcommand{\unit}{\mathop{}\!\mathrm}
%\newcommand{\eval}[1]{\bigg[ #1 \bigg]}
\newcommand{\eval}[1]{ #1 \bigg|}
\newcommand{\seq}[1]{\left( #1 \right)}
\renewcommand{\epsilon}{\varepsilon}
\renewcommand{\iff}{\Leftrightarrow}

\DeclareMathOperator{\arccot}{arccot}
\DeclareMathOperator{\arcsec}{arcsec}
\DeclareMathOperator{\arccsc}{arccsc}
\DeclareMathOperator{\si}{Si}
\DeclareMathOperator{\proj}{proj}
\DeclareMathOperator{\scal}{scal}
\DeclareMathOperator{\cis}{cis}
\DeclareMathOperator{\Arg}{Arg}
%\DeclareMathOperator{\arg}{arg}
\DeclareMathOperator{\Rep}{Re}
\DeclareMathOperator{\Imp}{Im}
\DeclareMathOperator{\sech}{sech}
\DeclareMathOperator{\csch}{csch}
\DeclareMathOperator{\Log}{Log}

\newcommand{\tightoverset}[2]{% for arrow vec
  \mathop{#2}\limits^{\vbox to -.5ex{\kern-0.75ex\hbox{$#1$}\vss}}}
\newcommand{\arrowvec}{\overrightarrow}
\renewcommand{\vec}{\mathbf}
\newcommand{\veci}{{\boldsymbol{\hat{\imath}}}}
\newcommand{\vecj}{{\boldsymbol{\hat{\jmath}}}}
\newcommand{\veck}{{\boldsymbol{\hat{k}}}}
\newcommand{\vecl}{\boldsymbol{\l}}
\newcommand{\utan}{\vec{\hat{t}}}
\newcommand{\unormal}{\vec{\hat{n}}}
\newcommand{\ubinormal}{\vec{\hat{b}}}

\newcommand{\dotp}{\bullet}
\newcommand{\cross}{\boldsymbol\times}
\newcommand{\grad}{\boldsymbol\nabla}
\newcommand{\divergence}{\grad\dotp}
\newcommand{\curl}{\grad\cross}
%% Simple horiz vectors
\renewcommand{\vector}[1]{\left\langle #1\right\rangle}


\outcome{Compute an indefinite integral}

\title{4.1 Indefinite Integrals}


\begin{document}

\begin{abstract}
In this section we learn to compute general anti-derivatives, also known as indefinite integrals.
\end{abstract}

\maketitle

\section{Anti-derivatives}
In a typical differentiation problem, we are given a function $f(x)$ and
we are asked to compute the derivative, $f'(x)$.
In a typical \textbf{anti-differentiation} problem, we are given the derivative of a function, $f'(x)$
and we are asked to compute the original function $f(x)$.  Anti-differentiation is thus the reverse process of differentiation.
There is a small twist, however as we will see in our first example.

\begin{example}[example 1]
Given that $f'(x) = 2x$, find $f(x)$. \\
By now, we are all familiar with the differentiation formula:
\[
\frac{d}{dx} x^2 = 2x,
\]
and this leads us the the answer:
\[
f(x) = x^2.
\]
So, the anti-derivative of $2x$ (with respect to $x$) is $x^2$.
But wait, notice that the derivative of $f(x) = x^2 + 5$ is also $2x$.
so $f(x) = x^2 + 5$ is also a solution to this anti-differentiation problem.
We now recognize that there are actually infinitely many solutions to this problem, 
because if $C$ is any constant, then
\[
\frac{d}{dx} \left(x^2 + C\right) = 2x.
\]
Thus the answer to the anti-differentiation problem is actually not just a single function but a family of functions each differing by a constant.
The graph below shows several members of this family.
\[
\graph{x^2 - 5, x^2 - 4, x^2 - 3, x^2 - 2, x^2 - 1, x^2, x^2 +1, x^2 +2, x^2 +3, x^2 +4, x^2 +5}
\]

\end{example}

Anti-differentiation problems arise naturally in projectile motion scenarios.


\begin{example}[example 2]
Suppose the velocity of a projectile is given by the equation $v(t) = 40 - 32t$. Find the position function $s(t)$.\\
We recall from previous discussions of rectilinear motion, of which projectile motion is a special case, that
\[
s'(t) = v(t).
\]
Hence to recover $s(t)$ from $v(t)$ as in this problem, we must anti-differentiate $v(t)$.
Restating the problem in terms of differentiation, we need to find the function $s(t)$ such that $s'(t) = 40 - 32t$.
Our familiarity with differentiating polynomials leads us to guess
\[
s(t) = 40t - 16t^2.
\]
We can verify that this is a correct answer by differentiating it:
\[
\frac{d}{dt}\left(40t - 16t^2\right) = 40 - 32t.
\]
But let us not forget the lesson of the previous example: there is more than one answer.  There answer is a family of functions
each pair differing by a constant. So a complete solution set to our problem is
\[
s(t) = 40t - 16t^2 + C.
\]
Noting that the constant in this equation represents the position of the object at time $t = 0$,
we denote this particular constant as $s_0$, the \textbf{initial position} of the object and we can write our answer as 
$s(t) = s_0 + 40t - 16t^2$.


\end{example}






We see from the previous example that in order to know the exact position of the object, 
we need more information than just the velocity function.
If we were also given the initial position of the object (or the position at any time), 
then we can find the precise position function.
We do this in the next example.

\begin{example}[example 3]
Suppose the velocity of a projectile (in ft/sec) is given by the equation $v(t) = 40 - 32t$. 
Find the height of the projectile at time $t = 1.5$ seconds, 
given that the projectile lands at time $t = 3$ seconds. \\

We recall from previous example that 
\[
s(t) = s_0 + 40t - 16t^2.
\]
The problem requires us to find $s(1.5)$ and in order to do this, we must first find $s_0$, 
the initial launch height of the projectile.
We are given that the object lands after three seconds, so $s(3) = 0$. We can use this to find $s_0$
by plugging $t=3$ into the position function:
\[
s(3) = s_0 + 40(3) - 16(3^2) = 0.
\]
Solving the second equation for $s_0$ yields $s_0 = 144 - 120 = 24$ ft.
Thus the position function is
\[
s(t) = 24 + 40t - 16t^2.
\]
Finally, we can find the position at time $t=1.5$ seconds
\[
s(1.5) = 24 + 40(1.5) - 16(1.5^2) = 24 + 60 - 16(2.25) = 84 - 36 = 48 \; \text{ft}.
\]
\end{example}


\begin{problem}(problem 3a)
Given $f'(x)$, find the family of anti-derivatives. Don't forget to add $+C$ to your answers.
\[
f'(x) = 3x + 5; \;\; f(x) = \answer{3/2x^2 + 5x +C}.
\]
\[
f'(x) = 2\cos(x) - 7\sin(x); \;\; f(x) = \answer{2\sin(x) + 7\cos(x) +C}.
\]
\[
f'(x) = e^{4x}; \;\; f(x) = \answer{1/4e^{4x} +C}.
\]

\end{problem}



\begin{problem}(problem 3b)
Find the precise anti-derivatives satisfying the given initial condition:
\[
f'(x) = x^3 + x, f(0) = 1;\;\; f(x) = \answer{1/4x^4 + 1/2x^2 + 1}.
\]
\[
f'(x) = 2\sec^2(x), f(\pi/4) = 1; \;\; f(x) = \answer{2\tan(x) - 1}.
\]
\[
f'(x) = 1/x, f(1) = 7/4; \;\; f(x) = \answer{7/4 + \ln(x)}.
\]

\end{problem}



\begin{problem}(problem 3c)
Find the height of the projectile at time $t = 2$ seconds if the total time in the air is 5 seconds 
and the velocity at time $t$ seconds is given in ft/sec as $v(t) = 80-32t$.\\
The initial height of the projectile is $s_0 = \answer{0}$ft.\\
The position at time $t$ is given by $s(t) = \answer{80t - 16t^2}$.\\
The height at time $t = 2$ seconds is $\answer{96}$ft.\\
\end{problem}



It turns out that the process of anti-differentiation is directly tied to finding the area under a curve.
We will study the area problem in detail in an upcoming lesson, but for now, we will learn an important notation that
is used in the discussion of the area problem.



\section{The Indefinite Integral}

The \textbf{indefinite integral} is denoted by $\int f(x) \ dx$ and it is read ``the integral of $f(x)$ dx." To compute an indefinite integral we need to 
find all anti-derivatives, $F(x) + C$ of $f(x)$. Notice the change in the notation from $f'(x)$ and $f(x)$
to $f(x)$ and $F(x)$. The process of computing an integral is called \textbf{integration} in addition to anti-differentiation.  
Since any two (continuous) functions that have the same derivative differ by a constant, 
if we can find one example of a function $F(x)$ whose derivative is $f(x)$, i.e., $F'(x) = f(x)$, then we can write
\[\int f(x) \ dx = F(x) + C\]
where C is any constant.  The symbol $\int$ is called the \textbf{integral sign}, 
the function $f(x)$ is called the \textbf{integrand} and $C$ is called the \textbf{constant of integration}.  

\section{Examples of Basic Indefinite Integrals}

A \textbf{basic indefinite integral} is one that can be computed either by recognizing the integrand as the 
derivative of a familiar function or by reversing the Power Rule for Derivatives.



\begin{example}[example 4]
\[
\int \cos(x) \ dx = \sin(x) + C,
\]
because the derivative of $\sin(x)$ is $\cos(x)$.

\end{example}

\begin{problem}(problem 4a)

\begin{hint}
\[
\frac{d}{dx} \cos(x) = -\sin(x)
\]
\end{hint}
\begin{hint}
\begin{center}
Do not add the +C to your answer
\end{center}
\end{hint}

\[
\int \sin(x) \ dx =
\answer[given]{-\cos(x)} \ + C
\]
\end{problem}

\begin{problem}(problem 4b)

\begin{hint}
\[
\frac{d}{dx} \sin(3x) = 3\cos(3x)
\]
\end{hint}
\begin{hint}
The answer is a minor modification of $\sin(3x)$
\end{hint}
\begin{hint}
\begin{center}
Do not add the +C to your answer
\end{center}
\end{hint}

\[
\int \cos(3x) \ dx =
\answer[given]{\frac13 \sin(3x)} \ + C
\]
\end{problem}

\begin{example}[example 5]
\[
\int \sec^2(x) \ dx = \tan(x) + C,
\]
because the derivative of $\tan(x)$ is $\sec^2(x)$.

\end{example}

\begin{problem}(problem 5a)
Compute 
%\[
%\int \csc^2(x) \ dx.
%\]

\begin{hint}
\[
\frac{d}{dx} \cot(x) = -\csc^2(x)
\]
\end{hint}
\begin{hint}
\begin{center}
Do not add the +C to your answer
\end{center}
\end{hint}

\[
\int \csc^2(x) \ dx =
\answer[given]{-\cot(x)} \ + C
\]
\end{problem}

\begin{problem}(problem 5b)
Compute 
%\[
%\int \sec^2(4x) \ dx.
%\]

\begin{hint}
\[
\frac{d}{dx} \tan(4x) = 4\sec^2(4x)
\]
\end{hint}
\begin{hint}
The answer is a minor modification of $\tan(4x)$
\end{hint}
\begin{hint}
\begin{center}
Do not add the +C to your answer
\end{center}
\end{hint}

\[
\int \sec^2(4x) \ dx =
\answer[given]{\frac14\tan(4x)} \ + C
\]
\end{problem}

\begin{example}[example 6]
\[
\int \sec(x)\tan(x) \ dx = \sec(x) + C,
\]
because the derivative of $\sec(x)$ is $\sec(x)\tan(x)$.

\end{example}


\begin{problem}(problem 6)
Compute 
%\[
%\int \csc(x)\cot(x) \ dx.
%\]

\begin{hint}
\[
\frac{d}{dx} \csc(x) = -\csc(x)\cot(x)
\]
\end{hint}
\begin{hint}
\begin{center}
Do not add the +C to your answer
\end{center}
\end{hint}

\[
\int \csc(x)\cot(x) \ dx =
\answer[given]{-\csc(x)} \ + C
\]
\end{problem}



\begin{example}[example 7]
\[
\int e^x \ dx = e^x + C,
\]
because the derivative of $e^x$ is $e^x$.

\end{example}

\begin{example} %example 4
\[
\int e^{2x} \ dx = \frac12 e^{2x} + C,
\]
because the derivative of $e^{2x}$ is $2e^{2x}$, by the chain rule,
and hence the derivative of $\frac12 e^{2x}$ is $e^{2x}$, by the constant multiple rule.

\end{example}

\begin{problem}(problem 7a)
Compute 
%\[
%\int 2^x \ln(2) \ dx.
%\]

\begin{hint}
\[
\frac{d}{dx} 2^x = 2^x \ln(2)
\]
\end{hint}
\begin{hint}
\begin{center}
Do not add the +C to your answer
\end{center}
\end{hint}

\[
\int 2^x \ln(2) \ dx =
\answer[given]{2^x} \ + C
\]
\end{problem}

\begin{problem}(problem 7b)
Compute 
%\[
%\int e^{5x} \ dx.
%\]

\begin{hint}
\[
\frac{d}{dx} e^{5x} = 5e^{5x}
\]
\end{hint}
\begin{hint}
The answer is a minor modification of $e^{5x}$
\end{hint}
\begin{hint}
\begin{center}
Do not add the +C to your answer
\end{center}
\end{hint}

\[
\int e^{5x} \ dx =
\answer[given]{\frac15 e^{5x}} \ + C
\]
\end{problem}

\begin{problem}(problem 7c)
Compute 
%\[
%\int e^{-x/2} \ dx.
%\]

\begin{hint}
\[
\frac{d}{dx} e^{-x/2} = -\frac12 e^{-x/2}
\]
\end{hint}
\begin{hint}
The answer is a minor modification of $e^{-x/2}$
\end{hint}
\begin{hint}
\begin{center}
Do not add the +C to your answer
\end{center}
\end{hint}

\[
\int e^{-x/2} \ dx =
\answer[given]{-2 e^{-x/2}} \ + C
\]
\end{problem}

\begin{example}[example 8]
\[
\int 2x \ dx = x^2 + C,
\]
because the derivative of $x^2$ is $2x$.

\end{example}


\begin{problem}(problem 8)
Compute 
%\[
%\int 3x^2 \ dx.
%\]

\begin{hint}
\[
\frac{d}{dx} x^3 = 3x^2
\]
\end{hint}
\begin{hint}
\begin{center}
Do not add the +C to your answer
\end{center}
\end{hint}

\[
\int 3x^2 \ dx =
\answer[given]{x^3} \ + C
\]
\end{problem}


\begin{example}[example 9]
\[
\int \frac{1}{2\sqrt x} \ dx = \sqrt x + C,
\]
because the derivative of $\sqrt x$ is $\frac{1}{2\sqrt x}$.

\end{example}



\begin{problem}(problem 9)
Compute 
%\[
%\int \frac{1}{2\sqrt w} \ dw.
%\]

\begin{hint}
\[
\frac{d}{dx} \sqrt x = \frac{1}{2\sqrt x}
\]
\end{hint}
\begin{hint}
Note that the variable is $w$
\end{hint}
\begin{hint}
Write your answer as $w$ to a power.
\end{hint}
\begin{hint}
\begin{center}
Do not add the +C to your answer
\end{center}
\end{hint}

\[
\int \frac{1}{2\sqrt w} \ dw =
\answer[given]{w^{1/2}} \ + C
\]
\end{problem}



\begin{example}[example 10]
\[
\int \frac{1}{1 + x^2} \ dx = \tan^{-1}(x) + C,
\]
because the derivative 
of $\tan^{-1}(x)$ is $\frac{1}{1 + x^2}$.

\end{example}

\begin{problem} (problem 10) %problem 7
Compute 
%\[
%\int \frac{1}{1+u^2} \ du.
%\]

\begin{hint}
\[
\frac{d}{dx} tan^{-1}(x) = \frac{1}{1+x^2}
\]
\end{hint}
\begin{hint}
Note that the variable is $u$
\end{hint}
\begin{hint}
\begin{center}
Do not add the +C to your answer
\end{center}
\end{hint}

\[
\int \frac{1}{1+u^2} \ du =
\answer[given]{\tan^{-1}(u)} \ + C
\]
\end{problem}(problem 10)


\begin{example}[example 11]
\[
\int \frac{1}{\sqrt{1 - x^2}} \ dx = \sin^{-1}(x) + C,
\]
because the derivative of $\sin^{-1}(x)$ 
is $\frac{1}{\sqrt{1 - x^2}}$.

\end{example}

\begin{problem}(problem 11)
Compute 
%\[
%\int \frac{1}{\sqrt{1 - t^2}} \ dt.
%\]

\begin{hint}
\[
\frac{d}{dx} \cos^{-1}(x) = -\frac{1}{\sqrt{1 - x^2}}
\]
\end{hint}
\begin{hint}
Note that the variable is $t$
\end{hint}
\begin{hint}
\begin{center}
Do not add the +C to your answer
\end{center}
\end{hint}

\[
\int -\frac{1}{\sqrt{1 - t^2}} \ dt =
\answer[given]{\cos^{-1}(t)} \ + C
\]
\end{problem}

\section{Power Rule for Indefinite Integrals}

The Power Rule for Indefinite Integrals reverses the Power Rule for Derivatives.
Instead of subtracting 1 from the exponent, we add 1 and instead of multiplying by the exponent, we divide.

\begin{theorem} Power Rule for Indefinite Integrals
\[
\int x^n \ dx = \frac{x^{n+1}}{n+1} +C 
\]
where $n$ is any number except $-1$.
\end{theorem}

%\begin{center}
%\bf{Examples of the Power Rule for Indefinite Integrals}
%\end{center}



\begin{example}[example 12]
Compute $$\int x^3 \ dx.$$
We use the Power Rule with $n= 3$ and we get
\[\int x^3 \ dx = \frac{x^4}{4} + C.\]  The answer can be confirmed by observing that the 
derivative of $\frac{x^4}{4} +C$ is $x^3$ (using the Constant Multiple Rule and the Power Rule for Derivatives).
\end{example}

\begin{problem}(problem 12a)
Compute 
%\[
%\int x^4 \ dx.
%\]

\begin{hint}
Use the Power Rule with $n=4$
\end{hint}
\begin{hint}
The Power Rule says $\int x^n \ dx = \frac{x^{n+1}}{n+1} +$ C
\end{hint}
\begin{hint}
\begin{center}
Do not add the +C to your answer
\end{center}
\end{hint}

\[
\int x^4 \ dx =
\answer[given]{x^5 /5} \ + C
\]
\end{problem}


\begin{problem}(problem 12b)
Compute 
%\[
%\int x^6 \ dx.
%\]

\begin{hint}
Use the Power Rule with $n=6$
\end{hint}
\begin{hint}
The Power Rule says $\int x^n \ dx = \frac{x^{n+1}}{n+1} +$ C
\end{hint}
\begin{hint}
\begin{center}
Do not add the +C to your answer
\end{center}
\end{hint}

\[
\int x^6 \ dx =
\answer[given]{x^7 /7} \ + C
\]
\end{problem}


\begin{example}[example 13]
Compute $$\int 1 \ dx.$$
We use the Power Rule with $n= 0$  ($x^0 = 1$) and we get
\[\int 1 \ dx = \int x^0 \ dx =\frac{x^1}{1} + C = x+C.\] The answer can be confirmed by observing that the 
derivative of $x + C$ is $1$ 
\end{example}

\begin{problem}(problem 13a)
Compute 
%\[
%\int 1 \ dt.
%\]


\begin{hint}
Use the Power Rule with $n=0$
\end{hint}
\begin{hint}
The Power Rule says $\int x^n \ dx = \frac{x^{n+1}}{n+1} +$ C
\end{hint}
\begin{hint}
Note that the variable is $t$
\end{hint}
\begin{hint}
\begin{center}
Do not add the +C to your answer
\end{center}
\end{hint}

\[
\int 1 \ dt =
\answer[given]{t} \ + C
\]
\end{problem}


\begin{problem}(problem 13b)
Compute 
\[
\int 1 \ du.
\]


\begin{hint}
Use the Power Rule with $n=0$
\end{hint}
\begin{hint}
The Power Rule says $\int x^n \ dx = \frac{x^{n+1}}{n+1} +$ C
\end{hint}
\begin{hint}
Note that the variable is $u$
\end{hint}
\begin{hint}
\begin{center}
Do not add the +C to your answer
\end{center}
\end{hint}

\[
\int 1 \ du =
\answer[given]{u} \ + C
\]
\end{problem}


\begin{example}[example 14]
Compute $$\int x \ dx.$$
We use the Power Rule with $n= 1$ and we get
\[\int x \ dx = \frac{x^2}{2} + C.\] The answer can be confirmed by observing that the 
derivative of $\frac{x^2}{2} +C$ is $x$ (using the constant multiple rule and the Power Rule for Derivatives).
\end{example}


\begin{problem}(problem 14a)
Compute 
%\[
%\int t \ dt.
%\]

\begin{hint}
Use the Power Rule with $n=1$
\end{hint}
\begin{hint}
The Power Rule says $\int x^n \ dx = \frac{x^{n+1}}{n+1} +$ C
\end{hint}
\begin{hint}
Note that the variable is $t$
\end{hint}
\begin{hint}
\begin{center}
Do not add the +C to your answer
\end{center}
\end{hint}

\[
\int t \ dt =
\answer[given]{t^2 /2} \ + C
\]
\end{problem}



\begin{problem}(problem 14b)
Compute 
%\[
%\int u \ du.
%\]

\begin{hint}
Use the Power Rule with $n=1$
\end{hint}
\begin{hint}
The Power Rule says $\int x^n \ dx = \frac{x^{n+1}}{n+1} +$ C
\end{hint}
\begin{hint}
Note that the variable is $u$
\end{hint}
\begin{hint}
\begin{center}
Do not add the +C to your answer
\end{center}
\end{hint}

\[
\int u \ du =
\answer[given]{u^2 /2} \ + C
\]
\end{problem}


\begin{example}[example 15]
Compute $$\int \sqrt x \ dx.$$
We rewrite $\sqrt x$ as $x^{1/2}$ and we use the Power Rule with $n= 1/2$ and we get
\begin{align*}
\int \sqrt x \ dx &= \int x^{1/2} \ dx\\
&= \frac{x^{3/2}}{3/2} + C\\
&= \tfrac{2}{3}x^{3/2} + C\\ 
&= \tfrac23 \sqrt {x^3} + C\\
&= \tfrac23 x\sqrt {x} + C.
\end{align*}
\end{example}

\begin{problem}(problem 15a)
Compute 
%\[
%\int \sqrt[3]x \ dx.
%\]

\begin{hint}
Rational exponents: $\sqrt[3]x = x^{1/3}$
\end{hint}
\begin{hint}
Use the Power Rule with $n=1/3$
\end{hint}
\begin{hint}
The Power Rule says $\int x^n \ dx = \frac{x^{n+1}}{n+1} +$ C
\end{hint}
\begin{hint}
\begin{center}
Do not add the +C to your answer
\end{center}
\end{hint}

\[
\int \sqrt[3]x \ dx =
\answer[given]{3x^{4/3} /4} \ + C
\]
\end{problem}


\begin{problem}(problem 15b)
Compute 
%\[
%\int \sqrt[4]x \ dx.
%\]

\begin{hint}
Rational exponents: $\sqrt[4]x = x^{1/4}$
\end{hint}
\begin{hint}
Use the Power Rule with $n=1/4$
\end{hint}
\begin{hint}
The Power Rule says $\int x^n \ dx = \frac{x^{n+1}}{n+1} +$ C
\end{hint}
\begin{hint}
\begin{center}
Do not add the +C to your answer
\end{center}
\end{hint}

\[
\int \sqrt[4]x \ dx =
\answer[given]{4x^{5/4} /5} \ + C
\]
\end{problem}



\begin{example}[example 16]
Compute $$\int \frac{1}{x^3} \ dx.$$
We rewrite $\frac{1}{x^3}$ as $x^{-3}$ and use the Power Rule with $n= -3$.
We get
\begin{align*}
\int \frac{1}{x^3} \ dx &= \int x^{-3} \ dx \\
&= \frac{x^{-2}}{-2} + C\\ 
&= -\tfrac{1}{2}x^{-2} + C \\
&= -\tfrac{1}{2}\cdot\frac{1}{x^2} + C \\
&= -\frac{1}{2x^2} +C.
\end{align*}
\end{example}

\begin{problem}(problem 16a)
Compute 
%\[
%\int \frac{1}{x^4} \ dx.
%\]

\begin{hint}
Negative exponents: $\frac{1}{x^4} = x^{-4}$
\end{hint}
\begin{hint}
Use the Power Rule with $n=-4$
\end{hint}
\begin{hint}
The Power Rule says $\int x^n \ dx = \frac{x^{n+1}}{n+1} +$ C
\end{hint}
\begin{hint}
\begin{center}
Do not add the +C to your answer
\end{center}
\end{hint}

\[
\int \frac{1}{x^4} \ dx =
\answer[given]{-x^{-3} /3} \ + C
\]
\end{problem}


\begin{problem}(problem 16b)
Compute 
%\[
%\int \frac{1}{x^7} \ dx.
%\]

\begin{hint}
Negative exponents: $\frac{1}{x^7} = x^{-7}$
\end{hint}
\begin{hint}
Use the Power Rule with $n=-7$
\end{hint}
\begin{hint}
The Power Rule says $\int x^n \ dx = \frac{x^{n+1}}{n+1} +$ C
\end{hint}
\begin{hint}
\begin{center}
Do not add the +C to your answer
\end{center}
\end{hint}

\[
\int \frac{1}{x^7} \ dx =
\answer[given]{-x^{-6} /6} \ + C
\]
\end{problem}



\begin{example}[example 17]
Compute $$\int \frac{1}{\sqrt x} \ dx.$$
We rewrite $\frac{1}{\sqrt x}$ as $x^{-1/2}$ and use the Power Rule 
with $n= -1/2$.
We get
\begin{align*}
\int \frac{1}{\sqrt x} \ dx &= \int x^{-1/2} \ dx \\
&= \frac{x^{1/2}}{1/2} + C \\ 
&=  2x^{1/2} + C \\
&= 2\sqrt x +C.
\end{align*} 

\end{example}


\begin{problem}(problem 17a)
Compute 
%\[
%\int \frac{1}{\sqrt[3]x} \ dx.
%\]

\begin{hint}
Negative rational exponents: $\frac{1}{\sqrt[3]x} = x^{-1/3}$
\end{hint}
\begin{hint}
Use the Power Rule with $n=-1/3$
\end{hint}
\begin{hint}
The Power Rule says $\int x^n \ dx = \frac{x^{n+1}}{n+1} +$ C
\end{hint}
\begin{hint}
\begin{center}
Do not add the +C to your answer
\end{center}
\end{hint}

\[
\int \frac{1}{\sqrt[3]x} \ dx =
\answer[given]{3x^{2/3} /2} \ + C
\]
\end{problem}



\begin{problem}(problem 17b)
Compute 
%\[
%\int \frac{1}{\sqrt[6]x} \ dx.
%\]

\begin{hint}
Negative rational exponents: $\frac{1}{\sqrt[6]x} = x^{-1/6}$
\end{hint}
\begin{hint}
Use the Power Rule with $n=-1/6$
\end{hint}
\begin{hint}
The Power Rule says $\int x^n \ dx = \frac{x^{n+1}}{n+1} +$ C
\end{hint}
\begin{hint}
\begin{center}
Do not add the +C to your answer
\end{center}
\end{hint}

\[
\int \frac{1}{\sqrt[6]x} \ dx =
\answer[given]{6x^{5/6} /5} \ + C
\]
\end{problem}



\begin{center}
\textbf{Special Case}
\end{center}

The Power Rule does not work when $n = -1$ so we consider this as a special case:

\begin{example}[example 18]
\[\int x^{-1} \ dx = \int \frac{1}{x}  \ dx = \ln |x| +C.\]
We can verify this by noting that the derivative of $\ln|x| + C$ is $\displaystyle{\frac{1}{x}}$.
\end{example}

\begin{problem}(problem 18a)
Compute 
%\[
%\int \frac{1}{t} \ dt.
%\]

\begin{hint}
Negative exponents: $\frac{1}{x} = x^{-1}$
\end{hint}
\begin{hint}
The Power Rule does not apply when $n = -1$
\end{hint}
\begin{hint}
Recall $\frac{d}{dx} \ln|x| = \frac{1}{x}$
\end{hint}
\begin{hint}
\begin{center}
Do not add the +C to your answer
\end{center}
\end{hint}

\[
\int \frac{1}{t} \ dt =
\answer[given]{\ln|t|} \ + C
\]
\end{problem}

\begin{problem}(problem 18b)
Compute 
%\[
%\int \frac{1}{u} \ du.
%\]

\begin{hint}
Negative exponents: $\frac{1}{x} = x^{-1}$
\end{hint}
\begin{hint}
The Power Rule does not apply when $n = -1$
\end{hint}
\begin{hint}
Recall $\frac{d}{dx} \ln|x| = \frac{1}{x}$
\end{hint}
\begin{hint}
\begin{center}
Do not add the +C to your answer
\end{center}
\end{hint}

\[
\int \frac{1}{u} \ du =
\answer[given]{\ln|u|} \ + C
\]
\end{problem}


\end{document}
















We can handle sums, differences and constant multiples in indefinite integrals 
the exact same way we handle them in differentiation.


Examples

\begin{center}
\bf{Constant Multiples}
\end{center}

\begin{description}

\item[CM 1.] $\displaystyle{\int 5\cos(x) \ dx =  5\sin(x) +C}$.

\item[CM 2.] $\displaystyle{\int 3\sec^2(x) \ dx =  3\tan(x) +C}$.

\item[CM 3.] $\displaystyle{\int 2e^x \ dx =  2e^x +C}$.

\item[CM 4.] $\displaystyle{\int \ffrac{4}{1+x^2} \ dx  = 4\tan^{-1}(x) +C}$.

\item[CM 5.] $\displaystyle{\int \ffrac{7}{12x} \ dx =  \ffrac{7}{12} \ln|x| +C}$.

\item[CM 6.] $\displaystyle{\int 4x^7 \ dx  = 4\cdot \frac{x^8}{\mbox{\footnotesize $8$}} +C 
= \frac{x^8}{\mbox{\footnotesize $2$}}}$.

\item[CM 7.] $\begin{aligned}[t]
\int \frac{3}{5x^2} \ dx &= \int \ffrac{3}{5}x^{-2} \ dx \\[3pt]
&= \ffrac{3}{5} \cdot \ffrac{x^{-1}}{-1} +C \\[5pt]
&= -\ffrac{3}{5} \cdot \ffrac{1}{x} +C \\[5pt]
&= -\ffrac{3}{5x} +C.
\end{aligned}$


\item[CM 8.] $\displaystyle{\int 4\sin(x) \ dx  = -4\cos(x) +C}$.

\item[CM 9.] $\displaystyle{\int 3\csc(x)\cot(x) \ dx = -3\csc(x) +C}$.


\item[CM 10.] $\begin{aligned}[t]
\int 4\sqrt[3]x  \ dx &=  \int 4x^{1/3} \ dx \\
&= 4 \cdot \frac{x^{4/3}}{\mbox{\footnotesize $4/3$}} +C \\[3pt]
&= 4 \cdot \ffrac{3}{4} \cdot x^{4/3} +C \\[3pt]
&= 3x^{4/3} +C \\
&= 3\sqrt[3] {x^4} +C \\
&= 3x\sqrt[3] x +C.
\end{aligned}$


\end{description}


\begin{center}
\bf{Sums}
\end{center}

\begin{description}

\item[SR 1.] $\int \cos(x) + e^x \ dx = \sin(x) + e^x + C.$

\item[SR 2.] $\int x + 2\sec^2(x) \ dx = \frac{x^2}{2} + 2\tan(x) + C.$

\item[SR 3.] $\int x^2 + \frac{1}{x^2} \ dx = \int x^2 + x^{-2} \ dx = \frac{x^3}{3} + \frac{x^{-1}}{\mbox{\footnotesize $-1$}} + C 
= \frac{x^3}{3} - \frac{1}{x} + C$

\item[SR 4.] $\int \frac{x+2}{x} \ dx = \int \frac{x}{x} + \frac{2}{x} \ dx = \int 1 + \frac{2}{x} \ dx = x + 2\ln|x| +C.$

\item[SR 5.] $\begin{aligned}[t]
\int \sqrt x + \sqrt[3] x\ dx &= \int x^{1/2} + x^{1/3} \ dx \\
&= \frac{x^{3/2}}{\mbox{\footnotesize $3/2$}} + \frac{x^{4/3}}{\mbox{\footnotesize $4/3$}} + C \\
&= \tfrac{2}{3} x^{3/2} + \tfrac{3}{4} x^{4/3} + C \\
&= \tfrac{2}{3} \sqrt {x^3} + \tfrac{3}{4}  \sqrt[3] {x^4} + C \\
&= \tfrac{2}{3} x\sqrt x + \tfrac{3}{4} x \sqrt[3] x + C.
\end{aligned}$

\item[SR 6.] $\int x^2 + 3x + 4 \ dx = \frac{x^3}{\mbox{\footnotesize $3$}} + \frac{3x^2}{\mbox{\footnotesize $2$}} + 4x + C.$

\end{description}

\begin{center}
\bf{Differences}
\end{center}

\begin{description}

\item[DR 1.] $\int (1 - e^x) \ dx = x - e^x + C.$

\item[DR 2.] $\int (x^2 - x) \ dx = \frac{x^3}{\mbox{\footnotesize $3$}} - \frac{x^2}{\mbox{\footnotesize $2$}} + C.$

\item[DR 3.] $\int [3\cos(x) - 2\sin(x)] \ dx = 3\sin(x) + 2\cos(x) + C.$

\item[DR 4.] $\int (3x^2 - \tfrac{3}{x} - \tfrac{1}{1+x^2}) \ dx = x^3 - 3\ln|x| - \tan^{-1}(x) + C.$

\item[DR 5.] $\int [4\csc(x)(\csc(x) - \cot(x))] \ dx = \int [4\csc^2(x)  - 4\csc(x)\cot(x)] \ dx = -4\cot(x) + 4\csc(x) + C
= 4\csc(x) - 4 \cot(x) + C.$

\end{description}


\begin{center}
\begin{foldable}
\unfoldable{Here is a detailed, lecture style video on indefinite integrals:}
\youtube{NAY2sGMKi5U}
\end{foldable}
\end{center}

\end{document}
