\documentclass{ximera}

%% You can put user macros here
%% However, you cannot make new environments



\newcommand{\ffrac}[2]{\frac{\text{\footnotesize $#1$}}{\text{\footnotesize $#2$}}}
\newcommand{\vasymptote}[2][]{
    \draw [densely dashed,#1] ({rel axis cs:0,0} -| {axis cs:#2,0}) -- ({rel axis cs:0,1} -| {axis cs:#2,0});
}


\graphicspath{{./}{firstExample/}}

\usepackage{amsmath}
\usepackage{amssymb}
\usepackage{array}
\usepackage[makeroom]{cancel} %% for strike outs
\usepackage{pgffor} %% required for integral for loops
\usepackage{tikz}
\usepackage{tikz-cd}
\usepackage{tkz-euclide}
\usetikzlibrary{shapes.multipart}


\usetkzobj{all}
\tikzstyle geometryDiagrams=[ultra thick,color=blue!50!black]


\usetikzlibrary{arrows}
\tikzset{>=stealth,commutative diagrams/.cd,
  arrow style=tikz,diagrams={>=stealth}} %% cool arrow head
\tikzset{shorten <>/.style={ shorten >=#1, shorten <=#1 } } %% allows shorter vectors

\usetikzlibrary{backgrounds} %% for boxes around graphs
\usetikzlibrary{shapes,positioning}  %% Clouds and stars
\usetikzlibrary{matrix} %% for matrix
\usepgfplotslibrary{polar} %% for polar plots
\usepgfplotslibrary{fillbetween} %% to shade area between curves in TikZ



%\usepackage[width=4.375in, height=7.0in, top=1.0in, papersize={5.5in,8.5in}]{geometry}
%\usepackage[pdftex]{graphicx}
%\usepackage{tipa}
%\usepackage{txfonts}
%\usepackage{textcomp}
%\usepackage{amsthm}
%\usepackage{xy}
%\usepackage{fancyhdr}
%\usepackage{xcolor}
%\usepackage{mathtools} %% for pretty underbrace % Breaks Ximera
%\usepackage{multicol}



\newcommand{\RR}{\mathbb R}
\newcommand{\R}{\mathbb R}
\newcommand{\C}{\mathbb C}
\newcommand{\N}{\mathbb N}
\newcommand{\Z}{\mathbb Z}
\newcommand{\dis}{\displaystyle}
%\renewcommand{\d}{\,d\!}
\renewcommand{\d}{\mathop{}\!d}
\newcommand{\dd}[2][]{\frac{\d #1}{\d #2}}
\newcommand{\pp}[2][]{\frac{\partial #1}{\partial #2}}
\renewcommand{\l}{\ell}
\newcommand{\ddx}{\frac{d}{\d x}}

\newcommand{\zeroOverZero}{\ensuremath{\boldsymbol{\tfrac{0}{0}}}}
\newcommand{\inftyOverInfty}{\ensuremath{\boldsymbol{\tfrac{\infty}{\infty}}}}
\newcommand{\zeroOverInfty}{\ensuremath{\boldsymbol{\tfrac{0}{\infty}}}}
\newcommand{\zeroTimesInfty}{\ensuremath{\small\boldsymbol{0\cdot \infty}}}
\newcommand{\inftyMinusInfty}{\ensuremath{\small\boldsymbol{\infty - \infty}}}
\newcommand{\oneToInfty}{\ensuremath{\boldsymbol{1^\infty}}}
\newcommand{\zeroToZero}{\ensuremath{\boldsymbol{0^0}}}
\newcommand{\inftyToZero}{\ensuremath{\boldsymbol{\infty^0}}}


\newcommand{\numOverZero}{\ensuremath{\boldsymbol{\tfrac{\#}{0}}}}
\newcommand{\dfn}{\textbf}
%\newcommand{\unit}{\,\mathrm}
\newcommand{\unit}{\mathop{}\!\mathrm}
%\newcommand{\eval}[1]{\bigg[ #1 \bigg]}
\newcommand{\eval}[1]{ #1 \bigg|}
\newcommand{\seq}[1]{\left( #1 \right)}
\renewcommand{\epsilon}{\varepsilon}
\renewcommand{\iff}{\Leftrightarrow}

\DeclareMathOperator{\arccot}{arccot}
\DeclareMathOperator{\arcsec}{arcsec}
\DeclareMathOperator{\arccsc}{arccsc}
\DeclareMathOperator{\si}{Si}
\DeclareMathOperator{\proj}{proj}
\DeclareMathOperator{\scal}{scal}
\DeclareMathOperator{\cis}{cis}
\DeclareMathOperator{\Arg}{Arg}
%\DeclareMathOperator{\arg}{arg}
\DeclareMathOperator{\Rep}{Re}
\DeclareMathOperator{\Imp}{Im}
\DeclareMathOperator{\sech}{sech}
\DeclareMathOperator{\csch}{csch}
\DeclareMathOperator{\Log}{Log}

\newcommand{\tightoverset}[2]{% for arrow vec
  \mathop{#2}\limits^{\vbox to -.5ex{\kern-0.75ex\hbox{$#1$}\vss}}}
\newcommand{\arrowvec}{\overrightarrow}
\renewcommand{\vec}{\mathbf}
\newcommand{\veci}{{\boldsymbol{\hat{\imath}}}}
\newcommand{\vecj}{{\boldsymbol{\hat{\jmath}}}}
\newcommand{\veck}{{\boldsymbol{\hat{k}}}}
\newcommand{\vecl}{\boldsymbol{\l}}
\newcommand{\utan}{\vec{\hat{t}}}
\newcommand{\unormal}{\vec{\hat{n}}}
\newcommand{\ubinormal}{\vec{\hat{b}}}

\newcommand{\dotp}{\bullet}
\newcommand{\cross}{\boldsymbol\times}
\newcommand{\grad}{\boldsymbol\nabla}
\newcommand{\divergence}{\grad\dotp}
\newcommand{\curl}{\grad\cross}
%% Simple horiz vectors
\renewcommand{\vector}[1]{\left\langle #1\right\rangle}


\outcome{Solve Related Rates word problems}

\title{2.17 Related Rates}




\begin{document}

\begin{abstract}
In this section we discover the relationship between the rates of change of two or more related quantities.
\end{abstract}

\maketitle


\section{Related Rates}

In this section, we use implicit differentiation to compute the relationship between 
the rates of change of related quantities.
If $F$ is a function of time, then 
\[ \frac{dF}{dt} \]
represents the rate of change of $F$ with respect to time, or simply, the rate of change of $F$.
For example, if $h$ is the height of a rising balloon, then $\frac{dh}{dt}$
is the rate of change of the height, i.e., it represents \textit{how fast}
the balloon is rising. If $z$ represents the distance between two moving 
objects then $\frac{dz}{dt}$ tells you how fast they are moving toward 
(or away from) each other.  If $V$ represents the volume of a melting snowball, 
then $\frac{dV}{dt}$ tells you how fast it is melting (and it should be negative, 
since the volume of a melting snowball is decreasing).

 %(time derivatives of related quantities)
 %(rates of change  of related quantities with respect to time)
%(rates of change with respect to time of related quantities).

\begin{center}
\bf{Examples of Related Rates}
\end{center}



\begin{example}[example 1]
Suppose $A$ and $s$ represent the area and side length of an expanding square. 
Then the symbols $\frac{dA}{dt}$ and $\frac{ds}{dt}$ represent the rates at which 
the area and the side length are increasing. Since the area of a square is related 
to side length (by the formula $A = s^2$), the rates of change of area and side 
length are also related. Since the square is expanding, both $A$ and $s$ are 
functions of time, $t$.
%To find this relationship we use implicit differentiation, noting that both $A$ and $s$ are functions of time since the square is expanding. 
Starting with the equation $A = s^2$, we differentiate both sides with respect 
to the implicit variable $t$ using the chain rule where appropriate:

%The area of a square is given by $A = s^2$. If $A$ and $s$ are functions of time, $t$, 
%then their rates of change are related as follows:
%\[A = s^2 \quad \text{we begin with the basic relationship between the variables}\]
\[\frac{d}{dt} (A) = \frac{d}{dt}(s^2) \]
Note the use of the symbol $\frac{d}{dt}$ to represent the derivative with 
respect to $t$ of each side of the equation $A = s^2$. Now we compute the derivatives:
%\quad \text{we differentiate both sides with respect to} \; t\]
\[\frac{dA}{dt}  = 2s\frac{ds}{dt}. \]
Note the use of the chain rule on the right hand side. In this application of the 
chain rule, $s$ is the inside function and the square is the outside function. 
So we get the derivative of the outside, $2s$, times the derivative of the inside, $\frac{ds}{dt}$.
In conclusion, the area and side length of a square are related by the formula
\[A = s^2\]
and their rates of change are related by the equation
\[\frac{dA}{dt} = 2s \frac{ds}{dt}.\]
\end{example}


\begin{problem}(problem 1a)
The volume of a cube is given by $V = s^3$. If $V$ and $s$ are functions of time, $t$, 
find the relationship between their rates of change.\\
\begin{hint}
Differentiate both sides with respect to $t$, using the symbol $\frac{d}{dt}$
\end{hint}
\begin{hint}
Use the chain rule on the right side with $s$ as the inside and the $cube$ as the outside
\end{hint}

The relationship between $\frac{dV}{dt}$ and $\frac{ds}{dt}$ is
\begin{multipleChoice}
\choice{$\frac{dV}{dt}= 2s \frac{ds}{dt}$}
\choice{$\frac{dV}{dt}= 3s \frac{ds}{dt}$}
\choice[correct]{$\frac{dV}{dt}= 3s^2 \frac{ds}{dt}$}
\choice{none of the above}
\end{multipleChoice}
\end{problem}


\begin{problem}(problem 1b)
The surface area of a cube is given by $S = 6s^2$. If $S$ and $s$ are functions of time, $t$, 
find the relationship between their rates of change.\\
\begin{hint}
Differentiate both sides with respect to $t$, using the symbol $\frac{d}{dt}$
\end{hint}
\begin{hint}
Use the chain rule on the right side with $s$ as the inside and the $square$ as the outside
\end{hint}

The relationship between $\frac{dS}{dt}$ and $\frac{ds}{dt}$ is
\begin{multipleChoice}
\choice{$\frac{dS}{dt}= 2s \frac{ds}{dt}$}
\choice{$\frac{dS}{dt}= 6s \frac{ds}{dt}$}
\choice[correct]{$\frac{dS}{dt}= 12s \frac{ds}{dt}$}
\choice{none of the above}
\end{multipleChoice}
\end{problem}


\begin{problem}(problem 2)
The circumference of a circle is given by $C = 2\pi r$. If $C$ and $r$ are functions of time, $t$,
find the relationship between their rates of change.\\
\begin{hint}
Differentiate both sides with respect to $t$, using the symbol $\frac{d}{dt}$
\end{hint}
\begin{hint}
Use the constant multiple rule on the right side
\end{hint}

The relationship between $\frac{dC}{dt}$ and $\frac{dr}{dt}$ is
\begin{multipleChoice}
\choice{$\frac{dC}{dt}=  \frac{dr}{dt}$}
\choice{$\frac{dC}{dt}= \pi \frac{dr}{dt}$}
\choice[correct]{$\frac{dC}{dt}= 2\pi \frac{dr}{dt}$}
\choice{none of the above}
\end{multipleChoice}
\end{problem}



\begin{example}[example 3]
The area of a circle is given by $A = \pi r^2$. If $A$ and $r$ are functions of time, $t$, 
then their rates of change are related as follows:
\[\frac{d}{dt} (A) = \frac{d}{dt}(\pi r^2)\]
\[\frac{dA}{dt}  = 2\pi r \frac{dr}{dt}.\]
\end{example}


\begin{problem}(problem 3a)
The volume of a sphere is given by $V = \tfrac{4}{3}\pi r^3$. If $V$ and $r$ are functions of time, $t$,
find the relationship between their rates of change.\\
\begin{hint}
Differentiate both sides with respect to $t$, using the symbol $\frac{d}{dt}$
\end{hint}
\begin{hint}
Use the chain rule on the right side with $r$ as the inside and the $cube$ as the outside
\end{hint}

The relationship between $\frac{dV}{dt}$ and $\frac{dr}{dt}$ is
\begin{multipleChoice}
\choice[correct]{$\frac{dV}{dt}= 4\pi r^2 \frac{dr}{dt}$}
\choice{$\frac{dV}{dt}= \frac43 \pi r \frac{dr}{dt}$}
\choice{$\frac{dV}{dt}= 4 \pi r \frac{dr}{dt}$}
\choice{none of the above}
\end{multipleChoice}
\end{problem}


\begin{problem}(problem 3b)
The surface area of a sphere is given by $S = 4\pi r^2$. If $S$ and $r$ are functions of time, $t$, 
find the relationship between their rates of change.\\
\begin{hint}
Differentiate both sides with respect to $t$, using the symbol $\frac{d}{dt}$
\end{hint}
\begin{hint}
Use the chain rule on the right side with $r$ as the inside and the $square$ as the outside
\end{hint}

The relationship between $\frac{dS}{dt}$ and $\frac{ds}{dt}$ is
\begin{multipleChoice}
\choice{$\frac{dS}{dt}= 2 \pi r \frac{dr}{dt}$}
\choice[correct]{$\frac{dS}{dt}= 8 \pi r\frac{dr}{dt}$}
\choice{$\frac{dS}{dt}= 4 \pi r^2 \frac{dr}{dt}$}
\choice{none of the above}
\end{multipleChoice}
\end{problem}



\begin{example}[example 4]
The lengths of the sides of a right triangle are related by $x^2 + y^2 = z^2$. 
If $x, y$ and $z$ are functions of time, $t$, 
differentiate with respect to $t$ to find the relationship between their rates of change:
\[\frac{d}{dt} (x^2 + y^2 ) = \frac{d}{dt}( z^2).\]
Use the chain rule on the squares:
%\[\frac{d}{dt} (x^2) +\frac{d}{dt}(y^2 ) = \frac{d}{dt}( z^2)\]
\[2x\frac{dx}{dt} + 2y\frac{dy}{dt} = 2z\frac{dz}{dt}.\]
Divide by 2:
\[x\frac{dx}{dt} + y\frac{dy}{dt} = z\frac{dz}{dt}.\]
\end{example}



\begin{example}[example 5]
In a right triangle with angle $\theta$, adjacent side $5$ and opposite side $x$, then 
$\theta$ and $x$ are related by $\tan (\theta) = \frac{x}{5}$. If $\theta$ and $x$ are functions of time, $t$, 
then their rates of change are related as follows:
\[\frac{d}{dt}[\tan (\theta)] = \frac{d}{dt} \left(\frac{x}{5}\right)\]
\[\sec^2(\theta) \frac{d\theta}{dt} = \frac{1}{5} \cdot \frac{dx}{dt}.\]
Note the use of the chain rule on the left side.
\end{example}



\begin{example}[example 6]
The area of a triangle is given by $A = \frac{1}{2} bh$. If $A, b$ and $h$ are functions of time, $t$, 
differentiate with respect to $t$ to find the relationship between their rates of change:
\[\frac{d}{dt}(A) = \frac{d}{dt}(\tfrac{1}{2} bh).\]
Use the product rule on the right side:
\[\frac{dA}{dt} = \frac{b}{2} \cdot \frac{dh}{dt}+ \frac{h}{2} \cdot \frac{db}{dt}.\]
\end{example}

\begin{problem}(problem 6a)
The volume of a cylinder is given by $V = \pi r^2h$. If $V, r$ and $h$ are functions of time, $t$,
find the relationship between their rates of change.\\
\begin{hint}
Differentiate both sides with respect to $t$, using the symbol $\frac{d}{dt}$
\end{hint}
\begin{hint}
Use the product rule on the right side and the chain rule on the square
\end{hint}

The relationship between $\frac{dV}{dt}, \frac{dr}{dt}$ and $\frac{dh}{dt}$ is
\begin{multipleChoice}
\choice{$\frac{dV}{dt} = \pi r^2\frac{dh}{dt} \cdot \frac{dr}{dt}$}
\choice{$\frac{dV}{dt} = \pi r^2 \frac{dh}{dt} + 2\pi r \frac{dr}{dt}$}
\choice[correct]{$\frac{dV}{dt} = \pi r^2\frac{dh}{dt} + 2\pi rh \frac{dr}{dt}$}
\choice{none of the above}
\end{multipleChoice}
\end{problem}




\begin{problem}(problem 6b)
The volume of a cone is given by $V = \frac13 \pi r^2h$. If $V, r$ and $h$ are functions of time, $t$,
find the relationship between their rates of change.\\
\begin{hint}
Differentiate both sides with respect to $t$, using the symbol $\frac{d}{dt}$
\end{hint}
\begin{hint}
Use the product rule on the right side and the chain rule on the square
\end{hint}

The relationship between $\frac{dV}{dt}, \frac{dr}{dt}$ and $\frac{dh}{dt}$ is
\begin{multipleChoice}
\choice[correct]{$\frac{dV}{dt} = \tfrac13 \pi r^2\frac{dh}{dt} + \tfrac23\pi rh \frac{dr}{dt}$}
\choice{$\frac{dV}{dt} = \tfrac23 \pi r^2\frac{dh}{dt} + \tfrac13\pi rh \frac{dr}{dt}$}
\choice{$\frac{dV}{dt} = 3 \pi r^2\frac{dh}{dt} + 2\pi rh \frac{dr}{dt}$}
\choice{none of the above}
\end{multipleChoice}
\end{problem}






\section{Related Rates Word Problems}




\begin{example}[example 7]
The radius of a circular ripple wave is increasing at a rate of 2 meters per second. How fast is the area of the wave 
increasing when the radius is 5 meters? \\
In mathematical notation, we are given $\frac{dr}{dt} = 2$ and we need to find
\[\frac{dA}{dt}\bigg|_{r = 5}.\]
We find the relationship between 
\[\frac{dA}{dt} \quad \text{and} \quad \frac{dr}{dt}\]
and then we plug in the given numerical information.
Since $A = \pi r^2$, we have
\[\frac{d}{dt} (A) = \frac{d}{dt}(\pi r^2)  \]
and using the chain rule on the right side we get:
\[\frac{dA}{dt}  = 2\pi r \frac{dr}{dt}.\]
Finally,
\[\frac{dA}{dt}\bigg|_{r = 5}  = 2\pi (5) (2) = 20\pi.\]
Thus, when the radius of the spill is 5 meters, the area is growing at a rate of $20\pi$ meters per second.
\end{example}

\begin{problem}(problem 7)
The radius of a circular aperture is decreasing at a rate of 3 centimeters per second.  
How fast is the area of the aperture changing when the radius is 15 centimeters?

\begin{hint}
$A = \pi r^2$ where both $A$ and $r$ are functions of time, $t$.
\end{hint}

\begin{hint}
Differentiate both sides with respect to $t$, using the symbol $\frac{d}{dt}$
\end{hint}
\begin{hint}
Use the chain rule on the square
\end{hint}

The area is changing at a rate of (in square centimeters per second):
\begin{multipleChoice}
\choice{$90 \pi$}
\choice[correct]{$-90 \pi$}
\choice{$-120 \pi$}
\choice{none of the above}
\end{multipleChoice}
\end{problem}


\begin{example}[example 8]
A ten foot ladder is sliding down a wall at a rate of 1 foot per second.  How fast is the base of the 
ladder sliding away from the wall when the top of the ladder is six feet above the ground?\\
 The ladder, the wall and the ground make a right triangle with the ladder as the hypotenuse.  
We label $x$ being the distance from the base of the ladder 
to the wall by $x$ and we label  the distance between the top of the ladder and the ground by $y$. 
The hypotenuse is the length of the
ladder, which is 10 feet. The variables $x$ and $y$ are related by the 
Pythagorean Theorem: $x^2 + y^2 = 10^2$. Since the ladder is
sliding down the wall, $x$ and $y$ are functions of time, $t$. 
In fact, $x$ is increasing and $y$ is decreasing. Furthermore, we are given 
the rate $\frac{dy}{dt} = -1$ (it is negative because $y$ is decreasing)
and we need to find the related rate
\[\frac{dx}{dt}\bigg|_{y = 6}.\]
We first find the relationship between 
\[\frac{dx}{dt} \quad \text{and} \quad \frac{dy}{dt}\]
and then we plug in the given numerical information.
Since $x^2 + y^2 = 100$, we have
\[\frac{d}{dt} (x^2 + y^2 ) = \frac{d}{dt}(100)\]
\[\frac{d}{dt} (x^2) +\frac{d}{dt}(y^2 ) = 0\]
\[2x\frac{dx}{dt} + 2y\frac{dy}{dt} = 0\]
\[x\frac{dx}{dt} + y\frac{dy}{dt} = 0\]
\[\frac{dx}{dt} =- \frac{y}{x}\cdot \frac{dy}{dt}. \]
Finally, when $y= 6$ we have $x=8$ since $x^2 + y^2 = 100$ and
\[\frac{dx}{dt}\bigg|_{y = 6}= -\frac68 (-1) = \frac34.\]
Thus, when the top of the ladder is six feet above the ground, 
the base of the ladder is sliding away from the wall 
at a rate of $\frac34$ foot per second.

\begin{center}
\geogebra{vV7WGnrc}{1000}{750}
\end{center}






\end{example}



\begin{problem}(problem 8a)
A 17 foot ladder is sliding down a wall at a rate of 3 feet per second.  How fast is the base of the 
ladder sliding away from the wall when the top of the ladder is 8 feet above the ground?\\

\begin{hint}
$x^2 + y^2 = 17^2$ where  $x$ and $y$ are functions of time, $t$.
\end{hint}

\begin{hint}
Differentiate both sides with respect to $t$, using the symbol $\frac{d}{dt}$
\end{hint}
\begin{hint}
Use the chain rule on the squares
\end{hint}

The base is sliding away from the wall at a rate of $\answer{8/5}$ ft/sec.
\end{problem}

\begin{problem}(problem 8b)
The base of 13 foot ladder is sliding away from a vertical wall at a rate of 2 
feet per second.  How fast is the top of the ladder sliding down the wall when the base of the 
ladder is 5 feet from the wall?
\begin{hint}
$x^2 + y^2 = 13^2$ where  $x$ and $y$ are functions of time, $t$.
\end{hint}

\begin{hint}
Differentiate both sides with respect to $t$, using the symbol $\frac{d}{dt}$
\end{hint}
\begin{hint}
Use the chain rule on the squares
\end{hint}

The top of the ladder is sliding down the wall at a rate of (in feet per second):
\begin{multipleChoice}
\choice[correct]{$5/6$}
\choice{$6/5$}
\choice{$5/12$}
\choice{none of the above}
\end{multipleChoice}
\end{problem}


\begin{problem}(problem 8c)
A kite, flying at a height of 120 feet is moving horizontally at a rate of 2 ft/sec.  How fast does the kite flier have to release the (taut) string when the kits is 200 feet away?//

The string must be released at a rate of $\answer{8/5}$ ft/sec.


\end{problem}

\begin{example}[example 9]
A street light is mounted at the top of a 15 ft tall pole.  
A man 6 ft tall walks away from the pole with a speed of 3 ft per sec along a straight path.  
How fast is the tip of his shadow moving when he is 35 ft from
the pole?\\
The diagram below illustrates the situation in the problem, with the variable $x$ representing the distance from the man to the pole
and $s$ representing the length of his shadow.
\begin{center}
\begin{tikzpicture}%[scale=1.25]%,cap=round,>=latex]

\coordinate [label=below:$light$] (A) at (2.5cm,0.0cm);
\coordinate [label=below:$man$] (B) at (5cm,0.0cm);
\coordinate  (C) at (6.5cm,0.0cm);
\coordinate  (D) at  (5cm, 0.75cm);
\coordinate  (E) at  (2.5cm, 2.0cm);
\draw (A) -- node[below] {$x$} (B) -- node[below] {$s$} (C) --  (E) -- node[left] {$15$} (A);
\draw (B) -- node[left] {$6$} (D);
\draw (2.5cm,0.25cm) rectangle (2.75cm,0.0cm);
\draw (5cm,0.1cm) rectangle (5.1cm,0.0cm);

\end{tikzpicture}
\end{center}

The small right triangle and the large right triangle in the diagram above are similar triangles.  
That means that the ratios of corresponding sides are equal.
Thus
\[\frac{s}{6}=\frac{s+x}{15}  \]
which can be simplified into
\[ 15s=6(s+x)\]
so
\[9s =6x\]
or
\[s = \frac23 x.\]
Here, $s$ and $x$ are functions of time, so differentiating both sides of this equation with respect to time, $t$,
we get
\[\frac{ds}{dt} = \frac23\frac{dx}{dt}.\]
We were given that 
\[\frac{dx}{dt} = 3 \; \text{ft/sec},\]
and so 
\[\frac{ds}{dt} = \frac23\frac{dx}{dt} = \frac23 \cdot 3 = 2 \;\text{ft/sec}.\]
Finally, to calculate how fast the tip of the shadow is moving, we must add the walking speed of the man, 
$dx/dt$, to the rate of lengthening of the shadow, $ds/dt$.
Thus, the tip of the shadow is moving at a rate of $3+2 =5$ ft/sec.
Note that this rate is independent of the distance of the man from the pole, so the answer is the same whether he
is 35 ft. from the pole of 350 ft. from the pole.


\end{example}


\begin{problem}(problem 9)
A street light is mounted at the top of a 20 ft tall pole.  
A woman 5 ft tall walks away from the pole with a speed of 2 ft per sec along a straight path.  
How fast is the tip of his shadow moving when she is 25 ft from
the pole?
\begin{hint}
We have similar triangles, so $\frac{s}{5}=\frac{s+x}{20}$
\end{hint}
\begin{hint}
Thus $s=\frac{x}{3}$
\end{hint}
\begin{hint}
Differentiate both sides with respect to $t$, using the symbol $\frac{d}{dt}$
\end{hint}
\begin{hint}
Add the walking speed of the woman to the rate of the lengthening of the shadow
\end{hint}

The tip of the shadow is moving at a rate of (in feet per second):
\begin{multipleChoice}
\choice{$4/3$}
\choice{$7/3$}
\choice[correct]{$8/3$}
\choice{none of the above}
\end{multipleChoice}
\end{problem}


\begin{example}[example 10]
A spherical snowball is melting at a rate of 20 cm$^3$/min. How fast is the radius of the snowball
decreasing when it is 10 cm. in diameter?\\
In mathematical notation, we are given $\frac{dV}{dt} = -20$ and we need to find
\[\frac{dr}{dt}\bigg|_{d = 10}.\]
The relationship between the volume,$V$, of a sphere and its radius, $r$ is given by the formula 
\[V = \frac43 \pi r^3.\]
To solve the problem, we need to find the relationship between 
\[\frac{dV}{dt} \quad \text{and} \quad \frac{dr}{dt}\]
and then we plug in the given numerical information.
Differentiate both sides with respect to $t$:


%Assuming the snowball is a sphere, its volume is given by $V = \tfrac43 \pi r^3$, where $r$ represents the radius of the sphere. 


\[\frac{d}{dt} (V) = \frac{d}{dt}(\tfrac43 \pi r^3)\]
Using the chain rule on the right hand side, we get the desired relationship between the rates:
\[\frac{dV}{dt}  = 4\pi r^2 \  \frac{dr}{dt}.\]

Lastly, we plug in the given numerical information. When the diameter,  $d = 10$, we have the radius, $r = 5$ and since $\frac{dV}{dt}=-20$ (given), we have
\[-20  = 4\pi (5)^2 \ \frac{dr}{dt}\bigg|_{d = 10}\]
and
\[\frac{dr}{dt}\bigg|_{d = 10} = -\frac{20}{4\pi (5)^2} = -\frac{1}{5\pi} \]
Thus, when the diameter of the snowball is  10 cm, the radius is decreasing at a rate of $\frac{1}{5\pi}$ centimeters per minute.
\end{example}



\begin{problem}(problem 10)
Air is being pumped into a spherical ballon at a rate of 64 in$^3$/min. How fast is the radius of the balloon
increasing when it is 6 inches in diameter?\\


The radius of the balloon is increasing at a rate of $\answer{\frac{16}{9\pi}}$ in/min.
\end{problem}



\begin{example}[example 11]
A camera is filming the launch of a hot air balloon from 100 feet away. 
If the balloon rises at a rate of 15 feet per second, how fast is the camera angle increasing when the balloon is 50 feet in the air?\\
Let $h$ be the height of the balloon and $\theta$ be the camera angle. Then, by right triangle trigonometry, 
\[\tan \theta  = \frac{h}{100}\]
where both $\theta$ and $h$ are functions of time, $t$.
We are given $\frac{dh}{dt} = 15$ and we have to find
\[\frac{d\theta}{dt}\bigg|_{h = 50}.\]
We find the relationship between 
\[\frac{d\theta}{dt} \text{ and } \frac{dh}{dt}\]
and then we plug in the given numerical information.
Since $\tan\theta = \frac{h}{100}$, we have
\[\frac{d}{dt}\big(\tan\theta\big) = \frac{d}{dt} \Big(\frac{h}{100}\Big)\]
 \[\sec^2 \theta \cdot \frac{d\theta}{dt} = \frac{1}{100} \cdot \frac{dh}{dt}\]
\[ \frac{d\theta}{dt} = \frac{1}{100}\cdot\cos^2 \theta  \cdot\frac{dh}{dt}.\]
Next, when the balloon is 50 feet in the air, the hypotenuse of the right triangle is 
\[\sqrt{100^2 + 50^2} = 50\sqrt{2^2 + 1^2} = 50\sqrt 5\]
which means 
\[\cos^2 \theta = \left(\frac{100}{50\sqrt 5}\right)^2 = \left(\frac{2}{\sqrt 5}\right)^2 = \frac45.\]
Lastly,
\[\frac{d\theta}{dt}\bigg|_{h = 50}= \frac{1}{100} \cdot \frac45 \cdot 15 = \frac{3}{25}.\]
Thus, when the balloon is 50 feet in the air, the camera angle is increasing at a rate of $\frac{3}{25}$ radians per second.
\end{example}


\begin{problem}(problem 11)
A camera is filming the landing of a rocket(!) from 500 feet away. 
If the rocket descends at a rate of 50 feet per second, how fast is the camera angle decreasing when the 
rocket is 250 feet in the air?\\

\begin{hint} 
A negative rate means the quantity is decreasing
\end{hint}

The camera angle is decreasing at a rate of $\answer{2/25}$ rad/sec.

\end{problem}



\begin{example}[example 12]
Oil is being spilled into the ocean at a rate of 75 cubic meters per hour.  
Assuming that the layer of oil on the oceans surface makes a cylinder  2 cm in height, 
how fast is the radius of the spill increasing when the radius is 300 m?\\
Let $V, r$ and $h$ be the volume, radius and height of the spill.  Then $V = \pi r^2 h = 0.02 \ \pi r^2$,
where $V$ and $r$ are functions of time, $t$. We are given
\[\frac{dV}{dt} = 75 \quad \text{and we need} \quad \frac{dr}{dt}\bigg|_{r = 300}.\]

We find the relationship between 
\[\frac{dr}{dt} \quad \text{and}\quad \frac{dV}{dt}\]
and then we plug in the given numerical information.
Since $V =  0.02\  \pi r^2$,
\[\frac{d}{dt}(V) = \frac{d}{dt}(0.02 \ \pi r^2)\]
\[\frac{dV}{dt} = 0.04 \pi r \cdot \frac{dr}{dt}.\]
Finally, when $r = 300$ we have
\[75 = 0.04 \ \pi \cdot 300 \cdot \frac{dr}{dt}\bigg|_{r= 300} = 12 \pi \cdot \frac{dr}{dt}\bigg|_{r= 300} \]
and
\[\frac{dr}{dt}\bigg|_{r= 300} = \frac{75}{12\pi} = 
\frac{25}{4\pi}.\]





Thus, when the radius of the spill is 300 meters, the radius is increasing at a rate of $\frac{25}{4\pi}$ meters per hour.
\end{example}



\begin{example}[example 13]
Sand is being dumped into a conical pile. The consistency of the sand is such that the 
diameter of the cone is always equal to its height. After a few minutes,  the pile is 6 feet high and
increasing at a rate of 1 ft/min. How fast is sand being added to the pile at that time?\\

Let $h, r$ and $V$ be the height, radius and volume of the cone of sand. Then we are given
\[\frac{dh}{dt}\bigg|_{h = 6} = 1\]
and due to the consistency of the sand, $h = 2r$, so $r = h/2$.
We need to find 
\[\frac{dV}{dt}\bigg|_{h = 6}.\]
We find the relationship between 
\[\frac{dV}{dt} \text{ and } \frac{dh}{dt}\]
and then we plug in the given numerical information.
Since 
\[V = \tfrac13 \pi r^2 h = \tfrac13 \pi \big(\tfrac{h}{2}\big)^2 h = \tfrac{1}{12}\pi h^3\]
where $V$ and $h$ are functions of $t$, we have
\[\frac{d}{dt}(V) = \frac{d}{dt}(\tfrac{1}{12}\pi h^3)\]
\[\frac{dV}{dt} = \tfrac{3}{12}\pi h^2 \  \frac{dh}{dt} = \tfrac{1}{4}\pi h^2 \ \frac{dh}{dt}.\]
Finally,

\[\frac{dV}{dt}\bigg|_{h = 6} = \tfrac{1}{4}\pi \cdot (6)^2 \cdot(1) = 9\pi.\]
Thus, when the pile is 6 feet high, sand is being added at a rate of $9\pi$ cubic feet per minute.
\end{example}


\begin{center}
\begin{foldable}
\unfoldable{Here is a detailed, lecture style video on related rates:}
\youtube{fnqv825oQPY}
\end{foldable}
\end{center}





\end{document}
