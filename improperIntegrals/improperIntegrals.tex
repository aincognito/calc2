\documentclass{ximera}

%% You can put user macros here
%% However, you cannot make new environments



\newcommand{\ffrac}[2]{\frac{\text{\footnotesize $#1$}}{\text{\footnotesize $#2$}}}
\newcommand{\vasymptote}[2][]{
    \draw [densely dashed,#1] ({rel axis cs:0,0} -| {axis cs:#2,0}) -- ({rel axis cs:0,1} -| {axis cs:#2,0});
}


\graphicspath{{./}{firstExample/}}

\usepackage{amsmath}
\usepackage{amssymb}
\usepackage{array}
\usepackage[makeroom]{cancel} %% for strike outs
\usepackage{pgffor} %% required for integral for loops
\usepackage{tikz}
\usepackage{tikz-cd}
\usepackage{tkz-euclide}
\usetikzlibrary{shapes.multipart}


\usetkzobj{all}
\tikzstyle geometryDiagrams=[ultra thick,color=blue!50!black]


\usetikzlibrary{arrows}
\tikzset{>=stealth,commutative diagrams/.cd,
  arrow style=tikz,diagrams={>=stealth}} %% cool arrow head
\tikzset{shorten <>/.style={ shorten >=#1, shorten <=#1 } } %% allows shorter vectors

\usetikzlibrary{backgrounds} %% for boxes around graphs
\usetikzlibrary{shapes,positioning}  %% Clouds and stars
\usetikzlibrary{matrix} %% for matrix
\usepgfplotslibrary{polar} %% for polar plots
\usepgfplotslibrary{fillbetween} %% to shade area between curves in TikZ



%\usepackage[width=4.375in, height=7.0in, top=1.0in, papersize={5.5in,8.5in}]{geometry}
%\usepackage[pdftex]{graphicx}
%\usepackage{tipa}
%\usepackage{txfonts}
%\usepackage{textcomp}
%\usepackage{amsthm}
%\usepackage{xy}
%\usepackage{fancyhdr}
%\usepackage{xcolor}
%\usepackage{mathtools} %% for pretty underbrace % Breaks Ximera
%\usepackage{multicol}



\newcommand{\RR}{\mathbb R}
\newcommand{\R}{\mathbb R}
\newcommand{\C}{\mathbb C}
\newcommand{\N}{\mathbb N}
\newcommand{\Z}{\mathbb Z}
\newcommand{\dis}{\displaystyle}
%\renewcommand{\d}{\,d\!}
\renewcommand{\d}{\mathop{}\!d}
\newcommand{\dd}[2][]{\frac{\d #1}{\d #2}}
\newcommand{\pp}[2][]{\frac{\partial #1}{\partial #2}}
\renewcommand{\l}{\ell}
\newcommand{\ddx}{\frac{d}{\d x}}

\newcommand{\zeroOverZero}{\ensuremath{\boldsymbol{\tfrac{0}{0}}}}
\newcommand{\inftyOverInfty}{\ensuremath{\boldsymbol{\tfrac{\infty}{\infty}}}}
\newcommand{\zeroOverInfty}{\ensuremath{\boldsymbol{\tfrac{0}{\infty}}}}
\newcommand{\zeroTimesInfty}{\ensuremath{\small\boldsymbol{0\cdot \infty}}}
\newcommand{\inftyMinusInfty}{\ensuremath{\small\boldsymbol{\infty - \infty}}}
\newcommand{\oneToInfty}{\ensuremath{\boldsymbol{1^\infty}}}
\newcommand{\zeroToZero}{\ensuremath{\boldsymbol{0^0}}}
\newcommand{\inftyToZero}{\ensuremath{\boldsymbol{\infty^0}}}


\newcommand{\numOverZero}{\ensuremath{\boldsymbol{\tfrac{\#}{0}}}}
\newcommand{\dfn}{\textbf}
%\newcommand{\unit}{\,\mathrm}
\newcommand{\unit}{\mathop{}\!\mathrm}
%\newcommand{\eval}[1]{\bigg[ #1 \bigg]}
\newcommand{\eval}[1]{ #1 \bigg|}
\newcommand{\seq}[1]{\left( #1 \right)}
\renewcommand{\epsilon}{\varepsilon}
\renewcommand{\iff}{\Leftrightarrow}

\DeclareMathOperator{\arccot}{arccot}
\DeclareMathOperator{\arcsec}{arcsec}
\DeclareMathOperator{\arccsc}{arccsc}
\DeclareMathOperator{\si}{Si}
\DeclareMathOperator{\proj}{proj}
\DeclareMathOperator{\scal}{scal}
\DeclareMathOperator{\cis}{cis}
\DeclareMathOperator{\Arg}{Arg}
%\DeclareMathOperator{\arg}{arg}
\DeclareMathOperator{\Rep}{Re}
\DeclareMathOperator{\Imp}{Im}
\DeclareMathOperator{\sech}{sech}
\DeclareMathOperator{\csch}{csch}
\DeclareMathOperator{\Log}{Log}

\newcommand{\tightoverset}[2]{% for arrow vec
  \mathop{#2}\limits^{\vbox to -.5ex{\kern-0.75ex\hbox{$#1$}\vss}}}
\newcommand{\arrowvec}{\overrightarrow}
\renewcommand{\vec}{\mathbf}
\newcommand{\veci}{{\boldsymbol{\hat{\imath}}}}
\newcommand{\vecj}{{\boldsymbol{\hat{\jmath}}}}
\newcommand{\veck}{{\boldsymbol{\hat{k}}}}
\newcommand{\vecl}{\boldsymbol{\l}}
\newcommand{\utan}{\vec{\hat{t}}}
\newcommand{\unormal}{\vec{\hat{n}}}
\newcommand{\ubinormal}{\vec{\hat{b}}}

\newcommand{\dotp}{\bullet}
\newcommand{\cross}{\boldsymbol\times}
\newcommand{\grad}{\boldsymbol\nabla}
\newcommand{\divergence}{\grad\dotp}
\newcommand{\curl}{\grad\cross}
%% Simple horiz vectors
\renewcommand{\vector}[1]{\left\langle #1\right\rangle}


\outcome{Compute definite integrals involving infinity}

\title{Improper Integrals}

\begin{document}

\begin{abstract}
We will compute definite integrals involving infinity.
\end{abstract}

\maketitle

\section{Introduction}
In this section we will consider definite integrals involving infinity.
There are two fundamentally different types of improper integrals.  The first involves infinity as an ``endpoint"
of integration and the second involves vertical asymptotes at or between the endpoints of integration.
Examples of the first type include
\[
\int_0^\infty e^{-2x} \; dx, \; \text{ and } \;  \int_{-\infty}^\infty \frac{1}{1+x^2}\; dx.
\]
The second of the above examples is called ``doubly improper" since it involves both $\infty$ and $-\infty$.
Examples of the second type include
\[
\int_0^{\pi/2} \tan(x)\; dx \; \text{ and } \; \int_0^1 \frac{1}{x} \; dx.
\]
If the value of an improper integral is a finite number, then we say that the integral \textbf{converges}.
Otherwise, we say that the integral \textbf{diverges}.
The method of computing improper integrals involves replacing an endpoint with a variable 
and then taking an appropriate limit as we will see in the examples below.




\section{Infinity as an endpoint}
\begin{definition}[Type 1 Improper Integral]
Suppose $f(x)$ is continuous on the interval $[a, \infty)$. Then we compute the improper integral at infinity as follows:
\[
\int_a^\infty f(x) \; dx = \lim_{b \to \infty} \int_a^b f(x) \; dx.
\]
The improper integral at negative infinity is defined similarly.
\end{definition}


\begin{example}
Compute the improper integral
\[
\int_0^\infty e^{-2x} \; dx.
\]

We will begin by replacing $\infty$ in the improper integral with the variable $b$, a typical choice for an upper endpoint of integration.
We compute
\begin{align*}
\int_0^b e^{-2x} \; dx &= \left(-\frac12 e^{-2x} \right) \bigg|_0^b\\
                       &= \left(-\frac12 e^{-2b} \right) - \left(-\frac12 e^{0} \right)\\
                       &=\frac12 -\frac12 e^{-2b}.
\end{align*}
                       
To complete the problem, we now take a limit as $b \to \infty$.
\begin{align*}
\int_0^\infty e^{-2x} \; dx &= \lim_{b \to \infty}\int_0^b e^{-2x} \; dx\\
                       &=\lim_{b \to \infty} \left(\frac12 -\frac12 e^{-2b}\right)\\
                       &= \frac12 -\frac12 e^{-\infty}\\
                       &= \frac12 - 0 \\
                       &= \frac12.
\end{align*}
Since the answer is a finite number we say that the integral converges.
More precisely, we can say that the integral converges to $1/2$.

\end{example}




\begin{problem}
Compute the improper integral
\[
\int_0^\infty 2e^{-5x} \; dx = \answer{2/5}
\]

\end{problem}






\begin{example}
Compute the improper integral
\[
\int_1^\infty \frac{3}{x^2} \; dx.
\]

We will begin by replacing $\infty$ in the improper integral with the variable $b$, a typical choice for an upper endpoint of integration.
We compute
\begin{align*}
\int_1^b \frac{3}{x^2} \; dx &= \int_1^b 3x^{-2} \; dx\\
                           &= \left(3\cdot \frac{x^{-1}}{-1} \right) \bigg|_1^b\\
                           &= \left(-\frac{3}{x} \right) \bigg|_1^b\\
                       &= \left(-\frac{3}{b} \right) - \left(-\frac{3}{1} \right)\\
                       &=3 - \frac{3}{b}.
\end{align*}
                       
To complete the problem, we now take a limit as $b \to \infty$.
\begin{align*}
\int_1^\infty \frac{3}{x^2} \; dx &= \lim_{b \to \infty}\int_1^b \frac{3}{x^2} \; dx\\
                       &=\lim_{b \to \infty} \left(3 - \frac{3}{b}\right)\\
                       &= 3 -\frac{3}{\infty}\\
                       &= 3 - 0 \\
                       &= 3.
\end{align*}
Since the answer is a finite number we say that the integral converges.
More precisely, we can say that the integral converges to $3$.

\end{example}



\begin{problem}
Compute the improper integral
\[
\int_2^\infty \frac{5}{x^3} \; dx = \answer{5/8}
\]

\end{problem}




\begin{example}
Compute the improper integral
\[
\int_1^\infty \frac{4}{x} \; dx.
\]

We will begin by replacing $\infty$ in the improper integral with the variable $b$, a typical choice for an upper endpoint of integration.
We compute
\begin{align*}
\int_1^b \frac{4}{x} \; dx &= \int_1^b \frac{4}{x} \; dx\\
                           &= \left(4\ln|x| \right) \bigg|_1^b\\
                       &= \left(4\ln|b| \right) - \left(4\ln|1| \right)\\
                       &=4\ln|b|.
\end{align*}
                       
To complete the problem, we now take a limit as $b \to \infty$.
\begin{align*}
\int_1^\infty \frac{4}{x} \; dx &= \lim_{b \to \infty}\int_1^b \frac{4}{x} \; dx\\
                       &=\lim_{b \to \infty} 4\ln|b|\\
                       &= 4\ln(\infty)\\
                       &= \infty.
\end{align*}
Since the answer is a \textbf{not} a finite number we say that the integral diverges.

\end{example}


\begin{problem}
Show that the improper integral diverges:
\[
\int_3^\infty \frac{7}{\sqrt x}.
\]
\end{problem}

Based on the previous examples and problems, we can expect the following theorem about integrals of the form
\[
\int_1^\infty \frac{1}{x^p} \; dx.
\]

\begin{theorem}[p-integrals]
The improper p-integral $\int_1^\infty \frac{1}{x^p} \; dx$
 converges if $p>1$ and diverges if $p \leq 1$
The result is the same if lower endpoint is replaced by any other positive number.
\end{theorem}
This theorem will be of value to us in future sections, but for now, we use it to help us with another improper integral.


\begin{example}
Compute the improper integral
\[
\int_2^\infty \frac{1}{x\ln(x)} \; dx.
\]
Using a u-substitution with $u = \ln(x)$ and $du = \frac{1}{x} \, dx$, the integral becomes
\[
\int_2^\infty \frac{1}{x\ln(x)} \; dx = \int_{\ln(2)}^\infty \frac{1}{u} \; du.
\]
This is a $p$-integral in the variable $u$ with $p=1$. Since $p \leq 1$, the theorem says that this integral diverges, and hence
\[
\int_2^\infty \frac{1}{x\ln(x)} \; dx \;\; \text{ diverges}.
\]
\end{example}



\section{Type 2 Improper Integrals}

In this section we examine definite integrals in which the integrand has a vertical asymptote at an endpoint of integration.

\begin{definition}[Type 2 Improper Integral]
Suppose $f(x)$ is continuous on the interval $(a, b]$ and that $f(x)$ has a vertical asymptote at $x = a$.
Then
\[
\int_a^b f(x) \; dx = \lim_{t \to a^+} \int_t^b f(x) \; dx.
\]
The improper integral is defined similarly if the vertical asymptote is at the upper endpoint of integration.
\end{definition}

\begin{example}
Compute the improper integral
\[
\int_0^1 \frac{3}{\sqrt x} \; dx.
\]
The integral is improper because the integrand has a vertical asymptote at $x = 0$. Hence
\begin{align*}
\int_0^1 \frac{3}{\sqrt x} \; dx &= \lim_{a \to 0^+} \int_a^1 \frac{3}{\sqrt x} \; dx\\
                                 &= \lim_{a \to 0^+} \int_a^1 3x^{-1/2} \; dx\\
                                 &= \lim_{a \to 0^+} \left( 6x^{1/2}  \right) \bigg|_a^1\\
                                 &= \lim_{a \to 0^+} \left( 6\sqrt{1} - 6\sqrt{a}  \right) \\
                                 &= \left( 6 - 6\sqrt{0}  \right) \\
                                 &= 6.
\end{align*}
 Hence the improper integral converges to 6.
 \end{example}
 
\begin{problem}
Compute the improper integral
\[
\int_2^6 \frac{4}{\sqrt{x-2}} \; dx = \answer{16}
\]
\end{problem}

                                 
\begin{center}
\begin{foldable}
\unfoldable{Here is a detailed, lecture style video on improper integrals:}
\youtube{gRJTe1i1muo}
\end{foldable}
\end{center}





\end{document}


\begin{example} %example #15
Find $h'(x)$ if $h(x) = x^{\sin(x)}$.\\
We will use the fact that the exponential and logarithm functions are inverses,
\[e^{\ln(x)} = x,\]
and the exponent property of logarithms, 
\[\ln(x^n) = n\ln(x),\]
to rewrite $h(x)$.  We have 
\[h(x) = x^{\sin(x)} = e^{\ln(x^{\sin(x)})} = e^{\sin(x)\ln(x)}\]
and we can now compute $h'(x)$ using a combination of the chain rule and product rule.
We can write $h(x)$ as a composition, $f(g(x))$ with 
\[g(x) = \sin(x)\ln(x) \quad \text{and} \quad f(x) = e^x.\]
Then to find $g'(x)$ we us the product rule and we get $g'(x) = \frac{\sin(x)}{x} + \cos(x)\ln(x)$.
Next $f'(x) = e^x$ and 
hence $f'(g(x)) = e^{g(x)} = e^{\sin(x)\ln(x)} = x^{\sin(x)}$.
We can then conclude $h'(x) = f'(g(x))g'(x) = x^{\sin(x)} \left[ \frac{\sin(x)}{x} + \cos(x)\ln(x)\right]$.
\end{example}

%more question formats below













%\begin{verbatim}
\begin{question}
What is your favorite color?
\begin{multipleChoice}
\choice[correct]{Rainbow}
\choice{Blue}
\choice{Green}
\choice{Red}
\end{multipleChoice}
\begin{freeResponse}
Hello
\end{freeResponse}
\end{question}
%\end{verbatim}





\begin{question}
  Which one will you choose?
  \begin{multipleChoice}
    \choice[correct]{I'm correct.}
    \choice{I'm wrong.}
    \choice{I'm wrong too.}
  \end{multipleChoice}
\end{question}


\begin{question}
  Which one will you choose?
  \begin{selectAll}
    \choice[correct]{I'm correct.}
    \choice{I'm wrong.}
    \choice[correct]{I'm also correct.}
    \choice{I'm wrong too.}
  \end{selectAll}
\end{question}


\begin{freeResponse}
What is the chain rule used for?
\end{freeResponse}
