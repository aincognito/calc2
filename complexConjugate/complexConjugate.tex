\documentclass[handout]{ximera}

%% You can put user macros here
%% However, you cannot make new environments



\newcommand{\ffrac}[2]{\frac{\text{\footnotesize $#1$}}{\text{\footnotesize $#2$}}}
\newcommand{\vasymptote}[2][]{
    \draw [densely dashed,#1] ({rel axis cs:0,0} -| {axis cs:#2,0}) -- ({rel axis cs:0,1} -| {axis cs:#2,0});
}


\graphicspath{{./}{firstExample/}}

\usepackage{amsmath}
\usepackage{amssymb}
\usepackage{array}
\usepackage[makeroom]{cancel} %% for strike outs
\usepackage{pgffor} %% required for integral for loops
\usepackage{tikz}
\usepackage{tikz-cd}
\usepackage{tkz-euclide}
\usetikzlibrary{shapes.multipart}


\usetkzobj{all}
\tikzstyle geometryDiagrams=[ultra thick,color=blue!50!black]


\usetikzlibrary{arrows}
\tikzset{>=stealth,commutative diagrams/.cd,
  arrow style=tikz,diagrams={>=stealth}} %% cool arrow head
\tikzset{shorten <>/.style={ shorten >=#1, shorten <=#1 } } %% allows shorter vectors

\usetikzlibrary{backgrounds} %% for boxes around graphs
\usetikzlibrary{shapes,positioning}  %% Clouds and stars
\usetikzlibrary{matrix} %% for matrix
\usepgfplotslibrary{polar} %% for polar plots
\usepgfplotslibrary{fillbetween} %% to shade area between curves in TikZ



%\usepackage[width=4.375in, height=7.0in, top=1.0in, papersize={5.5in,8.5in}]{geometry}
%\usepackage[pdftex]{graphicx}
%\usepackage{tipa}
%\usepackage{txfonts}
%\usepackage{textcomp}
%\usepackage{amsthm}
%\usepackage{xy}
%\usepackage{fancyhdr}
%\usepackage{xcolor}
%\usepackage{mathtools} %% for pretty underbrace % Breaks Ximera
%\usepackage{multicol}



\newcommand{\RR}{\mathbb R}
\newcommand{\R}{\mathbb R}
\newcommand{\C}{\mathbb C}
\newcommand{\N}{\mathbb N}
\newcommand{\Z}{\mathbb Z}
\newcommand{\dis}{\displaystyle}
%\renewcommand{\d}{\,d\!}
\renewcommand{\d}{\mathop{}\!d}
\newcommand{\dd}[2][]{\frac{\d #1}{\d #2}}
\newcommand{\pp}[2][]{\frac{\partial #1}{\partial #2}}
\renewcommand{\l}{\ell}
\newcommand{\ddx}{\frac{d}{\d x}}

\newcommand{\zeroOverZero}{\ensuremath{\boldsymbol{\tfrac{0}{0}}}}
\newcommand{\inftyOverInfty}{\ensuremath{\boldsymbol{\tfrac{\infty}{\infty}}}}
\newcommand{\zeroOverInfty}{\ensuremath{\boldsymbol{\tfrac{0}{\infty}}}}
\newcommand{\zeroTimesInfty}{\ensuremath{\small\boldsymbol{0\cdot \infty}}}
\newcommand{\inftyMinusInfty}{\ensuremath{\small\boldsymbol{\infty - \infty}}}
\newcommand{\oneToInfty}{\ensuremath{\boldsymbol{1^\infty}}}
\newcommand{\zeroToZero}{\ensuremath{\boldsymbol{0^0}}}
\newcommand{\inftyToZero}{\ensuremath{\boldsymbol{\infty^0}}}


\newcommand{\numOverZero}{\ensuremath{\boldsymbol{\tfrac{\#}{0}}}}
\newcommand{\dfn}{\textbf}
%\newcommand{\unit}{\,\mathrm}
\newcommand{\unit}{\mathop{}\!\mathrm}
%\newcommand{\eval}[1]{\bigg[ #1 \bigg]}
\newcommand{\eval}[1]{ #1 \bigg|}
\newcommand{\seq}[1]{\left( #1 \right)}
\renewcommand{\epsilon}{\varepsilon}
\renewcommand{\iff}{\Leftrightarrow}

\DeclareMathOperator{\arccot}{arccot}
\DeclareMathOperator{\arcsec}{arcsec}
\DeclareMathOperator{\arccsc}{arccsc}
\DeclareMathOperator{\si}{Si}
\DeclareMathOperator{\proj}{proj}
\DeclareMathOperator{\scal}{scal}
\DeclareMathOperator{\cis}{cis}
\DeclareMathOperator{\Arg}{Arg}
%\DeclareMathOperator{\arg}{arg}
\DeclareMathOperator{\Rep}{Re}
\DeclareMathOperator{\Imp}{Im}
\DeclareMathOperator{\sech}{sech}
\DeclareMathOperator{\csch}{csch}
\DeclareMathOperator{\Log}{Log}

\newcommand{\tightoverset}[2]{% for arrow vec
  \mathop{#2}\limits^{\vbox to -.5ex{\kern-0.75ex\hbox{$#1$}\vss}}}
\newcommand{\arrowvec}{\overrightarrow}
\renewcommand{\vec}{\mathbf}
\newcommand{\veci}{{\boldsymbol{\hat{\imath}}}}
\newcommand{\vecj}{{\boldsymbol{\hat{\jmath}}}}
\newcommand{\veck}{{\boldsymbol{\hat{k}}}}
\newcommand{\vecl}{\boldsymbol{\l}}
\newcommand{\utan}{\vec{\hat{t}}}
\newcommand{\unormal}{\vec{\hat{n}}}
\newcommand{\ubinormal}{\vec{\hat{b}}}

\newcommand{\dotp}{\bullet}
\newcommand{\cross}{\boldsymbol\times}
\newcommand{\grad}{\boldsymbol\nabla}
\newcommand{\divergence}{\grad\dotp}
\newcommand{\curl}{\grad\cross}
%% Simple horiz vectors
\renewcommand{\vector}[1]{\left\langle #1\right\rangle}


\pgfplotsset{compat=1.13}

\outcome{Learn properties of the complex conjugate}

\title{1.3 Complex Conjugate}

\begin{document}

\begin{abstract}
We learn properties of the complex conjugate.
\end{abstract}

\maketitle



We have seen that the complex conjugate is defined by
\[
\overline{a+bi} = a-bi
\]

The conjugate of the conjugate is the original complex number:

\[
\overline{\overline{a+bi}} = \overline{a-bi} = a+bi
\]

The conjugate of a real number is itself:
\[
\overline{a}=\overline{a+0i} = a-0i = a
\]

The conjugate of an imaginary number is its negative:
\[
\overline{bi}=\overline{0+bi} = 0-bi = -bi
\]


\section{Real and Imaginary Part}


If we add a complex number and it's conjugate, we get
\[
(a+bi) + (\overline{a+bi}) = 2a = 2\Rep(a+bi)
\]
Thus, we have a formula for the real part of a complex number in terms of its conjugate:
\[
\Rep(a+bi) = \frac12 \left[(a+bi) + (\overline{a+bi})\right]
\]
Similarly, subtracting the conjugate gives 
\[
a+bi - \overline{a+bi} = 2bi,
\]
and so
\[
\Imp(a+bi) = -\frac{i}{2}\left[(a+bi) + (\overline{a+bi})\right]
\]
 
\section{Modulus}

If we multiply a complex number by its conjugate, we get the square of the modulus:
\[
(a+bi)(\overline{a+bi}) = (a+bi)(a-bi) = a^2 + b^2 = |a+bi|^2
\]
Thus, we have a formula for the modulus of a complex number in terms of its conjugate:
\[
|a+bi|= \sqrt{(a+bi)(\overline{a+bi})}
\]


\section{Multiplicative Inverse}

For a non-zero complex number, $a+bi$, its multiplicative inverse is its conjugate divided by the square of its modulus:
\[
\frac{1}{a+bi} = \frac{1}{a+bi}\cdot \frac{\overline{a+bi}}{\overline{a+bi}} = \frac{\overline{a+bi}}{\ |a+bi|^2}
\]



\section{Addition and Multiplication}

The conjugate of a sum is the sum of the conjugates:
\[
\overline{(a+bi) + (c+di)} = (a+c) - (b+d)i = \overline{a+bi }+\overline{c+di}
\]


The conjugate of a product is the product of the conjugates:
\[
\overline{(a+bi) \cdot (c+di)} = \overline{a+bi }\cdot\overline{c+di}
\]
To see this, we can calculate the left and right sides separately and see that they are the same:
\[
\overline{(a+bi) \cdot (c+di)}= \overline{(ac-bd) + (ad+bc)i} = (ac-bd) - (ad+bc)i
\]
and
\[
\overline{a+bi }\cdot\overline{c+di} = (a-bi)(c-di) = (ac-bd) + (-ad-bc)i = (ac-bd) - (ad +bc)i
\]


\begin{problem}(problem 1) Prove that the conjugate of a difference is the difference of the conjugates.
\end{problem}


\begin{problem}(problem 2) Prove that the conjugate of a quotient is the quotient of the conjugates.
\end{problem}

\begin{problem}(problem 3) Prove that the conjugate of a finite sum is the sum of the conjugates, i.e., show that
\[
\overline{\sum_{i=1}^n z_i} =  \sum_{i=1}^n \overline{z_i}
\]

\end{problem}
\begin{problem}(problem 4) Prove that the roots of a polynomial with real coefficients come in conjugate pairs, i.e., if $\alpha$ is a root of $p(z)$,then 
$\overline{\alpha}$ is also a root.

\end{problem}
\end{document}




\begin{definition}
A {\bf complex number} is a number of the form $a+bi$ where $a,b \in \mathbb{R}$. 
The set of all complex numbers is $\mathbb{C}= \{a+bi \,|\, a,b \in \mathbb{R}\}$.
\end{definition}

examples of complex numbers
equality of complex numbers
real numbers are complex R is a subset of C
real part, imaginary part

\section{Addition of Complex Numbers}

Complex numbers are added component-wise:
\[
(a+bi) + (c+di) = (a+b) + (c+d)i
\]

problems
Complex addition is commuative and associative.


\section{Multiplication of Complex Numbers}
Complex numbers are multiplied in the same way as binomials, using the fact that $i^2 = -1$:
\[
(a+bi)(c+di) = ac + adi + bci + (bi)(di) = (ac-bd) + (ad+bc)i
\]

problems


\section{The Complex Conjugate}

\begin{definition}[Complex Conjugate]
The {\bf complex conjugate} of a complex number $a+bi$ is defined as
\[
\overline {a+bi} = a-bi
\]
\end{definition}
problems

Note that if we multiply a complex number by its conjugate, we get a real number:
\[
(a+bi)  \cdot \overline{(a+bi)} = (a+bi)(a-bi) = a^2 +b^2
\]

problems


\section{Division of Complex Numbers}
To divide complex numbers, we use the complex conjugate:
\[
\frac{a+bi}{c+di} = \frac{a+bi}{c+di}\cdot\frac{c-di}{c-di} = \frac{(a+bi)(c-di)}{c^2 + d^2} = \left(\frac{ac+bd}{c^2 + d^2}\right) + \left(\frac{bc-ad}{c^2 + d^2}\right)i
\]


problems

\end{document}














\begin{center}
\begin{foldable}
\unfoldable{Here is a video of Example 1}
%\youtube{###} %vid of example 1
\end{foldable}
\end{center}

\begin{problem} %problem #1
  Find the area between...
  \[
  f(x) =  ...
  \]
    \begin{hint}
      Set up a definite integral
    \end{hint}
    \begin{hint}
      Determine the top and bottom curves
    \end{hint}
    
		
		The area between the curves is
		 $\answer[given]{number}$
\end{problem}







\begin{example} %example #15
Find $h'(x)$ if $h(x) = x^{\sin(x)}$.\\
We will use the fact that the exponential and logarithm functions are inverses,
\[e^{\ln(x)} = x,\]
and the exponent property of logarithms, 
\[\ln(x^n) = n\ln(x),\]
to rewrite $h(x)$.  We have 
\[h(x) = x^{\sin(x)} = e^{\ln(x^{\sin(x)})} = e^{\sin(x)\ln(x)}\]
and we can now compute $h'(x)$ using a combination of the chain rule and product rule.
We can write $h(x)$ as a composition, $f(g(x))$ with 
\[g(x) = \sin(x)\ln(x) \quad \text{and} \quad f(x) = e^x.\]
Then to find $g'(x)$ we us the product rule and we get $g'(x) = \frac{\sin(x)}{x} + \cos(x)\ln(x)$.
Next $f'(x) = e^x$ and 
hence $f'(g(x)) = e^{g(x)} = e^{\sin(x)\ln(x)} = x^{\sin(x)}$.
We can then conclude $h'(x) = f'(g(x))g'(x) = x^{\sin(x)} \left[ \frac{\sin(x)}{x} + \cos(x)\ln(x)\right]$.
\end{example}

%more question formats below













%\begin{verbatim}
\begin{question}
What is your favorite color?
\begin{multipleChoice}
\choice[correct]{Rainbow}
\choice{Blue}
\choice{Green}
\choice{Red}
\end{multipleChoice}
\begin{freeResponse}
Hello
\end{freeResponse}
\end{question}
%\end{verbatim}





\begin{question}
  Which one will you choose?
  \begin{multipleChoice}
    \choice[correct]{I'm correct.}
    \choice{I'm wrong.}
    \choice{I'm wrong too.}
  \end{multipleChoice}
\end{question}


\begin{question}
  Which one will you choose?
  \begin{selectAll}
    \choice[correct]{I'm correct.}
    \choice{I'm wrong.}
    \choice[correct]{I'm also correct.}
    \choice{I'm wrong too.}
  \end{selectAll}
\end{question}


\begin{freeResponse}
What is the chain rule used for?
\end{freeResponse}
