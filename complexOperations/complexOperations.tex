\documentclass[handout]{ximera}

%% You can put user macros here
%% However, you cannot make new environments



\newcommand{\ffrac}[2]{\frac{\text{\footnotesize $#1$}}{\text{\footnotesize $#2$}}}
\newcommand{\vasymptote}[2][]{
    \draw [densely dashed,#1] ({rel axis cs:0,0} -| {axis cs:#2,0}) -- ({rel axis cs:0,1} -| {axis cs:#2,0});
}


\graphicspath{{./}{firstExample/}}

\usepackage{amsmath}
\usepackage{amssymb}
\usepackage{array}
\usepackage[makeroom]{cancel} %% for strike outs
\usepackage{pgffor} %% required for integral for loops
\usepackage{tikz}
\usepackage{tikz-cd}
\usepackage{tkz-euclide}
\usetikzlibrary{shapes.multipart}


\usetkzobj{all}
\tikzstyle geometryDiagrams=[ultra thick,color=blue!50!black]


\usetikzlibrary{arrows}
\tikzset{>=stealth,commutative diagrams/.cd,
  arrow style=tikz,diagrams={>=stealth}} %% cool arrow head
\tikzset{shorten <>/.style={ shorten >=#1, shorten <=#1 } } %% allows shorter vectors

\usetikzlibrary{backgrounds} %% for boxes around graphs
\usetikzlibrary{shapes,positioning}  %% Clouds and stars
\usetikzlibrary{matrix} %% for matrix
\usepgfplotslibrary{polar} %% for polar plots
\usepgfplotslibrary{fillbetween} %% to shade area between curves in TikZ



%\usepackage[width=4.375in, height=7.0in, top=1.0in, papersize={5.5in,8.5in}]{geometry}
%\usepackage[pdftex]{graphicx}
%\usepackage{tipa}
%\usepackage{txfonts}
%\usepackage{textcomp}
%\usepackage{amsthm}
%\usepackage{xy}
%\usepackage{fancyhdr}
%\usepackage{xcolor}
%\usepackage{mathtools} %% for pretty underbrace % Breaks Ximera
%\usepackage{multicol}



\newcommand{\RR}{\mathbb R}
\newcommand{\R}{\mathbb R}
\newcommand{\C}{\mathbb C}
\newcommand{\N}{\mathbb N}
\newcommand{\Z}{\mathbb Z}
\newcommand{\dis}{\displaystyle}
%\renewcommand{\d}{\,d\!}
\renewcommand{\d}{\mathop{}\!d}
\newcommand{\dd}[2][]{\frac{\d #1}{\d #2}}
\newcommand{\pp}[2][]{\frac{\partial #1}{\partial #2}}
\renewcommand{\l}{\ell}
\newcommand{\ddx}{\frac{d}{\d x}}

\newcommand{\zeroOverZero}{\ensuremath{\boldsymbol{\tfrac{0}{0}}}}
\newcommand{\inftyOverInfty}{\ensuremath{\boldsymbol{\tfrac{\infty}{\infty}}}}
\newcommand{\zeroOverInfty}{\ensuremath{\boldsymbol{\tfrac{0}{\infty}}}}
\newcommand{\zeroTimesInfty}{\ensuremath{\small\boldsymbol{0\cdot \infty}}}
\newcommand{\inftyMinusInfty}{\ensuremath{\small\boldsymbol{\infty - \infty}}}
\newcommand{\oneToInfty}{\ensuremath{\boldsymbol{1^\infty}}}
\newcommand{\zeroToZero}{\ensuremath{\boldsymbol{0^0}}}
\newcommand{\inftyToZero}{\ensuremath{\boldsymbol{\infty^0}}}


\newcommand{\numOverZero}{\ensuremath{\boldsymbol{\tfrac{\#}{0}}}}
\newcommand{\dfn}{\textbf}
%\newcommand{\unit}{\,\mathrm}
\newcommand{\unit}{\mathop{}\!\mathrm}
%\newcommand{\eval}[1]{\bigg[ #1 \bigg]}
\newcommand{\eval}[1]{ #1 \bigg|}
\newcommand{\seq}[1]{\left( #1 \right)}
\renewcommand{\epsilon}{\varepsilon}
\renewcommand{\iff}{\Leftrightarrow}

\DeclareMathOperator{\arccot}{arccot}
\DeclareMathOperator{\arcsec}{arcsec}
\DeclareMathOperator{\arccsc}{arccsc}
\DeclareMathOperator{\si}{Si}
\DeclareMathOperator{\proj}{proj}
\DeclareMathOperator{\scal}{scal}
\DeclareMathOperator{\cis}{cis}
\DeclareMathOperator{\Arg}{Arg}
%\DeclareMathOperator{\arg}{arg}
\DeclareMathOperator{\Rep}{Re}
\DeclareMathOperator{\Imp}{Im}
\DeclareMathOperator{\sech}{sech}
\DeclareMathOperator{\csch}{csch}
\DeclareMathOperator{\Log}{Log}

\newcommand{\tightoverset}[2]{% for arrow vec
  \mathop{#2}\limits^{\vbox to -.5ex{\kern-0.75ex\hbox{$#1$}\vss}}}
\newcommand{\arrowvec}{\overrightarrow}
\renewcommand{\vec}{\mathbf}
\newcommand{\veci}{{\boldsymbol{\hat{\imath}}}}
\newcommand{\vecj}{{\boldsymbol{\hat{\jmath}}}}
\newcommand{\veck}{{\boldsymbol{\hat{k}}}}
\newcommand{\vecl}{\boldsymbol{\l}}
\newcommand{\utan}{\vec{\hat{t}}}
\newcommand{\unormal}{\vec{\hat{n}}}
\newcommand{\ubinormal}{\vec{\hat{b}}}

\newcommand{\dotp}{\bullet}
\newcommand{\cross}{\boldsymbol\times}
\newcommand{\grad}{\boldsymbol\nabla}
\newcommand{\divergence}{\grad\dotp}
\newcommand{\curl}{\grad\cross}
%% Simple horiz vectors
\renewcommand{\vector}[1]{\left\langle #1\right\rangle}


\pgfplotsset{compat=1.13}

\outcome{Visualize operations on complex numbers}

\title{1.5 Operations in the Complex Plane}

\begin{document}

\begin{abstract}
We visualize operations on complex numbers in the complex plane.
\end{abstract}

\maketitle

\section{The Complex Conjugate}


In the complex plane, a point $z$ and its conjugate ${\overline z}$ are symmetric about the real axis.

\begin{image}
\begin{tikzpicture}
\draw[->, thick] (-0.4,0)--(3,0) node[right]{$x$};
%\node[blue] at (2.9,.5){Real Axis};
\draw[<->, thick] (0, -2)--(0,2) node[left]{$iy$};
%\node[blue] at (-1.5, 2.9){Imaginary Axis};

%\draw[ultra thin] (2.6,.1)--(2.6,-.1) node[below]{$a$};

\node[blue] at (1.3, -2.4){Conjugate symmetry};

%\draw[ultra thin] (.1,1.8)--(-.1,1.8) node[left]{$bi$};

\draw[blue] (0, 0) -- (1.8, 1.2);
\draw[blue] (0, 0) -- (1.8, -1.2);

%\node[blue] at (1.3, -0.2){$r \cos \theta$};
%[snake=brace, blue, mirror snake]

\draw[mark=*,mark size=1pt, blue] plot coordinates {(1.8,1.2)} node[above right, blue]{$z$};
\draw[mark=*,mark size=1pt, blue] plot coordinates {(1.8,-1.2)} node[below right, blue]{${\overline z}$};


%\draw[ultra thin, dashed, blue] (2.6,0.3) -- (2.6,1.6) ;
%\draw[snake=brace, blue, mirror snake] (2.7, 0.1) -- (2.7, 1.7) node[midway, right , blue]{Im($a+bi$)};
%\draw[ultra thin, dashed, blue] (0.2,1.8) -- (2.4,1.8) node[midway, above, blue]{Re($a+bi$)};
%\draw[snake=brace, blue] (0.1,0.1) -- (2.5,1.9) node[midway, above, blue, sloped]{$|a+bi|$};
%\draw[ ->, blue] (0,0) -- (2.58,1.78) node[midway, above, blue, sloped]{$r$};
%\node[blue] at (0.6, 0.2){$\theta$};
\end{tikzpicture}
\end{image}


\section{The Negative of a Complex Number}


In the complex plane, the points  $z$ and  $-z$ are symmetric about the origin.

\begin{image}
\begin{tikzpicture}
\draw[<->, thick] (-2,0)--(2,0) node[right]{$x$};
%\node[blue] at (2.9,.5){Real Axis};
\draw[<->, thick] (0, -2)--(0,2) node[left]{$iy$};
%\node[blue] at (-1.5, 2.9){Imaginary Axis};

%\draw[ultra thin] (2.6,.1)--(2.6,-.1) node[below]{$a$};

\node[blue] at (0, -2.4){Complex negatives};

%\draw[ultra thin] (.1,1.8)--(-.1,1.8) node[left]{$bi$};

\draw[blue] (0, 0) -- (.8, 1.2);
\draw[blue] (0, 0) -- (-.8, -1.2);

%\node[blue] at (1.3, -0.2){$r \cos \theta$};
%[snake=brace, blue, mirror snake]

\draw[mark=*,mark size=1pt, blue] plot coordinates {(.8, 1.2)} node[above right, blue]{$z$};
\draw[mark=*,mark size=1pt, blue] plot coordinates {(-.8, -1.2)} node[below left, blue]{$-z$};


%\draw[ultra thin, dashed, blue] (2.6,0.3) -- (2.6,1.6) ;
%\draw[snake=brace, blue, mirror snake] (2.7, 0.1) -- (2.7, 1.7) node[midway, right , blue]{Im($a+bi$)};
%\draw[ultra thin, dashed, blue] (0.2,1.8) -- (2.4,1.8) node[midway, above, blue]{Re($a+bi$)};
%\draw[snake=brace, blue] (0.1,0.1) -- (2.5,1.9) node[midway, above, blue, sloped]{$|a+bi|$};
%\draw[ ->, blue] (0,0) -- (2.58,1.78) node[midway, above, blue, sloped]{$r$};
%\node[blue] at (0.6, 0.2){$\theta$};
\end{tikzpicture}
\end{image}




\section{Complex Addition}


Addition of complex numbers can be visualized in the complex plane by vector addition.


\begin{image}
\begin{tikzpicture}
\draw[->, thick] (-0.4,0)--(3.5,0) node[right]{$x$};
%\node[blue] at (2.9,.5){Real Axis};
\draw[->, thick] (0, -0.4)--(0,2.8) node[left]{$iy$};
%\node[blue] at (-1.5, 2.9){Imaginary Axis};

%\draw[ultra thin] (2.6,.1)--(2.6,-.1) node[below]{$a$};

\node[blue] at (1.6, -0.8){Parallelogram method};

%\draw[ultra thin] (.1,1.8)--(-.1,1.8) node[left]{$bi$};

\draw[blue] (0, 0) -- (1.75, 0.5) node[blue, right]{$z_1$};
\draw[blue] (0, 0) -- (0.75, 1.5) node[blue, above]{$z_2$};
\draw[blue] (0, 0) -- (2.5, 2) node[blue, above right]{$z_1 + z_2$};

\draw[blue, dashed] (1.75, 0.5) -- (2.5, 2);
\draw[blue, dashed] (0.75,1.5) -- (2.5, 2) ;

%\node[blue] at (1.3, -0.2){$r \cos \theta$};
%[snake=brace, blue, mirror snake]

\draw[mark=*,mark size=1pt, blue] plot coordinates {(1.75,0.5)} ;
\draw[mark=*,mark size=1pt, blue] plot coordinates {(0.75,1.5)};
\draw[mark=*,mark size=1pt, blue] plot coordinates {(2.5,2)};

%\draw[ultra thin, dashed, blue] (2.6,0.3) -- (2.6,1.6) ;
%\draw[snake=brace, blue, mirror snake] (2.7, 0.1) -- (2.7, 1.7) node[midway, right , blue]{Im($a+bi$)};
%\draw[ultra thin, dashed, blue] (0.2,1.8) -- (2.4,1.8) node[midway, above, blue]{Re($a+bi$)};
%\draw[snake=brace, blue] (0.1,0.1) -- (2.5,1.9) node[midway, above, blue, sloped]{$|a+bi|$};
%\draw[ ->, blue] (0,0) -- (2.58,1.78) node[midway, above, blue, sloped]{$r$};
%\node[blue] at (0.6, 0.2){$\theta$};
\end{tikzpicture}
\end{image}

If $z_1 = x_1 + iy_1$ and $z_2 = x_2 + iy_2$ then the sum is formed by adding the corresponding components:
\[
z_1 +z_2= (x_1+x_2) + i(y_1+y_2)
\]

This can be expressed in terms of real and imaginary parts as
\[
\Rep(z_1 + z_2) = \Rep(z_1)+ \Rep(z_2)
\]
and
\[
\Imp(z_1+z_2) = \Imp(z_1)+\Imp(z_2)
\]

The above figure suggests the triangle inequality, which is proved at the end of the section:
\[
|z_1 + z_2| \leq |z_1| + |z_2|
\]


\begin{image}
\begin{tikzpicture}
%\draw[->, thick] (-0.4,0)--(3.5,0) node[right]{$x$};
%\node[blue] at (2.9,.5){Real Axis};
%\draw[->, thick] (0, -0.4)--(0,2.8) node[left]{$iy$};
%\node[blue] at (-1.5, 2.9){Imaginary Axis};

%\draw[ultra thin] (2.6,.1)--(2.6,-.1) node[below]{$a$};

\node[blue] at (1.6, -0.8){Triangle Inequality: $|z_1 + z_2| \leq |z_1| + |z_2|$};

%\draw[ultra thin] (.1,1.8)--(-.1,1.8) node[left]{$bi$};

\draw[blue] (0, 0) -- (1.75, 0.5) node[blue,  midway, below, xshift = 2mm]{$|z_1|$};
%\draw[blue] (0, 0) -- (0.75, 1.5) node[blue, above]{$z_2$};
\draw[blue] (0, 0) -- (2.5, 2) node[blue, midway, left, yshift = 2mm]{$|z_1 + z_2|$};

\draw[blue] (1.75, 0.5) -- (2.5, 2) node[blue, midway, right, yshift = -1mm]{$|z_2|$};
%\draw[blue, dashed] (0.75,1.5) -- (2.5, 2) ;

%\node[blue] at (1.3, -0.2){$r \cos \theta$};
%[snake=brace, blue, mirror snake]

\draw[mark=*,mark size=1pt, blue] plot coordinates {(1.75,0.5)} ;
\draw[mark=*,mark size=1pt, blue] plot coordinates {(0,0)};
\draw[mark=*,mark size=1pt, blue] plot coordinates {(2.5,2)};

%\draw[ultra thin, dashed, blue] (2.6,0.3) -- (2.6,1.6) ;
%\draw[snake=brace, blue, mirror snake] (2.7, 0.1) -- (2.7, 1.7) node[midway, right , blue]{Im($a+bi$)};
%\draw[ultra thin, dashed, blue] (0.2,1.8) -- (2.4,1.8) node[midway, above, blue]{Re($a+bi$)};
%\draw[snake=brace, blue] (0.1,0.1) -- (2.5,1.9) node[midway, above, blue, sloped]{$|a+bi|$};
%\draw[ ->, blue] (0,0) -- (2.58,1.78) node[midway, above, blue, sloped]{$r$};
%\node[blue] at (0.6, 0.2){$\theta$};
\end{tikzpicture}
\end{image}



The modulus of a difference gives the distance between the complex numbers.

\begin{image}
\begin{tikzpicture}
\draw[->, thick] (-0.4,0)--(3.5,0) node[right]{$x$};
%\node[blue] at (2.9,.5){Real Axis};
\draw[->, thick] (0, -0.4)--(0,2.8) node[left]{$iy$};
%\node[blue] at (-1.5, 2.9){Imaginary Axis};

%\draw[ultra thin] (2.6,.1)--(2.6,-.1) node[below]{$a$};

\node[blue] at (1.6, -0.8){Distance between complex numbers};

%\draw[ultra thin] (.1,1.8)--(-.1,1.8) node[left]{$bi$};

\draw[blue] (0, 0) -- (1.75, 0.5) node[blue, right]{$z_1$};
\draw[blue] (0, 0) -- (0.75, 1.5) node[blue, above]{$z_2$};
\draw[blue] (1.75, 0.5) -- (0.75, 1.5) node[blue, midway, right, yshift=2mm]{$|z_2 - z_1|$};





\draw[mark=*,mark size=1pt, blue] plot coordinates {(1.75,0.5)} ;
\draw[mark=*,mark size=1pt, blue] plot coordinates {(0.75,1.5)};

\end{tikzpicture}
\end{image}


\section{Complex Multiplication}

Using the polar form, we can prove that complex multiplication corresponds to
multiplying the moduli and adding the arguments.
If $z_1 = r_1 \cis \theta_1$ and $z_2 = r_2 \cis \theta_2$, then
\[
 z_1 z_2 = r_1 r_2\cis \left(\theta_1+\theta_2\right) 
 \]

This follows from the sum identities for the sine and cosine:
\[
\sin\left(\theta_1+\theta_2\right) = \sin\theta_1 \cos \theta_2 +  \cos \theta_1 \sin \theta_2
\]
and
\[
\cos\left(\theta_1+\theta_2\right) = \cos\theta_1 \cos \theta_2 -  \sin \theta_1 \sin \theta_2
\]

In particular,
\begin{align*}
\cis \theta_1 \cis \theta_2 & = \left( \cos\theta_1 +i\sin \theta_1 \right)  \left( \cos \theta_2 + i  \sin \theta_2 \right)\\
  &=  \left(\cos\theta_1 \cos \theta_2  - \sin\theta_1 \sin \theta_2\right)+i \left(\sin \theta_1\cos \theta_2 + \cos \theta_1 \sin \theta_2 \right)\\
 &= \cos\left(\theta_1+\theta_2\right) + i \sin \left(\theta_1+\theta_2\right) \\
 &= \cis \left(\theta_1+\theta_2\right) 
\end{align*}



\begin{image}
\begin{tikzpicture}
\draw[<->, thick] (-3,0)--(3,0) node[right]{$x$};
\draw[<->, thick] (0, -3)--(0,3) node[left]{$iy$};


\draw[blue] (0, 0) -- (.5, .85) node[blue, above right]{$|z_1| = 2, \theta_1 = \pi/3$} ;
\draw[blue] (0, 0) -- (-1.3, .75) node[blue, above, xshift = -5mm]{$|z_2| = 3, \theta_2 = 5\pi/6$} ;
\draw[blue] (0, 0) -- (-2.6, -1.5)node[blue, below, yshift = -1mm ]{$|z_3| = 6, \theta_3 = 7\pi/6$} ;


\draw[mark=*,mark size=1pt, blue] plot coordinates {(.5,.85)} ;
\draw[mark=*,mark size=1pt, blue] plot coordinates {(-1.3, .75)} ;
\draw[mark=*,mark size=1pt, blue] plot coordinates {(-2.6, -1.5)} ;

\node[blue] at (0, -3.4){Complex multiplication, $z_3 = z_1 z_2$};

\end{tikzpicture}
\end{image}



\section{Proofs of the Triangle Inequality}

The first proof uses the \link[law of cosines]{https://en.wikipedia.org/wiki/Law_of_cosines}:
\[
|z_1 + z_2|^2 = |z_1|^2 + |z_2|^2 - 2|z_1|\cdot |z_2|\cos \theta,
\]
where $\theta$ is the angle opposite the side of length $|z_1 + z_2|$.
Since $-1 \leq \cos \theta \leq 1$, we have
\[
|z_1 + z_2|^2 \leq |z_1|^2 + |z_2|^2 + 2|z_1|\cdot |z_2| = \left(|z_1|^2 + |z_2|^2\right),
\]
and the result follows by taking square roots.

The second proof uses the fact that for any real numbers $a$ and $b$,
\[
2ab \leq a^2 + b^2
\]
Let $z_1 = x_1 + iy_1$ and $z_2 = x_2 + iy_2$. Then $z_1 + z_2 = (x_1 + x_2) + i(y_1 + y_2)$ and so
\[
|z_1 + z_2|^2 = (x_1 +x_2)^2 + (y_1 +y_2)^2
\]
Expanding the right hand side gives
\[
|z_1 + z_2|^2 = x_1^2 +2x_1x_2 +x_2^2 + y_1^2 + 2y_1y_2 +y_2^2 = |z_1|^2 + |z_2|^2 + 2(x_1x_2 + y_1y_2)
\]
The result will follow once we show that $(x_1x_2 + y_1y_2)^2 \leq |z_1|^2 |z_2|^2$.
Expanding the left hand side gives
\[
(x_1x_2 + y_1y_2)^2 = (x_1x_2)^2 + 2(x_1x_2y_1y_2) + (y_1y_2)^2
\]
Expanding the right hand side gives
\[
|z_1|^2 |z_2|^2 = (x_1^2+y_1^2)(x_2^2+y_2^2) = (x_1x_2)^2 + x_1^2y_2^2 + x_2^2y_1^2 + (y_1y_2)^2
\]
Finally,
\[
2(x_1x_2y_1y_2) \leq x_1^2y_2^2 + x_2^2y_1^2
\]
using $2ab \leq a^2 + b^2$ with $a = x_1y_2$ and $b=x_2y_1$.

The third and final proof uses the following properties of complex numbers:


\begin{itemize}
\item[(1)] $|z|^2 = z{\overline z} $
\item[(2)] ${\overline {z_1 + z_2}}= {\overline z_1} +{\overline z_2} $
\item[(3)] ${\overline{z_1z_2}} = {\overline z_1} {\overline z_2}$
\item[(4)] ${\overline {\overline z}} = z $
\item[(5)]  $\Rep(z) = \frac12 (z+{\overline z}) $
\item[(6)]  $\Rep(z) \leq |z| $
\item[(7)] $|{\overline z}| = |z|$ 
\item[(8)] $|z_1z_2| = |z_1|\cdot |z_2|$
\end{itemize}

We have
\begin{align*}
|z_1 + z_2|^2 &= (z_1 + z_2)({\overline {z_1 + z_2}}), \; \mbox{by} \;  (1)\\
              &= (z_1 + z_2)({\overline z_1} +{\overline z_2}), \; \mbox{by} \; (2)\\
              &= z_1{\overline z_1} + z_1{\overline z_2}+ {\overline z_1}z_2 + z_2{\overline z_2} \\
              &= |z_1|^2 + z_1{\overline z_2} + {\overline {z_1{\overline z_2}}} + |z_2|^2 ,\; \mbox{by} \;(3) \, \mbox{and} \, (4)\\
              &= |z_1|^2 + 2\Rep(z_1{\overline z_2}) + |z_2|^2, \; \mbox{by} \; (5)  \\
              & \leq |z_1|^2 + 2 |z_1{\overline z_2}| + |z_2|^2, \; \mbox{by} \;(6)\\
              &= |z_1|^2 + 2 |z_1| \cdot |z_2| + |z_2|^2, \; \mbox{by} \;(7) \, \mbox{and} \, (8)\\
              &= |z_1 + z_2|^2
\end{align*}
The result follow by taking square roots.


\end{document}



\begin{align}
&|z|^2 = z{\overline z} \\
& {\overline {z_1 + z_2}}= {\overline z_1} +{\overline z_2} \\
 & {\overline {\overline z}} = z \\
 & {\overline{z_1z_2}} = {\overline z_1} {\overline z_2}\\
 &  \Rep(z) \leq |z| \\ 
&|{\overline z}| = |z| \\
& |z_1z_2| = |z_1|\cdot |z_2|
\end{align}
