\documentclass[handout]{ximera}

%% You can put user macros here
%% However, you cannot make new environments



\newcommand{\ffrac}[2]{\frac{\text{\footnotesize $#1$}}{\text{\footnotesize $#2$}}}
\newcommand{\vasymptote}[2][]{
    \draw [densely dashed,#1] ({rel axis cs:0,0} -| {axis cs:#2,0}) -- ({rel axis cs:0,1} -| {axis cs:#2,0});
}


\graphicspath{{./}{firstExample/}}

\usepackage{amsmath}
\usepackage{amssymb}
\usepackage{array}
\usepackage[makeroom]{cancel} %% for strike outs
\usepackage{pgffor} %% required for integral for loops
\usepackage{tikz}
\usepackage{tikz-cd}
\usepackage{tkz-euclide}
\usetikzlibrary{shapes.multipart}


\usetkzobj{all}
\tikzstyle geometryDiagrams=[ultra thick,color=blue!50!black]


\usetikzlibrary{arrows}
\tikzset{>=stealth,commutative diagrams/.cd,
  arrow style=tikz,diagrams={>=stealth}} %% cool arrow head
\tikzset{shorten <>/.style={ shorten >=#1, shorten <=#1 } } %% allows shorter vectors

\usetikzlibrary{backgrounds} %% for boxes around graphs
\usetikzlibrary{shapes,positioning}  %% Clouds and stars
\usetikzlibrary{matrix} %% for matrix
\usepgfplotslibrary{polar} %% for polar plots
\usepgfplotslibrary{fillbetween} %% to shade area between curves in TikZ



%\usepackage[width=4.375in, height=7.0in, top=1.0in, papersize={5.5in,8.5in}]{geometry}
%\usepackage[pdftex]{graphicx}
%\usepackage{tipa}
%\usepackage{txfonts}
%\usepackage{textcomp}
%\usepackage{amsthm}
%\usepackage{xy}
%\usepackage{fancyhdr}
%\usepackage{xcolor}
%\usepackage{mathtools} %% for pretty underbrace % Breaks Ximera
%\usepackage{multicol}



\newcommand{\RR}{\mathbb R}
\newcommand{\R}{\mathbb R}
\newcommand{\C}{\mathbb C}
\newcommand{\N}{\mathbb N}
\newcommand{\Z}{\mathbb Z}
\newcommand{\dis}{\displaystyle}
%\renewcommand{\d}{\,d\!}
\renewcommand{\d}{\mathop{}\!d}
\newcommand{\dd}[2][]{\frac{\d #1}{\d #2}}
\newcommand{\pp}[2][]{\frac{\partial #1}{\partial #2}}
\renewcommand{\l}{\ell}
\newcommand{\ddx}{\frac{d}{\d x}}

\newcommand{\zeroOverZero}{\ensuremath{\boldsymbol{\tfrac{0}{0}}}}
\newcommand{\inftyOverInfty}{\ensuremath{\boldsymbol{\tfrac{\infty}{\infty}}}}
\newcommand{\zeroOverInfty}{\ensuremath{\boldsymbol{\tfrac{0}{\infty}}}}
\newcommand{\zeroTimesInfty}{\ensuremath{\small\boldsymbol{0\cdot \infty}}}
\newcommand{\inftyMinusInfty}{\ensuremath{\small\boldsymbol{\infty - \infty}}}
\newcommand{\oneToInfty}{\ensuremath{\boldsymbol{1^\infty}}}
\newcommand{\zeroToZero}{\ensuremath{\boldsymbol{0^0}}}
\newcommand{\inftyToZero}{\ensuremath{\boldsymbol{\infty^0}}}


\newcommand{\numOverZero}{\ensuremath{\boldsymbol{\tfrac{\#}{0}}}}
\newcommand{\dfn}{\textbf}
%\newcommand{\unit}{\,\mathrm}
\newcommand{\unit}{\mathop{}\!\mathrm}
%\newcommand{\eval}[1]{\bigg[ #1 \bigg]}
\newcommand{\eval}[1]{ #1 \bigg|}
\newcommand{\seq}[1]{\left( #1 \right)}
\renewcommand{\epsilon}{\varepsilon}
\renewcommand{\iff}{\Leftrightarrow}

\DeclareMathOperator{\arccot}{arccot}
\DeclareMathOperator{\arcsec}{arcsec}
\DeclareMathOperator{\arccsc}{arccsc}
\DeclareMathOperator{\si}{Si}
\DeclareMathOperator{\proj}{proj}
\DeclareMathOperator{\scal}{scal}
\DeclareMathOperator{\cis}{cis}
\DeclareMathOperator{\Arg}{Arg}
%\DeclareMathOperator{\arg}{arg}
\DeclareMathOperator{\Rep}{Re}
\DeclareMathOperator{\Imp}{Im}
\DeclareMathOperator{\sech}{sech}
\DeclareMathOperator{\csch}{csch}
\DeclareMathOperator{\Log}{Log}

\newcommand{\tightoverset}[2]{% for arrow vec
  \mathop{#2}\limits^{\vbox to -.5ex{\kern-0.75ex\hbox{$#1$}\vss}}}
\newcommand{\arrowvec}{\overrightarrow}
\renewcommand{\vec}{\mathbf}
\newcommand{\veci}{{\boldsymbol{\hat{\imath}}}}
\newcommand{\vecj}{{\boldsymbol{\hat{\jmath}}}}
\newcommand{\veck}{{\boldsymbol{\hat{k}}}}
\newcommand{\vecl}{\boldsymbol{\l}}
\newcommand{\utan}{\vec{\hat{t}}}
\newcommand{\unormal}{\vec{\hat{n}}}
\newcommand{\ubinormal}{\vec{\hat{b}}}

\newcommand{\dotp}{\bullet}
\newcommand{\cross}{\boldsymbol\times}
\newcommand{\grad}{\boldsymbol\nabla}
\newcommand{\divergence}{\grad\dotp}
\newcommand{\curl}{\grad\cross}
%% Simple horiz vectors
\renewcommand{\vector}[1]{\left\langle #1\right\rangle}


\outcome{Find the power series representation of a function}

\title{3.12 Power Series Representation}

\begin{document}

\begin{abstract}
We find the power series representation of a function.
\end{abstract}

\maketitle

\section{Power Series}

In this section we will use power series to represent familiar functions.
A \textbf{power series representation} of a function is a convergent power series whose sum is equal to the given function.
Our motivation will be the geometric power series that we saw in the last section,
\[
\sum_{n=0}^\infty x^n,
\]
which converges when $|x| < 1$. Moreover, since this is a geometric series, we can find the sum of this series
and this sum gives us the primary example of a power series representation of a function.

\begin{example}[example 1]
Find a power series representation for the function 
\[
f(x) = \frac{1}{1-x}.
\]
The sum of the geometric series
\[
\sum_{n=0}^\infty x^n = 1 + x + x^2 + x^3 + \dots =  \frac{1}{1-x},
\]
as long as $|x| < 1$.  Thus, the power series representation of the function 
\[
f(x) = \frac{1}{1-x},
\]
is given by 
\[
\sum_{n=0}^\infty x^n,
\]
and this representation is valid as long as $|x| < 1$.
\end{example}

The result of this example will be used throughout the remainder of this section.


\begin{problem}(problem 1)
Find a power series representation for the function
\[
f(t) = \frac{1}{1-t}
\]

The power series representation is $f(t) = $
\begin{multipleChoice}
\choice[correct]{$\displaystyle{\sum_{n=0}^\infty t^n,\; |t|<1}$}
\choice{$\displaystyle{\sum_{n=0}^\infty t^n, \; |t|\leq 1}$}
\choice{$\displaystyle{\sum_{n=1}^\infty t^n, \;  |t|<1}$}
\end{multipleChoice}
\end{problem}




\begin{example}[example 2]
Find a power series representation for the function
\[
f(x) = \frac{2}{1-x},
\]
and determine the interval on which this representation is valid.\\
The function in this example is twice the function in the previous example, 
so we will simply double the power series representation from that example.
The power series representation is
\[
2\sum_{n=0}^\infty x^n,  \; \text{or} \; \sum_{n=0}^\infty 2x^n,
\]
and this representation is valid as long as $|x| < 1$.

\end{example}


\begin{problem}(problem 2)
Find a power series representation for the function
\[
f(x) = \frac{4}{1-x}
\]

The power series representation is $f(x) = $
\begin{multipleChoice}
\choice{$\displaystyle{\sum_{n=0}^\infty 4x^n, \;  |x|\leq 1}$}
\choice[correct]{$\displaystyle{\sum_{n=0}^\infty 4x^n, \;  |x|<1}$}
\choice{$\displaystyle{\sum_{n=1}^\infty 4x^n, \;  |x|<1}$}
\end{multipleChoice}
\end{problem}


\begin{example}[example 3]
Find a power series representation for the function
\[
f(x) = \frac{x}{1-x},
\]
and determine the interval on which this representation is valid.\\
The function in this example is $x$ times the function in example 1, 
so we will simply multiply the power series representation from that example by $x$.
The power series representation is
\[
f(x) = x\sum_{n=0}^\infty x^n  = \sum_{n=0}^\infty x\cdot x^n = \sum_{n=0}^\infty x^{n+1},
\]
and this representation is valid as long as $|x| < 1$.

\end{example}


\begin{problem}(problem 3)
Find a power series representation for the function
\[
f(x) = \frac{x^2}{1-x}
\]

The power series representation is $f(x) = $
\begin{multipleChoice}
\choice{$\displaystyle{\sum_{n=0}^\infty x^{n+2}, \;  |x|\leq 1}$}
\choice{$\displaystyle{\sum_{n=1}^\infty x^{n+2}, \;  |x|<1}$}
\choice[correct]{$\displaystyle{\sum_{n=0}^\infty x^{n+2}, \;  |x|<1}$}
\end{multipleChoice}
\end{problem}


\begin{example}[example 4]
Find a power series representation for the function
\[
f(x) = \frac{1}{1 + x},
\]
and determine the interval on which this representation is valid.\\
Since $1 + x = 1 - (-x)$, we can use the result of example 1 with $-x$ in the place of $x$.
Thus, the power series representation is

\[
f(x) = \sum_{n=0}^\infty (-x)^n = \sum_{n=0}^\infty (-1)^n x^n.
\]
This representation is valid as long as $|-x| < 1$, which is equivalent to $|x| < 1$.
Note the final form of the answer is the standard form of a power series,
\[
\sum_{n=0}^\infty c_nx^n.
\]

\end{example}


\begin{problem}(problem 4a)
Find a power series representation for the function
\[
f(x) = \frac{1}{1+ 2x}
\]

The power series representation is $f(x) = $
\begin{multipleChoice}
\choice{$\displaystyle{\sum_{n=0}^\infty  2^nx^n, \;  |x|< 1}$}
\choice[correct]{$\displaystyle{\sum_{n=0}^\infty (-1)^n2^nx^n, \;  |x|<1/2}$}
\choice{$\displaystyle{\sum_{n=0}^\infty (-1)^n 2^nx^n, \;  |x|<2}$}
\end{multipleChoice}
\end{problem}




\begin{problem}(problem 4b)
Find a power series representation for the function
\[
f(x) = \frac{x}{1-3x}
\]

The power series representation is $f(x) = $
\begin{multipleChoice}
\choice{$\displaystyle{\sum_{n=0}^\infty 3^nx^{n+1}, \;  |x|< 3}$}
\choice{$\displaystyle{\sum_{n=0}^\infty 3^nx^{n+1}, \;  |x|< 1}$}
\choice[correct]{$\displaystyle{\sum_{n=0}^\infty 3^nx^{n+1}, \;  |x|< 1/3}$}
\end{multipleChoice}
\end{problem}



\begin{example}[example 5]
Find a power series representation for the function
\[
f(x) = \frac{1}{1-x^2},
\]
and determine the interval on which this representation is valid.\\
We can use the result of example 1 with $x^2$ in the place of $x$.
Thus the power series representation is
\[
f(x) = \sum_{n=0}^\infty (x^2)^n  = \sum_{n=0}^\infty x^{2n}. 
\]
This representation is valid as long as $|x^2| < 1$, which is equivalent to $|x| < 1$.

\end{example}


\begin{problem}(problem 5)
Find a power series representation for the function
\[
f(x) = \frac{1}{1+x^2}
\]

The power series representation is $f(x) = $
\begin{multipleChoice}
\choice[correct]{$\displaystyle{\sum_{n=0}^\infty (-1)^nx^{2n}, \;  |x|<1}$}
\choice{$\displaystyle{\sum_{n=0}^\infty  x^{2n},  \; |x|< 1}$}
\choice{$\displaystyle{\sum_{n=1}^\infty (-1)^nx^{2n}, \;  |x|<1}$}
\end{multipleChoice}
\end{problem}

\begin{example}[example 6]
Find a power series representation for the function
\[
f(x) = \frac{1}{2-x},
\]
and determine the interval on which this representation is valid.\\
The function in this example resembles the function in example 1 except that there is a `2' in the denominator instead of a `1'.
This is an important distinction and we need the `1'. To create it, we will factor `2' from the denominator:
\[
f(x) = \frac{1}{2 -x} = \frac{1}{2(1 - \frac{x}{2})} = \frac12 \cdot \frac{1}{1 - \frac{x}{2}}.
\]
Now we have $1/2$ times a function like the one in example 1 with $x/2$ in the place of $x$.
Thus, the power series representation is
\[
f(x) = \frac12 \sum_{n=0}^\infty \left(\frac{x}{2}\right)^n  = \frac12 \sum_{n=0}^\infty \frac{x^n}{2^n}
=  \sum_{n=0}^\infty \frac12 \cdot \frac{x^n}{2^n} =  \sum_{n=0}^\infty \frac{x^n}{2^{n+1}}.
\]
This representation is valid as long as $|x/2| < 1$ which is equivalent to $|x| < 2$.

\end{example}


\begin{problem}(problem 6a)
Find a power series representation for the function
\[
f(x) = \frac{1}{3-x}
\]

The power series representation is $f(x) = $
\begin{multipleChoice}
\choice[correct]{$\displaystyle{ \sum_{n=0}^\infty \frac{x^n}{3^{n+1}}, \;  |x|< 3}$}
\choice{$\displaystyle{\sum_{n=0}^\infty \frac{x^n}{3^n},  \; |x|< 3}$}
\choice{$\displaystyle{\sum_{n=0}^\infty 3^{n+1}x^n,  \; |x|< 3}$}
\end{multipleChoice}
\end{problem}



\begin{problem}(problem 6b)
Find a power series representation for the function
\[
f(x) = \frac{2}{4+x}
\]

The power series representation is $f(x) = $
\begin{multipleChoice}
\choice{$\displaystyle{ \sum_{n=0}^\infty (-1)^n \frac{x^n}{4^{n+1}}, \;  |x|< 4}$}
\choice{$\displaystyle{\sum_{n=0}^\infty \frac{x^n}{2^{2n+1}}, \;  |x|< 4}$}
\choice[correct]{$\displaystyle{\sum_{n=0}^\infty (-1)^n \frac{x^n}{2^{2n+1}}, \;  |x|< 4}$}
\end{multipleChoice}
\end{problem}

In the next example, we combine the ideas of the previous examples.


\begin{example}[example 7]
Find a power series representation for the function
\[
f(x) = \frac{3x^3}{4 + x^2},
\]
and determine the interval on which this representation is valid.\\
We begin by factoring a `4' from the denominator:
\[
f(x) = \frac{3x^3}{4+x^2} = \frac{3x^3}{4(1 + \frac{x^2}{4})} = \frac{3x^3}{4} \cdot \frac{1}{1 - (-\frac{x^2}{4})}.
\]
We can now apply the result of example 1 with $-x^2 /4$ in place of $x$:
\[
f(x) = \frac{3x^3}{4} \sum_{n=0}^\infty \left(-\frac{x^2}{4}\right)^n = \frac{3x^3}{4} \sum_{n=0}^\infty (-1)^n \frac{(x^2)^n}{4^n}
=  \sum_{n=0}^\infty \frac{3x^3}{4} \cdot (-1)^n \frac{x^{2n}}{4^n} =  \sum_{n=0}^\infty 3 (-1)^n \frac{x^{2n+3}}{4^{n+1}}. 
\]
This representation is valid as long as $|-x^2/4| < 1$ which is equivalent to $|x^2| < 4$ which, in turn, is equivalent to $|x| < 2$.

\end{example}


\begin{problem}(problem 7)
Find a power series representation for the function
\[
f(x) = \frac{x^2}{x^3+8}
\]

the power series representation is $f(x) = $
\begin{multipleChoice}
\choice{$\displaystyle{ \sum_{n=0}^\infty (-1)^n \frac{x^{3n+2}}{8^n}, \; |x|< 2}$}
\choice[correct]{$\displaystyle{\sum_{n=0}^\infty (-1)^n\frac{x^{3n+2}}{8^{n+1}}, \;  |x|< 2}$}
\choice{$\displaystyle{\sum_{n=0}^\infty \frac{x^{3n}}{8^{n+1}}, \;  |x|< 2}$}
\end{multipleChoice}
\end{problem}


\begin{center}
\begin{foldable}
\unfoldable{Here is a detailed, lecture style video on power series representation of functions:}
\youtube{3zMHWB3TcEs}
\end{foldable}
\end{center}





\end{document}


\begin{example} %example #15
Find $h'(x)$ if $h(x) = x^{\sin(x)}$.\\
We will use the fact that the exponential and logarithm functions are inverses,
\[e^{\ln(x)} = x,\]
and the exponent property of logarithms, 
\[\ln(x^n) = n\ln(x),\]
to rewrite $h(x)$.  We have 
\[h(x) = x^{\sin(x)} = e^{\ln(x^{\sin(x)})} = e^{\sin(x)\ln(x)}\]
and we can now compute $h'(x)$ using a combination of the chain rule and product rule.
We can write $h(x)$ as a composition, $f(g(x))$ with 
\[g(x) = \sin(x)\ln(x) \quad \text{and} \quad f(x) = e^x.\]
Then to find $g'(x)$ we us the product rule and we get $g'(x) = \frac{\sin(x)}{x} + \cos(x)\ln(x)$.
Next $f'(x) = e^x$ and 
hence $f'(g(x)) = e^{g(x)} = e^{\sin(x)\ln(x)} = x^{\sin(x)}$.
We can then conclude $h'(x) = f'(g(x))g'(x) = x^{\sin(x)} \left[ \frac{\sin(x)}{x} + \cos(x)\ln(x)\right]$.
\end{example}

%more question formats below













%\begin{verbatim}
\begin{question}
What is your favorite color?
\begin{multipleChoice}
\choice[correct]{Rainbow}
\choice{Blue}
\choice{Green}
\choice{Red}
\end{multipleChoice}
\begin{freeResponse}
Hello
\end{freeResponse}
\end{question}
%\end{verbatim}





\begin{question}
  Which one will you choose?
  \begin{multipleChoice}
    \choice[correct]{I'm correct.}
    \choice{I'm wrong.}
    \choice{I'm wrong too.}
  \end{multipleChoice}
\end{question}


\begin{question}
  Which one will you choose?
  \begin{selectAll}
    \choice[correct]{I'm correct.}
    \choice{I'm wrong.}
    \choice[correct]{I'm also correct.}
    \choice{I'm wrong too.}
  \end{selectAll}
\end{question}


\begin{freeResponse}
What is the chain rule used for?
\end{freeResponse}
