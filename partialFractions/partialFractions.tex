\documentclass{ximera}

%% You can put user macros here
%% However, you cannot make new environments



\newcommand{\ffrac}[2]{\frac{\text{\footnotesize $#1$}}{\text{\footnotesize $#2$}}}
\newcommand{\vasymptote}[2][]{
    \draw [densely dashed,#1] ({rel axis cs:0,0} -| {axis cs:#2,0}) -- ({rel axis cs:0,1} -| {axis cs:#2,0});
}


\graphicspath{{./}{firstExample/}}

\usepackage{amsmath}
\usepackage{amssymb}
\usepackage{array}
\usepackage[makeroom]{cancel} %% for strike outs
\usepackage{pgffor} %% required for integral for loops
\usepackage{tikz}
\usepackage{tikz-cd}
\usepackage{tkz-euclide}
\usetikzlibrary{shapes.multipart}


\usetkzobj{all}
\tikzstyle geometryDiagrams=[ultra thick,color=blue!50!black]


\usetikzlibrary{arrows}
\tikzset{>=stealth,commutative diagrams/.cd,
  arrow style=tikz,diagrams={>=stealth}} %% cool arrow head
\tikzset{shorten <>/.style={ shorten >=#1, shorten <=#1 } } %% allows shorter vectors

\usetikzlibrary{backgrounds} %% for boxes around graphs
\usetikzlibrary{shapes,positioning}  %% Clouds and stars
\usetikzlibrary{matrix} %% for matrix
\usepgfplotslibrary{polar} %% for polar plots
\usepgfplotslibrary{fillbetween} %% to shade area between curves in TikZ



%\usepackage[width=4.375in, height=7.0in, top=1.0in, papersize={5.5in,8.5in}]{geometry}
%\usepackage[pdftex]{graphicx}
%\usepackage{tipa}
%\usepackage{txfonts}
%\usepackage{textcomp}
%\usepackage{amsthm}
%\usepackage{xy}
%\usepackage{fancyhdr}
%\usepackage{xcolor}
%\usepackage{mathtools} %% for pretty underbrace % Breaks Ximera
%\usepackage{multicol}



\newcommand{\RR}{\mathbb R}
\newcommand{\R}{\mathbb R}
\newcommand{\C}{\mathbb C}
\newcommand{\N}{\mathbb N}
\newcommand{\Z}{\mathbb Z}
\newcommand{\dis}{\displaystyle}
%\renewcommand{\d}{\,d\!}
\renewcommand{\d}{\mathop{}\!d}
\newcommand{\dd}[2][]{\frac{\d #1}{\d #2}}
\newcommand{\pp}[2][]{\frac{\partial #1}{\partial #2}}
\renewcommand{\l}{\ell}
\newcommand{\ddx}{\frac{d}{\d x}}

\newcommand{\zeroOverZero}{\ensuremath{\boldsymbol{\tfrac{0}{0}}}}
\newcommand{\inftyOverInfty}{\ensuremath{\boldsymbol{\tfrac{\infty}{\infty}}}}
\newcommand{\zeroOverInfty}{\ensuremath{\boldsymbol{\tfrac{0}{\infty}}}}
\newcommand{\zeroTimesInfty}{\ensuremath{\small\boldsymbol{0\cdot \infty}}}
\newcommand{\inftyMinusInfty}{\ensuremath{\small\boldsymbol{\infty - \infty}}}
\newcommand{\oneToInfty}{\ensuremath{\boldsymbol{1^\infty}}}
\newcommand{\zeroToZero}{\ensuremath{\boldsymbol{0^0}}}
\newcommand{\inftyToZero}{\ensuremath{\boldsymbol{\infty^0}}}


\newcommand{\numOverZero}{\ensuremath{\boldsymbol{\tfrac{\#}{0}}}}
\newcommand{\dfn}{\textbf}
%\newcommand{\unit}{\,\mathrm}
\newcommand{\unit}{\mathop{}\!\mathrm}
%\newcommand{\eval}[1]{\bigg[ #1 \bigg]}
\newcommand{\eval}[1]{ #1 \bigg|}
\newcommand{\seq}[1]{\left( #1 \right)}
\renewcommand{\epsilon}{\varepsilon}
\renewcommand{\iff}{\Leftrightarrow}

\DeclareMathOperator{\arccot}{arccot}
\DeclareMathOperator{\arcsec}{arcsec}
\DeclareMathOperator{\arccsc}{arccsc}
\DeclareMathOperator{\si}{Si}
\DeclareMathOperator{\proj}{proj}
\DeclareMathOperator{\scal}{scal}
\DeclareMathOperator{\cis}{cis}
\DeclareMathOperator{\Arg}{Arg}
%\DeclareMathOperator{\arg}{arg}
\DeclareMathOperator{\Rep}{Re}
\DeclareMathOperator{\Imp}{Im}
\DeclareMathOperator{\sech}{sech}
\DeclareMathOperator{\csch}{csch}
\DeclareMathOperator{\Log}{Log}

\newcommand{\tightoverset}[2]{% for arrow vec
  \mathop{#2}\limits^{\vbox to -.5ex{\kern-0.75ex\hbox{$#1$}\vss}}}
\newcommand{\arrowvec}{\overrightarrow}
\renewcommand{\vec}{\mathbf}
\newcommand{\veci}{{\boldsymbol{\hat{\imath}}}}
\newcommand{\vecj}{{\boldsymbol{\hat{\jmath}}}}
\newcommand{\veck}{{\boldsymbol{\hat{k}}}}
\newcommand{\vecl}{\boldsymbol{\l}}
\newcommand{\utan}{\vec{\hat{t}}}
\newcommand{\unormal}{\vec{\hat{n}}}
\newcommand{\ubinormal}{\vec{\hat{b}}}

\newcommand{\dotp}{\bullet}
\newcommand{\cross}{\boldsymbol\times}
\newcommand{\grad}{\boldsymbol\nabla}
\newcommand{\divergence}{\grad\dotp}
\newcommand{\curl}{\grad\cross}
%% Simple horiz vectors
\renewcommand{\vector}[1]{\left\langle #1\right\rangle}


\outcome{Integrate rational functions using a partial fraction decomposition}

\title{2.4 Partial Fractions}

\begin{document}

\begin{abstract}
We will integrate rational functions using partial fraction decompositions.
\end{abstract}

\maketitle

\begin{center}
\textbf{Partial Fractions}
\end{center}

In this section we will learn an important method for integrating rational functions.
Recall that a rational function is a ratio of polynomials. The key background skill is the ability to factor polynomials.
We will look at three examples that cover the scenarios that we will encounter in our integrals. 

First, consider the cubic polynomial  $x^3 - x$.  
We can factor out an $x$ and then factor again using the difference of two perfect squares:
\[
x^3 - x = x(x^2-1) = x(x+1)(x-1).
\]

This is a \textbf{complete factorization} consisting of three different linear factors, $x, x-1$ and $x+1$.
We can say that the original cubic polynomial factored into \textbf{distinct linear factors}.

Second, consider the cubic polynomial $x^3 + 2x^2 + x$.  This polynomial factors as $x(x^2 + 2x + 1)$ but this is not a complete factorization. 
The quadratic, $x^2 + 2x + 1$, can be factored since it is a perfect square.  The complete factorization is 
\[
x^3 + 2x^2 + x= x(x^2 + 2x + 1) = x(x+1)(x+1) = x(x+1)^2.
\]
In this case, our original cubic polynomial factored into a linear factor and a \textbf{repeated linear factor}.

Last, consider the cubic polynomial $x^3 + x^2$.  This polynomial factors as $x(x^2 + 1)$.
The quadratic $x^2 + 1$ does not factor (unless you are willing to 
use \link[imaginary numbers]{https://en.wikipedia.org/wiki/Imaginary_number}) and is called 
an \textbf{irreducible quadratic} polynomial.
So we say that the original cubic polynomial factored into a linear factor and an irreducible quadratic factor.

Now we will investigate the three decomposition forms we will encounter in our integrals.

\begin{itemize}

\item{\textbf{Distinct Linear Factors}} \\
A rational function whose denominator factors into a product of distinct linear factors
can be decomposed into a sum of rational functions whose numerators are constants and whose denominators are the factors of the denominator.
For example, 
\[
\frac{x^2 + 3x -5}{x^3 -x} = \frac{x^2 + 3x -5}{x(x-1)(x+1)} = \frac{A}{x} + \frac{B}{x-1} + \frac{C}{x+1},
\]
where $A, B$ and $C$ are constants.




\item{\textbf{Repeated Linear Factor}} \\
The repeated linear factor will contribute a number of terms to the partial fraction 
decomposition equal to the \textbf{multiplicity} of the factor. Each of these terms will have a constant in the numerator and a 
distinct power of the repeated factor in the denominator.
For example, 
\[
\frac{x^2 + 3x -5}{x^3 + 2x^2 + x} = \frac{x^2 + 3x -5}{x(x+1)^2} = \frac{A}{x} + \frac{B}{x+1} + \frac{C}{(x+1)^2},
\]
where $A, B$ and $C$ are constants. Note how the repeated linear factor $(x+1)^2$ contributes two terms whose denominators are $x+1$ and $(x+1)^2$
and whose numerators were constants.


\item{\textbf{Irreducible Quadratic Factor}} \\
An irreducible quadratic factor in the denominator will contribute one term to the decomposition whose numerator
is a linear polynomial and whose denominator is the irreducible quadratic factor.
For example, 
\[
\frac{x^2 + 3x -5}{x^3 +x} = \frac{x^2 + 3x -5}{x(x^2+1)} = \frac{A}{x} + \frac{Bx + C}{x^2 + 1},
\]
where $A, B$ and $C$ are constants.


\end{itemize}

Before we move on to integrating the special rational functions seen in the above decompositions, 
we will look at one more example consisting of all three of the ideas mentioned above:
 
\begin{example}[example 1]
Find the partial fraction decomposition of
\[
  f(x) = \frac{x^2 + 3x -5}{x(x-2)^3(x^2 + 4)}. 
\]
The denominator is already in a completely factored form.  This form consists of a linear factor $x$, 
a repeated linear factor $(x-2)^3$ and an irreducible quadratic $x^2 + 4$. 
The partial fraction decomposition will consist of one term for the factor $x$ and three terms for the factor
$(x-2)^3$.  Each of these terms will have a constant in the numerator.  
The decomposition will also contain one term for the irreducible quadratic, and this term 
will have a linear polynomial in the numerator and the irreducible quadratic in the denominator.
The form of the decomposition is:
\[
  \frac{x^2 + 3x -5}{x(x-2)^3(x^2 + 4)} = \frac{A}{x} + \frac{B}{x-2} + \frac{C}{(x-2)^2} + \frac{D}{(x-2)^3} +\frac{Ex+F}{x^2+4},
\]
where $A, B, C, D, E,$ and $F$ are constants.

\end{example}

\section{Special Integrals}
The point of a partial fraction decomposition is to replace a complicated rational function by a sum of simpler ones which we can integrate.
We will now look at some of these simpler forms and learn how to integrate them.

\begin{example}[example 2]
Compute the indefinite integral
\[
\int \frac{1}{ax+b} \; dx.
\]
Using a $u$-substitution with $u = ax+b$ and $du = a \, dx$,
we get
\[
\int \frac{1}{ax+b} \; dx = \frac{1}{a} \int \frac{1}{u} \; du = \frac{1}{a} \ln|u| + C = \frac{1}{a} \ln|ax+b| + C.
\]
In future examples, we will not perform the substitution.  Instead, we will just write
\[
\int \frac{1}{ax+b} \; dx = \frac{1}{a} \ln|ax+b| + C.
\]
\end{example}

\begin{problem}
Compute the indefinite integral
\[
\int \frac{3}{2x+5} \; dx = \answer{\frac32 \ln|2x+5|} + C.
\]
\end{problem}

\begin{example}
Compute the following indefinite integral, assuming that $n > 1$:
\[
\int \frac{1}{(ax+b)^n} \; dx.
\]
Using a $u$-substitution with $u = ax+b$ and $du = a \, dx$,
we get
\begin{align*}
\int \frac{1}{(ax+b)^n} \; dx &= \frac{1}{a} \int \frac{1}{u^n} \; du \\
                              &= \frac{1}{a} \int u^{-n} \; du \\
                              &= \frac{1}{a} \cdot \frac{u^{-n+1}}{-n+1} + C \\
                              &= -\frac{1}{a} \cdot \frac{1}{(n-1)u^{n-1}} + C\\
                              &= -\frac{1}{a(n-1)u^{n-1}} + C\\
                              &= -\frac{1}{a(n-1)(ax+b)^{n-1}} + C.
\end{align*}
In short, we use the power rule with exponent `$-n$' and we multiply by $1/a$.
\end{example}

\begin{problem}
Compute the indefinite integral
\[
\int \frac{4}{(6x+ 5)^3} \; dx = \answer{-\frac{1}{3(6x+5)^2}} + C.
\]

\end{problem}


\begin{example}
Compute the indefinite integral
\[
\int \frac{1}{a^2 + x^2} \; dx.
\]
Using a substitution with $u = x/a$ and $du = (1/a) \, dx$, so that $x = au$ and $dx = a \, du$, we get

\begin{align*}
 \int \frac{1}{a^2 + x^2} \; dx &= \int \frac{1}{a^2 + (au)^2} \cdot a\, du   \\
 &= a\int \frac{1}{a^2 + a^2u^2} \; du   \\
 &= \frac{a}{a^2} \int \frac{1}{1+u^2} \; du \\
 &= \frac{1}{a} \tan^{-1}(u) + C \\
 &= \frac{1}{a} \tan^{-1}\left(\frac{x}{a}\right) + C.
 \end{align*}
 In summary,
 \[
\int \frac{1}{a^2 + x^2} \; dx =  \frac{1}{a} \tan^{-1}\left(\frac{x}{a}\right) + C.
\]
 \end{example}
 
 \begin{problem}
 Compute the indefinite integral
 \[
 \int \frac{3}{4+x^2} \; dx = \answer{\frac32 \tan^{-1}(x/2)} + C.
\]

 \end{problem}
 
 
\begin{example}
Compute the indefinite integral
\[
\int \frac{x}{a^2 + x^2} \; dx.
\]
Using a substitution with $u = a^2 + x^2$ and $du = 2x \, dx$, we get

 \[
 \int \frac{x}{a^2 + x^2} \; dx = \frac12 \int \frac{1}{u} \; du   =  \frac{1}{2} \ln|u| + C =    
 \frac{1}{2} \ln(a^2 + x^2) + C.
 \]
 In summary,
 \[
\int \frac{x}{a^2 + x^2} \; dx =  \frac{1}{2}\ln(a^2 + x^2)  + C.
\]
 \end{example}
 
 \begin{problem}
 Compute the indefinite integral
 \[
 \int \frac{3x}{9+x^2} \; dx = \answer{\frac32 \ln(9+ x^2)} + C.
\]

 \end{problem}
               
                    
We are now ready to tackle more general scenarios.

\section{Distinct Linear Factors}


\begin{example}

Compute the indefinite integral
\[
\int \frac{2x+3}{x^3 -x} \; dx.
\]
We begin by factoring the denominator and writing the partial fraction decomposition:
\[
\int \frac{2x+3}{x^3 -x} \; dx = \int \frac{2x+3}{x(x-1)(x+1)} \; dx = \int \left( \frac{A}{x} + \frac{B}{x-1} + \frac{C}{x+1} \right) \; dx.
\]
We can now integrate using the results of the special integrals section above:
\[
\int \left( \frac{A}{x} + \frac{B}{x-1} + \frac{C}{x+1} \right) \; dx = A\ln|x| + B\ln|x-1| + C\ln|x+1| + K.
\]
We have used $K$ for the constant of integration since the letter $C$ was already in use.  We will convert back to $C$ in our final answer, below.
Now, we must determine the values of the parameters $A, B, $ and $C$ using algebraic methods.
Going back to the original decomposition, we have
\[
\frac{2x+3}{x(x-1)(x+1)} = \frac{A}{x}  + \frac{B}{x-1} + \frac{C}{x+1}.
\]
Multiplying both sides of this equation by $x(x-1)(x+1)$ we have
\[
2x+3 = A(x-1)(x+1) + Bx(x+1) + Cx(x-1).
\]
Now, we will plug in the values $0, 1, $ and $-1$ for $x$ to determine $A, B, $ and $C$ respectively.
Letting $x = 0$ in the equation above eliminates the $B$ and $C$ terms, yielding
\[
3 = A(-1)(1),
\]
so that $A = -3$.
Next, letting $x = 1$ eliminates the $A$ and $C$ terms, yielding
\[
5 = B(1)(2),
\]
so that $B = 5/2$. Finally, letting $x = -1$ eliminates the $A$ and $B$ terms, yielding
\[
1 = C(-1)(-2),
\]
so that $C = 1/2$.
We can now write our final answer:
\[
\int \frac{2x+3}{x^3 -x} \; dx = A\ln|x| + B\ln|x-1| + C\ln|x+1| + K
\]
\[
 = -3\ln|x| + \frac52 \ln|x-1| + \frac12 \ln|x+1| + C.
\]

\end{example}



\begin{center}
\begin{foldable}
\unfoldable{Here is a detailed, lecture style video on distinct linear factors:}
\youtube{q2AlzXZ3zUA}
\end{foldable}
\end{center}


\section{Repeated Linear Factors}

\begin{example}
Compute the indefinite integral
\[
\int \frac{3x - 4}{x^3 + 2x^2 + x} \; dx.
\]
We begin by factoring the denominator and writing the partial fraction decomposition:
\[
\int \frac{3x - 4}{x^3 + 2x^2 + x} \; dx = \int \frac{3x - 4}{x(x+1)^2} \; dx = \int \left(\frac{A}{x} + \frac{B}{x+1} + \frac{C}{(x+1)^2} \right) \; dx.
\]
We can now integrate using the results of the special integrals section above:
\[
\int \left(\frac{A}{x} + \frac{B}{x+1} + \frac{C}{(x+1)^2} \right) \; dx = A\ln|x| + B\ln|x+1| - \frac{C}{x+1} + K.
\]
Now, we must determine the values of the parameters $A, B, $ and $C$ using algebraic methods.
Going back to the original decomposition, we have
\[
\frac{3x - 4}{x(x+1)^2} = \frac{A}{x} + \frac{B}{x+1} + \frac{C}{(x+1)^2}.
\]
Multiplying both sides of this equation by $x(x+1)^2$ gives
\[
3x-4 = A(x+1)^2 + B(x+1) + Cx.
\]
Letting $x = -1$ will eliminate the $A$ and $B$ terms, allowing us to find $C$:
\[
-7 = C(-1),
\]
which implies that $C = 1/7$.  If we let $x = 0$ the $C$ term gets eliminated and we are left with:
\[
-4 = A + B.
\]
Choosing another $x$-value will leave us with another equation involving $A$ and $B$.  Since $C = 1/7$, choosing $x = 7$ yields
\[
17 = 64A + 8B + 1,
\]
which can be written as 
\[
16 = 64A + 8B.
\]
Dividing this by 8 and then substituting $B = -A - 4$ (from the equation -4 = A + B), we can find $A$:
\[
2 = 8A + B = 8A + (-A-4) = 7A -4,
\]
which means $7A = 6$, or $A = 6/7$.
Now that we know $A$, we can use $-4 = A + B$ to find $B$:
\[
-4 = (6/7) + B,
\]
so that $B = -4 - (6/7) = -34/7$. The final answer is thus:
\[
\int \frac{3x - 4}{x^3 + 2x^2 + x} \; dx = A\ln|x| + B\ln|x+1| - \frac{C}{x+1} + K.
\]
\[
= \frac67\ln|x| -\frac{34}{7}\ln|x+1| - \frac{1}{7(x+1)} + C.
\]


\end{example}

\begin{problem}

Compute $\displaystyle{\int \frac{2x^2 + 1}{(x-1)^2(x+2)}\, dx}$\\

The partial fraction decomposition has the form

\begin{multipleChoice}
\choice[correct]{$\displaystyle{\frac{A}{x-1} + \frac{B}{(x-1)^2} + \frac{C}{x+2}}$}
\choice{$\displaystyle{\frac{A}{x-1} + \frac{Bx}{(x-1)^2} + \frac{C}{x+2}}$}
\choice{$\displaystyle{\frac{A}{(x-1)^2} + \frac{B}{x+2}}$}
\end{multipleChoice}

The answer has the form

$\displaystyle{\int \frac{1}{(x-1)^2(x+2)} \, dx =}$
\begin{multipleChoice}
\choice[correct]{$\displaystyle{A\ln|x-1| - \frac{B}{x-1} + C\ln|x+2|+ K}$}
\choice[correct]{$\displaystyle{A\ln|x-1| + \frac{B}{x-1} + C\ln|x+2|+K}$}
\choice[correct]{$\displaystyle{A\ln|x-1| + B\ln|(x-1)^2| + C\ln|x+2|+K}$}
\end{multipleChoice}

The coefficients are\\
$A = \answer{1} \;\;\; B = \answer{1} \;\;\; C = \answer{1}$
\end{problem}




\begin{center}
\begin{foldable}
\unfoldable{Here is a detailed, lecture style video on repeated linear factors:}
\youtube{w8fw4l1RN64}
\end{foldable}
\end{center}

\section{Irreducible Quadratic Factors}

\begin{example}
Compute the indefinite integral:
\[
\int \frac{x^2 + 5x + 2}{x^3 + 4x} \; dx.
\]

We begin by factoring the denominator and writing the partial fraction decomposition:
\[
\int \frac{x^2 + 5x + 2}{x^3 + 4x} \; dx = \int \frac{x^2 + 5x + 2}{x(x^2 + 4)} \; dx = \int \left(\frac{A}{x} + \frac{Bx+C}{x^2+4} \right) \; dx.
\]
We can now integrate using the results of the special integrals section above:
\[
\int \left(\frac{A}{x} + \frac{Bx}{x^2+4} + \frac{C}{x^2+4} \right) \; dx = A\ln|x| + \frac{B}{2}\ln(x^2 +4 ) + \frac{C}{2}\tan^{-1}\left(\frac{x}{2}\right) + K.
\]
Now, we must determine the values of the parameters $A, B, $ and $C$ using algebraic methods.
Going back to the original decomposition, we have
\[
\frac{x^2 + 5x + 2}{x(x^2 + 4)} = \frac{A}{x} + \frac{Bx+C}{x^2+4}.
\]
Multiplying both sides of this equation by $x(x^2 +4)$ gives
\[
x^2 + 5x + 2 = A(x^2 +4)+ (Bx + C)x.
\]
Rather than plugging in certain values for $x$, we can find $A, B$ and $C$ by another method, called ``equating coefficients."
We will rewrite the right hand side in standard polynomial form:
\[
x^2 + 5x + 2 = Ax^2 + 4A + Bx^2 + Cx = (A+B)x^2 + Cx + 4A.
\]
Since these polynomials are equal, their coefficients must match.  Hence
\[
A+B = 1,\; C = 5, \text { and } \; 4A = 2.
\]
The third of these equations gives $A = 1/2$, and using this in the first equation gives $B = 1/2$ also.
The final answer is thus:
\[
\int \frac{x^2 + 5x + 2}{x^3 + 4x} \; dx = A\ln|x| + \frac{B}{2}\ln(x^2 +4 ) + \frac{C}{2}\tan^{-1}\left(\frac{x}{2}\right) + K
\]
\[
= \frac12\ln|x| + \frac{1}{4}\ln(x^2 +4 ) + \frac{5}{2}\tan^{-1}\left(\frac{x}{2}\right) + C.
\]




\end{example}



\begin{center}
\begin{foldable}
\unfoldable{Here is a detailed, lecture style video on irreducible quadratic factors:}
\youtube{abuPElILOZw&t}
\end{foldable}
\end{center}


\end{document}


\begin{center}
\begin{foldable}
\unfoldable{Here is a video of Example 1}
%\youtube{Yy6QXnFlnXs} %vid of example 1
\end{foldable}
\end{center}


\begin{example} %example #15
Find $h'(x)$ if $h(x) = x^{\sin(x)}$.\\
We will use the fact that the exponential and logarithm functions are inverses,
\[e^{\ln(x)} = x,\]
and the exponent property of logarithms, 
\[\ln(x^n) = n\ln(x),\]
to rewrite $h(x)$.  We have 
\[h(x) = x^{\sin(x)} = e^{\ln(x^{\sin(x)})} = e^{\sin(x)\ln(x)}\]
and we can now compute $h'(x)$ using a combination of the chain rule and product rule.
We can write $h(x)$ as a composition, $f(g(x))$ with 
\[g(x) = \sin(x)\ln(x) \quad \text{and} \quad f(x) = e^x.\]
Then to find $g'(x)$ we us the product rule and we get $g'(x) = \frac{\sin(x)}{x} + \cos(x)\ln(x)$.
Next $f'(x) = e^x$ and 
hence $f'(g(x)) = e^{g(x)} = e^{\sin(x)\ln(x)} = x^{\sin(x)}$.
We can then conclude $h'(x) = f'(g(x))g'(x) = x^{\sin(x)} \left[ \frac{\sin(x)}{x} + \cos(x)\ln(x)\right]$.
\end{example}

%more question formats below













%\begin{verbatim}
\begin{question}
What is your favorite color?
\begin{multipleChoice}
\choice[correct]{Rainbow}
\choice{Blue}
\choice{Green}
\choice{Red}
\end{multipleChoice}
\begin{freeResponse}
Hello
\end{freeResponse}
\end{question}
%\end{verbatim}





\begin{question}
  Which one will you choose?
  \begin{multipleChoice}
    \choice[correct]{I'm correct.}
    \choice{I'm wrong.}
    \choice{I'm wrong too.}
  \end{multipleChoice}
\end{question}


\begin{question}
  Which one will you choose?
  \begin{selectAll}
    \choice[correct]{I'm correct.}
    \choice{I'm wrong.}
    \choice[correct]{I'm also correct.}
    \choice{I'm wrong too.}
  \end{selectAll}
\end{question}


\begin{freeResponse}
What is the chain rule used for?
\end{freeResponse}
