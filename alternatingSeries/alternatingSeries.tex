\documentclass{ximera}

%% You can put user macros here
%% However, you cannot make new environments



\newcommand{\ffrac}[2]{\frac{\text{\footnotesize $#1$}}{\text{\footnotesize $#2$}}}
\newcommand{\vasymptote}[2][]{
    \draw [densely dashed,#1] ({rel axis cs:0,0} -| {axis cs:#2,0}) -- ({rel axis cs:0,1} -| {axis cs:#2,0});
}


\graphicspath{{./}{firstExample/}}

\usepackage{amsmath}
\usepackage{amssymb}
\usepackage{array}
\usepackage[makeroom]{cancel} %% for strike outs
\usepackage{pgffor} %% required for integral for loops
\usepackage{tikz}
\usepackage{tikz-cd}
\usepackage{tkz-euclide}
\usetikzlibrary{shapes.multipart}


\usetkzobj{all}
\tikzstyle geometryDiagrams=[ultra thick,color=blue!50!black]


\usetikzlibrary{arrows}
\tikzset{>=stealth,commutative diagrams/.cd,
  arrow style=tikz,diagrams={>=stealth}} %% cool arrow head
\tikzset{shorten <>/.style={ shorten >=#1, shorten <=#1 } } %% allows shorter vectors

\usetikzlibrary{backgrounds} %% for boxes around graphs
\usetikzlibrary{shapes,positioning}  %% Clouds and stars
\usetikzlibrary{matrix} %% for matrix
\usepgfplotslibrary{polar} %% for polar plots
\usepgfplotslibrary{fillbetween} %% to shade area between curves in TikZ



%\usepackage[width=4.375in, height=7.0in, top=1.0in, papersize={5.5in,8.5in}]{geometry}
%\usepackage[pdftex]{graphicx}
%\usepackage{tipa}
%\usepackage{txfonts}
%\usepackage{textcomp}
%\usepackage{amsthm}
%\usepackage{xy}
%\usepackage{fancyhdr}
%\usepackage{xcolor}
%\usepackage{mathtools} %% for pretty underbrace % Breaks Ximera
%\usepackage{multicol}



\newcommand{\RR}{\mathbb R}
\newcommand{\R}{\mathbb R}
\newcommand{\C}{\mathbb C}
\newcommand{\N}{\mathbb N}
\newcommand{\Z}{\mathbb Z}
\newcommand{\dis}{\displaystyle}
%\renewcommand{\d}{\,d\!}
\renewcommand{\d}{\mathop{}\!d}
\newcommand{\dd}[2][]{\frac{\d #1}{\d #2}}
\newcommand{\pp}[2][]{\frac{\partial #1}{\partial #2}}
\renewcommand{\l}{\ell}
\newcommand{\ddx}{\frac{d}{\d x}}

\newcommand{\zeroOverZero}{\ensuremath{\boldsymbol{\tfrac{0}{0}}}}
\newcommand{\inftyOverInfty}{\ensuremath{\boldsymbol{\tfrac{\infty}{\infty}}}}
\newcommand{\zeroOverInfty}{\ensuremath{\boldsymbol{\tfrac{0}{\infty}}}}
\newcommand{\zeroTimesInfty}{\ensuremath{\small\boldsymbol{0\cdot \infty}}}
\newcommand{\inftyMinusInfty}{\ensuremath{\small\boldsymbol{\infty - \infty}}}
\newcommand{\oneToInfty}{\ensuremath{\boldsymbol{1^\infty}}}
\newcommand{\zeroToZero}{\ensuremath{\boldsymbol{0^0}}}
\newcommand{\inftyToZero}{\ensuremath{\boldsymbol{\infty^0}}}


\newcommand{\numOverZero}{\ensuremath{\boldsymbol{\tfrac{\#}{0}}}}
\newcommand{\dfn}{\textbf}
%\newcommand{\unit}{\,\mathrm}
\newcommand{\unit}{\mathop{}\!\mathrm}
%\newcommand{\eval}[1]{\bigg[ #1 \bigg]}
\newcommand{\eval}[1]{ #1 \bigg|}
\newcommand{\seq}[1]{\left( #1 \right)}
\renewcommand{\epsilon}{\varepsilon}
\renewcommand{\iff}{\Leftrightarrow}

\DeclareMathOperator{\arccot}{arccot}
\DeclareMathOperator{\arcsec}{arcsec}
\DeclareMathOperator{\arccsc}{arccsc}
\DeclareMathOperator{\si}{Si}
\DeclareMathOperator{\proj}{proj}
\DeclareMathOperator{\scal}{scal}
\DeclareMathOperator{\cis}{cis}
\DeclareMathOperator{\Arg}{Arg}
%\DeclareMathOperator{\arg}{arg}
\DeclareMathOperator{\Rep}{Re}
\DeclareMathOperator{\Imp}{Im}
\DeclareMathOperator{\sech}{sech}
\DeclareMathOperator{\csch}{csch}
\DeclareMathOperator{\Log}{Log}

\newcommand{\tightoverset}[2]{% for arrow vec
  \mathop{#2}\limits^{\vbox to -.5ex{\kern-0.75ex\hbox{$#1$}\vss}}}
\newcommand{\arrowvec}{\overrightarrow}
\renewcommand{\vec}{\mathbf}
\newcommand{\veci}{{\boldsymbol{\hat{\imath}}}}
\newcommand{\vecj}{{\boldsymbol{\hat{\jmath}}}}
\newcommand{\veck}{{\boldsymbol{\hat{k}}}}
\newcommand{\vecl}{\boldsymbol{\l}}
\newcommand{\utan}{\vec{\hat{t}}}
\newcommand{\unormal}{\vec{\hat{n}}}
\newcommand{\ubinormal}{\vec{\hat{b}}}

\newcommand{\dotp}{\bullet}
\newcommand{\cross}{\boldsymbol\times}
\newcommand{\grad}{\boldsymbol\nabla}
\newcommand{\divergence}{\grad\dotp}
\newcommand{\curl}{\grad\cross}
%% Simple horiz vectors
\renewcommand{\vector}[1]{\left\langle #1\right\rangle}


\outcome{Use the Alternating Series Test}

\title{3.7 Alternating Series}

\begin{document}

\begin{abstract}
We use the Alternating Series Test to determine convergence of infinite series.
\end{abstract}

\maketitle

\section{The Alternating Series Test}

An alternating series is an infinite series whose terms alternate signs.
A typical \textbf{alternating series} has the form 
\[
\sum_{n=1}^\infty (-1)^n a_n,
\]
where $a_n > 0$ for all $n$. We will refer to the factor $(-1)^n$ as the \textbf{alternating symbol}.

Some examples of alternating series are
\begin{itemize}

\item $\displaystyle{\sum_{n=1}^\infty (-1)^{n+1} \frac{1}{n} = 1 - \frac12 + \frac13 - \frac14 + \cdots  }\, .$  This is the \textbf{Alternating Harmonic Series}.

\item $\displaystyle{\sum_{n=0}^\infty \left(-\frac12\right)^n = 1 - \frac12 + \frac14 - \frac18 + \cdots }\, .$  This is an alternating geometric series with $r=-1/2$.
Since $-1 < -\frac12 < 1$, we know from our earlier study of geometric series that this alternating series converges and 
the sum is 
\[
S = \frac{a}{1-r} = \frac{1}{1-(-1/2)} = \frac23.
\]

\item $\displaystyle{\sum_{n=0}^\infty \frac{\cos(n\pi)}{(n+1)^2} = 1 - \frac14 + \frac19 - \frac{1}{16} + \cdots}\, .$ 
In this example, $\cos(n\pi)$ could be replaced by the standard alternating symbol $(-1)^n$.


\item $\displaystyle{\sum_{n=1}^\infty (-1)^{n+1} \frac{1}{n^p} = 1 - \frac{1}{2^p} + \frac{1}{3^p} - \frac{1}{4^p} + \cdots  }\, .$  This is an \textbf{Alternating $p$-Series}.


\end{itemize}

\begin{theorem} Alternating Series Test (AST)\\
The alternating series, 
\[
\sum_{n=1}^\infty (-1)^{n+1} a_n = a_1 - a_2 + a_3 - a_4 + \cdots
\]
\textbf{converges} if the following two conditions are met:
\begin{itemize}
\item[1.] $\displaystyle{\lim_{n\to\infty} a_n = 0}$,
\item[2.] the terms, $a_n$, are decreasing, i.e., $a_{n+1} \leq a_n$ for all $n$.
\end{itemize}
\end{theorem}


\begin{remark} If the first condition does not hold, i.e., if 
$\displaystyle{\lim_{n\to\infty} a_n \neq 0}$, 
then the alternating series diverges (by the Test for Divergence).
\end{remark}

\begin{remark}
The second condition can be weakened to say the terms, $a_n$, are eventually decreasing, i.e., decreasing from some term onward.
\end{remark}

\begin{remark} 
If the first condition is satisfied but the second condition is not, then the AST gives \textbf{no conclusion} about the alternating series.
\end{remark}

\begin{example}[example 1] The alternating harmonic series, 
$\displaystyle{\sum_{n=1}^\infty (-1)^{n+1} \frac{1}{n}}$ converges by the AST. 
To see this, first, 
\[
\lim_{n\to\infty} \frac{1}{n} = 0,
\]
and second,
the terms are decreasing since 
\[
\frac{1}{n+1} < \frac{1}{n}
\]
for all $n \geq 1$. Since both conditions are satisfied, the alternating harmonic series converges by the AST.
Moreover, it is known that the alternating harmonic series converges to the value $\ln(2)$, as we will see in the section on Power Series.
\end{example}

\begin{problem}(problem 1a)
Consider the series $\displaystyle{\sum_{n=1}^\infty (-1)^{n+1} \frac{1}{\sqrt n}}$.\\
First, find the limit:
\[
\lim_{n\to\infty} \frac{1}{\sqrt n} = \answer{0}.
\]
Which of the following is true?
\begin{multipleChoice}
\choice[correct]{$\displaystyle{\frac{1}{\sqrt{n+1}} < \frac{1}{\sqrt n}}$ for $n \geq 1$}
\choice{$\displaystyle{\frac{1}{\sqrt{n+1}} > \frac{1}{\sqrt n}}$ for $n \geq 1$}
\choice{neither}
\end{multipleChoice}


Describe the behavior of the series $\displaystyle{\sum_{n=1}^\infty (-1)^{n+1} \frac{1}{\sqrt n}:}$
\begin{multipleChoice}
\choice[correct]{Converges by AST}
\choice{Diverges by Test for Divergence}
\choice{No Conclusion from AST}
\end{multipleChoice}

\end{problem}



\begin{problem}(problem 1b)
Consider the series $\displaystyle{\sum_{n=1}^\infty (-1)^{n+1} \frac{1}{n^2}}$.\\
First, find the limit:
\[
\lim_{n\to\infty} \frac{1}{n^2} = \answer{0}.
\]
Which of the following is true?
\begin{multipleChoice}
\choice[correct]{$\displaystyle{\frac{1}{(n+1)^2} < \frac{1}{n^2}}$ for $n \geq 1$}
\choice{$\displaystyle{\frac{1}{(n+1)^2} > \frac{1}{n^2}}$ for $n \geq 1$}
\choice{neither}
\end{multipleChoice}


Describe the behavior of the series $\displaystyle{\sum_{n=1}^\infty (-1)^{n+1} \frac{1}{n^2}:}$
\begin{multipleChoice}
\choice[correct]{Converges by AST}
\choice{Diverges by Test for Divergence}
\choice{No Conclusion from AST}
\end{multipleChoice}

\end{problem}



\begin{problem}(problem 1c)
Consider the series $\displaystyle{\sum_{n=1}^\infty (-1)^{n+1} \frac{n}{n+1}}$.\\
First, find the limit:
\[
\lim_{n\to\infty} \frac{n}{n+1} = \answer{1}.
\]
Which of the following is true?
\begin{multipleChoice}
\choice{$\displaystyle{\frac{n+1}{n+2} < \frac{n}{n+1}}$ for $n \geq 1$}
\choice[correct]{$\displaystyle{\frac{n+1}{n+2} > \frac{n}{n+1}}$ for $n \geq 1$}
\choice{neither}
\end{multipleChoice}


Describe the behavior of the series $\displaystyle{\sum_{n=1}^\infty (-1)^{n+1} \frac{n}{n+1}:}$
\begin{multipleChoice}
\choice{Converges by AST}
\choice[correct]{Diverges by Test for Divergence}
\choice{No Conclusion from AST}
\end{multipleChoice}

\end{problem}


\begin{problem}(problem 1d)
Consider the series $\displaystyle{\sum_{n=2}^\infty (-1)^n \frac{1}{\ln n}}$.\\
First, find the limit:
\[
\lim_{n\to\infty} \frac{1}{\ln n} = \answer{0}.
\]
Which of the following is true?
\begin{multipleChoice}
\choice[correct]{$\displaystyle{\frac{1}{\ln(n+1)} < \frac{1}{\ln n}}$ for $n \geq 2$}
\choice{$\displaystyle{\frac{1}{\ln(n+1)} > \frac{1}{\ln n}}$ for $n \geq 2$}
\choice{neither}
\end{multipleChoice}


Describe the behavior of the series $\displaystyle{\sum_{n=2}^\infty (-1)^n \frac{1}{\ln n}:}$
\begin{multipleChoice}
\choice[correct]{Converges by AST}
\choice{Diverges by Test for Divergence}
\choice{No Conclusion from AST}
\end{multipleChoice}

\end{problem}



\begin{example}[example 2]
For $p > 0$, the alternating $p$-series, 
$\displaystyle{\sum_{n=1}^\infty (-1)^{n+1} \frac{1}{n^p}}$ converges by the AST. 
To see this, first, 
\[
\lim_{n\to\infty} \frac{1}{n^p} = 0,
\]
and second,
the terms are decreasing since 
\[
\frac{1}{(n+1)^p} < \frac{1}{n^p}
\]
for all $n \geq 1$. Since both conditions are satisfied, the alternating $p$-series converges by the AST.
\end{example}

\section{Absolute Versus Conditional Convergence}
In general, the presence of the alternating symbol, $(-1)^n$, helps a series to converge. The alternating harmonic series and its non-alternating counterpart, the harmonic series, provide the quintessential example of this.
The harmonic series diverges, but with the addition of the alternating symbol, the alternating harmonic series converges. 
This leads us to investigate alternating series further by asking the question ``does a convergent alternating series require the alternating symbol to converge?''
If an alternating series converges without the alternating symbol, then we say that it is \textbf{absolutely convergent}.  On the other hand, if a convergent alternating series
diverges without the alternating symbol (like the harmonic series), then we say that the alternating series is \textbf{conditionally convergent}.
We formalize these terms in the definition below.

\begin{definition}[Absolute vs Conditional Convergence]
Consider the alternating series
\[
S = \sum_{n=1}^\infty (-1)^{n+1} a_n,
\]
where $a_n > 0$ for all $n$.
\begin{itemize}
\item If the non-alternating counterpart, $\displaystyle{\sum_{n=1}^\infty a_n}$, converges, then we say that the original alternating series, $S$,
\textbf{converges absolutely}.
\item If the non-alternating counterpart, $\displaystyle{\sum_{n=1}^\infty  a_n}$, diverges, then we say that the original alternating series, $S$,  
\textbf{converges conditionally}.
\end{itemize}

\end{definition}

\begin{example}[example 3]
Consider the alternating $p$-series
\[
\sum_{n=1}^\infty (-1)^{n+1} \frac{1}{n^p},
\]
where $p > 0$. In our study of non-alternating $p$-series in the section on the Integral Test, we discovered that 
the standard $p$-series, $\displaystyle{\sum_{n=1}^\infty \frac{1}{n^p}}$. converges if $p>1$ and diverges if $0 < p \leq 1$.
In the above discussion, we noted that alternating $p$-series converge for all $p>0$. 
Hence, in the terminology of absolute and conditional convergence, we can say that 
if $p>1$, the alternating $p$-series converges absolutely and if $0<p\leq 1$, the alternating $p$-series converges conditionally.
\end{example}



\begin{example}[example 4]
Determine whether the alternating series
\[
\sum_{n=0}^\infty (-1)^n \frac{n^2 + 2}{n^3 + 3}.
\]
converges absolutely, converges conditionally or diverges.\\
First, we consider the non-alternating counterpart of the given series:
\[
\sum_{n=0}^\infty \frac{n^2 + 2}{n^3 + 3}.
\]
This series diverges by the Limit Comparison Test with the divergent Harmonic Series since,

\[
\lim_{n\to\infty} \frac{\left(\frac{n^2 + 2}{n^3 + 3} \right)}{\left(\frac{1}{n} \right)} = \lim_{n\to\infty} \frac{n^3 + 2n}{n^3 + 3} = 1,
\]
by the ratio of the leading coefficients. Since $0 < 1< \infty$, the non-alternating counterpart also diverges by the LCT.
Next, we check for Conditional Convergence using the Alternating Series Test.
The first condition is satisfied since
\[
\lim_{n\to\infty} \frac{n^2 + 2}{n^3 + 3} = 0.
\]
For the second condition, we must determine if the terms are (eventually) decreasing.  To do this, we will compute the derivative (using the quotient rule) and check to see if it is negative:
\begin{align*}
\frac{d}{dn} \left(\frac{n^2 + 2}{n^3 + 3} \right) &= \frac{(n^3 + 3)(2n) - (n^2 + 2)(3n^2)}{(n^3 + 3)^2}\\
                                                            &= \frac{(2n^4 + 6n)-(3n^4+6n^2)}{(n^3 + 3)^2}\\
                                                            &= \frac{6n - 6n^2 - n^4}{(n^3 + 3)^2}.
\end{align*}
The derivative is negative for $n>1$ since the numerator is negative and the denominator is positive when $n>1$.
Hence, the terms of the series are eventually decreasing, and the second condition is also satisfied. Finally, since both conditions of the AST are satisfied,
we can conclude that the original alternating series converges conditionally.

\end{example}




\begin{problem}(problem 4)
Consider the series $\displaystyle{\sum_{n=1}^\infty (-1)^{n+1} \frac{n^3}{n^4 + 2}}$.\\
Does its non-alternating counterpart converge?
\begin{hint} 
Consider the series without the alternating symbol.
\end{hint}
\begin{multipleChoice}
\choice{Yes}
\choice[correct]{No}
\end{multipleChoice}

Why?
\begin{multipleChoice}
\choice[correct]{Limit Comparison Test with the Harmonic Series}
\choice{$p$-series with $0 <  p \leq 1$}
\choice{geometric series with $|r| \geq 1$}
\end{multipleChoice}




Does the original alternating series converge absolutely?
\begin{multipleChoice}
\choice{Yes}
\choice[correct]{No}
\end{multipleChoice}

We now use the AST to check for conditional convergence.

First, find the limit:
\[
\lim_{n\to\infty} \frac{n^3}{n^4 + 2} = \answer{0}.
\]

The derivative of $\displaystyle{\frac{n^3}{n^4 + 2}}$ is  

\begin{multipleChoice}
\choice[correct]{$\displaystyle{\frac{6n^2 - n^6 }{(n^4 + 2)^2}}$}
\choice{$\displaystyle{\frac{n^ 6 - 6n^2}{(n^4 + 2)^2}}$}
\end{multipleChoice}


The derivative is negative for $n > \answer{\sqrt[4]6}$.
\begin{hint}
Rewrite the numerator as $n^2(6-n^4)$
\end{hint}
\begin{hint}
Solve $6 - n^4 = 0$
\end{hint}

Are the terms eventually decreasing?
\begin{multipleChoice}
\choice[correct]{Yes}
\choice{No}
\end{multipleChoice}


Does the original alternating series converge conditionally?
\begin{multipleChoice}
\choice[correct]{Yes}
\choice{No}
\end{multipleChoice}


\end{problem}




\section{Video Lesson}




\begin{center}
\begin{foldable}
\unfoldable{Here is a detailed, lecture style video on the AST:}
\youtube{xeddzMP7FuM}
\end{foldable}
\end{center}


\section{More Problems}



\begin{problem}(problem 5)
Consider the series $\displaystyle{\sum_{n=1}^\infty (-1)^{n+1} \frac{1}{n\sqrt n}}$.\\
Does its non-alternating counterpart converge?
\begin{hint} 
Consider the series without the alternating symbol.
\end{hint}
\begin{multipleChoice}
\choice[correct]{Yes}
\choice{No}
\end{multipleChoice}

Why?
\begin{multipleChoice}
\choice{$p$-series with $0< p \leq 1$}
\choice[correct]{geometric series with $|r| \geq 1$}
\end{multipleChoice}


Does the original alternating series converge absolutely?
\begin{multipleChoice}
\choice[correct]{Yes}
\choice{No}
\end{multipleChoice}

\end{problem}




\begin{problem}(problem 6)
Consider the series $\displaystyle{\sum_{n=1}^\infty (-1)^{n+1} \frac{1}{\sqrt[3] n}}$.\\
Does its non-alternating counterpart converge?
\begin{hint} 
Consider the series without the alternating symbol.
\end{hint}
\begin{multipleChoice}
\choice{Yes}
\choice[correct]{No}
\end{multipleChoice}

Why?
\begin{multipleChoice}
\choice[correct]{$p$-series with $0< p \leq 1$}
\choice{$p$-series with $ p > 1$}
\end{multipleChoice}




Does the original alternating series converge absolutely?
\begin{multipleChoice}
\choice{Yes}
\choice[correct]{No}
\end{multipleChoice}

We now use the AST to check for conditional convergence.

First, find the limit:
\[
\lim_{n\to\infty} \frac{1}{\sqrt[3] n} = \answer{0}.
\]
Which of the following is true?
\begin{multipleChoice}
\choice[correct]{$\displaystyle{\frac{1}{\sqrt[3]{n+1}} < \frac{1}{\sqrt[3] n}}$ for $n \geq 1$}
\choice{$\displaystyle{\frac{1}{\sqrt[3]{n+1}} > \frac{1}{\sqrt[3] n}}$ for $n \geq 1$}
\choice{neither}
\end{multipleChoice}


Does the series $\displaystyle{\sum_{n=1}^\infty (-1)^{n+1} \frac{1}{\sqrt[3] n}}$ converge conditionally?
\begin{multipleChoice}
\choice[correct]{Yes}
\choice{No}
\end{multipleChoice}

\end{problem}




\begin{problem}(problem 7)
Consider the series $\displaystyle{\sum_{n=1}^\infty (-1)^{n+1} \frac{n}{n^2 + 1}}$.\\
Does its non-alternating counterpart converge?
\begin{hint} 
Consider the series without the alternating symbol.
\end{hint}
\begin{multipleChoice}
\choice{Yes}
\choice[correct]{No}
\end{multipleChoice}

Why?
\begin{multipleChoice}
\choice[correct]{Limit Comparison Test with the Harmonic Series}
\choice{$p$-series with $0 <  p \leq 1$}
\choice{geometric series with $|r| \geq 1$}
\end{multipleChoice}




Does the original alternating series converge absolutely?
\begin{multipleChoice}
\choice{Yes}
\choice[correct]{No}
\end{multipleChoice}

We now use the AST to check for conditional convergence.

First, find the limit:
\[
\lim_{n\to\infty} \frac{n}{n^2 + 1} = \answer{0}.
\]
Which of the following is true? 
\begin{hint}
Use the derivative to check for decreasing
\end{hint}
\begin{multipleChoice}
\choice[correct]{$\displaystyle{\frac{n+1}{(n+1)^2 + 1} < \frac{n}{n^2 + 1}}$ for $n \geq 1$}
\choice{$\displaystyle{\frac{n+1}{(n+1)^2 + 1} > \frac{n}{n^2 + 1}}$ for $n \geq 1$}
\choice{neither}
\end{multipleChoice}


Does the series $\displaystyle{\sum_{n=1}^\infty (-1)^{n+1} \frac{n}{n^2 + 1}}$ converge conditionally?
\begin{multipleChoice}
\choice[correct]{Yes}
\choice{No}
\end{multipleChoice}

\end{problem}



\begin{problem}(problem 8)
Consider the series $\displaystyle{\sum_{n=1}^\infty (-1)^{n+1} \frac{n}{n^3 + 1}}$.\\
Does its non-alternating counterpart converge?
\begin{hint} 
Consider the series without the alternating symbol.
\end{hint}
\begin{multipleChoice}
\choice[correct]{Yes}
\choice{No}
\end{multipleChoice}

Why?
\begin{multipleChoice}
\choice[correct]{Direct Comparison Test with $p$-series}
\choice{$p$-series with $p > 1$}
\choice{geometric series with $|r| < 1$}
\end{multipleChoice}



Does the original alternating series converge absolutely?
\begin{multipleChoice}
\choice[correct]{Yes}
\choice{No}
\end{multipleChoice}

\end{problem}




\begin{problem}(problem 9)
Consider the series $\displaystyle{\sum_{n=1}^\infty (-1)^{n+1} \frac{\sqrt n}{n + 1}}$.\\
Does its non-alternating counterpart converge?
\begin{hint} 
Consider the series without the alternating symbol.
\end{hint}
\begin{multipleChoice}
\choice{Yes}
\choice[correct]{No}
\end{multipleChoice}

Why?
\begin{multipleChoice}
\choice[correct]{Limit Comparison Test with $p$-series}
\choice{$p$-series with $0 < p \leq 1$}
\choice{geometric series with $|r| \geq 1$}
\end{multipleChoice}




Does the original alternating series converge absolutely?
\begin{multipleChoice}
\choice{Yes}
\choice[correct]{No}
\end{multipleChoice}

We now use the AST to check for conditional convergence.

First, find the limit:
\[
\lim_{n\to\infty} \frac{\sqrt n}{n + 1} = \answer{0}.
\]
Which of the following is true? 
\begin{hint}
Use the derivative to check for decreasing
\end{hint}
\begin{multipleChoice}
\choice[correct]{$\displaystyle{\frac{\sqrt{n+1}}{n+2} < \frac{\sqrt n}{n + 1}}$ for $n \geq 1$}
\choice{$\displaystyle{\frac{\sqrt{n+1}}{n+2} > \frac{\sqrt n}{n + 1}}$ for $n \geq 1$}
\choice{neither}
\end{multipleChoice}


Does the series $\displaystyle{\sum_{n=1}^\infty (-1)^{n+1} \frac{\sqrt n}{n + 1}}$ converge conditionally?
\begin{multipleChoice}
\choice[correct]{Yes}
\choice{No}
\end{multipleChoice}

\end{problem}






\begin{problem}(problem 10)
Consider the series $\displaystyle{\sum_{n=0}^\infty (-1)^n \frac{n^2 + 1}{n^2 + n + 1}}$.\\
Does its non-alternating counterpart converge?
\begin{hint} 
Consider the series without the alternating symbol.
\end{hint}
\begin{multipleChoice}
\choice{Yes}
\choice[correct]{No}
\end{multipleChoice}

Why?
\begin{multipleChoice}
\choice{Limit Comparison Test with $p$-series}
\choice{$p$-series with $0 < p \leq 1$}
\choice[correct]{Test for Divergence}
\end{multipleChoice}




Does the original alternating series converge absolutely?
\begin{multipleChoice}
\choice{Yes}
\choice[correct]{No}
\end{multipleChoice}



Find the limit:
\[
\lim_{n\to\infty} \frac{n^2 + 1}{n^2 + n + 1} = \answer{1}.
\]

Do we need to check the second condition of the AST?
\begin{multipleChoice}
\choice{Yes}
\choice[correct]{No}
\end{multipleChoice}


What can we say about the series $\displaystyle{\sum_{n=0}^\infty (-1)^n \frac{n^2 + 1}{n^2 + n + 1}}$?
\begin{multipleChoice}
\choice{Converges Conditionally}
\choice[correct]{Diverges}
\choice{No Conclusion}
\end{multipleChoice}

\end{problem}





\end{document}


























\begin{example} %example #15
Find $h'(x)$ if $h(x) = x^{\sin(x)}$.\\
We will use the fact that the exponential and logarithm functions are inverses,
\[e^{\ln(x)} = x,\]
and the exponent property of logarithms, 
\[\ln(x^n) = n\ln(x),\]
to rewrite $h(x)$.  We have 
\[h(x) = x^{\sin(x)} = e^{\ln(x^{\sin(x)})} = e^{\sin(x)\ln(x)}\]
and we can now compute $h'(x)$ using a combination of the chain rule and product rule.
We can write $h(x)$ as a composition, $f(g(x))$ with 
\[g(x) = \sin(x)\ln(x) \quad \text{and} \quad f(x) = e^x.\]
Then to find $g'(x)$ we us the product rule and we get $g'(x) = \frac{\sin(x)}{x} + \cos(x)\ln(x)$.
Next $f'(x) = e^x$ and 
hence $f'(g(x)) = e^{g(x)} = e^{\sin(x)\ln(x)} = x^{\sin(x)}$.
We can then conclude $h'(x) = f'(g(x))g'(x) = x^{\sin(x)} \left[ \frac{\sin(x)}{x} + \cos(x)\ln(x)\right]$.
\end{example}

%more question formats below













%\begin{verbatim}
\begin{question}
What is your favorite color?
\begin{multipleChoice}
\choice[correct]{Rainbow}
\choice{Blue}
\choice{Green}
\choice{Red}
\end{multipleChoice}
\begin{freeResponse}
Hello
\end{freeResponse}
\end{question}
%\end{verbatim}





\begin{question}
  Which one will you choose?
  \begin{multipleChoice}
    \choice[correct]{I'm correct.}
    \choice{I'm wrong.}
    \choice{I'm wrong too.}
  \end{multipleChoice}
\end{question}


\begin{question}
  Which one will you choose?
  \begin{selectAll}
    \choice[correct]{I'm correct.}
    \choice{I'm wrong.}
    \choice[correct]{I'm also correct.}
    \choice{I'm wrong too.}
  \end{selectAll}
\end{question}


\begin{freeResponse}
What is the chain rule used for?
\end{freeResponse}
