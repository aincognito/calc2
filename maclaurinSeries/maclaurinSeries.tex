\documentclass{ximera}

%% You can put user macros here
%% However, you cannot make new environments



\newcommand{\ffrac}[2]{\frac{\text{\footnotesize $#1$}}{\text{\footnotesize $#2$}}}
\newcommand{\vasymptote}[2][]{
    \draw [densely dashed,#1] ({rel axis cs:0,0} -| {axis cs:#2,0}) -- ({rel axis cs:0,1} -| {axis cs:#2,0});
}


\graphicspath{{./}{firstExample/}}

\usepackage{amsmath}
\usepackage{amssymb}
\usepackage{array}
\usepackage[makeroom]{cancel} %% for strike outs
\usepackage{pgffor} %% required for integral for loops
\usepackage{tikz}
\usepackage{tikz-cd}
\usepackage{tkz-euclide}
\usetikzlibrary{shapes.multipart}


\usetkzobj{all}
\tikzstyle geometryDiagrams=[ultra thick,color=blue!50!black]


\usetikzlibrary{arrows}
\tikzset{>=stealth,commutative diagrams/.cd,
  arrow style=tikz,diagrams={>=stealth}} %% cool arrow head
\tikzset{shorten <>/.style={ shorten >=#1, shorten <=#1 } } %% allows shorter vectors

\usetikzlibrary{backgrounds} %% for boxes around graphs
\usetikzlibrary{shapes,positioning}  %% Clouds and stars
\usetikzlibrary{matrix} %% for matrix
\usepgfplotslibrary{polar} %% for polar plots
\usepgfplotslibrary{fillbetween} %% to shade area between curves in TikZ



%\usepackage[width=4.375in, height=7.0in, top=1.0in, papersize={5.5in,8.5in}]{geometry}
%\usepackage[pdftex]{graphicx}
%\usepackage{tipa}
%\usepackage{txfonts}
%\usepackage{textcomp}
%\usepackage{amsthm}
%\usepackage{xy}
%\usepackage{fancyhdr}
%\usepackage{xcolor}
%\usepackage{mathtools} %% for pretty underbrace % Breaks Ximera
%\usepackage{multicol}



\newcommand{\RR}{\mathbb R}
\newcommand{\R}{\mathbb R}
\newcommand{\C}{\mathbb C}
\newcommand{\N}{\mathbb N}
\newcommand{\Z}{\mathbb Z}
\newcommand{\dis}{\displaystyle}
%\renewcommand{\d}{\,d\!}
\renewcommand{\d}{\mathop{}\!d}
\newcommand{\dd}[2][]{\frac{\d #1}{\d #2}}
\newcommand{\pp}[2][]{\frac{\partial #1}{\partial #2}}
\renewcommand{\l}{\ell}
\newcommand{\ddx}{\frac{d}{\d x}}

\newcommand{\zeroOverZero}{\ensuremath{\boldsymbol{\tfrac{0}{0}}}}
\newcommand{\inftyOverInfty}{\ensuremath{\boldsymbol{\tfrac{\infty}{\infty}}}}
\newcommand{\zeroOverInfty}{\ensuremath{\boldsymbol{\tfrac{0}{\infty}}}}
\newcommand{\zeroTimesInfty}{\ensuremath{\small\boldsymbol{0\cdot \infty}}}
\newcommand{\inftyMinusInfty}{\ensuremath{\small\boldsymbol{\infty - \infty}}}
\newcommand{\oneToInfty}{\ensuremath{\boldsymbol{1^\infty}}}
\newcommand{\zeroToZero}{\ensuremath{\boldsymbol{0^0}}}
\newcommand{\inftyToZero}{\ensuremath{\boldsymbol{\infty^0}}}


\newcommand{\numOverZero}{\ensuremath{\boldsymbol{\tfrac{\#}{0}}}}
\newcommand{\dfn}{\textbf}
%\newcommand{\unit}{\,\mathrm}
\newcommand{\unit}{\mathop{}\!\mathrm}
%\newcommand{\eval}[1]{\bigg[ #1 \bigg]}
\newcommand{\eval}[1]{ #1 \bigg|}
\newcommand{\seq}[1]{\left( #1 \right)}
\renewcommand{\epsilon}{\varepsilon}
\renewcommand{\iff}{\Leftrightarrow}

\DeclareMathOperator{\arccot}{arccot}
\DeclareMathOperator{\arcsec}{arcsec}
\DeclareMathOperator{\arccsc}{arccsc}
\DeclareMathOperator{\si}{Si}
\DeclareMathOperator{\proj}{proj}
\DeclareMathOperator{\scal}{scal}
\DeclareMathOperator{\cis}{cis}
\DeclareMathOperator{\Arg}{Arg}
%\DeclareMathOperator{\arg}{arg}
\DeclareMathOperator{\Rep}{Re}
\DeclareMathOperator{\Imp}{Im}
\DeclareMathOperator{\sech}{sech}
\DeclareMathOperator{\csch}{csch}
\DeclareMathOperator{\Log}{Log}

\newcommand{\tightoverset}[2]{% for arrow vec
  \mathop{#2}\limits^{\vbox to -.5ex{\kern-0.75ex\hbox{$#1$}\vss}}}
\newcommand{\arrowvec}{\overrightarrow}
\renewcommand{\vec}{\mathbf}
\newcommand{\veci}{{\boldsymbol{\hat{\imath}}}}
\newcommand{\vecj}{{\boldsymbol{\hat{\jmath}}}}
\newcommand{\veck}{{\boldsymbol{\hat{k}}}}
\newcommand{\vecl}{\boldsymbol{\l}}
\newcommand{\utan}{\vec{\hat{t}}}
\newcommand{\unormal}{\vec{\hat{n}}}
\newcommand{\ubinormal}{\vec{\hat{b}}}

\newcommand{\dotp}{\bullet}
\newcommand{\cross}{\boldsymbol\times}
\newcommand{\grad}{\boldsymbol\nabla}
\newcommand{\divergence}{\grad\dotp}
\newcommand{\curl}{\grad\cross}
%% Simple horiz vectors
\renewcommand{\vector}[1]{\left\langle #1\right\rangle}


\outcome{Find the Maclaurin Series representation of a function}

\title{Maclaurin Series}

\begin{document}

\begin{abstract}
We find the Maclaurin Series representation of a function.
\end{abstract}

\maketitle

\section{Maclaurin Series}


In this section we will examine a special type of Taylor Series, called a Maclaurin Series and we will develop the Maclaurin Series for 
some important functions.

\begin{definition}[Maclaurin series] A \textbf{Maclaurin Series} is a Taylor Series with center 0, i.e., a power series representation 
for a function, $f(x)$, of the form
\[
f(x) = \sum_{n=0}^\infty c_n x^n.
\]
\end{definition}
 
We have already seen the series representation for the function $f(x) = \frac{1}{1-x}$,
\[
\frac{1}{1-x} = \sum_{n=0}^\infty  x^n.
\]
In the current context, we refer to this series representation as the Maclaurin Series for the function $\frac{1}{1-x}$.
Furthermore, we have seen the formula for the coefficients of a Taylor Series representation centered at $a$,
\[
c_n = \frac{f^{(n)}(a)}{n!},
\]
and we can use this formula to find a Maclaurin Series representation by letting $a = 0$.
We will do this for the functions $e^x, \sin(x),$ and $\cos(x)$.

\begin{example}[The Maclaurin Series for $e^x$] Find the Maclaurin Series representation for the function $f(x) = e^x$.\\
To find the Maclaurin Series representation, we must determine the coefficients, $c_n$. These are particularly easy to find for the exponential function
since $f^{(n)}(x) = e^x$ for all whole numbers, $n$. Thus the coefficients are,
\[
c_n = \frac{f^{(n)}(0)}{n!} = \frac{e^0}{n!} = \frac{1}{n!},
\]
and the Maclaurin Series representation is 
\[
e^x = \sum_{n=0}^\infty c_n x^n = \sum_{n=0}^\infty \frac{x^n}{n!} = 1 + x + \frac{x^2}{2!} + \frac{x^3}{3!} + \cdots.
\]
\end{example}

\begin{example}[The Maclaurin Series for $\sin(x)$] Find the Maclaurin Series representation for the function $f(x) = \sin(x)$.\\
To find the Maclaurin series representation, we must determine the coefficients, $c_n$. 
We will take advantage of a cyclical pattern in the derivatives of $f(x) = \sin(x)$.
The first four derivatives are:
\[
f'(x) = \cos(x), f''(x) = -\sin(x), f'''(x) - -\cos(x) \text{ and } f^{(4)}(x) = \sin(x).
\]
Since we arrived back at $\sin(x)$, the next four derivatives will be exactly the same.
The coefficients require us to plug in 0:
\[
f(0) = \sin(0) = 0, f'(0) = \cos(0) = 1,
\]
\[
 f''(0) = -\sin(0) = 0, \text{ and } f'''(0) = -\cos(0) = -1.
\]
Hence the numerators of the coefficients will cycle through the values $0, 1, 0, -1...$.
Since the numerators of the even coefficients are 0, these coefficients are themselves 0 (because $0/n! = 0$) for any whole number, $n$.
The numerators of the odd coefficients will alternate between $1$ and $-1$ and hence $c_1 = 1/1!, c_3 = -1/3!, c_5 = 1/5!, c_7 = -1/7!$, etc.
We can now write the Maclaurin Series representation as
\[
\sin(x) = x - \frac{x^3}{3!} + \frac{x^5}{5!} - \frac{x^7}{7!} + \cdots.
\]
This can be writtien in summation notation as
\[
\sin(x) = \sum_{n=0}^\infty \frac{(-1)^{n+1}}{(2n+1)!}x^{2n+1}.
\]
We now make an observation about this formula. Recall that $\sin(x)$ is an \textit{odd} function, 
i.e., $\sin(-x) = -\sin(x)$, and now notice that the Maclaurin Series representation for $\sin(x)$ consists of only the 
odd powers of $x$.
\end{example}




\begin{example}[The Maclaurin Series for $\cos(x)$] Find the Maclaurin Series representation for the function $f(x) = \cos(x)$.\\
Most of the work has already been done for us in the proceeding example.  The numerators of the coefficients will cycle through the same
four values as $\sin(x)$, but instead of starting with 0, they start with $\cos(0) = 1$. 
Specfically, the numerators of the coefficients cycle throught he values $1, 0, -1, 0, ...$.  Again, when the numerator is 0, the coefficient is 0,
so the Maclaurin Series representation for $\cos(x)$ will only consist of even powers of $x$:
\[
\cos(x) = 1 - \frac{x^2}{2!} + \frac{x^4}{4!} - \frac{x^6}{6!} + \cdots.
\]
This can be writtien in summation notation as
\[
\sin(x) = \sum_{n=0}^\infty \frac{(-1)^n}{(2n)!}x^{2n}.
\]
We now make a similar observation about this formula than the one we made for $\sin(x)$. Recall that $\cos(x)$ is an \textit{even} function, 
i.e., $\cos(-x) = \cos(x)$, and as mentioned the Maclaurin Series representation for $\cos(x)$ consists of only the 
even powers of $x$.
\end{example}


\section{Video Lessons}


\begin{center}
\begin{foldable}
\unfoldable{Here is a detailed, lecture style video on Maclaurin Series:}
\youtube{aa9rcfNJrfk}
\end{foldable}
\end{center}





\end{document}


\begin{example} %example #15
Find $h'(x)$ if $h(x) = x^{\sin(x)}$.\\
We will use the fact that the exponential and logarithm functions are inverses,
\[e^{\ln(x)} = x,\]
and the exponent property of logarithms, 
\[\ln(x^n) = n\ln(x),\]
to rewrite $h(x)$.  We have 
\[h(x) = x^{\sin(x)} = e^{\ln(x^{\sin(x)})} = e^{\sin(x)\ln(x)}\]
and we can now compute $h'(x)$ using a combination of the chain rule and product rule.
We can write $h(x)$ as a composition, $f(g(x))$ with 
\[g(x) = \sin(x)\ln(x) \quad \text{and} \quad f(x) = e^x.\]
Then to find $g'(x)$ we us the product rule and we get $g'(x) = \frac{\sin(x)}{x} + \cos(x)\ln(x)$.
Next $f'(x) = e^x$ and 
hence $f'(g(x)) = e^{g(x)} = e^{\sin(x)\ln(x)} = x^{\sin(x)}$.
We can then conclude $h'(x) = f'(g(x))g'(x) = x^{\sin(x)} \left[ \frac{\sin(x)}{x} + \cos(x)\ln(x)\right]$.
\end{example}

%more question formats below













%\begin{verbatim}
\begin{question}
What is your favorite color?
\begin{multipleChoice}
\choice[correct]{Rainbow}
\choice{Blue}
\choice{Green}
\choice{Red}
\end{multipleChoice}
\begin{freeResponse}
Hello
\end{freeResponse}
\end{question}
%\end{verbatim}





\begin{question}
  Which one will you choose?
  \begin{multipleChoice}
    \choice[correct]{I'm correct.}
    \choice{I'm wrong.}
    \choice{I'm wrong too.}
  \end{multipleChoice}
\end{question}


\begin{question}
  Which one will you choose?
  \begin{selectAll}
    \choice[correct]{I'm correct.}
    \choice{I'm wrong.}
    \choice[correct]{I'm also correct.}
    \choice{I'm wrong too.}
  \end{selectAll}
\end{question}


\begin{freeResponse}
What is the chain rule used for?
\end{freeResponse}
