\documentclass{ximera}

%% You can put user macros here
%% However, you cannot make new environments



\newcommand{\ffrac}[2]{\frac{\text{\footnotesize $#1$}}{\text{\footnotesize $#2$}}}
\newcommand{\vasymptote}[2][]{
    \draw [densely dashed,#1] ({rel axis cs:0,0} -| {axis cs:#2,0}) -- ({rel axis cs:0,1} -| {axis cs:#2,0});
}


\graphicspath{{./}{firstExample/}}

\usepackage{amsmath}
\usepackage{amssymb}
\usepackage{array}
\usepackage[makeroom]{cancel} %% for strike outs
\usepackage{pgffor} %% required for integral for loops
\usepackage{tikz}
\usepackage{tikz-cd}
\usepackage{tkz-euclide}
\usetikzlibrary{shapes.multipart}


\usetkzobj{all}
\tikzstyle geometryDiagrams=[ultra thick,color=blue!50!black]


\usetikzlibrary{arrows}
\tikzset{>=stealth,commutative diagrams/.cd,
  arrow style=tikz,diagrams={>=stealth}} %% cool arrow head
\tikzset{shorten <>/.style={ shorten >=#1, shorten <=#1 } } %% allows shorter vectors

\usetikzlibrary{backgrounds} %% for boxes around graphs
\usetikzlibrary{shapes,positioning}  %% Clouds and stars
\usetikzlibrary{matrix} %% for matrix
\usepgfplotslibrary{polar} %% for polar plots
\usepgfplotslibrary{fillbetween} %% to shade area between curves in TikZ



%\usepackage[width=4.375in, height=7.0in, top=1.0in, papersize={5.5in,8.5in}]{geometry}
%\usepackage[pdftex]{graphicx}
%\usepackage{tipa}
%\usepackage{txfonts}
%\usepackage{textcomp}
%\usepackage{amsthm}
%\usepackage{xy}
%\usepackage{fancyhdr}
%\usepackage{xcolor}
%\usepackage{mathtools} %% for pretty underbrace % Breaks Ximera
%\usepackage{multicol}



\newcommand{\RR}{\mathbb R}
\newcommand{\R}{\mathbb R}
\newcommand{\C}{\mathbb C}
\newcommand{\N}{\mathbb N}
\newcommand{\Z}{\mathbb Z}
\newcommand{\dis}{\displaystyle}
%\renewcommand{\d}{\,d\!}
\renewcommand{\d}{\mathop{}\!d}
\newcommand{\dd}[2][]{\frac{\d #1}{\d #2}}
\newcommand{\pp}[2][]{\frac{\partial #1}{\partial #2}}
\renewcommand{\l}{\ell}
\newcommand{\ddx}{\frac{d}{\d x}}

\newcommand{\zeroOverZero}{\ensuremath{\boldsymbol{\tfrac{0}{0}}}}
\newcommand{\inftyOverInfty}{\ensuremath{\boldsymbol{\tfrac{\infty}{\infty}}}}
\newcommand{\zeroOverInfty}{\ensuremath{\boldsymbol{\tfrac{0}{\infty}}}}
\newcommand{\zeroTimesInfty}{\ensuremath{\small\boldsymbol{0\cdot \infty}}}
\newcommand{\inftyMinusInfty}{\ensuremath{\small\boldsymbol{\infty - \infty}}}
\newcommand{\oneToInfty}{\ensuremath{\boldsymbol{1^\infty}}}
\newcommand{\zeroToZero}{\ensuremath{\boldsymbol{0^0}}}
\newcommand{\inftyToZero}{\ensuremath{\boldsymbol{\infty^0}}}


\newcommand{\numOverZero}{\ensuremath{\boldsymbol{\tfrac{\#}{0}}}}
\newcommand{\dfn}{\textbf}
%\newcommand{\unit}{\,\mathrm}
\newcommand{\unit}{\mathop{}\!\mathrm}
%\newcommand{\eval}[1]{\bigg[ #1 \bigg]}
\newcommand{\eval}[1]{ #1 \bigg|}
\newcommand{\seq}[1]{\left( #1 \right)}
\renewcommand{\epsilon}{\varepsilon}
\renewcommand{\iff}{\Leftrightarrow}

\DeclareMathOperator{\arccot}{arccot}
\DeclareMathOperator{\arcsec}{arcsec}
\DeclareMathOperator{\arccsc}{arccsc}
\DeclareMathOperator{\si}{Si}
\DeclareMathOperator{\proj}{proj}
\DeclareMathOperator{\scal}{scal}
\DeclareMathOperator{\cis}{cis}
\DeclareMathOperator{\Arg}{Arg}
%\DeclareMathOperator{\arg}{arg}
\DeclareMathOperator{\Rep}{Re}
\DeclareMathOperator{\Imp}{Im}
\DeclareMathOperator{\sech}{sech}
\DeclareMathOperator{\csch}{csch}
\DeclareMathOperator{\Log}{Log}

\newcommand{\tightoverset}[2]{% for arrow vec
  \mathop{#2}\limits^{\vbox to -.5ex{\kern-0.75ex\hbox{$#1$}\vss}}}
\newcommand{\arrowvec}{\overrightarrow}
\renewcommand{\vec}{\mathbf}
\newcommand{\veci}{{\boldsymbol{\hat{\imath}}}}
\newcommand{\vecj}{{\boldsymbol{\hat{\jmath}}}}
\newcommand{\veck}{{\boldsymbol{\hat{k}}}}
\newcommand{\vecl}{\boldsymbol{\l}}
\newcommand{\utan}{\vec{\hat{t}}}
\newcommand{\unormal}{\vec{\hat{n}}}
\newcommand{\ubinormal}{\vec{\hat{b}}}

\newcommand{\dotp}{\bullet}
\newcommand{\cross}{\boldsymbol\times}
\newcommand{\grad}{\boldsymbol\nabla}
\newcommand{\divergence}{\grad\dotp}
\newcommand{\curl}{\grad\cross}
%% Simple horiz vectors
\renewcommand{\vector}[1]{\left\langle #1\right\rangle}


\outcome{Compute the derivative of an implicit function}

\title{2.13 Implicit Differentiation}


\begin{document}

\begin{abstract}
We learn to compute the derivative of an implicit function.
\end{abstract}

\maketitle

\section{Implicit Differentiation}

Implicit Differentiation is used to find $\frac{dy}{dx}$ in situations where $y$ is not written 
as an explicit function of $x$. 
Some examples of equations where implicit differentiation is necessary are:

\[x^2 + y^2 = 1, \text{ the unit circle} \]
\[4x^2 + 9y^2 = 36, \text{ an ellipse} \]
\[x^3 + y^3 =6xy, \text{ the Folium of Descartes}\]
\[x^2(x^2 + y^2) =y^2,  \text{ the kappa curve and}\]
\[(x^2 + y^2)^2 =50xy, \text{ a lemniscate}.\]

To compute $\frac{dy}{dx}$ in these situations, we make the assumption that $y$ is an unspecified 
function of $x$ and in most cases,
we employ the chain rule  with $y$ as the inside function.

\section{Warm-up Examples of Implicit Differentiation}

\begin{example} %example w1
Find $\frac{d}{dx} y^2$ where $y$ is an unspecified function of $x$.\\
We use the chain rule,
\[\displaystyle{[f(g(x))]' = f'(g(x))g'(x)}\]
with $g(x) = y$ and $f(x) = x^2$. Then 
\[g'(x) = \frac{dy}{dx} \text{ and}\]
\[f'(x) = 2x, \text{ so } f'(g(x)) = 2g(x) = 2y\]
and we can conclude that
\[\frac{d}{dx} y^2 = 2y \frac{dy}{dx}.\]
\end{example}


\begin{center}
\begin{foldable}
\unfoldable{Here is a video of Warm-up 1}
\youtube{Tk3Ykf3TOfQ} %vid of warm-up 1
\end{foldable}
\end{center}


\begin{question} %problem #1
  Compute
  \[
  \frac{d}{dx} y^3
  \]
  
	  \begin{hint}
		  Notice the mismatched letters, $x$ and $y$, so this is not an ordinary derivative.
    \end{hint}
		\begin{hint}
      Treat $y$ as an unspecified function of $x$, i.e., $y = g(x)$.
    \end{hint}
    \begin{hint}
      Use the Chain Rule with $y$ as the inside function.
    \end{hint}
    \begin{hint}
      The Chain Rule says $\dfrac{d}{dx} f(g(x)) = f'(g(x)) g'(x).$
    \end{hint}
		\begin{hint}
		  The derivative of the inside is written as $\dfrac{dy}{dx}$ or $y'$.
    \end{hint}
		\begin{hint}
      We have
      \[
      \frac{d}{dx} y^3 =  3y^2 \dfrac{dy}{dx}.
      \]
    \end{hint}
    
		The derivative of $y^3$ with respect to $x$ is
		 $\answer[given]{3y^2 \frac{dy}{dx}}$
		
\end{question}


  

\begin{example} %example w2
Find $\frac{d}{dx} \sin(y)$ where $y$ is an unspecified function of $x$.\\
We use the chain rule,
\[\displaystyle{[f(g(x))]' = f'(g(x))g'(x)}\]
with $g(x) = y$ and $f(x) = \sin(x)$. Then 
\[g'(x) = \frac{dy}{dx} \text{ and}\]
\[f'(x) = \cos(x), \text{ so } f'(g(x)) = \cos(g(x)) = \cos(y)\]
and we can conclude that
\[\frac{d}{dx} \sin(y) = \cos(y) \frac{dy}{dx}.\]
\end{example}


\begin{center}
\begin{foldable}
\unfoldable{Here is a video of Warm-up 2}
\youtube{1PCYgMMkiOA} %vid of warm-up 2
\end{foldable}
\end{center}

\begin{question} %problem #2a
  Compute
  \[
  \frac{d}{dx} \cos(y)
  \]
  
	  
    \begin{hint}
      Consider $y$ to be a function of $x$.
    \end{hint}
    \begin{hint}
      Use the chain rule with $y$ as the inside function.
    \end{hint}
    \begin{hint}
      The chain rule says:
      \[
      \frac{d}{dx} f(g(x)) = f'(g(x))\cdot g'(x)
      \]
    \end{hint}
    \begin{hint}
      The derivative of the inside is 
      \[
      \frac{dy}{dx}
      \]
    \end{hint}
    
		The derivative of $\cos(y)$ with respect to $x$ is
		 $\answer[given]{-\sin(y) \frac{dy}{dx}}$
		
\end{question}

\begin{question} %problem #2b
  Compute
  \[
  \frac{d}{dx} \sec(y)
  \]
  
	  
    \begin{hint}
      Consider $y$ to be a function of $x$.
    \end{hint}
    \begin{hint}
     The derivative of $\sec(x)$ with respect to x is $\sec(x)\tan(x)$.
    \end{hint}
		\begin{hint}
      Use the chain rule with $y$ as the inside function.
    \end{hint}
    \begin{hint}
      The chain rule says:
      \[
      \frac{d}{dx} f(g(x)) = f'(g(x))\cdot g'(x)
      \]
    \end{hint}
    \begin{hint}
      The derivative of the inside is 
      \[
      \frac{dy}{dx}
      \]
    \end{hint}
    
		The derivative of $\sec(y)$ with respect to $x$ is
		 $\answer[given]{\sec(y)\tan(y) \frac{dy}{dx}}$
		
\end{question}


\begin{question} %problem #2a
  Compute
  \[
  \frac{d}{dx} e^y
  \]
  
	  
    \begin{hint}
      Consider $y$ to be a function of $x$.
    \end{hint}
    \begin{hint}
      Use the chain rule with $y$ as the inside function.
    \end{hint}
    \begin{hint}
      The chain rule says:
      \[
      \frac{d}{dx} f(g(x)) = f'(g(x))\cdot g'(x)
      \]
    \end{hint}
    \begin{hint}
      The derivative of the inside is 
      \[
      \frac{dy}{dx}
      \]
    \end{hint}
    
		The derivative of $e^y$ with respect to $x$ is
		 $\answer[given]{e^y \frac{dy}{dx}}$
		
\end{question}


\begin{question} %problem #2a
  Compute
  \[
  \frac{d}{dx} 3^y
  \]
  
	  
    \begin{hint}
      Consider $y$ to be a function of $x$.
    \end{hint}
    \begin{hint}
      Use the chain rule with $y$ as the inside function.
    \end{hint}
    \begin{hint}
      The chain rule says:
      \[
      \frac{d}{dx} f(g(x)) = f'(g(x))\cdot g'(x)
      \]
    \end{hint}
    \begin{hint}
      The derivative of the inside is 
      \[
      \frac{dy}{dx}
      \]
    \end{hint}
    
		The derivative of $3^y$ with respect to $x$ is
		 $\answer[given]{3^y \ln(3) \frac{dy}{dx}}$
		
\end{question}


\begin{question} %problem #2a
  Compute
  \[
  \frac{d}{dx} \ln(y)
  \]
  
	  
    \begin{hint}
      Consider $y$ to be a function of $x$.
    \end{hint}
    \begin{hint}
      Use the chain rule with $y$ as the inside function.
    \end{hint}
    \begin{hint}
      The chain rule says:
      \[
      \frac{d}{dx} f(g(x)) = f'(g(x))\cdot g'(x)
      \]
    \end{hint}
    \begin{hint}
      The derivative of the inside is 
      \[
      \frac{dy}{dx}
      \]
    \end{hint}
    
		The derivative of $\ln(y)$ with respect to $x$ is
		 $\answer[given]{\frac{1}{y} \frac{dy}{dx}}$
		
\end{question}


\begin{question} %problem #2a
  Compute
  \[
  \frac{d}{dx} \log(y)
  \]
  
	  
    \begin{hint}
      Consider $y$ to be a function of $x$.
    \end{hint}
    \begin{hint}
      Use the chain rule with $y$ as the inside function.
    \end{hint}
    \begin{hint}
      The chain rule says:
      \[
      \frac{d}{dx} f(g(x)) = f'(g(x))\cdot g'(x)
      \]
    \end{hint}
    \begin{hint}
      The derivative of the inside is 
      \[
      \frac{dy}{dx}
      \]
    \end{hint}
    
		The derivative of $\log(y)$ with respect to $x$ is
		 $\answer[given]{\frac{1}{y \ln(10)} \frac{dy}{dx}}$
		
\end{question}


\begin{question} %problem #2c
  Compute
  \[
  \frac{d}{dx} \tan^{-1}(y)
  \]
  
	  
    \begin{hint}
      Consider y to be a function of x.
    \end{hint}
		\begin{hint}
     The derivative of $\tan^{-1}(x)$ with respect to x is $\frac{1}{1+x^2}$.
    \end{hint}
    \begin{hint}
      Use the chain rule with $y$ as the inside function.
    \end{hint}
    \begin{hint}
      The chain rule says:
      \[
      \frac{d}{dx} f(g(x)) = f'(g(x))\cdot g'(x)
      \]
    \end{hint}
    \begin{hint}
      The derivative of the inside is 
      \[
      \frac{dy}{dx}
      \]
    \end{hint}
    
		The derivative of $\tan^{-1}(y)$ with respect to $x$ is
		 $\answer[given]{\frac{1}{1+y^2} \frac{dy}{dx}}$
		
\end{question}



\begin{example} %example w3
Find $\frac{d}{dx} \big(\frac{1}{y}\big)$ where $y$ is an unspecified function of $x$.\\
We use the chain rule,
\[\displaystyle{[f(g(x))]' = f'(g(x))g'(x)}\]
with $g(x) = y$ and $f(x) = \frac{1}{x}$. Then 
\[g'(x) = \frac{dy}{dx} \text{ and}\]
\[f'(x) = -\frac{1}{x^2}, \text{ so } f'(g(x)) = -\frac{1}{g^2(x)} = -\frac{1}{y^2}\]
and we can conclude that
\[\frac{d}{dx} \Big(\frac{1}{y}\Big) = -\frac{1}{y^2} \frac{dy}{dx}.\]
\end{example}


\begin{center}
\begin{foldable}
\unfoldable{Here is a video of Warm-up 3}
\youtube{kqg8PlpHRCw} %vid of warm-up 3
\end{foldable}
\end{center}


\begin{question} %problem #3
  Compute
  \[
  \frac{d}{dx} \left( \frac{1}{y^2} \right)
  \]
  
	  \begin{hint}
      \[\frac{1}{y^2} = y^{-2}\]
    \end{hint}
    \begin{hint}
      Consider $y$ to be a function of $x$.
    \end{hint}
    \begin{hint}
      Use the chain rule with $y$ as the inside function.
    \end{hint}
    \begin{hint}
      The chain rule says:
      \[
      \frac{d}{dx} f(g(x)) = f'(g(x))\cdot g'(x)
      \]
    \end{hint}
    \begin{hint}
      The derivative of the inside is 
      \[
      \frac{dy}{dx}
      \]
    \end{hint}
    
		The derivative of $\frac{1}{y^2}$ with respect to $x$ is
		 $\answer[given]{-\frac{2}{y^3} \frac{dy}{dx}}$
		
\end{question}


\begin{example} %example w4
Find $\frac{d}{dx} (6xy)$ where $y$ is an unspecified function of $x$.\\
We use the product rule,
\[\displaystyle{[f(x) + g(x)]' = f(x)g'(x) + g(x)f'(x)}\]
with $f(x) = 6x$ and $g(x) = y$. Then 
\[f'(x) = 6 \text{ and } g'(x) = \frac{dy}{dx}.\]
We conclude that
\[\frac{d}{dx} (6xy) = 6x \frac{dy}{dx} + 6y.\]
\end{example}



\begin{center}
\begin{foldable}
\unfoldable{Here is a video of Warm-up 4}
\youtube{P8maNTP-hjk} %vid of warm-up 4
\end{foldable}
\end{center}


\begin{question} %problem #4
  Compute
  \[
  \frac{d}{dx} 3x^2y
  \]
  
	  
    \begin{hint}
      Consider $y$ to be a function of $x$.
    \end{hint}
		\begin{hint}
      Use the product rule with $3x^2$ and $y$.
    \end{hint}
    
    
		The derivative of $3x^2y$ with respect to $x$ is
		 $\answer[given]{3x^2  \frac{dy}{dx} + 6xy}$
		
\end{question}


\begin{example} %example w5
Find $\frac{d}{dx} (x^2y^3)$ where $y$ is an unspecified function of $x$.\\
We use the product rule,
\[\displaystyle{[f(x) + g(x)]' = f(x)g'(x) + g(x)f'(x)}\]
with $f(x) = x^2$ and $g(x) = y^3$. Then 
\[f'(x) = 2x \]
and by the chain rule,
\[g'(x) = 3y^2\frac{dy}{dx}.\]
We conclude that
\[\frac{d}{dx} (x^2y^3) = 3x^2y^2\frac{dy}{dx} + 2xy^3.\]
\end{example}



\begin{center}
\begin{foldable}
\unfoldable{Here is a video of Warm-up 5}
\youtube{Rp5WcRCuGJg} %vid of warm-up 5
\end{foldable}
\end{center}

\begin{question} %problem #5
  Compute
  \[
  \frac{d}{dx} x^3y^4
  \]
  
	  
    \begin{hint}
      Consider $y$ to be a function of $x$.
    \end{hint}
		\begin{hint}
      Use the product rule with $x^3$ and $y^4$.
    \end{hint}
    \begin{hint}
      Use the chain rule on $y^4$ with $y$ as the inside function.
    \end{hint}
    \begin{hint}
      The chain rule says:
      \[
      \frac{d}{dx} f(g(x)) = f'(g(x))\cdot g'(x)
      \]
    \end{hint}
    \begin{hint}
      The derivative of the inside is 
      \[
      \frac{dy}{dx}
      \]
    \end{hint}
    
		The derivative of $x^3 y^4$ with respect to $x$ is
		 $\answer[given]{4x^3 y^3 \frac{dy}{dx} + 3x^2y^4}$
		
\end{question}


\begin{problem} %problem #4
  Compute
  \[
  \frac{d}{dx} x^2e^y
  \]
  
	  
    \begin{hint}
      Consider $y$ to be a function of $x$.
    \end{hint}
		\begin{hint}
      Use the product rule with $x^2$ and $e^y$.
    \end{hint}
    \begin{hint}
      Use the chain rule on $e^y$ with $y$ as the inside function.
    \end{hint}
    \begin{hint}
      The chain rule says:
      \[
      \frac{d}{dx} f(g(x)) = f'(g(x))\cdot g'(x)
      \]
    \end{hint}
    \begin{hint}
      The derivative of the inside is 
      \[
      \frac{dy}{dx}
      \]
    \end{hint}
    
		The derivative of $x^2 e^y$ with respect to $x$ is
		 $\answer[given]{x^2 e^y \frac{dy}{dx} + 2xe^y}$
		
\end{problem}







We are now ready to find $\frac{dy}{dx}$ in situations where the formula describing a curve in the $x,y$-plane 
is not given in the form $y = f(x)$.

\section{Examples of Implicit Differentiation}


\begin{example} %example 1
Find $\displaystyle{\frac{dy}{dx}}$ if $x^2 + y^2 = 1$.\\
We assume that $y$ is a 
function of $x$ and we differentiate both 
sides with respect to $x$:
\[\frac{d}{dx}(x^2 + y^2)  = \frac{d}{dx} (1) \]
\[\frac{d}{dx}(x^2) + \frac{d}{dx} (y^2) = 0\]
\[2x + 2y\frac{dy}{dx} = 0.\]
We can now solve for $\displaystyle{\frac{dy}{dx}}$ algebraically:
\[2y\frac{dy}{dx} = -2x\]
\[\frac{dy}{dx} = -\frac{2x}{2y}= -\frac{x}{y}.\]
\end{example}



\begin{center}
\begin{foldable}
\unfoldable{Here is a video of Example 1}
\youtube{8QioZEjtOgE} %vid of example 1
\end{foldable}
\end{center}

\begin{problem}
Use the result of the example above to find the equation of the tangent line to the unit circle $x^2 + y^2 = 1$ at the point $(-\tfrac35,\tfrac45).$

\begin{hint}
$y-y_0 = m(x-x_0)$
\end{hint}
\begin{hint}
$m = \frac{dy}{dx}\bigg|_{(-\frac35, \frac45)}$
\end{hint}
The equation of the tangent line is $y = \answer{\frac34 x+\frac54}.$
\end{problem}


\begin{problem} %problem #1
  Find $\frac{dy}{dx}$ if
  \[
  x^2 + y^2 = 25
  \]
  
	  
    \begin{hint}
      Consider $y$ to be a function of $x$.
    \end{hint}
		\begin{hint}
		  Differentiate both sides with respect to $x$.
		\end{hint}
    \begin{hint}
      Use the chain rule on $y^2$ with $y$ as the inside function.
    \end{hint}
    \begin{hint}
      The derivative of the inside is 
      \[
      \frac{dy}{dx}
      \]
    \end{hint}
		\begin{hint}
		  \[
			2x + 2y\frac{dy}{dx} = 0
			\]
		\end{hint}
		\begin{hint}
		  Solve for $\frac{dy}{dx}$ by moving the terms with $\frac{dy}{dx}$
			to one side and the 
			rest of the terms to the other side.
		\end{hint}
    
		If $x^2 + y^2 = 25$ then $\frac{dy}{dx}$ is
		 $\answer[given]{-\frac{x}{y}}$
		
\end{problem}




\begin{example} %example 2
Find  $\frac{dy}{dx}$ if $4x^2 + 9y^2 = 36$.\\
We assume that $y$ is a function of $x$ and we differentiate both 
sides with respect to $x$:
\[\frac{d}{dx}(4x^2 + 9y^2)  = \frac{d}{dx} (36)\]
\[\frac{d}{dx}(4x^2) + \frac{d}{dx} (9y^2) = 0\]
\[8x + 18y\frac{dy}{dx} = 0.\]
We can now solve for $\displaystyle{\frac{dy}{dx}}$ algebraically:
\[18y\frac{dy}{dx} = -8x\]
\[\frac{dy}{dx} = -\frac{8x}{18y}= -\frac{4x}{9y}.\]
\end{example}



\begin{center}
\begin{foldable}
\unfoldable{Here is a video of Example 2}
\youtube{FjHRnCfmuuQ} %vid of example 2
\end{foldable}
\end{center}


\begin{problem} %problem #2
  Find $\frac{dy}{dx}$ if
  \[
  3x^2 + 5y^2 = 11
  \]
  
	  
    \begin{hint}
      Consider $y$ to be a function of $x$.
    \end{hint}
		\begin{hint}
		  Differentiate both sides with respect to $x$.
		\end{hint}
    \begin{hint}
      Use the chain rule on $y^2$ with $y$ as the inside function.
    \end{hint}
    \begin{hint}
      The derivative of the inside is 
      \[
      \frac{dy}{dx}
      \]
    \end{hint}
		\begin{hint}
		  \[
			6x + 10y\frac{dy}{dx} = 0
			\]
		\end{hint}
		\begin{hint}
		  Solve for $\frac{dy}{dx}$ by moving the terms with $\frac{dy}{dx}$ to one side and the 
			rest of the terms to the other side.
		\end{hint}
    
		If $3x^2 + 5y^2 = 11$ then $\frac{dy}{dx}$ is
		 $\answer[given]{-\frac{3x}{5y}}$
		
\end{problem}


\begin{problem}
Use the result of the problem above to find the equation of the tangent line to the ellipse $3x^2 + 5y^2 = 11$ at the point $(\sqrt 2,1).$

\begin{hint}
$y-y_0 = m(x-x_0)$
\end{hint}
\begin{hint}
$m = \frac{dy}{dx}\bigg|_{(\sqrt 2, 1)}$
\end{hint}
The equation of the tangent line is $y = \answer{-\frac{3\sqrt 2}{5} x + \frac{11}{5}}.$
\end{problem}



\begin{example} %example 3
Find  $\frac{dy}{dx}$ if $x^3 + y^3 =6xy$.\\
We assume that $y$ is a function of $x$ and we differentiate both 
sides with respect to $x$:
\[\frac{d}{dx}(x^3 + y^3)  = \frac{d}{dx} (6xy)\]
\[\frac{d}{dx}(x^3) + \frac{d}{dx}(y^3)  = 6x\frac{dy}{dx} + 6y\]
\[3x^2 + 3y^2\frac{dy}{dx}  = 6x\frac{dy}{dx} + 6y.\]
We can now solve for $\displaystyle{\frac{dy}{dx}}$ algebraically:
\[3y^2\frac{dy}{dx}  - 6x\frac{dy}{dx} = 6y-3x^2\]
\[(3y^2- 6x)\frac{dy}{dx}   = 6y-3x^2\]
\[\frac{dy}{dx}   = \frac{6y-3x^2}{3y^2- 6x} = \frac{2y-x^2}{y^2- 2x}.\]
\end{example}



\begin{center}
\begin{foldable}
\unfoldable{Here is a video of Example 3}
\youtube{6pr31E5ggyA} %vid of example 3
\end{foldable}
\end{center}


\begin{problem}
Use the result of the example above to find the equation of the tangent line to the curve $x^3 + y^3 = 6xy$ at the point $(3,3).$

\begin{hint}
$y-y_0 = m(x-x_0)$
\end{hint}
\begin{hint}
$m = \frac{dy}{dx}\bigg|_{(3,3)}$
\end{hint}
The equation of the tangent line is $y = \answer{-x + 6}.$
\end{problem}




\begin{problem} %problem #3
  Find $\frac{dy}{dx}$ if
  \[
  x^5 + y^5 = 3x^2y
  \]
  
	  
    \begin{hint}
      Consider $y$ to be a function of $x$.
    \end{hint}
		\begin{hint}
		  Differentiate both sides with respect to $x$.
		\end{hint}
    \begin{hint}
      Use the chain rule on $y^5$ with $y$ as the inside function.
    \end{hint}
    \begin{hint}
      The derivative of the inside is 
      \[
      \frac{dy}{dx}
      \]
    \end{hint}
		\begin{hint}
      Use the product rule on the right hand side.
    \end{hint}
		\begin{hint}
		  \[
			5x^4 + 5y^4\frac{dy}{dx} = 3x^2\frac{dy}{dx} + 6xy
			\]
		\end{hint}
		\begin{hint}
		  Solve for $\frac{dy}{dx}$ by moving the terms with $\frac{dy}{dx}$ 
			to one side and the 
			rest of the terms to the other side.
		\end{hint}
    
		If $x^5 + y^5 = 3x^2y$ then $\frac{dy}{dx}$ is
		 $\answer[given]{\frac{6xy - 5x^4}{5y^4 - 3x^2}}$
		
\end{problem}



\begin{example} %example 4
Find  $\frac{dy}{dx}$ if $x^2(x^2 + y^2) =y^2$.\\
We assume that $y$ is a function of $x$ and we differentiate both 
sides with respect to $x$:
\[\frac{d}{dx}[x^2(x^2 + y^2)]  = \frac{d}{dx} (y^2 )\]
\[\frac{d}{dx}(x^4 + x^2y^2) = 2y\frac{dy}{dx}\]
\[\frac{d}{dx}(x^4) + \frac{d}{dx}(x^2y^2)  = 2y\frac{dy}{dx}\]
\[4x^3 + 2x^2y\frac{dy}{dx} + 2xy^2  = 2y\frac{dy}{dx}.\]

We can now solve for $\displaystyle{\frac{dy}{dx}}$ algebraically:
\[2x^2y\frac{dy}{dx} - 2y\frac{dy}{dx}= -4x^3 - 2xy^2\]
\[(2x^2y- 2y)\frac{dy}{dx}= -4x^3 - 2xy^2\]
\[\frac{dy}{dx}= \frac{-4x^3 - 2xy^2}{2x^2y- 2y}\]
\[\frac{dy}{dx}= -\frac{2x^3 + xy^2}{x^2y- y}= 
\frac{2x^3 + xy^2}{y - x^2y}.\]
\end{example}


\begin{center}
\begin{foldable}
\unfoldable{Here is a video of Example 4}
\youtube{mctk6cD4m2Q} %vid of example 4
\end{foldable}
\end{center}



\begin{problem} %problem #4
  Find $\frac{dy}{dx}$ if
  \[
  x\sin(y) - y\cos(x) = xy
  \]
  
	  
    \begin{hint}
      Consider $y$ to be a function of $x$.
    \end{hint}
		\begin{hint}
		  Differentiate both sides with respect to $x$.
		\end{hint}
    \begin{hint}
      Use the chain rule on $\sin(y)$ with $y$ as the inside function.
    \end{hint}
    \begin{hint}
      The derivative of the inside is 
      \[
      \frac{dy}{dx}
      \]
    \end{hint}
		\begin{hint}
      Use the product rule on both terms on the left hand side and the term on the right hand side.
    \end{hint}
		\begin{hint}
		  \[
			x\cos(y)\frac{dy}{dx} + \sin(y) + y\sin(x) - \cos(x)\frac{dy}{dx} = 
			x\frac{dy}{dx} + y
			\]
		\end{hint}
		\begin{hint}
		  Solve for $\frac{dy}{dx}$ by moving the terms with $\frac{dy}{dx}$ to one side and the 
			rest of the terms to the other side.
		\end{hint}
    
		If $x\sin(y) - y\cos(x) = xy$ then $\frac{dy}{dx}$ is
		 $\answer[given]{\frac{y - \sin(y) - y\sin(x)}{x\cos(y) -\cos(x) -x}}$
		
\end{problem}



\begin{example} %example 5
Find  $\frac{dy}{dx}$ if $(x^2 + y^2)^2 =50xy$.\\
We assume that $y$ is a function of $x$ and we 
differentiate both sides with respect to $x$:
\[\frac{d}{dx}[(x^2 + y^2)^2]  = \frac{d}{dx} (50xy )\]
\[\frac{d}{dx}(x^4 + 2x^2y^2 + y^4) = 50x\frac{dy}{dx} + 50y\]
\[\frac{d}{dx}(x^4) + \frac{d}{dx}(2x^2y^2) + \frac{d}{dx}(y^4) = 50x\frac{dy}{dx} + 50y\]
\[4x^3 + 4x^2y\frac{dy}{dx} + 4xy^2 + 4y^3 \frac{dy}{dx}  = 50x\frac{dy}{dx} + 50y.\]

We can now solve for $\displaystyle{\frac{dy}{dx}}$ algebraically:
\[4x^2y\frac{dy}{dx} + 4y^3 \frac{dy}{dx}  - 50x\frac{dy}{dx} = 50y - 4x^3 - 4xy^2\]
\[(4x^2y + 4y^3  - 50x)\frac{dy}{dx} = 50y - 4x^3 - 4xy^2\]
\[\frac{dy}{dx} = \frac{50y - 4x^3 - 4xy^2}{4x^2y + 4y^3  - 50x}= \frac{25y - 2x^3 - 2xy^2}{2x^2y + 2y^3  - 25x}.\]
\end{example}


\begin{center}
\begin{foldable}
\unfoldable{Here is a video of Example 5}
\youtube{hWNqzQsShTA} %vid of example 5
\end{foldable}
\end{center}




\begin{example} %example 6
Find  $\frac{dy}{dx}$ if $\frac{1}{x} + \frac{1}{y} =e^{2y}$.\\
We assume that $y$ is a 
function of $x$ and we differentiate both 
sides with respect to $x$:
\[\frac{d}{dx}\big(\frac{1}{x} + \frac{1}{y}\big)  = \frac{d}{dx} (e^{2y} )\]
\[\frac{d}{dx}\big(\frac{1}{x}\big) + \frac{d}{dx}\big(\frac{1}{y}\big)  = \frac{d}{dx} (e^{2y} )\]
\[-\frac{1}{x^2} - \frac{1}{y^2} \ \frac{dy}{dx}  = 2e^{2y} \ \frac{dy}{dx}. \]
We can now solve for $\displaystyle{\frac{dy}{dx}}$ algebraically:

\[-\frac{1}{x^2}    = 2e^{2y} \ \frac{dy}{dx}+ \frac{1}{y^2}\ \frac{dy}{dx} \]
\[-\frac{1}{x^2}    = \Big(2e^{2y}+ \frac{1}{y^2}\Big) \ \frac{dy}{dx} \]
\[-\frac{y^2}{x^2}    = (2y^2e^{2y}+ 1) \ \frac{dy}{dx} \]
\[\frac{dy}{dx} = -\frac{y^2}{x^2 (1+ 2y^2e^{2y})}. \]
\end{example}



\begin{center}
\begin{foldable}
\unfoldable{Here is a video of Example 6}
\youtube{38AVwfIwgIM} %vid of example 6
\end{foldable}
\end{center}



\begin{question} %problem #6
  Find $\frac{dy}{dx}$ if
  \[
  e^{xy} + x^2y^2 = 1
  \]
  
	  
    \begin{hint}
      Consider $y$ to be a function of $x$.
    \end{hint}
		\begin{hint}
		  Differentiate both sides with respect to $x$.
		\end{hint}
    \begin{hint}
      Use the chain rule on $e^{xy}$ with $xy$ as the inside function.
    \end{hint}
    \begin{hint}
      The derivative of the inside is 
      \[
      x\frac{dy}{dx} + y
      \]
			by the product rule.
    \end{hint}
		\begin{hint}
      Use the product rule on the $x^2y^2$ term.
    \end{hint}
		\begin{hint}
		  \[
			e^{xy} \left( x\frac{dy}{dx} + y\right) + 2x^2y \frac{dy}{dx} + 2xy^2 = 0
			\]
		\end{hint}
		\begin{hint}
		  Solve for $\frac{dy}{dx}$ by moving the terms with $\frac{dy}{dx}$ to one side and the 
			rest of the terms to the other side.
		\end{hint}
    
		If $e^{xy} + x^2y^2 = 1$ then $\frac{dy}{dx}$ is
		 $\answer[given]{-\frac{ye^{xy} +2xy^2}{xe^{xy} + 2x^2y}}$
		
\end{question}




\begin{center}
\begin{foldable}
\unfoldable{Here are some detailed, lecture style videos on implicit differentiation:}
\youtube{5s7UfylRtPE}
\youtube{TTfTemnYkg8}
\end{foldable}
\end{center}









\end{document}
