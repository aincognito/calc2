\documentclass{ximera}
\usepackage{tcolorbox}
%% You can put user macros here
%% However, you cannot make new environments



\newcommand{\ffrac}[2]{\frac{\text{\footnotesize $#1$}}{\text{\footnotesize $#2$}}}
\newcommand{\vasymptote}[2][]{
    \draw [densely dashed,#1] ({rel axis cs:0,0} -| {axis cs:#2,0}) -- ({rel axis cs:0,1} -| {axis cs:#2,0});
}


\graphicspath{{./}{firstExample/}}

\usepackage{amsmath}
\usepackage{amssymb}
\usepackage{array}
\usepackage[makeroom]{cancel} %% for strike outs
\usepackage{pgffor} %% required for integral for loops
\usepackage{tikz}
\usepackage{tikz-cd}
\usepackage{tkz-euclide}
\usetikzlibrary{shapes.multipart}


\usetkzobj{all}
\tikzstyle geometryDiagrams=[ultra thick,color=blue!50!black]


\usetikzlibrary{arrows}
\tikzset{>=stealth,commutative diagrams/.cd,
  arrow style=tikz,diagrams={>=stealth}} %% cool arrow head
\tikzset{shorten <>/.style={ shorten >=#1, shorten <=#1 } } %% allows shorter vectors

\usetikzlibrary{backgrounds} %% for boxes around graphs
\usetikzlibrary{shapes,positioning}  %% Clouds and stars
\usetikzlibrary{matrix} %% for matrix
\usepgfplotslibrary{polar} %% for polar plots
\usepgfplotslibrary{fillbetween} %% to shade area between curves in TikZ



%\usepackage[width=4.375in, height=7.0in, top=1.0in, papersize={5.5in,8.5in}]{geometry}
%\usepackage[pdftex]{graphicx}
%\usepackage{tipa}
%\usepackage{txfonts}
%\usepackage{textcomp}
%\usepackage{amsthm}
%\usepackage{xy}
%\usepackage{fancyhdr}
%\usepackage{xcolor}
%\usepackage{mathtools} %% for pretty underbrace % Breaks Ximera
%\usepackage{multicol}



\newcommand{\RR}{\mathbb R}
\newcommand{\R}{\mathbb R}
\newcommand{\C}{\mathbb C}
\newcommand{\N}{\mathbb N}
\newcommand{\Z}{\mathbb Z}
\newcommand{\dis}{\displaystyle}
%\renewcommand{\d}{\,d\!}
\renewcommand{\d}{\mathop{}\!d}
\newcommand{\dd}[2][]{\frac{\d #1}{\d #2}}
\newcommand{\pp}[2][]{\frac{\partial #1}{\partial #2}}
\renewcommand{\l}{\ell}
\newcommand{\ddx}{\frac{d}{\d x}}

\newcommand{\zeroOverZero}{\ensuremath{\boldsymbol{\tfrac{0}{0}}}}
\newcommand{\inftyOverInfty}{\ensuremath{\boldsymbol{\tfrac{\infty}{\infty}}}}
\newcommand{\zeroOverInfty}{\ensuremath{\boldsymbol{\tfrac{0}{\infty}}}}
\newcommand{\zeroTimesInfty}{\ensuremath{\small\boldsymbol{0\cdot \infty}}}
\newcommand{\inftyMinusInfty}{\ensuremath{\small\boldsymbol{\infty - \infty}}}
\newcommand{\oneToInfty}{\ensuremath{\boldsymbol{1^\infty}}}
\newcommand{\zeroToZero}{\ensuremath{\boldsymbol{0^0}}}
\newcommand{\inftyToZero}{\ensuremath{\boldsymbol{\infty^0}}}


\newcommand{\numOverZero}{\ensuremath{\boldsymbol{\tfrac{\#}{0}}}}
\newcommand{\dfn}{\textbf}
%\newcommand{\unit}{\,\mathrm}
\newcommand{\unit}{\mathop{}\!\mathrm}
%\newcommand{\eval}[1]{\bigg[ #1 \bigg]}
\newcommand{\eval}[1]{ #1 \bigg|}
\newcommand{\seq}[1]{\left( #1 \right)}
\renewcommand{\epsilon}{\varepsilon}
\renewcommand{\iff}{\Leftrightarrow}

\DeclareMathOperator{\arccot}{arccot}
\DeclareMathOperator{\arcsec}{arcsec}
\DeclareMathOperator{\arccsc}{arccsc}
\DeclareMathOperator{\si}{Si}
\DeclareMathOperator{\proj}{proj}
\DeclareMathOperator{\scal}{scal}
\DeclareMathOperator{\cis}{cis}
\DeclareMathOperator{\Arg}{Arg}
%\DeclareMathOperator{\arg}{arg}
\DeclareMathOperator{\Rep}{Re}
\DeclareMathOperator{\Imp}{Im}
\DeclareMathOperator{\sech}{sech}
\DeclareMathOperator{\csch}{csch}
\DeclareMathOperator{\Log}{Log}

\newcommand{\tightoverset}[2]{% for arrow vec
  \mathop{#2}\limits^{\vbox to -.5ex{\kern-0.75ex\hbox{$#1$}\vss}}}
\newcommand{\arrowvec}{\overrightarrow}
\renewcommand{\vec}{\mathbf}
\newcommand{\veci}{{\boldsymbol{\hat{\imath}}}}
\newcommand{\vecj}{{\boldsymbol{\hat{\jmath}}}}
\newcommand{\veck}{{\boldsymbol{\hat{k}}}}
\newcommand{\vecl}{\boldsymbol{\l}}
\newcommand{\utan}{\vec{\hat{t}}}
\newcommand{\unormal}{\vec{\hat{n}}}
\newcommand{\ubinormal}{\vec{\hat{b}}}

\newcommand{\dotp}{\bullet}
\newcommand{\cross}{\boldsymbol\times}
\newcommand{\grad}{\boldsymbol\nabla}
\newcommand{\divergence}{\grad\dotp}
\newcommand{\curl}{\grad\cross}
%% Simple horiz vectors
\renewcommand{\vector}[1]{\left\langle #1\right\rangle}

\outcome{Compute derivatives involving inverse trig functions.}


\title{2.9 Derivatives of Inverse Trig Functions}


\begin{document}

\begin{abstract}
In this section we compute derivatives involving $\tan^{-1}(x), \sin^{-1}(x)$ and $\cos^{-1}(x)$.
\end{abstract}

\maketitle


We begin by computing the derivative of the inverse trigonometric function $f(x) = \tan^{-1}(x)$.  
The following Pythagorean trigonometric identity will be needed:


\[
1 + \tan^2(\theta) = \sec^2(\theta).
\]
This identity follows from $\cos^2(\theta) + \sin^2(\theta) = 1$ by dividing both sides by $\cos^2(\theta)$.

We begin the derivation by using the fact that $\tan(x)$ and $\tan^{-1}(x)$ are inverse functions, so that:

\[
\tan(\tan^{-1}(x)) = x.
\]

We differentiate both sides of this equation with respect to $x$:

\[
\dd{x} \tan(\tan^{-1}(x)) = \ddx x.
\]

Using the Chain Rule on the left side gives:

\[
\sec^2(\tan^{-1}(x)) \ddx \left[\tan^{-1}(x)\right] = 1.
\]

Now, we can solve this for the derivative of $tan^{-1}(x)$:

\[
 \ddx \left[\tan^{-1}(x)\right] = \frac{1}{\sec^2\left(\tan^{-1}(x)\right)}.
\]

Next, we use the Pythagorean Identity:

\[
 \ddx \left[\tan^{-1}(x)\right] = \frac{1}{1 + \tan^2\left(\tan^{-1}(x)\right)}.
\]

Finally, using the property of inverse functions:

\[
 \ddx \left[\tan^{-1}(x)\right] = \frac{1}{1 + x^2}.
\]


Similar arguments can show that:

\[
 \ddx \left[\sin^{-1}(x)\right] = \frac{1}{\sqrt{1 - x^2}} \;\;\text{  and  } \;\;\ddx \left[\cos^{-1}(x)\right] = -\frac{1}{\sqrt{1 - x^2}}.
\]



\begin{foldable}
\unfoldable{Below is a graph of $f(x) = \tan^{-1}(x)$ (in blue) and its derivative, $f'(x) = \tfrac{1}{1+x^2}$ (in purple).
Notice that the slope of the tangent line to $f(x)$ (in red) is the height of the corresponding 
point on $f'(x)$. Use it to see a graphical representation of the answers to the problem above.}
\includeinteractive{invtanwithderiv.js}
\end{foldable}







\begin{example}[example 1]

Find the equation of the tangent line to the graph of $f(x) = \tan^{-1}(x)$ at $x=0.$\\
The point of tangency is $(0, f(0)) = (0, 0)$ since $\tan^{-1}(0) = 0$.
The slope of the tangent line is given by  $m = f'(0) = \tfrac{1}{1+0^2} = 1$.
To create the equation of the line we use the
Point slope form: $y-y_0 = m(x-x_0)$ and we get $y-0 = 1(x-0)$ or $y=x$.

\end{example}





\begin{problem}(problem 1a)
Find the equation of the tangent line to the graph of $f(x) = \tan^{-1}(x)$ at $x=1.$

\begin{hint}
The point of tangency is $(1, f(1))$
\end{hint}
\begin{hint}
Use the derivative to find the slope, $m$
\end{hint}
\begin{hint}
Point slope form: $y-y_0 = m(x-x_0)$
\end{hint}

The equation of the tangent line is \ $y = \answer{x/2 + \pi/4 - 1/2}$

\end{problem}



\begin{problem}(problem 1b)

Find the equation of the tangent line to the graph of $f(x) = \sin^{-1}(x)$ at $x=0.$

\begin{hint}
The point of tangency is $(0, f(0))$
\end{hint}
\begin{hint}
Use the derivative to find the slope, $m$
\end{hint}
\begin{hint}
Point slope form: $y-y_0 = m(x-x_0)$
\end{hint}

The equation of the tangent line is \ $y = \answer{x}$

\end{problem}



\begin{problem}(problem 1c)

Find the equation of the tangent line to the graph of $f(x) = \cos^{-1}(x)$ at $x=0.$

\begin{hint}
The point of tangency is $(0, f(0))$
\end{hint}
\begin{hint}
Use the derivative to find the slope, $m$
\end{hint}
\begin{hint}
Point slope form: $y-y_0 = m(x-x_0)$
\end{hint}

The equation of the tangent line is \ $y = \answer{-x + \pi/2}$

\end{problem}


\begin{example}[example 2]
Find $h'(x)$ if $h(x) = (1+x^2)\tan^{-1}(x)$.\\
We write $h(x)$ as a product: $h(x) = f(x)\cdot g(x)$ with
\[f(x) = 1+x^2 \quad \text{and} \quad g(x) = \tan^{-1}(x).\]
To use the product rule we need the derivatives:
\[f'(x) = 2x \quad \text{and} \quad g'(x) = \frac{1}{1+x^2}.\]
We can now write the derivative:
\begin{align*}
h'(x) &= f'(x)g(x) + f(x)g'(x) \\
&= 2x\tan^{-1}(x)  + (1+x^2)\frac{1}{1+x^2} \\
&=  2x\tan^{-1}(x) + 1.
\end{align*}
\end{example}


\begin{center}
\begin{foldable}
\unfoldable{Here is a video of the preceding example}
\youtube{NvlgmABasiQ} %vid of example 6
\end{foldable}
\end{center}



\begin{problem}(problem 2a)
  Compute
  \[
  \frac{d}{dx} x^3 \tan^{-1}(x)
  \]
  
    \begin{hint}
      Use the Product Rule with $x^3$ and $\tan^{-1}(x)$.
    \end{hint}
    \begin{hint}
      $(fg)' = f'g+g'f$.
    \end{hint}
    \begin{hint}
      The derivative of $x^3$ is $3x^2$
    \end{hint}
    \begin{hint}
      The derivative of $\tan^{-1}(x)$ is $\frac{1}{1+x^2}$
    \end{hint}
		The derivative of $x^3\tan^{-1}(x)$ with respect to $x$ is
		 $\answer[given]{3x^2 \tan^{-1}(x) + \frac{x^3}{1+x^2}}$
		
\end{problem}

\begin{problem}(problem 2b)
  Compute
  \[
  \frac{d}{dx} (1-x^2) \sin^{-1}(x)
  \]
  
    \begin{hint}
      Use the Product Rule with $1-x^2$ and $\sin^{-1}(x)$.
    \end{hint}
    \begin{hint}
      $(fg)' = f'g+g'f$.
    \end{hint}
    \begin{hint}
      The derivative of $1 - x^2$ is $-2x$
    \end{hint}
    \begin{hint}
      The derivative of $\sin^{-1}(x)$ is $\frac{1}{\sqrt{1-x^2}}$
    \end{hint}
		\begin{hint}
       $(1-x^2) \frac{1}{\sqrt{1-x^2}} = \sqrt{1-x^2}$
    \end{hint}
		The derivative of $(1-x^2)\sin^{-1}(x)$ with respect to $x$ is
		 $\answer[given]{-2x \sin^{-1}(x) + \sqrt{1-x^2}}$
		
\end{problem}


The Chain Rule versions of these formulas are:


\[
 \ddx \left[\tan^{-1}(g(x))\right] = \frac{g'(x)}{1 + g^2(x)},
\]
\[
 \ddx \left[\sin^{-1}(g(x))\right] = \frac{g'(x)}{\sqrt{1 - g^2(x)}}
\]

and

\[
 \ddx \left[\cos^{-1}(g(x))\right] = -\frac{g'(x)}{\sqrt{1 - g^2(x)}}.
\]


\begin{example}[example 3]
Find $h'(x)$ if $h(x) = \tan^{-1}(4x)$.\\
We write $h(x)$ as a composition: $h(x)=f(g(x))$ with 
\[g(x) = 4x  \quad \text{and} \quad  f(x) = \tan^{-1}(x).\]
 To find $h'(x)$ we need $f'(g(x))$ and $g'(x)$.  First, 
\[g'(x) = 4; \quad \text{next} \] 
\[f'(x) = \frac{1}{1+x^2} , \quad \text{so}\]
\[ f'(g(x)) = \frac{1}{1+g^2(x)} =\frac{1}{1+(4x)^2}=\frac{1}{1+16x^2}.\]
By the chain rule,
\[h'(x) = f'(g(x))g'(x) = \frac{1}{1+16x^2} \cdot 4 = \frac{4}{1+16x^2}.\]
\end{example}

\begin{center}
\begin{foldable}
\unfoldable{Here is a video of the example}
\youtube{-0ZwkfU9Xec}  %vid of example 14
\end{foldable}
\end{center}


\begin{problem}(problem 3a)
  Compute
  \[
  \frac{d}{dx} \tan^{-1}(3x)
  \]
  
    \begin{hint}
      The chain rule says:
      \[
      \frac{d}{dx} f(g(x)) = f'(g(x))\cdot g'(x)
      \]
    \end{hint}
    \begin{hint}
      The outside function is $f(x) = \tan^{-1}(x)$ and the inside
      function is $g(x) = 3x$.
    \end{hint}
    \begin{hint}
		  Leave the inside in, $f'(g(x))$
		\end{hint}
		\begin{hint}
		  Multiply by the derivative of the inside, $g'(x)$
		\end{hint}
    
		The derivative of $\tan^{-1}(3x)$ with respect to $x$ is
		 $\answer[given]{\frac{3}{1+(3x)^2}}$
		
\end{problem}



\begin{problem}(problem 3b)
Find the equation of the tangent line to the graph of $f(x) = \cos(x)$ at $x=\pi/2.$


\begin{hint}
The point of tangency is $(\pi/2, f(\pi/2))$
\end{hint}
\begin{hint}
Use the derivative to find the slope, $m$
\end{hint}
\begin{hint}
Point slope form: $y-y_0 = m(x-x_0)$
\end{hint}

The equation of the tangent line is \ $y = \answer{-x + \pi/2}$

\end{problem}




\end{document}






%\newcommand{\ffrac}[2]{\frac{\mbox{\footnotesize $#1$}}{\mbox{\footnotesize $#2$}}}
%\newcommand{\vasymptote}[2][]{
 %   \draw [densely dashed,#1] ({rel axis cs:0,0} -| {axis cs:#2,0}) -- ({rel axis cs:0,1} -| {axis cs:#2,0});
%}
