\documentclass{ximera}
\usepgfplotslibrary{fillbetween}
%% You can put user macros here
%% However, you cannot make new environments



\newcommand{\ffrac}[2]{\frac{\text{\footnotesize $#1$}}{\text{\footnotesize $#2$}}}
\newcommand{\vasymptote}[2][]{
    \draw [densely dashed,#1] ({rel axis cs:0,0} -| {axis cs:#2,0}) -- ({rel axis cs:0,1} -| {axis cs:#2,0});
}


\graphicspath{{./}{firstExample/}}

\usepackage{amsmath}
\usepackage{amssymb}
\usepackage{array}
\usepackage[makeroom]{cancel} %% for strike outs
\usepackage{pgffor} %% required for integral for loops
\usepackage{tikz}
\usepackage{tikz-cd}
\usepackage{tkz-euclide}
\usetikzlibrary{shapes.multipart}


\usetkzobj{all}
\tikzstyle geometryDiagrams=[ultra thick,color=blue!50!black]


\usetikzlibrary{arrows}
\tikzset{>=stealth,commutative diagrams/.cd,
  arrow style=tikz,diagrams={>=stealth}} %% cool arrow head
\tikzset{shorten <>/.style={ shorten >=#1, shorten <=#1 } } %% allows shorter vectors

\usetikzlibrary{backgrounds} %% for boxes around graphs
\usetikzlibrary{shapes,positioning}  %% Clouds and stars
\usetikzlibrary{matrix} %% for matrix
\usepgfplotslibrary{polar} %% for polar plots
\usepgfplotslibrary{fillbetween} %% to shade area between curves in TikZ



%\usepackage[width=4.375in, height=7.0in, top=1.0in, papersize={5.5in,8.5in}]{geometry}
%\usepackage[pdftex]{graphicx}
%\usepackage{tipa}
%\usepackage{txfonts}
%\usepackage{textcomp}
%\usepackage{amsthm}
%\usepackage{xy}
%\usepackage{fancyhdr}
%\usepackage{xcolor}
%\usepackage{mathtools} %% for pretty underbrace % Breaks Ximera
%\usepackage{multicol}



\newcommand{\RR}{\mathbb R}
\newcommand{\R}{\mathbb R}
\newcommand{\C}{\mathbb C}
\newcommand{\N}{\mathbb N}
\newcommand{\Z}{\mathbb Z}
\newcommand{\dis}{\displaystyle}
%\renewcommand{\d}{\,d\!}
\renewcommand{\d}{\mathop{}\!d}
\newcommand{\dd}[2][]{\frac{\d #1}{\d #2}}
\newcommand{\pp}[2][]{\frac{\partial #1}{\partial #2}}
\renewcommand{\l}{\ell}
\newcommand{\ddx}{\frac{d}{\d x}}

\newcommand{\zeroOverZero}{\ensuremath{\boldsymbol{\tfrac{0}{0}}}}
\newcommand{\inftyOverInfty}{\ensuremath{\boldsymbol{\tfrac{\infty}{\infty}}}}
\newcommand{\zeroOverInfty}{\ensuremath{\boldsymbol{\tfrac{0}{\infty}}}}
\newcommand{\zeroTimesInfty}{\ensuremath{\small\boldsymbol{0\cdot \infty}}}
\newcommand{\inftyMinusInfty}{\ensuremath{\small\boldsymbol{\infty - \infty}}}
\newcommand{\oneToInfty}{\ensuremath{\boldsymbol{1^\infty}}}
\newcommand{\zeroToZero}{\ensuremath{\boldsymbol{0^0}}}
\newcommand{\inftyToZero}{\ensuremath{\boldsymbol{\infty^0}}}


\newcommand{\numOverZero}{\ensuremath{\boldsymbol{\tfrac{\#}{0}}}}
\newcommand{\dfn}{\textbf}
%\newcommand{\unit}{\,\mathrm}
\newcommand{\unit}{\mathop{}\!\mathrm}
%\newcommand{\eval}[1]{\bigg[ #1 \bigg]}
\newcommand{\eval}[1]{ #1 \bigg|}
\newcommand{\seq}[1]{\left( #1 \right)}
\renewcommand{\epsilon}{\varepsilon}
\renewcommand{\iff}{\Leftrightarrow}

\DeclareMathOperator{\arccot}{arccot}
\DeclareMathOperator{\arcsec}{arcsec}
\DeclareMathOperator{\arccsc}{arccsc}
\DeclareMathOperator{\si}{Si}
\DeclareMathOperator{\proj}{proj}
\DeclareMathOperator{\scal}{scal}
\DeclareMathOperator{\cis}{cis}
\DeclareMathOperator{\Arg}{Arg}
%\DeclareMathOperator{\arg}{arg}
\DeclareMathOperator{\Rep}{Re}
\DeclareMathOperator{\Imp}{Im}
\DeclareMathOperator{\sech}{sech}
\DeclareMathOperator{\csch}{csch}
\DeclareMathOperator{\Log}{Log}

\newcommand{\tightoverset}[2]{% for arrow vec
  \mathop{#2}\limits^{\vbox to -.5ex{\kern-0.75ex\hbox{$#1$}\vss}}}
\newcommand{\arrowvec}{\overrightarrow}
\renewcommand{\vec}{\mathbf}
\newcommand{\veci}{{\boldsymbol{\hat{\imath}}}}
\newcommand{\vecj}{{\boldsymbol{\hat{\jmath}}}}
\newcommand{\veck}{{\boldsymbol{\hat{k}}}}
\newcommand{\vecl}{\boldsymbol{\l}}
\newcommand{\utan}{\vec{\hat{t}}}
\newcommand{\unormal}{\vec{\hat{n}}}
\newcommand{\ubinormal}{\vec{\hat{b}}}

\newcommand{\dotp}{\bullet}
\newcommand{\cross}{\boldsymbol\times}
\newcommand{\grad}{\boldsymbol\nabla}
\newcommand{\divergence}{\grad\dotp}
\newcommand{\curl}{\grad\cross}
%% Simple horiz vectors
\renewcommand{\vector}[1]{\left\langle #1\right\rangle}


\outcome{Apply definite integrals}

\title{4.8 Applications of Definite Integrals}

%\newcommand{\ffrac}[2]{\frac{\mbox{\footnotesize $#1$}}{\mbox{\footnotesize $#2$}}}
%\newcommand{\vasymptote}[2][]{\draw [densely dashed,#1] 
%({rel axis cs:0,0} -| {axis cs:#2,0}) -- ({rel axis cs:0,1} -| {axis cs:#2,0});}


\begin{document}

\begin{abstract}
In this section we use definite integrals to study rectilinear motion and compute average value.
\end{abstract}

\maketitle



\section{Applications of Definite Integrals}


\subsection{Average Value}
\begin{definition}[Average Value]
If $f(x)$ is continuous on the interval $[a,b]$, then the average value 
of $f(x)$ on $[a,b]$ is given by
\[f_{ave} = \frac{1}{b-a}\int_a^b f(x) \ dx.\]
\end{definition}

\begin{example}
Compute the average value of $f(x) = x^2$ on the interval $[0,2]$.\\
The average value is given by 
\begin{align*}
f_{ave} &= \frac{1}{b-a}\int_a^b f(x) \ dx \\
 & = \frac12\int_0^2 x^2 \ dx \\
 & = \frac{x^3}{6} \Bigg|_0^2 \\
  & = \tfrac43
\end{align*}
\end{example}

\begin{problem}
Compute the average value of $f(x) = x(x+1) +1$ on the interval $[0,3]$.
\begin{hint}
Multiply first, then integrate
\end{hint}
\[f_{ave} = \answer{11/2}.\]
\end{problem} 

\begin{example}
Compute the average value of $f(x) = \sin(x)$ on the interval $[0,\pi]$.\\
The average value is given by 
\begin{align*}
f_{ave} &= \frac{1}{b-a}\int_a^b f(x) \ dx \\
 & = \frac{1}{\pi}\int_0^\pi \sin(x) \ dx \\
 & = -\frac{1}{\pi} \cos(x) \Bigg|_0^\pi \\
  & = \frac{2}{\pi}
\end{align*}
\end{example}

\begin{problem}
Compute the average value of $f(x) = \cos(2x) + \sin(2x)$ on the interval $[0,\pi/2]$.
\begin{hint}
An anti-derivative of $\cos(2x) + \sin(2x)$ is  $\tfrac12\sin(2x) - \tfrac12\cos(2x)$
\end{hint}
\[f_{ave} = \answer{2/\pi}.\]
\end{problem} 


\begin{example}
Compute the average value of $f(x) = e^{2x}$ on the interval $[-1,1]$.\\
The average value is given by 
\begin{align*}
f_{ave} &= \frac{1}{b-a}\int_a^b f(x) \ dx \\
 & = \frac12\int_{-1}^1 e^{2x} \ dx \\
 & = \frac14 e^{2x} \Bigg|_{-1}^1 \\
  & = \frac{e^2 - e^{-2}}{4}
\end{align*}
\end{example}


\begin{problem}
Compute the average value of $f(x) = e^{3x}$ on the interval $[-1,1]$.
\[f_{ave} = \answer{\frac{e^{3} - e^{-3}}{6}}.\]
\end{problem} 


\subsection{Rectilinear Motion}

Suppose that a projectile is launched vertically into the air.
Then, the height of the object at time $t$, the velocity at time $t$, and the acceleration at time $t$ are related as follows:
\[v(t) = s'(t)\]
and
\[a(t) = v'(t) = s''(t).\]

We can also use indefinite integrals to express these relationships:
\[s(t) = \int v(t) \ dt\]
and
\[v(t) = \int a(t) \ dt.\]

\begin{proposition}
Suppose that a vertical projectile has instantaneous velocity $v(t)$, then
\begin{enumerate}
\item the displacement over the interval $[a,b]$ is given by 
\[\int_a^b v(t) \ dt\]
and
\item the distance traveled over the interval $[a,b]$ is given by 
\[\int_a^b |v(t)| \ dt.\]
\end{enumerate}
\end{proposition}

\begin{example}
An object is launched vertically from a height of $16 ft$ with an initial velocity of $48 ft/sec$.  
\begin{enumerate}
\item Find a formula, $s(t)$, for the height of the object at time $t$
 seconds.
\item Find the displacement of the object from time $t = 0$ to time $t = 2$.
\item Find the total distance traveled from time $t = 0$ to time $t = 2$.
\end{enumerate}

Assuming that $a(t) = -32 ft/sec^2$, as it is on the surface of the earth, 
we can compute $v(t)$ by using an indefinite integral:
\[v(t) = \int a(t) \ dt = \int -32 \ dt = -32 + C.\]
To find the value of the constant $C$, we use the fact that the initial velocity was given as $48 ft/sec$:
\[v(0) = -32(0) + C = 48,\]
which implies that $C = 48$.  Thus
\[v(t) = -32t + 48.\]
Next, we repeat the process to find $s(t)$:
\[s(t) = \int v(t) \ dt = \int (-32t + 48) \ dt = -16t^2 + 48t + C.\]
To find $C$, we use the given information about the initial height of the object:
\[s(0) = -16t^2 + 48(0) + C = 16,\]
which implies $C = 16$.  Thus,
\[s(t) = -16t^2 + 48t + 16.\]
Next, to find the displacement from $t= 0$ to $t = 2$, we compute
\[s(2) - s(0) = (-64+96+16)-(16) = 32 \text{feet}.\]
Note that we could have computed $s(2) - s(0)$ as a definite integral:
\[\int_0^2 v(t) \ dt = s(2) - s(0),\]
since $s(t)$ is an anti-derivative of $v(t)$.
Finally, to find the total distance traveled from $t = 0$ to time $t = 2$,
we need integrate $|v(t)|$. To get a handle on $|v(t)|$, recall
\[|x| = \left\{
     \begin{array}{rc}
       x & \ \text{if} \  x \geq 0 \\
			 -x & \ \text{if} \ x <0
     \end{array}
   \right.
\]
Noting that $v(1.5) = -32(1.5) + 48 = 0$, we can see that
\[|v(t)| = |-32t + 48| = \left\{\begin{array}{rc}
       -32t + 48  &\ \text{if} \  t \leq 1.5 \\
			 32t - 48  &\ \text{if} \ t > 1.5
     \end{array}
   \right.
\]


Hence, the total distance traveled from $t=0$ to $t = 2$ is
\[\int_0^2 |v(t)| \ dt = \int_0^{1.5} (-32t + 48) \ dt + \int_{1.5}^2 (32t -48) \ dt = 36 + 4 = 40 \ \text{feet}.\]


\end{example}

\begin{problem}
An object is launched vertically from a height of $24 ft$ with an initial velocity of $64 ft/sec$.  
\begin{enumerate}
\item Find a formula, $s(t)$, for the height of the object at time $t$
 seconds.
\item Find the maximum height reached by the object.
\item Find the displacement of the object from time $t = 0$ to time $t = 3$.
\item Find the total distance traveled from time $t = 0$ to time $t = 3$.
\end{enumerate}
Assume $g = -32 \ ft/sec^2$.\\
$s(t) = \answer{-16t^2 + 64t + 24},$\\
$\text{maximum height} =\answer{88} \ ft,$\\
$\text{displacement} =\answer{48} \ ft,$\\
and
$\text{distance traveled} = \answer{80} \ ft.$
\end{problem}

\begin{problem}
An object is launched vertically from a height of $5 ft$ with an initial velocity of $16 ft/sec$.  
\begin{enumerate}
\item Find a formula, $s(t)$, for the height of the object at time $t$
 seconds.
\item Find the maximum height reached by the object.
\end{enumerate}
Assume $g = -32 \ ft/sec^2$.\\
$s(t) = \answer{-16t^2 + 16t + 5},$\\
and
$\text{maximum height} =\answer{9} \ ft.$\\


\end{problem}

\end{document}
