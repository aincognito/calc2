\documentclass{ximera}
\usepackage{tcolorbox}
%% You can put user macros here
%% However, you cannot make new environments



\newcommand{\ffrac}[2]{\frac{\text{\footnotesize $#1$}}{\text{\footnotesize $#2$}}}
\newcommand{\vasymptote}[2][]{
    \draw [densely dashed,#1] ({rel axis cs:0,0} -| {axis cs:#2,0}) -- ({rel axis cs:0,1} -| {axis cs:#2,0});
}


\graphicspath{{./}{firstExample/}}

\usepackage{amsmath}
\usepackage{amssymb}
\usepackage{array}
\usepackage[makeroom]{cancel} %% for strike outs
\usepackage{pgffor} %% required for integral for loops
\usepackage{tikz}
\usepackage{tikz-cd}
\usepackage{tkz-euclide}
\usetikzlibrary{shapes.multipart}


\usetkzobj{all}
\tikzstyle geometryDiagrams=[ultra thick,color=blue!50!black]


\usetikzlibrary{arrows}
\tikzset{>=stealth,commutative diagrams/.cd,
  arrow style=tikz,diagrams={>=stealth}} %% cool arrow head
\tikzset{shorten <>/.style={ shorten >=#1, shorten <=#1 } } %% allows shorter vectors

\usetikzlibrary{backgrounds} %% for boxes around graphs
\usetikzlibrary{shapes,positioning}  %% Clouds and stars
\usetikzlibrary{matrix} %% for matrix
\usepgfplotslibrary{polar} %% for polar plots
\usepgfplotslibrary{fillbetween} %% to shade area between curves in TikZ



%\usepackage[width=4.375in, height=7.0in, top=1.0in, papersize={5.5in,8.5in}]{geometry}
%\usepackage[pdftex]{graphicx}
%\usepackage{tipa}
%\usepackage{txfonts}
%\usepackage{textcomp}
%\usepackage{amsthm}
%\usepackage{xy}
%\usepackage{fancyhdr}
%\usepackage{xcolor}
%\usepackage{mathtools} %% for pretty underbrace % Breaks Ximera
%\usepackage{multicol}



\newcommand{\RR}{\mathbb R}
\newcommand{\R}{\mathbb R}
\newcommand{\C}{\mathbb C}
\newcommand{\N}{\mathbb N}
\newcommand{\Z}{\mathbb Z}
\newcommand{\dis}{\displaystyle}
%\renewcommand{\d}{\,d\!}
\renewcommand{\d}{\mathop{}\!d}
\newcommand{\dd}[2][]{\frac{\d #1}{\d #2}}
\newcommand{\pp}[2][]{\frac{\partial #1}{\partial #2}}
\renewcommand{\l}{\ell}
\newcommand{\ddx}{\frac{d}{\d x}}

\newcommand{\zeroOverZero}{\ensuremath{\boldsymbol{\tfrac{0}{0}}}}
\newcommand{\inftyOverInfty}{\ensuremath{\boldsymbol{\tfrac{\infty}{\infty}}}}
\newcommand{\zeroOverInfty}{\ensuremath{\boldsymbol{\tfrac{0}{\infty}}}}
\newcommand{\zeroTimesInfty}{\ensuremath{\small\boldsymbol{0\cdot \infty}}}
\newcommand{\inftyMinusInfty}{\ensuremath{\small\boldsymbol{\infty - \infty}}}
\newcommand{\oneToInfty}{\ensuremath{\boldsymbol{1^\infty}}}
\newcommand{\zeroToZero}{\ensuremath{\boldsymbol{0^0}}}
\newcommand{\inftyToZero}{\ensuremath{\boldsymbol{\infty^0}}}


\newcommand{\numOverZero}{\ensuremath{\boldsymbol{\tfrac{\#}{0}}}}
\newcommand{\dfn}{\textbf}
%\newcommand{\unit}{\,\mathrm}
\newcommand{\unit}{\mathop{}\!\mathrm}
%\newcommand{\eval}[1]{\bigg[ #1 \bigg]}
\newcommand{\eval}[1]{ #1 \bigg|}
\newcommand{\seq}[1]{\left( #1 \right)}
\renewcommand{\epsilon}{\varepsilon}
\renewcommand{\iff}{\Leftrightarrow}

\DeclareMathOperator{\arccot}{arccot}
\DeclareMathOperator{\arcsec}{arcsec}
\DeclareMathOperator{\arccsc}{arccsc}
\DeclareMathOperator{\si}{Si}
\DeclareMathOperator{\proj}{proj}
\DeclareMathOperator{\scal}{scal}
\DeclareMathOperator{\cis}{cis}
\DeclareMathOperator{\Arg}{Arg}
%\DeclareMathOperator{\arg}{arg}
\DeclareMathOperator{\Rep}{Re}
\DeclareMathOperator{\Imp}{Im}
\DeclareMathOperator{\sech}{sech}
\DeclareMathOperator{\csch}{csch}
\DeclareMathOperator{\Log}{Log}

\newcommand{\tightoverset}[2]{% for arrow vec
  \mathop{#2}\limits^{\vbox to -.5ex{\kern-0.75ex\hbox{$#1$}\vss}}}
\newcommand{\arrowvec}{\overrightarrow}
\renewcommand{\vec}{\mathbf}
\newcommand{\veci}{{\boldsymbol{\hat{\imath}}}}
\newcommand{\vecj}{{\boldsymbol{\hat{\jmath}}}}
\newcommand{\veck}{{\boldsymbol{\hat{k}}}}
\newcommand{\vecl}{\boldsymbol{\l}}
\newcommand{\utan}{\vec{\hat{t}}}
\newcommand{\unormal}{\vec{\hat{n}}}
\newcommand{\ubinormal}{\vec{\hat{b}}}

\newcommand{\dotp}{\bullet}
\newcommand{\cross}{\boldsymbol\times}
\newcommand{\grad}{\boldsymbol\nabla}
\newcommand{\divergence}{\grad\dotp}
\newcommand{\curl}{\grad\cross}
%% Simple horiz vectors
\renewcommand{\vector}[1]{\left\langle #1\right\rangle}


\outcome{Compute the derivative of function using the definition.}

\title{1.10 Definition of Derivative}

%\newcommand{\ffrac}[2]{\frac{\mbox{\footnotesize $#1$}}{\mbox{\footnotesize $#2$}}}
%\newcommand{\vasymptote}[2][]{
 %   \draw [densely dashed,#1] ({rel axis cs:0,0} -| {axis cs:#2,0}) -- ({rel axis cs:0,1} -| {axis cs:#2,0});
%}


\begin{document}

\begin{abstract}
In this section we learn the definition of derivative and we use it to solve the tangent line problem.
\end{abstract}

\maketitle


\section{Tangent Lines}







%Here is an interactive applet that shows the graph of a function and 
%the tangent line at a point. You can move the point along the graph to see the tangent line 
%at different points.

%\geogebra{eTqftsNC}{640}{480}


The idea of a tangent line is that if we zoom in on a point $(a, f(a))$ on the graph of a smooth function, $y = f(x)$, 
then the graph looks like a straight line. 
This line is the \textbf{tangent line} and the point $(a, f(a)$ is called the \textbf{point of tangency}.



The interactive graph below shows a function and its tangent line at a point.  
You can drag the point along the curve and see how the tangent line changes.
You can also zoom in on the curve to see that the tangent line approximates the curve
locally.  If you zoom in far enough, the curve and the tangent line become nearly indistinguishable.


\includeinteractive{dd00b.js}

%insert hand drawn pic of f(x), with m_tan = f'(x) (or not)


and a point on the graph, , 
the tangent line to the graph at this point has the property that when one zooms in on the graph of the function at this point,
its \textbf{derivative}, denoted $f'(x)$, is a formula for the slope of 
its \textbf{tangent lines}.

Conceptually, the derivative represents the instantaneous rate of change of the function. 
Other notations for the derivative include $y'$ and $\frac{dy}{dx}$.\\


The \textbf{tangent line problem} is to find the equation of the tangent line to the graph of $y = f(x)$
at the point $(a, f(a))$. From the point-slope form of the equation of a line,
we see that the equation of the tangent line has the form
\[
y -f(a) = m(x-a),
\]
where $m$ is the slope of the line.  The \textbf{derivative}, $f'(x)$, will give us the slope of the tangnet line,
so the equation of the tangent line can be written as
\[
y - f(a) = f'(a)(x-a), \quad \text{or} \quad y = f(a) + f'(a)(x-a).
\]


We will find the second form uselful later when we study linear approximation.
Note that the slope of the tangent line is $f'(a)$ at the point $(a, f(a))$. 
The general derivative, $f'(x)$, will give us the slope at any point.


\section{The Derivative}
Our objective is to discover the derivative, $f'(x)$.
To begin the discussion, we recall the formula for the slope of a line between two points, $(x_1, y_1)$ 
and $(x_2, y_2)$:
\[m = \frac{\text{rise}}{\text{run}} = \frac{\Delta y}{\Delta x} = \frac{y_2 - y_1}{x_2 - x_1}.\]

To find the slope of the tangent line to the graph of $y = f(x)$ at the general point $(x, f(x))$, we first 
compute the slope of the \textbf{secant line} connecting the point $(x, f(x))$ with the 
nearby point $(x+h, f(x+h))$:
\[m_{\text{sec}} = \frac{f(x+h) - f(x)}{(x+h)-x}.\]
Simplifying the denominator, we have
\[m_{\text{sec}} = \frac{f(x+h) - f(x)}{h}.\]
This quantity is known as the \textbf{difference quotient}.

The slope of the tangent line, is obtained by letting $h \to 0$ which has the effect of 
moving the point $(x+h, f(x+h))$ towards the point 
$(x, f(x))$.  

%This requires the use of a limit.  
%To make $x+h$ approach $x$ we must make $h$ approach $0$ (we are thinking of $x$ as being fixed here).

This gives us the \textbf{definition of the derivative}:

\begin{definition}[Derivative]
The derivative of the function $f(x)$, denoted by $f'(x)$ is defined by
\[f'(x) = \lim_{h \to 0} \frac{f(x+h)-f(x)}{h}.\]
This definition is valid for all values of $x$ for which the limit exists.
\end{definition}




Below is an interactive graph that shows both a tangent line (in red) and a secant line (in green).
Consider the red dot to be the point $(x, f(x)$. Move the red dot to a point of your choosing.
The green dot represents the point $(x+h, f(x+h))$. Move the green dot toward the red dot.
Observe that when the green dot is very close to the red dot (i.e. $h$ is very small), 
the secant line becomes indistinguishable from the tangent line. 
If the green dot is to the right of the red dot, then $h$ is positive, and if the green dot is to the left of 
the red dot, then $h$ is negative.

\includeinteractive{dd01b.js}



%The slope of the tangent line to the graph of the function $y=f(x)$ at the point $(a,f(a))$
%is given by
%\[
%m_{\text{tan}}= f'(a).
%\]


%\geogebra{GNskj5A7}{640}{480}


%Here is an interactive applet that shows both a tangent line and a secant line.
%Use the purple slider at the top left to adjust the point of tangency.
%Then move the blue slider, to adjust the increment, $\Delta x$, in the secant line.
%Notice that when $\Delta x$ is very small, the secant line becomes indistinguishable from the tangent line.
%\geogebra{V3gPpDkp}{640}{480}

We now compute the derivative of several different functions.\\

\section{Using the Definition to Find the Derivative}



\begin{example}[example 1]
We now compute the derivative of $x^2$ and interpret some of its values.
Using the formula for $f'(x)$, we have
\begin{align*}
f'(x) &= \lim_{h \to 0} \frac{f(x+h)-f(x)}{h}\\[5pt] 
&= \lim_{h \to 0} \frac{(x+h)^2- x^2}{h}\\[5pt] 
&= \lim_{h \to 0} \frac{(x^2 + 2xh + h^2)- x^2}{h} \\[5pt] 
&=  \lim_{h \to 0} \frac{ (2xh + h^2)}{h}\\[5pt] 
&=  \lim_{h \to 0} \frac{ h(2x + h)}{h}\\[5pt] 
&=  \lim_{h \to 0} (2x + h)\\
&= 2x.
\end{align*}

Thus the derivative of $x^2$ is $2x$ and this tells use the slope of the tangent line at a given $x$-value.
The process of obtaining the derivative is called \textbf{differentiation},
and the mathematical symbol for the differentiation operator is
\[
\frac{d}{dx}
\]
so that we can write:
\[
\frac{d}{dx}\left( x^2\right) = 2x.
\]
This is essentially the notation used by German polymath, philosopher and co-discoverer of calculus, 
Gottfried Wilhelm Leibniz circa 1700.
Now let's use the derivative to discuss tangent lines and their slopes.

At the point $(-3, 9)$ on the graph of the parabola, $f(x) = x^2$, the slope of the tangent line is given by 
\[
f'(-3) = 2x\big|_{x=-3} = 2(-3) = -6.
\]
That is, we plug the x-coordinate, $x=-3$, into the function $f'(x) = 2x$.\\
The points $(0,0)$ and $(2,4)$ are also on the parabola. 
At the point $(0,0)$, the slope of the tangent line is 
\[
m_{\text{tan}}= f'(0) =2x\big|_{x=0}= 2(0) = 0,
\]
 and at the point $(2, 4)$ the slope is 
\[
m_{\text{tan}}= f'(2) =2x\big|_{x=2}= 2(2) = 4.
\]

The graph below shows the parabola, $y = x^2$, along with the tangent lines at $x = -3, 0,$ and $2$.
\includeinteractive{dd02.js}

%include tangent lines with next update!

\end{example}




\begin{problem}(problem 1)
Use the fact that the derivative of $x^2$ is $2x$,
i.e., 
\[
\frac{d}{dx}\left( x^2 \right) = 2x,
\]
to find the equation of the tangent line to the graph of 
$f(x) = x^2$ at the point where $x = 1/2$.\\
First, we find a point on the tangent line. This point will be the point of tangency to the graph of $y = x^2$
at $x = 1/2$. Since $(1/2)^2 = 1/4$, the coordinates of this point are $(1/2,\; \answer{1/4}\;)$.
\begin{hint}
The derivative gives the slope of the tangent line
\end{hint}
\begin{hint}
The slope is $m_{\text{tan}} = f'(1/2) = (2x)\big|_{x=1/2}$
\end{hint}
The slope of the tangent line is $\answer{1}$.\\
The equation of the tangent line is $y = \; \answer{x - 1/4}$.
\begin{hint}
The point slope form of the equation of a line is given by
\[
 y - y_1 = m(x - x_1).
 \]
 \end{hint}
 
\end{problem}


%\begin{center}
%\begin{tikzpicture}
%\begin{axis}[axis lines = center, title={A Jump Discontinuity at $x=-1$}]
%\addplot[domain=-3:-1.02, 
%    samples=100, color=blue]{x+3};
%\addplot[smooth,mark=o,blue] plot coordinates {(-1,2)};
%\addplot[domain=-0.99:1, 
%    samples=100, color=blue]{-x^2 + 2};
%\addplot[smooth,mark=o,blue] plot coordinates {(-1,1)};
%\end{axis}
%\end{tikzpicture}
%\end{center}


% include figure of y = x^2 with the 3 tangent lines.






\begin{example}[example 2]
Find the derivative, $f'(x)$, for the function $f(x) = x^2 - 5x + 7$.  \\
Step 1. Compute $f(x+h)$. \\
Substitute $(x+h)$ into the formula for $f(x)$ in the place of $x$:  
\begin{align*}
f(x+h) &= (x+h)^2 - 5(x+h) + 7 \\
       &= x^2 + 2xh + h^2 - 5x -5h + 7.			
\end{align*}

Step 2. Compute the numerator of the difference quotient, $f(x+h) - f(x)$.\\
Using the result of step 1, we have
\[
f(x+h) - f(x) = \left(x^2 + 2xh + h^2 - 5x -5h + 7\right) - \left(x^2 - 5x + 7\right) 
\]
\[
= x^2 + 2xh + h^2 - 5x - 5h + 7 -x^2 + 5x - 7 = 2xh + h^2 - 5h.
\]

Step 3. Divide by $h$ to obtain the difference quotient. \\
Using the result of step 2,
\[
\frac{f(x+h) - f(x)}{h}	= \frac{2xh + h^2 - 5h}{h}	= \frac{h(2x + h - 5)}{h} = 2x + h -5.
\]

Step 4. Compute the limit of the difference quotient. \\
Using step 3, we have
\[
\lim_{h \to 0} \frac{f(x+h) - f(x)}{h}	= \lim_{h \to 0} (2x + h -5) = 2x - 5.
\]

Thus, the derivative of $x^2 - 5x + 7$ is $2x -5$, symbolized, $f'(x) =2x-5$,
or
\[
\frac{d}{dx} \left(x^2 - 5x + 7\right) = 2x-5.
\]
\end{example}



In general, if y = f(x) then the notations
\[
y' , \; \frac{dy}{dx},\; \frac{d}{dx} f(x) \; \text{and} \; f'(x),
\]
are all valid ways to express the derivative.
Furthermore, the following are valid ways to express the derivative evaluated at a point $(a, f(a))$, as when finding the slope of a tangent line:
\[
y'(a) = y' \big|_{x=a} = \frac{dy}{dx}\Big|_{x=a} = \frac{d}{dx} f(x)\Big|_{x=a} = f'(a).
\]


\begin{problem}(problem 2a)
Find the derivative of the function $f(x) = x^2 - 3x + 5$ using the definition of the derivative,
\[
f'(x) = \lim_{h \to 0} \frac{f(x+h) - f(x)}{h}.
\]

Step 1. Compute $f(x + h)$:
\begin{multipleChoice}
\choice{$x^2 + 2xh + h^2 - 3x - 3h - 5$}
\choice{$x^2 + 2xh + h^2 - 3x + 3h + 5$}
\choice[correct]{$x^2 + 2xh + h^2 - 3x - 3h + 5$}
\choice{$x^2 + 2xh + h^2 + 3x - 3h + 5$}
\end{multipleChoice}


Step 2. Compute the numerator of the difference quotient: $f(x+h) - f(x) =$
\begin{multipleChoice}
\choice{$2xh - h^2 - 3h$}
\choice{$2xh + h^2 + 3h$}
\choice{$2xh - h^2 + 3h$}
\choice[correct]{$2xh + h^2 - 3h$}
\end{multipleChoice}


Step 3. Divide by $h$ to obtain the difference quotient:
\[
\frac{f(x+h) - f(x)}{h} = 
\]
\begin{multipleChoice}
\choice{$2x - h - 3$}
\choice{$2x + h + 3$}
\choice{$2x - h + 3$}
\choice[correct]{$2x + h - 3$}
\end{multipleChoice}

Step 4. Take the limit as $h \to 0$. The derivative is
\[
f'(x) = \lim_{h \to 0} \frac{f(x+h) - f(x)}{h} = \; \answer{2x-3}.
\]
\end{problem} 


\begin{problem}(problem 2b)
Use the fact that the derivative of $x^2 -3x + 5$ is $2x-3$,
i.e., 
\[
\frac{d}{dx}\left( x^2 - 3x + 5 \right) = 2x-3,
\]
to find the slope of the tangent line to the graph of 
$f(x) = x^2 - 3x + 5$ at the point where $x = 1$.\\
\begin{hint}
The derivative gives the slope of the tangent line
\end{hint}
\begin{hint}
The slope is $m_{\text{tan}} = f'(1) = (2x-3)\big|_{x=1}$
\end{hint}
The slope is $\answer{-1}$
\end{problem}

\begin{problem}(problem 2c)
Use the answer to the previous problem to find the equation of the tangent line to the graph of 
\[
f(x) = x^2 - 3x + 5 \ \ \text{at} \ \ x=1.
\]
\begin{hint}
Use the point slope form: $y-y_1 = m(x-x_1)$
\end{hint}
\begin{hint}
The point $(x_1,y_1)$ is $(1, f(1))$
\end{hint}
\begin{hint}
Solve for $y$
\end{hint}
The equation of the tangent line is \ $y = \answer{-x+4}$
\end{problem}




\begin{problem}(problem 2d)
Find the derivative of the function $f(x) = 3x^2 - 2x -4$ using the definition of the derivative,
\[
f'(x) = \lim_{h \to 0} \frac{f(x+h) - f(x)}{h}.
\]

Step 1. Compute $f(x + h)$:
\begin{multipleChoice}
\choice{$3x^2 + 6xh + 3h^2 + 2x  + 2h - 4 $}
\choice{$3x^2 + 6xh + 3h^2 - 2x  + 2h - 4 $}
\choice[correct]{$3x^2 + 6xh + 3h^2 - 2x  - 2h - 4 $}
\choice{$3x^2 + 6xh + 3h^2 - 2x  - 2h + 4 $}
\end{multipleChoice}


Step 2. Compute the numerator of the difference quotient: $f(x+h) - f(x) =$
\begin{multipleChoice}
\choice{$6xh - 3h^2 - 2h$}
\choice{$6xh + 3h^2 + 2h$}
\choice{$6xh - 3h^2 + 2h$}
\choice[correct]{$6xh + 3h^2 - 2h$}
\end{multipleChoice}


Step 3. Divide by $h$ to obtain the difference quotient:
\[
\frac{f(x+h) - f(x)}{h} = 
\]
\begin{multipleChoice}
\choice{$6x - 3h - 2$}
\choice{$6x + 3h + 2$}
\choice{$6x - 3h + 2$}
\choice[correct]{$6x + 3h - 2$}
\end{multipleChoice}

Step 4. Take the limit as $h \to 0$. The derivative is
\[
f'(x) = \lim_{h \to 0} \frac{f(x+h) - f(x)}{h} = \; \answer{6x-2}.
\]
\end{problem} 


\begin{problem}(problem 2e)
Use the fact that the derivative of $3x^2 -2x - 4$ is $6x-2$,
i.e., 
\[
\frac{d}{dx}\left( 3x^2 - 2x -4  \right) = 6x-2,
\]
to find the slope of the tangent line to the graph of 
$f(x) = 3x^2 - 2x - 4$ at the point where $x = 1$.\\
\begin{hint}
The derivative gives the slope of the tangent line
\end{hint}
\begin{hint}
The slope is $m_{\text{tan}} = f'(1)$
\end{hint}
The slope is $\answer{4}$
\end{problem}

\begin{problem}(problem 2f)
Use the answer to the previous problem to find the equation of the tangent line to the graph of 
\[
f(x) = 3x^2 - 2x - 4 \ \ \text{at} \ \ x=1.
\]
\begin{hint}
Use the point slope form: $y-y_1 = m(x-x_1)$
\end{hint}
\begin{hint}
The point $(x_1,y_1)$ is $(1, f(1))$
\end{hint}
\begin{hint}
Solve for $y$
\end{hint}
The equation of the tangent line is \; $y = \answer{4x - 7}$
\end{problem}


\begin{example}[example 3]
If $f(x) = x^3$ then in order to find $f'(x)$ we first compute $f(x+h)$.

We have
\begin{align*}
f(x+h) &= (x+h)^3  \\
       &= (x+h)(x+h)^2 \\
			&= (x+h)(x^2 + 2xh + h^2)\\
			&= x^3 + 2x^2h + xh^2 + x^2h + 2xh^2 + h^3\\
			&= x^3 + 3x^2h + 3xh^2 + h^3.
\end{align*}
Now we use the definition:
\begin{align*}
f'(x) &= \lim_{h \to 0} \frac{f(x+h)-f(x)}{h}\\[5pt]
&= \lim_{h \to 0} \frac{(x^3 + 3x^2h + 3xh^2 + h^3)- x^3}{h}\\[5pt]
&= \lim_{h \to 0} \frac{3x^2h + 3xh^2 + h^3}{h}\\[5pt]
&= \lim_{h \to 0} \frac{h(3x^2 + 3xh + h^2)}{h}\\[5pt]
&= \lim_{h \to 0} (3x^2 + 3xh + h^2) \\
&= 3x^2.
\end{align*}
To recap, 
\[
\frac{d}{dx}x^3 = 3x^2.
\]
Try to notice a pattern in the form of the derivatives of $x^2$ and $x^3$.

\end{example}




\begin{problem}(problem 3a)
Use the fact that the derivative of $x^3$ is $3x^2$,
i.e., 
\[
\frac{d}{dx}\left( x^3 \right) = 3x^2,
\]
to find the slope of the tangent line to the graph of 
$f(x) = x^3$ at the point where $x = 2$.\\
\begin{hint}
The derivative gives the slope of the tangent line
\end{hint}
\begin{hint}
The slope is $m_{\text{tan}} = f'(2)$
\end{hint}
The slope is $\answer{12}$
\end{problem}




\begin{problem}(problem 3b)
Use the answer to the previous problem to find the equation of the tangent line to the graph of 
\[
f(x) = x^3 \ \ \text{at} \ \ x=2.
\]
\begin{hint}
Use the point slope form: $y-y_1 = m(x-x_1)$
\end{hint}
\begin{hint}
The point $(x_1,y_1)$ is $(2, f(2))$
\end{hint}
\begin{hint}
Solve for $y$
\end{hint}
The equation of the tangent line is \; $y = \answer{12x-16}$
\end{problem}




\begin{example}[example 4]
If $f(x) = \sqrt x$ then
\begin{align*}
f'(x) &= \lim_{h \to 0} \frac{f(x+h)-f(x)}{h}\\[5pt]
&= \lim_{h \to 0} \frac{\sqrt{x+h}- \sqrt x}{h}\\[5pt]
&= \lim_{h \to 0} \frac{\sqrt{x+h}- \sqrt x}{h} \cdot \frac{\sqrt{x+h}+ \sqrt x}{\sqrt{x+h}+ \sqrt x} \\[5pt]
&= \lim_{h \to 0} \frac{(x+h) - (x)}{h(\sqrt{x+h}+ \sqrt x)}\\[5pt]
&= \lim_{h \to 0} \frac{h}{h(\sqrt{x+h}+ \sqrt x)}\\[5pt]
&= \lim_{h \to 0} \frac{1}{\sqrt{x+h}+ \sqrt x}\\[5pt]
&=  \frac{1}{2 \sqrt x}.
\end{align*}
To recap, 
\[
\frac{d}{dx}\sqrt{x} = \frac{1}{2\sqrt{x}},
\]
which in the notation of exponents, where $\sqrt x = x^{1/2}$, we can write
\[
\frac{d}{dx}x^{1/2} = \frac12 x^{-1/2}.
\]

\end{example}





\begin{problem}(problem 4a)
Use the fact that the derivative of $\sqrt{x}$ is $\frac{1}{2\sqrt{x}}$,
i.e., 
\[
\frac{d}{dx}\sqrt{x} = \frac{1}{2\sqrt{x}},
\]
to find the slope of the tangent line to the graph of 
$f(x) = \sqrt{x}$ at the point where $x = 4$.\\
\begin{hint}
The derivative gives the slope of the tangent line
\end{hint}
\begin{hint}
The slope is $m_{\text{tan}} = f'(4)$
\end{hint}
The slope is $\answer{1/4}$
\end{problem}




\begin{problem}(problem 4b)
Use the answer to the previous problem to find the equation of the tangent line to the graph of 
\[
f(x) = \sqrt{x} \ \ \text{at} \ \ x=4.
\]
\begin{hint}
Use the point slope form: $y-y_1 = m(x-x_1)$
\end{hint}
\begin{hint}
The point $(x_1,y_1)$ is $(4, f(4))$
\end{hint}
\begin{hint}
Solve for $y$
\end{hint}
The equation of the tangent line is \ $y = \answer{x/4 +1}$
\end{problem}




\begin{example}[example 5]
If $f(x) = \sqrt{2x+1}$ then
\begin{align*}
f(x+h) &= \sqrt{2(x+h)+1} \\
       &= \sqrt{2x +2h+1}.			
\end{align*}
Now for the derivative:
\begin{align*}
f'(x) &= \lim_{h \to 0} \frac{f(x+h)-f(x)}{h}\\[5pt]
&= \lim_{h \to 0} \frac{\sqrt{2x+2h+1}- \sqrt{2x+1}}{h}\\[5pt]
&= \lim_{h \to 0} \frac{\sqrt{2x+2h+1}- \sqrt{2x+1}}{h} \cdot
\frac{\sqrt{2x+2h+1}+ \sqrt{2x+1}}{\sqrt{2x+2h+1}+ \sqrt{2x+1}} \\[5pt]
&= \lim_{h \to 0} \frac{(2x+2h+1) - (2x+1)}{h(\sqrt{2x+2h+1}+ \sqrt{2x+1})}\\[5pt]
&= \lim_{h \to 0} \frac{2h}{h\sqrt{2x+2h+1}+ \sqrt{2x+1}}\\[5pt]
&= \lim_{h \to 0} \frac{2}{\sqrt{2x+2h+1}+ \sqrt{2x+1}}\\[5pt]
&=  \frac{2}{\sqrt{2x+1}+ \sqrt{2x+1}}\\[5pt]
&=  \frac{2}{2\sqrt{2x+1}}\\[5pt]
&=  \displaystyle{\frac{1}{\sqrt{2x+1}}}.
\end{align*}
In the notation of Leibniz, 
\[
\frac{d}{dx}\sqrt{2x+1}= \frac{1}{\sqrt{2x+1}}.
\]

\end{example}





\begin{problem}(problem 5a)
Use the fact that the derivative of $\sqrt{2x+1}$ is $\frac{1}{\sqrt{2x+1}}$,
i.e., 
\[
\frac{d}{dx}\sqrt{2x+1} = \frac{1}{\sqrt{2x+1}},
\]
to find the slope of the tangent line to the graph of 
$f(x) = \sqrt{2x+1}$ at the point where $x = 4$.\\
\begin{hint}
The derivative gives the slope of the tangent line
\end{hint}
\begin{hint}
The slope is $m_{\text{tan}} = f'(4)$
\end{hint}
The slope is $\answer{1/3}$
\end{problem}





\begin{problem}(problem 5b)
Use the answer to the previous problem to find the equation of the tangent line to the graph of 
\[
f(x) = \sqrt{2x+1} \ \ \text{at} \ \ x=4.
\]
\begin{hint}
Use the point slope form: $y-y_1 = m(x-x_1)$
\end{hint}
\begin{hint}
The point $(x_1,y_1)$ is $(4, f(4))$
\end{hint}
\begin{hint}
Solve for $y$
\end{hint}
The equation of the tangent line is \ $y = \answer{x/3 + 5/3}$
\end{problem}

\begin{problem}(problem 5c)
Find the derivative of the function $f(x) = \sqrt{2x + 5}$ using the definition of the derivative,
\[
f'(x) = \lim_{h \to 0} \frac{f(x+h) - f(x)}{h}.
\]

Step 1. Compute $f(x + h)$:
\begin{multipleChoice}
\choice[correct]{$\sqrt{2x + 2h + 5}$}
\choice{$\sqrt{2x + h + 5}$}
\choice{$\sqrt{2x + 5} + h$}
\choice{$\sqrt{2x + 5} + 2h $}
\end{multipleChoice}


Step 2. Compute the numerator of the difference quotient, $f(x+h) - f(x)$:
\begin{multipleChoice}
\choice{$\sqrt{2x + h + 5}  - \sqrt{2x + 5}$}
\choice{$\sqrt{2x + 5} + h- \sqrt{2x + 5}$}
\choice{$\sqrt{2x + 5} + 2h - \sqrt{2x + 5}$}
\choice[correct]{$\sqrt{2x + 2h + 5} - \sqrt{2x + 5}$}
\end{multipleChoice}


Step 3. Divide by $h$ to obtain the difference quotient:
\[
\frac{f(x+h) - f(x)}{h} = 
\]
(multiply by the conjugate radical and cancel $h$)
\begin{multipleChoice}
\choice[correct]{$\frac{2}{\sqrt{2x + 2h + 5}  + \sqrt{2x + 5}}$}
\choice{$\frac{2}{\sqrt{2x + 2h + 5}  - \sqrt{2x + 5}}$}
\choice{$\frac{1}{\sqrt{2x + 2h + 5}  + \sqrt{2x + 5}}$}
\choice{$\frac{1}{\sqrt{2x + 2h + 5}  - \sqrt{2x + 5}}$}
\end{multipleChoice}

Step 4. Take the limit as $h \to 0$. The derivative is
\[
f'(x) = \lim_{h \to 0} \frac{f(x+h) - f(x)}{h} = \; \answer{1/\sqrt{2x+5}}.
\]
\end{problem} 


\begin{problem}(problem 5d)
Use the fact that the derivative of $\sqrt{2x + 5}$ is $1/\sqrt{2x+5}$,
i.e., 
\[
\frac{d}{dx}\sqrt{2x+5} = \frac{1}{\sqrt{2x+5}},
\]
to find the slope of the tangent line to the graph of 
$f(x) = \sqrt{2x+5}$ at the point where $x = 2$.\\
\begin{hint}
The derivative gives the slope of the tangent line
\end{hint}
\begin{hint}
The slope is $m_{\text{tan}} = f'(1)$
\end{hint}
The slope is $\answer{1/3}$
\end{problem}

\begin{problem}(problem 5e)
Use the answer to the previous problem to find the equation of the tangent line to the graph of 
\[
f(x) = \sqrt{2x+5} \ \ \text{at} \ \ x=2.
\]
\begin{hint}
Use the point slope form: $y-y_1 = m(x-x_1)$
\end{hint}
\begin{hint}
The point $(x_1,y_1)$ is $(2, f(2))$
\end{hint}
\begin{hint}
Solve for $y$
\end{hint}
The equation of the tangent line is \; $y = \answer{x/3 + 7/3}$
\end{problem}




\begin{example}[example 6]
If $f(x) = \displaystyle{\frac{1}{x}}$ then
%\begin{center}
%$\begin{aligned}
\begin{align*}
f'(x) &= \lim_{h \to 0} \frac{f(x+h)-f(x)}{h}\\[5pt]
&= \lim_{h \to 0} \frac{\frac{1}{x+h}- \frac{1}{x}}{h}\\[5pt]
&= \lim_{h \to 0} \frac{\frac{(x) - (x+h)}{x(x+h)}}{h} \\[5pt]
&=  \lim_{h \to 0} \frac{-h}{x(x+h)}\cdot \frac{1}{h}\\[5pt]
&= \lim_{h \to 0} \frac{-1}{x(x+h)} \\[5pt]
&= -\frac{1}{x^2}.
%\end{aligned}$
%\end{center}
\end{align*}
Using the differential operator, $d/dx$, we can write:
\[
 \frac{d}{dx}\left({\frac{1}{x}}\right) = -\frac{1}{x^2}.
\]
\end{example}




\begin{problem} %problem #5a
Use the fact that the derivative of $\frac{1}{x}$ is $-\frac{1}{x^2}$,
i.e., 
\[
\frac{d}{dx}\left(\frac{1}{x}\right) = -\frac{1}{x^2},
\]
to find the slope of the tangent line to the graph of 
$f(x) = \frac{1}{x}$ at the point where $x = 2$.\\
\begin{hint}
The derivative gives the slope of the tangent line
\end{hint}
\begin{hint}
The slope is $m_{\text{tan}} = f'(2)$
\end{hint}
The slope is $\answer{-1/4}$
\end{problem}




\begin{problem} %problem #5b
Use the answer to the previous problem to find the equation of the tangent line to the graph of 
\[
f(x) = \frac{1}{x} \ \ \text{at} \  \ x=2.
\]
\begin{hint}
Use the point slope form: $y-y_1 = m(x-x_1)$
\end{hint}
\begin{hint}
The point $(x_1,y_1)$ is $(2, f(2))$
\end{hint}
\begin{hint}
Solve for $y$
\end{hint}
The equation of the tangent line is \ $y = \answer{-x/4 + 1}$
\end{problem}



\begin{example}[example 7]
If $f(x) = \displaystyle{\frac{2}{x+3}}$ then

\begin{align*}
f'(x) &= \lim_{h \to 0} \frac{f(x+h)-f(x)}{h}\\[5pt]
&= \lim_{h \to 0} \frac{\frac{2}{x+h +3}- \frac{2}{x+3}}{h}\\[5pt]
&= \lim_{h \to 0} \frac{\frac{2(x+3) - 2(x+h+3)}{(x+h+3)(x+3)}}{h} \\[5pt]
&= \lim_{h \to 0} \frac{\frac{2x+6 - 2x-2h-6}{(x+h+3)(x+3)}}{h} \\[5pt]
&=  \lim_{h \to 0} \frac{-2h}{(x+h+3)(x+3)}\cdot \frac{1}{h}\\[5pt]
&= \lim_{h \to 0} \frac{-2}{(x+h+3)(x+3)} \\[5pt]
&= -\frac{2}{(x+3)^2}.
\end{align*}
To recap, the derivative of $\displaystyle{\frac{2}{x+3}}$ is $\displaystyle{-\frac{2}{(x+3)^2}}$.
\end{example}




\begin{problem} %problem #6a
Use the fact that the derivative of $\frac{2}{x+3}$ is $-\frac{2}{(x+3)^2}$,
i.e., 
\[
\frac{d}{dx}\left(\frac{2}{x+3}\right) = -\frac{2}{(x+3)^2},
\]
to find the slope of the tangent line to the graph of 
$f(x) = \frac{2}{x+3}$ at the point where $x = -2$.\\
\begin{hint}
The derivative gives the slope of the tangent line
\end{hint}
\begin{hint}
The slope is $m_{\text{tan}} = f'(-2)$
\end{hint}
The slope is $\answer{-2}$
\end{problem}




\begin{problem} %problem #6b
Use the answer to the previous problem to find the equation of the tangent line to the graph of 
\[
f(x) = \frac{2}{x+3} \  \  \text{at} \  \ x=-2.
\]

\begin{hint}
Use the point slope form: $y-y_1 = m(x-x_1)$
\end{hint}
\begin{hint}
The point $(x_1,y_1)$ is $(-2, f(-2))$
\end{hint}
\begin{hint}
Solve for $y$
\end{hint}
The equation of the tangent line is \  $y = \answer{-2x-2}$
\end{problem}






\section{Video Lessons}

\begin{center}
\begin{foldable}
\unfoldable{Here are two detailed, lecture style videos on the derivative:}
\youtube{jkLA6rvZVJo}
\youtube{XJ56sfFBltY}
\end{foldable}
\end{center}


\end{document}


The next two examples us the special limits:

\[\lim_{h \to 0} \frac{\cos(h) -1}{h}=0 \;\;\mbox{and} \;\; \lim_{h \to 0}\frac{\sin(h)}{h} = 1.\]


\begin{example}[example 8]
If $f(x) = \sin(x)$ then
\begin{center}
$\begin{aligned}
%\begin{align*}
f'(x) &= \lim_{h \to 0} \frac{f(x+h)-f(x)}{h} \\[5pt]
&= \lim_{h \to 0} \frac{\sin(x+h) - \sin(x)}{h}\\[5pt]
&=  \lim_{h \to 0} \frac{\sin(x)\cos(h) + \cos(x)\sin(h) - \sin(x)}{h}\\[5pt]
&=  \lim_{h \to 0} \frac{\sin(x)[\cos(h) -1] + \cos(x)\sin(h)}{h}\\[5pt]
&=  \lim_{h \to 0} \frac{\sin(x)[\cos(h) -1]}{h} + \frac{\cos(x)\sin(h)}{h}\\[5pt]
&=  \sin(x) \cdot 0 + \cos(x) \cdot 1 \\[5pt]
&= \cos(x).
%\end{align*}
\end{aligned}$
\end{center}
This can be expressed simply as
\[
\frac{d}{dx}\sin(x) = \cos(x),
\]
or in terms of any other independent variable, such as the common radian unit, $\theta$:
\[
\frac{d}{d\theta}\sin(\theta) = \cos(\theta).
\]
\end{example}




\begin{problem} %problem #1a
Use the fact that the derivative of $y = \sin(x)$ is \ $y' = \cos(x)$,
i.e., 
\[
\frac{d}{dx}\sin(x) = \cos(x),
\]
to find the slope of the tangent line to the graph of 
$f(x) = \sin(x)$ at the point where $x = 0$.\\
\begin{hint}
The derivative gives the slope of the tangent line
\end{hint}
\begin{hint}
The slope is $m_{\text{tan}} = f'(0)$
\end{hint}
The slope is $\answer{1}$
\end{problem}

\begin{problem} %problem #1b
Use the answer to the previous problem to find the equation of the tangent line to the graph of 
\[
f(x) = \sin(x) \ \ \text{at} \ \ x=0.
\]
\begin{hint}
Use the point slope form: $y-y_1 = m(x-x_1)$
\end{hint}
\begin{hint}
The point $(x_1,y_1)$ is $(0, f(0))$
\end{hint}
\begin{hint}
Solve for $y$
\end{hint}
The equation of the tangent line is \ $y = \answer{x}$
\end{problem}





\begin{example}[example 9]
If $f(x) = \cos(x)$ then\\
\begin{center}
$\begin{aligned}
f'(x) &= \lim_{h \to 0} \frac{f(x+h)-f(x)}{h} \\[5pt]
&= \lim_{h \to 0} \frac{\cos(x+h) - \cos(x)}{h}\\[5pt]
&=  \lim_{h \to 0} \frac{\cos(x)\cos(h) - \sin(x)\sin(h) - \cos(x)}{h}\\[5pt]
&=  \lim_{h \to 0} \frac{\cos(x)[\cos(h) -1] - \sin(x)\sin(h)}{h}\\[5pt]
&=  \lim_{h \to 0} \frac{\cos(x)[\cos(h) -1]}{h} - \frac{\sin(x)\sin(h)}{h}\\[5pt]
&=   \cos(x) \cdot 0 - \sin(x)\cdot 1\\[5pt]
&= -\sin(x).\\[5pt]
\end{aligned}$
\end{center}
Using other variables, this can be written as
\[
\frac{d}{dt} \cos(t) = -\sin(t) \\
\text{or}\\
\frac{d}{du} \cos(u) = -\sin(u).
\]

\end{example}




\begin{problem} %problem #8a
Use the fact that the derivative of $\cos(x)$ is $-\sin(x)$,
i.e., 
\[
\frac{d}{dx}\cos(x) = -\sin(x),
\]
to find the slope of the tangent line to the graph of 
$f(x) = \cos(x)$ at the point where $x = \pi$.\\
\begin{hint}
The derivative gives the slope of the tangent line
\end{hint}
\begin{hint}
The slope is $m_{\text{tan}} = f'(\pi)$
\end{hint}
The slope is $\answer{0}$
\end{problem}




\begin{problem} %problem #8b
Use the answer to the previous problem to find the equation of the tangent line to the graph of 
\[
f(x) = \cos(x) \ \ \text{at} \ \ x=\pi.
\]
\begin{hint}
Use the point slope form: $y-y_1 = m(x-x_1)$
\end{hint}
\begin{hint}
The point $(x_1,y_1)$ is $(\pi, f(\pi))$
\end{hint}
\begin{hint}
Solve for $y$
\end{hint}
The equation of the tangent line is \ $y = \answer{-1}$
\end{problem}



The next example uses the special limit 

\[\lim_{h \to 0} \frac{e^h-1}{h} = 1.\]



\begin{example}[example 10]
If $f(x) = e^x$ then\\[-15pt]
\begin{center}
$\begin{aligned}
f'(x) &= \lim_{h \to 0} \frac{f(x+h)-f(x)}{h} \\[5pt]
&= \lim_{h \to 0}\frac{e^{x+h}-e^x}{h}\\[5pt]
&= \lim_{h \to 0} \frac{e^x e^h-e^x}{h}\\[5pt]
&= \lim_{h \to 0} \frac{e^x (e^h-1)}{h}\\[5pt]
&= e^x \cdot 1 = e^x.\\[-5pt]
\end{aligned}$
\end{center}
Here we encounter the fascinating fact that the rate of change of the natural exponential function
is equal to the function itself, symbolized by the system
\[
y=e^x\\
y' = e^x,
\]
so that 
\[
y' = y.
\]

\end{example}




\begin{problem} %problem #9a
Use the fact that the derivative of $e^x$ is itself,
i.e., 
\[
\frac{d}{dx}e^x=e^x,
\]
to find the slope of the tangent line to the graph of 
$f(x) = e^x$ at the point where $x = 0$.\\
\begin{hint}
The derivative gives the slope of the tangent line
\end{hint}
\begin{hint}
The slope is $m_{\text{tan}} = f'(0)$
\end{hint}
The slope is $\answer{1}$
\end{problem}





\begin{problem} %problem #9b
Use the answer to the previous problem to find the equation of the tangent line to the graph of 
\[
f(x) = e^x \ \ \text{at} \ \ x=0.
\]
\begin{hint}
Use the point slope form: $y-y_1 = m(x-x_1)$
\end{hint}
\begin{hint}
The point $(x_1,y_1)$ is $(0, f(0))$
\end{hint}
\begin{hint}
Solve for $y$
\end{hint}
The equation of the tangent line is \ $y = \answer{x+1}$
\end{problem}



The next example uses the special limit 

\[\lim_{k \to 0} \frac{\ln(1 + k)}{k} = 1.\]



\begin{example}[example 11]
If $f(x) = \ln(x)$ then\\[10pt]
\begin{center}
$\begin{aligned}
f'(x) &= \lim_{h \to 0} \frac{f(x+h)-f(x)}{h}\\[5pt]
&= \lim_{h \to 0}\frac{\ln(x+h)-\ln(x)}{h}\\[5pt]
&= \lim_{h \to 0} \frac{\ln(\frac{x+h}{x})}{h}\\[5pt]
&= \lim_{h \to 0}\frac{\ln(1 + \frac{h}{x})}{h}\\[5pt]
&= \lim_{k \to 0} \frac{\ln(1 + k)}{kx} \\[5pt]
&= \frac{1}{x}.
\end{aligned}$
\end{center}

Note that we made the substitution $k =\frac{h}{x}$ so that $h = kx$ and  also note 
that $h\to 0$ is equivalent to $k\to 0$.
To recap, 
\[
\frac{d}{dx} \ln(x) = \frac{1}{x}
\]
which gives a memorable mathematical relationship between the \textbf{transcendental} natural logarithm function 
the \textbf{rational} reciprocal function.

\end{example}




\begin{problem} %problem #10a
Use the fact that the derivative of $\ln(x)$ is $\frac{1}{x}$,
i.e., 
\[
\frac{d}{dx}\ln(x) = \frac{1}{x},
\]
to find the slope of the tangent line to the graph of 
$f(x) = \ln(x)$ at the point where $x = 1$.\\
\begin{hint}
The derivative gives the slope of the tangent line
\end{hint}
\begin{hint}
The slope is $m_{\text{tan}} = f'(1)$
\end{hint}
The slope is $\answer{1}$
\end{problem}




\begin{problem} %problem #10b
Use the answer to the previous problem to find the equation of the tangent line to the graph of 
\[
f(x) = \ln(x) \ \ \text{at} \ \ x=1.
\]
\begin{hint}
Use the point slope form: $y-y_1 = m(x-x_1)$
\end{hint}
\begin{hint}
The point $(x_1,y_1)$ is $(1, f(1))$
\end{hint}
\begin{hint}
Solve for $y$
\end{hint}
The equation of the tangent line is \ $y = \answer{x-1}$
\end{problem}






\end{document}




\begin{align*}
f'(x) &= \lim_{h \to 0} \frac{f(x+h)-f(x)}{h} \\[5pt] 
&= \lim_{h \to 0} \frac{(x^2 + 2xh + h^2 - 5x - 5h +7)- (x^2 - 5x + 7)}{h}\\[5pt]
&= \lim_{h \to 0} \frac{ 2xh + h^2 - 5h }{h} \\[5pt]
&= \lim_{h \to 0} \frac{ h(2x + h -5)}{h}\\[5pt]
&= \lim_{h \to 0} (2x + h -5)\\
&= 2x-5.
\end{align*}
