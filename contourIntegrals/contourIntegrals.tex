\documentclass[handout]{ximera}

%% You can put user macros here
%% However, you cannot make new environments



\newcommand{\ffrac}[2]{\frac{\text{\footnotesize $#1$}}{\text{\footnotesize $#2$}}}
\newcommand{\vasymptote}[2][]{
    \draw [densely dashed,#1] ({rel axis cs:0,0} -| {axis cs:#2,0}) -- ({rel axis cs:0,1} -| {axis cs:#2,0});
}


\graphicspath{{./}{firstExample/}}

\usepackage{amsmath}
\usepackage{amssymb}
\usepackage{array}
\usepackage[makeroom]{cancel} %% for strike outs
\usepackage{pgffor} %% required for integral for loops
\usepackage{tikz}
\usepackage{tikz-cd}
\usepackage{tkz-euclide}
\usetikzlibrary{shapes.multipart}


\usetkzobj{all}
\tikzstyle geometryDiagrams=[ultra thick,color=blue!50!black]


\usetikzlibrary{arrows}
\tikzset{>=stealth,commutative diagrams/.cd,
  arrow style=tikz,diagrams={>=stealth}} %% cool arrow head
\tikzset{shorten <>/.style={ shorten >=#1, shorten <=#1 } } %% allows shorter vectors

\usetikzlibrary{backgrounds} %% for boxes around graphs
\usetikzlibrary{shapes,positioning}  %% Clouds and stars
\usetikzlibrary{matrix} %% for matrix
\usepgfplotslibrary{polar} %% for polar plots
\usepgfplotslibrary{fillbetween} %% to shade area between curves in TikZ



%\usepackage[width=4.375in, height=7.0in, top=1.0in, papersize={5.5in,8.5in}]{geometry}
%\usepackage[pdftex]{graphicx}
%\usepackage{tipa}
%\usepackage{txfonts}
%\usepackage{textcomp}
%\usepackage{amsthm}
%\usepackage{xy}
%\usepackage{fancyhdr}
%\usepackage{xcolor}
%\usepackage{mathtools} %% for pretty underbrace % Breaks Ximera
%\usepackage{multicol}



\newcommand{\RR}{\mathbb R}
\newcommand{\R}{\mathbb R}
\newcommand{\C}{\mathbb C}
\newcommand{\N}{\mathbb N}
\newcommand{\Z}{\mathbb Z}
\newcommand{\dis}{\displaystyle}
%\renewcommand{\d}{\,d\!}
\renewcommand{\d}{\mathop{}\!d}
\newcommand{\dd}[2][]{\frac{\d #1}{\d #2}}
\newcommand{\pp}[2][]{\frac{\partial #1}{\partial #2}}
\renewcommand{\l}{\ell}
\newcommand{\ddx}{\frac{d}{\d x}}

\newcommand{\zeroOverZero}{\ensuremath{\boldsymbol{\tfrac{0}{0}}}}
\newcommand{\inftyOverInfty}{\ensuremath{\boldsymbol{\tfrac{\infty}{\infty}}}}
\newcommand{\zeroOverInfty}{\ensuremath{\boldsymbol{\tfrac{0}{\infty}}}}
\newcommand{\zeroTimesInfty}{\ensuremath{\small\boldsymbol{0\cdot \infty}}}
\newcommand{\inftyMinusInfty}{\ensuremath{\small\boldsymbol{\infty - \infty}}}
\newcommand{\oneToInfty}{\ensuremath{\boldsymbol{1^\infty}}}
\newcommand{\zeroToZero}{\ensuremath{\boldsymbol{0^0}}}
\newcommand{\inftyToZero}{\ensuremath{\boldsymbol{\infty^0}}}


\newcommand{\numOverZero}{\ensuremath{\boldsymbol{\tfrac{\#}{0}}}}
\newcommand{\dfn}{\textbf}
%\newcommand{\unit}{\,\mathrm}
\newcommand{\unit}{\mathop{}\!\mathrm}
%\newcommand{\eval}[1]{\bigg[ #1 \bigg]}
\newcommand{\eval}[1]{ #1 \bigg|}
\newcommand{\seq}[1]{\left( #1 \right)}
\renewcommand{\epsilon}{\varepsilon}
\renewcommand{\iff}{\Leftrightarrow}

\DeclareMathOperator{\arccot}{arccot}
\DeclareMathOperator{\arcsec}{arcsec}
\DeclareMathOperator{\arccsc}{arccsc}
\DeclareMathOperator{\si}{Si}
\DeclareMathOperator{\proj}{proj}
\DeclareMathOperator{\scal}{scal}
\DeclareMathOperator{\cis}{cis}
\DeclareMathOperator{\Arg}{Arg}
%\DeclareMathOperator{\arg}{arg}
\DeclareMathOperator{\Rep}{Re}
\DeclareMathOperator{\Imp}{Im}
\DeclareMathOperator{\sech}{sech}
\DeclareMathOperator{\csch}{csch}
\DeclareMathOperator{\Log}{Log}

\newcommand{\tightoverset}[2]{% for arrow vec
  \mathop{#2}\limits^{\vbox to -.5ex{\kern-0.75ex\hbox{$#1$}\vss}}}
\newcommand{\arrowvec}{\overrightarrow}
\renewcommand{\vec}{\mathbf}
\newcommand{\veci}{{\boldsymbol{\hat{\imath}}}}
\newcommand{\vecj}{{\boldsymbol{\hat{\jmath}}}}
\newcommand{\veck}{{\boldsymbol{\hat{k}}}}
\newcommand{\vecl}{\boldsymbol{\l}}
\newcommand{\utan}{\vec{\hat{t}}}
\newcommand{\unormal}{\vec{\hat{n}}}
\newcommand{\ubinormal}{\vec{\hat{b}}}

\newcommand{\dotp}{\bullet}
\newcommand{\cross}{\boldsymbol\times}
\newcommand{\grad}{\boldsymbol\nabla}
\newcommand{\divergence}{\grad\dotp}
\newcommand{\curl}{\grad\cross}
%% Simple horiz vectors
\renewcommand{\vector}[1]{\left\langle #1\right\rangle}


\pgfplotsset{compat=1.13}

\outcome{Compute complex integrals}

\title{5.1 Contour Integrals}

\begin{document}

\begin{abstract}
We compute integrals of complex functions along contours.
\end{abstract}

\maketitle


Let $C$ be a contour parameterized by $\gamma(t) = x(t) + iy(t), \; a\leq t \leq b$ and let $f(z)$ be a complex function defined along $C$.
Then the integral of $f$ along $C$ is defined by
\[
\int_C f(z) dz = \int_a^b f(\gamma(t)) \gamma'(t) dt
\]

\begin{example}[example 1]
Compute $\dis \int_C \overline{z} \, dz$ where $C$ is the line segment from $-1$ to $1+i$.\\
We first parameterize the line segment. Since the equation of the line in the $xy$-plane is $y = \frac12x + \frac12$,
one possible parametrization of the line segment is
\[
\gamma(t) = x(t) + iy(t) = t + i\left(\frac{t}{2} + \frac12\right), \; -1 \leq t \leq 1
\]
The derivative of $\gamma(t)$ is 
\[
\gamma'(t) = 1 + \frac{i}{2}
\]
Next, $f(z) = \overline{z} = x-iy$ and hence
\[
f(\gamma(t)) = x(t) - iy(t) = t - i\left(\frac{t}{2} + \frac12\right)
\]
We are now ready to compute the contour integral:
\begin{align*}
\int_C \overline{z} \, dz &= \int_{-1}^1 \left[t - i\left(\frac{t}{2} + \frac12\right)\right]\cdot \left[1 + \frac{i}{2}\right] \, dt\\[6pt]
&= \int_{-1}^1 \left[t + \frac12\left(\frac{t}{2} + \frac12\right)\right] + i\left[-\left(\frac{t}{2} +\frac12\right) + \frac{t}{2}\right] \, dt\\[6pt]
&= \int_{-1}^1 \left[\frac54 t + \frac14\right]+i\left[-\frac12\right] \, dt\\[6pt]
&=\int_{-1}^1 \left[\frac54 t + \frac14\right] \, dt - i\int_{-1}^1  \frac12 \, dt\\[6pt]
                         &= \frac12 - i
\end{align*}
\end{example}

\begin{remark}
A line segment from $z_1$ to $z_2$ can be parameterized by $\gamma(t) = z_1 + t(z_2 - z_1), \, 0\leq t \leq 1$.
Since $(1+i) - (-1) = 2 + i$, the contour $C$ in the previous example can thus also be parametrized by
\[
\gamma(t) = -1 + t(2+i), \, 0 \leq t \leq 1
\]
\end{remark}

\begin{problem}(problem 1a)
Use the parametrization given in the remark above to compute the integral in example 1.
\end{problem}

\begin{problem}(problem 1b)
Let $C$ be the line segment from $0$ to $1+i$. Compute \\
$\dis \int_C \overline{z} \, dz = \answer{1} $\\
\end{problem}

\begin{problem}(problem 1c)
Let $C= C(0, 1)$ be the unit circle traversed in the counter-clockwise direction. Compute \\
$\dis \int_C \overline{z} \, dz = \answer{2\pi i} $\\
\begin{hint}
To parametrize the circle $C(z_0, r)$ traversed in the counter-clockwise direction, use
\[
\gamma(t) = z_0 + re^{it}, \, 0 \leq t \leq 2\pi
\]
\end{hint}
\end{problem}


\begin{proposition}
If the contour $C$ has two different parametrizations, $\gamma_1(t)\,$ with  $\, a_1 \leq t \leq b_1$ and $\gamma_2(t)\,$ with $ \, a_2 \leq t \leq b_2$, then
\[
\int_C f(z) \, dz = \int_{a_1}^{b_1} f(\gamma_1(t)) \gamma_1'(t) \, dt = \int_{a_2}^{b_2} f(\gamma_2(t)) \gamma_2'(t) \, dt
\]
That is, the value of a contour integral is independent of the parametrization used to describe the contour.
\end{proposition}

What happens if we reverse the direction of a contour , $C$?


If $C$ is a contour parameterized by $\gamma(t) = x(t) + iy(t), \, a \leq t \leq b$ then the contour $-C$ which represents the 
same points as $C$ but traced in the opposite 
direction has parametrization $\zeta(s) = x(a+b-s) + iy(a+b-s), \, a \leq s \leq b$.


\begin{proposition}
Suppose $f$ is continuous along a contour $C$ %parameterized by $C: \gamma(t) = x(t) + iy(t), \, a \leq t \leq b$
and suppose the contour $-C$ represents the same points as $C$ traced in the opposite direction, 
then 
\[
\int_{-C} f(z) \, dz = -\int_C f(z) \, dz
\]
\end{proposition}
\begin{proof}
Using the substitution $t = a+b-s$, we have
\begin{align*}
\int_{-C} f(z) \, dz &= \int_a^b f(\zeta(s)) \zeta'(s) \, ds \\
&= \int_a^b f\big(x(a+b-s) + iy(a+b-s)\big) \big[x'(a+b-s)(-1)+ iy'(a+b-s)(-1)\big]\, ds\\
                     &= \int_b^a f\big(x(t)+iy(t)\big) \big[x'(t)+ iy'(t)\big] \, dt\\
                     &= -\int_a^b f(\gamma(t)) \gamma'(t) \, dt\\
                   &= - \int_C f(z) \, dz
\end{align*}
\end{proof}

\begin{example}[example 2]
Let $C$ be the semi-circle parametrized by $\gamma(t) = 2e^{it}, \, 0\leq t \leq \pi$. Compute
\[
\int_C \frac{1}{z} \, dz \quad \text{and} \quad \int_{-C }\frac{1}{z} \, dz
\]
and verify the the above proposition.\\
Note that $\gamma'(t) = 2ie^{it}$. We begin with the integral along $C$:
\[
\int_C \frac{1}{z} \, dz = \int_0^\pi \frac{1}{\gamma(t)} \gamma'(t) \, dt = \int_0^\pi \frac{2ie^{it}}{2e^{it}} \, dt
                        =\int_0^\pi i \, dt \,= \, it\, \bigg|_0^\pi \, = \pi i
\]
Next we integrate along $-C$.  To do this we parametrize $-C$ as $\zeta(t) = 2e^{i(\pi - t)}, \, 0\leq t \leq \pi$.
Note that $\zeta'(t) = -2ie^{i(\pi - t)}$. Thus
\[
\int_{-C} \frac{1}{z} \, dz = \int_0^\pi \frac{1}{\zeta(t)} \zeta'(t) \, dt = \int_0^\pi \frac{-2ie^{i(\pi-t)}}{2e^{i(\pi -t)}} \, dt
                        =\int_0^\pi -i \, dt
                        = -it \bigg|_0^\pi
                        = -\pi i
\]
From our calculations, we see that $\dis \int_C \frac{1}{z} \, dz = -\int_{-C} \frac{1}{z} \, dz$
thereby verifying the conclusion of the previous proposition.
\end{example}


\begin{problem}(problem 2a)
Let $C= C(0, 1)$ be the unit circle traversed in the counter-clockwise direction. Compute \\
$\dis \int_C \frac{1}{z} \, dz = \answer{2\pi i} $\\
\begin{hint}
To parametrize the circle $C(z_0, r)$ traversed in the counter-clockwise direction, use
\[
\gamma(t) = z_0 + re^{it}, \, 0 \leq t \leq 2\pi
\]
\end{hint}
\end{problem}

\begin{problem}(problem 2b)
Let $C= C(i, 2)$ be the circle centered at $i$ of radius $2$, traversed in the clockwise direction. Compute \\
$\dis \int_C \frac{1}{z-i} \, dz = \answer{-2\pi i} $\\
\begin{hint}
To parametrize the circle $C(z_0, r)$ traversed in the counter-clockwise direction, use
\[
\gamma(t) = z_0 + re^{it}, \, 0 \leq t \leq 2\pi
\]
\end{hint}
\end{problem}

\begin{problem}(problem 2c)
Let $C= C(z_0, r)$ be the circle centered at $z_0$ of radius $r$, traversed in the counter-clockwise direction. Compute \\
$\dis \int_C \frac{1}{z-z_0} \, dz = \answer{2\pi i} $\\
\begin{hint}
To parametrize the circle $C(z_0, r)$ traversed in the counter-clockwise direction, use
\[
\gamma(t) = z_0 + re^{it}, \, 0 \leq t \leq 2\pi
\]
\end{hint}
\end{problem}

Compare your answers to problems 2b and 2c. Notice that they verify the proposition.

\begin{problem}(problem 2d)
Let $C= C(z_0, r)$ be the circle centered at $z_0$ of radius $r$, traversed in the counter-clockwise direction. Compute \\
$\dis \int_C \frac{1}{(z-z_0)^2} \, dz = \answer{0} $\\
\begin{hint}
To parametrize the circle $C(z_0, r)$ traversed in the counter-clockwise direction, use
\[
\gamma(t) = z_0 + re^{it}, \, 0 \leq t \leq 2\pi
\]
\end{hint}
\end{problem}

\begin{proposition}
Suppose $f(t) = u(t) + iv(t)$ is a continuous function of the real variable $t$. Then
\[
\left| \int_a^b f(t) \, dt \right| \leq \int_a^b \left| f(t) \right| \, dt
\]
\end{proposition}

\begin{proposition}[$ML$-formula]
Let $C$ be a contour and suppose $f$ is continuous along $C$. Then
\[
\left| \int_C f(z) \, dz \right| \leq ML
\]
where $M$ is the maximum value of $|f(z)|$ along $C$ and $L$ is the length of $C$. 
\end{proposition}   

\begin{proof}
Suppose $C$ is parametrized by $\gamma(t), \, a \leq t \leq b$. Since $|f(z)| \leq M$ for all $z \in C$, the previous proposition implies
\[
\left| \int_C f(z) \, dz \right| = \left| \int_a^b f(\gamma(t))\gamma'(t) \, dt \right| \leq  \int_a^b \left|f(\gamma(t))\right| \cdot \left|\gamma'(t)\right| \, dt \leq M \int_a^b \left|\gamma'(t)\right| \, dt
\]
But this last integral is exactly $L$, the length of the curve $C$ and hence
\[
\left| \int_C f(z) \, dz \right| \leq M \int_a^b \left|\gamma'(t)\right| \, dt = ML
\]
\end{proof}

\begin{example}[example 3]
Use the $ML$-formula to estimate the modulus of the integral: $\dis \int_C z^\frac12 \, dz$,
where $C$ is the circle centered at $3+4i$ of radius $4$.\\
The length $L$ of the circle, $C$, is $L = 2\pi \cdot \text{radius} = 8\pi$. 
In polar form, $z^\frac12 = r^\frac12 e^{i\theta/2}$ and hence $|z^\frac12| = r^\frac12 = |z|^\frac12$.
From the triangle inequality, the modulus of any point $z$ on $C=C(3+4i, 4)$ can be estimated by
\[
|z| = |z-(3+4i) + (3+4i)| \leq |z-(3+4i)| + |3+4i| = 4+5 = 9.
\]
Hence for any $z$ on the circle, $|z^\frac12| \leq 9^\frac12 = 3 = M$.
Finally, by the $ML$-formula,
\[
\left| \int_C z^\frac12 \, dz \right| \leq ML = (3)(8\pi) = 24\pi.
\]
\end{example}

\begin{problem}(problem 3)
Use the $ML$-formula to estimate the modulus of the integral: $\dis \int_C e^z \, dz$,
where $C$ is the circle centered at $i$ of radius $2$.\\
\begin{hint}
$|e^z| = e^x$
\end{hint}
The maximum modulus of $e^z$ on $C$ is $M = \answer{e^2}$\\
The length of $C$ is $L = \answer{4\pi}$\\
By the $ML$-formula, $\dis \int_C e^z \, dz \leq \answer{4e^2 \pi}$
\end{problem}
If the function $f(z)$ is analytic at every point on the contour $C$, then the computation of the integral along $C$
can be simplified as long as an anti-derivative of $f$ can be found.

\begin{theorem}[Fundamental Theorem of Calculus]
Suppose $C$ is a contour with initial point $z_1$ and terminal point $z_2$. If $f$ is analytic along $C$ and 
if $F$ is an anti-derivative of $f$ along $C$, i.e., $F' = f$ at every $z \in C$,
then
\[
\int_C f(z) \, dz = F(z_2) -F(z_1)
\]
\end{theorem}

\begin{example}[example 4]
Let $C$ be any contour from $0$ to $i$. Compute $\dis \int_C \cos z \, dz$.\\
Since $\cos z$ is an entire function, it is analytic at every point on $C$. Moreover, since
\[
\frac{d}{dz} \sin z = \cos z \quad \text{for all} \,\, z \in \C
\]
we have an anti-derivative for $\cos z$ for any $z$ on $C$. Thus
\[
\int_C \cos z \, dz = \sin(i) - \sin(0) = \sin(i) = i\sinh(1).
\]
\end{example}

\begin{problem}(problem 4)
Let $C$ be any contour from $1$ to $1+i$ that does not cross the negative real axis or go through the origin. 
Compute $\dis \int_C \frac{1}{z} \, dz$.\\
\[
\int_C \frac{1}{z} \, dz = \answer{\frac12 \ln 2+ i\pi/4}.
\]
\begin{hint}
Use the principal branch of the logarithm, $\Log z$, as the anti-derivative on $C$.
\end{hint}
\end{problem}


\end{document}


                         
                         



                         
















\begin{definition} 
A complex series  is defined as a limit of its sequence partial sums:
\[
\sum_{k=1}^\infty c_k = \lim_{n \to \infty}  \sum_{k=1}^n c_k = \lim_{n \to \infty} S_n
\]
If the sequence $\{S_n\}$ converges, then we say the series converges.  Otherwise, we say that the series diverges. 
\end{definition}

\begin{theorem}[Test for Divergence]
If $\dis \sum_{k=1}^\infty c_k$ converges, then $\dis \lim_{k\to\infty} c_k = 0$
\end{theorem}
\begin{proof}
Note that $c_n = S_n - S_{n-1}$, the difference in partial sums. Since the sequence $\{S_n\}$ converges, given $\epsilon >0$,
there exists a number $N$ and a number $S$ such that if $n>N$ then $|S_n - S| < \epsilon/2$. Hence for $n > N+1$, the triangle inequality gives
\begin{align*}
|c_n| &= |S_n - S_{n-1}| = |(S_n -S) + (S-S_{n-1})| \\
&\leq |S_n -S| + |S-S_{n-1}| < \frac{\epsilon}{2} + \frac{\epsilon}{2} =\epsilon
\end{align*}
Hence, the sequence $\{c_n\}$ converges to zero.
\end{proof}


\begin{example}[example 1] 
Determine whether the complex series $\dis \sum_{k=1}^\infty \frac{1+ik}{k}$ converges.\\
Since, $\dis \lim_{k \to \infty} \frac{1+ik}{k} = i$. Since $i \neq 0$, the Test for Divergence says that the series $\dis \sum_{k=1}^\infty \frac{1+ik}{k}$
does not converge, i.e., it diverges.
\end{example}

\begin{problem}(problem 1) 
Determine whether the complex series $\dis \sum_{k=0}^\infty \frac{k!}{(2i)^k}$ converges.\\
Since, $\dis \lim_{k \to \infty} \left|\frac{k!}{(2i)^k}\right| = \answer{\infty}$,\\
 $\dis \lim_{k \to \infty} \frac{k!}{(2i)^k} \neq \answer{0}$,\\
 the series $\dis \sum_{k=0}^\infty \frac{k!}{(2i)^k}\;\;$ \wordChoice{\choice{converges}\choice[correct]{diverges}}\\
 by the \wordChoice{\choice{Test for Convergence}\choice[correct]{Test for Divergence}}
\end{problem}

\begin{theorem}[Geometric Series]
Let $c \in \C$ and consider the series
\[
\sum_{k=0}^\infty c^k = 1 + c + c^2 + c^3 + \cdots
\]
This series converges if and only if $|c| < 1$.  Moreover, if it converges, its sum is 
$\displaystyle \frac{1}{1-c}$.
\end{theorem}

\begin{remark}
The complex number $c$ in the geometric series $\dis \sum_{k=0}^\infty c^k$ is called the common ratio of the series.
\end{remark}

\begin{proof}
Suppose $|c| < 1$. Then the partial sums have a closed form whose limit can be computed directly.
Since
\[
S_n = \sum_{k=1}^n c^k = 1+c+c^2+\cdots + c^n
\]
we have
\[
cS_n = c+c^2+\cdots + c^{n+1}
\]
Subtracting gives
\[
S_n - cS_n = 1-c^{n+1}
\]
from which we obtain the closed form
\[
S_n = \frac{1-c^{n+1}}{1-c}
\]
Now, since $|c|<1$, we have $c^n \to 0$ as $n \to \infty$ and hence
\[
\sum_{k=0}^\infty c^k = \lim_{n\to \infty} S_n = \lim_{n\to \infty} \frac{1-c^{n+1}}{1-c} = \frac{1}{1-c}
\]
Thus, the geometric series converges and its sum is also known.\\
If $|c| \geq 1$ then $c^n \nrightarrow 0$ as $n \to \infty$ which implies 
that the geometric series diverges (by the Test for Divergence).
\end{proof}


\begin{example}[example 2] 
Determine whether the geometric series converges or diverges:\\
a) $\;\;\dis \sum_{k=0}^\infty \frac{1}{(1+i)^k}\quad$  b) $\;\;\dis \sum_{k=0}^\infty \left(\frac{1+i}{1-i}\right)^k$ \\
Part a) This is a geometric series with common ratio $\dis \frac{1}{1+i}$. The modulus of the common ratio is 
\[
\left|\frac{1}{1+i}\right| = \frac{1}{\sqrt 2} <1
\]
Hence the geometric series converges and its sum is
\[
\sum_{k=0}^\infty \frac{1}{(1+i)^k} = \frac{1}{1-\frac{1}{1+i}} = \frac{1+i}{i} = 1-i
\]
Part b) This is a geometric series with common ratio $\dis \frac{1+i}{1-i}$. The modulus of the common ratio is 
\[
\left|\frac{1+i}{1-i}\right| = \frac{\sqrt 2}{\sqrt 2} = 1 \nless 1
\]
Hence the geometric series diverges.
\end{example}


\begin{problem}(problem 2)
Determine whether the geometric series converges or diverges. If it converges, find its sum: $\;\;\dis \sum_{k=0}^\infty \left(\frac{2i}{2+i}\right)$ \\
The common ratio is $\; \answer{\frac{2i}{2+i}}$\\
The modulus of the common ratio is $\; \answer{\frac{2}{\sqrt 5}}$\\
Is this less than 1? \wordChoice{\choice[correct]{yes}\choice{no}}\\
The geometric series  \wordChoice{\choice[correct]{converges}\choice{diverges}}\\
The sum of the series is $\; \answer{\frac{2+i}{2-i}}$
\end{problem}


\begin{theorem}[Absolute Convergence]
If the series $\dis \sum_{k=1}^\infty |c_k|$ converges, then so does the complex series $\dis \sum_{k=1}^\infty c_k$.
\end{theorem}


\begin{remark}
To prove this result, we will need the Monotone Convergence Theorem which states that a bounded, monotone sequence of real numbers converges.
\end{remark}

\begin{proof}
Suppose $\displaystyle \sum_{k=1}^\infty |c_k|$ converges to the real number $S$, and break the partial sums of  the complex series 
$\displaystyle \sum_{k=1}^\infty c_k$ into their real and imaginary parts:
\[
\sum_{k=1}^n c_k =  \sum_{k=1}^n a_k + i\sum_{k=1}^n b_k
\]
where $a_k, b_k \in \R$ for all $k$.  Let $a_k^+ = a_k$ if $a_k \geq 0$ and let $a_k^+ = 0$ if $a_k < 0$.
Similarly, let $a_k^- = -a_k$ if $a_k < 0$ and let $a_k^- = 0$ if $a_k \geq 0$. Then 
\[
\sum_{k=1}^n a_k = \sum_{k=1}^n a_k^+ - \sum_{k=1}^n a_k^- = A_n^+ - A_n^-
\]
The sequences $\{A_n^+\}$ and $\{A_n^-\}$ are both increasing sequences and since $|a_k| \leq |c_k|$ for all $k$, they are both bounded above by $S$.
Hence, by the Monotone Convergence Theorem, both sequences converge.
Thus, the real series $\dis \sum_{k=1}^n a_k$ converges, to a sum, $A$. Applying the same argument to the partial sums of the imaginary part, 
$\dis \sum_{k=1}^n b_k$, we see that it too converges to a sum, $B$. Finally, we can conclude that $\dis \sum_{k=1}^\infty c_k$ converges to $A + iB$. 
\end{proof}

\begin{example}[example 3]
Show that the complex series converges: $\dis \sum_{k=1}^\infty \frac{2e^i}{(k+ik)^2}$\\
The modulus of $c_k$ is
\[
\left|\frac{2e^i}{(k+ik)^2}\right| = \frac{2}{2k^2} = \frac{1}{k^2}
\]
and the series
\[
\sum_{k=1}^\infty \frac{1}{k^2}
\]
is a convergent $p$-series with $p = 2 > 1$. Moreover the sum of this well known series is $\frac{\pi^2}{6}$.
\end{example}

\begin{problem}(problem 3) 
Determine whether the complex series $\dis \sum_{k=0}^\infty \frac{i^k}{k!}$ converges.\\
The modulus of $c_k$ is $\; \answer{1/k!}$\\
The series $\dis \sum_{k=0}^\infty \frac{1}{k!}\;\;$ \wordChoice{\choice[correct]{converges}\choice{diverges}}\\
The series $\dis \sum_{k=0}^\infty \frac{i^k}{k!}\;\;$ \wordChoice{\choice[correct]{converges}\choice{diverges}}
\end{problem}


\begin{definition}
We define the limit superior of a sequence of real numbers $\{a_n\}, \; \limsup_{n \to \infty} a_n,\;$to be the largest limit of any subsequence of $\{a_n\}$.\
\end{definition}


\begin{remark}
If the sequence $\{a_n\}$ converges to $a$, then $\limsup_{n \to \infty} a_n =a$.
\end{remark}

\begin{remark}
It is possible that $\limsup_{n \to \infty} a_n =\pm\infty$
\end{remark}

\begin{example}[example 4]
Find the limit superior of the sequence $\{c_k\}= \{1, 2, 3, 1, 2, 3, 1, 2, 3, \ldots\}$
The sequence $\{c_k\}$ has subsequences that converge to $1, 2$ and $3$. Since the largest of these limits is $3$, we have
\[
\limsup_{k\to\infty} c_k = 3
\]
\end{example}

\begin{problem}(problem 4)
Find the limit superior of the sequence:\\
a) $\dis\; \{(-1)^k\} \quad \limsup_{k\to\infty} (-1)^k = \; \answer{1}$\\
a) $\dis\; \{c_k\} = \{1, -1, \frac12, -1, \frac13, -1, \ldots\} \quad \limsup_{k\to\infty} c_k = \; \answer{0}$\\
\end{problem}


\begin{theorem}[Root Test]
Let $\displaystyle \sum_{k=1}^\infty c_k$ be a complex series and let 
\[
L = \limsup_{k \to \infty} \sqrt[k]{|c_k|}
\]
Then the series converges if $L <1$ and diverges if $L >1$.
\end{theorem}

\begin{proof}
First, suppose $\dis L = \limsup_{k \to \infty} \sqrt[k]{|c_k|} < 1$. Then there exists a natural number $N$ such that for $k>N 
|c_k| < L^k$.  Hence the series $\dis \sum_{k = N+1}^\infty |c_k|$ converges by direct comparison to the geometric series
$\dis \sum_{k = N+1}^\infty L^k$. Thus, the series $\dis\sum_{k = 1}^\infty c_k$ converges absolutely and is therefore convergent.\\
Next, suppose $\dis L = \limsup_{k \to \infty} \sqrt[k]{|c_k|} > 1$ Then there exists a subsequence of $\{c_k\}$ such that $\sqrt[k]{|c_k|} > 1$
which implies that $|c_k| > 1$ for each of the infinitely many terms in this subsequence. Hence, $c_k \nrightarrow 0$ as $k \to \infty$.
By the Test for Divergence, the series $\displaystyle \sum_{k=1}^\infty c_k$ diverges.
\end{proof}

\begin{remark}
The series $\dis \sum_{k=1}^\infty \frac{1}{k^2}$ and $\dis \sum_{k=1}^\infty \frac{1}{k}$ show that if $L = 1$ in the Root Test,
then no conclusion can be drawn. This is because the first series converges and the second series diverges but $L=1$ in both cases.
\end{remark}

\begin{example}[example 5]
Determine if the series converges or diverges: $\dis \sum_{k=1}^\infty \left(\frac{k+i}{k^2 - i}\right)^k$\\
Computing the $k^{th}$ root of the modulus of $c_k$ gives
\[
\sqrt[k]{|c_k|} = \sqrt[k]{\left|\left(\frac{k+i}{k^2 - i}\right)^k\right|} = \sqrt[k]{\left|\frac{k+i}{k^2 - i}\right|^k} = \sqrt{\frac{k^2 + 1}{k^4 + 1}}
\]
and the limit superior is
\[
\limsup_{k \to \infty} \sqrt[k]{|c_k|} = \limsup_{k \to \infty}  \sqrt{\frac{k^2 + 1}{k^4 + 1}} = 0 <1
\]
so the series converges.
\end{example}


\begin{problem}(problem 5)
Determine if the series converges or diverges: 
\[
\sum_{k=1}^\infty \left(\frac{e^{ik}}{1+i\sqrt k}\right)^k
\]
$\sqrt[k]{|c_k|} = \; \answer{\frac{1}{\sqrt{1+k}}}$\\
The limit superior is $\; \limsup_{k\to\infty}\sqrt[k]{|c_k|}= \; \answer{0}$\\
Is this less than 1? \wordChoice{\choice[correct]{yes}\choice{no}}\\
The series  \wordChoice{\choice[correct]{converges}\choice{diverges}}
\end{problem}

\begin{theorem}[Ratio Test]
Let $\displaystyle \sum_{k=1}^\infty c_k$ be a complex series and let 
\[
L = \limsup_{k \to \infty} \left|\frac{c_{k+1}}{c_k}\right|
\]
Then the series converges if $L <1$ and diverges if $L >1$.
\end{theorem}

\begin{remark} 
If $c_k = 0$ for infinitely many values of $k$, then the ratio test cannot be used.
\end{remark}
\begin{remark} 
The proof of the ratio test is similar to the proof of the root test.
\end{remark}

\begin{example}[example 6]
Determine if the series converges or diverges: 
\[
\sum_{k=0}^\infty \frac{i^k2^k}{k!}
\]
The modulus of the ratio of successive terms is
\[
\left|\frac{i^{k+1}2^{k+1}}{(k+1)!}\cdot \frac{k!}{i^k 2^k}\right| = \frac{2}{k+1}
\]
Taking the limit gives
\[
\limsup_{k \to \infty} \frac{2}{k+1} = 0 < 1
\]
By the Ratio Test, the series converges.
\end{example}

\begin{problem}(problem 6)
Determine if the series converges or diverges: 
\[
\sum_{k=0}^\infty \frac{2e^{ik}}{(2k)!}
\]
$\left|\frac{c_{k+1}}{c_k}\right| = \; \answer{\frac{1}{(k+1)(2k+1)}}$\\
The limit superior is $\; \limsup_{k\to\infty}\left|\frac{c_{k+1}}{c_k}\right|= \; \answer{0}$\\
Is this less than 1? \wordChoice{\choice[correct]{yes}\choice{no}}\\
The series  \wordChoice{\choice[correct]{converges}\choice{diverges}}
\end{problem}


\end{document}


