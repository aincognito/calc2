\documentclass[handout]{ximera}

%% You can put user macros here
%% However, you cannot make new environments



\newcommand{\ffrac}[2]{\frac{\text{\footnotesize $#1$}}{\text{\footnotesize $#2$}}}
\newcommand{\vasymptote}[2][]{
    \draw [densely dashed,#1] ({rel axis cs:0,0} -| {axis cs:#2,0}) -- ({rel axis cs:0,1} -| {axis cs:#2,0});
}


\graphicspath{{./}{firstExample/}}

\usepackage{amsmath}
\usepackage{amssymb}
\usepackage{array}
\usepackage[makeroom]{cancel} %% for strike outs
\usepackage{pgffor} %% required for integral for loops
\usepackage{tikz}
\usepackage{tikz-cd}
\usepackage{tkz-euclide}
\usetikzlibrary{shapes.multipart}


\usetkzobj{all}
\tikzstyle geometryDiagrams=[ultra thick,color=blue!50!black]


\usetikzlibrary{arrows}
\tikzset{>=stealth,commutative diagrams/.cd,
  arrow style=tikz,diagrams={>=stealth}} %% cool arrow head
\tikzset{shorten <>/.style={ shorten >=#1, shorten <=#1 } } %% allows shorter vectors

\usetikzlibrary{backgrounds} %% for boxes around graphs
\usetikzlibrary{shapes,positioning}  %% Clouds and stars
\usetikzlibrary{matrix} %% for matrix
\usepgfplotslibrary{polar} %% for polar plots
\usepgfplotslibrary{fillbetween} %% to shade area between curves in TikZ



%\usepackage[width=4.375in, height=7.0in, top=1.0in, papersize={5.5in,8.5in}]{geometry}
%\usepackage[pdftex]{graphicx}
%\usepackage{tipa}
%\usepackage{txfonts}
%\usepackage{textcomp}
%\usepackage{amsthm}
%\usepackage{xy}
%\usepackage{fancyhdr}
%\usepackage{xcolor}
%\usepackage{mathtools} %% for pretty underbrace % Breaks Ximera
%\usepackage{multicol}



\newcommand{\RR}{\mathbb R}
\newcommand{\R}{\mathbb R}
\newcommand{\C}{\mathbb C}
\newcommand{\N}{\mathbb N}
\newcommand{\Z}{\mathbb Z}
\newcommand{\dis}{\displaystyle}
%\renewcommand{\d}{\,d\!}
\renewcommand{\d}{\mathop{}\!d}
\newcommand{\dd}[2][]{\frac{\d #1}{\d #2}}
\newcommand{\pp}[2][]{\frac{\partial #1}{\partial #2}}
\renewcommand{\l}{\ell}
\newcommand{\ddx}{\frac{d}{\d x}}

\newcommand{\zeroOverZero}{\ensuremath{\boldsymbol{\tfrac{0}{0}}}}
\newcommand{\inftyOverInfty}{\ensuremath{\boldsymbol{\tfrac{\infty}{\infty}}}}
\newcommand{\zeroOverInfty}{\ensuremath{\boldsymbol{\tfrac{0}{\infty}}}}
\newcommand{\zeroTimesInfty}{\ensuremath{\small\boldsymbol{0\cdot \infty}}}
\newcommand{\inftyMinusInfty}{\ensuremath{\small\boldsymbol{\infty - \infty}}}
\newcommand{\oneToInfty}{\ensuremath{\boldsymbol{1^\infty}}}
\newcommand{\zeroToZero}{\ensuremath{\boldsymbol{0^0}}}
\newcommand{\inftyToZero}{\ensuremath{\boldsymbol{\infty^0}}}


\newcommand{\numOverZero}{\ensuremath{\boldsymbol{\tfrac{\#}{0}}}}
\newcommand{\dfn}{\textbf}
%\newcommand{\unit}{\,\mathrm}
\newcommand{\unit}{\mathop{}\!\mathrm}
%\newcommand{\eval}[1]{\bigg[ #1 \bigg]}
\newcommand{\eval}[1]{ #1 \bigg|}
\newcommand{\seq}[1]{\left( #1 \right)}
\renewcommand{\epsilon}{\varepsilon}
\renewcommand{\iff}{\Leftrightarrow}

\DeclareMathOperator{\arccot}{arccot}
\DeclareMathOperator{\arcsec}{arcsec}
\DeclareMathOperator{\arccsc}{arccsc}
\DeclareMathOperator{\si}{Si}
\DeclareMathOperator{\proj}{proj}
\DeclareMathOperator{\scal}{scal}
\DeclareMathOperator{\cis}{cis}
\DeclareMathOperator{\Arg}{Arg}
%\DeclareMathOperator{\arg}{arg}
\DeclareMathOperator{\Rep}{Re}
\DeclareMathOperator{\Imp}{Im}
\DeclareMathOperator{\sech}{sech}
\DeclareMathOperator{\csch}{csch}
\DeclareMathOperator{\Log}{Log}

\newcommand{\tightoverset}[2]{% for arrow vec
  \mathop{#2}\limits^{\vbox to -.5ex{\kern-0.75ex\hbox{$#1$}\vss}}}
\newcommand{\arrowvec}{\overrightarrow}
\renewcommand{\vec}{\mathbf}
\newcommand{\veci}{{\boldsymbol{\hat{\imath}}}}
\newcommand{\vecj}{{\boldsymbol{\hat{\jmath}}}}
\newcommand{\veck}{{\boldsymbol{\hat{k}}}}
\newcommand{\vecl}{\boldsymbol{\l}}
\newcommand{\utan}{\vec{\hat{t}}}
\newcommand{\unormal}{\vec{\hat{n}}}
\newcommand{\ubinormal}{\vec{\hat{b}}}

\newcommand{\dotp}{\bullet}
\newcommand{\cross}{\boldsymbol\times}
\newcommand{\grad}{\boldsymbol\nabla}
\newcommand{\divergence}{\grad\dotp}
\newcommand{\curl}{\grad\cross}
%% Simple horiz vectors
\renewcommand{\vector}[1]{\left\langle #1\right\rangle}


\outcome{Determine differentiability}

\title{1.11 Differentiability}

\begin{document}

\begin{abstract}
We determine differentiability at a point
\end{abstract}

\maketitle

\begin{center}
\textbf{Differentiability}
\end{center}


\begin{definition}[Differentiability].  The function $f(x)$ is said to be 
\textbf{differentiable} at the point $x= x_0$
if the following limit exists:
\[
\lim_{h\to 0} \frac{f(x_0+h) -f(x_0)}{h}.
\]
We denote the limit by $f'(x_0)$, the derivative of $f$ at $x_0$.

\end{definition}

Geometrically, the derivative $f'(x_0)$ gives us the \it{slope} of a tangent line.  
Conceptually, it represents an \it{instantaneous} rate of change.
In this section, we will explore the three main reasons that a function is \textbf{not} differentiable at a point. These are
discontinuities, corner points, and vertical tangent lines.

\begin{theorem}
If $f(x)$ is differentiable at $x = x_0$, then $f(x)$ is continuous at $x=x_0$, i.e.,
differentiability implies continuity.
\end{theorem}

An immediate consequence of this theorem is that if $f(x)$ is \textbf{not} continuous at $x = x_0$,
then it cannot be differentiable there.  

\begin{corollary}
If $f(x)$ is not continuous at $x = x_0$ then $f(x)$ is not differentiable at $x = x_0$.
\end{corollary}

If the graph of a function has a removable, jump or infinite discontinuity, 
then the function is not differentiable at the corresponding point.

There are two other common reasons that a function might \textbf{not} be differentiable.

\begin{definition}[Corner Point] The function $f(x)$ has a \textbf{corner point} at $x = x_0$ if
the difference quotient has a jump discontinuity at $x = x_0$, i.e.,
\[
\lim_{h\to 0^-} \frac{f(x_0 +h)-f(x_0)}{h} \text{  and  }  \lim_{h\to 0^+} \frac{f(x_0 +h)-f(x_0)}{h}
\]
are both finite, but they are different.

\end{definition}

\begin{example}[example 1]
The function $f(x) = |x|$ has a corner point at $x = 0$.
The left-hand limit, 
\[
\lim_{h\to 0^-} \frac{f(x_0 +h)-f(x_0)}{h} = \lim_{h\to 0^-} \frac{|h|}{h} = -1
\]
whereas the right-hand limit,
\[
\lim_{h\to 0^+} \frac{f(x_0 +h)-f(x_0)}{h} = \lim_{h\to 0^+} \frac{|h|}{h} = 1.
\]

\[
\graph{abs(x)}
\]
\end{example}

The other common occurrence of a point of non-differentiability is at a vertical tangent line.


\begin{definition}[Vertical Tangent Line] The function $f(x)$ has a \textbf{vertical tangent line} at $x = x_0$ if
the vertical line $x = x_0$ is tangent to the graph of $y = f(x)$ at the point $(x_0, f(x_0))$.
\end{definition}

\begin{example}[example 2]
The function $f(x) = \sqrt[3] x$ has a vertical tangent line at $x = 0$.
\[
\graph{x^{1/3}}
\]
\end{example}


\begin{problem}(problem 1)
Below is the graph of $y = f(x)$.  Answer the questions below the graph.  You can zoom in on the graph if you need to.
\[
\graph{x^2}
\]
Is $f(x)$ differentiable at $x = 1$?
\begin{multipleChoice}
\choice[correct]{Yes}
\choice{No}
\end{multipleChoice}
If yes, is $f'(1)$ positive, negative or zero?
\begin{multipleChoice}
\choice[correct]{$f'(1) > 0$}
\choice{$f'(1) < 0$}
\choice{$f'(1) = 0$}
\end{multipleChoice}
\end{problem}




\begin{problem}(problem 2)
Below is the graph of $y = f(x)$.  Answer the questions below the graph.  You can zoom in on the graph if you need to.
\[
\graph{3-abs(x-1)}
\]
Is $f(x)$ differentiable at $x = 1$?
\begin{multipleChoice}
\choice{Yes}
\choice[correct]{No}
\end{multipleChoice}
If no, why not?
\begin{multipleChoice}
\choice{discontinuity at $x = 1$}
\choice[correct]{corner point at $x = 1$}
\choice{vertical tangent line at $x = 1$}
\end{multipleChoice}
\end{problem}



\begin{problem}(problem 3)
Below is the graph of $y = f(x)$.  Answer the questions below the graph.  You can zoom in on the graph if you need to.
\[
\graph{1 - (x-1)^{1/3}}
\]
Is $f(x)$ differentiable at $x = 1$?
\begin{multipleChoice}
\choice{Yes}
\choice[correct]{No}
\end{multipleChoice}
If no, why not?
\begin{multipleChoice}
\choice{discontinuity at $x = 1$}
\choice{corner point at $x = 1$}
\choice[correct]{vertical tangent line at $x = 1$}
\end{multipleChoice}
\end{problem}



\begin{problem}(problem 4)
Below is the graph of $y = f(x)$.  Answer the questions below the graph.  You can zoom in on the graph if you need to.
\[
\graph{(x-2)^2}
\]
Is $f(x)$ differentiable at $x = 1$?
\begin{multipleChoice}
\choice[correct]{Yes}
\choice{No}
\end{multipleChoice}
If yes, is $f'(1)$ positive, negative or zero?
\begin{multipleChoice}
\choice{$f'(1) > 0$}
\choice[correct]{$f'(1) < 0$}
\choice{$f'(1) = 0$}
\end{multipleChoice}
\end{problem}


\begin{problem}(problem 5)
Below is the graph of $y = f(x)$.  Answer the questions below the graph.  You can zoom in on the graph if you need to.
\[
\graph{1/(x-1)}
\]
Is $f(x)$ differentiable at $x = 1$?
\begin{multipleChoice}
\choice{Yes}
\choice[correct]{No}
\end{multipleChoice}
If no, why not?
\begin{multipleChoice}
\choice[correct]{discontinuity at $x = 1$}
\choice{corner point at $x = 1$}
\choice{vertical tangent line at $x = 1$}
\end{multipleChoice}
\end{problem}



\begin{problem}(problem 6)
Below is the graph of $y = f(x)$.  Answer the questions below the graph.  You can zoom in on the graph if you need to.
\[
\graph{1 + (x-1)^2}
\]
Is $f(x)$ differentiable at $x = 1$?
\begin{multipleChoice}
\choice[correct]{Yes}
\choice{No}
\end{multipleChoice}
If yes, is $f'(1)$ positive, negative or zero?
\begin{multipleChoice}
\choice{$f'(1) > 0$}
\choice{$f'(1) < 0$}
\choice[correct]{$f'(1) = 0$}
\end{multipleChoice}
\end{problem}


\end{document}












\[
\graph{2^x \left \{-2<x<2\right \}}
\]

\[
\graph{2^x \left \{-2<x<2\right \}, x \left \{2<x<4 \right \}}
\]


\[
\graph[color:red]{2^x}
\]

\[
\graph[style:dashed]{2^x}
\]
















\begin{problem}
\begin{image}
\begin{tikzpicture}
\begin{axis}[axis lines = center, title={Graph of $y=f(x)$}]
\addplot[domain= -1: 2, color=blue]{2-abs(x-1)};
\end{axis}
\end{tikzpicture}
\end{image}
Is $f(x)$ differentiable at $x = 1$?
\begin{multipleChoice}
\choice{Yes}
\choice[correct]{No}
\end{multipleChoice}
If no, why not?
\begin{multipleChoice}
\choice{discontinuity at $x = 1$}
\choice[correct]{corner point at $x = 1$}
\choice{vertical tangent line at $x = 1$}
\end{multipleChoice}
\end{problem}














\addplot[smooth,mark=o,blue] plot coordinates {(-4,3)};
\addplot[domain=-3:-.16, color=blue]{-1/x};
\addplot[smooth,mark=*,blue] plot coordinates {(-4,-4)};
\addplot[domain=.16:2,color=blue]{-1/x}



\begin{leash}{220}
Is $f(x)$ differentiable at $x = -2$?


Is $f(x)$ differentiable at $x = 0$?
\begin{multipleChoice}
\choice[correct]{Yes}
\choice{No}
\end{multipleChoice}
Is $f'(0)$ positive, negative or zero?
\begin{multipleChoice}
\choice[correct]{$f'(0) > 0$}
\choice{$f'(0) < 0$}
\choice{$f'(0) = 0$}
\end{multipleChoice}


\end{leash}






\[
\lim_{h\to 0^-} \frac{f(x_0 +h)-f(x_0)}{h} \text{  and  }  \lim_{h\to 0^+} \frac{f(x_0 +h)-f(x_0)}{h}}
both exist, but are unequal.
\end{definition}


  


\begin{theorem} If $f(x)$ and $g(x)$ are differentiable functions and $h(x) = f(g(x))$, then
\[
h'(x) = f'(g(x))g'(x).
\]
This important differentiation rule can also be written as 
\[
f(g(x))' = f'(g(x))g'(x)
\]
or, using Leibniz notation, as
\[
\frac{d}{dx}f(g(x)) = f'(g(x))g'(x).
\]
\end{theorem}



In words, the derivative of a composition is the derivative of the outside, with the inside left in,   
times the derivative of the inside. Or, we can shorten the phrasing slightly to simply `the derivative of the outside   
times the derivative of the inside'.



\begin{example} %example #1
Find $h'(x)$ if $h(x) = \sin(2x)$.\\
We write $h(x)$ as a composition: $h(x)=f(g(x))$ with
\[g(x) = 2x   \quad \text{and} \quad  f(x) = \sin(x).\]


To find $h'(x)$ we need $f'(g(x))$ and $g'(x)$. First,
\[g'(x) = 2; \quad \text{next} \]
\[f'(x) = \cos(x), \quad \text{so}\]
\[f'(g(x)) =\cos(g(x)) = \cos(2x).\]
By the chain rule,
\[h'(x) = f'(g(x))g'(x) = \cos(2x) \cdot 2 = 2\cos(2x).\]
\end{example}

\begin{center}
\begin{foldable}
\unfoldable{Here is a video of Example 1}
\youtube{Yy6QXnFlnXs} %vid of example 1
\end{foldable}
\end{center}

\begin{question} %problem #1
  Compute
  \[
  \frac{d}{dx} \sin(5x)
  \]
    \begin{hint}
      The chain rule says:
      \[
      \frac{d}{dx} f(g(x)) = f'(g(x))\cdot g'(x)
      \]
    \end{hint}
    \begin{hint}
      The ``outside'' function is $f(x) = \sin(x)$ and the ``inside''
      function is $g(x) = 5x$.
    \end{hint}
    \begin{hint}
		  Leave the inside in, $f'(g(x))$
		\end{hint}
		\begin{hint}
		  Multiply by the derivative of the inside, $g'(x)$
		\end{hint}
		
		The derivative of $\sin(5x)$ with respect to $x$ is
		 $\answer{5\cos(5x)}$
\end{question}

\begin{example} %example #2
Find $h'(x)$ if $h(x) = e^{5x}$.\\
We write $h(x)$ as a composition: $h(x)=f(g(x))$ with
\[
g(x) = 5x  \quad \text{and} \quad f(x) = e^x.
\]
To find $h'(x)$ we need $f'(g(x))$ and $g'(x)$. First:
\[
g'(x) = 5; \quad \text{next} 
\]
\[
f'(x) = e^x, \quad \text{so}
\]
\[
f'(g(x)) =e^{g(x)} = e^{5x}.
\]
By the chain rule,
\[
h'(x) = f'(g(x))g'(x) = e^{5x} \cdot 5 = 5e^{5x}.
\]
\end{example}


\begin{center}
\begin{foldable}
\unfoldable{Here is a video of the example}
\youtube{BG63VLFnlsA} %vid of example 2
\end{foldable}
\end{center}



\begin{question} %problem #2
  Compute
  \[
  \frac{d}{dx} e^{3x}
  \]
  
    \begin{hint}
      The chain rule says:
      \[
      \frac{d}{dx} f(g(x)) = f'(g(x))\cdot g'(x)
      \]
    \end{hint}
    \begin{hint}
      The outside function is $f(x) = e^x$ and the inside
      function is $g(x) = 3x$.
    \end{hint}
    \begin{hint}
		  Leave the inside in, $f'(g(x))$
		\end{hint}
		\begin{hint}
		  Multiply by the derivative of the inside, $g'(x)$
		\end{hint}
    
		The derivative of $e^{3x}$ with respect to $x$ is
		 $\answer[given]{3e^{3x}}$
		
\end{question}


\begin{example} %example 3
Find $h'(x)$ if $h(x) = \cos(x^2 + 1)$.\\
We write $h(x)$ as a composition: $h(x)=f(g(x))$ with
\[
g(x) = x^2 + 1  \quad \text{and} \quad  f(x) = \cos(x).
\]
To find $h'(x)$ we need $f'(g(x))$ and $g'(x)$. First,
\[
g'(x) = 2x; \quad \text{next}
\]
\[
f'(x) = -\sin(x), \quad \text{so}
\]
\[
f'(g(x))= -\sin(g(x)) = -\sin(x^2 + 1).
\]
By the chain rule,
\[
h'(x) = f'(g(x))g'(x) = -\sin(x^2 + 1) \cdot 2x = -2x\sin(x^2 + 1).
\]

\end{example}


\begin{center}
\begin{foldable}
\unfoldable{Here is a video of the example}
\youtube{W1adUySCZxM} %vid of example 3
\end{foldable}
\end{center}

\begin{question} %problem #3
  Compute
  \[
  \frac{d}{dx} \cos(2-x^3)
  \]
  
    \begin{hint}
      The chain rule says:
      \[
      \frac{d}{dx} f(g(x)) = f'(g(x))\cdot g'(x)
      \]
    \end{hint}
    \begin{hint}
      The outside function is $f(x) = \cos(x)$ and the inside
      function is $g(x) = 2-x^3$.
    \end{hint}
    \begin{hint}
		  Leave the inside in, $f'(g(x))$
		\end{hint}
		\begin{hint}
		  Multiply by the derivative of the inside, $g'(x)$
		\end{hint}
    
		The derivative of $\cos(2-x^3)$ with respect to $x$ is
		 $\answer{3x^2\sin(2-x^3)}$
		
\end{question}



\begin{example} %example #4

Find $h'(x)$ if $h(x) = (x^3 - 4x + 2)^7$.\\
We write $h(x)$ as a composition: $h(x)=f(g(x))$ with 
\[
g(x) = x^3 - 4x + 2  \quad \text{and} \quad  f(x) = x^7.
\]
To find $h'(x)$ we need $f'(g(x))$ and $g'(x)$.  First,
\[
g'(x) =3x^2 - 4; \quad \text{next} 
\]
\[
f'(x) = 7x^6 , \quad \text{so}
\]
\[
f'(g(x)) = 7[g(x)]^6 = 7(x^3 - 4x + 2)^6 .
\]
By the chain rule,
\[
h'(x) = f'(g(x))g'(x) = 7(x^3 + 4x + 2)^6(3x^2 - 4)=7(3x^2 - 4)(x^3 - 4x + 2)^6.
\]
\end{example}

\begin{center}
\begin{foldable}
\unfoldable{Here is a video of the example}
\youtube{FKoN4Ul8XcM}  %vid of example 4
\end{foldable}
\end{center}




\begin{question} %problem #4
  Compute
  \[
  \frac{d}{dx} (x^3 - 4x + 5)^8
  \]
  
    \begin{hint}
      The chain rule says:
      \[
      \frac{d}{dx} f(g(x)) = f'(g(x))\cdot g'(x)
      \]
    \end{hint}
    \begin{hint}
      The outside function is $f(x) = x^8$ and the inside
      function is $x^3 - 4x + 5$.
    \end{hint}
    \begin{hint}
		  Leave the inside in, $f'(g(x))$
		\end{hint}
		\begin{hint}
		  Multiply by the derivative of the inside, $g'(x)$
		\end{hint}
    
		The derivative of $(x^3 - 4x + 5)^8$ with respect to $x$ is
		 $\answer{8(3x^2 - 4)(x^3 - 4x + 5)^7 }$
		
\end{question}




\begin{example} %example #5

Find $h'(x)$ if $h(x) = \sqrt{x^2 + 4}$.\\
We write $h(x)$ as a composition: $h(x)=f(g(x))$ with 
\[
g(x) = x^2 + 4  \quad \text{and} \quad  f(x) = \sqrt x.
\] 
To find $h'(x)$ we need $f'(g(x))$ and $g'(x)$.  First,
\[
g'(x) = 2x; \quad \text{next} 
\]
\[
f'(x) = \frac{1}{2\sqrt x} , \quad \text{so}
\] 
\[
f'(g(x)) = \frac{1}{2\sqrt {g(x)}}=\frac{1}{2\sqrt {x^2 + 4}}.
\]
By the chain rule,
\begin{align*}
h'(x) &= f'(g(x))g'(x)\\
&= \frac{1}{2\sqrt {x^2 + 4}}\cdot 2x\\
&= \frac{2x}{2\sqrt {x^2 + 4}}\\
=\frac{x}{\sqrt {x^2 + 4}}.
\end{align*}
\end{example}

\begin{center}
\begin{foldable}
\unfoldable{Here is a video of the example}
\youtube{bmUe8uKxxos} %vid of example 5
\end{foldable}
\end{center}



\begin{question} %problem #5
  Compute
  \[
  \frac{d}{dx} \sqrt{x^2 - 3}
  \]
  
    \begin{hint}
      The chain rule says:
      \[
      \frac{d}{dx} f(g(x)) = f'(g(x))\cdot g'(x)
      \]
    \end{hint}
    \begin{hint}
      The outside function is $f(x) = \sqrt{x}$ and the inside
      function is $g(x) = x^2 - 3$.
    \end{hint}
    
    \begin{hint}
		  Leave the inside in, $f'(g(x))$
		\end{hint}
		\begin{hint}
		  Multiply by the derivative of the inside, $g'(x)$
		\end{hint}
    
		The derivative of $\sqrt{x^2 - 3}$ with respect to $x$ is
		 $\answer{\frac{x}{\sqrt{x^2 - 3}}}$
		
\end{question}




\begin{example} %example #6

Find $h'(x)$ if $h(x) = \tan(3x)$.\\
We write $h(x)$ as a composition: $h(x)=f(g(x))$ with 
\[
g(x) = 3x  \quad \text{and} \quad  f(x) =\tan(x).
\]
To find $h'(x)$ we need $f'(g(x))$ and $g'(x)$.  First, 
\[
g'(x) =3; \quad \text{next} 
\]
\[
f'(x) =\sec^2(x) , \quad \text{so}
\]
\[
f'(g(x)) = \sec^2(g(x)) = \sec^2(3x).
\]
By the chain rule, 
\[
h'(x) = f'(g(x))g'(x) = \sec^2(3x) \cdot 3= 3\sec^2(3x).
\]
\end{example}


\begin{center}
\begin{foldable}
\unfoldable{Here is a video of the example}
\youtube{1L1z-e7yvRo} %vid of example 6
\end{foldable}
\end{center}


\begin{question} %problem #6a
  Compute
  \[
  \frac{d}{dx} \tan(x^4)
  \]
  
    \begin{hint}
      The chain rule says:
      \[
      \frac{d}{dx} f(g(x)) = f'(g(x))\cdot g'(x)
      \]
    \end{hint}
    \begin{hint}
      The outside function is $f(x) = \tan(x)$ and the inside
      function is $g(x) = x^4$.
    \end{hint}
    \begin{hint}
		  Leave the inside in, $f'(g(x))$
		\end{hint}
		\begin{hint}
		  Multiply by the derivative of the inside, $g'(x)$
		\end{hint}
    
		The derivative of $\tan(x^4)$ with respect to $x$ is
		 $\answer{4x^3 \sec^2(x^4)}$
		
\end{question}

\begin{question} %problem 6b
What is the derivative of $\sec(2x)$?
\begin{multipleChoice}
  \choice{$\sec(2x)\tan(2x)$}
  \choice{$2\sec(x)\tan(x)$}
  \choice{$-2\sec(2x)\tan(2x)$}
  \choice[correct]{$2\sec(2x)\tan(2x)$}
\end{multipleChoice}
\end{question}



\begin{example} %example #7
Find $h'(x)$ if $h(x) = \frac{3}{(2x + 5)^4}$.\\
We write $h(x)$ as a composition: $h(x)=f(g(x))$ with 
\[
g(x) = 2x+ 5  \quad \text{and} \quad  f(x) = \frac{3}{x^4} = 3x^{-4}.
\]
 To find $h'(x)$ we need $f'(g(x))$ and $g'(x)$.  First,
\[g'(x) = 2; \quad \text{next}\]
\[f'(x) = -12x^{-5}, \quad \text{so}\]
\[f'(g(x)) = -12[g(x)]^{-5} = -12(2x+5)^{-5} = -\frac{12}{(2x+5)^5}.\]
By the chain rule,
\[h'(x) = f'(g(x))g'(x) = -\frac{12}{(2x+5)^5}\cdot 2 = -\frac{24}{(2x+5)^5}.\]
\end{example}

\begin{center}
\begin{foldable}
\unfoldable{Here is a video of the example}
\youtube{kddUPQBZJEM} %vid of example 7
\end{foldable}
\end{center}



\begin{question} %problem #7
  Compute
  \[
  \frac{d}{dx} \frac{2}{(3x - 8)^5}
  \]
  
    \begin{hint}
      The chain rule says:
      \[
      \frac{d}{dx} f(g(x)) = f'(g(x))\cdot g'(x)
      \]
    \end{hint}
    \begin{hint}
      The outside function is $f(x) = 2x^{-5}$ and the inside
      function is $g(x) = 3x - 8$.
    \end{hint}
    
    \begin{hint}
		  Leave the inside in, $f'(g(x))$
		\end{hint}
		\begin{hint}
		  Multiply by the derivative of the inside, $g'(x)$
		\end{hint}
    
		The derivative of $\frac{2}{(3x - 8)^5}$ with respect to $x$ is
		 $\answer{\frac{-30}{(3x - 8)^6}}$
		
\end{question}


\begin{example} %example #8
Find $h'(x)$ if $h(x) = \sin(x^3)$.\\
We write $h(x)$ as a composition: $h(x)=f(g(x))$ with 
\[g(x) = x^3  \quad \text{and} \quad  f(x) = \sin(x).\] 
To find $h'(x)$ we need $f'(g(x))$ and $g'(x)$.  First, 
\[g'(x) = 3x^2; \quad \text{next} \]
\[f'(x) =\cos(x) , \quad \text{so}\]
\[ f'(g(x)) = \cos(g(x)) = \cos(x^3).\]
By the chain rule,
\[h'(x) = f'(g(x))g'(x) = \cos(x^3)\cdot 3x^2 = 3x^2\cos(x^3).\]
\end{example}

\begin{center}
\begin{foldable}
\unfoldable{Here is a video of the example}
\youtube{rA8RotbzDJ0} %vid of example 8
\end{foldable}
\end{center}


\begin{question} %problem #8
  Compute
  \[
  \frac{d}{dx} \sin(3x^5)
  \]
  
    \begin{hint}
      The chain rule says:
      \[
      \frac{d}{dx} f(g(x)) = f'(g(x))\cdot g'(x)
      \]
    \end{hint}
    \begin{hint}
      The outside function is $f(x) = \sin(x)$ and the inside
      function is $g(x) = 3x^5$.
    \end{hint}
    \begin{hint}
		  Leave the inside in, $f'(g(x))$
		\end{hint}
		\begin{hint}
		  Multiply by the derivative of the inside, $g'(x)$
		\end{hint}
    
		The derivative of $\sin(3x^5)$ with respect to $x$ is
		 $\answer{\cos(3x^5)\cdot 15x^4}$
		
\end{question}



\begin{example} %example #9
Find $h'(x)$ if $h(x) = \sin^3(x)$.\\
We write $h(x)$ as a composition: $h(x)=f(g(x))$ with 
\[g(x) =\sin(x)   \quad \text{and} \quad  f(x) = x^3.\]
 To find $h'(x)$ we need $f'(g(x))$ and $g'(x)$.  First, 
\[g'(x) =\cos(x); \quad \text{next} \] 
\[f'(x) = 3x^2, \quad \text{so}\] 
\[f'(g(x)) = 3g^2(x) = 3\sin^2(x).\]
By the chain rule,
\[h'(x) = f'(g(x))g'(x) =  3\sin^2(x)\cos(x).\]
\end{example}

\begin{center}
\begin{foldable}
\unfoldable{Here is a video of the example}
\youtube{2xh2g1f5Ysc} %vid of example 9
\end{foldable}
\end{center}
 


\begin{question} %problem #9
  Compute
  \[
  \frac{d}{dx} \sin^4(x)
  \]
  
    \begin{hint}
      The chain rule says:
      \[
      \frac{d}{dx} f(g(x)) = f'(g(x))\cdot g'(x)
      \]
    \end{hint}
    \begin{hint}
      The outside function is $f(x) = x^4$ and the inside
      function is $g(x) = \sin(x)$.
    \end{hint}
    \begin{hint}
		  Leave the inside in, $f'(g(x))$
		\end{hint}
		\begin{hint}
		  Multiply by the derivative of the inside, $g'(x)$
		\end{hint}
    
		The derivative of $\sin^4(x)$ with respect to $x$ is
		 $\answer{4\sin^3(x)\cos(x)}$
		
\end{question}



\begin{example} %example #10
Find $h'(x)$ if $h(x) = e^{-x^2}$.\\
We write $h(x)$ as a composition: $h(x)=f(g(x))$ with 
\[g(x) = -x^2  \quad \text{and} \quad  f(x) =e^x.\]
 To find $h'(x)$ we need $f'(g(x))$ and $g'(x)$. First, 
\[g'(x) = -2x; \quad \text{next} \] 
\[f'(x) = e^x, \quad \text{so} \]
\[f'(g(x)) = e^{g(x)} = e^{-x^2}.\]
By the chain rule,
\[h'(x) = f'(g(x))g'(x) = e^{-x^2} (-2x) \quad \text{or} \quad -2xe^{-x^2}.\]
\end{example}

\begin{center}
\begin{foldable}
\unfoldable{Here is a video of the example}
\youtube{ubKbCULGXDg} %vid of example 10
\end{foldable}
\end{center}
 

\begin{question} %problem #10
  Compute
  \[
  \frac{d}{dx} e^{\sqrt x}
  \]
  
    \begin{hint}
      The chain rule says:
      \[
      \frac{d}{dx} f(g(x)) = f'(g(x))\cdot g'(x)
      \]
    \end{hint}
    \begin{hint}
      The outside function is $f(x) = e^x$ and the inside
      function is $g(x) = \sqrt x$.
    \end{hint}
    \begin{hint}
		  Leave the inside in, $f'(g(x))$
		\end{hint}
		\begin{hint}
		  Multiply by the derivative of the inside, $g'(x)$
		\end{hint}
    
		The derivative of $e^{\sqrt x}$ with respect to $x$ is
		 $\answer{\frac{e^{\sqrt x}}{2\sqrt x}}$
		
\end{question}



\begin{example} %example #11
Find $h'(x)$ if $h(x) = \ln(x^4 +1)$.\\
We write $h(x)$ as a composition: $h(x)=f(g(x))$ with 
\[g(x) = x^4 + 1  \quad \text{and} \quad  f(x) = \ln(x).\]
 To find $h'(x)$ we need $f'(g(x))$ and $g'(x)$.  First, 
\[g'(x) = 4x^3; \quad \text{next} \] 
\[f'(x) = \frac{1}{x}, \quad \text{so}\]
\[ f'(g(x)) =\frac{1}{g(x)} = \frac{1}{x^4 + 1} .\]
By the chain rule,
\[h'(x) = f'(g(x))g'(x) = \frac{1}{x^4 + 1} \cdot 4x^3 = \frac{4x^3}{x^4 + 1}.\]
\end{example}

\begin{center}
\begin{foldable}
\unfoldable{Here is a video of the example}
\youtube{7FOx5dq88NI} %vid of example 11
\end{foldable}
\end{center}
 


\begin{question} %problem #11
  Compute
  \[
  \frac{d}{dx} \ln(x^3 -2)
  \]
  
    \begin{hint}
      The chain rule says:
      \[
      \frac{d}{dx} f(g(x)) = f'(g(x))\cdot g'(x)
      \]
    \end{hint}
    \begin{hint}
      The outside function is $f(x) = \ln(x)$ and the inside
      function is $g(x) = x^3 - 2$.
    \end{hint}
    \begin{hint}
		  Leave the inside in, $f'(g(x))$
		\end{hint}
		\begin{hint}
		  Multiply by the derivative of the inside, $g'(x)$
		\end{hint}
    
		The derivative of $\ln(x^3 - 2)$ with respect to $x$ is
		 $\answer{\frac{3x^2}{x^3 - 2}}$
		
\end{question}



\begin{example} %example #12
Find $h'(x)$ if $h(x) = \ln^5(x)$.\\
We write $h(x)$ as a composition: $h(x)=f(g(x))$ with 
\[g(x) = \ln(x)  \quad \text{and} \quad  f(x) = x^5.\]
 To find $h'(x)$ we need $f'(g(x))$ and $g'(x)$.  First, 
\[g'(x) = \frac{1}{x}; \quad \text{next} \] 
\[f'(x) = 5x^4, \quad \text{so} \]
\[f'(g(x)) = 5g^4(x) = 5\ln^4(x).\]
By the chain rule,
\[h'(x) = f'(g(x))g'(x) = 5\ln^4(x)\cdot \frac{1}{x}  = \frac{5\ln^4(x)}{x}.\]
\end{example}

\begin{center}
\begin{foldable}
\unfoldable{Here is a video of the example}
\youtube{5Fb6cph_7jE}  %vid of example 12
\end{foldable}
\end{center}


\begin{question} %problem #12
  Compute
  \[
  \frac{d}{dx} \ln^3(x)
  \]
  
    \begin{hint}
      The chain rule says:
      \[
      \frac{d}{dx} f(g(x)) = f'(g(x))\cdot g'(x)
      \]
    \end{hint}
    \begin{hint}
      The outside function is $f(x) = x^3$ and the inside
      function is $g(x) = \ln(x)$.
    \end{hint}
    \begin{hint}
		  Leave the inside in, $f'(g(x))$
		\end{hint}
		\begin{hint}
		  Multiply by the derivative of the inside, $g'(x)$
		\end{hint}
    
		The derivative of $\ln^3(x)$ with respect to $x$ is
		 $\answer{\frac{3\ln^2(x)}{x}}$
		
\end{question}



\begin{example} %example #13
Find $h'(x)$ if $h(x) = \ln(\sin(x))$.\\
We write $h(x)$ as a composition: $h(x)=f(g(x))$ with 
\[g(x) = \sin(x)  \quad \text{and} \quad  f(x) = \ln(x).\]
 To find $h'(x)$ we need $f'(g(x))$ and $g'(x)$.  First,
\[g'(x) = \cos(x); \quad \text{next} \]
\[f'(x) = \frac{1}{x}, \quad \text{so}\]
\[f'(g(x)) = \frac{1}{g(x)} = \frac{1}{\sin(x)}.\]
By the chain rule,
\[h'(x) = f'(g(x))g'(x) = \frac{1}{\sin(x)} \cdot \cos(x) = \frac{\cos(x)}{\sin(x)} = \cot(x).\]
\end{example}

\begin{center}
\begin{foldable}
\unfoldable{Here is a video of the example}
\youtube{sVXyra9ElKo}  %vid of example 13
\end{foldable}
\end{center}


\begin{question} %problem #13a
  Compute
  \[
  \frac{d}{dx} \ln(\sec(x))
  \]
  
    \begin{hint}
      The chain rule says:
      \[
      \frac{d}{dx} f(g(x)) = f'(g(x))\cdot g'(x)
      \]
    \end{hint}
    \begin{hint}
      The outside function is $f(x) = \ln(x)$ and the inside
      function is $g(x) = \sec(x)$.
    \end{hint}
    \begin{hint}
		  Leave the inside in, $f'(g(x))$
		\end{hint}
		\begin{hint}
		  Multiply by the derivative of the inside, $g'(x)$
		\end{hint}
    
		The derivative of $\ln(\sec(x))$ with respect to $x$ is
		 $\answer{\tan(x)}$
		
\end{question}

\begin{question} %problem #13b
  Compute
  \[
  \frac{d}{dx} \cos(\ln(x))
  \]
  
    \begin{hint}
      The chain rule says:
      \[
      \frac{d}{dx} f(g(x)) = f'(g(x))\cdot g'(x)
      \]
    \end{hint}
    \begin{hint}
      The outside function is $f(x) = \cos(x)$ and the inside
      function is $g(x) = \ln(x)$.
    \end{hint}
    \begin{hint}
		  Leave the inside in, $f'(g(x))$
		\end{hint}
		\begin{hint}
		  Multiply by the derivative of the inside, $g'(x)$
		\end{hint}
    
		The derivative of $\cos(\ln(x))$ with respect to $x$ is
		 $\answer{\frac{-\sin(\ln(x))}{x}}$
		
\end{question}


\begin{example} %example #14
Find $h'(x)$ if $h(x) = \tan^{-1}(4x)$.\\
We write $h(x)$ as a composition: $h(x)=f(g(x))$ with 
\[g(x) = 4x  \quad \text{and} \quad  f(x) = \tan^{-1}(x).\]
 To find $h'(x)$ we need $f'(g(x))$ and $g'(x)$.  First, 
\[g'(x) = 4; \quad \text{next} \] 
\[f'(x) = \frac{1}{1+x^2} , \quad \text{so}\]
\[ f'(g(x)) = \frac{1}{1+g^2(x)} =\frac{1}{1+(4x)^2}=\frac{1}{1+16x^2}.\]
By the chain rule,
\[h'(x) = f'(g(x))g'(x) = \frac{1}{1+16x^2} \cdot 4 = \frac{4}{1+16x^2}.\]
\end{example}

\begin{center}
\begin{foldable}
\unfoldable{Here is a video of the example}
\youtube{-0ZwkfU9Xec}  %vid of example 14
\end{foldable}
\end{center}


\begin{question} %problem #14
  Compute
  \[
  \frac{d}{dx} \tan^{-1}(3x)
  \]
  
    \begin{hint}
      The chain rule says:
      \[
      \frac{d}{dx} f(g(x)) = f'(g(x))\cdot g'(x)
      \]
    \end{hint}
    \begin{hint}
      The outside function is $f(x) = \tan^{-1}(x)$ and the inside
      function is $g(x) = 3x$.
    \end{hint}
    \begin{hint}
		  Leave the inside in, $f'(g(x))$
		\end{hint}
		\begin{hint}
		  Multiply by the derivative of the inside, $g'(x)$
		\end{hint}
    
		The derivative of $\tan^{-1}(3x)$ with respect to $x$ is
		 $\answer{\frac{3}{1+(3x)^2}}$
		
\end{question}


\begin{center}
\begin{foldable}
\unfoldable{Here are some detailed, lecture style videos on the chain rule:}
\youtube{WkiZcDAwlYE}
\youtube{DiXsWYXXI00}
\youtube{hLAV8r9ARzU}
\end{foldable}
\end{center}





\end{document}


\begin{example} %example #15
Find $h'(x)$ if $h(x) = x^{\sin(x)}$.\\
We will use the fact that the exponential and logarithm functions are inverses,
\[e^{\ln(x)} = x,\]
and the exponent property of logarithms, 
\[\ln(x^n) = n\ln(x),\]
to rewrite $h(x)$.  We have 
\[h(x) = x^{\sin(x)} = e^{\ln(x^{\sin(x)})} = e^{\sin(x)\ln(x)}\]
and we can now compute $h'(x)$ using a combination of the chain rule and product rule.
We can write $h(x)$ as a composition, $f(g(x))$ with 
\[g(x) = \sin(x)\ln(x) \quad \text{and} \quad f(x) = e^x.\]
Then to find $g'(x)$ we us the product rule and we get $g'(x) = \frac{\sin(x)}{x} + \cos(x)\ln(x)$.
Next $f'(x) = e^x$ and 
hence $f'(g(x)) = e^{g(x)} = e^{\sin(x)\ln(x)} = x^{\sin(x)}$.
We can then conclude $h'(x) = f'(g(x))g'(x) = x^{\sin(x)} \left[ \frac{\sin(x)}{x} + \cos(x)\ln(x)\right]$.
\end{example}

%more question formats below













%\begin{verbatim}
\begin{question}
What is your favorite color?
\begin{multipleChoice}
\choice[correct]{Rainbow}
\choice{Blue}
\choice{Green}
\choice{Red}
\end{multipleChoice}
\begin{freeResponse}
Hello
\end{freeResponse}
\end{question}
%\end{verbatim}





\begin{question}
  Which one will you choose?
  \begin{multipleChoice}
    \choice[correct]{I'm correct.}
    \choice{I'm wrong.}
    \choice{I'm wrong too.}
  \end{multipleChoice}
\end{question}


\begin{question}
  Which one will you choose?
  \begin{selectAll}
    \choice[correct]{I'm correct.}
    \choice{I'm wrong.}
    \choice[correct]{I'm also correct.}
    \choice{I'm wrong too.}
  \end{selectAll}
\end{question}


\begin{freeResponse}
What is the chain rule used for?
\end{freeResponse}
