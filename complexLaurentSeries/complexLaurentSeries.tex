\documentclass[handout]{ximera}

%% You can put user macros here
%% However, you cannot make new environments



\newcommand{\ffrac}[2]{\frac{\text{\footnotesize $#1$}}{\text{\footnotesize $#2$}}}
\newcommand{\vasymptote}[2][]{
    \draw [densely dashed,#1] ({rel axis cs:0,0} -| {axis cs:#2,0}) -- ({rel axis cs:0,1} -| {axis cs:#2,0});
}


\graphicspath{{./}{firstExample/}}

\usepackage{amsmath}
\usepackage{amssymb}
\usepackage{array}
\usepackage[makeroom]{cancel} %% for strike outs
\usepackage{pgffor} %% required for integral for loops
\usepackage{tikz}
\usepackage{tikz-cd}
\usepackage{tkz-euclide}
\usetikzlibrary{shapes.multipart}


\usetkzobj{all}
\tikzstyle geometryDiagrams=[ultra thick,color=blue!50!black]


\usetikzlibrary{arrows}
\tikzset{>=stealth,commutative diagrams/.cd,
  arrow style=tikz,diagrams={>=stealth}} %% cool arrow head
\tikzset{shorten <>/.style={ shorten >=#1, shorten <=#1 } } %% allows shorter vectors

\usetikzlibrary{backgrounds} %% for boxes around graphs
\usetikzlibrary{shapes,positioning}  %% Clouds and stars
\usetikzlibrary{matrix} %% for matrix
\usepgfplotslibrary{polar} %% for polar plots
\usepgfplotslibrary{fillbetween} %% to shade area between curves in TikZ



%\usepackage[width=4.375in, height=7.0in, top=1.0in, papersize={5.5in,8.5in}]{geometry}
%\usepackage[pdftex]{graphicx}
%\usepackage{tipa}
%\usepackage{txfonts}
%\usepackage{textcomp}
%\usepackage{amsthm}
%\usepackage{xy}
%\usepackage{fancyhdr}
%\usepackage{xcolor}
%\usepackage{mathtools} %% for pretty underbrace % Breaks Ximera
%\usepackage{multicol}



\newcommand{\RR}{\mathbb R}
\newcommand{\R}{\mathbb R}
\newcommand{\C}{\mathbb C}
\newcommand{\N}{\mathbb N}
\newcommand{\Z}{\mathbb Z}
\newcommand{\dis}{\displaystyle}
%\renewcommand{\d}{\,d\!}
\renewcommand{\d}{\mathop{}\!d}
\newcommand{\dd}[2][]{\frac{\d #1}{\d #2}}
\newcommand{\pp}[2][]{\frac{\partial #1}{\partial #2}}
\renewcommand{\l}{\ell}
\newcommand{\ddx}{\frac{d}{\d x}}

\newcommand{\zeroOverZero}{\ensuremath{\boldsymbol{\tfrac{0}{0}}}}
\newcommand{\inftyOverInfty}{\ensuremath{\boldsymbol{\tfrac{\infty}{\infty}}}}
\newcommand{\zeroOverInfty}{\ensuremath{\boldsymbol{\tfrac{0}{\infty}}}}
\newcommand{\zeroTimesInfty}{\ensuremath{\small\boldsymbol{0\cdot \infty}}}
\newcommand{\inftyMinusInfty}{\ensuremath{\small\boldsymbol{\infty - \infty}}}
\newcommand{\oneToInfty}{\ensuremath{\boldsymbol{1^\infty}}}
\newcommand{\zeroToZero}{\ensuremath{\boldsymbol{0^0}}}
\newcommand{\inftyToZero}{\ensuremath{\boldsymbol{\infty^0}}}


\newcommand{\numOverZero}{\ensuremath{\boldsymbol{\tfrac{\#}{0}}}}
\newcommand{\dfn}{\textbf}
%\newcommand{\unit}{\,\mathrm}
\newcommand{\unit}{\mathop{}\!\mathrm}
%\newcommand{\eval}[1]{\bigg[ #1 \bigg]}
\newcommand{\eval}[1]{ #1 \bigg|}
\newcommand{\seq}[1]{\left( #1 \right)}
\renewcommand{\epsilon}{\varepsilon}
\renewcommand{\iff}{\Leftrightarrow}

\DeclareMathOperator{\arccot}{arccot}
\DeclareMathOperator{\arcsec}{arcsec}
\DeclareMathOperator{\arccsc}{arccsc}
\DeclareMathOperator{\si}{Si}
\DeclareMathOperator{\proj}{proj}
\DeclareMathOperator{\scal}{scal}
\DeclareMathOperator{\cis}{cis}
\DeclareMathOperator{\Arg}{Arg}
%\DeclareMathOperator{\arg}{arg}
\DeclareMathOperator{\Rep}{Re}
\DeclareMathOperator{\Imp}{Im}
\DeclareMathOperator{\sech}{sech}
\DeclareMathOperator{\csch}{csch}
\DeclareMathOperator{\Log}{Log}

\newcommand{\tightoverset}[2]{% for arrow vec
  \mathop{#2}\limits^{\vbox to -.5ex{\kern-0.75ex\hbox{$#1$}\vss}}}
\newcommand{\arrowvec}{\overrightarrow}
\renewcommand{\vec}{\mathbf}
\newcommand{\veci}{{\boldsymbol{\hat{\imath}}}}
\newcommand{\vecj}{{\boldsymbol{\hat{\jmath}}}}
\newcommand{\veck}{{\boldsymbol{\hat{k}}}}
\newcommand{\vecl}{\boldsymbol{\l}}
\newcommand{\utan}{\vec{\hat{t}}}
\newcommand{\unormal}{\vec{\hat{n}}}
\newcommand{\ubinormal}{\vec{\hat{b}}}

\newcommand{\dotp}{\bullet}
\newcommand{\cross}{\boldsymbol\times}
\newcommand{\grad}{\boldsymbol\nabla}
\newcommand{\divergence}{\grad\dotp}
\newcommand{\curl}{\grad\cross}
%% Simple horiz vectors
\renewcommand{\vector}[1]{\left\langle #1\right\rangle}


\pgfplotsset{compat=1.13}

\outcome{Determine Laurent series}

\title{4.5 Laurent Series}

\begin{document}

\begin{abstract}
We determine the Laurent series for a given function.
\end{abstract}

\maketitle

Consider the function $\dis f(z) = e^\frac{1}{z}$. This function is analytic in the punctured plane $\C\backslash\{0\}$.
Using the Maclaurin series for $e^z$ we can create a series representation for $e^\frac{1}{z}$ as follows:
\[
e^\frac{1}{z} = \sum_{k=0}^\infty \frac{1}{k!} \cdot \left(\frac{1}{z}\right)^k = \sum_{k=0}^\infty \frac{1}{k!z^k} = 1+ \frac{1}{z} + \frac{1}{2! z^2} + \cdots
\]
 The ratio test shows that this series converges for all $z$ except $0$. This type of series is called a Laurent series.
 

\begin{definition} 
A series of the form 
\[
\sum_{k=-\infty}^\infty c_k(z-z_0)^k = \sum_{k=1}^\infty c_{-k}(z-z_0)^{-k} + \sum_{k=0}^\infty c_k(z-z_0)^k
\]
is called the \textbf{Laurent series} centered at $z_0$.
\end{definition}

\begin{definition}
A function $f$ is said to have an \textbf{isolated singularity} at $z_0$ if $f$ is analytic on a punctured disk $D(z_0, r)\backslash\{z_0\}$ for some $r>0$
but not analytic at $z_0$.
\end{definition}

\begin{theorem}
Suppose $f$ has an isolated singularity at $z_0$. Then $f$ has a Laurent series representation centered at $z_0$.
\end{theorem}

\begin{example}[example 1]
Find the Laurent series representation for $f(z) = \frac{e^z}{z^3}$ centered at the origin.\\
The function $f(z) = e^z/z^3$ is analytic in the punctured plane, so it has an isolated singularity at the origin.
We can find its Laurent series representation centered at the origin by using the Mauriac series of $e^z$.
We have
\[
\frac{e^z}{z^3} = \frac{1}{z^3} \sum_{k=0}^\infty \frac{z^k}{k!} = \sum_{k=0}^\infty \frac{z^{k-3}}{k!} 
\]
\[
= \frac{1}{z^3} + \frac{1}{z^2} + \frac{1}{2!z} + \frac{1}{3!} + \frac{z}{4!} + \frac{z^2}{5!} + \cdots
\]
This representation is valid for all $z$ in the punctured plane.
\end{example}

\begin{problem}(problem 1a)
Find the Laurent series for $\dis f(z) = \frac{\sin z}{z}$ centered at the origin.
\end{problem}

\begin{problem}(problem 1b)
Find the Laurent series for $\dis f(z) = \frac{\cos z}{z}$ centered at the origin.
\end{problem}

\begin{problem}(problem 1c)
Find the Laurent series for $\dis f(z) = \frac{e^z}{z^2}$ centered at the origin.
\end{problem}

\begin{problem}(problem 1d)
Find the Laurent series for $\dis f(z) = \frac{1}{z^4}$ centered at the origin.
\end{problem}

Isolated singularities of a function can be classified according to the largest negative power appearing in
its Laurent series.

\begin{definition}
A function $f(z)$ with an isolated singularity at $z_0$ is said to have a \textbf{pole of order} $\boldsymbol{k}$ at $z_0$
if is Laurent series centered at $z_0$ has the form
\[
f(z) = \frac{c_{-k}}{(z-z_0)^k} +  \frac{c_{-k+1}}{(z-z_0)^{k-1}} + \cdots + c_0 + c_1(z-z_0) + c_2(z-z_0)^2 + \cdots
\]
where $c_{-k} \neq 0$.
A pole of order $1$ is called a {\bf simple pole}. A pole of order $\infty$ is called an {\bf essential singularity}.
If the Laurent series has no terms with negative powers of $z-z_0$, then the singularity is a discontinuity.
\end{definition}

\begin{example}[example 2]
Classify the singularity at the origin for the function $\dis f(z) = \frac{e^z}{z^3}$.\\
The Laurent series for $f$ is
\[
\frac{e^z}{z^3} = \frac{1}{z^3} + \frac{1}{z^2} + \frac{1}{2!z} + \frac{1}{3!} + \frac{z}{4!} + \frac{z^2}{5!} + \cdots
\]
Since the largest power of $z$ appearing in the denominators is $3$, the function $\dis f(z) = \frac{e^z}{z^3}$
has a pole of order $3$ at the origin.
\end{example}

\begin{problem}(problem 2a)
Classify the isolated singularity at the origin for the function $\dis f(z) = \frac{\sin z}{z}$
\begin{multipleChoice}
\choice[correct]{discontinuity}
\choice{simple pole}
\choice{pole of order 2}
\choice{pole of order 5}
\choice{essential singularity}
\end{multipleChoice}
\end{problem}


\begin{problem}(problem 2b)
Classify the isolated singularity at the origin for the function $\dis f(z) = \frac{\cos z}{z}$
\begin{multipleChoice}
\choice{discontinuity}
\choice[correct]{simple pole}
\choice{pole of order 3}
\choice{pole of order 6}
\choice{essential singularity}
\end{multipleChoice}\end{problem}


\begin{problem}(problem 2c)
Classify the isolated singularity at the origin for the function $\dis f(z) = \frac{e^z}{z^2}$
\begin{multipleChoice}
\choice{discontinuity}
\choice{simple pole}
\choice[correct]{pole of order 2}
\choice{pole of order 5}
\choice{essential singularity}
\end{multipleChoice}
\end{problem}


\begin{problem}(problem 2d)
Classify the isolated singularity at the origin for the function $\dis f(z) = \frac{1}{z^4}$
\begin{multipleChoice}
\choice{discontinuity}
\choice{simple pole}
\choice[correct]{pole of order 4}
\choice{pole of order 8}
\choice{essential singularity}
\end{multipleChoice}
\end{problem}


\begin{problem}(problem 2e)
Classify the isolated singularity at the origin for the function $\dis f(z) = e^\frac{1}{z}$
\begin{multipleChoice}
\choice{discontinuity}
\choice{simple pole}
\choice{pole of order 7}
\choice{pole of order 9}
\choice[correct]{essential singularity}
\end{multipleChoice}\end{problem}

The term $\dis \frac{c_{-1}}{z-z_0}$ in a Laurent series is of particular importance in the theory of integration.

\begin{definition}
The coefficient $c_{-1}$ in the Laurent series for a function $f$ centered at $z_0$ is called the \textbf{residue} of $f$ at $z_0$, and it is denoted by
\[
\text{Res}(f,z_0) = c_{-1}
\]
\end{definition}


\begin{example}[example 3]
Find Res$(f,0)$ for $f(z) = \frac{e^z}{z^3}$.\\
The Laurent series is
\[
\frac{e^z}{z^3} = \frac{1}{z^3} + \frac{1}{z^2} + \frac{1}{2!z} + \frac{1}{3!} + \frac{z}{4!} + \frac{z^2}{5!} + \cdots
\]
and the term of interest is $ \frac{1}{2!z}$. Thus the residue, being the coefficient of this term is
\[
\text{Res}(f,0) = \frac{1}{2!} = \frac12
\]
\end{example}

\begin{problem}(problem 3a)
Find Res$(f, 0)$ for the function $\dis f(z) = \frac{\sin z}{z}$\\
$\dis \text{Res}(f,0) = \answer{0}$
\end{problem}


\begin{problem}(problem 3b)
Find Res$(f, 0)$ for the function $\dis f(z) = \frac{\cos z}{z}$\\
$\dis \text{Res}(f,0) = \answer{1}$
\end{problem}


\begin{problem}(problem 3c)
Find Res$(f, 0)$ for the function $\dis f(z) = \frac{e^z}{z^2}$\\
$\dis \text{Res}(f,0) = \answer{1}$
\end{problem}


\begin{problem}(problem 3d)
Find Res$(f, 0)$ for the function $\dis f(z) = \frac{1}{z^4}$\\
$\dis \text{Res}(f,0) = \answer{0}$
\end{problem}


\begin{problem}(problem 3e)
Find Res$(f, 0)$ for the function $\dis f(z) = e^\frac{1}{z}$\\
$\dis \text{Res}(f,0) = \answer{1}$
\end{problem}

Consider the function $\dis f(z) = \frac{1}{1-z}$. This function is analytic on $\C\backslash\{1\}$. 
We may recognize this function as the sum of the geometric series
\[
\sum_{k=0}^\infty z^k = 1+z+z^2 + \cdots = \frac{1}{1-z}, \, |z|<1
\]
This series represents the function inside of the unit disk, but what about the rest of the domain of analyticity of the function? 
On the set $|z|>1$ we can write a Laurent series centered at the origin for the function as follows:
\begin{align*}
\frac{1}{1-z} &= \frac{1}{z\left(\frac{1}{z} - 1\right)} = -\frac{1}{z}\cdot\frac{1}{1-\frac{1}{z}} \\
&= -\frac{1}{z} \sum_{k=0}^\infty \left(\frac{1}{z}\right)^k = -\sum_{k=0}^\infty \frac{1}{z^{k+1}}\\
&= -\frac{1}{z} - \frac{1}{z^2}- \frac{1}{z^3} - \cdots, \, |z|>1
\end{align*}

Hence we have multiple representations of $\dis \frac{1}{1-z}$ with each representation being valid in a different region.
We can generalize this idea.

\begin{theorem}
If $f(z)$ is analytic in an annulus centered at $z_0$ with inner radius $r$ and outer radius $R$, i.e. $A = A(z_0, r, R) = \{z\,\big|\, 0\leq r <|z-z_0| < R \leq \infty\}$, then 
$f$ has a Laurent series representation which is valid in $A$.
\end{theorem}

\begin{example}[example 4]
Find Laurent series expansions centered at $i$ for the function $\dis f(z) = \frac{2i}{z^2+1}$.\\
This function has isolated singularities at $\pm i$. Hence it is analytic in the annuli
$A_1 = A(i,0,2)$ and $A_2 = A(i, 2, \infty)$. The function has a Laurent series representation on each of these annuli.
To create these Laurent series, we make a partial fraction decomposition of $f$:
\[
\frac{2i}{z^2+1} = \frac{1}{z-i} - \frac{1}{z+i} \quad \text{(verify)}
\]

For the annulus $A_1$, we need a Taylor series centered at $i$ for the second fraction: $\dis \frac{1}{z+i}$.
We have
\[
\frac{1}{z+i} = \frac{1}{2i+(z-i)} = \frac{1}{2i} \cdot \frac{1}{1+ \frac{z-i}{2i}}
\]
We can now make a geometric series expansion 
\[
\frac{1}{z+i} = \frac{1}{2i} \cdot \sum_{k=0}^\infty \left(-\frac{z-i}{2i}\right)^k, \; \left|\frac{z-i}{2i}\right| < 1
\]
Simplifying and combining with $\dis \frac{1}{z-i}$, we obtain the Laurent series
\[
\frac{2i}{z^2+1} = \frac{1}{z-i} - \sum_{k=0}^\infty \frac{i^{k-1}}{2^{k+1}}(z-i)^k 
\]
\[
= \frac{1}{z-i} + \frac{i}{2} - \frac{1}{4}(z-i) - \frac{i}{8}(z-i)^2 + \cdots
\]
which is valid on $A_1$.
To find the Laurent series on $A_2$ we again use the geometric series formula, but with a twist so that the series converges in the correct region:

\[
\frac{1}{z+i} = \frac{1}{2i + (z-i)} = \frac{1}{z-i} \cdot  \frac{1}{\frac{2i}{z-i} +1}
\]

\[
= \frac{1}{z-i} \sum_{k=0}^\infty \left(-\frac{2i}{z-i}\right)^k = \sum_{k=0}^\infty \frac{(-2i)^k}{(z-i)^{k+1}}, \quad \left|\frac{2i}{z-i}\right| < 1
\]

Thus, on $A_2$ we have
\begin{align*}
\frac{2i}{z^2 +1} &= \frac{1}{z-i} -\frac{1}{z+i}\\
&=\frac{1}{z-i} - \sum_{k=0}^\infty \frac{(-2i)^k}{(z-i)^{k+1}}\\
&= \frac{1}{z-i} - \left(\frac{1}{z-i}-\frac{2i}{(z-i)^2}-\frac{4}{(z-i)^3}+\frac{8i}{(z-i)^4} \cdots \right)\\
&= \frac{2i}{(z-i)^2}+\frac{4}{(z-i)^3}-\frac{8i}{(z-i)^4} -\cdots
\end{align*}


\end{example}


\begin{problem}(problem 4)
Find the Laurent series for $\dis f(z) = \frac{1}{z(z+1)}$ on the annuli $A_1 = A(0, 0, 1)$ and $A_2 = A(0, 1, \infty)$
\end{problem}


\end{document}










