\documentclass{ximera}

%% You can put user macros here
%% However, you cannot make new environments



\newcommand{\ffrac}[2]{\frac{\text{\footnotesize $#1$}}{\text{\footnotesize $#2$}}}
\newcommand{\vasymptote}[2][]{
    \draw [densely dashed,#1] ({rel axis cs:0,0} -| {axis cs:#2,0}) -- ({rel axis cs:0,1} -| {axis cs:#2,0});
}


\graphicspath{{./}{firstExample/}}

\usepackage{amsmath}
\usepackage{amssymb}
\usepackage{array}
\usepackage[makeroom]{cancel} %% for strike outs
\usepackage{pgffor} %% required for integral for loops
\usepackage{tikz}
\usepackage{tikz-cd}
\usepackage{tkz-euclide}
\usetikzlibrary{shapes.multipart}


\usetkzobj{all}
\tikzstyle geometryDiagrams=[ultra thick,color=blue!50!black]


\usetikzlibrary{arrows}
\tikzset{>=stealth,commutative diagrams/.cd,
  arrow style=tikz,diagrams={>=stealth}} %% cool arrow head
\tikzset{shorten <>/.style={ shorten >=#1, shorten <=#1 } } %% allows shorter vectors

\usetikzlibrary{backgrounds} %% for boxes around graphs
\usetikzlibrary{shapes,positioning}  %% Clouds and stars
\usetikzlibrary{matrix} %% for matrix
\usepgfplotslibrary{polar} %% for polar plots
\usepgfplotslibrary{fillbetween} %% to shade area between curves in TikZ



%\usepackage[width=4.375in, height=7.0in, top=1.0in, papersize={5.5in,8.5in}]{geometry}
%\usepackage[pdftex]{graphicx}
%\usepackage{tipa}
%\usepackage{txfonts}
%\usepackage{textcomp}
%\usepackage{amsthm}
%\usepackage{xy}
%\usepackage{fancyhdr}
%\usepackage{xcolor}
%\usepackage{mathtools} %% for pretty underbrace % Breaks Ximera
%\usepackage{multicol}



\newcommand{\RR}{\mathbb R}
\newcommand{\R}{\mathbb R}
\newcommand{\C}{\mathbb C}
\newcommand{\N}{\mathbb N}
\newcommand{\Z}{\mathbb Z}
\newcommand{\dis}{\displaystyle}
%\renewcommand{\d}{\,d\!}
\renewcommand{\d}{\mathop{}\!d}
\newcommand{\dd}[2][]{\frac{\d #1}{\d #2}}
\newcommand{\pp}[2][]{\frac{\partial #1}{\partial #2}}
\renewcommand{\l}{\ell}
\newcommand{\ddx}{\frac{d}{\d x}}

\newcommand{\zeroOverZero}{\ensuremath{\boldsymbol{\tfrac{0}{0}}}}
\newcommand{\inftyOverInfty}{\ensuremath{\boldsymbol{\tfrac{\infty}{\infty}}}}
\newcommand{\zeroOverInfty}{\ensuremath{\boldsymbol{\tfrac{0}{\infty}}}}
\newcommand{\zeroTimesInfty}{\ensuremath{\small\boldsymbol{0\cdot \infty}}}
\newcommand{\inftyMinusInfty}{\ensuremath{\small\boldsymbol{\infty - \infty}}}
\newcommand{\oneToInfty}{\ensuremath{\boldsymbol{1^\infty}}}
\newcommand{\zeroToZero}{\ensuremath{\boldsymbol{0^0}}}
\newcommand{\inftyToZero}{\ensuremath{\boldsymbol{\infty^0}}}


\newcommand{\numOverZero}{\ensuremath{\boldsymbol{\tfrac{\#}{0}}}}
\newcommand{\dfn}{\textbf}
%\newcommand{\unit}{\,\mathrm}
\newcommand{\unit}{\mathop{}\!\mathrm}
%\newcommand{\eval}[1]{\bigg[ #1 \bigg]}
\newcommand{\eval}[1]{ #1 \bigg|}
\newcommand{\seq}[1]{\left( #1 \right)}
\renewcommand{\epsilon}{\varepsilon}
\renewcommand{\iff}{\Leftrightarrow}

\DeclareMathOperator{\arccot}{arccot}
\DeclareMathOperator{\arcsec}{arcsec}
\DeclareMathOperator{\arccsc}{arccsc}
\DeclareMathOperator{\si}{Si}
\DeclareMathOperator{\proj}{proj}
\DeclareMathOperator{\scal}{scal}
\DeclareMathOperator{\cis}{cis}
\DeclareMathOperator{\Arg}{Arg}
%\DeclareMathOperator{\arg}{arg}
\DeclareMathOperator{\Rep}{Re}
\DeclareMathOperator{\Imp}{Im}
\DeclareMathOperator{\sech}{sech}
\DeclareMathOperator{\csch}{csch}
\DeclareMathOperator{\Log}{Log}

\newcommand{\tightoverset}[2]{% for arrow vec
  \mathop{#2}\limits^{\vbox to -.5ex{\kern-0.75ex\hbox{$#1$}\vss}}}
\newcommand{\arrowvec}{\overrightarrow}
\renewcommand{\vec}{\mathbf}
\newcommand{\veci}{{\boldsymbol{\hat{\imath}}}}
\newcommand{\vecj}{{\boldsymbol{\hat{\jmath}}}}
\newcommand{\veck}{{\boldsymbol{\hat{k}}}}
\newcommand{\vecl}{\boldsymbol{\l}}
\newcommand{\utan}{\vec{\hat{t}}}
\newcommand{\unormal}{\vec{\hat{n}}}
\newcommand{\ubinormal}{\vec{\hat{b}}}

\newcommand{\dotp}{\bullet}
\newcommand{\cross}{\boldsymbol\times}
\newcommand{\grad}{\boldsymbol\nabla}
\newcommand{\divergence}{\grad\dotp}
\newcommand{\curl}{\grad\cross}
%% Simple horiz vectors
\renewcommand{\vector}[1]{\left\langle #1\right\rangle}


\outcome{Determine series behavior based on a definite integral}

\title{3.4 Integral Test}

\begin{document}

\begin{abstract}
We determine the convergence or divergence of an infinite series using a related improper integral.
\end{abstract}

\maketitle

\section{Integral Test}

\begin{theorem}[Integral Test]
Suppose that the function $f(x)$ is positive, continuous, and decreasing on the interval $(1, \infty)$, and 
suppose that the terms of an infinite series
are given by the corresponding function values, i.e., $a_n = f(n)$. \\
Then the improper integral
\[
\int_1^\infty f(x) \; dx.
\]
and the associated infinite series
\[
\sum_{n=1}^\infty a_n
\]
both converge or both diverge. In other words, the behavior of the improper integral 
determines the behavior of the associated series.
\end{theorem}

\begin{remark}
It is sufficient for the function to be decreasing eventually, i.e., on an interval of the form $(a, \infty)$
for some number $a$.
\end{remark}


The ratioanl for the integral test is that the sum of an infinite series can be interpretred as an area problem in the sense of integral calculus.
Since each term in the sum represents the area of area a rectangle of unit width, 
it stands to reason that we can use the comaprison theorem of improper integrals that can be roughly 
summarized as "less than convergent is convergent and greater than divergent is divergent". The two figures below show that if 
the function $f(x)$ and terms $a_n$ are in accord, then we can use the behavior of the integral to make conclusions about the behavior of the series.

\begin{center}
\begin{tikzpicture}

\begin{axis}[axis x line=middle, axis y line= middle, xlabel={$x$}, ylabel={$y$}, xtick={1, 2, 3, 4, 5, 6},
xmin=0, xmax = 6.5, ymax = 1.3,legend pos= north east,
xticklabels={1, 2, 3, 4, 5, 6}, ymin = 0, ytick={1/4, 1/3, 1/2, 1}, yticklabels={$a_4$, $a_3$, $a_2$, $a_1$}, 
title={
\begin{tabular} {c} 
If the improper integral $\displaystyle{\int_1^\infty f(x) \, dx}$ diverges, \\
then so does the associated series  $\displaystyle{\sum_{n=1}^\infty a_n}$ 
\end{tabular}  
}]
%title={\begin{tabular} {c}The area of each rectangle is $a_n$ and \\ 
%$\displaystyle{\int_1^\infty f(x) dx < \sum_{n=1}^\infty a_n}$\end{tabular}}]

\addplot+[fill][mark=none, domain=1:6, color=blue, thick, fill=blue!25]{1/x} \closedcycle;

\addplot[mark=*,cyan] coordinates {(1,1)};


\legend{$y = f(x)$, $(n\text{,}a_n)$};

\addplot[mark=*,cyan] coordinates {(2,1/2)} ;
\addplot[mark=*,cyan] coordinates {(3,1/3)} ;
\addplot[mark=*,cyan] coordinates {(4,1/4)} ;
\addplot[mark=*,cyan] coordinates {(5,1/5)} ;
\addplot[thick, cyan] coordinates{(1, 0) (1, 1) (2,1) (2,0)};
\addplot[thick, cyan] coordinates{ (2,1/2) (3, 1/2)  (3,0)};
\addplot[thick, cyan] coordinates{(3, 1/3) (4, 1/3)  (4,0)};
\addplot[thick, cyan] coordinates{(4,0) (4, 1/4) (5, 1/4)  (5,0)};
\addplot[thick, cyan] coordinates{(5, 1/5) (6, 1/5)  (6,0)};
\end{axis}
\end{tikzpicture}
\hspace{.51 in}
\begin{tikzpicture}

\begin{axis}
[
axis x line=middle, axis y line= middle, xlabel={$x$}, ylabel={$y$}, xtick={1, 2, 3, 4, 5, 6},
xmin=0, xmax = 6.5, ymax = 1.3, legend pos= north east,
xticklabels={1, 2, 3, 4, 5, 6}, ymin = 0, ytick={1/4, 1/3, 1/2, 1}, yticklabels={$a_4$, $a_3$, $a_2$, $a_1$}, 
title={
\begin{tabular} {c} 
If the improper integral $\displaystyle{\int_1^\infty f(x) \, dx}$ converges, \\
then so does the associated series  $\displaystyle{\sum_{n=2}^\infty a_n}$ 
\end{tabular}  
}
]


%title={\begin{tabular} {c} The area of each rectangle is $a_n$ and \\ 
%$\displaystyle{ \int_1^\infty f(x) dx > \sum_{n=2}^\infty a_n}$\end{tabular}}]

\addplot[domain=1:6, color=blue, thick,name path=f, ]{1/x};
fill between[
        of=f and axis,
    ];


\addplot[mark=*,cyan] coordinates {(2,1/2)} ;

\legend{$y = f(x)$, $(n\text{,}a_n)$};

\addplot[mark=*,cyan] coordinates {(1,1)};
\addplot[mark=*,cyan] coordinates {(3,1/3)} ;
\addplot[mark=*,cyan] coordinates {(4,1/4)} ;
\addplot[mark=*,cyan] coordinates {(5,1/5)} ;
\addplot[mark=*,cyan] coordinates {(6,1/6)} ;
\addplot[thick, cyan] coordinates{(0, 0) (0, 1) (1,1) (1,0)};
\addplot[thick, cyan, fill=cyan!25!white] coordinates{(1,0) (1,1/2) (2, 1/2)  (2,0)};
\addplot[thick, cyan, fill=cyan!25!white] coordinates{(2,0) (2, 1/3) (3, 1/3)  (3,0)};
\addplot[thick, cyan, fill=cyan!25!white] coordinates{(3,0) (3, 1/4) (4, 1/4)  (4,0)};
\addplot[thick, cyan, fill=cyan!25!white] coordinates{(4,0) (4, 1/5) (5, 1/5)  (5,0)};
\addplot[thick, cyan, fill=cyan!25!white] coordinates{(5,0) (5, 1/6) (6, 1/6)  (6,0)};
\end{axis}
\end{tikzpicture}
\end{center}


\begin{example}
Determine whether the series
\[
\sum_{n=0}^\infty \frac{1}{1+n^2}
\]
converges or diverges.\\
Consider the function, $f(x) = \frac{1}{1+x^2}$, so that $a_n = f(n)$.
This function is positive and continuous on the interval $(0, \infty)$.
To see that the function is decreasing, we recall that if the derivative is negative then the function is decreasing.
Using the Quotient Rule, we can see that the derivative is given by 
\[
f'(x) = -\frac{2x}{(1+x^2)^2}.
\]
Since $f'(x)$ is negative on the interval $(0, \infty)$, the original function $f(x)$ is decreasing on that interval.
Hence the conditions of the Integral Test are met and the behavior of the series is determined by the behavior of the improper integral:
\begin{align*}
\int_0^\infty \frac{1}{1+x^2} \; dx &= \lim_{b \to \infty} \int_0^b \frac{1}{1+x^2} \; dx\\
&= \lim_{b \to \infty} \tan^{-1}(x) \bigg|_0^b \\
&= \lim_{b \to \infty} \left[\tan^{-1}(b) - \tan^{-1}(0)\right]\\
&= \lim_{b \to \infty} \left[\tan^{-1}(b)\right]\\
&= \frac{\pi}{2}.
\end{align*}
We see that the improper integral converges. By the Integral Test, the associated infinite series
\[
\sum_{n=0}^\infty \frac{1}{1+n^2}
\]
also converges.
\end{example}



\begin{example}
Determine whether the series
\[
\sum_{n=2}^\infty \frac{\ln(n)}{n}
\]
converges or diverges.\\
Consider the function, $f(x) = \frac{\ln(x)}{x}$, so that $a_n = f(n)$.
This function is positive and continuous on the interval $(2, \infty)$.
To see that the function is decreasing, compute the derivative:
\[
f'(x) = \frac{x\cdot\frac{1}{x}- \ln(x)}{x^2} = \frac{1-\ln(x)}{x^2},
\]
which is negative if $x > e$. Hence $f(x)$ is eventually decreasing and we can use the integral test:

\begin{align*}
\int_2^\infty \frac{\ln(x)}{x} \; dx &= \lim_{t \to \infty} \int_2^t \frac{\ln(x)}{x} \; dx\\
&= \lim_{t \to \infty} \int_{\ln(2)}^{\ln(t)} \frac{1}{u} \, du \;\; \left(\text{u-sub with $u = \ln(x), du = \frac{1}{x} dx$}\right)\\
&= \lim_{t \to \infty} \ln|u| \bigg|_{\ln(2)}^{\ln(t)}\\
&= \lim_{t \to \infty} \big(\ln|\ln(t)| - \ln|\ln 2|\big)\\
&= \infty.
\end{align*}
We see that the improper integral diverges. By the Integral Test, the associated infinite series
\[
\sum_{n=2}^\infty \frac{\ln(n)}{n}
\]
also diverges.
\end{example}




\begin{definition}[p-series]
Let $p > 0$. The infinite series
\[
\sum_{n=1}^\infty \frac{1}{n^p}
\]
is called a \textbf{$p$-series}. In the special case $p = 1$, the series is known as the \textbf{harmonic series}.
\end{definition}

We have seen in the section on improper integrals that the $p$-integral

\[
\int_1^\infty \frac{1}{x^p} \; dx.
\]
converges if $p > 1$ and diverges if $p \leq 1$.
Since the function $f(x) = 1/x^p \; (p > 0)$ is continuous, positive, and decreasing on the interval $(1, \infty)$,
the integral test can be applied to conclude that the associated $p$-series has the same behavior.  This is stated in the following theorem.


\begin{theorem}[$p$-series]
The series
\[
\sum_{n=1}^\infty \frac{1}{n^p}
\]
converges if $p > 1$ and diverges if $p \leq 1$.
\end{theorem}


\section{Video Lesson}

\begin{center}
\begin{foldable}
\unfoldable{Here is a detailed, lecture style video on the Integral Test:}
\youtube{db7NqcbCO4w&t}
\end{foldable}
\end{center}



\section{More Problems}



\end{document}


\begin{example} %example #15
Find $h'(x)$ if $h(x) = x^{\sin(x)}$.\\
We will use the fact that the exponential and logarithm functions are inverses,
\[e^{\ln(x)} = x,\]
and the exponent property of logarithms, 
\[\ln(x^n) = n\ln(x),\]
to rewrite $h(x)$.  We have 
\[h(x) = x^{\sin(x)} = e^{\ln(x^{\sin(x)})} = e^{\sin(x)\ln(x)}\]
and we can now compute $h'(x)$ using a combination of the chain rule and product rule.
We can write $h(x)$ as a composition, $f(g(x))$ with 
\[g(x) = \sin(x)\ln(x) \quad \text{and} \quad f(x) = e^x.\]
Then to find $g'(x)$ we us the product rule and we get $g'(x) = \frac{\sin(x)}{x} + \cos(x)\ln(x)$.
Next $f'(x) = e^x$ and 
hence $f'(g(x)) = e^{g(x)} = e^{\sin(x)\ln(x)} = x^{\sin(x)}$.
We can then conclude $h'(x) = f'(g(x))g'(x) = x^{\sin(x)} \left[ \frac{\sin(x)}{x} + \cos(x)\ln(x)\right]$.
\end{example}

%more question formats below













%\begin{verbatim}
\begin{question}
What is your favorite color?
\begin{multipleChoice}
\choice[correct]{Rainbow}
\choice{Blue}
\choice{Green}
\choice{Red}
\end{multipleChoice}
\begin{freeResponse}
Hello
\end{freeResponse}
\end{question}
%\end{verbatim}





\begin{question}
  Which one will you choose?
  \begin{multipleChoice}
    \choice[correct]{I'm correct.}
    \choice{I'm wrong.}
    \choice{I'm wrong too.}
  \end{multipleChoice}
\end{question}


\begin{question}
  Which one will you choose?
  \begin{selectAll}
    \choice[correct]{I'm correct.}
    \choice{I'm wrong.}
    \choice[correct]{I'm also correct.}
    \choice{I'm wrong too.}
  \end{selectAll}
\end{question}


\begin{freeResponse}
What is the chain rule used for?
\end{freeResponse}
