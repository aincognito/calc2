\documentclass[handout]{ximera}

%% You can put user macros here
%% However, you cannot make new environments



\newcommand{\ffrac}[2]{\frac{\text{\footnotesize $#1$}}{\text{\footnotesize $#2$}}}
\newcommand{\vasymptote}[2][]{
    \draw [densely dashed,#1] ({rel axis cs:0,0} -| {axis cs:#2,0}) -- ({rel axis cs:0,1} -| {axis cs:#2,0});
}


\graphicspath{{./}{firstExample/}}

\usepackage{amsmath}
\usepackage{amssymb}
\usepackage{array}
\usepackage[makeroom]{cancel} %% for strike outs
\usepackage{pgffor} %% required for integral for loops
\usepackage{tikz}
\usepackage{tikz-cd}
\usepackage{tkz-euclide}
\usetikzlibrary{shapes.multipart}


\usetkzobj{all}
\tikzstyle geometryDiagrams=[ultra thick,color=blue!50!black]


\usetikzlibrary{arrows}
\tikzset{>=stealth,commutative diagrams/.cd,
  arrow style=tikz,diagrams={>=stealth}} %% cool arrow head
\tikzset{shorten <>/.style={ shorten >=#1, shorten <=#1 } } %% allows shorter vectors

\usetikzlibrary{backgrounds} %% for boxes around graphs
\usetikzlibrary{shapes,positioning}  %% Clouds and stars
\usetikzlibrary{matrix} %% for matrix
\usepgfplotslibrary{polar} %% for polar plots
\usepgfplotslibrary{fillbetween} %% to shade area between curves in TikZ



%\usepackage[width=4.375in, height=7.0in, top=1.0in, papersize={5.5in,8.5in}]{geometry}
%\usepackage[pdftex]{graphicx}
%\usepackage{tipa}
%\usepackage{txfonts}
%\usepackage{textcomp}
%\usepackage{amsthm}
%\usepackage{xy}
%\usepackage{fancyhdr}
%\usepackage{xcolor}
%\usepackage{mathtools} %% for pretty underbrace % Breaks Ximera
%\usepackage{multicol}



\newcommand{\RR}{\mathbb R}
\newcommand{\R}{\mathbb R}
\newcommand{\C}{\mathbb C}
\newcommand{\N}{\mathbb N}
\newcommand{\Z}{\mathbb Z}
\newcommand{\dis}{\displaystyle}
%\renewcommand{\d}{\,d\!}
\renewcommand{\d}{\mathop{}\!d}
\newcommand{\dd}[2][]{\frac{\d #1}{\d #2}}
\newcommand{\pp}[2][]{\frac{\partial #1}{\partial #2}}
\renewcommand{\l}{\ell}
\newcommand{\ddx}{\frac{d}{\d x}}

\newcommand{\zeroOverZero}{\ensuremath{\boldsymbol{\tfrac{0}{0}}}}
\newcommand{\inftyOverInfty}{\ensuremath{\boldsymbol{\tfrac{\infty}{\infty}}}}
\newcommand{\zeroOverInfty}{\ensuremath{\boldsymbol{\tfrac{0}{\infty}}}}
\newcommand{\zeroTimesInfty}{\ensuremath{\small\boldsymbol{0\cdot \infty}}}
\newcommand{\inftyMinusInfty}{\ensuremath{\small\boldsymbol{\infty - \infty}}}
\newcommand{\oneToInfty}{\ensuremath{\boldsymbol{1^\infty}}}
\newcommand{\zeroToZero}{\ensuremath{\boldsymbol{0^0}}}
\newcommand{\inftyToZero}{\ensuremath{\boldsymbol{\infty^0}}}


\newcommand{\numOverZero}{\ensuremath{\boldsymbol{\tfrac{\#}{0}}}}
\newcommand{\dfn}{\textbf}
%\newcommand{\unit}{\,\mathrm}
\newcommand{\unit}{\mathop{}\!\mathrm}
%\newcommand{\eval}[1]{\bigg[ #1 \bigg]}
\newcommand{\eval}[1]{ #1 \bigg|}
\newcommand{\seq}[1]{\left( #1 \right)}
\renewcommand{\epsilon}{\varepsilon}
\renewcommand{\iff}{\Leftrightarrow}

\DeclareMathOperator{\arccot}{arccot}
\DeclareMathOperator{\arcsec}{arcsec}
\DeclareMathOperator{\arccsc}{arccsc}
\DeclareMathOperator{\si}{Si}
\DeclareMathOperator{\proj}{proj}
\DeclareMathOperator{\scal}{scal}
\DeclareMathOperator{\cis}{cis}
\DeclareMathOperator{\Arg}{Arg}
%\DeclareMathOperator{\arg}{arg}
\DeclareMathOperator{\Rep}{Re}
\DeclareMathOperator{\Imp}{Im}
\DeclareMathOperator{\sech}{sech}
\DeclareMathOperator{\csch}{csch}
\DeclareMathOperator{\Log}{Log}

\newcommand{\tightoverset}[2]{% for arrow vec
  \mathop{#2}\limits^{\vbox to -.5ex{\kern-0.75ex\hbox{$#1$}\vss}}}
\newcommand{\arrowvec}{\overrightarrow}
\renewcommand{\vec}{\mathbf}
\newcommand{\veci}{{\boldsymbol{\hat{\imath}}}}
\newcommand{\vecj}{{\boldsymbol{\hat{\jmath}}}}
\newcommand{\veck}{{\boldsymbol{\hat{k}}}}
\newcommand{\vecl}{\boldsymbol{\l}}
\newcommand{\utan}{\vec{\hat{t}}}
\newcommand{\unormal}{\vec{\hat{n}}}
\newcommand{\ubinormal}{\vec{\hat{b}}}

\newcommand{\dotp}{\bullet}
\newcommand{\cross}{\boldsymbol\times}
\newcommand{\grad}{\boldsymbol\nabla}
\newcommand{\divergence}{\grad\dotp}
\newcommand{\curl}{\grad\cross}
%% Simple horiz vectors
\renewcommand{\vector}[1]{\left\langle #1\right\rangle}


\outcome{Sample Final Exams}

\title{Sample Final Exams}

\begin{document}

\begin{abstract}
Sample Final Exams
\end{abstract}

\maketitle

\section{Sample Final, version A}


\begin{problem}(problem 1) 
Find the area bounded by the graphs of the parabola $y = x^2 + x - 3$ and the line $y = 2x - 1$.

\end{problem}

\begin{problem}(problem 2)
Find the volume of the solid obtained by revolving the region bounded by the graphs of $y = \tan(x)$ and $y = 0$
from $x = 0$ to $x = \pi/4$ about the $x$-axis.\\
answer: $\displaystyle \pi\left(1-\frac{\pi}{4}\right)$

\end{problem}


\begin{problem}(problem 3)
Find the volume of the solid obtained by revolving the region bounded by the graphs of $y = \sin(x)$ and $y = 0$
from $x = 0$ to $x = \pi$ about the $y$-axis.\\
answer: $\displaystyle 2\pi^2$

\end{problem}


\begin{problem}(problem 4)
Find the length of the graph of
$y = \frac{x^3}{12} + \frac{1}{x}$ from $x = 1$ to $x = 2$.

\end{problem}

\begin{problem}(problem 5)
Find the average value of the function $f(x) = xe^{2x}$ over the interval $(0,2)$.\\
answer: $\displaystyle \frac38 e^4 + \frac18$

\end{problem}


\begin{problem}(problem 6)
Solve the initial value problem $ \displaystyle (1+t^2)\frac{dy}{dt} = 2ty, \; y(1) = 8 $.

\end{problem}



\begin{problem}(problem 7)
Compute $\displaystyle \int x^3 \sqrt{1-x^2} \, dx$

\end{problem}


\begin{problem}(problem 8)
Compute $\displaystyle \int \frac{3x-4}{x^3 +2x^2 +x} \, dx$

\end{problem}



\begin{problem}(problem 9)
Find the sum of the infinite series, if it converges: $\displaystyle\sum_{n=1}^\infty \frac{1}{n^2 +n}$

\end{problem}




\begin{problem}(problem 10)
Use the Integral Test to determine whether the infinite series converges or diverges:  $\displaystyle \sum_{n=0}^\infty \frac{1}{1+n^2}$

\end{problem}

\begin{problem}(problem 11)
Determine whether the infinite series converges or diverges: $\displaystyle \sum_{n=1}^\infty \frac{n+1}{2n+3}$

\end{problem}





\begin{problem}(problem 12)
Determine whether the infinite series converges or diverges: $\displaystyle \sum_{n=1}^\infty \frac{n^2 + 5n + 12}{n^3 + 6n^2 + 2n}$

\end{problem}



\begin{problem}(problem 13)
Find the interval of convergence of the power series $\displaystyle{\sum_{n=1}^\infty \frac{(x+2)^n}{(2n+1)3^n}}$

\end{problem}


\begin{problem}(problem 14)
Find a power series representation for the function
$ \displaystyle f(x) = \frac{1}{2-x} $. Be sure to include the interval of convergence in your answer.

\end{problem}



\begin{problem}(problem 15)
Find the first four terms of the Taylor Series $f(x) = \sqrt x$ centered at $x = 4$.

\end{problem}



\begin{problem}(problem 16)
Find the Maclaurin series representation of the function $f(x) = x^2 e^{2x}$.

\end{problem}






\section{Sample Final, version B}


\begin{problem}(problem 1) 
Find the area between the curves $y = \cos(x)$ and $y=\sin(x)$ over the interval $[0, \pi]$.

\end{problem}



\begin{problem}(problem 2)
Find the volume of the solid obtained by revolving the region bounded by the graphs of $y = \sin(x)$ and $y = 0$
from $x = 0$ to $x = \pi/2$ about the $x$-axis.\\
answer: $\displaystyle \frac{\pi^2}{4}$

\end{problem}



\begin{problem}(problem 3)
Find the volume of the solid obtained by revolving the region bounded by the graphs of $y = e^x$ and $y = 0$
from $x = 0$ to $x = 1$ about the $y$-axis.\\
answer: $\displaystyle 2\pi$

\end{problem}



\begin{problem}(problem 4)
Find the length of the graph of 
$y = 1+ x^{3/2}$ 
from  $x = 0$ to $x = 4$.

\end{problem}



\begin{problem}(problem 5)
Find the average value of the function $f(x) = \frac{\ln(x)}{x}$ over the interval $(1,e)$.\\
answer: $\displaystyle \frac{1}{2e-2}$

\end{problem}



\begin{problem}(problem 6)
Solve the differential equation $y' = xy^2$.

\end{problem}



\begin{problem}(problem 7)
Compute $\displaystyle \int \frac{x^3}{ \sqrt{9+x^2}} \, dx$

\end{problem}



\begin{problem}(problem 8)
Compute $\displaystyle \int \frac{x^2 + 5x + 2}{x^3 + 4x} \, dx$

\end{problem}






\begin{problem}(problem 9)
Find the sum of the infinite series, if it converges: $\displaystyle \sum_{n=0}^\infty \frac{2^{2n}}{3^n}$

\end{problem}


\begin{problem}(problem 10)
Use the integral test to determine whether the infinite series $\displaystyle \sum_{n=1}^\infty \frac{n}{e^n}$
converges or diverges.

\end{problem}


\begin{problem}(problem 11)
Determine whether the infinite series converges or diverges: $\displaystyle \sum_{n=1}^\infty \frac{\sin^2(n)}{n^3}$

\end{problem}


\begin{problem}(problem 12)
Determine whether the infinite series converges absolutely, converges conditionally or diverges: $\displaystyle \sum_{n=0}^\infty (-1)^n \frac{n^2 + 2}{n^3 + 3}$


\end{problem}


\begin{problem}(problem 13)
Find the interval of convergence of the power series $\displaystyle \sum_{n=1}^\infty \frac{(x-4)^n}{n2^n}$

\end{problem}


\begin{problem}(problem 14)
Find a power series representation for the function $\displaystyle f(x) = \frac{3x^3}{4 + x^2}$.
Be sure to include the interval of convergence in your answer.

\end{problem}



\begin{problem}(problem 15)
Find the first four terms of the Taylor Series $f(x) = \ln(x)$ centered at $x = 1$.

\end{problem}



\begin{problem}(problem 16)
Find the Maclaurin series representation for the function $f(x) = x\sin(3x)$.

\end{problem}













\end{document}













