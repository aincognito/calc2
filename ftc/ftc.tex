\documentclass[handout]{ximera}
\usepgfplotslibrary{fillbetween}
%% You can put user macros here
%% However, you cannot make new environments



\newcommand{\ffrac}[2]{\frac{\text{\footnotesize $#1$}}{\text{\footnotesize $#2$}}}
\newcommand{\vasymptote}[2][]{
    \draw [densely dashed,#1] ({rel axis cs:0,0} -| {axis cs:#2,0}) -- ({rel axis cs:0,1} -| {axis cs:#2,0});
}


\graphicspath{{./}{firstExample/}}

\usepackage{amsmath}
\usepackage{amssymb}
\usepackage{array}
\usepackage[makeroom]{cancel} %% for strike outs
\usepackage{pgffor} %% required for integral for loops
\usepackage{tikz}
\usepackage{tikz-cd}
\usepackage{tkz-euclide}
\usetikzlibrary{shapes.multipart}


\usetkzobj{all}
\tikzstyle geometryDiagrams=[ultra thick,color=blue!50!black]


\usetikzlibrary{arrows}
\tikzset{>=stealth,commutative diagrams/.cd,
  arrow style=tikz,diagrams={>=stealth}} %% cool arrow head
\tikzset{shorten <>/.style={ shorten >=#1, shorten <=#1 } } %% allows shorter vectors

\usetikzlibrary{backgrounds} %% for boxes around graphs
\usetikzlibrary{shapes,positioning}  %% Clouds and stars
\usetikzlibrary{matrix} %% for matrix
\usepgfplotslibrary{polar} %% for polar plots
\usepgfplotslibrary{fillbetween} %% to shade area between curves in TikZ



%\usepackage[width=4.375in, height=7.0in, top=1.0in, papersize={5.5in,8.5in}]{geometry}
%\usepackage[pdftex]{graphicx}
%\usepackage{tipa}
%\usepackage{txfonts}
%\usepackage{textcomp}
%\usepackage{amsthm}
%\usepackage{xy}
%\usepackage{fancyhdr}
%\usepackage{xcolor}
%\usepackage{mathtools} %% for pretty underbrace % Breaks Ximera
%\usepackage{multicol}



\newcommand{\RR}{\mathbb R}
\newcommand{\R}{\mathbb R}
\newcommand{\C}{\mathbb C}
\newcommand{\N}{\mathbb N}
\newcommand{\Z}{\mathbb Z}
\newcommand{\dis}{\displaystyle}
%\renewcommand{\d}{\,d\!}
\renewcommand{\d}{\mathop{}\!d}
\newcommand{\dd}[2][]{\frac{\d #1}{\d #2}}
\newcommand{\pp}[2][]{\frac{\partial #1}{\partial #2}}
\renewcommand{\l}{\ell}
\newcommand{\ddx}{\frac{d}{\d x}}

\newcommand{\zeroOverZero}{\ensuremath{\boldsymbol{\tfrac{0}{0}}}}
\newcommand{\inftyOverInfty}{\ensuremath{\boldsymbol{\tfrac{\infty}{\infty}}}}
\newcommand{\zeroOverInfty}{\ensuremath{\boldsymbol{\tfrac{0}{\infty}}}}
\newcommand{\zeroTimesInfty}{\ensuremath{\small\boldsymbol{0\cdot \infty}}}
\newcommand{\inftyMinusInfty}{\ensuremath{\small\boldsymbol{\infty - \infty}}}
\newcommand{\oneToInfty}{\ensuremath{\boldsymbol{1^\infty}}}
\newcommand{\zeroToZero}{\ensuremath{\boldsymbol{0^0}}}
\newcommand{\inftyToZero}{\ensuremath{\boldsymbol{\infty^0}}}


\newcommand{\numOverZero}{\ensuremath{\boldsymbol{\tfrac{\#}{0}}}}
\newcommand{\dfn}{\textbf}
%\newcommand{\unit}{\,\mathrm}
\newcommand{\unit}{\mathop{}\!\mathrm}
%\newcommand{\eval}[1]{\bigg[ #1 \bigg]}
\newcommand{\eval}[1]{ #1 \bigg|}
\newcommand{\seq}[1]{\left( #1 \right)}
\renewcommand{\epsilon}{\varepsilon}
\renewcommand{\iff}{\Leftrightarrow}

\DeclareMathOperator{\arccot}{arccot}
\DeclareMathOperator{\arcsec}{arcsec}
\DeclareMathOperator{\arccsc}{arccsc}
\DeclareMathOperator{\si}{Si}
\DeclareMathOperator{\proj}{proj}
\DeclareMathOperator{\scal}{scal}
\DeclareMathOperator{\cis}{cis}
\DeclareMathOperator{\Arg}{Arg}
%\DeclareMathOperator{\arg}{arg}
\DeclareMathOperator{\Rep}{Re}
\DeclareMathOperator{\Imp}{Im}
\DeclareMathOperator{\sech}{sech}
\DeclareMathOperator{\csch}{csch}
\DeclareMathOperator{\Log}{Log}

\newcommand{\tightoverset}[2]{% for arrow vec
  \mathop{#2}\limits^{\vbox to -.5ex{\kern-0.75ex\hbox{$#1$}\vss}}}
\newcommand{\arrowvec}{\overrightarrow}
\renewcommand{\vec}{\mathbf}
\newcommand{\veci}{{\boldsymbol{\hat{\imath}}}}
\newcommand{\vecj}{{\boldsymbol{\hat{\jmath}}}}
\newcommand{\veck}{{\boldsymbol{\hat{k}}}}
\newcommand{\vecl}{\boldsymbol{\l}}
\newcommand{\utan}{\vec{\hat{t}}}
\newcommand{\unormal}{\vec{\hat{n}}}
\newcommand{\ubinormal}{\vec{\hat{b}}}

\newcommand{\dotp}{\bullet}
\newcommand{\cross}{\boldsymbol\times}
\newcommand{\grad}{\boldsymbol\nabla}
\newcommand{\divergence}{\grad\dotp}
\newcommand{\curl}{\grad\cross}
%% Simple horiz vectors
\renewcommand{\vector}[1]{\left\langle #1\right\rangle}


\outcome{Compute the value of a definite integral using the fundamental theorem.}

\title{4.6 The Fundamental Theorem of Calculus}

%\newcommand{\ffrac}[2]{\frac{\mbox{\footnotesize $#1$}}{\mbox{\footnotesize $#2$}}}
%\newcommand{\vasymptote}[2][]{\draw [densely dashed,#1] 
%({rel axis cs:0,0} -| {axis cs:#2,0}) -- ({rel axis cs:0,1} -| {axis cs:#2,0});}


\begin{document}

\begin{abstract}
In this section we learn to compute the value of a definite integral using the fundamental theorem of calculus.
\end{abstract}

\maketitle



\section{The Fundamental Theorem of Calculus}

We begin by recalling the definition of the definite integral:
\[
\int_a^b f(x) \ dx \equiv \lim_{n\to \infty} \sum_{i=1}^n f(x_i^*)\Delta x.
\]
where $\Delta x = (b-a)/n$ and $x_i^*$ is a sample point for the $i$-th sub-interval, $[x_{i-1}, x_i]$.


Computing the value of a definite integral from this definition can be cumbersome and 
require knowledge of special summation formulas.  Fortunately, there is a theorem which states 
that computing definite integrals can done via anti-differentiation!


\begin{theorem}[Fundamental Theorem of Calculus]
If $f(x)$ is continuous on the interval $[a, b]$ and if $F(x)$ is an 
anti-derivative of $f(x)$, i.e., $F'(x) = f(x)$, then
\[
\int_a^b f(x) \ dx = F(b) - F(a).
\]
\end{theorem}


The proof of the Fundamental Theorem of Calculus can be obtained by applying the Mean Value Theorem to $F(x)$ on each of the sub-intervals $[x_{i-1}, x_i]$
and using the value of $c$ in each case as the sample point.

%using the values of $`c'$ from the Mean Value Theorem 
%(applied to the anti-derivative, $F(x)$) as the values of the sample points $x_i^*$ in the Riemann Sum in 
%the definition of the definite integral.

The process of calculating the numerical value of a definite integral is performed in 
two main steps: first, find the anti-derivative
$F(x)$ and  second, plug the endpoints of integration, $a$ and $b$ to compute $F(b) - F(a)$. 
To symbolize the steps, we use a vertical evaluation bar:

\[\int_a^b f(x) \ dx = F(x)\Big|_a^b = F(b) - F(a)\]
In the first step, the function $F(x)$ is introduced and in the second step, the difference of the values $F(b)$ and $F(a)$ is computed.

In the following examples, we compute definite integrals using the FTC in order to solve the area problem.
Recall that the definite integral of a non-negative function gives the area under the curve.

\begin{example}[example 1]
Find the exact area under the graph of $y = x^2$ from $x = 0$ to $x = 1$. \\
Since the function $x^2$ is non-negative on the interval $[0, 1]$, the 
exact area under the curve is the definite integral 
\[\int_0^1 x^2 \ dx.\]
We can compute the value of this definite integral using the Fundamental Theorem of Calculus as follows:
\[\int_0^1 x^2 \ dx = \frac{x^3}{3} \Bigg|_0^1 = \frac{1^3}{3} - \frac{0^3}{3} = \frac13.\]




\begin{image}
\begin{tikzpicture}
\begin{axis}[axis x line=  bottom, axis y line = left, xtick={-1, -0.5,0,0.5,1},  
ytick={0, 1, 2}, title={The area under $f(x)=x^2$ on $[0,1]$}]
\addplot[name path = A, domain=0:1, 
    samples=100, color=black]{x^2};
\addplot[name path = B, domain=0:1, samples=100, color=black]{0};
\addplot[blue!10] fill between[of=A and B];

%\addplot [domain= .1:.2, samples=10, color=black]{-x + .2};
%\addplot [domain= .2:.24, samples=10, color=black]{.04-(x- .2)};
%\addplot [domain= .3:.39, samples=10, color=black]{.09-(x- .3)};
%\addplot [domain= .4:.56, samples=10, color=black]{.16-(x- .4)};
%\addplot [domain= .5:.75, samples=10, color=black]{.25-(x- .5)};
%\addplot [domain= .6:.96, samples=10, color=black]{.36-(x- .6)};
\node at (axis cs: .7,.2){Area = 1/3};
%\addplot [domain= .7:1, samples=10, color=black]{.49-(x- .7)};
%\addplot [domain= .8:1, samples=10, color=black]{.64-(x- .8)};
%\addplot [domain= .9:1, samples=10, color=black]{.81-(x- .9)};

%\addplot [domain= .2:.4, samples=10, color=black]{-x + .05};
%\addplot coordinates {(0.5,0) (0.5, 1/sqrt e)};
\addplot[thick, samples = 100, color=black] coordinates {(1,0) (1,1)};

\end{axis}
%\legend{$y = f(x)$, , secant line, tangent line, };
\end{tikzpicture}
\end{image}



\end{example}



\begin{problem}(problem 1)
Find the exact area under the graph of $y = x^3$ from $x = 0$ to $x = 1$.\\
The area is $\answer{1/4}$

\begin{hint}
  \[
  \text{Area} = \int_0^{1} x^3 \ dx 
  \]
\end{hint}
    \begin{hint}
      Use the power rule to find an anti-derivative of $x^3$
    \end{hint}
    
    \begin{hint}
      The Fundamental Theorem of Calculus says:
      \[
      \int_a^b f(x) \ dx = F(b) - F(a)
      \]
    \end{hint}    
		
		
\end{problem}



\begin{example}[example 2]
Find the exact area under the curve $y = \cos(x)$ from $x=0$ to $x= \pi/2$.\\
Since the function $\cos(x)$ is non-negative on the interval $[0, \frac{\pi}{2}]$, the exact area can be expressed as the definite integral
\[\int_0^{\pi/2} \cos(x) \ dx.\]
We can use the Fundamental Theorem of Calculus to compute the value of this definite integral as follows:
\[\int_0^{\pi/2} \cos(x) \ dx = \sin(x) \Big|_0^{\pi/2} = \sin(\tfrac{\pi}{2}) - \sin(0)= 1- 0 = 1.\]


\begin{image}
\begin{tikzpicture}
\begin{axis}[axis x line=  bottom, axis y line = left,xtick={0, 0.785, 1.57}, xticklabels={$0$, $\frac{\pi}{4}$,$\frac{\pi}{2}$}, ytick={0, .5, 1}, title={The area under $f(x)=\cos(x)$ on $[0,\pi/2]$}]
\addplot[name path = A, domain=0:1.57, 
    samples=100, color=black]{cos(deg(x))};
\addplot[name path = B, domain=0:1.57, samples=100, color=black]{0};
\addplot[blue!10] fill between[of=A and B];
\node at (axis cs: .4,.4){Area = 1};


\end{axis}
%\legend{$y = f(x)$, , secant line, tangent line, };
\end{tikzpicture}
\end{image}

\end{example}



\begin{problem}(problem 2)
Find the exact area under the graph of $y = \sin(x)$ from $x = 0$ to $x = \pi$.\\
The area is $\answer{2}$

\begin{hint}
  \[
  \text{Area} = \int_0^{\pi} \sin(x) \ dx 
  \]
\end{hint}
    \begin{hint}
      An anti-derivative of $\sin(x)$ is $-\cos(x)$
    \end{hint}
    
    \begin{hint}
      The Fundamental Theorem of Calculus says:
      \[
      \int_a^b f(x) \ dx = F(b) - F(a)
      \]
    \end{hint}    
		\begin{hint}
      $\cos(0) = 1$ and $\cos(\pi) = -1$
    \end{hint}
		
\end{problem}


\begin{example}[example 3]
Find the exact area under the curve $y = e^x$ from $x = 0$ to $x = 1$.\\
Since the function $e^x$ is non-negative on the interval $[0, 1]$, the area can be expressed as the definite integral
\[\int_0^1 e^x \ dx.\]
We can use the Fundamental Theorem of Calculus to compute the value of this definite integral as follows:
\[\int_0^{1} e^x \ dx = e^x \Big|_0^{1} = e^{1} - e^0= e- 1.\]


\begin{image}
\begin{tikzpicture}
\begin{axis}[axis x line=  bottom, axis y line = left,xtick={0, .5, 1}, ytick={0, .5, 1,1.5, 2, 2.5}, title={The area under $f(x)=e^x$ on $[0,1]$}]
\addplot[name path = A, domain=0:1, 
    samples=100, color=black]{e^x};
\addplot[name path = B, domain=0:1, samples=100, color=black]{0};
\addplot[blue!10] fill between[of=A and B];

\node at (axis cs: .5,.7){Area = $e-1$};

\end{axis}
%\legend{$y = f(x)$, , secant line, tangent line, };
\end{tikzpicture}
\end{image}

\end{example}

\begin{problem}(problem 3a)
 Find the exact area under the graph of $y = e^x$ from $x=1$ to $x = \ln(5)$.\\
The area is $\answer{5-e}$.
 \begin{hint}
 \[
 \text{Area} = \int_1^{\ln(5)} e^x \ dx 
  \]
  \end{hint}
    \begin{hint}
      An anti-derivative of $e^x$ is itself
    \end{hint}
    
    \begin{hint}
      The Fundamental Theorem says:
      \[
      \int_a^b f(x) \ dx = F(b) - F(a)
      \]
    \end{hint}   
		\begin{hint}
      $e^{\ln(a)} = a$ if $a > 0$
    \end{hint}
\end{problem}



%\begin{example} %example 3
%$\displaystyle{\int_1^2 x^2 \ dx = \frac{x^3}{\mbox{\footnotesize $3$}} \Bigg|_1^2 = 
%\tfrac{\hphantom{|}2^3}{\hphantom{|}3^\hphantom{3}} - \tfrac{\hphantom{|}1^3}{\hphantom{|}3^\hphantom{3}} = \tfrac{7}{3}}$.
%\end{example}

\begin{problem}(problem 3b)
Find the exact area under the graph of $y = x^3$ from $x = 2$ to $x = 4$.\\
The area is $\answer{60}$.
 \begin{hint}
 \[
  \text{Area}=\int_2^{4} x^3 \ dx
  \]
  \end{hint}
    \begin{hint}
      An anti-derivative of $x^3$ is $x^4 /4$
    \end{hint}
    \begin{hint}
      The Fundamental Theorem says:
      \[
      \int_a^b f(x) \ dx = F(b) - F(a)
      \]
    \end{hint}    
		
		
\end{problem}



\begin{example}[example 4]
Find the exact area under the curve $y = 1/x$ from $x=1$ to $x= e$.\\
Since the function $1/x$ is non-negative on the interval $[1, e]$, the exact area can be expressed as the definite integral
\[\int_1^{e} \frac{1}{x} \ dx.\]
We can use the Fundamental Theorem of Calculus to compute the value of this definite integral as follows:
\[
\int_1^e \frac{1}{x} \ dx = \ln|x| \Big|_1^e = \ln(e) - \ln(1)= 1- 0 = 1.
\]


\begin{image}
\begin{tikzpicture}
\begin{axis}[axis x line=  bottom, axis y line = left, xtick={0,1,2,2.72, 3}, xticklabels ={0, 1, 2, $e$, 3}, ytick={0, .5, 1}, title={The Area Under $f(x)=\dfrac{1}{x}$ on $[1,e]$}]
\addplot[name path = A, domain=1:e, 
    samples=100, color=black]{1/x};
		\addplot[domain=0:1, 
    samples=100, color=black]{0};
\addplot[name path = B, domain=1:e, samples=100, color=black]{0};
\addplot[blue!10] fill between[of=A and B];
\addplot[domain=.75:1, 
    samples=100, color=black]{1/x};
\addplot[domain=e:3, 
    samples=100, color=black]{1/x};

\node at (axis cs: 1.8,.3){Area = $1$};
\addplot[thin, samples = 100, color=black] coordinates {(1,0) (1,1)};
\addplot[thin, samples = 100, color=black] coordinates {(e,0) (e,1/e)};


\end{axis}
%\legend{$y = f(x)$, , secant line, tangent line, };
\end{tikzpicture}
\end{image}

\end{example}



\begin{problem}(problem 4)
Find the exact area under the graph of $y = 1/x$ from $x = 2$ to $x = 5$.\\
The area is $\answer{\ln(5/2)}$.
 \begin{hint}
  \[
  \text{Area} = \int_2^{5} \frac{1}{x} \ dx
  \]
  \end{hint}
    \begin{hint}
      An anti-derivative of $1/x$ is $\ln(x)$
    \end{hint}
    \begin{hint}
      The Fundamental Theorem says:
      \[
      \int_a^b f(x) \ dx = F(b) - F(a)
      \]
    \end{hint}    
		
		
\end{problem}


\begin{example}[example 5]
Find the exact area under the curve $y = 2x+1$ from $x=0$ to $x= 2$.\\
Since the function $2x+1$ is non-negative on the interval $[0, 2]$, the exact area can be expressed as the definite integral
\[\int_0^2 (2x+1) \ dx.\]
We can use the Fundamental Theorem of Calculus to compute the value of this definite integral as follows:
\[
\int_0^2 (2x+ 1) \ dx = (x^2 + x) \Big|_0^2 = (2^2 + 2) - (0^2 + 0) = 6.
\]



\end{example}


\begin{problem}(problem 5)
Find the exact area under the graph of $y = 4x-3$ from $x = 1$ to $x = 3$.\\
The area is $\answer{10}$.
 \begin{hint}
  \[
  \text{Area} =  \int_1^3 (4x - 3) \ dx
  \]
 \end{hint}
    \begin{hint}
      An anti-derivative of $4x-3$ is $2x^2 - 3x$
    \end{hint}
    \begin{hint}
      The Fundamental Theorem says:
      \[
      \int_a^b f(x) \ dx = F(b) - F(a)
      \]
    \end{hint}    
		
		
\end{problem}


\begin{example}[example 6]
Find the exact area under the curve $y = \sec^2(x)$ from $x=0$ to $x= \pi/4$.\\
Since the function $\cos(x)$ is non-negative on the interval $[0, \frac{\pi}{4}]$, the exact area can be expressed as the definite integral
\[\int_0^{\pi/4} \sec^2(x) \ dx.\]
We can use the Fundamental Theorem of Calculus to compute the value of this definite integral as follows:
\[
\int_0^{\pi/4} \sec^2(x) \ dx = \tan(x) \Big|_0^{\pi/4} = \tan(\tfrac{\pi}{4}) - \tan(0) = 1-0 =1.
\] 


\begin{image}
\begin{tikzpicture}
\begin{axis}[axis x line=  bottom, axis y line = left,xtick={0, .785},
xticklabels={$0$, $\frac{\pi}{4}$}, ytick={0, 1, 2}, title={The Area Under $f(x)=\sec^2(x)$ on $[0,\pi/4]$}]
\addplot[name path = A, domain=0:.785, samples=100, color=black]{1/cos(deg(x))^2};
\addplot[name path = B, domain=0:.785, samples=100, color=black]{0};
\addplot[blue!10] fill between[of=A and B];
\addplot[thin, samples = 100, color=black] coordinates {(0,0) (0,1)};
\addplot[thin, samples = 100, color=black] coordinates {(.785,0) (.785,2)};
\node at (axis cs: .38,.6){Area = $1$};
\end{axis}
%\legend{$y = f(x)$, , secant line, tangent line, };
\end{tikzpicture}
\end{image}




\end{example}

\begin{problem}(problem 6)
Find the exact area under the graph of $y = \csc^2(x)$ from $x = \pi/4$ to $x = \pi/2$.\\
The area is $\answer{1}$.
 \begin{hint}
  \[
  \text{Area} = \int_{\pi/4}^{\pi/2} \csc^2(x) \ dx
  \]
\end{hint}  
    \begin{hint}
      An anti-derivative of $\csc^2(x)$ is $-\cot(x)$
    \end{hint}
    
    \begin{hint}
      The Fundamental Theorem says:
      \[
      \int_a^b f(x) \ dx = F(b) - F(a)
      \]
    \end{hint}    
		\begin{hint}
      $\cot(x) = \frac{\cos(x)}{\sin(x)}$
    \end{hint}
		
\end{problem}


\begin{example}[example 7]
Find the exact area under the curve $y = \dfrac{1}{1+x^2}$ from $x=0$ to $x= 1$.\\
Since the function $\frac{1}{1+x^2}$ is non-negative on the interval $[0, 1]$, the exact area can be expressed as the definite integral
\[\int_0^{1} \frac{1}{1+x^2} \ dx.\]
We can use the Fundamental Theorem of Calculus to compute the value of this definite integral as follows:
\[
\int_0^{1} \frac{1}{1+x^2} \ dx = \tan^{-1}(x) \Big|_0^1 
= \tan^{-1}(1)- \tan^{-1}(0) = \tfrac{\pi}{4}-0 =\tfrac{\pi}{4}.
\]


\begin{image}
\begin{tikzpicture}
\begin{axis}[axis x line=  bottom, axis y line = center,xtick={-3, -2, -1, 0, 1, 2, 3}, ytick={0, .5, 1,1.5, 2, 2.5}, title={The Area Under $f(x)=\dfrac{1}{1+x^2}$ on $[0,1]$}]
\addplot[name path = A, domain=0:1, 
    samples=100, color=black]{1/(1+x^2)};
\addplot[domain=0:1, 
    samples=100, color=black]{1/(1+x^2)};
\addplot[name path = B, domain=0:1, samples=100, color=black]{0};
\addplot[thin, samples = 100, color=black] coordinates {(1,0) (1,.5)};
%\addplot[thin, samples = 100, color=black] coordinates {(0,1) (0,1.5)};
\addplot[blue!10] fill between[of=A and B];
%\addplot[thin, samples = 100, color=black] coordinates {(e,0) (e,1/e)};
%xtick={0, $\pi/4$,$\pi/2$},
\node at (axis cs: .5,.4){Area = $\frac{\pi}{4}$};
\end{axis}
%\legend{$y = f(x)$, , secant line, tangent line, };
\end{tikzpicture}
\end{image}


\end{example}

\begin{problem}(problem 7)
Find the exact area under the graph of $y = \dfrac{1}{\sqrt{1-x^2}}$ from $x = 0$ to $x = 1$.\\
The area is $\answer{\pi/2}$.
 \begin{hint}
  \[
  \text{Area} =  \int_0^1 \frac{1}{\sqrt{1-x^2}} \ dx
  \]
 \end{hint} 
    \begin{hint}
      An anti-derivative of $\frac{1}{\sqrt{1-x^2}}$ is $\sin^{-1}(x)$
    \end{hint}
    
    \begin{hint}
      The Fundamental Theorem says:
      \[
      \int_a^b f(x) \ dx = F(b) - F(a)
      \]
    \end{hint}    
		\begin{hint}
      $\sin^{-1}(0) = 0$ and $\sin^{-1}(1) = \frac{\pi}{2}$ 
    \end{hint}
		
		
\end{problem}


We now consider the definite integral of negative functions.
If $f(x) \leq 0$ for all $x$ in the interval $[a,b]$, then 
\[\sum_{i=1}^n f(x_i^*) \Delta x \leq 0,\]
for any Riemann Sum of $f(x)$ on $[a, b]$.
Hence, the definite integral,
\[
\int_a^b f(x) \ dx = \lim_{n \to \infty} \sum_{i=1}^n f(x_i^*) \Delta x \leq 0
\]
as well. How should we interpret the definite integral in this case? The graph of $y = f(x)$ will lie below the $x$-axis, 
and the definite integral will equal $-1$ times the area of the region above the curve. 

\begin{proposition}
If $f(x)$ is continuous on the interval $[a, b]$ and if $f(x) \leq 0$ on $[a,b]$,
then 
\[ \int_a^b f(x) \ dx \leq 0,\]

and 

\[\int_a^b f(x) \ dx = (-1) \cdot \text{area above the curve}.\]
\end{proposition}

To put it another way, if $f(x) \leq 0$ on $[a,b]$ then
\[\text{area above the curve} = -\int_a^b f(x) \ dx.\]

\begin{example}[example 8]
Find the exact area above the graph of $y = x^3$ from $x = -1$ to $x = 0$. \\
Since the function $x^3$ is negative (or zero) on the interval $[-1, 0]$ , the exact area above the curve is the negative of the definite integral 
\[\int_{-1}^0 x^3 \ dx.\]
We can compute the value of this definite integral using the Fundamental Theorem of Calculus as follows:
\[\int_{-1}^0 x^3 \ dx = \frac{x^4}{4} \Big|_{-1}^0 = \frac{0^4}{4} - \frac{(-1)^4}{4} = -\frac14.\]

Hence the area of the region above the curve is $\frac14$.


\begin{image}
\begin{tikzpicture}
\begin{axis}[axis x line=  center, axis y line = center, xtick={-1, 0, 1},  
ytick={-1, 0, 1}, title={The Area Above $f(x)=x^3$ on $[-1,0]$}]
\addplot[name path = B, domain=-1:0, samples=100, color=black]{x^3};
\addplot[domain=0:1, samples=100, color=black]{x^3};
\addplot[name path = A, domain=-1:0, samples=100, color=black]{0};
\addplot[red!10] fill between[of=A and B];


\node at (axis cs: -.6,.1){Area = 1/4};

\addplot[thick, samples = 100, color=black] coordinates {(-1,0) (-1,-1)};

\end{axis}
%\legend{$y = f(x)$, , secant line, tangent line, };
\end{tikzpicture}
\end{image}
\end{example}


\begin{problem}(problem 8)
Find the exact area above the graph of $y = x^3$ from $x = -2$ to $x = -1$. \\
\begin{hint}
Area above the curve $= -\int_a^b f(x) \ dx$
\end{hint}
\[\int_{-2}^{-1} x^3 \ dx = \answer{-15/4},\]
\[\text{area above the curve} = \answer{15/4}.\]
\end{problem}


\begin{example}[example 9]
Find the exact area above the graph of $y = x^2 - 2$ from $x = 0$ to $x = 1$. \\
Since the function $x^2 - 2 \leq 0$  on the interval $[0,1]$ , the exact area above the curve \textit{is the negative of} the definite integral 
\[\int_0^1 (x^2 -2) \ dx.\]
We can compute the value of this definite integral using the Fundamental Theorem of Calculus as follows:
\[\int_0^1 (x^2 -2) \ dx = \left(\frac{x^3}{3}-2x\right) \Bigg|_{0}^1 = \left(\frac{1^3}{3}-2(1)\right) - \left(\frac{0^3}{3}- 2(0)\right) = \frac13 - 2 =-\frac53.\]

Hence the area of the region above the curve is $-\left(-\frac53\right) = \frac53$.

%xticklabel pos=upper,
\begin{image}
\begin{tikzpicture}
\begin{axis}[axis x line=  center, axis y line = center, xtick={-1, 0, 1},  
ytick={-2, -1, 0, 1},  title={The Area Above $f(x)=x^2 -2$ on $[0,1]$}]
\addplot[name path = B, domain=0:1, 
    samples=100, color=black]{x^2 - 2};
%\addplot[domain=-1:0, samples=100, color=black]{x^2 - 2};
\addplot[name path =A,domain=0:1, samples = 100, color = black]{0};
\addplot[domain=-1:0, samples = 100, color = black]{0};
\addplot[red!10] fill between [of=A and B];

\node at (axis cs: .5,-.7){Area = 5/3};

\addplot[thin, samples = 100, color=black] coordinates {(1,0) (1,-1)};
\addplot[thin, samples = 100, color=black] coordinates {(0,0) (0,1)};
\end{axis}
%\legend{$y = f(x)$, , secant line, tangent line, };
\end{tikzpicture}
\end{image}
\end{example}


\begin{problem}(problem 9)
Find the exact area above the graph of $y = 1 - \sqrt x$ from $x = 1$ to $x = 4$. \\
\begin{hint}
Area above the curve $= -\int_a^b f(x) \ dx$
\end{hint}
\[\int_1^4 1 - \sqrt x \ dx = \answer{-2/3},\]
\[\text{area above the curve} = \answer{2/3}.\]
\end{problem}


\begin{example}[example 10]
Find the exact area above the graph of $y = -\sin(x)$ from $x = 0$ to $x = \pi$. \\
Since the function $-\sin(x) \leq 0$ on the interval $[0, \pi]$ , the exact area above the curve is \textit{the negative of} the definite integral 
\[\int_{0}^\pi -\sin(x) \ dx.\]
We can compute the value of this definite integral using the Fundamental Theorem of Calculus as follows:
\[\int_0^\pi -\sin(x) \ dx = \cos(x) \Big|_{0}^\pi = \cos(\pi) - \cos(0) = -1 -1 = -2.\]

Hence the area of the region above the curve is $2$.


\begin{image}
\begin{tikzpicture}
\begin{axis}[axis x line=  center, axis y line = center, xtick={0, 1.57, 3.14}, 
xticklabels={0, $\frac{\pi}{2}$, $\pi$}, 
ytick={-1, 0, 1}, title={The Area Above $f(x)=-\sin(x)$ on $[0,\pi]$}]
\addplot[name path = B, domain=0:3.14, 
    samples=100, color=black]{-sin(deg(x))};
\addplot[name path = A, domain=0:3.14, samples=100, color=black]{0};
\addplot[red!10] fill between[of=A and B];


\node at (axis cs: 1.57,-.5){Area = 2};

\addplot[thin, samples = 100, color=black] coordinates {(0,0) (0,1)};

\end{axis}
%\legend{$y = f(x)$, , secant line, tangent line, };
\end{tikzpicture}
\end{image}
\end{example}



\begin{problem}(problem 10a)
Find the exact area above the graph of $y = \cos(x)$ from $x = \frac{\pi}{2}$ to $x = \frac{3\pi}{2}$. \\
\begin{hint}
Area above the curve $= -\int_a^b f(x) \ dx$
\end{hint}
\[\int_{\pi/2}^{3\pi/2} \cos(x) \ dx = \answer{-2},\]
\[\text{area above the curve} = \answer{2}.\]
\end{problem}


\begin{problem}(problem 10b)
Find the exact area above the graph of $y = \dfrac{1}{x}$ from $x = -3$ to $x = -1$. \\
\begin{hint}
Area above the curve $= -\int_a^b f(x) \ dx$
\end{hint}
\[\int_{-3}^{-1} \frac{1}{x} \ dx = \answer{-\ln(3)},\]
\[\text{area above the curve} = \answer{\ln(3)}.\]
\end{problem}



\begin{center}
\begin{foldable}
\unfoldable{Here are some detailed, lecture style videos on the definite integral:}
\youtube{Xrot6lPfcKQ}
\youtube{aGCbdHPo3kE}
\end{foldable}
\end{center}

\section{More Definite Integrals}


\begin{problem}(problem 11) Compute the definite integrals:
\begin{itemize}
\item[11a)] \; $\displaystyle{\int_0^2 \left(t^2 - 5t + 2\right) \; dt = \answer{-10/3}}$
\item[11b)] \; $\displaystyle{\int_1^ 3 \left(4u - \frac{5}{u}\right) \; du = \answer{16 - 5\ln(3)}}$
\item[11c)] \; $\displaystyle{\int_1^4 \frac{r + 1}{\sqrt r} \; dr = \answer{20/3}}$
\item[11d)] \; $\displaystyle{\int_0^\pi \left[3\cos(\theta) - 4\sin(\theta)\right] \; d\theta = \answer{-8}}$
\item[11e)] \; $\displaystyle{\int_0^1 2^x \; dx = \answer{1/\ln(2)}}$
\end{itemize}
\end{problem}



\section{Definite Integrals and Substitution}
We now compute definite integrals that require a u-substitution. The key is to change the limits of integration when we change the variable.
Suppose that
\[
\int f(x) \; dx = F(x) + C.
\]
Now, consider the definite integral
\[
\int_a^b f(g(x)) g'(x) \; dx.
\]
We will make the substitution, $u = g(x)$, so that $du = g'(x) \; dx$.
When we convert to the integral in terms of the variable $u$, we will change the limits of integration from $x = a$ and $x = b$ 
to $u = g(a)$ and $u = g(b)$ respectively. This will give us
\[\int_a^b f(g(x)) g'(x) \; dx = \int_{g(a)}^{g(b)} f(u) \; du = F(u)\bigg|_{g(a)}^{g(b)} = F(g(b)) - F(g(a)).\]
In this way, we do not need to go basck to the variable $x$, as we did with a substitution in an indefinite integral.

\begin{example}[example12]
Compute the definite integral 
\[
\int_0^1 x(x^2 + 2)^3 \; dx.
\]
We let $u = x^2 + 2$ so that $du = 2x \; dx$. Now we change the limits of integration: 
\[
\text{when} \quad x = 0 \quad \text{we have} \quad u = 0^2 + 2 = 2  \quad \text{and}
\]
\[ 
\text{when} \quad x = 1 \quad \text{we have} \quad u = 1^2 + 2 = 3.
\]
Thus
\[
\int_1^2 x(x^2 + 2)^3 \; dx = \frac12 \int_2^3 u^3 \; du = \frac{u^4}{8}\bigg|_2^3 = \frac{3^4}{8} - \frac{2^4}{8} = \frac{65}{8}.
\]
\end{example}

\begin{problem}(problem 12) Compute the definite integrals:
\begin{itemize}
\item[12a)] \; $\displaystyle{\int_{-1}^1 x^2(x^3 + 1)^4 \; dx = \answer{32/15}}$
\item[12b)] \; $\displaystyle{\int_0^1 \frac{x}{x^2 + 1}\; dx = \answer{\ln(2)/2}}$
\item[12c)] \; $\displaystyle{\int_0^{\pi/2} e^{\sin(x)} \cos(x) \; dx = \answer{e-1}}$
\end{itemize}
\end{problem}


\end{document}









\begin{problem}(problem 9b)
Use geometry to find the value of the definite integral $\displaystyle{\int_{-4}^{4}-\sqrt{16-x^2} \ dx}.$
\begin{hint}
$-\sqrt{16-x^2} \leq 0$ on the interval $[-4,4]$.
\end{hint}
\begin{hint}
The definite integral gives $(-1) \cdot$ area above the semi-circle
\end{hint}
\[\int_{-4}^{4}-\sqrt{16-x^2} \ dx = \answer{-8\pi}.\]
\end{problem}




\begin{image}
\begin{tikzpicture}[scale=.75]
\draw[->](0,0)--(1.5,0);
\draw[->](0, 0)--(0, 1.5);
\draw (0,0) parabola (1.1, 1.21) node[right] {$y=x^2$};
\draw (1,1) -- (1,0) node[below] {$1$};
%\node at (0.5, 0.5) {$y = x^2$};
%\node at (1, -0.25) {$1$};
\end{tikzpicture}
\end{image}



\begin{example} %example 8
$\begin{aligned}[t]
\int_{-1}^1 (3x+ 5) \ dx &= \left(\tfrac32 x^2 + 5x \right) \bigg|_{-1}^1 \\
&= \left(\tfrac32 (1^2) + 5(1)\right) - \left(\tfrac{3}{2} (-1)^2 + 5(-1) \right) \\ 
&= \left(\tfrac32 + 5\right) - \left(\tfrac32 - 5\right)\\
&= 10.
\end{aligned}$
\end{example}


\begin{example} %example 8
$\begin{aligned}[t]
\int_{-1}^1 (3x+ 5) \ dx &= \left(\tfrac32 x^2 + 5x \right) \bigg|_{-1}^1 \\
&= \left(\tfrac32 (1^2) + 5(1)\right) - \left(\tfrac{3}{2} (-1)^2 + 5(-1) \right) \\ 
&= \left(\tfrac32 + 5\right) - \left(\tfrac32 - 5\right)\\
&= 10.
\end{aligned}$
\end{example}


%\begin{problem} %problem #2
 % \[
  %\int_1^{\ln(5)} e^x \ dx=\answer[given]{5-e}
  %\]
  
   % \begin{hint}
    %  An antiderivative of $e^x$ is itself
    %\end{hint}
    %\begin{hint}
     % $e^{\ln(a)} = a$ if $a > 0$
    %\end{hint}
    %\begin{hint}
     % The Fundamental Theorem says:
     % \[
     % \int_a^b f(x) \ dx = F(b) - F(a)
     % \]
    %\end{hint}    
		
		
%\end{problem}
