\documentclass{ximera}

%% You can put user macros here
%% However, you cannot make new environments



\newcommand{\ffrac}[2]{\frac{\text{\footnotesize $#1$}}{\text{\footnotesize $#2$}}}
\newcommand{\vasymptote}[2][]{
    \draw [densely dashed,#1] ({rel axis cs:0,0} -| {axis cs:#2,0}) -- ({rel axis cs:0,1} -| {axis cs:#2,0});
}


\graphicspath{{./}{firstExample/}}

\usepackage{amsmath}
\usepackage{amssymb}
\usepackage{array}
\usepackage[makeroom]{cancel} %% for strike outs
\usepackage{pgffor} %% required for integral for loops
\usepackage{tikz}
\usepackage{tikz-cd}
\usepackage{tkz-euclide}
\usetikzlibrary{shapes.multipart}


\usetkzobj{all}
\tikzstyle geometryDiagrams=[ultra thick,color=blue!50!black]


\usetikzlibrary{arrows}
\tikzset{>=stealth,commutative diagrams/.cd,
  arrow style=tikz,diagrams={>=stealth}} %% cool arrow head
\tikzset{shorten <>/.style={ shorten >=#1, shorten <=#1 } } %% allows shorter vectors

\usetikzlibrary{backgrounds} %% for boxes around graphs
\usetikzlibrary{shapes,positioning}  %% Clouds and stars
\usetikzlibrary{matrix} %% for matrix
\usepgfplotslibrary{polar} %% for polar plots
\usepgfplotslibrary{fillbetween} %% to shade area between curves in TikZ



%\usepackage[width=4.375in, height=7.0in, top=1.0in, papersize={5.5in,8.5in}]{geometry}
%\usepackage[pdftex]{graphicx}
%\usepackage{tipa}
%\usepackage{txfonts}
%\usepackage{textcomp}
%\usepackage{amsthm}
%\usepackage{xy}
%\usepackage{fancyhdr}
%\usepackage{xcolor}
%\usepackage{mathtools} %% for pretty underbrace % Breaks Ximera
%\usepackage{multicol}



\newcommand{\RR}{\mathbb R}
\newcommand{\R}{\mathbb R}
\newcommand{\C}{\mathbb C}
\newcommand{\N}{\mathbb N}
\newcommand{\Z}{\mathbb Z}
\newcommand{\dis}{\displaystyle}
%\renewcommand{\d}{\,d\!}
\renewcommand{\d}{\mathop{}\!d}
\newcommand{\dd}[2][]{\frac{\d #1}{\d #2}}
\newcommand{\pp}[2][]{\frac{\partial #1}{\partial #2}}
\renewcommand{\l}{\ell}
\newcommand{\ddx}{\frac{d}{\d x}}

\newcommand{\zeroOverZero}{\ensuremath{\boldsymbol{\tfrac{0}{0}}}}
\newcommand{\inftyOverInfty}{\ensuremath{\boldsymbol{\tfrac{\infty}{\infty}}}}
\newcommand{\zeroOverInfty}{\ensuremath{\boldsymbol{\tfrac{0}{\infty}}}}
\newcommand{\zeroTimesInfty}{\ensuremath{\small\boldsymbol{0\cdot \infty}}}
\newcommand{\inftyMinusInfty}{\ensuremath{\small\boldsymbol{\infty - \infty}}}
\newcommand{\oneToInfty}{\ensuremath{\boldsymbol{1^\infty}}}
\newcommand{\zeroToZero}{\ensuremath{\boldsymbol{0^0}}}
\newcommand{\inftyToZero}{\ensuremath{\boldsymbol{\infty^0}}}


\newcommand{\numOverZero}{\ensuremath{\boldsymbol{\tfrac{\#}{0}}}}
\newcommand{\dfn}{\textbf}
%\newcommand{\unit}{\,\mathrm}
\newcommand{\unit}{\mathop{}\!\mathrm}
%\newcommand{\eval}[1]{\bigg[ #1 \bigg]}
\newcommand{\eval}[1]{ #1 \bigg|}
\newcommand{\seq}[1]{\left( #1 \right)}
\renewcommand{\epsilon}{\varepsilon}
\renewcommand{\iff}{\Leftrightarrow}

\DeclareMathOperator{\arccot}{arccot}
\DeclareMathOperator{\arcsec}{arcsec}
\DeclareMathOperator{\arccsc}{arccsc}
\DeclareMathOperator{\si}{Si}
\DeclareMathOperator{\proj}{proj}
\DeclareMathOperator{\scal}{scal}
\DeclareMathOperator{\cis}{cis}
\DeclareMathOperator{\Arg}{Arg}
%\DeclareMathOperator{\arg}{arg}
\DeclareMathOperator{\Rep}{Re}
\DeclareMathOperator{\Imp}{Im}
\DeclareMathOperator{\sech}{sech}
\DeclareMathOperator{\csch}{csch}
\DeclareMathOperator{\Log}{Log}

\newcommand{\tightoverset}[2]{% for arrow vec
  \mathop{#2}\limits^{\vbox to -.5ex{\kern-0.75ex\hbox{$#1$}\vss}}}
\newcommand{\arrowvec}{\overrightarrow}
\renewcommand{\vec}{\mathbf}
\newcommand{\veci}{{\boldsymbol{\hat{\imath}}}}
\newcommand{\vecj}{{\boldsymbol{\hat{\jmath}}}}
\newcommand{\veck}{{\boldsymbol{\hat{k}}}}
\newcommand{\vecl}{\boldsymbol{\l}}
\newcommand{\utan}{\vec{\hat{t}}}
\newcommand{\unormal}{\vec{\hat{n}}}
\newcommand{\ubinormal}{\vec{\hat{b}}}

\newcommand{\dotp}{\bullet}
\newcommand{\cross}{\boldsymbol\times}
\newcommand{\grad}{\boldsymbol\nabla}
\newcommand{\divergence}{\grad\dotp}
\newcommand{\curl}{\grad\cross}
%% Simple horiz vectors
\renewcommand{\vector}[1]{\left\langle #1\right\rangle}


\outcome{Compute limits using algebraic techniques}

\title{1.4 Computing Limits}


\begin{document}

\begin{abstract}
Compute limits using algebraic techniques.
\end{abstract}

\maketitle


\section{Limit laws}
 
In section 1.2, Numerical Limits, we briefly discussed ``plugging in" to compute limits.
 The example we investigated was
 \[
 \lim_{x \to 3^+} \left(x^2 + 2 \right) = 3^2 + 2 = 11.
 \]

The reason plugging in worked in the previous example is a direct consequence of the limit laws presented below.
 
In each of the following laws, all of the limits are assumed to exist.

\begin{enumerate}

\item[1.]  
The limit of a constant is the constant:
 \[
 \lim_{x \to c} k = k.
 \]
 
\item[2.]
 The next law is self-evident:
 \[
 \lim_{x \to c} x = c.
 \]
 
 \item[3.]
 The limit of a multiple of a function is the multiple of the limit:
 \[
 \lim_{x \to c} kf(x) = k \cdot \lim_{x \to c} f(x). 
 \]
 
 \item[4.]
 The limit of a sum is the sum of the limits:
 \[
 \lim_{x \to c} \left[f(x) + g(x) \right] = \lim_{x \to c} f(x) + \lim_{x \to c} g(x). 
 \]
 
\item[5.]
 The limit of a difference is the difference of the limits:
 \[
 \lim_{x \to c} \left[f(x) - g(x) \right] = \lim_{x \to c} f(x) - \lim_{x \to c} g(x). 
 \]
 
\item[6.]
 The limit of a product is the product of the limits:
 \[
 \lim_{x \to c} f(x) g(x) = \lim_{x \to c} f(x) \cdot \lim_{x \to c} g(x). 
 \]


\item[7.]
 The limit of a quotient is the quotient of the limits:
 \[
 \lim_{x \to c} \frac{f(x)}{g(x)} = \frac{\lim_{x \to c} f(x)}{\lim_{x \to c} g(x)}, 
 \]
 provided that the limit in the denominator is \textbf{not equal to zero}.
 
 \item[8.]
 Limits can be moved inside of radicals:
 \[
 \lim_{x \to c} \sqrt[n]{f(x)} = \sqrt[n]{\lim_{x \to c} f(x)}. 
 \]

\end{enumerate}


Each of the above limit laws is valid if $x \to c^+, x\to c^-, x \to \infty$ or $x \to -\infty$.



\begin{example}[example 1] Given that 
\[
\lim_{x \to 3} f(x) = -5 \quad \text{and} \quad \lim_{x \to 3} g(x) = \tfrac12,
\]
find 
\[
\lim_{x \to 3} \left[x^2f(x) - 6g(x)\right].
\]
Using limit laws 2, 3 and 5, we can write
\[
\lim_{x \to 3} \left[x^2f(x) - 6g(x)\right] = \lim_{x \to 3} x^2 \cdot \lim_{x \to 3}f(x) - 6\cdot\lim_{x \to 3} g(x) = 9(-5) - 6(\tfrac12) = -48.
\]
\end{example}


\begin{problem}(problem 1a)
Given that 
\[
\lim_{x \to -2} f(x) = 2 \quad \text{and} \quad \lim_{x \to -2} g(x) = 0,
\]
\[
\lim_{x \to -2} \left[3f(x) + xg(x)\right] = \answer{6}.
\]
\end{problem}


\begin{problem}(problem 1b)
Given that 
\[
\lim_{x \to 1^+} f(x) = \tfrac23 \quad \text{and} \quad \lim_{x \to 1+} g(x) = 4,
\]
\[
\lim_{x \to 1^+} \frac{f(x)}{\sqrt{g(x)}} = \answer{1/3}.
\]
\end{problem}


\begin{problem}(problem 1c)
Given that 
\[
\lim_{x \to \infty} f(x) = -9 \quad \text{and} \quad \lim_{x \to \infty} g(x) = 1,
\]
\[
\lim_{x \to \infty} \sqrt[3]{f(x) + g(x)} = \answer{-2}.
\]
\end{problem}


%\section{Computing Limits}
 %\graph command must contain all 4 boundaries if it contains any (xmin, xmax, ymin, ymax)

%\[
%\graph[xmin=-2, xmax=4, ymin=-2, ymax=6]{(x-2)^2 + (y-4)^2 = 0.01, x^2 \left{-1\leq x \leq 1.97\right}}
%\]
%\[\graph[xAxisLabel=x, yAxisLabel=y, xmin=-3, xmax=10, ymin=-10, ymax=10]{y = 1/x}\]

%\[
%\graph[xAxisLabel=x, yAxisLabel=y, xmin=-3, xmax=10, ymin=-10, ymax=10]{y = 1/x, y = x/2}
%\]

%\[
%\graph{ \sin(x)\left\{x<0\right\}, 2x\left\{ x>=0 \right\} }
%\]

%\[
%\graph[color=Desmos.Colors.BLUE]{y = 1/x, y = x/2}
%\]
%\[\graph{x^2 \left\{ -1 \leq x \leq 1.97 \right\}, (x-2)^2 + (y-4)^2 = 0.01 }\]

%\[
%\graph{(2, 3)}
%\]

%\geogebra{mG34YseZ}{640}{480}
%\geogebra{PMcY6Q8j}{640}{480}


%We have seen examples of limit problems where plugging in the terminal $x$-value both led to 
%reasonable answers and meaningless forms.
%In the section we will work with examples in which plugging in the terminal $x$-value initially 
%yields a meaningless expression,
%but after performing algebraic manipulations on the function in the problem, 
%plugging in the terminal value
%yields a credible solution.



\section{Factor and cancel}

\begin{example}[example 2]
Compute the limit: \[\lim_{x \to 4} \frac{x^2 - 16}{x^2 - 4x}.\]


Plugging in the terminal value, $x=4$, yields 
the indeterminate form $\frac00$.  To find this limit analytically, we will factor the numerator 
and denominator and simplify the fraction. In the numerator we have a \textbf{difference of squares}, and 
such a form always factors in the following way:
\[a^2 - b^2 = (a+b)(a-b).\]
Applying this general formula to our example, we get 
\[x^2 - 16 = (x+4)(x-4).\]  
Next, in the denominator, we can factor out a common factor
of $x$ from each of the terms:
\[x^2 - 4x = x(x-4).\]
With these factorizations, we can simplify the fraction and find the limit:

\begin{align*}
\lim_{x \to 4} \frac{x^2 - 16}{x^2 - 4x} &= \lim_{x \to 4} \frac{(x+4)(x-4)}{x(x-4)} \enspace & \text{(factoring)} \\
                                         &= \lim_{x \to 4} \frac{x+4}{x} & \text{(canceling)} \\
                                         &= \frac84  &  \\
																				 &= 2. &
\end{align*}																				
																				
Hence,
\[\lim_{x \to 4} \frac{x^2 - 16}{x^2 - 4x} = 2.\]
This example uses the \textbf{factor and cancel} method.
\end{example}

\begin{problem}(problem 2)
  Compute the limit:
  \[  \lim_{x \to 3} \frac{x^2 - 9}{x^2 - 3x} = \answer{2}.  \]
    \begin{hint}
      When you plug in $x = 3$, you get $\frac00$
    \end{hint}
    \begin{hint}
      Factor the numerator and the denominator
    \end{hint}
    \begin{hint}
      Difference of squares: $a^2 - b^2 = (a+b)(a-b)$
    \end{hint}
    \begin{hint}
      Common factor: $ab \pm ac = a(b \pm c)$
    \end{hint}
\end{problem}


\begin{example}[example 3]
Compute the limit: \[\lim_{x \to 2} \frac{x^2 + 3x - 10}{x^3 - 8}.\]
\\
Plugging in the terminal value $x = 2$ yields the indeterminate form $\frac00$.
The numerator factors  by finding two numbers which multiply to give $-10$ and add to give $+3$.  
The two numbers are 
$-2$ and $+5$, hence the factorization is  
\[x^2 + 3x - 10 = (x-2)(x+5).\]
In the denominator, we have a \textbf{difference of cubes}, and we use the formula:
\[a^3 - b^3 = (a-b)(a^2+ab +b^2).\]
Applying this to our example gives 
\[x^3 - 8 = (x-2)(x^2 + 2x + 4).\]
With these factorizations, we can simplify the fraction and find the limit:

\begin{align*}
\lim_{x \to 2} \frac{x^2 + 3x - 10}{x^3 - 8} &= \lim_{x \to 2}\frac{(x-2)(x+5)}{(x-2)(x^2 + 2x + 4)} 
\enspace & \text{(factoring)} \\
&= \lim_{x \to 2} \frac{x+5}{x^2 + 2x + 4} & \text{(canceling)} \\
&= \frac{2+5}{2^2 + (2\cdot 2) + 4} & \text{(plugging in)} \\
&= \frac{7}{12}. 
\end{align*}

Hence,
\[\lim_{x \to 2} \frac{x^2 + 3x - 10}{x^3 - 8} = \frac{7}{12}.\]
\end{example}




\begin{problem}(problem 3a)
  Compute the following limit which we investigated numerically in the previous section:
  \[
  \lim_{x \to 1} \frac{x^3 - 1}{x^2 -1}.
  \]
  
    \begin{hint}
      When you plug in $x = 1$, you get $\frac00$
    \end{hint}
    \begin{hint}
      Factor the numerator and the denominator
    \end{hint}
    \begin{hint}
      Difference of cubes: $a^3 - b^3 = (a-b)(a^2 + ab +b^2)$
    \end{hint}
		\begin{hint}
      Difference of squares: $a^2 - b^2 = (a-b)(a+b)$
    \end{hint}

		The value of the limit is
		 $\answer[given]{\frac32}$
		
\end{problem}


\begin{problem}(problem 3b)
  Compute the limit:
  \[
  \lim_{x \to 3} \frac{x^3 - 27}{x^2 -2x - 3}.
  \]
  
    \begin{hint}
      When you plug in $x = 3$, you get $\frac00$
    \end{hint}
    \begin{hint}
      Factor the numerator and the denominator
    \end{hint}
    \begin{hint}
      Difference of cubes: $a^3 - b^3 = (a-b)(a^2 + ab +b^2)$
    \end{hint}

		The value of the limit is
		 $\answer[given]{\frac{27}{4}}$
		
\end{problem}


\begin{example}[example 4]
Compute the limit: \[\lim_{x \to 4} \frac{4x^2 - 19x + 12}{6x^2 -19x -20}.\]
\\
Plugging in the terminal value $x = 4$ yields the indeterminate form $\frac00$.
To factor the numerator and denominator in this case, we will use the important fact that
for a polynomial $p(x)$ and a number $a$, 
\[\text{If} \quad p(a) = 0, \quad \text{then} \quad  x-a \quad \text{is a factor of} \; p(x).\]
And once we know one factor of a polynomial, it is usually much easier to find the other.
Since plugging in $x=4$ gave $0$ in both the numerator and the denominator, both polynomials 
have $x-4$ as a factor.
In the numerator, the factorization is $(x-4)(4x-3)$ and in the denominator, 
the factorization is $(x-4)(6x+5)$.

With these factorizations, we can simplify the fraction and find the limit:

\begin{align*}
\lim_{x \to 4} \frac{4x^2 - 19x + 12}{6x^2 -19x -20} &= \lim_{x \to 4}\frac{(x-4)(4x-3)}{(x-4)(6x + 5)} 
\enspace & \text{(factoring)} \\
&= \lim_{x \to 4} \frac{4x-3}{6x + 5} & \text{(canceling)} \\
&= \frac{16-3}{24+5} & \text{(plugging in)}\\
&= \frac{13}{29}. 
\end{align*}

Hence,
\[\lim_{x \to 4} \frac{4x^2 -19x +12}{6x^2 -19x -20} = \frac{13}{29}.\]
\end{example}



\begin{problem}(problem 4a)
  Compute the limit:
  \[
  \lim_{x \to 2} \frac{3x^2 -8x + 4}{2x^2 + x - 10}.
  \]
  
    \begin{hint}
      When you plug in $x = 2$, you get $\frac00$
    \end{hint}
    \begin{hint}
      Factor the numerator and the denominator
    \end{hint}
    \begin{hint}
      If $p(a) = 0$, then $(x-a)$ is a factor
    \end{hint}
    \begin{hint}
      $x-2$ is a factor of both the numerator and the denominator
    \end{hint}
    
		The value of the limit is
		 $\answer[given]{4/9}$
		
\end{problem}


\begin{problem}(problem 4b)
  Compute the limit:
  \[
  \lim_{x \to -2} \frac{2x^2 + x - 6}{x^2 + 5x + 6}.
  \]
  
    \begin{hint}
      When you plug in $x = -2$, you get $\frac00$
    \end{hint}
    \begin{hint}
      Factor the numerator and the denominator
    \end{hint}
    \begin{hint}
      If $p(a) = 0$, then $(x-a)$ is a factor
    \end{hint}
    \begin{hint}
      $x+2$ is a factor of both the numerator and the denominator
    \end{hint}
    
		The value of the limit is
		 $\answer[given]{-7}$
		
\end{problem}


\section{Conjugate radicals}


%The difference of two squares formula, , 
%is helpful for dealing with limits involving radicals. 
%When dealing with radials, one of $a$ or $b$ (or possibly even both) 
%will involve a radical symbol, like $\sqrt{x+3}$.
The expressions 
\[\sqrt u + \sqrt v \quad \text{and} \quad \sqrt u - \sqrt v\]
 are called
\textbf{conjugate radicals}. When we multiply conjugate radicals
using the difference of squares formula, $(a+b)(a-b) = a^2 - b^2$,
we get an expression that is free of radicals: 

\[(\sqrt u + \sqrt v)(\sqrt u - \sqrt v) = (\sqrt u)^2 - (\sqrt v)^2 = u - v.\]

We will now take advantage of this technique to find limits.

\begin{example}[example 5]
Compute the limit:
\[\lim_{x \to 4} \frac{\sqrt{x} - 2}{x-4}.\]

To solve this limit problem, we will use the conjugate radical of the
numerator, which is $\sqrt{x} + 2$.  In order to maintain the value of
the expression given in the problem, we multiply by the conjugate radical
over itself (which is equal to one):



\begin{align*}
\lim_{x \to 4} \frac{\sqrt{x}- 2}{x-4} &=
\lim_{x \to 4} \frac{\sqrt{x} -2}{x-4}\cdot \frac{\sqrt{x} +2}{\sqrt{x}+2} \\
 &= \lim_{x \to 4} \frac{x-4}{(x-4)(\sqrt{x}+2)} \\
&= \lim_{x \to 4}\frac{1}{\sqrt{x}+2} \\
&= \frac 14.
\end{align*}

In the last step, we plugged in the terminal value $x=4$ to get the final answer $\frac14$.

\end{example}



\begin{problem}(problem 5)
  Compute the limit:
  \[
  \lim_{x \to 9} \frac{\sqrt{x}-3}{x-9}
  \]
  
    \begin{hint}
      When you plug in $x = 9$, you get $\frac00$
    \end{hint}
    \begin{hint}
      Multiply by the conjugate radical
    \end{hint}
    \begin{hint}
      $\sqrt a + \sqrt b$ and $\sqrt a - \sqrt b$ are conjugates
    \end{hint}
    \begin{hint}
      Use the difference of squares formula in the numerator: $(a-b)(a+b) = a^2 - b^2$ 
    \end{hint}
    
		The value of the limit is
		 $\answer[given]{1/6}$
		
\end{problem}



\begin{example}[example 6]
Compute the limit:
\[\lim_{x \to 1} \frac{x- \sqrt{x}}{x-1}.\]

To solve this limit problem, we will use the conjugate radical of the
numerator, which is $x+ \sqrt{x}$.  In order to not change the value of
the expression given in the problem, we multiply by the conjugate radical
divided by itself (which is equal to one):



\begin{align*}
\lim_{x \to 1} \frac{x- \sqrt{x}}{x-1} &=
\lim_{x \to 1} \frac{x- \sqrt{x}}{x-1}\cdot \frac{x+ \sqrt{x}}{x+ \sqrt{x}} \\
 &= \lim_{x \to 1} \frac{x^2- x}{(x-1)(x+ \sqrt{x})} \\
&= \lim_{x \to 1} \frac{x(x-1)}{(x-1)(x+ \sqrt{x})} \\
&= \lim_{x \to 1}\frac{x}{x+ \sqrt{x}} \\
&= \frac 12.
\end{align*}

In the last step, we plugged in the terminal value $x=1$ to get the final answer $\frac12$.

\end{example}


\begin{problem}(problem 6)
  Compute the limit:
  \[
  \lim_{x \to 2} \frac{x- \sqrt{2x}}{x-2}
  \]
  
    \begin{hint}
      When you plug in $x = 2$, you get $\frac00$
    \end{hint}
    \begin{hint}
      Multiply by the conjugate radical
    \end{hint}
    \begin{hint}
      $\sqrt a + \sqrt b$ and $\sqrt a - \sqrt b$ are conjugates
    \end{hint}
    \begin{hint}
      Use the difference of squares formula in the numerator: $(a-b)(a+b) = a^2 - b^2$ 
    \end{hint}
    
		The value of the limit is
		 $\answer[given]{\frac12}$
		
\end{problem}



\begin{example}[example 7]
Compute the limit: $\displaystyle{\lim_{x \to -3} \frac{x^2 - 9}{4 - \sqrt{13 -x}}}$.\\
\\
If we plug in $x = -3$, the numerator and denominator are both 0. The conjugate of the radical expression in the denominator is
\[ 4 + \sqrt{13 -x}.\]
To simplify our calculations a little bit, let's multiply the conjugates together separately:
\[(4 - \sqrt{13 -x})(4 + \sqrt{13 -x}) = 16 - (13-x) = 3+x = x+3.\]
With this in mind, we have:

\begin{align*}
\lim_{x \to -3} \frac{x^2 - 9}{4 - \sqrt{13 -x}} &= 
\lim_{x \to -3} \left(\frac{x^2 - 9}{4 - \sqrt{13 -x}}\right) \cdot \left(\frac{4 + \sqrt{13 -x}}{4 + \sqrt{13 -x}}\right) \\
&=\lim_{x \to -3} \frac{(x^2 - 9)(4 + \sqrt{13 -x})}{x+3}\\
&=\lim_{x \to -3} \frac{(x+3)(x-3)(4 + \sqrt{13 -x})}{x+3} \\
&= \lim_{x \to -3} (x-3)(4 + \sqrt{13 -x})\\
&= (-3-3)(4 + \sqrt{13- (-3)}) \\
&= (-6)(4 + 4) \\
&= -48.
\end{align*}
\end{example}



		
\begin{problem}(problem 7a)
  Compute the limit:
  \[
  \lim_{x \to -4} \frac{x^2 - 16}{3 - \sqrt{5 -x}}
  \]
  
    \begin{hint}
      When you plug in $x = -4$, you get $\frac00$
    \end{hint}
    \begin{hint}
      Multiply by the conjugate radical
    \end{hint}
    \begin{hint}
      $\sqrt a + \sqrt b$ and $\sqrt a - \sqrt b$ are conjugates
    \end{hint}
    \begin{hint}
      Use the difference of squares formula in the denominator: $(a-b)(a+b) = a^2 - b^2$
    \end{hint}
    \begin{hint}
      $\sqrt a \cdot \sqrt a = a$
    \end{hint}
		The value of the limit is
		 $\answer[given]{-48}$
		
\end{problem}



\begin{problem}(problem 7b)
  Compute the limit:
  \[
  \lim_{x \to 0} \frac{4- \sqrt{x + 16}}{x}.
  \]
  
    \begin{hint}
      When you plug in $x = 0$, you get $\frac00$
    \end{hint}
    \begin{hint}
      Multiply by the conjugate radical
    \end{hint}
    \begin{hint}
      $\sqrt a + \sqrt b$ and $\sqrt a - \sqrt b$ are conjugates
    \end{hint}
    \begin{hint}
      Use the difference of squares formula in the numerator: $(a-b)(a+b) = a^2 - b^2$
    \end{hint}
    \begin{hint}
      $\sqrt a \cdot \sqrt a = a$ 
    \end{hint}
		The value of the limit is
		 $\answer[given]{-\frac18}$
		
\end{problem}

\section{Complex fractions}


We now consider examples involving fractions within fractions, called \textbf{complex fractions}.




\begin{example}[example 8]
Compute the limit: $\displaystyle{\lim_{x \to 4} \frac{\frac{1}{x} - \frac{1}{4}}{x-4}}.$ \\
\\
Plugging in $x=4$ yields the familiar $\frac00$ indeterminate form. The algebra skills necessary to transform the 
function in the problem to one which allows plugging in $x=4$ involve subtraction and division of fractions. 
The rules are summarized as:
\[\frac{a}{b} - \frac{c}{d} = \frac{ad-bc}{bd} \quad \text{and} \quad \frac{\frac{a}{b}}{\frac{c}{d}} = \frac{ad}{bc}.\]
Applying these to our problem, we get

\begin{align*}
\lim_{x \to 4} \frac{\frac{1}{x} - \frac{1}{4}}{x-4} &= \lim_{x \to 4} \frac{\frac{4-x}{4x}}{x-4}\\
&=\lim_{x \to 4} \frac{4-x}{4x(x-4)}\\
&= \lim_{x \to 4} -\frac{1}{4x} \\
&= -\frac{1}{16}.
\end{align*}

It is important to note that $a-b$ and $b-a$ are opposites which cancel, leaving $-1$: 
\[ \frac{a-b}{b-a} = -1. \]
The ratio of opposites is $-1$.
\end{example}



\begin{problem}(problem 8)
  Compute the limit:
  \[
  \lim_{x \to 5} \frac{\frac{1}{x} - \frac{1}{5}}{x-5}.
  \]
  
    \begin{hint}
      When you plug in $x = 5$, you get $\frac00$
    \end{hint}
    \begin{hint}
      Subtract the fractions in the numerator
    \end{hint}
    \begin{hint}
      $\frac{a}{b} \pm \frac{c}{d} = \frac{ad \pm bc}{bd}$.
    \end{hint}
    \begin{hint}
      To divide, multiply by the reciprocal: $\frac{a}{b} \div \frac{c}{d} = \frac{a}{b} \cdot \frac{d}{c}$ 
    \end{hint}
    \begin{hint}
      Simplify the fraction by canceling 
    \end{hint}
		The value of the limit is
		 $\answer[given]{-\frac{1}{25}}$
		
\end{problem}




\begin{example}[example 9]
Compute the limit:
\[\lim_{x \to -2} \frac{\frac{2}{x-3} + \frac{x+4}{5}}{x^2 + 5x + 6}.\]
Carefully plugging in $x=-2$ gives $\frac00$, so we begin simplifying the complex fraction.
Since this function is bulky, let's do the addition of fractions from the numerator separately:
\begin{align*}
\frac{2}{x-3} + \frac{x+4}{5} &= \frac{10 + (x-3)(x+4)}{5(x-3)}\\
&= \frac{x^2 +x -2}{5(x-3)} \\
&= \frac{(x-1)(x+2)}{5(x-3)}.
\end{align*}
Now back to the original problem:
\begin{align*}
\lim_{x \to -2} \frac{\frac{2}{x-3} + \frac{x+4}{5}}{x^2 + 5x + 6} &= 
\lim_{x \to -2} \frac{\frac{(x-1)(x+2)}{5(x-3)}}{(x+2)(x+3)}\\
&= \lim_{x \to -2} \frac{(x-1)(x+2)}{5(x-3)(x+2)(x+3)} \\
&= \lim_{x \to -2} \frac{(x-1)}{5(x-3)(x+3)}\\
&= \frac{-3}{5(-5)(1)} = \frac{3}{25}.
\end{align*}
\end{example}


\begin{problem}(problem 9a)
  Compute the limit:
  \[
  \lim_{x \to -2} \frac{\frac{3}{x+1} + \frac{6}{x+4}}{x+2}.
  \]
  
    \begin{hint}
      When you plug in $x = -2$, you get $\frac00$
    \end{hint}
    \begin{hint}
      Add the fractions in the numerator
    \end{hint}
    \begin{hint}
      $\frac{a}{b} \pm \frac{c}{d} = \frac{ad \pm bc}{bd}$
    \end{hint}
    \begin{hint}
      To divide, multiply by the reciprocal: $\frac{a}{b} \div \frac{c}{d} = \frac{a}{b} \cdot \frac{d}{c}$
    \end{hint}
    \begin{hint}
      Cancel a common factor 
    \end{hint}
		The value of the limit is
		 $\answer[given]{-\frac{9}{2}}$
		
\end{problem}



\begin{problem}(problem 9b)
  Compute the limit:
  \[
    \lim_{x \to 3} \frac{\frac{2}{x-5} + \frac{x+2}{5}}{x^2 - 5x + 6}.
  \]
  
    \begin{hint}
      When you plug in $x = 3$, you get $\frac00$
    \end{hint}
    \begin{hint}
      Add the fractions in the numerator
    \end{hint}
    \begin{hint}
      $\frac{a}{b} \pm \frac{c}{d} = \frac{ad \pm bc}{bd}$
    \end{hint}
    \begin{hint}
      To divide, multiply by the reciprocal $\frac{a}{b} \div \frac{c}{d} = \frac{a}{b} \cdot \frac{d}{c}$ 
    \end{hint}
    \begin{hint}
      Cancel a common factor
    \end{hint}
		The value of the limit is
		 $\answer[given]{-\frac{3}{10}}$
		
\end{problem}




\section{Absolute values}


The definition of the absolute value is:

\[
\left | x \right | = 
\begin{cases}
\hfill x & \text{if $x\geq 0$}\\         -x & \text{if $x<0$}
\end{cases}
\] 

To calculate a limit involving an absolute value, we will need to remove the absolute value bars. 
To do this correctly, we can see from the definition that it is necessary to know whether the 
quantity in the absolute value bars is positive or negative.


\begin{example}[example 10]
Compute the limit: $\displaystyle{\lim_{x \to 3^{-}} \frac{|x-3|}{x-3}}$.\\
Plugging in $x = 3$ gives the indeterminate form $\frac00$. To resolve this limit, 
we will do a sign analysis on the quantity in the absolute value bars, $x-3$.

Since $x \to 3^-$, we have $x<3$ and hence $x-3 <0$. Since the quantity in the absolute value bars is negative, 
we can compute its absolute value as follows:

\[|x-3| = -(x-3).\]

Using this in the limit, we get

\begin{align*}
\lim_{x \to 3^-} \frac{|x-3|}{x-3} &= \lim_{x \to 3^-} \frac{-(x-3)}{x-3} \\
&= \lim_{x \to 3^-} (-1) \\
&= -1.
\end{align*}
\end{example}

\begin{problem}(problem 10a)
  Compute the limit:
  \[
  \lim_{x \to 4^-} \frac{4-x}{|x-4|}
  \]
  
    \begin{hint}
      Since $x \to 4^-$, we have $x<4$
    \end{hint}
    \begin{hint}
      Is $x-4$ positive or negative?
    \end{hint}
    \begin{hint}
      Remove the absolute value bars; include a negative sign if necessary
    \end{hint}
		\begin{hint}
      Simplify the fraction
    \end{hint}
		The value of the limit is
		 $\answer[given]{1}$
		
\end{problem}


\begin{problem}(problem 10b)
  Compute the limit:
  \[
  \lim_{x \to -3^-} \frac{|x+3|}{2x+6}
  \]
  
    \begin{hint}
      Since $x \to -3^-$, we have $x<-3$
    \end{hint}
    \begin{hint}
      Is $x+3$ positive or negative?
    \end{hint}
    \begin{hint}
      Remove the absolute value bars; include a negative sign if necessary
    \end{hint}
		\begin{hint}
      Simplify the fraction
    \end{hint}
		The value of the limit is
		 $\answer[given]{-1/2}$
		
\end{problem}


\begin{center}
\begin{foldable}
\unfoldable{Here are some detailed, lecture style videos on finding limits analytically:}
\youtube{JroM3rmBH50}
\youtube{zN8wuKZatxk}
\end{foldable}
\end{center}


\end{document}





















In the next few examples, we will investigate infinite limits of rational functions. 
These occur at points where the denominator of the rational function is approaching zero, 
but the numerator is not approaching zero.

\begin{example} %example 7
Compute the limit: 

\[
\lim_{x \to 0^+} \frac{1}{x}.
\]

Plugging in $x=\infty$ yields the {\bf undefined} expression $\frac{1}{0}$. 
As the denominator gets smaller and smaller, it will divide into the numerator more and more times, 
causing the fraction to ``blow up". In terms of the limit, it is reasonable to expect an 
answer of either $\infty$ or $-\infty$.  A sign analysis will determine which one.
Since the values of $x$ are positive as $x \to 0^+$, the values of $f(x) = \frac{1}{x}$ are also positive.
Furthermore, as the values of $x$ decrease towards zero, the values of the reciprocal, $\frac{1}{x}$ increase without bound.

Hence, we can conclude that 

\[\lim_{x \to 0^+} \frac{1}{x} =  \infty. \]

The result geometric significance of this result is that the line $x=0$ (the $y$-axis) 
is a vertical asymptote for the graph of the function $f(x) = \frac{1}{x}.$ See the graph below.


\begin{tikzpicture}
\begin{axis}
\addplot[domain=.01:4, 
    samples=100, color=red]{1/x};
\end{axis}
\end{tikzpicture}
\end{example}


\begin{question} %problem #7
  
	Compute the limit:
  \[
  \lim_{x \to 0^-} \frac{3}{x}.
  \]
  
    \begin{hint}
      Since $x \to 0^-$, we have $x<0$
    \end{hint}
    \begin{hint}
      Is $\frac{3}{x}$ positive or negative?
    \end{hint}
    \begin{hint}
      $\frac{a}{0}$ gives $\pm \infty$
    \end{hint}
		The value of the limit  is
		(type infinity for $\infty$ or -infinity for $-\infty$)
		 $\answer[given]{-infinity}$
		
\end{question}


\begin{example} %example 8
Determine the limit: 
\[
\lim_{x \to 3^-} \frac{2}{x-3}.
\] 


First we observe that plugging in the value $x=3$ gives $\frac{2}{0}$ which is undefined and 
the fraction is ``blowing up" in the limit.  A sign analysis will tell us if the limit is $\pm\infty$.
Since $x \to 3^-$, we have $x<3$ and hence $x-3 <0$. 
We see that the numerator is $2$ and the denominator is approaching $0$ through negative values, and
the values of $f(x) = \frac{2}{x-3}$ are negative.  We conclude that

\[\lim_{x \to 3^-} \frac{2}{x-3} =  -\infty. \]

This geometric significance of the result is that the line $x=3$ is a vertical asymptote for the 
graph of the function $f(x) = \frac{2}{x-3}$, as shown below.

\begin{center}
\begin{tikzpicture}
\begin{axis}[axis lines = center, xlabel = $x$,
    ylabel = {$f(x)$}]
\addplot[domain=1:2.9, 
    samples=100, color=blue]{2/(x-3)};

\end{axis}
\end{tikzpicture}
\end{center}
\end{example}

\begin{question} %problem #8
  
	Detrermine the limit:
  \[
  \lim_{x \to 2^+} \frac{4}{x-2}.
  \]
  
    \begin{hint}
      Since $x \to 2^+$, we have $x>2$
    \end{hint}
    \begin{hint}
      Is $\frac{4}{x-2}$ positive or negative?
    \end{hint}
    \begin{hint}
      $\frac{a}{0}$ gives $\pm \infty$
    \end{hint}
		The value of the limit is
		(type infinity for $\infty$ or -infinity for $-\infty$)
		 $\answer[given]{infinity}$
		
\end{question}

\begin{example} %example 9
Analyze the limit:
\[
\lim_{x\to 2} 
\frac{x + 1}{x-2}
\]

Plugging $x = 2$ into the rational function
\[f(x) = \frac{x + 1}{x-2}\]
gives the undefined expression $\frac{3}{0}$. From this information, we can conclude that the graph 
of the function has a vertical asymptote at $x = 2$. This means that the one-sided limits as $x$ approaches 2
will give either $\infty$ or $-\infty$, i.e., 
\[\lim_{x \to 2^-} \frac{x+1}{x-2}= \infty \ \text{or} \ -\infty.\]
and
\[\lim_{x \to 2^+} \frac{x+1}{x-2}= \infty \ \text{or} \ -\infty.\]
To determine which, we will do a sign analysis as follows. Consider the left hand limit first:
\[\lim_{x \to 2^-} \frac{x+1}{x-2}.\]
The numerator is approaching 3, which is positive. The denominator is approaching zero which is neither 
positive nor negative, but since $x \to 2^-$, we know that $x<2$ and therefore $x-2 <0$.  
Hence the denominator is negative as $x$ approaches 2 from the left. Since a positive divided by a 
negative is negative, we get:
\[\lim_{x \to 2^-} \frac{x+1}{x-2} =\frac{\text{pos}}{\text{neg}} = -\infty,\]
since the choices were only $\infty$ and $-\infty$.
Now, we will do a sign analysis on the right hand limit:
\[\lim_{x \to 2^-} \frac{x+1}{x-2}.\]
The numerator is approaching 3, which is positive. In the denominator, since $x \to 2^+$, 
we know that $x>2$ and therefore $x-2 >0$.  
Hence the denominator is positive as $x$ approaches 2 from the right. Since a positive divided by a 
positive is positive, we get:
\[\lim_{x \to 2^-} \frac{x+1}{x-2} =\frac{\text{pos}}{\text{pos}} = \infty,\]
since the choices were only $\infty$ and $-\infty$.
The graph of $f(x) = \frac{x+1}{x-2}$ near $x = 2$ looks like this:

\begin{center}
\begin{tikzpicture}
\begin{axis}[axis lines = center, title={The graph of $y=\dfrac{x+1}{x-2}$}]
\addplot[domain=0:1.9, 
    samples=100, color=blue]{(x+1)/(x-2)};
\addplot[domain=2.1:4, 
    samples=100, color=blue]{(x+1)/(x-2)};
\vasymptote {2}
\end{axis}
\end{tikzpicture}
\end{center}
\end{example}



\begin{question} %problem #9
  
	Analyze the limit:
  \[
  \lim_{x \to 3} \frac{x-1}{x-3}.
  \]
  
    \begin{hint}
      This is a two-sided limit. Check the one-sided limits separately
    \end{hint}
    \begin{hint}
      Both one-sided limits involve division by zero
    \end{hint}
    \begin{hint}
      The one-sided limits are either $\pm \infty$
    \end{hint}
		\begin{hint}
		  If the one-sided limits are different, then the two-sided limit DNE
		\end{hint}	
		The value of the limit is
		(type infinity for $\infty$, -infinity for $-\infty$ or DNE)
		 $\answer[given]{DNE}$
		
\end{question}





\begin{example} %example 10
Analyze the limit:
\[
\lim_{x\to -1} 
\frac{2x}{(x+1)^3}
\]

Plugging $x = -1$ into the rational function
\[f(x) = \frac{2x}{(x+1)^3}\]
gives the undefined expression $\frac{-2}{0}$. From this information, we can conclude that the graph 
of the function has a vertical asymptote at $x = -1$. This means that the one-sided limits as $x$ approaches -1
will give either $\infty$ or $-\infty$, i.e., 
\[\lim_{x \to -1^-} \frac{2x}{(x+1)^3}= \infty \ \text{or} \ -\infty.\]
and
\[\lim_{x \to -1^+} \frac{2x}{(x+1)^3}= \infty \ \text{or} \ -\infty.\]
To determine which, we will do a sign analysis as follows. Consider the left hand limit first:
\[\lim_{x \to -1^-} \frac{2x}{(x+1)^3}.\]
The numerator is approaching -2, which is negative. The denominator is approaching zero which is neither 
positive nor negative, but since $x \to =1^-$, we know that $x<-1$ and therefore $x+1 <0$ and $(x+1)^3 < 0$.  
Hence the denominator is negative as $x$ approaches -1 from the left. Since a negative divided by a 
negative is positive, we get:
\[\lim_{x \to -1^-} \frac{2x}{(x+1)^3} =\frac{\text{neg}}{\text{neg}} = \infty,\]
since the choices were only $\infty$ and $-\infty$.
Now, we will do a sign analysis on the right hand limit:
\[\lim_{x \to -1^+} \frac{2x}{(x+1)^3}.\]
The numerator is approaching -2, which is negative. In the denominator, since $x \to -1^+$, 
we know that $x>-1$ and therefore $x+1 >0$ and $(x+1)^3 > 0$.  
Hence the denominator is positive as $x$ approaches -1 from the right. Since a negative divided by a 
positive is negative, we get:
\[\lim_{x \to -1^+} \frac{2x}{(x+1)^3} =\frac{\text{neg}}{\text{pos}} = -\infty,\]
since the choices were only $\infty$ and $-\infty$.
Near $x = 2$, the graph looks like this:

\begin{center}
\begin{tikzpicture}
\begin{axis}[axis lines = center,  title={The graph of $y=\dfrac{2x}{(x+1)^3}$}]
\addplot[domain=-3:-1.1, 
    samples=100, color=blue]{(2*x)/(x+1)^3};
\addplot[domain=-0.9:1, 
    samples=100, color=blue]{(2*x)/(x+1)^3};
\vasymptote {-1}
\end{axis}
\end{tikzpicture}
\end{center}
\end{example}


\begin{question} %problem #10
  
	Analyze the limit:
  \[
  \lim_{x \to -3} \frac{x+1}{(x+3)^2}.
  \]
  
    \begin{hint}
      This is a two-sided limit. Check the one-sided limits separately
    \end{hint}
    \begin{hint}
      Both one-sided limits involve division by zero
    \end{hint}
    \begin{hint}
      The one-sided limits are either $\pm \infty$
    \end{hint}
		\begin{hint}
		  If the one-sided limits are different, then the two-sided limit DNE
		\end{hint}	
		The value of the limit is
		(type infinity for $\infty$, -infinity for $-\infty$ or DNE)
		 $\answer[given]{infinity}$
		
\end{question}



\begin{center}
{\bf End Behavior}
\end{center}




\begin{example} %example 12
Find $\lim_{x \to \infty} \dfrac{1}{x}.$
  We argue as follows. If $x$ is a very large number, 
	then $1/x$ will be a very small number, near zero.  Furthermore, as $x$ increases, $1/x$ will decrease.
	
	Hence
	\[\lim_{x \to \infty} \frac{1}{x}= 0.\]
	This is evident from the graph of $y=1/x$ shown below.
	
	%\[\graph[panel, xAxisLabel=``x'', yAxisLabel=``y'', xmin=-3, xmax=75]{y = 1/x}\]
	
	The graph of the function $f(x) = \frac{1}{x}$ also provides evidence for this conclusion.  
	
	\[\graph{y=1/x}\]
	
	Lastly, the value of the limit corresponds to the horizontal asymptote of the graph, namely
	$y = 0$ in this example.
\end{example}




\begin{question} %problem #12
  Compute the limit:
  \[
  \lim_{x \to \infty} \frac{7}{x^2}
  \]
  
    \begin{hint}
      As $x$ goes to infinity, so does $x^2$
    \end{hint}
    \begin{hint}
      As the denominator grows, the fraction shrinks
    \end{hint}
    \begin{hint}
      The fraction never shrinks below 0
    \end{hint}
		The value of the limit is
		 $\answer[given]{0}$
		
\end{question}

An important generalization of the last example and problem, which follows from a similar analysis is 
\[\lim_{x \to \pm \infty} \frac{a}{x^n} = 0 \]
for any constant, $a$, and any positive exponent, $n$.

We will exploit this important fact in the next two examples.


\begin{example} %example 13
Compute $\displaystyle{\lim_{x \to \infty} \frac{3x^2 + 5x + 2}{2x^2 -x- 4}}.$

First note that as $x\to \pm \infty$, a polynomial, $p(x) \to \pm \infty$ according to it leading term.
In this example, since the lead terms $3x^2$ and $2x^2$ both go to $\infty$ as $x \to \infty$, 
our limit has the indeterminate form
$\frac{\infty}{\infty}$.
To resolve this issue, we will factor out of both the numerator and the denominator 
the highest power of $x$ seen in the denominator.  So, in this example, we will factor $x^2$ from both.
In the numerator, 
\[3x^2 + 5x + 2 = x^2(3 + \frac{5}{x} + \frac{2}{x^2})\]
and in the denominator,
\[2x^2 -x -4 = x^2(2- \frac{1}{x} - \frac{4}{x^2}).\]

With these factorizations, our limit becomes

\begin{align*}
\lim_{x \to \infty} \frac{3x^2 + 5x + 2}{2x^2 -x- 4} &= 
\lim_{x \to \infty} \frac{x^2(3 + \frac{5}{x} + \frac{2}{x^2})}{x^2(2- \frac{1}{x} - \frac{4}{x^2})} \\ \\
&=\lim_{x \to \infty} \frac{3 + \frac{5}{x} + \frac{2}{x^2}}{2- \frac{1}{x} - \frac{4}{x^2}} \\ \\
&=\frac{3 + 0 + 0}{2- 0 - 0} \\ \\
&= \frac32.
\end{align*}

The result of this limit means that the line $y = 3/2$ is a horizontal asymptote
for the graph of $y = \dfrac{3x^2 + 5x + 2}{2x^2 -x- 4}$ on the right end.
\end{example}

\begin{question} %problem #13
  Compute the limit:
  \[
  \lim_{x \to \infty} \frac{3 + 2x^3 - x^3}{5x^3 - 2x -4}
  \]
  
    \begin{hint}
      When you `plug in' $x = \infty$, you get $\frac{-\infty}{\infty}$
    \end{hint}
    \begin{hint}
      Determine the highest power of $x$ in the denominator
    \end{hint}
    \begin{hint}
      Factor this power from both the numerator and denominator and cancel it
    \end{hint}
    \begin{hint}
      $\frac{a}{x^n} \to 0$ as $x \to \infty$.
    \end{hint}
		The value of the limit is
		 $\answer[given]{-\frac15}$
		
\end{question}




\begin{example} %example 14
Compute $\displaystyle{\lim_{x \to -\infty} \frac{4x^2 - 3x - 6}{x^3 +8x^2 -3x + 7}}.$

First note that as $x\to \pm \infty$, a polynomial, $p(x) \to \pm \infty$ according to it leading term.
In this example, the lead terms $4x^2$ and $x^3$  go to $\infty$ and $-\infty$ respectively, as $x \to \infty$. 
Hence, our limit has the indeterminate form
$\frac{\infty}{-\infty}$.
To resolve this issue, we will factor out 
the highest power of $x$ in the denominator, namely $x^3$, from both the numerator and the denominator.  
In the numerator we get, 
\[4x^2 - 3x - 6 = x^3(\frac{4}{x} - \frac{3}{x^2} - \frac{6}{x^3})\]
and in the denominator we get,
\[x^3 +8x^2 -3x + 7 = x^3(1 +  \frac{8}{x} - \frac{3}{x^2} + \frac{7}{x^3}).\]

Factoring the $x^3$ and canceling, the limit can be resolved as follows:

\begin{align*}
\lim_{x \to -\infty}\frac{4x^2 - 3x - 6}{x^3 +8x^2 -3x + 7} &= 
\lim_{x \to -\infty} \frac{x^3(\frac{4}{x}-\frac{3}{x^2} -\frac{6}{x^3})}
{x^3(1 + \frac{8}{x}-\frac{3}{x^2}+\frac{7}{x^3})} \\ \\
&=\lim_{x \to -\infty} \frac{\frac{4}{x}-\frac{3}{x^2} -\frac{6}{x^3}}{1 + \frac{8}{x}-\frac{3}{x^2}+\frac{7}{x^3}} \\ \\
&=\frac{0 - 0 - 0}{1 +0- 0 + 0} \\ \\
&= \frac01 \\
&= 0.
\end{align*}

The result of this limit means that the line $y = 0$ (the $x$-axis) is a horizontal asymptote
for the graph of $y = \dfrac{4x^2 - 3x - 6}{x^3 +8x^2 -3x + 7}$ on the right end.
\end{example}

\begin{question} %problem #14
  Compute the limit:
  \[
  \lim_{x \to \infty} \frac{3x^3 - 5x + 6}{3-x+ 2x^3- x^4}
  \]
  
    \begin{hint}
      When you `plug in' $x = \infty$, you get $\frac{\infty}{-\infty}$
    \end{hint}
    \begin{hint}
      Determine the highest power of $x$ in the denominator
    \end{hint}
    \begin{hint}
      Factor this power from both the numerator and denominator
    \end{hint}
    \begin{hint}
      Cancel the power of $x$ that was factored out
    \end{hint}
    \begin{hint}
      $\frac{a}{x^n} \to 0$ as $x \to \infty$.
    \end{hint}
		The value of the limit is
		 $\answer[given]{0}$
		
\end{question}




\end{document}


\begin{example} %example 9
Analyze the limit:
\[
\lim_{x\to -1} 
\frac{1}{x^2}
\]
Plugging $x = 0$ into the rational function
\[f(x) = \frac{1}{x^2}\]
gives the undefined expression $\frac{1}{0}$. From this information, we can conclude that the graph 
of the function has a vertical asymptote at $x = 0$. This means that the one-sided limits as $x$ approaches 0
will give either $\infty$ or $-\infty$, i.e., 
\[\lim_{x \to -1^-} \frac{1}{x^2}= \infty \ \text{or} \ -\infty.\]
and
\[\lim_{x \to -1^+} \frac{1}{x^2}= \infty \ \text{or} \ -\infty.\]
To determine which, we will do a sign analysis as follows. Because of the perfect square, 
we can consider both case simultaneously.
The numerator is 1, which is positive and the denominator is $x^2$ which is positive whether $x \to 0^-$
or $x \to 0^+$.  Hence 
\[\lim_{x \to 0^-} \frac{1}{x^2} =\frac{\text{pos}}{\text{pos}} = \infty,\]
and
\[\lim_{x \to 0^+} \frac{1}{x^2} =\frac{\text{pos}}{\text{pos}} = \infty,\]
since the choices were only $\infty$ and $-\infty$.
The graph of $f(x) = \frac{1}{x^2}$ near $x = 0$ looks like this:

\begin{center}
\begin{tikzpicture}
\begin{axis}[axis lines = center, xlabel = $x$,
    ylabel = {$f(x)$}]
\addplot[domain=-2:-0.1, 
    samples=100, color=blue]{1/x^2};
\addplot[domain=0.1:2, 
    samples=100, color=blue]{1/x^2};
\end{axis}
\end{tikzpicture}
\end{center}
\end{example}


Here is a cool truth table

\begin{problem}
Fill in the truth table below using your amazing logic skillz!

\begin{prompt}
\begin{center}
\[
\begin{array}{c|c|c|c}
		p & q & p \implies q & p \vee (p \implies q) \\
		\hline
		T & T & T & \answer{T} \\
		T & F & F & \answer{T} \\
		F & T & T & \answer{T} \\
		F & F & T & \answer{T}
	\end{array}
    \]
\end{center}
\end{prompt}
\end{problem}

\begin{center}
\begin{tikzpicture}
\begin{axis}[axis lines = center, xlabel = $x$,
    ylabel = {$f(x)$}]
\addplot[domain= 0.5:100, 
    samples=1000, color=blue]{1/x};

\end{axis}
\end{tikzpicture}
\end{center}

\youtube{AkYCqV75EOw}
\youtube{FyhXNtLGr2o}

The idea used in this problem is to multiply by the \bf conjugate radical}. 
Expressions of the form $a+b$ and $a-b$ are called conjugate radical expressions if 
either of the terms $a$ or $b$ (or both) contains a square root. In our problem, the expression 
$x- \sqrt x$ in the numerator contains a square root symbol and hence it has a conjugate radical expression, namely $x + \sqrt x$.
The significance of conjugate radicals is revealed when they are multiplied together. 
Applying the difference of two squares formula 
(from right to left) we see that $(x- \sqrt x)(x + \sqrt x) = x^2 -x$. 
The resulting expression no longer contains a square root symbol. 

Let us perform the following operation on the function in our problem:
\[\lim_{x \to 1} \frac{x- \sqrt{x}}{x-1}= 
\lim_{x \to 1} \left(\frac{x- \sqrt{x}}{x-1}\right)\cdot \left(\frac{x+ \sqrt{x}}{x+ \sqrt{x}}\right) .\]
Notice that we have multiplied our original fraction by one in the form of the conjugate radical over itself. The point of
undertaking this seemingly complexifying step can be seen in what follows.

In conclusion, by introducing a the conjugate radical to the problem, 
then using the difference of two squares formula and finally factoring and canceling, we have 
\[\lim_{x \to 1} \frac{x- \sqrt{x}}{x-1} = \frac{1}{2}.\]
