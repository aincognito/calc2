
\documentclass{ximera}
%% You can put user macros here
%% However, you cannot make new environments



\newcommand{\ffrac}[2]{\frac{\text{\footnotesize $#1$}}{\text{\footnotesize $#2$}}}
\newcommand{\vasymptote}[2][]{
    \draw [densely dashed,#1] ({rel axis cs:0,0} -| {axis cs:#2,0}) -- ({rel axis cs:0,1} -| {axis cs:#2,0});
}


\graphicspath{{./}{firstExample/}}

\usepackage{amsmath}
\usepackage{amssymb}
\usepackage{array}
\usepackage[makeroom]{cancel} %% for strike outs
\usepackage{pgffor} %% required for integral for loops
\usepackage{tikz}
\usepackage{tikz-cd}
\usepackage{tkz-euclide}
\usetikzlibrary{shapes.multipart}


\usetkzobj{all}
\tikzstyle geometryDiagrams=[ultra thick,color=blue!50!black]


\usetikzlibrary{arrows}
\tikzset{>=stealth,commutative diagrams/.cd,
  arrow style=tikz,diagrams={>=stealth}} %% cool arrow head
\tikzset{shorten <>/.style={ shorten >=#1, shorten <=#1 } } %% allows shorter vectors

\usetikzlibrary{backgrounds} %% for boxes around graphs
\usetikzlibrary{shapes,positioning}  %% Clouds and stars
\usetikzlibrary{matrix} %% for matrix
\usepgfplotslibrary{polar} %% for polar plots
\usepgfplotslibrary{fillbetween} %% to shade area between curves in TikZ



%\usepackage[width=4.375in, height=7.0in, top=1.0in, papersize={5.5in,8.5in}]{geometry}
%\usepackage[pdftex]{graphicx}
%\usepackage{tipa}
%\usepackage{txfonts}
%\usepackage{textcomp}
%\usepackage{amsthm}
%\usepackage{xy}
%\usepackage{fancyhdr}
%\usepackage{xcolor}
%\usepackage{mathtools} %% for pretty underbrace % Breaks Ximera
%\usepackage{multicol}



\newcommand{\RR}{\mathbb R}
\newcommand{\R}{\mathbb R}
\newcommand{\C}{\mathbb C}
\newcommand{\N}{\mathbb N}
\newcommand{\Z}{\mathbb Z}
\newcommand{\dis}{\displaystyle}
%\renewcommand{\d}{\,d\!}
\renewcommand{\d}{\mathop{}\!d}
\newcommand{\dd}[2][]{\frac{\d #1}{\d #2}}
\newcommand{\pp}[2][]{\frac{\partial #1}{\partial #2}}
\renewcommand{\l}{\ell}
\newcommand{\ddx}{\frac{d}{\d x}}

\newcommand{\zeroOverZero}{\ensuremath{\boldsymbol{\tfrac{0}{0}}}}
\newcommand{\inftyOverInfty}{\ensuremath{\boldsymbol{\tfrac{\infty}{\infty}}}}
\newcommand{\zeroOverInfty}{\ensuremath{\boldsymbol{\tfrac{0}{\infty}}}}
\newcommand{\zeroTimesInfty}{\ensuremath{\small\boldsymbol{0\cdot \infty}}}
\newcommand{\inftyMinusInfty}{\ensuremath{\small\boldsymbol{\infty - \infty}}}
\newcommand{\oneToInfty}{\ensuremath{\boldsymbol{1^\infty}}}
\newcommand{\zeroToZero}{\ensuremath{\boldsymbol{0^0}}}
\newcommand{\inftyToZero}{\ensuremath{\boldsymbol{\infty^0}}}


\newcommand{\numOverZero}{\ensuremath{\boldsymbol{\tfrac{\#}{0}}}}
\newcommand{\dfn}{\textbf}
%\newcommand{\unit}{\,\mathrm}
\newcommand{\unit}{\mathop{}\!\mathrm}
%\newcommand{\eval}[1]{\bigg[ #1 \bigg]}
\newcommand{\eval}[1]{ #1 \bigg|}
\newcommand{\seq}[1]{\left( #1 \right)}
\renewcommand{\epsilon}{\varepsilon}
\renewcommand{\iff}{\Leftrightarrow}

\DeclareMathOperator{\arccot}{arccot}
\DeclareMathOperator{\arcsec}{arcsec}
\DeclareMathOperator{\arccsc}{arccsc}
\DeclareMathOperator{\si}{Si}
\DeclareMathOperator{\proj}{proj}
\DeclareMathOperator{\scal}{scal}
\DeclareMathOperator{\cis}{cis}
\DeclareMathOperator{\Arg}{Arg}
%\DeclareMathOperator{\arg}{arg}
\DeclareMathOperator{\Rep}{Re}
\DeclareMathOperator{\Imp}{Im}
\DeclareMathOperator{\sech}{sech}
\DeclareMathOperator{\csch}{csch}
\DeclareMathOperator{\Log}{Log}

\newcommand{\tightoverset}[2]{% for arrow vec
  \mathop{#2}\limits^{\vbox to -.5ex{\kern-0.75ex\hbox{$#1$}\vss}}}
\newcommand{\arrowvec}{\overrightarrow}
\renewcommand{\vec}{\mathbf}
\newcommand{\veci}{{\boldsymbol{\hat{\imath}}}}
\newcommand{\vecj}{{\boldsymbol{\hat{\jmath}}}}
\newcommand{\veck}{{\boldsymbol{\hat{k}}}}
\newcommand{\vecl}{\boldsymbol{\l}}
\newcommand{\utan}{\vec{\hat{t}}}
\newcommand{\unormal}{\vec{\hat{n}}}
\newcommand{\ubinormal}{\vec{\hat{b}}}

\newcommand{\dotp}{\bullet}
\newcommand{\cross}{\boldsymbol\times}
\newcommand{\grad}{\boldsymbol\nabla}
\newcommand{\divergence}{\grad\dotp}
\newcommand{\curl}{\grad\cross}
%% Simple horiz vectors
\renewcommand{\vector}[1]{\left\langle #1\right\rangle}


\outcome{Find limits using graphs}

\title{1.3 Graphical Limits}


\begin{document}

\begin{abstract}
In this section we use the graph of a function to find limits.
\end{abstract}

\maketitle








\begin{center}
\textbf{Finding Limits Graphically}
\end{center}


\begin{center}
\begin{foldable}
\unfoldable{Here are some detailed, lecture style videos on graphical limits:}
\youtube{PZCrPXqjUBI}
\youtube{qHQgMRXI4-k}
\end{foldable}
\end{center}


%\includeinteractive{desmos.js}



%\[
%\interactive{
%var calc = Desmos.Calculator( holderElementFromSomewhereMagical );
%calc.setExpression({latex:"(x^2-1)/(x-1)",color:"blue"});
%calc.setExpression({latex:"(1,2)",color:"blue",style:"open"});
%}
%\]


In this section, functions will be presented graphically. Recall that the graph of a 
function must pass the \textbf{vertical line test} which states that a vertical line 
can intersect the graph of a function in at most one point.
To understand graphical representations of functions, consider the following graph of a function, $y = f(x)$.



\includeinteractive{gl00b.js}

%\[\graph[xmin=-9, xmax=4, ymin=-5, ymax=11]{y=(x^3 +7x^2)/6}\]



%\begin{center}
%\includegraphics[width = 4 in, height = 3 in, angle= 1]{parabola.jpeg}
%\end{center}

The graph of a function is created by letting the $x$-coordinate represent 
the input of the function and the $y$-coordinate represent the corresponding output, i.e., $y = f(x)$.  The general form of a point on the graph of $y = f(x)$ is 
$(x_0, f(x_0))$ for any input value, $x_0$, in the \textbf{domain} of $f$.\\

Notice that the point $(-1,2)$ is on the graph of $f$, shown above.  This means that $f(-1) = 2$.  
Similarly, since the points $(3,-2)$ and $(5,1)$ are also on the graph, we have $f(3) = -2$ and $f(5) = 1$.  


We now consider several examples of limits of functions presented graphically.



\begin{example}[example 1]

For the function, $f$, whose graph is shown below, determine the value of the \textbf{one-sided limit}, 
\[
\lim_{x\to 3^-} f(x).
\]
Note that this is called a one-sided limit because $x$ is approaching $3$ from the left hand side.
%\includeinteractive{gl02.js}
\includeinteractive{gl01b.js}
%\includeinteractive{gl01.js}

%\[\graph[xmin=-5, xmax=8, ymin=-1.5, ymax=5.5]{2^x \left\{ -1 \leq x \leq 1.97 \right\}, (x-2)^2 + (y-4)^2 = 0.01}\]


%\begin{center}
%\includegraphics[width = 4 in, height = 2.5 in]{graphical-limit01}
%\end{center}


\vspace{.25in}
Solution: Since $x \to 3^-$ we know that the $x$-value is approaching $3$ and we also know that $x<3$. 
Looking at the graph, we can see that for $x$-values slightly less than $3$ (to the left of 3), 
the $y$-values on the graph are very close to $2$. And as the $x$-value moves toward 3, the corresponding $y$-value on the graph moves toward 2. We can express this using limits as 

\[
\lim_{x\to 3^-} f(x) = 2.
\]

In this example, it is important to note that the open circle at the point $(3,2)$ indicates that that point is not on the graph.  Furthermore, there are no points on the graph with $x$-coordinate equal to 3, which means that $f(3)$ is undefined.
Thus, we see in this example that a function can have a limit as $x$ approaches a value that is not in the domain of the function. 
Interactivity note: if you click on the graph,
a dot will appear on the graph.  
%on the graph will pop-up at the place where you clicked.  
You can then drag that dot along the curve with your mouse.

\end{example}


\begin{problem}(problem 1)




Use the graph of $y = f(x)$ given below to determine
  \[
  \lim_{x\to -4^+} f(x).
  \]
  
    \begin{hint}
      $x$ is to the right of -4
    \end{hint}
    \begin{hint}
      The y-coordinates determine the limit
    \end{hint}
		\begin{hint}
		  Click on the graph and move the dot
		\end{hint}
		The value of the limit is
		 $\answer[given]{3}$

\includeinteractive{glp01.js}
\end{problem}


\begin{example}[example 2]
For the function, $f$, whose graph is shown below, determine the value of the \textbf{two-sided limit}, 
\[
\lim_{x\to 3} f(x).
\]
Note that this is called a two-sided limit because $x$ can 
approach $3$ from either the left hand side or the right hand side.

%\begin{center}
%\includegraphics[width = 4 in, height = 3 in]{graphical-limit02}
%\end{center}

\includeinteractive{gl02.js}
%\[
%\graph[xmin=-5, xmax=12, ymin=-1, ymax=6]{
%x-1 \left\{-10 \leq x \leq 2.94 \right\}, 
%x-1 \left\{3.06 \leq x \leq 10 \right\}, (x-3)^2 + (y-2)^2 = 0.01,
%(x-3)^2 + (y-4)^2 = 0.01, (x-3)^2 + (y-4)^2 = 0.001, (x-3)^2 + (y-4)^2 = 0.0001,
%(x-3)^2 + (y-4)^2 = 0.009, (x-3)^2 + (y-4)^2 = 0.007, (x-3)^2 + (y-4)^2 = 0.005,
%}
%\]

\vspace{.25in}
Solution: Since $x \to 3$ we know that the $x$-value is approaching $3$ and we also know 
that $x$ can be on either side of $3$ (but not equal to 3). 
Looking at the graph, we can see that for $x$-values either slightly less than $3$ or slightly greater than $3$, 
the $y$-values on the graph are very close to $2$. Thus, 

\[
\lim_{x\to 3} f(x) = 2.
\]

In this example it is important to observe that, even though the function value at $x= 3$ is $4$, 
as indicated by the dot at the point $(3,4)$, the limit of the function as $x$ approaches $3$ is $2$ and not $4$.  
The limit of a function is determined by the behavior of the function \textit{near} the 
indicated $x$-value and not \textit{at} that $x$-value.\\

\end{example}

\begin{problem}(problem 2)




Use the graph of $y = f(x)$ given below to determine
  \[
  \lim_{x\to 3} f(x).
  \]
  
    \begin{hint}
      This is a two-sided limit
    \end{hint}
    \begin{hint}
      The y-coordinates determine the limit
    \end{hint}
		\begin{hint}
		  Click on the graph and move the dot
		\end{hint}
		The value of the limit is
		 $\answer[given]{1}$

\includeinteractive{glp02.js}
	
\end{problem}

\begin{example}[example 3]
For the function, $f$, whose graph is shown below, find the one-sided limits,
\[\lim_{x \to 1^-}f(x), \lim_{x \to 1^+}f(x),
\] 
and discuss the two-sided limit, 
\[\lim_{x \to 1} f(x).
\]

\includeinteractive{gl03.js}

%\[
%\graph{
%2^x -3 \left\{-10 \leq x \leq 0.93 \right\}, 
%-1/x +3 \left\{1 < x \leq 10 \right\}, (x-1)^2 + (y+1)^2 = 0.015, (x-1)^2 + (y-2)^2 = 0.005,
%(x-1)^2 + (y-2)^2 = 0.015, (x-1)^2 + (y-2)^2 = 0.01, (x-1)^2 + (y-2)^2 = 0.001, (x-1)^2 + (y-2)^2 = 0.0001,
%}
%\]


%\begin{center}
%\includegraphics[width = 4 in, height = 3 in]{graphical-limit03}
%\end{center}

\vspace{.25in}
Solution: As in example 1, we can determine the value of the left hand limit, $\lim_{x \to 1^-}f(x)$, 
by observing that if $x$ is on the left side of 1 and very close to 1 then the $y$-values on the 
graph are very close to -1 and hence, 

\[
\lim_{x \to 1^-}f(x) = -1.
\]

For the right hand limit $\lim_{x \to 1^+}f(x)$, we inspect the $y$-values on the 
graph that correspond to $x$-values on the right side of 1 and very close to $1$.  
We see that these $y$-values are very close to 2, hence

\[
\lim_{x \to 1^+}f(x) = 2.
\]

We see that the one-sided limits as $x$ approaches 1 have different values, and we therefore 
conclude that the two-sided limit, $\lim_{x \to 1}f(x)$, \textbf{does not exist}.  We write

\[
\lim_{x \to 1}f(x) \;\;\text{DNE}.
\]

\end{example}

In general, when the one-sided limits are different then the 
corresponding two-sided limit does not exist.

\begin{problem}(problem 3)


  Use the graph of $y = f(x)$ given below to determine
  \[
  \lim_{x\to -1^-} f(x), \lim_{x\to -1^+} f(x) \ \text{and}
	\]
	\[
	\lim_{x\to -1} f(x).
	\]
  
  
    \begin{hint}
      The y-coordinates determine the limit
    \end{hint}
		\begin{hint}
		  Click on the graph and move the dot
		\end{hint}
		\begin{hint}
		  It is possible that a limit DNE
		\end{hint}
		The value are $\lim_{x\to -1^-} f(x)=\answer[given]{-2}$, 
		$\lim_{x\to -1^+} f(x)=\answer[given]{3}$ and 
		$\lim_{x\to -1} f(x)=\answer[given]{DNE}$


\includeinteractive{glp03.js}

	
\end{problem}


\begin{example}[example 4]
For the function, $f$, whose graph is shown below, find the one-sided limit,
\[
\lim_{x \to 3^+}f(x).
\]

\includeinteractive{gl04.js}

%\[\graph[xmin=2, xmax=6, ymin=-10, ymax=3]{2+log(x-3.01)}\]

%\begin{center}
%\includegraphics[width = 4 in, height = 2.5 in]{graphical-limit04}
%\end{center}

\vspace{.25in}
Solution: From the graph, we can see that as the $x$-values approach 3 from the right hand side, the $y$-values decrease without bound.  In this case we write

\[
\lim_{x \to 3^+}f(x) = -\infty.
\]

To describe the phenomenon of an \textbf{infinite limit} as $x$ approaches a finite value (in this case, 3), 
we say that the line $x = 3$ is a \textbf{vertical asymptote} for the graph of $f$.

\end{example}


\begin{problem}(problem 4)

  Use the graph of $y = f(x)$ given below to determine
  \[
  \lim_{x\to 2^-} f(x).
  \]
  
    \begin{hint}
      $x$ is to the left of 2
    \end{hint}
    \begin{hint}
      The y-coordinates determine the limit
    \end{hint}
		\begin{hint}
		  Click on the graph and move the dot
		\end{hint}
		\begin{hint}
		 Type infinity for $\infty$ and -infinity for $-\infty$
		\end{hint}
		The value of the limit is
		 $\answer[given]{\infty}$

\includeinteractive{glp04.js}

	
\end{problem}

\begin{example}[example 5]
For the function, $f$, whose graph is shown below, find the limit,
\[
\lim_{x \to \infty}f(x).
\] 
This is an example of a \textbf{limit at infinity} and it is used to help 
us describe the end behavior of the function.

\includeinteractive{gl05.js}

%\[\graph[xmin=-5, xmax=20, ymin=-3, ymax=4]{4arctan(x)/\pi}\]

%\begin{center}
%\includegraphics[width = 4 in, height = 3 in, angle= -3]{graphical-limit05}
%\end{center}

\vspace{.25in}
Solution:  Since $x$ is approaching infinity, we look for a pattern on the right end of the graph.  
From the graph, we can see that the $y$-values are getting closer and closer to $2$ as the $x$ values 
increase without bound. We can conclude that 

\[
\lim_{x \to \infty}f(x) = 2,
\]

and we can describe the end behavior by saying that the line $y = 2$ is a \textbf{horizontal asymptote} for the graph of $f$.


\end{example}


\begin{problem}(problem 5)

  Use the graph of $y = f(x)$ given below to determine
  \[
  \lim_{x\to -\infty} f(x).
  \]
  
    \begin{hint}
      Look at the left end of the graph
    \end{hint}
    \begin{hint}
      The y-coordinates determine the limit
    \end{hint}
		\begin{hint}
		  Click on the graph and move the dot
		\end{hint}
		The value of the limit is
		 $\answer[given]{0}$

\includeinteractive{glp05.js}
	
\end{problem}

Here is a problem that encompasses all of the ideas in this section.

\begin{problem}(problem 6)
  \includeinteractive{glproblem.js}
	\begin{leash}{220}
	Use the above graph of $y = f(x)$ to answer the following questions.\\
	
  
  What are the following function values (undefined is a possibility)?  
		 \[f(-4) = \answer[given]{3}, \ f(-1) = \answer[given]{-1} \]
		 \[f(0) = \answer[given]{undefined}, \ f(1) = \answer[given]{2} \]
	   \[f(4) = \answer[given]{3}, \ f(6) = \answer[given]{-3} \]
	Find the following one-sided limits. (You can click on the graph and drag a point along it.)
	\[\lim_{x\to -4^-} f(x) = \answer[given]{3}, \ \lim_{x\to -4^+} f(x) = \answer[given]{1}\]
	\[\lim_{x\to -1^-} f(x) = \answer[given]{4}, \ \lim_{x\to -1^+} f(x) = \answer[given]{-1}\]
	Type infinity for $\infty$ and -infinity for $-\infty$.
	\[\lim_{x\to 0^-} f(x) = \answer[given]{-\infty}, \ \lim_{x\to 0^+} f(x) = \answer[given]{\infty}\]
	\[\lim_{x\to 1^-} f(x) = \answer[given]{1}, \ \lim_{x\to 1^+} f(x) = \answer[given]{2}\]
	\[\lim_{x\to 4^-} f(x) = \answer[given]{3}, \ \lim_{x\to 4^+} f(x) = \answer[given]{2}\]
	\[\lim_{x\to 6^-} f(x) = \answer[given]{0}, \ \lim_{x\to 6^+} f(x) = \answer[given]{2}\]
	Find the following limits at infinity. (Click on the grid and drag it if necessary.  
	Press the home button on the right to reset the grid.)
	\[\lim_{x\to -\infty} f(x) = \answer[given]{2}, \ \lim_{x\to \infty} f(x) = \answer[given]{1}\]
	
	\end{leash}
\end{problem}




\end{document}


\begin{tikzpicture}
\begin{axis}
\addplot[domain=-11:-5.1] {1/(x+5)};
\addplot[domain=-4.9:0] {-1/(x+5)};
\addplot[domain=0:2] {x^3+2};
\addplot[domain=2:10] {abs(x-7)-7};
\addplot[mark=*,only marks] coordinates {(-3,4)(0,-3)(2,10)};
\addplot[mark=*,fill=white,only marks] coordinates {(-3,-.5)
(0,-.2)(0,2)(2,-2)};
\end{axis}
\end{tikzpicture}


\[
\graph{(-3,1), (-2, 1), (-1, 1), (-1, 2),  (-1, 3), (-2, 2), (-3, 2); x^2}
\]

\[
\graph{2^x -3 \left\{ -10 \leq x \leq 1 \right\}, x^2 \left\{1 \leq x < 2 \right\}, x/4 \left\{ 2 \leq x < 3\right\}, 1/x \left\{3 \leq x \leq 10 \right\}}
\]

\[
\graph{(1,2); x^2}
\]


%fiddling around with sage

From Sage
\begin{sageOutput}
    show(diff(x^2))
\end{sageOutput}

From Geogebra
\geogebra{J4fcjvP9}{640}{480}


\begin{center}
\begin{tikzpicture}
\begin{axis}[axis lines = center, title={A Removable Discontinuity at $x=1$}]
\addplot[domain=0:0.98, 
    samples=100, color=blue]{x+1};
\addplot[smooth,mark=*,blue] plot coordinates {(1,1.5)};
\addplot[domain=1.02:2, 
    samples=100, color=blue]{x+1};
\addplot[smooth,mark=o,blue] plot coordinates {(1,2)};
\end{axis}
\end{tikzpicture}
\end{center}

\begin{center}
\begin{tikzpicture}
\begin{axis}[axis lines = center, title={A Removable Discontinuity at $x=2$}]
\addplot[domain=1:1.985, 
    samples=100, color=blue]{x^2 - 3};
\addplot[smooth,mark=o,blue] plot coordinates {(2,1)};
\addplot[domain=2.015:3, 
    samples=100, color=blue]{x^2 - 3};
\end{axis}
\end{tikzpicture}
\end{center}


\begin{center}
\begin{tikzpicture}
\begin{axis}[axis lines = center, title={A Jump Discontinuity at $x=1$}]
\addplot[domain=-1:1, 
    samples=100, color=blue]{x^2};
\addplot[smooth,mark=*,blue] plot coordinates {(1,1)};
\addplot[domain=1.02:2, 
    samples=100, color=blue]{x+1};
\addplot[smooth,mark=o,blue] plot coordinates {(1,2)};
\end{axis}
\end{tikzpicture}
\end{center}

\begin{center}
\begin{tikzpicture}
\begin{axis}[axis lines = center, title={A Jump Discontinuity at $x=2$}]
\addplot[domain=0:1.99, 
    samples=100, color=blue]{x^2 - 1};
\addplot[smooth,mark=o,blue] plot coordinates {(2,3)};
\addplot[domain=2:4, 
    samples=100, color=blue]{3-x};
\addplot[smooth,mark=*,blue] plot coordinates {(2,1)};
\end{axis}
\end{tikzpicture}
\end{center}

\begin{center}
\begin{tikzpicture}
\begin{axis}[axis lines = center, title={A Jump Discontinuity at $x=2$}]
\addplot[domain=0:1.99, 
    samples=100, color=blue]{x^2 - 1};
\addplot[smooth,mark=o,blue] plot coordinates {(2,3)};
\addplot[domain=2:4, 
    samples=100, color=blue]{3-x};
\addplot[smooth,mark=*,blue] plot coordinates {(2,1)};
\end{axis}
\end{tikzpicture}
\end{center}

\begin{center}
\begin{tikzpicture}
\begin{axis}[axis lines = center, title={A Jump Discontinuity at $x=-1$}]
\addplot[domain=-3:-1.02, 
    samples=100, color=blue]{x+3};
\addplot[smooth,mark=o,blue] plot coordinates {(-1,2)};
\addplot[domain=-0.99:1, 
    samples=100, color=blue]{-x^2 + 2};
\addplot[smooth,mark=o,blue] plot coordinates {(-1,1)};
\end{axis}
\end{tikzpicture}
\end{center}


\begin{center}
\begin{tikzpicture}
\begin{axis}[axis lines = center, title={An infinite discontinuity at $x=1$}]
\addplot[domain=-1:0.9, 
    samples=100, color=blue]{1/(x-1)^2};
\addplot[domain=1.1:3, 
    samples=100, color=blue]{1/(x-1)^2};
\vasymptote {1}
\end{axis}
\end{tikzpicture}
\end{center}
