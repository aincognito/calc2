\documentclass[handout]{ximera}
\usepgfplotslibrary{fillbetween}
%% You can put user macros here
%% However, you cannot make new environments



\newcommand{\ffrac}[2]{\frac{\text{\footnotesize $#1$}}{\text{\footnotesize $#2$}}}
\newcommand{\vasymptote}[2][]{
    \draw [densely dashed,#1] ({rel axis cs:0,0} -| {axis cs:#2,0}) -- ({rel axis cs:0,1} -| {axis cs:#2,0});
}


\graphicspath{{./}{firstExample/}}

\usepackage{amsmath}
\usepackage{amssymb}
\usepackage{array}
\usepackage[makeroom]{cancel} %% for strike outs
\usepackage{pgffor} %% required for integral for loops
\usepackage{tikz}
\usepackage{tikz-cd}
\usepackage{tkz-euclide}
\usetikzlibrary{shapes.multipart}


\usetkzobj{all}
\tikzstyle geometryDiagrams=[ultra thick,color=blue!50!black]


\usetikzlibrary{arrows}
\tikzset{>=stealth,commutative diagrams/.cd,
  arrow style=tikz,diagrams={>=stealth}} %% cool arrow head
\tikzset{shorten <>/.style={ shorten >=#1, shorten <=#1 } } %% allows shorter vectors

\usetikzlibrary{backgrounds} %% for boxes around graphs
\usetikzlibrary{shapes,positioning}  %% Clouds and stars
\usetikzlibrary{matrix} %% for matrix
\usepgfplotslibrary{polar} %% for polar plots
\usepgfplotslibrary{fillbetween} %% to shade area between curves in TikZ



%\usepackage[width=4.375in, height=7.0in, top=1.0in, papersize={5.5in,8.5in}]{geometry}
%\usepackage[pdftex]{graphicx}
%\usepackage{tipa}
%\usepackage{txfonts}
%\usepackage{textcomp}
%\usepackage{amsthm}
%\usepackage{xy}
%\usepackage{fancyhdr}
%\usepackage{xcolor}
%\usepackage{mathtools} %% for pretty underbrace % Breaks Ximera
%\usepackage{multicol}



\newcommand{\RR}{\mathbb R}
\newcommand{\R}{\mathbb R}
\newcommand{\C}{\mathbb C}
\newcommand{\N}{\mathbb N}
\newcommand{\Z}{\mathbb Z}
\newcommand{\dis}{\displaystyle}
%\renewcommand{\d}{\,d\!}
\renewcommand{\d}{\mathop{}\!d}
\newcommand{\dd}[2][]{\frac{\d #1}{\d #2}}
\newcommand{\pp}[2][]{\frac{\partial #1}{\partial #2}}
\renewcommand{\l}{\ell}
\newcommand{\ddx}{\frac{d}{\d x}}

\newcommand{\zeroOverZero}{\ensuremath{\boldsymbol{\tfrac{0}{0}}}}
\newcommand{\inftyOverInfty}{\ensuremath{\boldsymbol{\tfrac{\infty}{\infty}}}}
\newcommand{\zeroOverInfty}{\ensuremath{\boldsymbol{\tfrac{0}{\infty}}}}
\newcommand{\zeroTimesInfty}{\ensuremath{\small\boldsymbol{0\cdot \infty}}}
\newcommand{\inftyMinusInfty}{\ensuremath{\small\boldsymbol{\infty - \infty}}}
\newcommand{\oneToInfty}{\ensuremath{\boldsymbol{1^\infty}}}
\newcommand{\zeroToZero}{\ensuremath{\boldsymbol{0^0}}}
\newcommand{\inftyToZero}{\ensuremath{\boldsymbol{\infty^0}}}


\newcommand{\numOverZero}{\ensuremath{\boldsymbol{\tfrac{\#}{0}}}}
\newcommand{\dfn}{\textbf}
%\newcommand{\unit}{\,\mathrm}
\newcommand{\unit}{\mathop{}\!\mathrm}
%\newcommand{\eval}[1]{\bigg[ #1 \bigg]}
\newcommand{\eval}[1]{ #1 \bigg|}
\newcommand{\seq}[1]{\left( #1 \right)}
\renewcommand{\epsilon}{\varepsilon}
\renewcommand{\iff}{\Leftrightarrow}

\DeclareMathOperator{\arccot}{arccot}
\DeclareMathOperator{\arcsec}{arcsec}
\DeclareMathOperator{\arccsc}{arccsc}
\DeclareMathOperator{\si}{Si}
\DeclareMathOperator{\proj}{proj}
\DeclareMathOperator{\scal}{scal}
\DeclareMathOperator{\cis}{cis}
\DeclareMathOperator{\Arg}{Arg}
%\DeclareMathOperator{\arg}{arg}
\DeclareMathOperator{\Rep}{Re}
\DeclareMathOperator{\Imp}{Im}
\DeclareMathOperator{\sech}{sech}
\DeclareMathOperator{\csch}{csch}
\DeclareMathOperator{\Log}{Log}

\newcommand{\tightoverset}[2]{% for arrow vec
  \mathop{#2}\limits^{\vbox to -.5ex{\kern-0.75ex\hbox{$#1$}\vss}}}
\newcommand{\arrowvec}{\overrightarrow}
\renewcommand{\vec}{\mathbf}
\newcommand{\veci}{{\boldsymbol{\hat{\imath}}}}
\newcommand{\vecj}{{\boldsymbol{\hat{\jmath}}}}
\newcommand{\veck}{{\boldsymbol{\hat{k}}}}
\newcommand{\vecl}{\boldsymbol{\l}}
\newcommand{\utan}{\vec{\hat{t}}}
\newcommand{\unormal}{\vec{\hat{n}}}
\newcommand{\ubinormal}{\vec{\hat{b}}}

\newcommand{\dotp}{\bullet}
\newcommand{\cross}{\boldsymbol\times}
\newcommand{\grad}{\boldsymbol\nabla}
\newcommand{\divergence}{\grad\dotp}
\newcommand{\curl}{\grad\cross}
%% Simple horiz vectors
\renewcommand{\vector}[1]{\left\langle #1\right\rangle}


\outcome{Use Riemann sums to approximate the area under a curve}

\title{4.4 Riemann Sums}

\begin{document}

\begin{abstract}
We compute Riemann Sums to approximate the area under a curve.
\end{abstract}

\maketitle

The first major problem in calculus was the tangent line problem. We now turn to the second major problem in calculus, the area problem.
We saw that the derivative solved the tangent line problem and it turns out that the anti-derivative solves the area problem.
\section{The Area Problem}


Let $f(x)$ be a non-negative, continuous function on the interval $[a,b]$. The \textbf{area problem} asks us to find the area under the graph of $y = f(x)$
over the interval $[a,b]$.

\begin{image}
\begin{tikzpicture}
\begin{axis}[axis x line=  center, axis y line = none, xtick={-1, 1}, 
xticklabels={$x=a$, $x=b$}, title={The Area Problem}] 

\addplot[name path = A, domain=-1:1, samples = 100, color=black, thick]{x^3-x+1};
\addplot[name path = B, domain=-1:1, samples=100, color=black]{0};

\addplot[blue!10] fill between[of=A and B];
%\addplot[red!10] fill between[of=C and D];
\addplot[thick] coordinates {(-1,0) (-1, 1)};
\addplot[thick] coordinates {(1,0) (1, 1)};
%\addplot[white] coordinates {(1.5,-0.3) (1.8,-.3)};
\addplot[<->] coordinates {(-1.5,-.5) (-1.5, 2)};
\addplot[<-] coordinates {(-1.7,0) (1.5, 0)};


\node at (axis cs: 0,-.45){Find the area of the shaded region};
\node at (axis cs: 0.2,1.25){$y = f(x)$};
%\node at (axis cs: 1.6,0){$x$};
\node at (axis cs: -1.5,2.1){$y$};


\end{axis}
%\legend{$y = f(x)$, , secant line, tangent line, };
\end{tikzpicture}
\end{image}

To solve this problem, we begin by approximating the area under the curve using rectangles. The sum of the areas of these rectangles is called a Riemann Sum.

\begin{image}
\begin{tikzpicture}
\begin{axis}[axis x line=  center, axis y line = none, xtick={-1, 1}, 
xticklabels={$a$, $b$}, title={Riemann Sum}] 

\addplot[domain=-1:1, samples = 100, color=black, thick]{x^3-x+1};
\addplot[name path = A, thick, domain=-1:-0.5, samples=100, color=black]{1.1};
\addplot[name path = B, thick, domain=-0.5:0, samples=100, color=black]{1.3};
\addplot[name path = C, thick, domain= 0:0.5, samples=100, color=black]{0.8};
\addplot[name path = D, thick, domain= 0.5:1, samples=100, color=black]{1};
\addplot[name path = E, domain=-1:-.5, samples=100, color=black]{0};
\addplot[name path = F, domain=-.5:0, samples=100, color=black]{0};
\addplot[name path = G, domain=0:.5, samples=100, color=black]{0};
\addplot[name path = H, domain=.5:1, samples=100, color=black]{0};

\addplot[blue!10] fill between[of=A and E];
\addplot[blue!10] fill between[of=B and F];
\addplot[blue!10] fill between[of=C and G];
\addplot[blue!10] fill between[of=D and H];

\addplot[thick] coordinates {(-1,0) (-1, 1.1)};
\addplot[thick] coordinates {(-0.5,0) (-0.5, 1.3)};
\addplot[thick] coordinates {(0,0) (0, 1.3)};
\addplot[thick] coordinates {(0.5,0) (0.5, 1)};
\addplot[thick] coordinates {(1,0) (1, 1)};

\addplot[<->] coordinates {(-1.5,-.5) (-1.5, 2)};
\addplot[<-] coordinates {(-1.7,0) (1.5, 0)};

\node at (axis cs: 0,-.45){Sum of the areas of the rectangles};
%\node at (axis cs: 0.2,1.25){$y = f(x)$};
\node at (axis cs: -1.5,2.1){$y$};

\end{axis}
\end{tikzpicture}
\end{image}


To find the exact area under the curve we will need to use infinitely many rectangles.  This will lead us into the next section on the Definite Integral.
For now, we begin with a special notation that will allow us to write the sum of the areas of many rectangles in a compact form.

\section{Summation Notation}

In mathematics, the symbol $\sum$ is used to denote summation.
\begin{definition}[Summation Notation]
\[\sum_{i = 1}^n \ a_i = a_1 + a_2 + \dots + a_n.\]
The parameter $i$ is called the \textbf{index of summation} and it begins at the indicated value, $i=1$ in this case; 
and it increments by one unit until it reaches the terminal value (in this case, $n$).
\end{definition}

\begin{example}[example1]
\[\sum_{i = 1}^4 \ (i^2 + 3) = 4 + 7 + 12 + 19 = 42.\]
\end{example}

\begin{example}[example 2]
\[\sum_{i = 2}^7 \ (2i +5) = 9 + 11 + 13 + 15 + 17 + 19 = 84.\]
\end{example}

\begin{problem}(problem 2a)
Find the value of the sum:
\[\sum_{i = 1}^5 \ (4i -1).\]
The sum is $\answer{55}.$
\end{problem}

\begin{problem}(problem 2b)
Find the value of the sum:
\[\sum_{i = 3}^6 \ (i^2 - i + 1).\]
The sum is $\answer{72}.$
\end{problem}


\begin{theorem}[Special Sums] Sums of powers of $i$:

%\begin{align*}
%\sum_{i = 1}^n \ i &= 1 + 2 + 3+ \cdots + n = \frac{n(n+1)}{2},\\
%\sum_{i = 1}^n \ i^2 &= 1^2 + 2^2 + 3^2 + \cdots + n^2 = \frac{n(n+1)(2n+1)}{6}\; \; \text{and}\\
%\sum_{i = 1}^n \ i^3 &= 1^3 + 2^3+ 3^3 + \cdots + n^3 = \frac{n^2(n+1)^2}{4}.
%\end{align*}


\begin{itemize}
\item[1.] $\displaystyle{\sum_{i = 1}^n \ i = 1 + 2 + 3+ \cdots + n = \frac{n(n+1)}{2}},$
\item[2.] $\displaystyle{\sum_{i = 1}^n \ i^2 = 1^2 + 2^2 + 3^2 + \cdots + n^2 = \frac{n(n+1)(2n+1)}{6}}$ and
\item[3.] $\displaystyle{\sum_{i = 1}^n \ i^3 = 1^3 + 2^3+ 3^3 + \cdots + n^3 = \frac{n^2(n+1)^2}{4}}$.
\end{itemize}
\end{theorem}

\begin{example}[example 3]
Find the value of the sum:
\[\sum_{i = 5}^{12} \ i^2.\]
This sum can be found by rewriting it as a difference of two sums and using formula (2), above, twice:
 \[\sum_{i = 5}^{12} \ i^2 = \sum_{i = 1}^{12}\ i^2 - \sum_{i = 1}^{4} \ i^2 = \frac{12\cdot 13\cdot 25}{6} - \frac{4\cdot 5\cdot 9}{6} = 620. \]
\end{example}


\begin{problem}(problem 3a)
Find the value of the sum:
\[\sum_{i = 1}^{100} \ i.\]
The sum is $\answer{5050}.$
\end{problem}

\begin{problem}(problem 3b)
Find the value of the sum:
\[\sum_{i = 1}^{10} \ i^2.\]
The sum is $\answer{385}.$
\end{problem}

\begin{problem}(problem 3c)
Find the value of the sum:
\[\sum_{i = 1}^5 \ i^3.\]
The sum is $\answer{225}.$
\end{problem}

\section{Riemann Sums}


Let $f(x)$ be defined on a closed interval $[a,b]$. Partition the interval 
$[a,b]$ into $n$ sub-intervals, each of width
\[\Delta x = \frac{b-a}{n}.\]
 Denote the endpoints of the sub-intervals by $x_i$, 
so that 
\[x_i =  a + i\cdot \Delta x,\]
 where $i = 0, 1, 2, \ldots, n$. (Note that $x_0 = a$ and $x_n = b$.)  

\begin{image}
\begin{tikzpicture}
\node at (5,1.2) {A partition of $[a,b]$ into $n$ sub-intervals};
\draw(0,0)--(10,0);
%\draw(5,0)--(6,0) node[below] {$\ldots$};
\node at (6,-.4){$\dots$};
\draw[<-](3,.4)--(3.75,.4) node[above] {$\Delta x$};
\draw[->](3.75,.4)--(4.5,.4);

  \draw(0,5pt)--(0,-5pt) node[below] {$a = x_0$};
	\draw(1.5,5pt)--(1.5,-5pt) node[below] {$x_1$};
	\draw(3,5pt)--(3,-5pt) node[below] {$x_{2}$};
	\draw(4.5,5pt)--(4.5,-5pt) node[below] {$x_{3}$};
	\draw(8.5,5pt)--(8.5,-5pt) node[below] {$x_{n-1}$};
	\draw(10,5pt)--(10,-5pt);
\node at (10, -.35){$x_n=b$};
 \end{tikzpicture}
\end{image}


To create rectangles under the curve, we next select \textbf{sample points}- one in each sub-interval
so as to space them out roughly evenly over the interval $[a,b]$. The sample points are denoted by an asterisk,
like $x_4^*$ for the fourth sample point and $x_i^*$ for the $i$-th sample point.


\begin{image}
\begin{tikzpicture}
\node at (5,1.2) {The sample points, $x_i^*$};
\draw(0,0)--(10,0);
%\draw(5,0)--(6,0) node[below] {$\ldots$};
\node at (6,-.4){$\dots$};


  \draw(0,5pt)--(0,-5pt) node[below] {$a$};
  \draw[thick, blue](.6,-5pt)--(.6,5pt) node[above] {$x_1^*$};
	\draw(1.5,5pt)--(1.5,-5pt) node[below] {$x_1$};
	\draw[thick, blue](2.1,-5pt)--(2.1,5pt) node[above] {$x_2^*$};
	\draw(3,5pt)--(3,-5pt) node[below] {$x_{2}$};
	\draw[thick, blue](3.9,-5pt)--(3.9,5pt) node[above] {$x_3^*$};
	\draw(4.5,5pt)--(4.5,-5pt) node[below] {$x_{3}$};
	\draw(8.5,5pt)--(8.5,-5pt) node[below] {$x_{n-1}$};
	\draw[thick, blue](9.1,-5pt)--(9.1,5pt) node[above] {$x_n^*$};
	\draw(10,5pt)--(10,-5pt);
\node at (10, -.35){$b$};
 \end{tikzpicture}
\end{image}

In general, the $i$-th sample point, $x_i^*$ lies in the interval $[x_{i-1}, x_i]$. The sample points 
are used to determine the heights of the rectangles used in our Riemann Sum. Specifically, the height of the first rectangle
will be $f(x_1^*)$, the height of the second rectangle will be $f(x_2^*)$ and so on until we reach the $n$-th and final rectangle, 
whose height will be $f(x_n^*)$.

\begin{image}
\begin{tikzpicture}
\begin{axis}[axis x line=  center, axis y line = none, xtick={-0.87, -0.32, 0.21, 0.8}, 
xticklabels={$x_1^*$, $x_2^*$, $x_3^*$,  $x_4^*$}, title={Riemann Sum}] 

\addplot[domain=-1:1, samples = 100, color=black, thick]{x^3-x+1};
\addplot[name path = A, thick, domain=-1:-0.5, samples=100, color=black]{1.2};
\addplot[name path = B, thick, domain=-0.5:0, samples=100, color=black]{1.3};
\addplot[name path = C, thick, domain= 0:0.5, samples=100, color=black]{0.8};
\addplot[name path = D, thick, domain= 0.5:1, samples=100, color=black]{.71};
\addplot[name path = E, domain=-1:-.5, samples=100, color=black]{0};
\addplot[name path = F, domain=-.5:0, samples=100, color=black]{0};
\addplot[name path = G, domain=0:.5, samples=100, color=black]{0};
\addplot[name path = H, domain=.5:1, samples=100, color=black]{0};

\addplot[blue!10] fill between[of=A and E];
\addplot[blue!10] fill between[of=B and F];
\addplot[blue!10] fill between[of=C and G];
\addplot[blue!10] fill between[of=D and H];

\addplot[thick] coordinates {(-1,0) (-1, 1.2)};
\addplot[dashed, blue] coordinates {(-0.87,0) (-0.87, 1.2)};
\addplot[dashed, blue] coordinates {(-1.5,1.2) (-1,1.2)};
\node at (axis cs: -1.8, 1.2){$f(x_1^*)$};
\addplot[thick] coordinates {(-0.5,0) (-0.5, 1.3)};
\addplot[dashed, blue] coordinates {(-0.32,0) (-0.32, 1.3)};
\addplot[thick] coordinates {(0,0) (0, 1.3)};
\addplot[dashed, blue] coordinates {(0.21,0) (0.21, 0.8)};
\addplot[thick] coordinates {(0.5,0) (0.5, .8)};
\addplot[thick] coordinates {(1,0) (1, 1)};
\addplot[dashed, blue] coordinates {(0.8,0) (0.8, 0.71)};
%axes
\addplot[<->] coordinates {(-1.5,-.5) (-1.5, 2)};
\addplot[<-] coordinates {(-2,0) (1.8, 0)};

\node at (axis cs: -0.3,-.6){The height of the $i$-th rectangle is $f(x_i^*)$};
\node at (axis cs: -.7,1.5){$y = f(x)$};
\node at (axis cs: -1.5,2.1){$y$};

\end{axis}
\end{tikzpicture}
\end{image}

%select one sample point from each subinterval
%call the sample points x_1^*, x_2^* etc
%in general, the i^{th} sample point is x_i^*
%plot the sample points on the curve y = f(x)
%over the first sub interval, construct a rectangle with height f(x_1^*)
% over the second subinterval, create a rectangle...
%over the nth subinterval...
%over the i^{th} sub interval...

%now add the sample points and the rectangles
%include a figure or 2 with sample points and rectangles.

\begin{definition}[Riemann Sum]
The \textbf{Riemann Sum} of $f(x)$ on $[a,b]$ with $n$ sub-intervals is given by 
\[\sum_{i = 1}^n \ f(x_i^*) \Delta x,\]
where $x_{i-1} \leq x_i^* \leq x_i$.  The numbers $x_i^*$ are called \textbf{sample points} for the sub-intervals $[x_{i-1}, x_i]$.
\end{definition}

\begin{remark}
For a function $f(x)$ which is positive on the interval $[a,b]$
the Riemann Sum represents the sum of the areas of $n$ rectangles.  
The heights of the rectangles are given by the function values at the sample points, $f(x_i^*)$ and the widths are given by $\Delta x = (b-a)/n$.
\end{remark}


\begin{example}[example 4]
Below is an interactive graph of the parabola $y=x^2$.  The Riemann Sum uses the rectangles in the figure to approximate the area under the curve.
The sample points $x_i^*$ are taken to be endpoints of the sub-interval $[x_{i-1}, x_i]$. The orange rectangles use $x_i^* = x_{i-1}$, i.e., 
a \textbf{left-endpoint} approximation and the purple rectangles use a \textbf{right-endpoint} approximation with $x_i^* = x_i$.
Use the slider to convince yourself that as the number of rectangles, $n$ increases, the width $\Delta x$ decreases and
the area of the rectangles approaches the area under the curve. 
\begin{onlineOnly}
  \begin{center}
    \desmos{j9hdqgwiwg}{800}{600}
  \end{center}
\end{onlineOnly}
\end{example}




%Below is an applet that shows Riemann Sums visually.  You can select the number of rectangles, $n$ and the location of the sample points $x_i^*$ within the sub-intervals.


%\begin{foldable}
%\geogebra{bsC9Pwu3}{900}{600}
%\end{foldable}

%\begin{center}
%\textbf{Left and Right Riemann Sums}
%\end{center}
%Typically, the sample points $x_i^*$ are chosen according to a prescribed rule, such as ``$x_i^* = $ left end point'', i.e. $x_i^* = x_{i-1}$, or 
%``$x_i^* = $ right  end point'', i.e. $x_i^* = x_i$.\\
%Below is an applet that shows Riemann Sums using left and right endpoints as the sample points.

%``$x_i^* = $ mid-point'', i.e. $x_i^* = (x_{i-1} + x_i)/2$

%\begin{foldable}
%\geogebra{Q54eSw2n}{900}{600}
%\end{foldable}

%\begin{center}
%\textbf{Upper and Lower Sums}
%\end{center}
%We can also choose the sample points $x_i^*$ to be at locations where the function $f(x)$ has a certain property.  For example, we can choose $x_i^*$
%such that $f(x)$ has an absolute max or min on the interval $[x_{i-1}, x_i]$ 
%at $x = x_i^*$.  The Riemann sums in these cases are called upper or lower sums, respectively. \\
%Below is an applet where the sample points are chosen so that the Riemann Sums are upper and lower sums.


%\begin{foldable}
%\geogebra{Z5Nvhks7}{900}{600}
%\end{foldable}



%\geogebra{jF23GzmS}{800}{600}



\begin{example}[example 5]

Below is the graph corresponding to a Riemann Sum for the curve $f(x) = e^{-x}$ over the interval $[-1,1]$ using 4 rectangles 
and left endpoints as the sample points.

\begin{image}
\begin{tikzpicture}
\begin{axis}[axis x line=  bottom, axis y line = center, xtick={-1, -0.5, 0, 0.5, 1},  
ytick={0, 1, 2}, title={Riemann Sum for $f(x)=e^{-x} $ on $[-1,1]$ with $n=4$ using Left Endpoints}]
\addplot[domain=-1:1, 
    samples=100, color=black]{1/e^x};
\addplot[smooth,mark=*,black] plot coordinates {(0,0)};
\addplot[smooth,mark=*,blue] plot coordinates {(-1,e)};
\addplot[smooth,mark=*,blue] plot coordinates {(-0.5,sqrt e)};      
\addplot[smooth,mark=*,blue] plot coordinates {(0,1)};
\addplot[smooth,mark=*,blue] plot coordinates {(0.5,1/ sqrt e)};
%\addplot[smooth,mark=*,blue] plot coordinates {(1,1.64)};
%\legend{$y = f(x)$, , secant line, tangent line, };
\addplot[blue] coordinates {(-1, e)  (-0.5, e)};
\addplot[blue] coordinates {(-0.5,sqrt e) (0, sqrt e)};
\addplot[blue] coordinates {(0, 1) (0.5,1)};
\addplot[blue] coordinates {(0.5,1/ sqrt e) (1, 1/sqrt e)};
\addplot[blue] coordinates {(-1, 0) (-1, e)};
\addplot[blue] coordinates {(-0.5,0) (-0.5, e)};
\addplot[blue] coordinates {(0,0) (0, sqrt e)};
\addplot[blue] coordinates {(0.5,0) (0.5,1)};
\addplot[blue] coordinates {(1,0) (1, 1/ sqrt e)};
\end{axis}
\end{tikzpicture}
\end{image}
The value of this Left Riemann Sum is 
\[f(-1) \Delta x + f(-0.5) \Delta x +f(0) \Delta x +f(0.5) \Delta x,\]
which equals
\[e^{1} (0.5) + e^{0.5} (0.5) + e^0 (0.5) + e^{-0.5} (0.5) \approx 2.987.\]
\end{example}


\begin{problem}(problem 5)
Below is the graph corresponding to a Riemann Sum for the curve $f(x) = x^2 + 1$ over the interval $[0,1]$ using 5 rectangles 
and left endpoints as the sample points.

\begin{image}
\begin{tikzpicture}
\begin{axis}[axis x line=  bottom, axis y line = left, xtick={0, 0.2, 0.4, 0.6, 0.8, 1},  
ytick={0, 1, 2}, title={Riemann Sum for $f(x)=x^2 + 1 $ on $[0,1]$ with $n=5$ using Left Endpoints}]
\addplot[domain=0:1, 
    samples=100, color=black]{x^2 + 1};
\addplot[smooth,mark=*,black] plot coordinates {(0,0)};
\addplot[smooth,mark=*,blue] plot coordinates {(0,1)};
\addplot[smooth,mark=*,blue] plot coordinates {(0.2,1.04)};      
\addplot[smooth,mark=*,blue] plot coordinates {(0.4,1.16)};
\addplot[smooth,mark=*,blue] plot coordinates {(0.6,1.36)};
\addplot[smooth,mark=*,blue] plot coordinates {(0.8,1.64)};
%\legend{$y = f(x)$, , secant line, tangent line, };
\addplot[blue] coordinates {(0,1)  (0.2,1)};
\addplot[blue] coordinates {(0.2,1.04) (0.4,1.04)};
\addplot[blue] coordinates {(0.4,1.16) (0.6,1.16)};
\addplot[blue] coordinates {(0.6,1.36) (0.8,1.36)};
\addplot[blue] coordinates {(0.8, 1.64) (1,1.64)};
\addplot[blue] coordinates {(0.2,0) (0.2,1.04)};
\addplot[blue] coordinates {(0.4,0) (0.4, 1.16)};
\addplot[blue] coordinates {(0.6,0) (0.6,1.36)};
\addplot[blue] coordinates {(0.8,0)  (0.8, 1.64)};
\addplot[blue] coordinates {(1,0) (1, 1.64)};
\end{axis}
\end{tikzpicture}
\end{image}

Compute the Riemann Sum for the function $f(x) = x^2 + 1$ on the interval $[0, 1]$ using $n = 5$ rectangles and choosing the sample points to be left end-points.\\

The Left Riemann Sum is $\answer{1.24}$.
\end{problem} 


\begin{example}[example 6]
Below is the graph corresponding to the Riemann Sum for the curve $f(x) = e^{-x}$ over the interval $[-1,1]$ using 4 rectangles
and right endpoints as the sample points.


\begin{image}
\begin{tikzpicture}
\begin{axis}[axis x line=  bottom, axis y line = center, xtick={-1, -0.5,0,0.5,1},  
ytick={0, 1, 2}, title={Riemann Sum for $f(x)=e^{-x} $ on $[-1,1]$ with $n=4$ using Right Endpoints}]
\addplot[domain=-1:1, 
    samples=100, color=black]{1/e^x};
\addplot[smooth,mark=*,black] plot coordinates {(0,0)};
\addplot[smooth,mark=*,purple] plot coordinates {(-0.5,sqrt e)}; 
\addplot[smooth,mark=*,purple] plot coordinates {(0,1)};  
\addplot[smooth,mark=*,purple] plot coordinates {(0.5,1/ sqrt e)};  
\addplot[smooth,mark=*,purple] plot coordinates {(1,1/e)};
\addplot[purple] coordinates {(-1,sqrt e)  (-0.5,sqrt e)};
\addplot[purple] coordinates {(-0.5,1) (0,1)};
\addplot[purple] coordinates {(0,1/ sqrt e) (0.5,1/sqrt e)};
\addplot[purple] coordinates {(0.5,1/e) (1, 1/e)};
\addplot[purple] coordinates {(-1,0) (-1,sqrt e)};
\addplot[purple] coordinates {(-0.5,0) (-0.5,sqrt e)};
\addplot[purple] coordinates {(0,0) (0,1)};
\addplot[purple] coordinates {(0.5,0) (0.5, 1/sqrt e)};
\addplot[purple] coordinates {(1,0) (1,1/e)};

\end{axis}
%\legend{$y = f(x)$, , secant line, tangent line, };
\end{tikzpicture}
\end{image}
The value of this Right Riemann Sum is 
\[ f(-0.5) \Delta x + f(0) \Delta x +f(0.5) \Delta x + f(1) \Delta x ,\]
which equals
\[e^{0.5} (0.5) + e^0 (0.5) + e^{-0.5} (0.5) +  e^{-1} (0.5) \approx 1.812.\]
\end{example}


\begin{problem}(problem 6)
Below is the graph corresponding to a Riemann Sum for the curve $f(x) = x^2 + 1$ over the interval $[0,1]$ using 5 rectangles 
and right endpoints as the sample points.

\begin{image}
\begin{tikzpicture}
\begin{axis}[axis x line=  bottom, axis y line = left, xtick={0, 0.2,0.4,0.6,0.8,1},  
ytick={0, 1, 2}, title={Riemann Sum for $f(x)=x^2 + 1 $ on $[0,1]$ with $n=5$ using Right Endpoints}]
\addplot[domain=0:1, 
    samples=100, color=black]{x^2 + 1};
\addplot[smooth,mark=*,black] plot coordinates {(0,0)};
\addplot[smooth,mark=*,purple] plot coordinates {(0.2,1.04)};
\addplot[smooth,mark=*,purple] plot coordinates {(0.4,1.16)}; 
\addplot[smooth,mark=*,purple] plot coordinates {(0.6,1.36)};  
\addplot[smooth,mark=*,purple] plot coordinates {(0.8,1.64)};  
\addplot[smooth,mark=*,purple] plot coordinates {(1,2)};
\addplot[purple] coordinates {(0,1.04)  (0.2,1.04)};
\addplot[purple] coordinates {(0.2,1.16) (0.4,1.16)};
\addplot[purple] coordinates {(0.4,1.36) (0.6,1.36)};
\addplot[purple] coordinates {(0.6,1.64) (0.8,1.64)};
\addplot[purple] coordinates {(0.8, 2) (1,2)};
\addplot[purple] coordinates {(0.2,0) (0.2,1.16)};
\addplot[purple] coordinates {(0.4,0) (0.4,1.36)};
\addplot[purple] coordinates {(0.6,0) (0.6,1.64)};
\addplot[purple] coordinates {(0.8,0) (0.8, 2)};
\addplot[purple] coordinates {(1,2) (1,0)};

\end{axis}
%\legend{$y = f(x)$, , secant line, tangent line, };
\end{tikzpicture}
\end{image}
Compute the Riemann Sum for the function $f(x) = x^2 + 1$ on the interval $[0, 1]$ using $n = 5$ rectangles and choosing the sample points to be right end-points.\\

The Right Riemann Sum is $\answer{1.44}$.
\end{problem} 




\end{document}














\begin{image}

\begin{tikzpicture}
\foreach \x in {1,2,...,5,7,8,...,12}
\foreach \y in {1,...,5}
{
\draw (\x,\y) +(-.5,-.5) rectangle ++(.5,.5);
\draw (\x,\y) node{\x,\y};
}
\end{tikzpicture}
\end{image}


\begin{image}
\begin{tikzpicture}
\foreach \x in {1,2,...,10}
{
\draw (\x,0) -- (\x ,1/ \x);
\draw (\x -1, 1/ \x) -- ( \x,  1/ \x);
}
\end{tikzpicture}
\end{image}




[color=blue]

\draw (0,0) --(0,1.04);
\draw (0,1.04) --(0.2,1.04);
\draw (0.2,0) --(0.2,1.16);
\draw[color=blue] (0.2,1.16) --(0.4,1.16);
\draw[color=blue] (0.4,0) --(0.4,1.36);
\draw[color=blue] (0.4,1.36) --(0.6,1.36);
\draw[color=blue] (0.6,0) --(0.6,1.64);
\draw[color=blue] (0.6,1.64) --(0.8,1.64);
\draw[color=blue] (0.8,0) --(0.8,2);
\draw[color=blue] (0.8,2) --(1,2);
\draw[color=blue] (1,0) --(1,2);

$x_0$, $x_1$, $x_2$, $x_3$, $x_4$, $x_5$
xticklabels={$x_0$,$x_1$,$x_2$,$x_3$,$x_4$,$x_5$},

xticklabels={0, 0.2, 0.4, 0.6, 0.8, 1},
