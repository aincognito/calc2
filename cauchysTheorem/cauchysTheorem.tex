\documentclass[handout]{ximera}

%% You can put user macros here
%% However, you cannot make new environments



\newcommand{\ffrac}[2]{\frac{\text{\footnotesize $#1$}}{\text{\footnotesize $#2$}}}
\newcommand{\vasymptote}[2][]{
    \draw [densely dashed,#1] ({rel axis cs:0,0} -| {axis cs:#2,0}) -- ({rel axis cs:0,1} -| {axis cs:#2,0});
}


\graphicspath{{./}{firstExample/}}

\usepackage{amsmath}
\usepackage{amssymb}
\usepackage{array}
\usepackage[makeroom]{cancel} %% for strike outs
\usepackage{pgffor} %% required for integral for loops
\usepackage{tikz}
\usepackage{tikz-cd}
\usepackage{tkz-euclide}
\usetikzlibrary{shapes.multipart}


\usetkzobj{all}
\tikzstyle geometryDiagrams=[ultra thick,color=blue!50!black]


\usetikzlibrary{arrows}
\tikzset{>=stealth,commutative diagrams/.cd,
  arrow style=tikz,diagrams={>=stealth}} %% cool arrow head
\tikzset{shorten <>/.style={ shorten >=#1, shorten <=#1 } } %% allows shorter vectors

\usetikzlibrary{backgrounds} %% for boxes around graphs
\usetikzlibrary{shapes,positioning}  %% Clouds and stars
\usetikzlibrary{matrix} %% for matrix
\usepgfplotslibrary{polar} %% for polar plots
\usepgfplotslibrary{fillbetween} %% to shade area between curves in TikZ



%\usepackage[width=4.375in, height=7.0in, top=1.0in, papersize={5.5in,8.5in}]{geometry}
%\usepackage[pdftex]{graphicx}
%\usepackage{tipa}
%\usepackage{txfonts}
%\usepackage{textcomp}
%\usepackage{amsthm}
%\usepackage{xy}
%\usepackage{fancyhdr}
%\usepackage{xcolor}
%\usepackage{mathtools} %% for pretty underbrace % Breaks Ximera
%\usepackage{multicol}



\newcommand{\RR}{\mathbb R}
\newcommand{\R}{\mathbb R}
\newcommand{\C}{\mathbb C}
\newcommand{\N}{\mathbb N}
\newcommand{\Z}{\mathbb Z}
\newcommand{\dis}{\displaystyle}
%\renewcommand{\d}{\,d\!}
\renewcommand{\d}{\mathop{}\!d}
\newcommand{\dd}[2][]{\frac{\d #1}{\d #2}}
\newcommand{\pp}[2][]{\frac{\partial #1}{\partial #2}}
\renewcommand{\l}{\ell}
\newcommand{\ddx}{\frac{d}{\d x}}

\newcommand{\zeroOverZero}{\ensuremath{\boldsymbol{\tfrac{0}{0}}}}
\newcommand{\inftyOverInfty}{\ensuremath{\boldsymbol{\tfrac{\infty}{\infty}}}}
\newcommand{\zeroOverInfty}{\ensuremath{\boldsymbol{\tfrac{0}{\infty}}}}
\newcommand{\zeroTimesInfty}{\ensuremath{\small\boldsymbol{0\cdot \infty}}}
\newcommand{\inftyMinusInfty}{\ensuremath{\small\boldsymbol{\infty - \infty}}}
\newcommand{\oneToInfty}{\ensuremath{\boldsymbol{1^\infty}}}
\newcommand{\zeroToZero}{\ensuremath{\boldsymbol{0^0}}}
\newcommand{\inftyToZero}{\ensuremath{\boldsymbol{\infty^0}}}


\newcommand{\numOverZero}{\ensuremath{\boldsymbol{\tfrac{\#}{0}}}}
\newcommand{\dfn}{\textbf}
%\newcommand{\unit}{\,\mathrm}
\newcommand{\unit}{\mathop{}\!\mathrm}
%\newcommand{\eval}[1]{\bigg[ #1 \bigg]}
\newcommand{\eval}[1]{ #1 \bigg|}
\newcommand{\seq}[1]{\left( #1 \right)}
\renewcommand{\epsilon}{\varepsilon}
\renewcommand{\iff}{\Leftrightarrow}

\DeclareMathOperator{\arccot}{arccot}
\DeclareMathOperator{\arcsec}{arcsec}
\DeclareMathOperator{\arccsc}{arccsc}
\DeclareMathOperator{\si}{Si}
\DeclareMathOperator{\proj}{proj}
\DeclareMathOperator{\scal}{scal}
\DeclareMathOperator{\cis}{cis}
\DeclareMathOperator{\Arg}{Arg}
%\DeclareMathOperator{\arg}{arg}
\DeclareMathOperator{\Rep}{Re}
\DeclareMathOperator{\Imp}{Im}
\DeclareMathOperator{\sech}{sech}
\DeclareMathOperator{\csch}{csch}
\DeclareMathOperator{\Log}{Log}

\newcommand{\tightoverset}[2]{% for arrow vec
  \mathop{#2}\limits^{\vbox to -.5ex{\kern-0.75ex\hbox{$#1$}\vss}}}
\newcommand{\arrowvec}{\overrightarrow}
\renewcommand{\vec}{\mathbf}
\newcommand{\veci}{{\boldsymbol{\hat{\imath}}}}
\newcommand{\vecj}{{\boldsymbol{\hat{\jmath}}}}
\newcommand{\veck}{{\boldsymbol{\hat{k}}}}
\newcommand{\vecl}{\boldsymbol{\l}}
\newcommand{\utan}{\vec{\hat{t}}}
\newcommand{\unormal}{\vec{\hat{n}}}
\newcommand{\ubinormal}{\vec{\hat{b}}}

\newcommand{\dotp}{\bullet}
\newcommand{\cross}{\boldsymbol\times}
\newcommand{\grad}{\boldsymbol\nabla}
\newcommand{\divergence}{\grad\dotp}
\newcommand{\curl}{\grad\cross}
%% Simple horiz vectors
\renewcommand{\vector}[1]{\left\langle #1\right\rangle}


\pgfplotsset{compat=1.13}

\outcome{Compute complex integrals around closed curves}

\title{5.2 Cauchy's Theorem}

\begin{document}

\begin{abstract}
We compute integrals of complex functions around closed curves.
\end{abstract}

\maketitle

A contour is called closed if its initial and terminal points coincide.  Further it is called simple if those are the only points which coincide. 
\begin{theorem}[Cauchy's Theorem]
If $f$ is analytic along a simple closed contour $C$ and also analytic inside $C$, then
\[
\int_C f(z) \, dz = 0
\]
\end{theorem}

\begin{example}
Let $C$ be a simple closed contour that does not pass through $z_0$ or contain $z_0$ in its interior. Compute $\dis \int_C \frac{1}{z-z_0} \, dz$.\\
The function $\dis f(z) = \frac{1}{z-z_0}$ is analytic everywhere except at $z_0$.  From the description of the contour $C$,
$f(z)$ is analytic along $C$ and in its interior.  Hence, by Cauchy's Theorem,
\[
\int_C \frac{1}{z-z_0} \, dz = 0
\]
\end{example}
\begin{remark}
If the contour $C$ in the previous example passed through the point $z_0$ or contained $z_0$ in its interior, Cauchy's Theorem would not apply
and the integral could be non-zero.
\end{remark}

\begin{problem}(problem 1a)
Let $C = C(z_0, r)$ traversed clockwise.\\
\[
\int_C e^z \, dz = \answer{0}
\]
\end{problem}

\begin{problem}(problem 1b)
Let $C = C(z_0, r)$ traversed clockwise.\\
\[
\int_C \sin(z) \, dz = \answer{0}
\]
\end{problem}

\begin{problem}(problem 1c)
Let $C = C(1+i, 1)$ traversed counter-clockwise.\\
\[
\int_C \Log(z) \, dz = \answer{0}
\]
\end{problem}

\begin{problem}(problem 1d)
Let $C = C(1+i, 1)$ traversed counter-clockwise.\\
\[
\int_C \frac{1}{z} \, dz = \answer{0}
\]
\end{problem}



A \textbf{positively oriented} simple closed contour is one in which the interior of the contour is to the left as the contour is traced out.
In the case of a circle, positive orientation is counter-clockwise. 

\begin{theorem}[Deformation of Contour]
Suppose $C_1$ and $C_2$ are positively oriented, simple closed contours such that $C_1$ lies entirely inside $C_2$. If $f$ is analytic 
along $C_1$ and $C_2$ and also analytic in the region between them, then
\[
\int_{C_1} f(z) \, dz = \int_{C_2} f(z) \, dz
\]
\end{theorem}

\begin{remark} 
The Deformation of Contour theorem is typically used to replace arbitrary positively oriented, simple closed curves with circles traversed counter-clockwise.
\end{remark}


\begin{example}[example 2]
Let $C$ be a positively oriented, simple closed contour containing the origin in its interior. Compute $\dis \int_C \frac{1}{z} \, dz$.\\
Since the origin is interior to $C$, there exists a radius $r$ such that the circle $C(0, r)$ lies in the interior of $C$.
By the Deformation of Contour theorem,
\[
\int_C \frac{1}{z} \, dz = \int_{C(0,r)} \frac{1}{z} \, dz
\]
where $C(0,r)$ is traversed in the positive direction, i.e., counter-clockwise. Using the parametrization
$\gamma(t) = re^{it}, \, 0\leq t \leq 2\pi$ for $C(0, r)$, we have
\[
\int_{C(0,r)} \frac{1}{z} \, dz = \int_0^{2\pi} \frac{\gamma'(t)}{\gamma(t)} \, dt = \int_0^{2\pi} i \, dt = 2\pi i
\]
Hence, 
\[
\int_C \frac{1}{z} \, dz = 2\pi i
\]
as well.
\end{example}

\begin{problem}(problem 2a)
Let $C$ be a positively oriented simple closed curve containing the point $z_0$ in its interior.\\
\[
\int_C \frac{1}{z-z_0} \, dz = \answer{2\pi i}
\]
\end{problem}

\begin{problem}(problem 2b)
Let $C$ be a positively oriented simple closed curve containing the point $z_0$ in its interior.\\
\[
\int_C \frac{1}{(z-z_0)^2} \, dz = \answer{0}
\]
\end{problem}


\begin{problem}(problem 2c)
Let $C$ be a positively oriented simple closed curve containing the point $z_0$ in its interior and let $n > 2$.\\
\[
\int_C \frac{1}{(z-z_0)^n} \, dz = \answer{0}
\]
\end{problem}




\begin{theorem}[Extended Cauchy's Theorem]
Suppose $C_1, C_2, ..., C_n$ are non-intersecting positively oriented, simple closed contours each lying inside of another positively oriented, simple closed contour $C$.  
If $f$ is analytic along $C, C_1, C_2, ..., C_n$ and also analytic in the region between them (i.e. interior to $C$ but exterior to each of $C_1, C_2, ..., C_n$), then
\[
\int_C f(z) \, dz = \sum_{k=1}^n \int_{C_k} f(z) \, dz
\]
\end{theorem}



\begin{theorem}[Cauchy's Integral Formula]
Suppose $f$ is analytic along and inside a positively oriented, simple closed contour $C$ containing the point $z_0$ in its interior.
Then
\[
f^{(n)}(z_0) = \frac{n!}{2\pi i} \int_C \frac{f(z)}{(z-z_0)^{n+1}} \, dz
\]
\end{theorem}

\begin{remark}
The case $n=0$ corresponds to 
\[
f(z_0) = \frac{1}{2\pi i} \int_C \frac{f(z)}{z-z_0} \, dz
\]
\end{remark}


\begin{example}[example 3a]
Let $C = C(0, 1)$ traversed counter-clockwise. Compute the contour integral: $\dis \int_C \frac{\sin z}{z^2} \, dz$\\
We use Cauchy's Integral Formula.  To do so, we first need to identify $z_0, n$ and $f(z)$ and verify that $f(z)$ is analytic in an on $C$.
We have $z_0 = 0, n=1, $ and $f(z) = \sin z$ which is entire. Thus, we can apply the formula and we obtain
\[
\int_C \frac{\sin z}{z^2} \, dz = \frac{2\pi i}{1!} f'(0) = 2\pi i \cos 0  = 2 \pi i
\]
\end{example}

\begin{example}[example 3b]
Let $C = C(2, 1)$ traversed counter-clockwise. Compute the contour integral: $\dis \int_C \frac{\sin z}{z(z-2)} \, dz$\\
We use Cauchy's Integral Formula.  To do so, we first need to identify $z_0, n$ and $f(z)$ and verify that $f(z)$ is analytic in an on $C$.
We have $z_0 = 2, n=0, $ and $f(z) = \frac{\sin z}{z}$ which is analytic in and on $C(2,1)$. Thus, we can apply the formula and we obtain
\[
\int_C \frac{\sin z}{z(z-2)} \, dz = \frac{2\pi i}{0!} f(2) = 2\pi i \frac{\sin 2}{2}  =  \pi i \sin 2
\]
\end{example}


\begin{problem}(problem 3a)
Let $C = C(0, 1)$ traversed counter-clockwise. Compute the contour integral: $\dis \int_C \frac{e^z}{z^2} \, dz$\\
\[
z_0 = \answer{0} \quad n = \answer{1} \quad f(z) = \answer{e^z}
\]
\[
\int_C \frac{e^z}{z^2} \, dz = \answer{2\pi i}
\]
\end{problem}
\begin{problem}(problem 3b)
Let $C = C(0, 1)$ traversed counter-clockwise. Compute the contour integral: $\dis \int_C \frac{e^z}{z^4} \, dz$\\
\[
z_0 = \answer{0} \quad n = \answer{3} \quad f(z) = \answer{\pi i /3}
\]
\[
\int_C \frac{e^z}{z^4}\, dz = \answer{}
\]
\end{problem}
\begin{problem}(problem 3c)
Let $C = C(i, 1)$ traversed clockwise. Compute the contour integral: $\dis \int_C \frac{\cos z}{z-i} \, dz$\\
\[
z_0 = \answer{i} \quad n = \answer{0} \quad f(z) = \answer{\cos z}
\]
\[
\int_C \frac{\cos z}{z-i}\, dz = \answer{-2\pi i \cosh 1}
\]
\end{problem}
\begin{problem}(problem 3d)
Let $C = C(4, 2)$ traversed counter-clockwise. Compute the contour integral: $\dis \int_C \frac{\sin z}{(z-1)(z-3)} \, dz$\\
\[
z_0 = \answer{3} \quad n = \answer{0} \quad f(z) = \answer{\frac{\sin z}{z-1}}
\]
\[
\int_C \frac{\sin z}{(z-1)(z-3)} \, dz = \answer{\pi i \sin 3}
\]
\end{problem}

\begin{example}[example 4]
Let $C = C(0, 2)$ traversed counter-clockwise. Compute the contour integral: $\dis \int_C \frac{e^z}{z^2 +1} \, dz$\\
The integrand has singularities at $\pm i$, so we use the Extended Deformation of Contour Theorem before we use Cauchy's Integral Formula.
By the Extended Deformation of Contour Theorem we can write
\[
\int_C \frac{e^z}{z^2 +1} \, dz = \int_{C_1} \frac{e^z}{z^2 +1} \, dz +  \int_{C_2} \frac{e^z}{z^2 +1} \, dz 
\]
where $C_1 = C(i, 1/2)$ traversed counter-clockwise and $C_2 = C(-i, 1/2)$ traversed counter-clockwise.
Now by Cauchy's Integral Formula with $n = 0$, we have
\[
\int_{C_1} \frac{e^z}{z^2 +1} \, dz = \int_{C_1} \frac{e^z}{(z+i)(z-i)} \, dz = 2\pi i f(i)
\]
where $f(z) = \frac{e^z}{z+i}$.
Hence
\[
\int_{C_1} \frac{e^z}{z^2 +1} \, dz = 2\pi i \cdot \frac{e^i}{2i} = \pi e^i
\]
Similarly,
\[
\int_{C_2} \frac{e^z}{z^2 +1} \, dz = \int_{C_2} \frac{e^z}{(z+i)(z-i)} \, dz = 2\pi i f(-i)
\]
where $f(z) = \frac{e^z}{z-i}$.
Hence
\[
\int_{C_1} \frac{e^z}{z^2 +1} \, dz = 2\pi i \cdot \frac{e^{-i}}{-2i} = -\pi e^{-i}
\]
Combining these we get
\[
\int_C \frac{e^z}{z^2 +1} \, dz = \pi e^i - \pi e^{-i} = 2\pi \sin(1)
\]
\end{example}

\begin{problem}(problem 4)
Let $C = C(0, 2)$ traversed counter-clockwise. Compute the contour integral:\\
\[
\int_C \frac{e^z}{z^2 -1} \, dx  \answer{\pi i (e-1/e)}
\]
\end{problem}

\end{document}






