\documentclass[handout]{ximera}

%% You can put user macros here
%% However, you cannot make new environments



\newcommand{\ffrac}[2]{\frac{\text{\footnotesize $#1$}}{\text{\footnotesize $#2$}}}
\newcommand{\vasymptote}[2][]{
    \draw [densely dashed,#1] ({rel axis cs:0,0} -| {axis cs:#2,0}) -- ({rel axis cs:0,1} -| {axis cs:#2,0});
}


\graphicspath{{./}{firstExample/}}

\usepackage{amsmath}
\usepackage{amssymb}
\usepackage{array}
\usepackage[makeroom]{cancel} %% for strike outs
\usepackage{pgffor} %% required for integral for loops
\usepackage{tikz}
\usepackage{tikz-cd}
\usepackage{tkz-euclide}
\usetikzlibrary{shapes.multipart}


\usetkzobj{all}
\tikzstyle geometryDiagrams=[ultra thick,color=blue!50!black]


\usetikzlibrary{arrows}
\tikzset{>=stealth,commutative diagrams/.cd,
  arrow style=tikz,diagrams={>=stealth}} %% cool arrow head
\tikzset{shorten <>/.style={ shorten >=#1, shorten <=#1 } } %% allows shorter vectors

\usetikzlibrary{backgrounds} %% for boxes around graphs
\usetikzlibrary{shapes,positioning}  %% Clouds and stars
\usetikzlibrary{matrix} %% for matrix
\usepgfplotslibrary{polar} %% for polar plots
\usepgfplotslibrary{fillbetween} %% to shade area between curves in TikZ



%\usepackage[width=4.375in, height=7.0in, top=1.0in, papersize={5.5in,8.5in}]{geometry}
%\usepackage[pdftex]{graphicx}
%\usepackage{tipa}
%\usepackage{txfonts}
%\usepackage{textcomp}
%\usepackage{amsthm}
%\usepackage{xy}
%\usepackage{fancyhdr}
%\usepackage{xcolor}
%\usepackage{mathtools} %% for pretty underbrace % Breaks Ximera
%\usepackage{multicol}



\newcommand{\RR}{\mathbb R}
\newcommand{\R}{\mathbb R}
\newcommand{\C}{\mathbb C}
\newcommand{\N}{\mathbb N}
\newcommand{\Z}{\mathbb Z}
\newcommand{\dis}{\displaystyle}
%\renewcommand{\d}{\,d\!}
\renewcommand{\d}{\mathop{}\!d}
\newcommand{\dd}[2][]{\frac{\d #1}{\d #2}}
\newcommand{\pp}[2][]{\frac{\partial #1}{\partial #2}}
\renewcommand{\l}{\ell}
\newcommand{\ddx}{\frac{d}{\d x}}

\newcommand{\zeroOverZero}{\ensuremath{\boldsymbol{\tfrac{0}{0}}}}
\newcommand{\inftyOverInfty}{\ensuremath{\boldsymbol{\tfrac{\infty}{\infty}}}}
\newcommand{\zeroOverInfty}{\ensuremath{\boldsymbol{\tfrac{0}{\infty}}}}
\newcommand{\zeroTimesInfty}{\ensuremath{\small\boldsymbol{0\cdot \infty}}}
\newcommand{\inftyMinusInfty}{\ensuremath{\small\boldsymbol{\infty - \infty}}}
\newcommand{\oneToInfty}{\ensuremath{\boldsymbol{1^\infty}}}
\newcommand{\zeroToZero}{\ensuremath{\boldsymbol{0^0}}}
\newcommand{\inftyToZero}{\ensuremath{\boldsymbol{\infty^0}}}


\newcommand{\numOverZero}{\ensuremath{\boldsymbol{\tfrac{\#}{0}}}}
\newcommand{\dfn}{\textbf}
%\newcommand{\unit}{\,\mathrm}
\newcommand{\unit}{\mathop{}\!\mathrm}
%\newcommand{\eval}[1]{\bigg[ #1 \bigg]}
\newcommand{\eval}[1]{ #1 \bigg|}
\newcommand{\seq}[1]{\left( #1 \right)}
\renewcommand{\epsilon}{\varepsilon}
\renewcommand{\iff}{\Leftrightarrow}

\DeclareMathOperator{\arccot}{arccot}
\DeclareMathOperator{\arcsec}{arcsec}
\DeclareMathOperator{\arccsc}{arccsc}
\DeclareMathOperator{\si}{Si}
\DeclareMathOperator{\proj}{proj}
\DeclareMathOperator{\scal}{scal}
\DeclareMathOperator{\cis}{cis}
\DeclareMathOperator{\Arg}{Arg}
%\DeclareMathOperator{\arg}{arg}
\DeclareMathOperator{\Rep}{Re}
\DeclareMathOperator{\Imp}{Im}
\DeclareMathOperator{\sech}{sech}
\DeclareMathOperator{\csch}{csch}
\DeclareMathOperator{\Log}{Log}

\newcommand{\tightoverset}[2]{% for arrow vec
  \mathop{#2}\limits^{\vbox to -.5ex{\kern-0.75ex\hbox{$#1$}\vss}}}
\newcommand{\arrowvec}{\overrightarrow}
\renewcommand{\vec}{\mathbf}
\newcommand{\veci}{{\boldsymbol{\hat{\imath}}}}
\newcommand{\vecj}{{\boldsymbol{\hat{\jmath}}}}
\newcommand{\veck}{{\boldsymbol{\hat{k}}}}
\newcommand{\vecl}{\boldsymbol{\l}}
\newcommand{\utan}{\vec{\hat{t}}}
\newcommand{\unormal}{\vec{\hat{n}}}
\newcommand{\ubinormal}{\vec{\hat{b}}}

\newcommand{\dotp}{\bullet}
\newcommand{\cross}{\boldsymbol\times}
\newcommand{\grad}{\boldsymbol\nabla}
\newcommand{\divergence}{\grad\dotp}
\newcommand{\curl}{\grad\cross}
%% Simple horiz vectors
\renewcommand{\vector}[1]{\left\langle #1\right\rangle}


\outcome{Learn the fundamental limit laws}

\title{1.4a Limit Laws}


\begin{document}

\begin{abstract}
Learn the fundamental limit laws.
\end{abstract}

\maketitle

\section{Limit Laws}
 
 In section 1.2, Numerical Limits, we briefly discussed ``plugging in" to compute limits.
 The example we investigated was
 \[
 \lim_{x \to 3+} \left(x^2 + 2 \right) = 3^2 + 2 = 11.
 \]
 The reason plugging in worked in the previous example was a direct consequence of the limit laws presented below.
 
In each of the following, all of the limits are assumed to exist.

\begin{enumerate}

\item[1.]  
The limit of a constant is the constant:
 \[
 \lim_{x \to c} k = k.
 \]
 
\item[2.]
 The next law is self-evident:
 \[
 \lim_{x \to c} x = c.
 \]
 
 \item[3.]
 The limit of a multiple of a function is the multiple of the limit:
 \[
 \lim_{x \to c} kf(x) = k \cdot \lim_{x \to c} f(x). 
 \]
 
 \item[4.]
 The limit of a sum is the sum of the limits:
 \[
 \lim_{x \to c} \left[f(x) + g(x) \right] = \lim_{x \to c} f(x) + \lim_{x \to c} g(x). 
 \]
 
\item[5.]
 The limit of a difference is the difference of the limits:
 \[
 \lim_{x \to c} \left[f(x) - g(x) \right] = \lim_{x \to c} f(x) - \lim_{x \to c} g(x). 
 \]
 
\item[6.]
 The limit of a product is the product of the limits:
 \[
 \lim_{x \to c} f(x) g(x) = \lim_{x \to c} f(x) \cdot \lim_{x \to c} g(x). 
 \]


\item[7.]
 The limit of a quotient is the quotient of the limits:
 \[
 \lim_{x \to c} \frac{f(x)}{g(x)} = \frac{\lim_{x \to c} f(x)}{\lim_{x \to c} g(x)}, 
 \]
 provided that the limit in the denominator is \textbf{not equal to zero}.
 
 \item[8.]
 Limits can be moved inside of radicals:
 \[
 \lim_{x \to c} \sqrt[n]{f(x)} = \sqrt[n]{\lim_{x \to c} f(x)}. 
 \]

\end{enumerate}


Each of the above limit laws is valid if $x \to c^+, x\to c^-, x \to \infty$ or $x \to -\infty$.


\section{Problems}

\begin{example}[example 1] Given that 
\[
\lim_{x \to 3} f(x) = -5 \quad \text{and} \quad \lim_{x \to 3} g(x) = \tfrac12,
\]
find 
\[
\lim_{x \to 3} \left[x^2f(x) - 6g(x)\right].
\]
Using limit laws 2, 3 and 5, we can write
\[
\lim_{x \to 3} \left[x^2f(x) - 6g(x)\right] = \lim_{x \to 3} x^2 \cdot \lim_{x \to 3}f(x) - 6\lim_{x \to 3} g(x) = 9(-5) - 6(\tfrac12) = -48.
\]
\end{example}


\begin{problem}(problem 1a)
Given that 
\[
\lim_{x \to -2} f(x) = 2 \quad \text{and} \quad \lim_{x \to -2} g(x) = 0,
\]
\[
\lim_{x \to -2} \left[3f(x) + xg(x)\right] = \answer{6}.
\]
\end{problem}


\begin{problem}(problem 1b)
Given that 
\[
\lim_{x \to 1^+} f(x) = \tfrac23 \quad \text{and} \quad \lim_{x \to 1+} g(x) = 4,
\]
\[
\lim_{x \to 1^+} \frac{f(x)}{\sqrt{g(x)}} = \answer{1/3}.
\]
\end{problem}


\begin{problem}(problem 1c)
Given that 
\[
\lim_{x \to \infty} f(x) = -9 \quad \text{and} \quad \lim_{x \to \infty} g(x) = 1,
\]
\[
\lim_{x \to \infty} \sqrt[3]{f(x) + g(x)} = \answer{-2}.
\]
\end{problem}




\end{document}
