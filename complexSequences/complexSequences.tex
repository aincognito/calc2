\documentclass[handout]{ximera}

%% You can put user macros here
%% However, you cannot make new environments



\newcommand{\ffrac}[2]{\frac{\text{\footnotesize $#1$}}{\text{\footnotesize $#2$}}}
\newcommand{\vasymptote}[2][]{
    \draw [densely dashed,#1] ({rel axis cs:0,0} -| {axis cs:#2,0}) -- ({rel axis cs:0,1} -| {axis cs:#2,0});
}


\graphicspath{{./}{firstExample/}}

\usepackage{amsmath}
\usepackage{amssymb}
\usepackage{array}
\usepackage[makeroom]{cancel} %% for strike outs
\usepackage{pgffor} %% required for integral for loops
\usepackage{tikz}
\usepackage{tikz-cd}
\usepackage{tkz-euclide}
\usetikzlibrary{shapes.multipart}


\usetkzobj{all}
\tikzstyle geometryDiagrams=[ultra thick,color=blue!50!black]


\usetikzlibrary{arrows}
\tikzset{>=stealth,commutative diagrams/.cd,
  arrow style=tikz,diagrams={>=stealth}} %% cool arrow head
\tikzset{shorten <>/.style={ shorten >=#1, shorten <=#1 } } %% allows shorter vectors

\usetikzlibrary{backgrounds} %% for boxes around graphs
\usetikzlibrary{shapes,positioning}  %% Clouds and stars
\usetikzlibrary{matrix} %% for matrix
\usepgfplotslibrary{polar} %% for polar plots
\usepgfplotslibrary{fillbetween} %% to shade area between curves in TikZ



%\usepackage[width=4.375in, height=7.0in, top=1.0in, papersize={5.5in,8.5in}]{geometry}
%\usepackage[pdftex]{graphicx}
%\usepackage{tipa}
%\usepackage{txfonts}
%\usepackage{textcomp}
%\usepackage{amsthm}
%\usepackage{xy}
%\usepackage{fancyhdr}
%\usepackage{xcolor}
%\usepackage{mathtools} %% for pretty underbrace % Breaks Ximera
%\usepackage{multicol}



\newcommand{\RR}{\mathbb R}
\newcommand{\R}{\mathbb R}
\newcommand{\C}{\mathbb C}
\newcommand{\N}{\mathbb N}
\newcommand{\Z}{\mathbb Z}
\newcommand{\dis}{\displaystyle}
%\renewcommand{\d}{\,d\!}
\renewcommand{\d}{\mathop{}\!d}
\newcommand{\dd}[2][]{\frac{\d #1}{\d #2}}
\newcommand{\pp}[2][]{\frac{\partial #1}{\partial #2}}
\renewcommand{\l}{\ell}
\newcommand{\ddx}{\frac{d}{\d x}}

\newcommand{\zeroOverZero}{\ensuremath{\boldsymbol{\tfrac{0}{0}}}}
\newcommand{\inftyOverInfty}{\ensuremath{\boldsymbol{\tfrac{\infty}{\infty}}}}
\newcommand{\zeroOverInfty}{\ensuremath{\boldsymbol{\tfrac{0}{\infty}}}}
\newcommand{\zeroTimesInfty}{\ensuremath{\small\boldsymbol{0\cdot \infty}}}
\newcommand{\inftyMinusInfty}{\ensuremath{\small\boldsymbol{\infty - \infty}}}
\newcommand{\oneToInfty}{\ensuremath{\boldsymbol{1^\infty}}}
\newcommand{\zeroToZero}{\ensuremath{\boldsymbol{0^0}}}
\newcommand{\inftyToZero}{\ensuremath{\boldsymbol{\infty^0}}}


\newcommand{\numOverZero}{\ensuremath{\boldsymbol{\tfrac{\#}{0}}}}
\newcommand{\dfn}{\textbf}
%\newcommand{\unit}{\,\mathrm}
\newcommand{\unit}{\mathop{}\!\mathrm}
%\newcommand{\eval}[1]{\bigg[ #1 \bigg]}
\newcommand{\eval}[1]{ #1 \bigg|}
\newcommand{\seq}[1]{\left( #1 \right)}
\renewcommand{\epsilon}{\varepsilon}
\renewcommand{\iff}{\Leftrightarrow}

\DeclareMathOperator{\arccot}{arccot}
\DeclareMathOperator{\arcsec}{arcsec}
\DeclareMathOperator{\arccsc}{arccsc}
\DeclareMathOperator{\si}{Si}
\DeclareMathOperator{\proj}{proj}
\DeclareMathOperator{\scal}{scal}
\DeclareMathOperator{\cis}{cis}
\DeclareMathOperator{\Arg}{Arg}
%\DeclareMathOperator{\arg}{arg}
\DeclareMathOperator{\Rep}{Re}
\DeclareMathOperator{\Imp}{Im}
\DeclareMathOperator{\sech}{sech}
\DeclareMathOperator{\csch}{csch}
\DeclareMathOperator{\Log}{Log}

\newcommand{\tightoverset}[2]{% for arrow vec
  \mathop{#2}\limits^{\vbox to -.5ex{\kern-0.75ex\hbox{$#1$}\vss}}}
\newcommand{\arrowvec}{\overrightarrow}
\renewcommand{\vec}{\mathbf}
\newcommand{\veci}{{\boldsymbol{\hat{\imath}}}}
\newcommand{\vecj}{{\boldsymbol{\hat{\jmath}}}}
\newcommand{\veck}{{\boldsymbol{\hat{k}}}}
\newcommand{\vecl}{\boldsymbol{\l}}
\newcommand{\utan}{\vec{\hat{t}}}
\newcommand{\unormal}{\vec{\hat{n}}}
\newcommand{\ubinormal}{\vec{\hat{b}}}

\newcommand{\dotp}{\bullet}
\newcommand{\cross}{\boldsymbol\times}
\newcommand{\grad}{\boldsymbol\nabla}
\newcommand{\divergence}{\grad\dotp}
\newcommand{\curl}{\grad\cross}
%% Simple horiz vectors
\renewcommand{\vector}[1]{\left\langle #1\right\rangle}


\pgfplotsset{compat=1.13}

\outcome{Find limits of complex sequences}

\title{4.1 Complex Sequences}

\begin{document}

\begin{abstract}
We find limits of complex sequences.
\end{abstract}

\maketitle


\begin{definition} 
A complex sequence $\{c_n\}$ is a function from the natural numbers to the complex numbers, $\C$.
\end{definition}

\begin{definition}
A complex sequence $\{c_n\}$ is said to {\bf converge} to a complex number $c$ if given $\epsilon >0$, there exists $N \in \N$
such that 
\[
|c_n - c| < \epsilon \quad \text{whenever} \quad n>N
\]
If no such complex number $c$ exists, then we say the sequence $\{c_n\}$ {\bf diverges}.
\end{definition}

 

\begin{example}[example 1]
Show that $\displaystyle \lim_{z \to 1+i} z^2 = 2i$.\\[6pt]
Suppose $|z-(1+i)| < \delta$ for some $0< \delta < 1$.
Note that
\[
z ^2 - 2i=\left[z-(1+i)\right] \cdot \left[z + (1+i)\right]
\]
By the Triangle Inequality
\[
|z+(1+i)| \leq |z| + |1+i| = |z| + \sqrt 2
\]
Now, since $|z-(1+i)| < \delta$ the Triangle Inequality also gives
\[
|z| = |z-(1+i) +(1+i)| \leq |z-(1+i)| + |1+i| < \delta + \sqrt 2 
\]
Thus,
\[
|z+(1+i)| \leq \delta + 2\sqrt 2 <  4 \; \mbox{since} \; \delta < 1
\]
Returning to $z^2 - 2i$, we have
\[
|z ^2 - 2i| = |z-(1+i)| \cdot | z+(1+i)| < \delta \cdot 4
\]

Now, let $\epsilon > 0$ and choose $\delta> 0$ such that $\delta < \min\left\{\frac{\epsilon}{4}, 1\right\}$. 
From the computations above, if $|z-(1+i)| < \delta$ then
\[
|z^2 - 2i| < 4\delta < \epsilon
\]
Hence,
\[
\lim_{z \to 1+i} z^2 = 2i
\]


\end{example}

\begin{problem}(problem 1) Use the definition of limit to show each of the following:
\begin{align*}
i) & \displaystyle \lim_{z \to i} z^2 = -1\\
ii) & \displaystyle \lim_{z \to z_0} (az+b) = az_0 + b \;\;(a \neq 0)\\
iii) & \displaystyle \lim_{z \to z_0} \overline{z} = \overline{z_0}
\end{align*}
\end{problem}

Complex limits have the same familiar properties as real limits.

\begin{theorem}
Suppose $f$ and $g$ are defined on the punctured disk $D_\circ(z_0, r)$ for some $r>0$ and suppose
\[
\lim_{z \to z_0} f(z) = w_f \quad \mbox{and} \quad \lim_{z \to z_0} g(z) = w_g
\]
then the following limits hold:
\begin{align*}
a) \; & \lim_{z \to z_0} \left[f(z)+ g(z)\right] = w_f +w_g\\
b) \; & \lim_{z \to z_0} \left[\alpha f(z)\right] = \alpha w_f, \;\; \mbox{for any} \;\;\alpha \in \C\\
c) \; & \lim_{z \to z_0} \left[f(z)\cdot g(z)\right] = w_f \cdot w_g\\
d) \; & \lim_{z \to z_0} \frac{f(z)}{g(z)} = \frac{w_f}{w_g}, \;\; \mbox{if}\;\; w_g \neq 0
\end{align*}
\end{theorem}

\begin{proof}
Part a)  Since $\lim_{z \to z_0} f(z) = w_f$, for a given $\epsilon >0$, there exists $\delta_f >0$ such that
\[
0<|z-z_0| < \delta_f \Rightarrow |f(z) - w_f| < \frac{\epsilon}{2}
\]
Similarly, there exists $\delta_g >0$ such that
\[
0<|z-z_0| < \delta_f \Rightarrow |g(z) - w_g| < \frac{\epsilon}{2}
\]
Let $\delta = \min\{\delta_f, \delta_g \}$ then for $0< |z-z_0| < \delta$, the Triangle Inequality gives
\[
|\left(f(z) + g(z)\right) - (w_f + w_g)| \leq |f(z) - w_f|+|g(z) - w_g| < \frac{\epsilon}{2}+\frac{\epsilon}{2} = \epsilon
\]
as required.\\[6pt]

Part d) It is sufficient to prove that 
\[
\lim_{z \to z_0} \frac{1}{g(z)} = \frac{1}{w_g}, \;\; \mbox{if}\;\; w_g \neq 0
\]
for then the result follows by observing that $\displaystyle \frac{f}{g} = f \cdot \frac{1}{g}$ and applying part c.\\
Let $\epsilon > 0$ and Let $\epsilon' = \min\left\{\frac{|w_g|}{2}, \frac{\epsilon |w_g|^2}{2}\right\}$.

Then there exists $\delta >0$ such that
\[
 0< |z-z_0| < \delta \Rightarrow |g(z) - w_g| < \epsilon'
 \]
By the Reverse Triangle Inequality and the definition of $\epsilon'$, if $0 < |z-z_0| < \delta $ then
\[
|g(z)| > |w_g| - \epsilon' \geq |w_g| - \frac{|w_g|}{2} = \frac{|w_g|}{2}
\]
and so 
\[
\frac{1}{|g(z)|} < \frac{2}{|w_g|}
\]
Now, for $z$ such that $0 < |z-z_0| < \delta $ we have 
\begin{align*}
\left| \frac{1}{g(z)} - \frac{1}{w_g} \right| & = \frac{|w_g - g(z)|}{|g(z)|\cdot |w_g|}\\
& < \frac{2\epsilon'}{|w_g|^2}\\
& \leq \frac{2}{|w_g|^2} \cdot \frac{\epsilon |w_g|^2}{2}\\
& = \epsilon
\end{align*}
as required.

\end{proof}


\begin{problem} 
Use the Reverse Triangle Inequality $\left(|z_2 - z_1| \geq |z_2| -|z_1|\right)$ to prove 
that if $|g(z) - w_g| < \epsilon'$, then $|g(z)| > |w_g| - \epsilon'$.
\end{problem}

\begin{problem} 
Prove parts b and c of the theorem above.
\end{problem}

\begin{problem}
Use parts a and b of the theorem above to show that the limit of a difference is the difference
of the limits
\end{problem}


\begin{problem}
Use parts a, b and c of the theorem above to prove that for a polynomial $p(z)$
\[
\lim_{z\to z_0} p(z) = p(z_0)
\]
for any $z_0 \in \C$.
\end{problem}

The limit of a complex function can be determined from the limits of its real and imaginary parts.
\begin{proposition}
Let $z_0 = x_0 + iy_0 \in \C$ and suppose $f(z)=f(x+iy) = u(x,y) + iv(x,y)$ is defined 
on $D_\circ(z_0,r)$ for some $r>0$. Let $w_0 = u_0 + iv_0$, then
\[
\lim_{z \to z_0} f(z) = w_0 
\]
if and only if
\[
\lim_{(x,y) \to (x_0, y_0)} u(x,y) =  u_0 
\]
and
\[
\lim_{(x,y) \to (x_0, y_0)} v(x,y) =  v_0 
\]
%where $z_0 = x_0 + iy_0$.
\end{proposition}

To prove the proposition, we need to recall that 
\[
|\Rep z|, |\Imp z| \leq |z| \leq |\Rep z| + |\Imp z|
\]
for all $z \in \C$.
Also, we will use the notation 
\[
|(x,y) - (x_0,y_0)|
\]
 for the distance between the points
$(x,y)$ and $(x_0,y_0)$ in $\R^2$.  Thus
\[
|(x,y) - (x_0,y_0)| = |(x+iy) - (x_0 + iy_0)|
\]

%For ease of readability we will also abbreviate $u(x,y)$ and $v(x,y)$ with $u$  and $v$ respectively. 

\begin{proof} %We now prove the proposition.\\[6pt]
Suppose $\lim_{z \to z_0} f(z) = w_0 = u_0 + iv_0$.
Then for any $\epsilon > 0$ there exists $\delta > 0$ such that
\[
0< |z-z_0| < \delta \Rightarrow |f(z) - w_0| < \epsilon
\]

%In particular, this statement holds for all $|z - z_0|< \epsilon$ with the additional constraint that $y = y_0$

This statement can be rewritten as
\[
|(x+iy)-(x_0+i y_0)| < \delta \Rightarrow |u(x,y) + iv(x,y) - (u_0 + iv_0)|< \epsilon
\]
Since $|u(x,y) - u_0| \leq |u(x,y) + iv(x,y) - (u_0 + iv_0)|$
we have
\[
|(x,y)-(x_0, y_0)| < \delta \Rightarrow |u(x,y)  - u_0 |< \epsilon
\]
Thus
\[
\lim_{(x,y) \to (x_0, y_0)} u(x,y) =  u_0 
\]
 

%since $|v(x,y) - v_0|<  |f(z) -w_0|$
%we have
%\[
%|(x,y)-(x_0, y_0)| < \delta \Rightarrow |v(x,y)  - v_0 |< \epsilon
%\]
%Thus

Similarly,
\[
\lim_{(x,y) \to (x_0, y_0)} v(x,y) =  v_0 
\]
Next, suppose
\[
\lim_{(x,y) \to (x_0, y_0)} u(x,y) =  u_0  \quad \mbox{and} \quad \lim_{(x,y) \to (x_0, y_0)} v(x,y) =  v_0 
\]
Then for $\epsilon > 0$ there exists $\delta_u > 0$ and $\delta_v > 0$ such that
\[
0<|(x,y)-(x_0, y_0)| < \delta_u \Rightarrow |u(x,y)  - u_0 |< \frac{\epsilon}{2}
\]
and
\[
0<|(x,y)-(x_0, y_0)| < \delta_v \Rightarrow |v(x,y)  - v_0 |< \frac{\epsilon}{2}
\]
Let $\delta = \min\{\delta_u, \delta_v\}$ and suppose $|z-z_0| = |(x+iy) - (x_0 + iy_0)| <  \delta$. Then 

%by the Triangle Inequality

\[
|f(z) - w_0| \leq  |u(x,y)  - u_0 | + |v(x,y)  - v_0 | < \frac{\epsilon}{2}+ \frac{\epsilon}{2} = \epsilon
\]
which shows that
\[
\lim_{z \to z_0} f(z) = w_0
\]
\end{proof}

Let's examine some limits of a real functions of two variables.

\begin{example}
Find
\[
\lim_{(x,y) \to (0,0)} \frac{x^3}{x^2 + y^2}
\]
if it exists.\\
We will use the Squeeze Theorem.  
Note that 
\[
0 \leq \frac{x^2}{x^2 + y^2} \leq 1
\]
for any $(x,y) \neq (0,0)$. Furthermore
\[
-|x| \leq x \leq |x|
\]
for any $x \in \R$. Hence,
\[
-|x| \leq - \frac{|x|x^2}{x^2 + y^2}\leq \frac{x^3}{x^2 + y^2} \leq  \frac{|x|x^2}{x^2 + y^2} \leq |x|
\]
Finally, since
\[
\lim_{(x,y) \to (0,0)} -|x| = 0 \;\; \mbox{and} \;\; \lim_{(x,y) \to (0,0)} |x| = 0
\]
we can use the Squeeze Theorem to can conclude that
\[
\lim_{(x,y) \to (0,0)} \frac{x^3}{x^2 + y^2} = 0
\]
as well.
\end{example}

\begin{problem}
Find
\[
\lim_{(x,y) \to (0,0)} \frac{x^2y}{x^2 + y^2} =\answer{0}
\]
if it exists.

\end{problem}


\begin{problem}
Find
\[
\lim_{(x,y) \to (0,0)} \frac{2x^2y^3}{x^4 + y^4} =\answer{0}
\]
if it exists.
\begin{hint}
$a^2 + b^2 \geq 2ab$ for any $a, b \in \R$
\end{hint}

\end{problem}



In the next example we show that a limit does not exist because different paths lead to different limits. This is akin to 
a two-sided limit not existing in the single variable case when the one-sided are different.

\begin{example}
Find
\[
\lim_{(x,y) \to (0,0)} \frac{xy}{x^2 + y^2}
\]
if it exists.\\
We will let $(x,y)$ approach $(0,0)$ along different lines. If we let $y = mx, m \in \R$, then the limit becomes
\begin{align*}
\lim_{(x,y) \to (0,0)} \frac{xy}{x^2 + y^2} &= \lim_{(x,y) \to (0,0)} \frac{mx^2}{x^2 + (mx)^2} \\
                                            &= \lim_{(x,y) \to (0,0)} \frac{mx^2}{(1+m^2)x^2}\\
                                            &  = \frac{m}{1+m^2}
\end{align*}
Hence the limit depends on the path and so the limit does not exist.
\end{example}

\begin{problem}
Find the limit if it exists.
\begin{align*}
i) \; & \lim_{(x,y) \to (0,0)} \frac{y^2}{x^2 + y^2} \\
ii) \; & \lim_{(x,y) \to (0,0)} \frac{x}{x^2 + y^2}\\
iii) \; & \lim_{(x,y) \to (0,0)} \frac{x^2y}{x^4 + y^2} \;\; \mbox{Hint: use parabolas instead of lines}
\end{align*}
\end{problem}

\section{Continuity}
A significant application of limits is to continuity. Recall that we define a function of a single real variable to be continuous at $x = x_0$
if 
\[
\lim_{x \to x_0} f(x) = f(x_0)
\]
We define continuity for a complex function analagously.

\begin{definition}
Let $f$ be a complex function defined on the disk $D(z_0, r)$ for some $r>0$. We say that $f$ is continuous ar $z_0$ if
\[
\lim_{z \to z_0} f(z) = f(z_0)
\]
\end{definition}

\begin{example}
If $p(z)$ is a polynomial (with complex coefficients), then $p$ is continuous in $\C$ (see the related problem above).
\end{example}

\begin{proposition}
A complex function $f(z) = u(x,y) + iv(x,y)$ is continuous at $z_0$ if and only if
$u(x,y)$ and $v(x,y)$ are continuous at $(x_0,y_0)$.
\end{proposition}

The proof of this proposition is a direct application of the earlier proposition relating limits 
of a complex function to the limits of its real and imaginary parts.\\
Recalling that the real exponential and trigonometric functions are continuous on their domains makes it easy to see that their complex analogues are also continuous.
\begin{example}
The complex exponential function $e^z$ is continuous on $\C$.\\
Since
\[
e^z = e^x \cis y = e^x \cos y + ie^x \sin y
\]
and the real functions $e^x \cos y$ and $e^x \sin y$ are continuous on $\R^2$,
the complex exponential function, $e^z$, is continuous on $\C$.
\end{example}

\begin{problem}
Show that the functions $\sin z$ and $\cos z$ are continuous on $\C$.
\end{problem}

\begin{example}
The principal branch of the complex logarithm, $\Log z$ is not continuous on the non-positive real axis.\\
To see this, first note that $\Log 0$ is undefined, so $\Log z$ is not continuous at $0$. Now let $c<0$ be any negative real number.
Then $\Log c = \ln |c| + \pi i$, but if we compute the limit as $z \to c$ along a path in the third quadrant, we get
\[
\lim_{z \to c \atop y < 0} \Log z = \ln|c| - \pi i \neq \Log c
\]
\end{example}

\begin{problem}
Show that the principal argument function $\Arg z$ is not continuous on the negative real axis.
\end{problem}

\end{document}




























\section{Linear Functions}
A linear function has the form $f(z) = az+b$ where $a, b \in \C$ and $a \neq 0$ (if $a=0$, then $f$ is a constant function).
The effect of a linear function on a set in the complex plane is to scale, rotate and translate that set.  The constant term $b$ acts as a translation vector, 
and the modulus and argument of $a$ are responsible for producing scaling and rotation, respectively.

\begin{example}[example 1]
Find the image of the unit square under the mapping $f(z) = (1+i)z + i$.\\
In this case, $a = 1+i$ which has modulus $\sqrt 2$ and argument $\pi/4$, and $b = i$.
The effect on the square of multiplying by $1+i$ is to scale the square by a factor of $\sqrt 2$, rotate counter-clockwise by $\pi/4$ and then translate it
by $i$. The scaling and the rotation can be done in either order, but the translation must come last.


\begin{image}
\begin{tikzpicture}
\draw[<->, thick] (-3,0)--(3,0) node[right]{x};

\draw[<->, thick] (0, -3) node[right]{$z$-plane} --(0,3) node[right]{iy};

\draw[thick,blue, fill= blue!20] (0,0) -- (1.2,0) node[below, black]{1} -- (1.2,1.2) -- (0, 1.2) node[left, black]{i}  -- (0,0);

%\draw[mark=*,mark size=1pt,mark options={color=blue}] plot coordinates {(-0.5,.4)} node[right, blue]{$z_0 = x_0 + iy_0$};
%\draw[mark=*,mark size=1pt,mark options={color=blue}] plot coordinates {(-0.5,1.4)} node[right, blue]{$z_0+ 2\pi i$};


%\draw[mark=*,mark size=1pt,mark options={color=blue}] plot coordinates {(-0.5,-.6)} node[right, blue]{$z_0- 2\pi i$};
%\draw[mark=*,mark size=1pt,mark options={color=blue}] plot coordinates {(-0.5,-1.6)} node[right, blue]{$z_0- 4\pi i$};


\draw[<->, thick] (6,0)--(12,0)node[right]{u};



\draw[thick,blue, fill= blue!20] (9,.5) node[left, black]{i}-- (10,1.5) node[right, black]{$1+2i$} -- (9,2.5) node[left, black]{3i} -- 
(8, 1.5) node[left, black]{$-1+2i$}  -- (9,.5);
\draw[<->, thick] (9, -3) node[right]{$w$-plane}--(9,3) node[right]{iv};
%\draw[mark=*,mark size=1pt,mark options={color=blue}] plot coordinates {(8.5,1.3)} node[right, blue]{$w_0$};
%\draw[dashed] (7,0) -- (8.5, 1.3) node[above, midway]{$e^{x_0}$};
%\draw (7.4,0) arc (0:42:.4) node[midway, right]{$y_0$} ;
\draw[->, thick, blue] (3.5,1)--(5.5, 1) node[above,midway, blue]{$w = f(z)$};
\node at (4.5, -4){\large The effect of $f(z) = (1+i)z + i$ on the unit square};
\end{tikzpicture}
\end{image}


\end{example}


\begin{problem}(problem 1)
Find the image of the unit square under the given mapping:
\begin{align*}
i) & f(z) = iz\\
ii) & f(z) = 2z\\
iii) & f(z) = z+1\\
iv) & f(z) = (1-i)z +i
\end{align*}
\end{problem}



\section{Squaring}


\begin{example}[example 2]
Find the image of the sector of an annulus given by:
\[
S = \left\{r\cis \theta : \frac12 < r < 2, \;\frac{\pi}{6} < \theta < \frac{\pi}{2}\right\}
\]
under the mapping $f(z) = z^2$.\\
The effect of squaring is to square moduli and double arguments. Hence the image, $f(S)$ is the sector
\[
f(S) = \left\{r\cis \theta : \frac14 < r < 4, \;\frac{\pi}{3} < \theta < \pi \right\}
\]


\begin{image}
\begin{tikzpicture}
\draw[dashed, blue, fill=blue!20] (90:.75) node[left]{$\frac{i}{2}$} arc (90:30:.75) coordinate (alpha) -- (30:2.25) arc (30:90:2.25) node[left]{$2i$} -- cycle;
\draw[->, thin, black] (1.5,0) arc (0:30:1.5) node[right, midway]{$\pi/6$};
\draw[<->, thick] (-3,0)--(3,0) node[right]{$x$};

\draw[<->, thick] (0, -3) node[right]{$z$-plane} --(0,3) node[right]{$iy$};

\draw[dashed, blue, fill=blue!20, shift ={(9,0)}] (180:.5) node[below]{$-\frac{1}{4}$} arc (180:60:.5) 
coordinate (alpha) -- (60:2.5) arc (60:180:2.5) node[below]{$-4$} -- cycle;


\draw[->, thin, black, shift={(9,0)}] (1.5,0) arc (0:60:1.5) node[right, midway]{$\pi/3$};

\draw[<->, thick] (6,0)--(12,0)node[right]{$u$};


\draw[<->, thick] (9, -3) node[right]{$w$-plane}--(9,3) node[right]{$iv$};

\draw[->, thick, blue] (3.5,1)--(5.5, 1) node[above,midway, blue]{$w = z^2$};
\node at (4.5, -4){\large The effect of $f(z) = z^2$ on a sector of an annulus};
\end{tikzpicture}
\end{image}

\end{example}


\begin{problem}(problem 2)
Find the image of the given set under the mapping $f(z) = z^2$:
\begin{align*}
i) & S = \left\{r\cis \theta : 1 < r < 2, \;0 < \theta < \frac{\pi}{4}\right\} \\
ii) & S = \left\{r\cis \theta : 2 < r < 3, \;\frac{\pi}{2} < \theta < \frac{3\pi}{4}\right\}\\
iii) & \mbox{Upper Half Plane}\; = \left\{ x+iy: y> 0\right\}\\
iv) & \mbox{Quadrant IV}\; = \left\{ x+iy: x> 0, y< 0\right\}
\end{align*}
\end{problem}



\begin{example}[example 3]
Find the image of vertical line $\Rep z = 1$ under the mapping $f(z) = z^2$.\\
Instead of a polar analysis, we will use rectangular coordinates with $x = 1$. We have
\[
w = u+iv = (1+iy)^2 = (1-y^2) + 2iy
\]
Thus $u = 1-y^2$ and $v = 2y$, where $y \in R$. Eliminating $y$ gives
\[
u= 1-\frac{v^2}{4}
\]
which is a parabola in the $uv$-plane which opens to the left and has vertex at $(1,0)$.


\begin{image}
\begin{tikzpicture}

\draw[<->, thick] (-3,0)--(3,0) node[right]{$x$};

\draw[<->, thick] (0, -3) node[right]{$z$-plane} --(0,3) node[right]{$iy$};

\draw[blue, thick, <->] (1, -2.5) -- (1, 2.5) ;
\node[blue] at (1.6, 1){$x = 1$};

\draw[<->,thick, blue, rotate=-90, shift={(0,9)}] (-3,-1.25) parabola bend (0,1) (3,-1.25) ;
\node[blue] at (11,1){$u = 1-\frac{v^2}{4}$};

\draw[<->, thick] (6,0)--(12,0)node[right]{$u$};


\draw[<->, thick] (9, -3) node[right]{$w$-plane}--(9,3) node[right]{$iv$};

\draw[->, thick, blue] (3.5,1)--(5.5, 1) node[above,midway, blue]{$w = z^2$};
\node at (4.5, -4){\large The effect of $f(z) = z^2$ on a vertical line};
\end{tikzpicture}
\end{image}

\end{example}


\begin{problem}(problem 3)
Find the image of the given line under the mapping $f(z) = z^2$:
\begin{align*}
i) & \Rep z = 2 \\
ii) & \Rep z = -1\\
iii) & \Imp z = 1\\
iv) & \Imp z = -2
\end{align*}
\end{problem}


\section{Conjugates and Reciprocals}
The function $f(z) = \overline{z}$ maps a set in the complex plane to its mirror image in the real axis.

\begin{example}[example 4]
Find the image of the closed disk $D(z_0, r)$ under the mapping $f(z) = 2\overline{z}$.\\
The image of the center of the disk is $f(z_0) = 2\overline{z_0}$ and so the image of the disk
$D(z_0, r)$ is the disk of twice the radius, $D(2\overline{z_0}, 2r)$.

\begin{image}
\begin{tikzpicture}

\draw[<->, thick] (-3,0)--(3,0) node[right]{$x$};

\draw[<->, thick] (0, -3) node[right]{$z$-plane} --(0,3) node[right]{$iy$};

\draw[dashed, blue, fill = blue!20] (1.25, 1.25) circle (0.75) ;
\draw[mark=*,mark size=1pt,mark options={color=blue}] plot coordinates {(1.25, 1.25)} node[left, blue]{$z_0$};
\draw[blue] (1.25, 1.25)--(1.78, 1.78) node[right, midway]{$r$};


\draw[dashed, blue, fill= blue!20] (11.5, -2.5)circle (1.5) ;
\draw[mark=*,mark size=1pt,mark options={color=blue}] plot coordinates {(11.25, -2.5)} node[left, blue]{$2\overline{z_0}$};
\draw[blue] (11.25, -2.5)--(12.36, -1.33) node[right, midway]{$2r$};

\draw[->, thick, blue] (3.5,1)--(5.5, 1) node[above,midway, blue]{$w = 2\overline{z}$};
\draw[<->, thick] (6,0)--(12,0)node[right]{$u$};


\draw[<->, thick] (9, -3) node[left]{$w$-plane}--(9,3) node[right]{$iv$};


\node at (4.5, -4){\large The effect of $f(z) = 2\overline{z}$ on a disk};
\end{tikzpicture}
\end{image}

\end{example}


\begin{problem}(problem 4)
Find the image of the disk $D(1+i, 1)$ under the given mapping:
\begin{align*}
i) & f(z) = \overline{z} \\
ii) & f(z) = -2\overline{z} \\
iii) & f(z) = 3i\overline{z} \\
iv) & f(z) = i+\overline{z} 
\end{align*}
\end{problem}


To understand the mapping properties of $f(z) = \frac{1}{z}$ it is useful to rewrite it in the following way:
\[
\frac{1}{z} = \frac{1}{z} \cdot \frac{\overline{z}}{\overline{z}}= \frac{\overline{z}}{|z|^2}
\]
Thus the mapping $f(z) = \frac{1}{z}$ has both a reflecting property like the conjugate mapping and a scaling property
due to the division of the positive real number $|z|^2$. As a result of the scaling, the image of a point inside the unit circle 
will map to a point outside of the circle and vice versa. This is shown in the figure below.

\begin{image}
\begin{tikzpicture}


\begin{scope}
\draw[clip, draw = none] (1,0) arc (0:180:2) --cycle;
\draw[outer color = blue!50, inner color = white, shading = radial] (-1,0) circle (2);
%\draw[ color = blue!50,shading = radial] (-2,0) circle (2);
\end{scope}

 

 \begin{scope}
\draw[clip, draw = none] (1,0) arc (0:-180:2) --cycle;
\draw[outer color = red!50, inner color = white] (-1,0) circle (2);
\end{scope}

\draw[<->, thick] (-4,0)--(2,0) node[right]{$x$};
\draw[<->, thick] (-1, -3) node[right]{$z$-plane} --(-1,3) node[right]{$iy$};

\begin{scope}
\draw[clip, draw = none] (6.2,0) rectangle (11.8,2.8);
\filldraw[draw = none, inner color = red, outer color = red!10, shading = radial] (6.2,-2.8) rectangle (11.8,2.8) ;
\end{scope}

\begin{scope}
\draw[clip, draw = none] (6.2,-2.8) rectangle (11.8,0);
\filldraw[draw = none, inner color = blue, outer color = blue!10, shading = radial] (6.2,-2.8) rectangle (11.8,2.8) ;
\end{scope}

\draw[dashed, fill = white] (9,0)circle (1) ;
\draw[white, thick] (6.2, -2.8) rectangle (11.8, 2.8);
\draw[thick, color = blue!20] (6.5,0)--(8,0);
\draw[thin, color = blue!20] (10,0)--(11.3,0);

\draw[<->, thick] (6,0)--(12,0)node[right]{$u$};


\draw[<->, thick] (9, -3) node[right]{$w$-plane}--(9,3) node[right]{$iv$};

\draw[->, thick, blue] (3,1)--(5, 1) node[above,midway, blue]{$w = \frac{1}{z}$};

\draw[dashed, fill = white] (-1,0)circle (.05) ;
\draw[dashed] (-1,0)circle (2) ;

\node at (4, -4){\large The effect of $f(z) = \frac{1}{z}$ on the open punctured unit disk};

\draw[mark=*,mark size=1pt,] plot coordinates {(-.3, 1.25)} ;
\node at (-.45,1) {$z_0$};

\draw[mark=*,mark size=1pt] plot coordinates {(9.8, -1.4)} ;
\node at (10.2,-1.4) {$\frac{1}{z_0}$};

\end{tikzpicture}
\end{image}




The color coding in the figure is intended to convey reflection in the real axis. A blue point in the upper half plane
inside the disk maps to a blue point in the lower half plane outside the disk. Similarly, a red point in the lower half plane inside the disk maps to a
red point outside the disk.

\begin{example}[example 5]
Find the image of the vertical ray $\Rep z = \frac12, \Imp z > 0$ under the mapping $f(z) = \frac{1}{z}$.\\
We write $1/z$ in rectangular form:
\[
\frac{1}{z} = \frac{\overline{z}}{|z|^2} = \frac{x-iy}{x^2 + y^2}
\]
Substituting $x = 1/2$, we have
\[
u = \frac{2}{1+4y^2} \quad \mbox{and} \quad v = -\frac{4y}{1+4y^2}
\]
Note $v < 0$ since $y > 0$. Furthermore, observe that $u$ and $v$ are related by $v = -2uy$. 
To eliminate $y$ from this last equation, we rewrite the expression for $u$, which gives
\[
y = \sqrt{\frac{2-u}{4u}} 
\]
Substituting into $v = -2uy$ gives
\[
v = -2u\sqrt{\frac{2-u}{4u}} = -\sqrt{u(2-u)} = -\sqrt{2u-u^2}
\]
To see what type of curve this is, square both sides :
\[
v^2 = 2u - u^2
\]
and so $u^2 - 2u + v^2 = 0$.  Completing the square we get
\[
(u-1)^2 + v^2 = 1
\]
which is the circle $C(1,1)$. Recalling that $v <0$
we conclude that the image of the ray $x = 1/2, y >0$ is a semi circle in the lower half plane with 
center at $1$ and radius $1$.


\begin{image}
\begin{tikzpicture}

%z-plane
\draw[<->, thick] (-2.5,0)--(2.5,0) node[right]{$x$}; %real axis
\draw[<->, thick] (0, -2.5) node[right]{$z$-plane} --(0,2.5) node[left]{$iy$}; %imaginary axis

\draw[blue, thick, ->] (0.5, 0) -- (0.5, 2.5) ; %domain
\draw[thick,blue, fill = white] (0.5,0) circle (0.07) ;
\node[blue, thick] at (1.5, 1.5){$x = 1, y > 0$}; %domain label

\draw[->, thick, blue] (3,-1.5)--(5, -1.5) node[above,midway, blue]{$w = \frac{1}{z}$};
%w-plane
\draw[<->, thick] (5,0)--(10,0)node[right]{$u$}; %real axis
\draw[<->, thick] (7.5, -2.5) node[right]{$w$-plane}--(7.5,2.5) node[left]{$iv$}; %imaginary axis

\draw[blue, thick, ->] (9.5,0) arc (360:180:1); %image
\draw[thick,blue, fill = white] (9.5,0) circle (0.07) ;
\node[blue, thick] at (10, -1.5){$(u-1)^2 + v^2 = 1, v< 0$}; %image label

\node at (3.75, -3.5){\large The effect of $f(z) = \frac{1}{z}$ on a vertical ray}; %caption
\end{tikzpicture}
\end{image}

\end{example}


\begin{problem}(problem 5)
Find the image of the following sets under the mapping $f(z) = \frac{1}{z}$:
\begin{align*}
i)\; & \Imp z = \frac13, \Rep z < 0 \\
ii)\; & \Rep z = 2, \Imp z >\leq 0 \\
iii)\; & \Rep z + \Imp z = 1 \\
iv)\; & \mbox{The ray:} \; \theta = \theta_0
\end{align*}
\end{problem}



\subsection{Stereographic Projection}


The mapping $f(z) = \frac{1}{z}$ in some sense inverts the unit disk, with the singularity at the origin corresponding to a ring of points far from 0.
In the complex plane this entire ring is considered to be ``$\infty$". Stereographic projection enables us to identify the complex plane with a punctured 
sphere $S\setminus N$ where $N$ is the north pole of the sphere (sorry Santa) and further, to identitify ``$\infty$" with the north pole, $N$.
%(North Pole, like batteries, not included).

At the heart of stereographic projection lies a 1-1, onto map $\varphi: \C\to S\setminus N $ where $S$ is the sphere with 
center at $\left(0,0,\tfrac12\right)$ and radius $\tfrac12$, so that $N = (0,0,1)$. 




\begin{image}
\begin{tikzpicture}

\draw (0,2) circle (2);
%\draw (-2,2) .. controls (0, 1.5) .. (2,2);
%\draw[dashed] (2,2) -- (-2,2);
\begin{scope}
\draw[clip, draw = none] (-2,2) rectangle (2,-4);
\draw (0,2) ellipse (2 and 0.5);
\end{scope}

\begin{scope}
\draw[clip, draw = none] (-2,2) rectangle (2,4);
\draw[dashed] (0,2) ellipse (2 and 0.5);
\end{scope}

\draw (-5,-1) -- (-3,1) --(5,1) -- (3,-1) --cycle;
\node at (0.3, 4.3) {$N$};
\node at (.2, -.2) {$0$};
\draw (0, 4) -- (-2, -0.5) node[below right] {$z_0$};
\node at (4.2, -.3) {$\C$};
\node at (2.2, 3.3) {$S\setminus N$};
\draw[fill=white] (0,4) circle (0.03);
%\draw[mark=*,mark size=1pt,] plot coordinates {(0, 4)} ;
\draw[mark=*,mark size=1pt,] plot coordinates {(0, 0)} ;
\draw[mark=*,mark size=1pt,] plot coordinates {(-2, -0.5)} ;
\draw[mark=*,mark size=1pt,red] plot coordinates {(-1.08, 1.58)}  ;
\node[red] at (-0.7, 1.7) {$w_0$};
\node at (0, -1.7) {\Large Stereographic Projection: $\varphi (z_0) = w_0$};
\draw[->] (0,0) -- (0, 5);
\end{tikzpicture}
\end{image}



To find the precise formula for the mapping $\varphi$, recall that a line segment in three dimensions can be expressed in vector form as
\[
\gamma(t) = {\bf u} + t{\bf v},  \; 0 \leq t \leq 1
\]

This segment goes from ${\bf u}$ to ${\bf u} + {\bf v}$ as $t$ goes from $0$ to $1$:

\begin{image}
\begin{tikzpicture}

\draw[thin, dashed, blue!80, ->] (0,0) -- (2, 2) node[midway, left] {${\bf u}$};
\draw[blue, ->, thick] (2,2) -- (5,3) node[midway, blue, above] {$\bf{v}$};
\draw[mark=*,mark size=1pt, blue!80] plot coordinates {(0, 0)} node[below left]{$0$} ;
\draw[mark=*,mark size=1pt, blue] plot coordinates {(2, 2)} ;
\draw[mark=*,mark size=1pt, blue] plot coordinates {(5, 3)} ;
\draw[mark=*,mark size=1pt,red] plot coordinates {(3, 2.33)} node[below right] {$\gamma(t) = {\bf u}+ t{\bf v}$} ;
\node at (2.5, -0.8){The segment from ${\bf u} \;\; \mbox{to}\;\; {\bf u}+{\bf v}$};
\end{tikzpicture}
\end{image}

In our case, the vector ${\bf u}$ is $(x,y,0)$ and the vector ${\bf v}$ goes from $(x, y, 0)$ to $(0, 0, 1)$, so ${\bf v} = (-x, -y, 1)$. 
Now, we wish to find $t$ so that the point
\[
{\bf u}+ t{\bf v} = (x- tx,y- ty, t) = (x(1-t), y(1-t), t)
\]
lies on the sphere, $S$, given by

\[
x^2 + y^2 + \left(z-\frac12 \right)^2 = \left(\frac12\right)^2
\]
Substituting gives
\[
x^2(1-t)^2 + y^2(1-t)^2 + \left(t-\frac12 \right)^2 = \left(\frac12\right)^2
\]
which is equivalent to 
\[
x^2(1-t)^2 + y^2(1-t)^2 + t(t-1) = 0
\]
Dividing by $t-1$ gives
\[
x^2(t-1) + y^2(t-1) + t = 0
\]
Solving this linear equation for $t$ we have
\[
t= \frac{x^2 + y^2}{1+x^2+y^2}
\]
Noting that $\displaystyle 1-t = \frac{1}{1+x^2 +y^2}$, we can define $\varphi$:
\[
\varphi(x+iy)   = \left( \frac{x}{1+x^2 +y^2}, \frac{y}{1+x^2 +y^2}, \frac{x^2 + y^2}{1+x^2+y^2} \right)
\]

The bijection $\varphi$ can be extended in a natural way to include $\infty$ in its domain: $\varphi(\infty) = N$.

\begin{question} The mapping $\varphi$ is a bijection from $\C$ to the punctured sphere $S \setminus N$.
What is the formula for $\varphi^{-1} : S \setminus N \to \C$?
\end{question}



\section{Exponential and Logs}
Next, we turn to the exponential function, $f(z) = e^z$ and its mapping properties.
Recall that $e^z$ is $2\pi i$ periodic.  As a result, the image of the band of complex numbers satisfying
\(-\pi < \Imp z \leq \pi\) is its full image, $\C\{0\}$.

\begin{example}[example 6]
Find the image of the vertical segment $x = 2, -\pi < y \leq \pi$ under the mapping $f(z) = e^z$.\\
For $z$ on this segment, 
\[
e^z = e^2 \cis y
\]
and since $y$ takes on all values in an interval of length $2\pi$, $\cis y$ will take on all values on the unit circle.
Hence the image of the segment is the circle centered at the origin with radius $e^2$.
\begin{image}
\begin{tikzpicture}

%z-plane
\draw[<->, thick] (-2.5,0)--(2.5,0) node[right]{$x$}; %real axis
\draw[<->, thick] (0, -2.5) node[left]{$z$-plane} --(0,2.5) node[left]{$iy$}; %imaginary axis

\draw[blue, thick] (1, -1.2) -- (1, 1.2) ; %domain
\draw[thick,blue, fill = white] (1,-1.2) circle (0.07) ;
\draw[thick,blue, fill = blue] (1, 1.2) circle (0.07) ;
\node[blue, thick] at (2, 1.5){$x = 2, -\pi < y \leq \pi$}; %domain label

\draw[->, thick, blue] (3,-1.5)--(5, -1.5) node[above,midway, blue]{$w = e^z$};
%w-plane
\draw[<->, thick] (5,0)--(10,0)node[right]{$u$}; %real axis
\draw[<->, thick] (7.5, -2.5) node[right]{$w$-plane}--(7.5,2.5) node[left]{$iv$}; %imaginary axis

\draw[blue, thick, ->] (7.5,0) circle (2.2); %image
\draw[dashed,thin,blue!80] (7.5,0) -- (9.1, 1.6) node[midway, below right]{$e^2$} ;
\node[blue, thick] at (10.2, 1.7){$u^2 + v^2 = e^4$}; %image label

\node at (3.75, -3.5){\large The effect of $f(z) = \frac{1}{z}$ on a vertical segment}; %caption
\end{tikzpicture}
\end{image}
An alternative method to arrive at the conclusion of the last example is to write
\[
u = e^2\cos  y \quad \mbox{and} \quad v = e^2 \sin y
\]
and note that the Pythagorean trig identity implies that
\[
u^2 + v^2 = (e^2)^2 = e^4
\]
To see that the image is this entire circle, note that since $-\pi < y \leq \pi$ we have
$-e^2 \leq u \leq e^2$ and similarly for $v$.
\end{example}


\begin{problem}(problem 6)
Find the image of the given line segment under the map $w = e^z$:
\begin{align*}
i)\; & \Rep z = 1, 0<\Imp z < \pi\\
ii)\; & \Rep z = -1, \pi/2 < \Imp z < 3\pi/2
\end{align*}
\end{problem}



\begin{example}[example 7]
Find the image of the horizontal line $y = \frac{\pi}{4}$ under the mapping $f(z) = e^z$.\\
For $z$ on this horizontal line, 
\[
e^z = e^x \cis \left(\frac{\pi}{4}\right)
\]
where $-\infty < x < \infty$. Such a point lies on the polar ray $\theta = \pi/4$ at a distance $e^x$ units from the origin.
Since $e^x$ goes from $0$ to $\infty$ as $x$ goes from $-\infty$ to $\infty$, the image of the horizontal line is the polar
ray, $\theta = \pi/4$ with $r>0$.

\begin{image}
\begin{tikzpicture}

%z-plane
\draw[<->, thick] (-2.5,0)--(2.5,0) node[right]{$x$}; %real axis
\draw[<->, thick] (0, -2.5) node[right]{$z$-plane} --(0,2.5) node[left]{$iy$}; %imaginary axis

\draw[blue, thick, <->] (-2, 1.6) -- (2, 1.6) ; %domain

\node[blue, thick] at (1.5, 1.1){$ y = 2$}; %domain label

\draw[->, thick, blue] (3,-1.5)--(5, -1.5) node[above,midway, blue]{$w = e^z$};
%w-plane
\draw[<->, thick] (5,0)--(10,0)node[right]{$u$}; %real axis
\draw[<->, thick] (7.5, -2.5) node[right]{$w$-plane}--(7.5,2.5) node[left]{$iv$}; %imaginary axis

\draw[blue, thick, ->] (7.5,0) -- (9.5,2); %image
\draw[thick,blue, fill = white] (7.5,0) circle (0.07) ;
\node[blue, thick] at (10, .8){$\theta = \pi/4, r>0$}; %image label

\node at (3.75, -3.5){\large The effect of $f(z) = e^z$ on a horizontal line}; %caption
\end{tikzpicture}
\end{image}
An alternative method to arrive at the conclusion of the last example is to write
\[
u = e^x\cos\left(\frac{\pi}{4}\right) = \frac{e^x}{\sqrt 2} = e^x\sin\left(\frac{\pi}{4}\right) = v
\]
where $u$ and $v$ are both strictly positive. This gives the portion of the line $u = v$ in the first quadrant.

\end{example}



\begin{problem}(problem 7)
Find the image of the line $\Imp z = \pi/3$ under the map $w = e^z$.
\end{problem}



\begin{example}[example 8]
Find the image of the semi-circle $z = 3\cis \theta, \; \pi/2 < \theta \leq 3\pi/2$ 
under the mapping $f(z) = \Log z$.\\
On the given semi-circle, $|z| = 3$, so 
\[
w = \Log z = \ln 3 + i\Arg \theta
\]
Thus $u = \ln 3$ and so the image lies on a vertical line. By definition of $\Log z$, 
the values of $v$ must lie in the interval $\left(-\pi, \pi\right]$.
For $\pi/2 <\theta\leq \pi$ we have $\pi/2 <\Arg z \leq  \pi$ and for $\pi < \theta \leq 3\pi/2$ we have $-\pi/2 \geq \Arg z  > -\pi$.
Hence the range of values for $v$ is $(\pi/2, \pi] \cup (-\pi/2, -\pi)$


\begin{image}
\begin{tikzpicture}

%z-plane
\draw[<->, thick] (-2.5,0)--(2.5,0) node[right]{$x$}; %real axis
\draw[<->, thick] (0, -2.5) node[right]{$z$-plane} --(0,2.5) node[left]{$iy$}; %imaginary axis

\draw[blue, thick, ->] (0,1.5) node[right]{$3i$} arc (90:140:1.5) ; %domain
\draw[blue, thick] (-1.0605, 1.0605) arc (135:180:1.5) node[anchor = north east]{$-3$}; %domain
\draw[red, thick, ->] (-1.5,0) arc (180:230:1.5); %domain
\draw[red, thick] (-1.0605,-1.0605) arc (225:270:1.5) node[right]{$-3i$}; %domain

\draw[thick,blue, fill = white] (0,1.5) circle (0.04) ;
\draw[thick,blue, fill = blue] (-1.5,0) circle (0.04) ;
\draw[thick,red, fill=red] (0,-1.5) circle (0.04) ;
\node[blue, thick] at (-1.5, 1.6){$x^2 + y^2 = 9$}; %domain label

\draw[->, thick, blue] (3,-1.5)--(5, -1.5) node[above,midway, blue]{$w = \Log z$};

%w-plane
\draw[<->, thick] (5,0)--(10,0)node[right]{$u$}; %real axis
\draw[<->, thick] (7.5, -2.5) node[right]{$w$-plane}--(7.5,2.5) node[left]{$iv$}; %imaginary axis

\draw[blue, thick, ->] (8.4, 1) -- (8.4, 1.6) ; %image
\draw[blue, thick] (8.4, 1.5) -- (8.4, 2) node[right]{$u = \ln 3$}; %image
\draw[thick,blue, fill=blue] (8.4,2) circle (0.04) ;
\draw[thick,blue, fill=white] (8.4,1) circle (0.04) ;

\draw[red, thick, ->] (8.4, -2) -- (8.4, -1.4) ; %image
\draw[red, thick] (8.4, -1.5) -- (8.4, -1) ; %image
\draw[thick,red, fill=red] (8.4,-1) circle (0.04) ;
\draw[thick,red, fill=white] (8.4,-2) circle (0.04) ;
%\node[red, thick] at (9.5, 1.5){$x = 1, y > 0$}; %image label

\draw[thin] (7.6, 2) -- (7.4, 2) node[left]{$\pi i$};
\draw[thin] (7.6, -2) -- (7.4, -2) node[left]{$-\pi i$};



\node at (3.75, -3.5){\large The effect of $f(z) = \Log z $ on a semi-circle}; %caption
\end{tikzpicture}
\end{image}


\end{example}

\begin{problem}(problem 8)
Find the image of the given line under the map $w = \Log z$:
\begin{align*}
i)\; & C(0,1); \; \Log z\\
ii)\; & \theta = -\pi/4; \; \Log z
\end{align*}
\end{problem}


Below is a Geogebra applet that shows the image of a line under $1/z, e^z$ and $z^2$.  Move the red line and observe the effects!
\begin{center}
\geogebra{Ptnx5Z3m}{600}{600}
\end{center}


Below is a Geogebra applet that shows the image of a circle under $1/z, e^z$ and $z^2$.  Move the red circle and observe the effects!
\begin{center}
\geogebra{WuXRYxUm}{600}{600}
\end{center}




\section{Sine and Cosine}

\begin{example}[example 9]
Find the image of the horizontal line $\Imp z =1$ under the mapping $f(z) = \cos z$.\\
Recall
\[
\cos z = \cos x \cosh y - i \sin x \sinh y
\]
Writin $w = u+iv = \cos z$ and substituting $y = 1$ gives
\[
u = \cos x \cosh 1 \quad \mbox{and} \quad v = -\sin x \sinh y
\]
We square $u$ and $v$ to take advantage of the Pythagorean Identity,
\[
\frac{u^2}{\cosh^2 1} + \frac{v^2}{\sinh^2 1} = \cos^2 x + \sin^2 x = 1
\]
This is the equation of an ellipse with $u$-intercepts $u = \pm \cosh 1$
and $v$-intercepts $v = \pm \sinh 1$.


\begin{image}
\begin{tikzpicture}

\draw[<->, thick] (-3,0)--(3,0) node[right]{$x$};

\draw[<->, thick] (0, -3) node[right]{$z$-plane} --(0,3) node[right]{$iy$};

\draw[blue, thick, <->] (-2.5, 1) -- (2.5, 1) ;
\node[blue] at (1, 1.6){$y = 1$};

\draw[thick, blue] (9,0) ellipse (1.54 and 1.17);


\node[blue] at (11.5,2){$\frac{u^2}{\cosh^2 1} + \frac{v^2}{\sinh^2 1} = 1$};

\draw[<->, thick] (6,0)--(12,0)node[right]{$u$};


\draw[<->, thick] (9, -3) node[right]{$w$-plane}--(9,3) node[right]{$iv$};

\draw[->, thick, blue] (3.5,-1.5)--(5.5, -1.5) node[above,midway, blue]{$w = \cos z$};
\node at (4.5, -4){\large The effect of $f(z) = \cos z$ on a horizontal line};
\end{tikzpicture}
\end{image}


\end{example}


\begin{problem}(problem 9)
Find the image of the given line under the given map:
\begin{align*}
i)\; & \Rep z = \pi/4; \; \cos z \\
ii)\; & \Imp z = -2; \; \cos z \\
iii)\; & \Rep z = \pi/6; \; \sin z\\
iv)\; & \Imp z = 1; \; \sin z
\end{align*}
\end{problem}


\end{document}









\pgfdeclareradialshading{ballshading}{\pgfpoint{-10bp}{10bp}}
 {color(0bp)=(gray!40!white); 
 color(9bp)=(gray!75!white);
 color(18bp)=(gray!70!black); 
 color(25bp)=(gray!50!black); 
 color(50bp)=(black)}

\begin{pgfpicture}
  \pgfpathcircle{\pgfpoint{0cm}{0cm}}{2cm}
  \pgfshadepath{ballshading}{20}
  \pgfusepath{}
\end{pgfpicture} 























\begin{image}
\begin{tikzpicture}


\begin{scope}
\draw[clip, draw = none] (1,0) arc (0:180:2) --cycle;
\draw[outer color = blue!50, inner color = white, shading = radial] (-1,0) circle (2);
%\draw[ color = blue!50,shading = radial] (-2,0) circle (2);
\end{scope}

 

 \begin{scope}
\draw[clip, draw = none] (1,0) arc (0:-180:2) --cycle;
\draw[outer color = red!50, inner color = white] (-1,0) circle (2);
\end{scope}

\draw[<->, thick] (-4,0)--(2,0) node[right]{$x$};
\draw[<->, thick] (-1, -3) node[right]{$z$-plane} --(-1,3) node[right]{$iy$};

\begin{scope}
\draw[clip, draw = none] (6.2,0) rectangle (11.8,2.8);
\filldraw[draw = none, inner color = red, outer color = red!10, shading = radial] (6.2,-2.8) rectangle (11.8,2.8) ;
\end{scope}

\begin{scope}
\draw[clip, draw = none] (6.2,-2.8) rectangle (11.8,0);
\filldraw[draw = none, inner color = blue, outer color = blue!10, shading = radial] (6.2,-2.8) rectangle (11.8,2.8) ;
\end{scope}

\draw[dashed, fill = white] (9,0)circle (1) ;
\draw[white, thick] (6.2, -2.8) rectangle (11.8, 2.8);
\draw[thick, color = blue!20] (6.5,0)--(8,0);
\draw[thin, color = blue!20] (10,0)--(11.3,0);

\draw[<->, thick] (6,0)--(12,0)node[right]{$u$};


\draw[<->, thick] (9, -3) node[right]{$w$-plane}--(9,3) node[right]{$iv$};

\draw[->, thick, blue] (3,1)--(5, 1) node[above,midway, blue]{$w = \frac{1}{z}$};

\draw[dashed, fill = white] (-1,0)circle (.05) ;
\draw[dashed] (-1,0)circle (2) ;

\node at (4, -4){\large The effect of $f(z) = \frac{1}{z}$ on the open punctured unit disk};

\draw[mark=*,mark size=1pt,] plot coordinates {(-.3, 1.25)} ;
\node at (-.45,1) {$z_0$};

\draw[mark=*,mark size=1pt] plot coordinates {(9.8, -1.4)} ;
\node at (10.2,-1.4) {$\frac{1}{z_0}$};

\end{tikzpicture}
\end{image}





\begin{image}
\begin{tikzpicture}
\filldraw[dashed, bottom color = red!30, top color = blue!30] (0,0) circle (1) ;
\draw[mark=*,mark size=1pt,] plot coordinates {(.3, .75)} ;
\node at (.45,.55) {$z_0$};
\draw[<->, thick] (-3,0)--(3,0) node[right]{$x$};

\draw[<->, thick] (0, -3) node[right]{$z$-plane} --(0,3) node[right]{$iy$};
\draw[dashed, fill = white] (0,0)circle (.05) ;



\filldraw[draw = none, top color = red!30, bottom color = blue!30] (6.2,-2.8) rectangle (11.8,2.8) ;
\draw[dashed, fill = white] (9,0)circle (1) ;
\draw[mark=*,mark size=1pt] plot coordinates {(10, -2.5)} ;
\node at (10.3,-2.2) {$1/z_0$};
\draw[->, thick, blue] (3.5,1)--(5.5, 1) node[above,midway, blue]{$w = \frac{1}{z}$};
\draw[<->, thick] (6,0)--(12,0)node[right]{$u$};


\draw[<->, thick] (9, -3) node[right]{$w$-plane}--(9,3) node[right]{$iv$};


\node at (4.5, -4){\large The effect of $f(z) = \frac{1}{z}$ on the open punctured unit disk};
\end{tikzpicture}
\end{image}

\begin{scope}
    \clip (1,0) arc (180:360:2)-- cycle;
    \draw[outer color = blue, inner color = red] (1,0) arc (0:3600:2) -- cycle;
\end{scope}
