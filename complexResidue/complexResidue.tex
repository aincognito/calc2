\documentclass[handout]{ximera}

%% You can put user macros here
%% However, you cannot make new environments



\newcommand{\ffrac}[2]{\frac{\text{\footnotesize $#1$}}{\text{\footnotesize $#2$}}}
\newcommand{\vasymptote}[2][]{
    \draw [densely dashed,#1] ({rel axis cs:0,0} -| {axis cs:#2,0}) -- ({rel axis cs:0,1} -| {axis cs:#2,0});
}


\graphicspath{{./}{firstExample/}}

\usepackage{amsmath}
\usepackage{amssymb}
\usepackage{array}
\usepackage[makeroom]{cancel} %% for strike outs
\usepackage{pgffor} %% required for integral for loops
\usepackage{tikz}
\usepackage{tikz-cd}
\usepackage{tkz-euclide}
\usetikzlibrary{shapes.multipart}


\usetkzobj{all}
\tikzstyle geometryDiagrams=[ultra thick,color=blue!50!black]


\usetikzlibrary{arrows}
\tikzset{>=stealth,commutative diagrams/.cd,
  arrow style=tikz,diagrams={>=stealth}} %% cool arrow head
\tikzset{shorten <>/.style={ shorten >=#1, shorten <=#1 } } %% allows shorter vectors

\usetikzlibrary{backgrounds} %% for boxes around graphs
\usetikzlibrary{shapes,positioning}  %% Clouds and stars
\usetikzlibrary{matrix} %% for matrix
\usepgfplotslibrary{polar} %% for polar plots
\usepgfplotslibrary{fillbetween} %% to shade area between curves in TikZ



%\usepackage[width=4.375in, height=7.0in, top=1.0in, papersize={5.5in,8.5in}]{geometry}
%\usepackage[pdftex]{graphicx}
%\usepackage{tipa}
%\usepackage{txfonts}
%\usepackage{textcomp}
%\usepackage{amsthm}
%\usepackage{xy}
%\usepackage{fancyhdr}
%\usepackage{xcolor}
%\usepackage{mathtools} %% for pretty underbrace % Breaks Ximera
%\usepackage{multicol}



\newcommand{\RR}{\mathbb R}
\newcommand{\R}{\mathbb R}
\newcommand{\C}{\mathbb C}
\newcommand{\N}{\mathbb N}
\newcommand{\Z}{\mathbb Z}
\newcommand{\dis}{\displaystyle}
%\renewcommand{\d}{\,d\!}
\renewcommand{\d}{\mathop{}\!d}
\newcommand{\dd}[2][]{\frac{\d #1}{\d #2}}
\newcommand{\pp}[2][]{\frac{\partial #1}{\partial #2}}
\renewcommand{\l}{\ell}
\newcommand{\ddx}{\frac{d}{\d x}}

\newcommand{\zeroOverZero}{\ensuremath{\boldsymbol{\tfrac{0}{0}}}}
\newcommand{\inftyOverInfty}{\ensuremath{\boldsymbol{\tfrac{\infty}{\infty}}}}
\newcommand{\zeroOverInfty}{\ensuremath{\boldsymbol{\tfrac{0}{\infty}}}}
\newcommand{\zeroTimesInfty}{\ensuremath{\small\boldsymbol{0\cdot \infty}}}
\newcommand{\inftyMinusInfty}{\ensuremath{\small\boldsymbol{\infty - \infty}}}
\newcommand{\oneToInfty}{\ensuremath{\boldsymbol{1^\infty}}}
\newcommand{\zeroToZero}{\ensuremath{\boldsymbol{0^0}}}
\newcommand{\inftyToZero}{\ensuremath{\boldsymbol{\infty^0}}}


\newcommand{\numOverZero}{\ensuremath{\boldsymbol{\tfrac{\#}{0}}}}
\newcommand{\dfn}{\textbf}
%\newcommand{\unit}{\,\mathrm}
\newcommand{\unit}{\mathop{}\!\mathrm}
%\newcommand{\eval}[1]{\bigg[ #1 \bigg]}
\newcommand{\eval}[1]{ #1 \bigg|}
\newcommand{\seq}[1]{\left( #1 \right)}
\renewcommand{\epsilon}{\varepsilon}
\renewcommand{\iff}{\Leftrightarrow}

\DeclareMathOperator{\arccot}{arccot}
\DeclareMathOperator{\arcsec}{arcsec}
\DeclareMathOperator{\arccsc}{arccsc}
\DeclareMathOperator{\si}{Si}
\DeclareMathOperator{\proj}{proj}
\DeclareMathOperator{\scal}{scal}
\DeclareMathOperator{\cis}{cis}
\DeclareMathOperator{\Arg}{Arg}
%\DeclareMathOperator{\arg}{arg}
\DeclareMathOperator{\Rep}{Re}
\DeclareMathOperator{\Imp}{Im}
\DeclareMathOperator{\sech}{sech}
\DeclareMathOperator{\csch}{csch}
\DeclareMathOperator{\Log}{Log}

\newcommand{\tightoverset}[2]{% for arrow vec
  \mathop{#2}\limits^{\vbox to -.5ex{\kern-0.75ex\hbox{$#1$}\vss}}}
\newcommand{\arrowvec}{\overrightarrow}
\renewcommand{\vec}{\mathbf}
\newcommand{\veci}{{\boldsymbol{\hat{\imath}}}}
\newcommand{\vecj}{{\boldsymbol{\hat{\jmath}}}}
\newcommand{\veck}{{\boldsymbol{\hat{k}}}}
\newcommand{\vecl}{\boldsymbol{\l}}
\newcommand{\utan}{\vec{\hat{t}}}
\newcommand{\unormal}{\vec{\hat{n}}}
\newcommand{\ubinormal}{\vec{\hat{b}}}

\newcommand{\dotp}{\bullet}
\newcommand{\cross}{\boldsymbol\times}
\newcommand{\grad}{\boldsymbol\nabla}
\newcommand{\divergence}{\grad\dotp}
\newcommand{\curl}{\grad\cross}
%% Simple horiz vectors
\renewcommand{\vector}[1]{\left\langle #1\right\rangle}


\pgfplotsset{compat=1.13}

\outcome{Use the Residue Theorem to compute complex integrals}

\title{5.3 Residue Theorem}

\begin{document}

\begin{abstract}
We use the Residue Theorem to compute integrals of complex functions around closed contours.
\end{abstract}

\maketitle


Recall that the coefficient $c_{-1}$ in the Laurent series of a function centered at $z_0$,
\[
f(z) = \sum_{k=-\infty}^\infty c_k (z-z_0)^k
\]
is called the residue of $f$ at $z_0$, denoted Res$(f,z_0)$.

If we integrated this Laurent series term by term around a positively oriented simple closed curve, $C$, containing no other 
singularities in its interior besides (possibly) $z_0$ then we would obtain:
\begin{align*}
\int_C \left(\sum_{k=-\infty}^\infty c_k (z-z_0)^k \right)\, dz &= \sum_{k=-\infty}^\infty \left( \int_C c_k (z-z_0)^k \, dz \right)\\
&= c_{-1} \int_C \frac{1}{z-z_0} \, dz \quad \text{(verify)}\\
&= 2\pi i c_{-1} \quad \text{(verify)}\\
&= 2\pi i \text{Res}(f, z_0)
\end{align*}
Combining the above result with the Extended Deformation of Contour Theorem, we obtain the Residue Theorem:

\begin{theorem}[Residue Theorem]
Let $C$ be a positively oriented, simple closed contour and suppose the function $f$ is analytic along $C$ and inside $C$,
except at finitely many isolated singularities at $z_1, z_2, \dots, z_n$. Then
\[
\int_C f(z) \, dz = 2\pi i \sum_{k=1}^n \text{Res}(f,z_k)
\]
\end{theorem}


\begin{example}[example 1a]
Let $C$ be a positively oriented, simple closed contour containing the origin in its interior.
Compute the contour integral
\[
\int_C e^{1/z} \, dz
\]
The integrand has a singularity at the origin (and nowhere else) and its Laurent series centered at $0$ is
\[
e^{1/z} = 1 + \frac{1}{z} + \frac{1}{2!z^2} + \frac{1}{3! z^3} + \cdots
\]
Hence, Res$(e^{1/z},0) = c_{-1} = 1$. By the Residue Theorem we obtain
\[
\int_C e^{1/z} \, dz = 2\pi i \text{Res}(e^{1/z}, 0) = 2\pi i
\]
\end{example}


\begin{example}[example 1b]
Let $C$ be a positively oriented, simple closed contour containing the points $0$ and $1$ in its interior.
Compute the contour integral
\[
\int_C \frac{e^{1/z}}{1-z} \, dz
\]

The function $\dis f(z) = \frac{e^{1/z}}{1-z}$ has singularities at both $0$ and $1$, we need to find Laurent series for $f$
centered at both of these points.
At $z = 1$, the function $e^{1/z}$ is analytic an hence has a Taylor series centered at $1$ of the form:
\[
e^{1/z} = c_0 + c_1(z-1) + c_2(z-1)^2 + \cdots
\]
The Laurent series for $\frac{1}{1-z}$ centered at $1$ is simply
\[
\frac{1}{1-z} = -\frac{1}{z-1}
\]
Multiplying these Taylor and Laurent series gives the Laurent series for $f$:
\[
\frac{e^{1/z}}{1-z} = -\frac{c_0}{z-1} -c_1 -c_2(z-1) - c_3(z-1)^2 - \cdots
\]
Thus Res$(f, 1) = -c_0 = -e$ since the is the constant term of the Taylor series of $e^{1/z} =$ centered at $1$ is $e^{1/1} =e$.\\
Next, at $z = 0$, the function $\frac{1}{1-z}$ is analytic and has Taylor series
\[
\frac{1}{1-z} = 1 + z + z^2 + z^3 + \cdots
\]
and the function $e^{1/z}$ has Laurent series
\[
e^{1/z} = 1 + \frac{1}{z} + \frac{1}{2! z^2} + \frac{1}{3! z^3} + \cdots
\]
The Laurent series of $f$ is obtained by multiplying these Taylor and Laurent series together
\[
\frac{e^{1/z}}{1-z} = \left(1 + z + z^2 + z^3 + \cdots\right) \cdot \left( 1 + \frac{1}{z} + \frac{1}{2! z^2} + \frac{1}{3! z^3} + \cdots\right)
\]
We only need the residue of $f$, so let's compute the coefficient of $\frac{1}{z}$ in the product above.
We get:
\[
\text{Res}(f, 0) = 1 + \frac{1}{2!} +\frac{1}{3!} + \frac{1}{4!} + \cdots = e-1
\]
Finally, we can evaluate the contour integral using the Residue Theorem:
\[
\int_C \frac{e^{1/z}}{1-z} \, dz = 2\pi i \left[\text{Res}(f,0) + \text{Res}(f,1)\right] = -2\pi i
\]
\end{example}


\begin{problem}(problem 1a)
Let $C$ be a positively oriented, simple closed curve containing the origin in its interior. Compute
\[
\int_C \sin\left(\frac{1}{z}\right) \, dz = \answer{2\pi i}
\]

\end{problem}

\begin{problem}(problem 1b)
Let $C$ be a positively oriented, simple closed curve containing $0$ and $1$ in its interior. Compute
\[
\int_C f(z) \, dz \quad \text{where} \quad f(z) = \frac{\sin\left(\frac{1}{z}\right)}{1-z}
\]
%Let $f(z) = \frac{\sin\left(\frac{1}{z}\right)}{1-z} $\quad 
\center{Res$(f, 0) = \answer{\sin(1)}$ \quad Res$(f, 1) = \answer{-\sin(1)}$}\\
\[
 \int_C \frac{\sin\left(\frac{1}{z}\right)}{1-z} \, dz = \answer{0}
 \]

\end{problem}

\end{document}



\[
\int_C \frac{1}{z} \, dz = \answer{\frac12 \ln 2+ i\pi/4}.
\]
\begin{hint}
Use the principal branch of the logarithm, $\Log z$, as the anti-derivative on $C$.
\end{hint}
\end{problem}


\end{document}


                         
                         



                         
