\documentclass{ximera}
%\usepackage{tcolorbox}
%% You can put user macros here
%% However, you cannot make new environments



\newcommand{\ffrac}[2]{\frac{\text{\footnotesize $#1$}}{\text{\footnotesize $#2$}}}
\newcommand{\vasymptote}[2][]{
    \draw [densely dashed,#1] ({rel axis cs:0,0} -| {axis cs:#2,0}) -- ({rel axis cs:0,1} -| {axis cs:#2,0});
}


\graphicspath{{./}{firstExample/}}

\usepackage{amsmath}
\usepackage{amssymb}
\usepackage{array}
\usepackage[makeroom]{cancel} %% for strike outs
\usepackage{pgffor} %% required for integral for loops
\usepackage{tikz}
\usepackage{tikz-cd}
\usepackage{tkz-euclide}
\usetikzlibrary{shapes.multipart}


\usetkzobj{all}
\tikzstyle geometryDiagrams=[ultra thick,color=blue!50!black]


\usetikzlibrary{arrows}
\tikzset{>=stealth,commutative diagrams/.cd,
  arrow style=tikz,diagrams={>=stealth}} %% cool arrow head
\tikzset{shorten <>/.style={ shorten >=#1, shorten <=#1 } } %% allows shorter vectors

\usetikzlibrary{backgrounds} %% for boxes around graphs
\usetikzlibrary{shapes,positioning}  %% Clouds and stars
\usetikzlibrary{matrix} %% for matrix
\usepgfplotslibrary{polar} %% for polar plots
\usepgfplotslibrary{fillbetween} %% to shade area between curves in TikZ



%\usepackage[width=4.375in, height=7.0in, top=1.0in, papersize={5.5in,8.5in}]{geometry}
%\usepackage[pdftex]{graphicx}
%\usepackage{tipa}
%\usepackage{txfonts}
%\usepackage{textcomp}
%\usepackage{amsthm}
%\usepackage{xy}
%\usepackage{fancyhdr}
%\usepackage{xcolor}
%\usepackage{mathtools} %% for pretty underbrace % Breaks Ximera
%\usepackage{multicol}



\newcommand{\RR}{\mathbb R}
\newcommand{\R}{\mathbb R}
\newcommand{\C}{\mathbb C}
\newcommand{\N}{\mathbb N}
\newcommand{\Z}{\mathbb Z}
\newcommand{\dis}{\displaystyle}
%\renewcommand{\d}{\,d\!}
\renewcommand{\d}{\mathop{}\!d}
\newcommand{\dd}[2][]{\frac{\d #1}{\d #2}}
\newcommand{\pp}[2][]{\frac{\partial #1}{\partial #2}}
\renewcommand{\l}{\ell}
\newcommand{\ddx}{\frac{d}{\d x}}

\newcommand{\zeroOverZero}{\ensuremath{\boldsymbol{\tfrac{0}{0}}}}
\newcommand{\inftyOverInfty}{\ensuremath{\boldsymbol{\tfrac{\infty}{\infty}}}}
\newcommand{\zeroOverInfty}{\ensuremath{\boldsymbol{\tfrac{0}{\infty}}}}
\newcommand{\zeroTimesInfty}{\ensuremath{\small\boldsymbol{0\cdot \infty}}}
\newcommand{\inftyMinusInfty}{\ensuremath{\small\boldsymbol{\infty - \infty}}}
\newcommand{\oneToInfty}{\ensuremath{\boldsymbol{1^\infty}}}
\newcommand{\zeroToZero}{\ensuremath{\boldsymbol{0^0}}}
\newcommand{\inftyToZero}{\ensuremath{\boldsymbol{\infty^0}}}


\newcommand{\numOverZero}{\ensuremath{\boldsymbol{\tfrac{\#}{0}}}}
\newcommand{\dfn}{\textbf}
%\newcommand{\unit}{\,\mathrm}
\newcommand{\unit}{\mathop{}\!\mathrm}
%\newcommand{\eval}[1]{\bigg[ #1 \bigg]}
\newcommand{\eval}[1]{ #1 \bigg|}
\newcommand{\seq}[1]{\left( #1 \right)}
\renewcommand{\epsilon}{\varepsilon}
\renewcommand{\iff}{\Leftrightarrow}

\DeclareMathOperator{\arccot}{arccot}
\DeclareMathOperator{\arcsec}{arcsec}
\DeclareMathOperator{\arccsc}{arccsc}
\DeclareMathOperator{\si}{Si}
\DeclareMathOperator{\proj}{proj}
\DeclareMathOperator{\scal}{scal}
\DeclareMathOperator{\cis}{cis}
\DeclareMathOperator{\Arg}{Arg}
%\DeclareMathOperator{\arg}{arg}
\DeclareMathOperator{\Rep}{Re}
\DeclareMathOperator{\Imp}{Im}
\DeclareMathOperator{\sech}{sech}
\DeclareMathOperator{\csch}{csch}
\DeclareMathOperator{\Log}{Log}

\newcommand{\tightoverset}[2]{% for arrow vec
  \mathop{#2}\limits^{\vbox to -.5ex{\kern-0.75ex\hbox{$#1$}\vss}}}
\newcommand{\arrowvec}{\overrightarrow}
\renewcommand{\vec}{\mathbf}
\newcommand{\veci}{{\boldsymbol{\hat{\imath}}}}
\newcommand{\vecj}{{\boldsymbol{\hat{\jmath}}}}
\newcommand{\veck}{{\boldsymbol{\hat{k}}}}
\newcommand{\vecl}{\boldsymbol{\l}}
\newcommand{\utan}{\vec{\hat{t}}}
\newcommand{\unormal}{\vec{\hat{n}}}
\newcommand{\ubinormal}{\vec{\hat{b}}}

\newcommand{\dotp}{\bullet}
\newcommand{\cross}{\boldsymbol\times}
\newcommand{\grad}{\boldsymbol\nabla}
\newcommand{\divergence}{\grad\dotp}
\newcommand{\curl}{\grad\cross}
%% Simple horiz vectors
\renewcommand{\vector}[1]{\left\langle #1\right\rangle}


\outcome{Discuss the continuity of a function.}

\title{1.8 Continuity}

%\newcommand{\ffrac}[2]{\frac{\mbox{\footnotesize $#1$}}{\mbox{\footnotesize $#2$}}}
%\newcommand{\vasymptote}[2][]{
 %   \draw [densely dashed,#1] ({rel axis cs:0,0} -| {axis cs:#2,0}) -- ({rel axis cs:0,1} -| {axis cs:#2,0});
%}


\begin{document}

\begin{abstract}
In this section we learn the definition of continuity and we study the types of discontinuities.
\end{abstract}

\maketitle






\section{Continuity}

One of the tools we used in the last section for evaluating limits analytically was plugging in. For example, to compute
\[\lim_{x \to 5} x^2 \]
we would simply plug in $x = 5$ and get the correct answer of 25. In most of the examples in the last section, 
plugging in 
did not give an acceptable answer at first, so more work had to be done before eventually plugging in. 
If plugging does work at the outset, then the function is called continuous.

\begin{definition}[Continuity]
The function $f(x)$ is \textbf{continuous} at $x = a$ if:
\[\lim_{x \to a} f(x) = f(a).\]
\end{definition}

\begin{example}[example 1]
Since
\[\lim_{x \to 5} x^2 = 5^2 = 25,\]
we can say that the function $f(x) = x^2$ is continuous at $x = 5$.
\end{example}

\begin{problem}(problem 1)
Since
\[\lim_{x \to 4} \sqrt{x} = \sqrt{4} = 2,\]
\begin{center}
we can say that the function \\
$f(x) = \answer{\sqrt x}$ \\
is continuous at \\
$x = \answer{4}.$
\end{center}
\end{problem}


\begin{example}[example 2]
Since
\[\lim_{x \to \pi} \cos(x) = \cos(\pi) = -1,\]
we can say that the function $f(x) = \cos(x)$ is continuous at $x = \pi$.
\end{example}

\begin{problem}(problem 2)
Since
\[\lim_{x \to 3} 2^x = 2^3 = 8,\]
\begin{center}
we can say that the function \\
$f(x) = \answer{2^x}$ \\
is continuous at \\
$x = \answer{3}.$
\end{center}
\end{problem}

\begin{example}[example 3]
Since
\[\lim_{x \to 1} \ln(x) = \ln(1) = 0,\]
we can say that the function $f(x) = \ln(x)$ is continuous at $x = 1$.
\end{example}

\begin{problem}(problem 3)
Since
\[\lim_{x \to \sqrt 3} \tan^{-1}(x) = \tan^{-1}(\sqrt 3) = \pi/3,\]
\begin{center}
we can say that the function \\
$f(x) = \answer{\tan^{-1}(x)}$ \\
is continuous at \\
$x = \answer{\sqrt 3}.$
\end{center}
\end{problem}

If a function is continuous at each point in an interval, $I$, then we say that $f(x)$ is \textbf{continuous on $I$}.

A function $f(x)$ is called \textbf{continuous from the left at $x=a$} if 
\[\lim_{x \to a^-} f(x) = f(a),\]
and it is called \textbf{continuous from the right at $x = a$} if
\[\lim_{x \to a^+} f(x) = f(a).\]
Note that if $f(x)$ is continuous at $x=a$ then it is both continuous from the left and 
continuous from the right at $x = a$.

If for some reason, a limit cannot be computed by plugging in, then we say that the function is discontinuous.
In other words, if
\[\lim_{x \to a} f(x) \neq f(a),\]
then we say that $f(x)$ has a \textbf{discontinuity} at $x = a$.
Next, we explore the types of discontinuities.

%As an example the function $f(x) = \tan(x)$ is not continuous at $x = \frac{\pi}{2}$ because $\tan(\frac{\pi}{2})$
%is undefined.

% and hence
%\[\lim_{x \to \frac{\pi}{2}} \tan(x) \neq \tan(\frac{\pi}{2}).\]



%We will study examples of continuity from the left and right in the section on piecewise functions below.


\begin{center}
\textbf{Types of Discontinuities}
\end{center}

A function $f(x)$ has a discontinuity at $x = a$ if
\[\lim_{x \to a} f(x) \neq f(a).\]
There are four main types of discontinuities: removable, jump, infinite and essential. 
%They are called removable, jump, infinite and essential. 
%Suppose $f(x)$ has a  discontinuity at $x=a$, then

%Depending on the reason for these two quantities not being equal, we will get a different type of discontinuity.
First, a discontinuity is called a \textbf{removable discontinuity} if 
\[\lim_{x \to a} f(x)  \  \  \text{exists and is finite }.\]
%In other words, the limit is equal to a number but that number is not $f(a)$.  
%This might be because $f(a)$ is undefined
%or it could be that even though $f(a)$ is defined, it just doesn't equal the limit. Such a discontinuity is 
%called removable because if we were to define the function appropriately at $x=a$ it would become 
%continuous, i.e., the discontinuity would have been removed.

Here are two examples of graphs of functions that have removable discontinuities:
\begin{center}
\begin{tikzpicture}
\begin{axis}[axis lines = center, title={A Removable Discontinuity at $x=1$}]
\addplot[domain=0:0.98, color=blue]{x+1};
\addplot[smooth,mark=*,blue] plot coordinates {(1,1.5)};
\addplot[domain=1.02:2, 
    samples=100, color=blue]{x+1};
\addplot[smooth,mark=o,blue] plot coordinates {(1,2)};
\end{axis}
\end{tikzpicture}
\hspace{1.5 in}
\begin{tikzpicture}
\begin{axis}[axis lines = center, title={A Removable Discontinuity at $x=2$}]
\addplot[domain=1:1.985, color=blue]{x^2 - 3};
\addplot[smooth,mark=o,blue] plot coordinates {(2,1)};
\addplot[domain=2.015:3, 
    samples=100, color=blue]{x^2 - 3};
\end{axis}
\end{tikzpicture}
\end{center}


The second type of discontinuity is called a \textbf{jump discontinuity}. 
This happens when the one-sided
limits are different numbers, so that
%\[\lim_{x \to a} f(x) \ \ \text{does not exist}\]
%for the following reason:
%\[\lim_{x \to a^-} f(x) \ \ \text{is finite, and}\]
%\[\lim_{x \to a^+} f(x) 

%\ \ \text{is finite, but}\]
\[\lim_{x \to a^-} f(x) \neq \lim_{x \to a^+} f(x).\]
and hence
\[
\lim_{x \to a} f(x) \ \text{DNE}.
\]

It is important in this case that the one-sided limits are both finite.

% finite numbers but they are not equal to each other, 
%and therefore the two sided limit does not exist.
\begin{example}[example 4]
As an example, the function 
\[f(x) = \frac{|x|}{x}\]
 has a jump discontinuity at $x = 0$ because
\[\lim_{x \to 0^-} \frac{|x|}{x} = -1 \text{  but} \]
\[\lim_{x \to 0^+} \frac{|x|}{x} = 1. \]

A graph of this function is shown below.

\includeinteractive{cf01.js}
\end{example}

Here are two examples of graphs of functions that have jump discontinuities:

\begin{center}
\begin{tikzpicture}
\begin{axis}[axis lines = center, title={A Jump Discontinuity at $x=1$}]
\addplot[domain=-1:1, color=blue]{x^2};
\addplot[smooth,mark=*,blue] plot coordinates {(1,1)};
\addplot[domain=1.02:2, 
    samples=100, color=blue]{x+1};
\addplot[smooth,mark=o,blue] plot coordinates {(1,2)};
\end{axis}
\end{tikzpicture}
\hspace{1.5 in}
\begin{tikzpicture}
\begin{axis}[axis lines = center, title={A Jump Discontinuity at $x=2$}]
\addplot[domain=0:1.99, color=blue]{x^2 - 1};
\addplot[smooth,mark=o,blue] plot coordinates {(2,3)};
\addplot[domain=2.02:4, 
    color=blue]{3-x};
\addplot[smooth,mark=o,blue] plot coordinates {(2,1)};
\end{axis}
\end{tikzpicture}
\end{center}


The third type of discontinuity is called an \textbf{infinite discontinuity}. 
This occurs when either of the one-sided limits is either $\pm \infty$, i.e., 
\[\lim_{x \to a^-} f(x) = \pm\infty \ \text{or}  \]
\[\lim_{x \to a^+} f(x) = \pm\infty. \]
The graph of the function $f(x)$ has a vertical asymptote at an infinite discontinuity.
The function $f(x) = 1/x$,  has an infinite discontinuity at $x=0$ since
\[\lim_{x \to 0^-} \frac{1}{x} = -\infty \ \text {  and}\]
\[\lim_{x \to 0^+} \frac{1}{x} = \infty.\]
The graph of $f(x) = 1/x$ is shown below.
\[
\graph{1/x}
\]
%\includeinteractive{cf02.js}

\begin{example}[example 5]
The function $f(x) = \tan(x)$ has an infinite discontinuity at $x = \frac{\pi}{2}$
since
\[\lim_{x \to \frac{\pi}{2}^-} \tan(x) = \infty \ \text {  and}\]
\[\lim_{x \to \frac{\pi}{2}^+} \tan(x) = -\infty.\]
In fact, $f(x) = \tan(x)$ has infinite discontinuities at all odd multiples of $\pi/2$. 
The graph of $f(x) = \tan(x)$ is below.

\[
\graph{tan(x)}
\]
%\includeinteractive{cf03.js}

\end{example}


Finally, if a discontinuity is not one of the first three types, it is called an 
\textbf{essential discontinuity}.
As an example, the function $f(x) = \sin(\frac{1}{x})$, shown below, has an essential discontinuity at $x = 0$.
Neither of one-sided limits at $x=0$ exist due to oscillation of the function, 
and the function does not have a vertical asymptote since 
\[-1 \leq \sin\big(\frac{1}{x}\big) \leq 1 \]
for all values of $x$ except $x = 0$, where the function is undefined.

\[
\graph{sin(1/x)}
\]
%Here is an example of the graph of a function that has an essential discontinuity:


%\begin{center}
%\begin{tikzpicture}
%\begin{axis}[axis lines = center, title={An Essential Discontinuity at $x=0$}]
%\addplot[domain=-1:-0.01, color=blue]{sin(deg(1/x))};
%\addplot[domain=0.01:1, color=blue]{sin(deg(1/x))};

%\end{axis}
%\end{tikzpicture}
%\end{center}

\begin{problem}(problem 5a)
Which type of discontinuity does $f(x)$ have at $x=2$ if
\[
\lim_{x \to 2^+} f(x) = -\infty?
\]

\begin{multipleChoice}
  \choice{removable}
  \choice{jump}
  \choice[correct]{infinite}
  \choice{essential}
\end{multipleChoice}
\end{problem}

\begin{problem}(problem 5b)
Which type of discontinuity does $f(x)$ have at $x=2$ if
\[
\lim_{x \to 2^-} f(x) = \lim_{x \to 2^+} f(x),
\]
but $f(2)$ is undefined?
\begin{multipleChoice}
  \choice[correct]{removable}
  \choice{jump}
  \choice{infinite}
  \choice{essential}
\end{multipleChoice}
\end{problem}

\begin{problem}(problem 5c)
Which type of discontinuity does $f(x)$ have at $x=2$ if
\[
\lim_{x \to 2^+} f(x) \ \text{DNE}?
\]

\begin{multipleChoice}
  \choice{removable}
  \choice{jump}
  \choice{infinite}
  \choice[correct]{essential}
\end{multipleChoice}
\end{problem}

\begin{problem}(problem 5d)
Which type of discontinuity does $f(x)$ have at $x=2$ if
\[
\lim_{x \to 2^-} f(x) = 3
\]
and
\[
\lim_{x \to 2^+} f(x) = -1
\]
\begin{multipleChoice}
  \choice{removable}
  \choice[correct]{jump}
  \choice{infinite}
  \choice{essential}
\end{multipleChoice}
\end{problem}



\begin{center}
\textbf{Piecewise Functions}
\end{center}

In this section we will examine the continuity of piecewise defined functions.

\begin{example}[example 6]
Is the function defined below continuous at $x = 2$?
\[
f(x) = \left\{
     \begin{array}{lr}
       x-1 & \ \text{for} \  x \leq 2 \\
       3 & \ \text{for} \ x > 2
     \end{array}
   \right.
\]


We need to  compute $f(2)$ and the one-sided limits as $x$ approaches 2.
If all three of these are equal, then $f(x)$ is continuous at $x=2$.
Otherwise, $f(x)$ has a discontinuity at $x=2$.\\
First, 
\[
f(2) = 2-1 = 1.
\]
Next,  
\[\lim_{x \to 2^-} f(x) = \lim_{x \to 2^-} x-1 = 2-1 = 1.\]
Finally,
\[\lim_{x \to 2^+} f(x) = \lim_{x \to 2^+}3 = 3.\]
Since the one-sided limits are different, the two-sided limit
\[
\lim_{x \to 2} f(x) \ \text{DNE}
\]
and hence $f(x)$ is not continuous at $x = 2$.
In this case, $f(x)$ has a jump discontinuity at $x=2$ since the one-sided limits are finite but different.
\end{example}

\begin{problem}(problem 6)
Determine whether the function $f(x)$ given below is continuous at $x = 1$.
\[
f(x) = \left\{
     \begin{array}{lr}
       2x & \ \text{for} \  x < 1 \\
       x^2 + 2 & \ \text{for} \ x \geq 1
     \end{array}
   \right.
\]
\[
f(1) = \answer{3}
\]
\[
\lim_{x \to 1^-} f(x) = \answer{2}
\]
\[
\lim_{x \to 1^+} f(x) = \answer{3}
\]
Is $f(x)$ continuous at $x = 1$?
\begin{multipleChoice}
  \choice{Yes}
  \choice[correct]{No}
\end{multipleChoice}
\end{problem}





\begin{example}[example 7]
Is the function define below continuous at $x = 2$?
\[f(x) = \left\{
     \begin{array}{lr}
       x-1 & \ \text{for} \  x \leq 2 \\
       x^2 - 3 & \ \text{for} \ x > 2
     \end{array}
   \right.
\]

We need to  compute $f(2)$ and the one-sided limits as $x$ approaches 2.
If all three of these are equal, then $f(x)$ is continuous at $x=2$.
Otherwise, $f(x)$ has a discontinuity at $x=2$.\\
First, 
\[
f(2) = 2-1 = 1.
\]
Next, 
\[
\lim_{x \to 2^-} f(x) = \lim_{x \to 2^-} x-1 = 2-1 = 1,
\]
and finally,
\[
\lim_{x \to 2^+} f(x) = \lim_{x \to 2^+} x^2 - 3 = 2^2 - 3 = 1.
\]
Since all three of these values are equal, we can write
\[
\lim_{x \to 2} f(x) = f(2)
\]
and conclude that $f(x)$ is continuous at $x = 2$.
\end{example}


\begin{problem}(problem 7)
Determine whether the function $f(x)$ given below is continuous at $x = 1$.
\[
f(x) = \left\{
     \begin{array}{lr}
       2x & \ \text{for} \  x < 1 \\
       x^2 + 1 & \ \text{for} \ x \geq 1
     \end{array}
   \right.
\]
\[
f(1) = \answer{2}
\]
\[
\lim_{x \to 1^-} f(x) = \answer{2}
\]
\[
\lim_{x \to 1^+} f(x) = \answer{2}
\]
Is $f(x)$ continuous at $x = 1$?
\begin{multipleChoice}
  \choice[correct]{Yes}
  \choice{No}
\end{multipleChoice}
\end{problem}




\begin{example}[example 8]
Is the function $f(x)$ defined below continuous at $x = 2$?
\[
f(x) = \left\{
     \begin{array}{lr}
       x-1 & \ \text{for} \  x < 2 \\
			 3 & \ \text{for} \  x = 2 \\
       x^2 - 3 & \ \text{for} \ x > 2
     \end{array}
   \right.
\]


We need to  compute $f(2)$ and the one-sided limits as $x$ approaches 2.
If all three of these are equal, then $f(x)$ is continuous at $x=2$.
Otherwise, $f(x)$ has a discontinuity at $x=2$.\\
First, 
\[
f(2) = 3 \ \text{(given)}.
\]
Next, 
\[
\lim_{x \to 2^-} f(x) = \lim_{x \to 2^-} x-1 = 2-1 = 1
\]
and
\[
\lim_{x \to 2^+} f(x) = \lim_{x \to 2^+} x^2 - 3 = 2^2 - 3 = 1.
\]
Since the one-sided limits are equal, the two-sided limit exists:
\[
\lim_{x \to 2} f(x) = 1.
\]
However,
\[
\lim_{x \to 2} f(x) \neq f(2) = 3.
\]
Hence $f(x)$ is not continuous at $x = 2$.
This is an example of a removable discontinuity, since 
\[\lim_{x \to 2} f(x) \]
is a finite number.
\end{example}


\begin{problem}(problem 8)
Determine whether the function $f(x)$ given below is continuous at $x = 1$.
\[
f(x) = \left\{
     \begin{array}{lr}
       2x & \ \text{for} \  x < 1 \\
			 5 & \ \text{for} \  x = 1 \\
       x^2 + 1 & \ \text{for} \ x > 1
     \end{array}
   \right.
\]
\[
f(1) = \answer{5}
\]
\[
\lim_{x \to 1^-} f(x) = \answer{2}
\]
\[
\lim_{x \to 1^+} f(x) = \answer{2}
\]
Is $f(x)$ continuous at $x = 1$?
\begin{multipleChoice}
  \choice{Yes}
  \choice[correct]{No}
\end{multipleChoice}
\end{problem}







\begin{example}[example 9]
Determine the value of $k$ that will make the function $f(x)$ given below 
continuous at $x = 2$.
 
\[f(x) = \left\{
     \begin{array}{lr}
       kx-5 & \ \text{for} \  x \leq 2 \\
			 x^2 - k & \ \text{for} \ x > 2
     \end{array}
   \right.
\]

We need to  compute $f(2)$ and the one-sided limits as $x$ approaches 2.
We then set these expressions equal to each other and solve for $k$.\\
First,
\[
f(2) = 2k-5.
\]
Next,
\[
\lim_{x \to 2^-} f(x) = \lim_{x \to 2^-} kx-5 = 2k-5,
\]
and
\[
\lim_{x \to 2^+} f(x) = \lim_{x \to 2^+} x^2 - k = 2^2 - k = 4 - k.
\]
Setting these expressions equal to one another gives the equation:
\[2k - 5 = 4 - k.\]
Solving for $k$ gives
\[3k = 9\]
and hence,
\[k = 3.\]
This value of $k$ will make the function $f(x)$ continuous at $x=2$.
\end{example}

\begin{problem}(problem 9)
Determine the value of $k$ that will make the function $f(x)$ given below 
continuous at $x = 1$.
 
\[f(x) = \left\{
     \begin{array}{lr}
        x^2 - k & \ \text{for} \  x < 1 \\
			 kx-3 & \ \text{for} \ x \geq 1
     \end{array}
   \right.
\]

\[
f(1) = \answer{k-3}
\]
\[
\lim_{x \to 1^-} f(x) = \answer{1-k}
\]
\[
\lim_{x \to 1^+} f(x) = \answer{k-3}
\]
The value of $k$ that makes $f(x)$ continuous at $x = 1$ is
$k = \answer{2}$
\end{problem}


\begin{center}
\textbf{Continuity of Familiar Functions}
\end{center}

Consider the function $f(x) = x^2$ and any number $a$. We can compute the limit 
\[\lim_{x \to a} x^2 = a^2\]
by plugging in. Therefore, we can say that $f(x) = x^2$ is continuous for all real numbers.
We can also say that $f(x) = x^2$ is continuous on the interval $(-\infty, \infty)$.
Actually, this is true for all polynomials, including constant functions.
If $p(x)$ is a polynomial, then $p(x)$ is continuous on the interval $(-\infty, \infty)$.\\
There are some other familiar functions which are also continuous on the interval $(-\infty, \infty)$.
These are: 
\[f(x) = \sin(x), \cos(x), e^x, \tan^{-1}(x), \sqrt[3] x \ \text{and} \ |x|.\]
The function $f(x) = \ln(x)$ is only defined for $x$ in the interval $(0, \infty)$ and it is 
continuous on this interval.\\
The function $f(x) = \tan(x)$ has vertical asymptotes at odd multiples of $\frac{\pi}{2}$. 
It is continuous between these vertical asymptotes, so, for example, $f(x) = \tan(x)$ is continuous on the 
interval $(-\frac{\pi}{2},\frac{\pi}{2})$.
The function $f(x) = \sec(x)$ is similar to $\tan(x)$ in that it has vertical asymptotes at odd 
multiples of $\frac{\pi}{2}$, and it is continuous on the intervals between them.
The functions $\cot(x)$ and $\csc(x)$ have vertical asymptotes at multiples of $\pi$ and like 
$\tan(x)$ and $\sec(x)$, they are continuous between their asymptotes. For example, the function
$f(x) = \cot(x)$ is continuous on the interval $(0, \pi)$.\\



The function $f(x) = \sqrt x$ is only defined for $x \geq 0$ and it is continuous for all of these values of $x$.
In other words, $f(x) = \sqrt x$ is continuous on the interval $[0, \infty)$. 
To say that the function is continuous at the left endpoint of this interval ($x = 0$) it is sufficient 
that the function is right continuous at this point. And it is indeed true that
\[\lim_{x \to 0^+} \sqrt x = \sqrt 0 = 0.\]
In general, a root function, $f(x) = \sqrt[n] x$ is continuous on the interval $[0, \infty)$ if $n$ is even and 
it is continuous on the interval $(-\infty, \infty)$ if $n$ is odd.\\

The last type of familiar function that we will discuss here is the rational function.  
A rational function is a ratio of polynomials, 
\[f(x) = \frac{p_1(x)}{p_2(x)}, \]
where $p_1(x)$ and $p_2(x)$ are both polynomials and the degree of $p_2(x)$ is at least 1.
Such a function is continuous for all values of $x$ such that $p_2(x) \neq 0$.
For example, the function
\[f(x) = \frac{x}{x^2 + 1}\]
is continuous on the interval $(-\infty, \infty)$ since $x^2 + 1 \neq 0$ for any $x$.
On the other hand, the function
\[f(x) = \frac{x}{x^2 - 1}\]
is continuous on the intervals $(-\infty, -1), (-1, 1)$ and $(1, \infty)$ since $ x^2 - 1 = 0$
when $x = \pm 1$.



\begin{center}
\textbf{Properties of Continuity}
\end{center}



Continuous functions combine nicely with respect to the operations addition, subtraction, multiplication, 
division and composition. Specifically, if $f(x)$ and $g(x)$ are both continuous at $x = a$, then so are
\[f(x) + g(x),\]
\[f(x) - g(x) \ \text{and}\]
\[f(x) \cdot g(x).\]
Furthermore, if $g(a) \neq 0$ then 
\[\frac{f(x)}{g(x)}\]
is also continuous at $x = a$. 
In words, we say that the sum, difference, product and quotient of continuous functions is continuous 
(with the understanding that $g(a) \neq 0$ in the case of the quotient.)
Things are slightly more complicated for the composition. If $g(x)$ is continuous at $x = a$ and $f(x)$ is 
continuous at $x = g(a)$ then then composition $f(g(x))$ is continuous at $x = a$. This situation is 
different from the four basic operations because in composition, when plugging in $x = a$, we plug $a$ 
into $g(x)$ and then plug $g(a)$ into $f(x)$,
whereas for the first four basic operations we plug $x = a$ into both $f(x)$ and $g(x)$.



 
\begin{center}
\begin{foldable}
\unfoldable{Here is a detailed, lecture style video on discontinuities:}
\youtube{3T5jQUiW5FY}
\end{foldable}
\end{center}

\end{document}


\begin{center}
\begin{tikzpicture}
\begin{axis}[axis lines = center, title={Crazy Function}]
\addplot[domain=-9:-4.05, color=blue]{-1/(-x-4)^.5};
\addplot[domain=-3.99:-2.1, color=blue]{-2 + 1/(x+4)^.5};
\addplot[domain=-1.9:-0.1, color=blue]{2-x^2};
\addplot[domain=0.1:1.8, color=blue]{2 + 1/(2-x)^.5};
\addplot[domain=2.2:3.9, color=blue]{1/4 -1/(x-2)^.5};
\addplot[domain=4.1:5.9, color=blue]{x/2 -2};
\addplot[domain=6.1:9, color=blue]{1+(x-6)^2};
\vasymptote {-4}
\vasymptote {2}
\end{axis}
\end{tikzpicture}
\end{center}


\begin{tikzpicture}[yscale=.8]
\draw [help lines, <->] (-6,0) -- (6,0);
\draw [help lines, <->] (0,-4) -- (0,4);
\draw [green,domain=0:2*pi] plot (\x, {(2*sin(\x r)* ln(\x+1))/2});
\draw [red,domain=-pi:pi/2] plot (\x, {3*sin(\x r)});
\draw [blue, domain=-2*pi:2*pi] plot (\x, {cos(\x r)*exp(\x/exp(2*pi))});
\end{tikzpicture}




%from the tikz manual, page 33
\begin{tikzpicture}[scale=3]
\clip (-2,-0.2) rectangle (2,0.8); − 1
2
\draw[step=.5cm,gray,very thin] (-1.4,-1.4) grid (1.4,1.4);
\filldraw[fill=green!20,draw=green!50!black] (0,0) -- (3mm,0mm) arc
(0:30:3mm) -- cycle;
\draw[->] (-1.5,0) -- (1.5,0) coordinate (x axis);
−1
\draw[->] (0,-1.5) -- (0,1.5) coordinate
(y axis);
\draw (0,0) circle (1cm);
\draw[very thick,red]
(30:1cm) -- node[left=1pt,fill=white] {$\sin \alpha$} (30:1cm |- x axis);
\draw[very thick,blue]
(30:1cm |- x axis) -- node[below=2pt,fill=white] {$\cos \alpha$} (0,0);
\draw[very thick,orange] (1,0) -- node [right=1pt,fill=white]
{$\displaystyle \tan \alpha \color{black}=
\frac{{\color{red}\sin \alpha}}{\color{blue}\cos \alpha}$}
(intersection of 0,0--30:1cm and 1,0--1,1) coordinate (t);
\draw (0,0) -- (t);
\foreach \x/\xtext in {-1, -0.5/-\frac{1}{2}, 1}
\draw (\x cm,1pt) -- (\x cm,-1pt) node[anchor=north,fill=white] {$\xtext$};
\foreach \y/\ytext in {-1, -0.5/-\frac{1}{2}, 0.5/\frac{1}{2}, 1}
\draw (1pt,\y cm) -- (-1pt,\y cm) node[anchor=east,fill=white] {$\ytext$};
\end{tikzpicture}






%from the tikz manual, page 160
\begin{tikzpicture}[domain=0:4]
\draw[very thin,color=gray] (-0.1,-1.1) grid (3.9,3.9);
\draw[->] (-0.2,0) -- (4.2,0) node[right] {$x$};
\draw[->] (0,-1.2) -- (0,4.2) node[above] {$f(x)$};
\draw[color=red]
plot[id=x]
function{x}
node[right] {$f(x) =x$};
\draw[color=blue]
plot[id=sin] function{sin(x)}
node[right] {$f(x) = \sin x$};
\draw[color=orange] plot[id=exp] function{0.05*exp(x)} node[right] {$f(x) = \frac{1}{20} \mathrm e^x$};
\end{tikzpicture}





%from the tikz manual, page 245
\begin{tikzpicture}[scale=2]
\shade[top color=blue,bottom color=gray!50] (0,0) parabola (1.5,2.25) |- (0,0);
\draw (1.05cm,2pt) node[above] {$\displaystyle\int_0^{3/2} \!\!x^2\mathrm{d}x$};
\draw[style=help lines] (0,0) grid (3.9,3.9)
[step=0.25cm]
(1,2) grid +(1,1);
\draw[->] (-0.2,0) -- (4,0) node[right] {$x$};
\draw[->] (0,-0.2) -- (0,4) node[above] {$f(x)$};
\foreach \x/\xtext in {1/1, 1.5/1\frac{1}{2}, 2/2, 3/3}
\draw[shift={(\x,0)}] (0pt,2pt) -- (0pt,-2pt) node[below] {$\xtext$};
\foreach \y/\ytext in {1/1, 2/2, 2.25/2\frac{1}{4}, 3/3}
\draw[shift={(0,\y)}] (2pt,0pt) -- (-2pt,0pt) node[left] {$\ytext$};
\draw (-.5,.25) parabola bend (0,0) (2,4) node[below right] {$x^2$};
\end{tikzpicture}



%from the tikz manual, page 276
\begin{tikzpicture}
\draw[gray,very thin] (-1.9,-1.9) grid (2.9,3.9)
[step=0.25cm] (-1,-1) grid (1,1);
\draw[blue] (1,-2.1) -- (1,4.1); % asymptote
\draw[->] (-2,0) -- (3,0) node[right] {$x(t)$};
\draw[->] (0,-2) -- (0,4) node[above] {$y(t)$};
\foreach \pos in {-1,2}
\draw[shift={(\pos,0)}] (0pt,2pt) -- (0pt,-2pt) node[below] {$\pos$};
\foreach \pos in {-1,1,2,3}
\draw[shift={(0,\pos)}] (2pt,0pt) -- (-2pt,0pt) node[left] {$\pos$};
\fill (0,0) circle (0.064cm);
\draw[thick,parametric,domain=0.4:1.5,samples=200]
% The plot is reparameterised such that there are more samples
% near the center.
plot[id=asymptotic-example] function{(t*t*t)*sin(1/(t*t*t)),(t*t*t)*cos(1/(t*t*t))}
node[right] {$\bigl(x(t),y(t)\bigr) = (t\sin \frac{1}{t}, t\cos \frac{1}{t})$};
\fill[red] (0.63662,0) circle (2pt)
node [below right,fill=white,yshift=-4pt] {$(\frac{2}{\pi},0)$};
\end{tikzpicture}




%from the tikz manual, page 301
\begin{pgfpicture}
% Half-parabola going ‘‘up and right’’
\pgfpathmoveto{\pgfpointorigin}
\pgfpathparabola{\pgfpointorigin}{\pgfpoint{2cm}{4cm}}
\color{red}
\pgfusepath{stroke}
% Half-parabola going ‘‘down and right’’
\pgfpathmoveto{\pgfpointorigin}
\pgfpathparabola{\pgfpoint{-2cm}{4cm}}{\pgfpointorigin}
\color{blue}
\pgfusepath{stroke}
% Full parabola
\pgfpathmoveto{\pgfpoint{-2cm}{2cm}}
\pgfpathparabola{\pgfpoint{1cm}{-1cm}}{\pgfpoint{2cm}{4cm}}
\color{orange}
\pgfusepath{stroke}
\end{pgfpicture}


\includegraphics{standalone01.pdf}

\includegraphics{galaxy.png}



%IVT image from mooculus
%\begin{image}
\begin{tikzpicture}
\begin{axis}[
            domain=0:6, ymin=0, ymax=2.2,xmax=6,
            axis lines =left, xlabel=$x$, ylabel=$y$,
            every axis y label/.style={at=(current axis.above origin),anchor=south},
            every axis x label/.style={at=(current axis.right of origin),anchor=west},
            xtick={1,3.597,5}, ytick={.203,1,1.679},
            xticklabels={$a$,$u$,$b$}, yticklabels={$f(a)$,$r$,$f(b)$},
            axis on top,
          ]
          \addplot [draw=none, fill=white, domain=(0:7)] {1.679} \closedcycle;
          \addplot [draw=none, fill=background, domain=(0:7)] {.203} \closedcycle;
          \addplot [textColor,dashed] plot coordinates {(0,1.679) (6,1.679)};
          \addplot [textColor,dashed] plot coordinates {(0,.203) (6,.203)};
          \addplot [textColor,dashed] plot coordinates {(5,0) (5,1.679)};
          \addplot [textColor,dashed] plot coordinates {(1,0) (1,.203)};
          \addplot [textColor,dashed] plot coordinates {(3.587,0) (3.597,1)};
          \addplot [blue,domain=(0:6)] {1};
          \addplot [very thick,black, smooth,domain=(0:2.5)] {sin(deg((x - 4)/2)) + 1.2};
          \addplot [very thick,black, smooth,domain=(4:6)] {sin(deg((x - 4)/2)) + 1.2};
          \addplot [very thick,dashed,black, smooth,domain=(2.5:4)] {sin(deg((x - 4)/2)) + 1.2}; 
          \addplot [color=blue,fill=blue,only marks,mark=*] coordinates{(3.587,1)};  %% closed hole          
          \addplot [color=black,fill=black,only marks,mark=*] coordinates{(1,.203)};  %% closed hole          
          \addplot [color=black,fill=black,only marks,mark=*] coordinates{(5,1.679)};  %% closed hole          
        \end{axis}
\end{tikzpicture}
%\end{image}

\end{document}
























































































Consider the function 
\[f(x) = \left\{
     \begin{array}{lr}
       kx^2 -8 & \ \text{for} \  x \leq 2 \\
			 x^2 - kx & \ \text{for} \ x > 2
     \end{array}
   \right.
\]
If we choose the parameter $k$ correctly, we the function will be continuous at $x = 2$.
To determine the correct value of $k$ we compare the one-sided limits:
\[\lim_{x \to 2^-} f(x) = \lim_{x \to 2^-} kx^2 -8 = 4k -8 \]
and
\[\lim_{x \to 2^+} f(x) = \lim_{x \to 2^+} x^2 - kx = 4 - 2k.\]
In order for the function to be continuous, these need to be the same. 
Hence, we set them equal to each other and we solve for $k$:
\[4k -8 = 4 - 2k\]
\[6k = 12\]
\[k = 2.\]
With this choice of $k$, the function becomes
\[f(x) = \left\{
     \begin{array}{lr}
       2x^2 - 8 & \ \text{for} \  x \leq 2 \\
			 x^2 - 2x & \ \text{for} \ x > 2
     \end{array}
   \right.
\]
and it is continuous at $x = 2$ because
\[\lim_{x \to 2} f(x) = f(2) = 0. \]

\end{description}




\begin{center}
{\bf Intermediate Value Theorem}
\end{center}


In this section we discuss an important theorem related to continuous functions. Before we present the theorem, 
lets consider two real life situations and observe an important difference in their behavior. 
First, consider the ambient temperature as a  and second, consider the amount of money in a bank account, 
both as a functions of time.

First, suppose that the temperature is $65^{\circ}$ at 8am and then suppose it is $75^{\circ}$ at noon. 
Because of the continuous nature of temperature variation, we can be sure that at some time 
between 8am and noon  the temperature was exactly $70^{\circ}$.
Can we make a similar claim about money in a bank account?  
Suppose the account has \$65  in it at 8am and then it has \$75 in it at noon.  
Did it have exactly \$70 in it at some time between 8 am and noon? 
We cannot answer that question with any certainty from the given information.  
On one hand, it is possible that a \$10 deposit was made at 11am and so the total in the 
bank would have jumped from \$65 dollars to \$75 without ever being exactly \$70. 
On the other hand, it is possible that the \$10 was added in \$5 increments, 
like \$5 at 10 am and another \$5 at 11 am.   In this case, the account did have 
exactly \$70 in it at some time, namely 10am. Again, in the first case, we can say with 
certainty that the temperature was exactly 70 degrees at some time between 8am and noon but in 
the second case we cannot be sure whether there was ever exactly \$70 dollars in the account between 8am and noon. 
The fundamental reason why we can make certain conclusions in the first case but cannot in the second, 
is that temperature varies continuously, whereas money in a bank account does not.  
When a quantity is known to vary continuously, then if the quantity is observed to have different 
values at different times then we can conclude that the quantity took on any value between these 
two at some time between our two observations. Mathematically, this property is stated in the 
Intermediate Value Theorem:
Suppose a function $f(x)$ is continuous on an interval $[a, b]$ and suppose $I$ is a value between $f(a)$ and $f(b)$.
Then there is some input, $c$ between $a$ and $b$ such that $f(c) = I$.  The value $I$ is called an intermediate value.
The IVT therefore can be re-stated by saying that a continuous function takes on all of its intermediate values on an interval.

If a function is not continuous on an interval, then we cannot be sure whether or not it takes on a 
particular intermediate value.

\begin{center}
\bf{Examples of the Intermediate Value Theorem}
\end{center}

\begin{description}

\item[IV 1.] To show that the function $f(x) = x^3$ takes on the value 10 somewhere between $x = 2$ and $x = 3$,
we observe that since the function is a polynomial, it is continuous on the 
interval $(-\infty, \infty)$ and hence it is continuous on the sub-interval, $[2, 3]$. Next, observe that $f(2) = 2^3 = 8$
and $f(3) = 3^3 = 27$ so that 10 is an intermediate value:
\[8 < 10 < 27.\]
All of the hypotheses of the Intermediate Value Theorem are satisfied so we can now 
conclude that there exists a value $c$ between 2 and 3 such that $f(c) = c^3 =  10$.

\item[IV 2.] To show that the function $f(x) = e^x$ takes on the value 2 somewhere between $x = 0$ and $x = 1$,
we observe that the function $f(x) = e^x$ is continuous on the 
interval $(-\infty, \infty)$ and hence it is continuous on the sub-interval, $[0, 1]$. Next, observe that $f(0) = e^0 = 1$
and $f(1) = e^1 = e \approx 2.7$ so that 2 is an intermediate value:
\[1 < 2 < e.\]
All of the hypotheses of the Intermediate Value Theorem are satisfied so we can now 
conclude that there exists a value $c$ between 0 and 1 such that $f(c) = e^c = 2$.

\item[IV 3.] To show that the function $f(x) = x^4 - 3x - 7$ has a root in the interval $(-2, 1)$ we use the IVT.
Note that having a root means that $f(x)$ takes on the value 0. Since $f(x)$ is a polynomial, it is continuous on the interval 
$[-2, 1]$. Plugging in the endpoints shows that 0 is an intermediate value:
\[f(-2) = (-2)^4 - 3(-2) - 7 = 16 + 6 - 7 = 15 > 0\]
and
\[f(1) = (1)^4 - 3(1) - 7 = 1 -3 - 7 = -9 < 0.\]
By the IVT, we can conclude that there exists a value $c$ between $x = -2$ and $x = 1$ such that $f(c) = 0$,
i.e., $f(x)$ has a root in the interval $(-2, 1)$.

\item[IV 4.] We can use the IVT to prove that the equation $\cos(x) = x$ has a solution 
between $x = 0$ and $x = \frac{\pi}{2}$. Let 
\[f(x) = x - \cos(x).\] 
 Then $f(x)$ is the difference between two functions
which are each continuous on the interval $(-\infty, \infty)$. Hence $f(x)$ is also continuous on the interval
$(-\infty, \infty)$.  For our purposes, it is sufficient to note that $f(x)$ is continuous on the
interval $[0, \frac{\pi}{2}]$. Next we compute $f(0)$ and $f(\frac{\pi}{2})$ and show that 0 is an intermediate value:
\[f(0) = 0 - \cos(0) = 0-1 = -1 < 0 \]
and
\[f\big(\ffrac{\pi}{2}\big) = \frac{\pi}{2} - \cos\big(\ffrac{\pi}{2}\big) = \ffrac{\pi}{2}-0 = \ffrac{\pi}{2} > 0. \]
By the IVT, there exists a value $c$ in the interval $(0, \frac{\pi}{2})$ such that $f(c) = 0$. Finally, since $f(c) = 0$,
we have $c - \cos(c) = 0$ which can be rewritten $\cos(c) = c$.


\item[IV 5.] Repeated use of the IVT can help us to approximate solutions to equations. 
In this example, we will use the Bisection Method to approximate a root of a polynomial. Let 
\[p(x) = x^3 - 3x^2 + 4x - 6.\]
We would like to approximate a root of $p(x)$.  We begin by noting that 
\[p(0) = -6 < 0 \ \text{and} \ p(8) = 346 > 0.\]
Since polynomials are continuous for all values of $x$, we can apply the IVT to a polynomial function on any interval.
In this case, the IVT tells us that $p(x)$ has a root somewhere in the interval $(0, 8)$. 
To implement the Bisection Method, we now determine the midpoint of this interval: $(0+8)/2 = 4$ and we 
plug this in to $p(x)$:
\[p(4) =  14 > 0.\]
Combining this with
\[p(0) < 0,\]
we can use the IVT to conclude that $p(x)$ has a root on the interval $(0, 4)$. 
This interval is half of the original interval- the original interval has been bisected. 
We can do this again. The midpoint of the interval $(0, 4)$ is $x=2$. We plug this into $p(x)$:
\[p(2) =  -2 > 0.\]
Combining this with
\[p(4) > 0,\]
we can use the IVT to conclude that $p(x)$ has a root on the interval $(2,4)$. 
This interval is half of the previous interval- the previous interval has been bisected. 
We do this once again, noting that $x = 3$ is the midpoint of the interval $(2,4)$:
\[p(3) = 6 > 0.\]
Combining this with
\[p(2) < 0,\]
we can use the IVT to conclude that $p(x)$ has a root on the interval $(2, 3)$. 
Our approximation of the root is $x = 2.5$ which is the midpoint of this interval. 
Our error is then no more than $0.5$ units, which is half the width of the interval.
We will stop here, but the method could theoretically be continued indefinitely giving a better 
approximation to the root each time.



\end{description}


testing...


%\hspace{.5 in}

%\begin{tikzpicture}
%\begin{axis}[axis lines = center, title={A Jump Discontinuity at $x=-1$}]
%\addplot[domain=-3:-1.02, 
%    samples=100, color=blue]{x+3};
%\addplot[smooth,mark=o,blue] plot coordinates {(-1,2)};
%\addplot[domain=-0.99:1, 
%    samples=100, color=blue]{-x^2 + 2};
%\addplot[smooth,mark=o,blue] plot coordinates {(-1,1)};
%\end{axis}
%\end{tikzpicture}



\begin{center}
\begin{tikzpicture}
\begin{axis}[axis x line=middle, axis y line= middle, xlabel={$x$}, ylabel={$y$}, axis equal, xtick={.75, 1.5},
xticklabels={$x_1$, $x_2$}, ytick={1.682, 2.828}, yticklabels={$f(x_1)$, $f(x_2)$}, title={An Increasing Function}]

\addplot[domain=-1:2]{2^x};
\addplot[dashed] coordinates{(0.75, 0)  (0.75, 1.682)};
\addplot[dashed] coordinates{ (0.75, 1.682) (0, 1.682)};
\addplot[dashed] coordinates{(1.5, 0) (1.5, 2.828)};
\addplot[dashed] coordinates{(1.5, 2.828) (0, 2.828)};

\end{axis}
%\addplot[mark=*,fill=white] coordinates {(0,0)}
\node at (4, -1) {$x_1 < x_2 \implies f(x_1) < f(x_2)$};				
\node at (6.5, 5) {$y = f(x)$};
 

\end{tikzpicture}
\end{center}



\begin{tikzpicture}
\begin{axis}[
            domain=-2:2,
            width=6in,
            height=3in,
            axis lines =middle, xlabel=$x$, ylabel=$y$,
            every axis y label/.style={at=(current axis.above origin),anchor=south},
            every axis x label/.style={at=(current axis.right of origin),anchor=west},
          ]
 \addplot [very thick, blue, smooth] {x^3};
        \end{axis}
\end{tikzpicture}



%Below is an example of the graph of a function that has an infinite discontinuity.

\begin{center}
\begin{tikzpicture}
\begin{axis}[axis lines = center, title={An infinite discontinuity at $x=1$}]
\addplot[domain=-1:0.9, 
    samples=100, color=blue]{1/(x-1)^2};
\addplot[domain=1.1:3, color=blue]{1/(x-1)^2};
\vasymptote {1}
\end{axis}
\end{tikzpicture}
\end{center}

%The graphs of these functions are shown below.

\begin{center}
\begin{tikzpicture}
\begin{axis}[axis lines = center, title={reciprocal}]
\addplot[domain=-2:-0.1, color=blue]{1/x};
\addplot[domain=0.1:2, color=blue]{1/x};
%\vasymptote {1}
\end{axis}
\end{tikzpicture}
\end{center}

\begin{center}
\begin{tikzpicture}
\begin{axis}[axis lines = center, title={tangent}]
\addplot[domain=-1:1.56, color=blue]{tan(deg(x))};
\addplot[domain=1.58:4, color=blue]{tan(deg(x))};
\vasymptote {1.57}
\end{axis}
\end{tikzpicture}
\end{center}


\begin{center}
\begin{tikzpicture}
\begin{axis}[axis lines = center, title={$y = \tan(x)$}]
\addplot[domain=-4.7:-1.58, color=blue]{tan(deg(x))};
\addplot[domain=-1.57:1.57, color=blue]{tan(deg(x))};
\addplot[domain=1.58:4.71, color=blue]{tan(deg(x))};
\vasymptote {1.57}
\vasymptote {-1.57}
\vasymptote {4.72}
\vasymptote {-4.72}
\end{axis}
\end{tikzpicture}
\end{center}
