\documentclass{ximera}

%% You can put user macros here
%% However, you cannot make new environments



\newcommand{\ffrac}[2]{\frac{\text{\footnotesize $#1$}}{\text{\footnotesize $#2$}}}
\newcommand{\vasymptote}[2][]{
    \draw [densely dashed,#1] ({rel axis cs:0,0} -| {axis cs:#2,0}) -- ({rel axis cs:0,1} -| {axis cs:#2,0});
}


\graphicspath{{./}{firstExample/}}

\usepackage{amsmath}
\usepackage{amssymb}
\usepackage{array}
\usepackage[makeroom]{cancel} %% for strike outs
\usepackage{pgffor} %% required for integral for loops
\usepackage{tikz}
\usepackage{tikz-cd}
\usepackage{tkz-euclide}
\usetikzlibrary{shapes.multipart}


\usetkzobj{all}
\tikzstyle geometryDiagrams=[ultra thick,color=blue!50!black]


\usetikzlibrary{arrows}
\tikzset{>=stealth,commutative diagrams/.cd,
  arrow style=tikz,diagrams={>=stealth}} %% cool arrow head
\tikzset{shorten <>/.style={ shorten >=#1, shorten <=#1 } } %% allows shorter vectors

\usetikzlibrary{backgrounds} %% for boxes around graphs
\usetikzlibrary{shapes,positioning}  %% Clouds and stars
\usetikzlibrary{matrix} %% for matrix
\usepgfplotslibrary{polar} %% for polar plots
\usepgfplotslibrary{fillbetween} %% to shade area between curves in TikZ



%\usepackage[width=4.375in, height=7.0in, top=1.0in, papersize={5.5in,8.5in}]{geometry}
%\usepackage[pdftex]{graphicx}
%\usepackage{tipa}
%\usepackage{txfonts}
%\usepackage{textcomp}
%\usepackage{amsthm}
%\usepackage{xy}
%\usepackage{fancyhdr}
%\usepackage{xcolor}
%\usepackage{mathtools} %% for pretty underbrace % Breaks Ximera
%\usepackage{multicol}



\newcommand{\RR}{\mathbb R}
\newcommand{\R}{\mathbb R}
\newcommand{\C}{\mathbb C}
\newcommand{\N}{\mathbb N}
\newcommand{\Z}{\mathbb Z}
\newcommand{\dis}{\displaystyle}
%\renewcommand{\d}{\,d\!}
\renewcommand{\d}{\mathop{}\!d}
\newcommand{\dd}[2][]{\frac{\d #1}{\d #2}}
\newcommand{\pp}[2][]{\frac{\partial #1}{\partial #2}}
\renewcommand{\l}{\ell}
\newcommand{\ddx}{\frac{d}{\d x}}

\newcommand{\zeroOverZero}{\ensuremath{\boldsymbol{\tfrac{0}{0}}}}
\newcommand{\inftyOverInfty}{\ensuremath{\boldsymbol{\tfrac{\infty}{\infty}}}}
\newcommand{\zeroOverInfty}{\ensuremath{\boldsymbol{\tfrac{0}{\infty}}}}
\newcommand{\zeroTimesInfty}{\ensuremath{\small\boldsymbol{0\cdot \infty}}}
\newcommand{\inftyMinusInfty}{\ensuremath{\small\boldsymbol{\infty - \infty}}}
\newcommand{\oneToInfty}{\ensuremath{\boldsymbol{1^\infty}}}
\newcommand{\zeroToZero}{\ensuremath{\boldsymbol{0^0}}}
\newcommand{\inftyToZero}{\ensuremath{\boldsymbol{\infty^0}}}


\newcommand{\numOverZero}{\ensuremath{\boldsymbol{\tfrac{\#}{0}}}}
\newcommand{\dfn}{\textbf}
%\newcommand{\unit}{\,\mathrm}
\newcommand{\unit}{\mathop{}\!\mathrm}
%\newcommand{\eval}[1]{\bigg[ #1 \bigg]}
\newcommand{\eval}[1]{ #1 \bigg|}
\newcommand{\seq}[1]{\left( #1 \right)}
\renewcommand{\epsilon}{\varepsilon}
\renewcommand{\iff}{\Leftrightarrow}

\DeclareMathOperator{\arccot}{arccot}
\DeclareMathOperator{\arcsec}{arcsec}
\DeclareMathOperator{\arccsc}{arccsc}
\DeclareMathOperator{\si}{Si}
\DeclareMathOperator{\proj}{proj}
\DeclareMathOperator{\scal}{scal}
\DeclareMathOperator{\cis}{cis}
\DeclareMathOperator{\Arg}{Arg}
%\DeclareMathOperator{\arg}{arg}
\DeclareMathOperator{\Rep}{Re}
\DeclareMathOperator{\Imp}{Im}
\DeclareMathOperator{\sech}{sech}
\DeclareMathOperator{\csch}{csch}
\DeclareMathOperator{\Log}{Log}

\newcommand{\tightoverset}[2]{% for arrow vec
  \mathop{#2}\limits^{\vbox to -.5ex{\kern-0.75ex\hbox{$#1$}\vss}}}
\newcommand{\arrowvec}{\overrightarrow}
\renewcommand{\vec}{\mathbf}
\newcommand{\veci}{{\boldsymbol{\hat{\imath}}}}
\newcommand{\vecj}{{\boldsymbol{\hat{\jmath}}}}
\newcommand{\veck}{{\boldsymbol{\hat{k}}}}
\newcommand{\vecl}{\boldsymbol{\l}}
\newcommand{\utan}{\vec{\hat{t}}}
\newcommand{\unormal}{\vec{\hat{n}}}
\newcommand{\ubinormal}{\vec{\hat{b}}}

\newcommand{\dotp}{\bullet}
\newcommand{\cross}{\boldsymbol\times}
\newcommand{\grad}{\boldsymbol\nabla}
\newcommand{\divergence}{\grad\dotp}
\newcommand{\curl}{\grad\cross}
%% Simple horiz vectors
\renewcommand{\vector}[1]{\left\langle #1\right\rangle}


\outcome{Learn the Mean Value Theorem and apply it to examples}

\title{3.5 Mean Value Theorem}

\begin{document}

\begin{abstract}
We apply the Mean Value Theorem.
\end{abstract}

\maketitle

\section{Rolle's Theorem}
We begin with a special case of the Mean Value Theorem known as Rolle's Theorem.
This theorem uses the Extreme Value Theorem to guarantee a critical number of a differentiable function under certain circumstances.

\begin{theorem}[Rolle's Theorem]
Suppose the function $f(x)$ is continuous on the closed interval $[a,b]$ and differentiable on the open interval $(a,b)$.
If $f(a) = f(b)$, then there exists a number $c$ between $a$ and $b$ such that 
\[
f'(c) = 0.
\]
\end{theorem}

\begin{image}
\begin{tikzpicture}
\begin{axis}[axis x line=  center, axis y line = none, xmin= -2.5, xmax=2.5, xtick={-2, 0, 2}, xticklabels={$a$,$c$,$b$},
title={Rolle's Theorem for $f(x) $ on $[a,b]$}]
\addplot[domain=-2:2, 
    samples=100, color=black]{-0.25*x^2 + 1.5 };
\addplot[smooth,mark=*,blue] plot coordinates {(-2,0.5)  (2,0.5)} node[below, midway]{$f(a) = f(b)$};
\addplot[domain=-1.2:1.2, 
    samples=100, color=red]{1.5};
\addplot[smooth,mark=*,red] plot coordinates {(0, 1.5)} node[above] {$f'(c) = 0$};
\addplot[domain=-2.5:2.5, 
    samples=100, color=black]{0};
\end{axis}
\end{tikzpicture}
\end{image}
The rationale behind the theorem is that since the function is at the same height at the endpoints, the Extreme Value Theorem implies that
there must be a local extreme (and hence a critical number) between $x = a$ and $x = b$.

\begin{example} Show that the function $f(x) = \sin(x)$ has a critical number in the interval $(0, \pi)$.\\
The function $\sin(x)$ is differentiable (and hence also continuous) on the interval $(-\infty, \infty)$.
We can apply Rolle's Theorem to this function on the interval $[0, \pi]$ since
\[
\sin 0 = \sin \pi = 0.
\]
In this case, the conclusion of the theorem is that $f(x) = \sin(x)$ has a critical number in the interval $(0, \pi)$.
Moreover, we know extly where, since $f'(x) = \cos(x)$ and the equation $\cos(x) = 0$ has a solution at $x = \pi/2$, which is in the interval.
\end{example}

\begin{example} Use Rolle's Theorem  to show that a cubic polynomial can have at most 3 roots.\\
Recall that a root of a polynomial, $p(x)$, is a value $x = a$, such that $p(a) = 0$.
So we would like to show that an equation of the form $p(x) = 0$, where $p(x)$ is a cubic polynomial, 
\[
p(x) = ax^3 + bx^2 + cx + d, \;\; \text{where} \;\; a\neq 0.
\]
Also recall that a quadratic polynomial has at most two roots, since if
\[
ax^2 + bx + c = 0
\]
then
\[
x = \frac{-b \pm \sqrt{b^2 - 4ac}}{2a}.
\]
Now we can argue by way of contradiction that a cubic polynomial can have at most 3 roots.
Suppose that the cubic polynomial, $p(x)$, has 4 roots, $x_1, x_2, x_3$ and $x_4$, where these are in order from smallest to largest.
Then, by definition of root,
\[
p(x_1) = p(x_2) = p(x_3) = p(x_4) = 0.
\]
We can apply Rolle's Theorem to $p(x)$ on the intervals $[x_1, x_2], [x_2, x_3]$ and $[x_3, x_4]$ to conclude
that $p'(x)$ has a root on each of these intervals, i.e.,
\[
p'(c_1) = p'(c_2) = p'(c_3) = 0,
\]
where
\[
x_1 < c_1 < x_2 < c_2 < x_3 < c_3 < x_4.
\]
\begin{image}
\begin{tikzpicture}
\begin{axis}[axis x line=  center, axis y line = none, xmin= -0.5, xmax=6.5, xtick={0, 1, 2, 3, 4, 5, 6}, 
xticklabels={$x_1$,$c_1$,$x_2$,$c_2$,$x_3$,$c_3$,$x_4$}, title={By Rolle's Theorem, $p'(x)$ has 3 roots}]
\addplot[domain=-0.5:6.5, 
    samples=100, color=black]{0};
\addplot [samples=100,smooth,domain=0:6] {sin(deg(1.57*x))} ;
\addplot[smooth,mark=*,blue] plot coordinates {(0,0)   (2,0)  (4,0) (6,0)} ;
\addplot[smooth,mark=*,red] plot coordinates {(1, 1)} ;
\addplot[smooth,mark=*,red] plot coordinates {(3, -1)} ;
\addplot[smooth,mark=*,red] plot coordinates {(5, 1)} ;
\addplot[domain=0.3:1.7, 
    samples=100, color=red]{1} node[above]{$p'(c_1) = 0$};
\addplot[domain=2.3:3.7, 
    samples=100, color=red]{-1} node[below]{$p'(c_2) = 0$};
\addplot[domain=4.3:5.7, 
    samples=100, color=red]{1} node[above]{$p'(c_3) = 0$};
\end{axis}
\end{tikzpicture}
\end{image}

But this is not possible. Since $p(x)$ is a cubic polynomial, its derivative, $p'(x)$ is a quadratic polynomial and so it has at most 2 roots.
This contradiction implies that a cubic polynomial can have at most 3 roots.
\end{example}




\section{The Mean Value Theorem}


The Mean Value Theorem is one of the most far-reaching theorems in calculus. It states that for a continuous 
and differentiable function, the average rate of change over an interval is attained as an 
instantaneous rate of change at some point inside the interval. The precise mathematical statement is as follows.\\

\begin{theorem}[Mean Value Theorem]
If $f(x)$ is continuous on the closed interval $[a,b]$ and differentiable on the 
open interval $(a,b)$, then the Mean Value Equation:
\[f'(c) = \frac{f(b) - f(a)}{b-a}\]
holds for some value $c$ between $a$ and $b$.\\
\end{theorem}




The left hand side of the Mean Value Theorem equation represents the instantaneous rate of change of $f(x)$ at $x = c$ and the
right hand side represents the average rate of change of $f(x)$ over the interval from $x=a$ to $x=b$.

\begin{image}
\begin{tikzpicture}
\begin{axis}[axis x line=  center, axis y line = none, xtick={-1, 0.5, 2}, xticklabels={$a$,$c$,$b$}, 
legend pos=outer north east, title={MVT for $f(x) $ on $[a,b]$}]
\addplot[domain=-1.4:2.4, 
    samples=100, color=black]{-x^2 + 5 };
\addplot[smooth,mark=*,blue] plot coordinates {(-1,4)  (2,1)};
\addplot[domain=-1:2, 
    samples=100, color=blue]{-x + 3};
\addplot[domain=-1:2, 
    samples=100, color=red]{-x + 5.25 };
\addplot[smooth,mark=*,red] plot coordinates {(0.5, 4.75)};
\legend{$y = f(x)$, , secant line, tangent line, };
\addplot[dashed] coordinates{(-1,0)(-1,4)};
\addplot[dashed] coordinates{(0.5,0)(0.5,4.75)};
\addplot[dashed] coordinates{(2,0)(2,1)};
node[label=below]{some label};
\end{axis}
\end{tikzpicture}
\end{image}








The Mean Value Equation is also frequently presented in the form:
\[f(b) - f(a)=f'(c)(b-a).\]

As a real life example, if a car averages 57.4 miles per hour on a long trip, then by the MVT the car must have been 
traveling at exactly 57.4 miles per hour at some instant during the trip.

The MVT is considered an existence theorem because it asserts that there exists a special value $c$ inside the interval $(a,b)$
that satisfies Mean Value Equation 
\[f'(c) = \frac{f(b) - f(a)}{b-a}.\]
In our examples, we will determine the exact value of $c$. 

\begin{example}[example 1]
We will verify that the function $f(x) = x^2$ satisfies the hypotheses of the MVT
on the interval $[0,3]$ and we will find the special value of $c$ that satisfies the Mean Value Equation.
Since $x^2$ is a polynomial function, it is continuous and differentiable on any interval. 
Hence, it is continuous on the closed interval $[0, 3]$ and differentiable on the open interval $(0, 3)$. 
Then, by the MVT,  the Mean Value Equation holds for some number 
$c$ in the interval $(0, 3)$. To find $c$, we first compute
\[\frac{f(b) - f(a)}{b-a},\]
then we compute $f'(x)$
and finally we solve the equation
\[f'(x) = \frac{f(b) - f(a)}{b-a}\]
for $x$, discarding any solutions that do not lie in the interval $(0, 3)$.  
Keep in mind that the MVT guarantees that we will find at least one solution in this interval.
Let's begin:
\[\frac{f(b) - f(a)}{b-a} = \frac{f(3) - f(0)}{3-0} \]
\[= \frac{3^2 - 0^2}{3}= \frac{9 - 0}{3} = 3.\]
Next, $f'(x) = 2x$
and finally, we solve the equation
\[2x = 3\]
which gives
\[x = \frac{3}{2}.\]
Note that the value $\frac32$ is in the interval $(0,3)$ and so the special value, $c$, guaranteed to exist by the MVT,
is $\frac32$ in this example.


\begin{image}
\begin{tikzpicture}
\begin{axis}[axis lines = center, legend pos=outer north east, title={MVT for $f(x) = x^2$ on $[0,3]$}]
\addplot[domain=0:3, 
    samples=100, color=black]{x^2};
\addplot[smooth,mark=*,blue] plot coordinates {(0,0)  (3,9)};
\addplot[domain=0:3, 
    samples=100, color=blue]{3*x};
\addplot[domain=0:3, 
    samples=100, color=red]{3*x - 2.25 };
\addplot[smooth,mark=*,red] plot coordinates {(1.5, 2.25)};
\legend{$y=x^2$, , \text{secant line}, \text{tangent line}, }
\end{axis}
\end{tikzpicture}
\end{image}

In the above figure, the blue line in the secant line for $f(x) = x^2$ on the interval $[0, 3]$, 
and the red line is the tangent line at $x = \frac32$. The Mean Value Equation asserts that these lines are parallel, and this
is clear in the figure.
\end{example}

\begin{problem}(problem 1a)
  Given that the function $f(x) = x^2 + 2$ satisfies the hypotheses of the MVT on the
	interval $[1,3]$, find the value of $c$ in the open interval $(1,3)$ which satisfies 
	the conclusion of the theorem.
	
    \begin{hint}
      Compute $f'(x)$ and $\dfrac{f(3) - f(1)}{3-1}$
    \end{hint}
		\begin{hint}
		  Solve $f'(x) = \dfrac{f(3) - f(1)}{3-1}$
		\end{hint}
		
		The value of $c$ is:
		 $\answer[given]{2}$
\end{problem}

\begin{problem}(problem 1b)
  Given that the function $f(x) = x^2 -3x + 5$ satisfies the hypotheses of the MVT on the
	interval $[-1,2]$, find the value of $c$ in the open interval $(-1,2)$ which satisfies 
	the conclusion of the theorem.
	
    \begin{hint}
      Compute $f'(x)$ and $\dfrac{f(2) - f(-1)}{2-(-1)}$
    \end{hint}
		\begin{hint}
		  Solve $f'(x) = \dfrac{f(2) - f(-1)}{2-(-1)}$
		\end{hint}
		
		The value of $c$ is:
		 $\answer[given]{1/2}$
\end{problem}

\begin{problem}(problem 1c)
  Given that the function $f(x) = 3x^2 -5x + 8$ satisfies the hypotheses of the MVT on the
	interval $[1,6]$, find the value of $c$ in the open interval $(1,6)$ which satisfies 
	the conclusion of the theorem.
	
    \begin{hint}
      Compute $f'(x)$ and $\dfrac{f(6) - f(1)}{6-1}$
    \end{hint}
		\begin{hint}
		  Solve $f'(x) = \dfrac{f(6) - f(1)}{6-1}$
		\end{hint}
		
		The value of $c$ is:
		 $\answer[given]{7/2}$
\end{problem}

\begin{problem}(problem 1d)
  Given that the function $f(x) = x^3 + 2x -9$ satisfies the hypotheses of the MVT on the
	interval $[-2,2]$, find the values of $c$ in the open interval $(-2,2)$ which satisfy 
	the conclusion of the theorem.
	
    \begin{hint}
      Compute $f'(x)$ and $\dfrac{f(2) - f(-2)}{2-(-2)}$
    \end{hint}
		\begin{hint}
		  Solve $f'(x) = \dfrac{f(2) - f(-2)}{2-(-2)}$
		\end{hint}
		
		The values of $c$ in ascending order are:
		 $\answer[given]{-2/\sqrt 3}$ and $\answer[given]{2/\sqrt 3}$
\end{problem}




\begin{example}[example 2]
We will verify that the function $f(x) = \sqrt x$ satisfies the hypotheses of the MVT
on the interval $[0,4]$ and we will find the special value of $c$ that satisfies the Mean Value Equation.
The function $\sqrt x$ is continuous on the interval $[0, \infty)$  and differentiable on the interval $(0, \infty)$. 
Hence, it is continuous on the closed interval $[0, 4]$ and differentiable on the open interval $(0, 4)$. 
Then, by the MVT,  the Mean Value Equation holds for some number 
$c$ in the interval $(0, 4)$. To find $c$, we first compute
\[\frac{f(b) - f(a)}{b-a} = \frac{\sqrt 4 - \sqrt 0}{4-0} = \frac{2}{4} = \frac{1}{2}.\]
Next we compute
\[f'(x) = \frac{d}{dx} \sqrt x = \frac{d}{dx} x^{1/2} = (1/2)x^{-1/2} = \frac{1}{2\sqrt x}.\]
Finally we solve the equation
\[\frac{1}{2\sqrt x} = \frac{1}{2}\]
which gives
\[\sqrt x = 1\]
and so 
\[ x=1.\]
Note that the value $x = 1$ is in the interval $(0,4)$ and so the special value, $c$, guaranteed to exist by the MVT,
is $1$ in this example.


\begin{image}
\begin{tikzpicture}
\begin{axis}[axis lines = center, legend pos=outer north east, title={MVT for $f(x) = \sqrt x$ on $[0,4]$}]
\addplot[domain=0:4, 
    samples=100, color=black]{sqrt(x)};
\addplot[smooth,mark=*,blue] plot coordinates {(0,0)  (4,2)};
\addplot[domain=0:3, 
    samples=100, color=blue]{0.5*x};
\addplot[domain=0:3, 
    samples=100, color=red]{0.5*x + 0.5 };
\addplot[smooth,mark=*,red] plot coordinates {(1, 1)};
\legend{$y=\sqrt x$, , secant line, tangent line, }
\end{axis}
\end{tikzpicture}
\end{image}

In the above figure, the blue line in the secant line for $f(x) = \sqrt x$ on the interval $[0, 4]$, 
and the red line is the tangent line at $x = 1$. The Mean Value Equation asserts that these lines are parallel, and this
is clear in the figure.
\end{example}

\begin{problem}(problem 2a)
  Given that the function $f(x) = \sqrt x$ satisfies the hypotheses of the MVT on the
	interval $[0,9]$, find the value of $c$ in the open interval $(0,9)$ which satisfies 
	the conclusion of the theorem.
	
    \begin{hint}
      Compute $f'(x)$ and $\dfrac{f(9) - f(0)}{9-0}$
    \end{hint}
		\begin{hint}
		  Solve $f'(x) = \dfrac{f(9) - f(0)}{9-0}$
		\end{hint}
		
		The value of $c$ is:
		 $\answer[given]{9/4}$
\end{problem}


\begin{problem}(problem 2b)
  Given that the function $f(x) = \sqrt x$ satisfies the hypotheses of the MVT on the
	interval $[4,16]$, find the value of $c$ in the open interval $(4,16)$ which satisfies 
	the conclusion of the theorem.
	
    \begin{hint}
      Compute $f'(x)$ and $\dfrac{f(16) - f(4)}{16-4}$
    \end{hint}
		\begin{hint}
		  Solve $f'(x) = \dfrac{f(16) - f(4)}{16-4}$
		\end{hint}
		
		The value of $c$ is:
		 $\answer[given]{9}$
\end{problem}


\begin{problem}(problem 2c)
  Given that the function $f(x) = \sqrt{2x+1}$ satisfies the hypotheses of the MVT on the
	interval $[0,4]$, find the value of $c$ in the open interval $(0,4)$ which satisfies 
	the conclusion of the theorem.
	
    \begin{hint}
      Compute $f'(x)$ and $\dfrac{f(4) - f(0)}{4-0}$
    \end{hint}
		\begin{hint}
		  Solve $f'(x) = \dfrac{f(4) - f(0)}{4-0}$
		\end{hint}
		
		The value of $c$ is:
		 $\answer[given]{3/2}$
\end{problem}



\begin{example}[example 3]
We will verify that the function $f(x) = e^x$ satisfies the hypotheses of the MVT
on the interval $[0,2]$ and we will find the special value of $c$ that satisfies the Mean Value Equation.
The function $e^x$ is continuous and differentiable on the interval $(-\infty, \infty)$. 
Hence, it is continuous on the closed interval $[0, 2]$ and differentiable on the open interval $(0, 2)$. 
Then, by the MVT,  the Mean Value Equation holds for some number 
$c$ in the interval $(0, 2)$. To find $c$, we first compute
\[\frac{f(b) - f(a)}{b-a} = \frac{e^2 - e^0}{2-0} = \frac{e^2 - 1}{2}.\]
Next we compute
\[f'(x) = \frac{d}{dx} e^x = e^x.\]
Finally we solve the equation
\[e^x = \frac{e^2 - 1}{2}\]
which gives
\[ x = \ln(\frac{e^2 - 1}{2}) \approx 1.16.\]

Note that the value $\ln(\frac{e^2 - 1}{2})$ is in the interval $(0,4)$ and so the special value, $c$, 
guaranteed to exist by the MVT,
is $\ln(\frac{e^2 - 1}{2})$ in this example.


\begin{image}
\begin{tikzpicture}
\begin{axis}[axis lines = center, legend pos=outer north east, title={MVT for $f(x) = e^x$ on $[0,2]$}]
\addplot[domain=0:2, 
    samples=100, color=black]{e^x};
\addplot[smooth,mark=*,blue] plot coordinates {(0,1)  (2,e^2)};
\addplot[domain=0:2, 
    samples=100, color=blue]{3.194528049*x + 1};
\addplot[domain=0:2, 
    samples=100, color=red]{3.194528049*x - 0.51572257};
\addplot[smooth,mark=*,red] plot coordinates {(1.161439362, 3.194528049)};
\legend{$y=e^x$, , secant line, tangent line, }
\end{axis}
\end{tikzpicture}
\end{image}

In the above figure, the blue line in the secant line for $f(x) = e^x$ on the interval $[0, 2]$, 
and the red line is the tangent line at $x = \ln(\frac{e^2 - 1}{2})$. 
The Mean Value Equation asserts that these lines are parallel, and this
is clear in the figure.
\end{example}


\begin{problem}(problem 3)
  Given that the function $f(x) = e^{2x}$ satisfies the hypotheses of the MVT on the
	interval $[0,1]$, find the value of $c$ in the open interval $(0,1)$ which satisfies 
	the conclusion of the theorem.
	
    \begin{hint}
      Compute $f'(x)$ and $\dfrac{f(1) - f(0)}{1-0}$
    \end{hint}
		\begin{hint}
		  Solve $f'(x) = \dfrac{f(1) - f(0)}{1-0}$
		\end{hint}
		
		The value of $c$ is:
		 $\answer[given]{(1/2) \ln((e^2 -1)/2)}$
\end{problem}


\begin{example}[example 4] %example 4
We will verify that the function $f(x) = \ln(x)$ satisfies the hypotheses of the MVT
on the interval $[1,e]$ and we will find the special value of $c$ that satisfies the Mean Value Equation.
The function $\ln(x)$ is continuous and differentiable on the interval $(0, \infty)$. 
Hence, it is continuous on the closed interval $[1, e]$ and differentiable on the open interval $(1, e)$. 
Then, by the MVT,  the Mean Value Equation holds for some number 
$c$ in the interval $(1, e)$. To find $c$, we first compute
\[\frac{f(b) - f(a)}{b-a} = \frac{\ln(e) - \ln(1)}{e-1} = \frac{1 - 0}{e-1} = \frac{1}{e-1}.\]
Next we compute
\[f'(x) = \frac{d}{dx} \ln(x) = \frac{1}{x}.\]
Finally we solve the equation
\[\frac{1}{x} = \frac{1}{e-1}\]
which gives
\[ x = e-1 \approx 1.72.\]

Note that the value $x = e-1$ is in the interval $(1,e)$ and so the special value, $c$, 
guaranteed to exist by the MVT,
is $e-1$ in this example.


\begin{image}
\begin{tikzpicture}
\begin{axis}[axis lines = center, legend pos=outer north east, title={MVT for $f(x) = \ln(x)$ on $[1,e]$}]
\addplot[domain=1:e, 
    samples=100, color=black]{ln(x)};
\addplot[smooth,mark=*,blue] plot coordinates {(1,0)  (e,1)};
\addplot[domain=1:e, 
    samples=100, color=blue]{0.581976707*x - 0.581976707};
\addplot[domain=1:e, 
    samples=100, color=red]{0.581976707*x - 0.458675145};
\addplot[smooth,mark=*,red] plot coordinates {(1.718281828, 0.541324855)};
\legend{$y=\ln(x)$, , secant line, tangent line, }
\end{axis}
\end{tikzpicture}
\end{image}

In the above figure, the blue line in the secant line for $f(x) = \ln(x)$ on the interval $[1, e]$, 
and the red line is the tangent line at $x = e-1$. 
The Mean Value Equation asserts that these lines are parallel, and this
is clear in the figure.
\end{example}

\begin{problem}(problem 4a) 
  Given that the function $f(x) = \ln(x)$ satisfies the hypotheses of the MVT on the
	interval $[1,e]$, find the value of $c$ in the open interval $(1,e)$ which satisfies 
	the conclusion of the theorem.
	
    \begin{hint}
      Compute $f'(x)$ and $\dfrac{f(e) - f(1)}{e-1}$
    \end{hint}
		\begin{hint}
		  Solve $f'(x) = \dfrac{f(e) - f(1)}{e-1}$
		\end{hint}
		
		The value of $c$ is:
		 $\answer[given]{e-1}$
\end{problem}
%example with |x| to show MVE does not always have a solution

\begin{problem}(problem 4b)
  Given that the function $f(x) = \sin(x)$ satisfies the hypotheses of the MVT on the
	interval $[0, \pi]$, find the value of $c$ in the open interval $(0, \pi)$ which satisfies 
	the conclusion of the theorem.
	
    \begin{hint}
      Compute $f'(x)$ and $\dfrac{f(\pi) - f(0)}{\pi - 0}$
    \end{hint}
		\begin{hint}
		  Solve $f'(c) = \dfrac{f(\pi) - f(0)}{\pi - 0}$
		\end{hint}
		
		The value of $c$ is:
		 $\answer[given]{\pi /2}$
\end{problem}

\begin{center}
\begin{foldable}
\unfoldable{Here is a detailed, lecture style video on the Mean Value Theorem:}
\youtube{suJx3pB_cVI}
\end{foldable}
\end{center}



\end{document}



























%\begin{image}
%\begin{tikzpicture}
%\begin{axis}[
 %           ymin=-5,
	%		ymax=5,
   %         axis lines =center, xlabel=$k$, ylabel=$P$,
    %          every axis y label/.style={at=(current axis.above origin),anchor=south},
     %         every axis x label/.style={at=(current axis.right of origin),anchor=west},
      %      domain=-5:5,
       %     grid = major,
        %    xtick={-4,...,4},
         %   ytick={-4,...,4},
          %]
          %\addplot [very thick, smooth] {1 + (3.5 + x)*(-0.5714285714285714 + (-3.5 + x)*(0.16326530612244897 + (-0.3327149041434756 + (-0.20522334808049095 + 0.04019472590901159*(-3 + x))*(-2 + x))*x))};
%        \end{axis}
%\end{tikzpicture}
%\end{image}


%\begin{image}
%\begin{tikzpicture}
%\begin{axis}[
%            ymin=-5,
%			ymax=5,
%            axis lines =center, xlabel=$k$, ylabel=$P$,
%              every axis y label/.style={at=(current axis.above origin),anchor=south},
%              every axis x label/.style={at=(current axis.right of origin),anchor=west},
%            domain=-5:5,
%            grid = major,
%            xtick={-4,...,4},
%            ytick={-4,...,4},
%          ]
%          \addplot [very thick, smooth] {x^2};
%        \end{axis}
%\end{tikzpicture}
%\end{image}



\begin{image}
\begin{tikzpicture}%[scale=\thescale*1.2]
    \draw[thick] (1,1) node[point,fill=black] (a) {} parabola bend (3,3) (4,2.5) node[point,fill=black] (b) {};
    \draw[thick] (1,1) -- (4,2.5);
    \draw (1,1+9/8) -- (4,2.5+9/8) coordinate (topright);
    \node[point,fill=black] (x0) at (2.5,2.875) {};

    \coordinate (origin) at (0,0);
    \draw[<->] (topright -| origin) -- (origin) -- (origin -| topright) -- +(1,0);
    \draw[dotted,very thick] (a) -- (a|-origin) node[below,black] {$a$};
    \draw[dotted,very thick] (b) -- (b|-origin) node[below] {$b$};
    \draw[dashed] (x0) -- (x0|-origin) node[below] {$c$};
\end{tikzpicture}
}
\end{image}

\begin{image}
\subcaptionbox{two choices for $c$}{
\begin{tikzpicture}[scale=\thescale]
    \begin{scope}
    \clip (-3,-2) rectangle (3,2);
    \draw[thick,smooth,domain=-3:3] plot (\x,{\x^3/3 - \x});
    \end{scope}
    \node[point,fill=black] (a) at (-2,-2/3) {};
    \node[point,fill=black] (b) at (2,2/3) {};
    \draw[thick] (a) -- (b);
    \coordinate (origin) at (-4,-3);
    \coordinate (topright) at (4,2);
    \draw[<->] (topright -| origin) -- (origin) -- (origin -| topright);
    \draw[dotted,very thick] (a) -- (a|-origin) node[below] {$a$};
    \draw[dotted,very thick] (b) -- (b|-origin) node[below] {$b$};

    \node[point,fill=black] (x0) at ({-2/sqrt(3)},{(1/3)*(-2/sqrt(3))^3+2/sqrt(3)}) {};
    \draw (x0) +(-1,-1/3) -- +(1,1/3);
    \node[point,fill=black] (x1) at ({2/sqrt(3)},{(1/3)*(2/sqrt(3))^3-2/sqrt(3)}) {};
    \draw (x1) +(-1,-1/3) -- +(1,1/3);
    \draw[dashed] (x0) -- (x0 |- origin) node[below]{$c_1$};
    \draw[dashed] (x1) -- (x1 |- origin) node[below]{$c_2$};
\end{tikzpicture}
}


\end{image}
