\documentclass{ximera}

%% You can put user macros here
%% However, you cannot make new environments



\newcommand{\ffrac}[2]{\frac{\text{\footnotesize $#1$}}{\text{\footnotesize $#2$}}}
\newcommand{\vasymptote}[2][]{
    \draw [densely dashed,#1] ({rel axis cs:0,0} -| {axis cs:#2,0}) -- ({rel axis cs:0,1} -| {axis cs:#2,0});
}


\graphicspath{{./}{firstExample/}}

\usepackage{amsmath}
\usepackage{amssymb}
\usepackage{array}
\usepackage[makeroom]{cancel} %% for strike outs
\usepackage{pgffor} %% required for integral for loops
\usepackage{tikz}
\usepackage{tikz-cd}
\usepackage{tkz-euclide}
\usetikzlibrary{shapes.multipart}


\usetkzobj{all}
\tikzstyle geometryDiagrams=[ultra thick,color=blue!50!black]


\usetikzlibrary{arrows}
\tikzset{>=stealth,commutative diagrams/.cd,
  arrow style=tikz,diagrams={>=stealth}} %% cool arrow head
\tikzset{shorten <>/.style={ shorten >=#1, shorten <=#1 } } %% allows shorter vectors

\usetikzlibrary{backgrounds} %% for boxes around graphs
\usetikzlibrary{shapes,positioning}  %% Clouds and stars
\usetikzlibrary{matrix} %% for matrix
\usepgfplotslibrary{polar} %% for polar plots
\usepgfplotslibrary{fillbetween} %% to shade area between curves in TikZ



%\usepackage[width=4.375in, height=7.0in, top=1.0in, papersize={5.5in,8.5in}]{geometry}
%\usepackage[pdftex]{graphicx}
%\usepackage{tipa}
%\usepackage{txfonts}
%\usepackage{textcomp}
%\usepackage{amsthm}
%\usepackage{xy}
%\usepackage{fancyhdr}
%\usepackage{xcolor}
%\usepackage{mathtools} %% for pretty underbrace % Breaks Ximera
%\usepackage{multicol}



\newcommand{\RR}{\mathbb R}
\newcommand{\R}{\mathbb R}
\newcommand{\C}{\mathbb C}
\newcommand{\N}{\mathbb N}
\newcommand{\Z}{\mathbb Z}
\newcommand{\dis}{\displaystyle}
%\renewcommand{\d}{\,d\!}
\renewcommand{\d}{\mathop{}\!d}
\newcommand{\dd}[2][]{\frac{\d #1}{\d #2}}
\newcommand{\pp}[2][]{\frac{\partial #1}{\partial #2}}
\renewcommand{\l}{\ell}
\newcommand{\ddx}{\frac{d}{\d x}}

\newcommand{\zeroOverZero}{\ensuremath{\boldsymbol{\tfrac{0}{0}}}}
\newcommand{\inftyOverInfty}{\ensuremath{\boldsymbol{\tfrac{\infty}{\infty}}}}
\newcommand{\zeroOverInfty}{\ensuremath{\boldsymbol{\tfrac{0}{\infty}}}}
\newcommand{\zeroTimesInfty}{\ensuremath{\small\boldsymbol{0\cdot \infty}}}
\newcommand{\inftyMinusInfty}{\ensuremath{\small\boldsymbol{\infty - \infty}}}
\newcommand{\oneToInfty}{\ensuremath{\boldsymbol{1^\infty}}}
\newcommand{\zeroToZero}{\ensuremath{\boldsymbol{0^0}}}
\newcommand{\inftyToZero}{\ensuremath{\boldsymbol{\infty^0}}}


\newcommand{\numOverZero}{\ensuremath{\boldsymbol{\tfrac{\#}{0}}}}
\newcommand{\dfn}{\textbf}
%\newcommand{\unit}{\,\mathrm}
\newcommand{\unit}{\mathop{}\!\mathrm}
%\newcommand{\eval}[1]{\bigg[ #1 \bigg]}
\newcommand{\eval}[1]{ #1 \bigg|}
\newcommand{\seq}[1]{\left( #1 \right)}
\renewcommand{\epsilon}{\varepsilon}
\renewcommand{\iff}{\Leftrightarrow}

\DeclareMathOperator{\arccot}{arccot}
\DeclareMathOperator{\arcsec}{arcsec}
\DeclareMathOperator{\arccsc}{arccsc}
\DeclareMathOperator{\si}{Si}
\DeclareMathOperator{\proj}{proj}
\DeclareMathOperator{\scal}{scal}
\DeclareMathOperator{\cis}{cis}
\DeclareMathOperator{\Arg}{Arg}
%\DeclareMathOperator{\arg}{arg}
\DeclareMathOperator{\Rep}{Re}
\DeclareMathOperator{\Imp}{Im}
\DeclareMathOperator{\sech}{sech}
\DeclareMathOperator{\csch}{csch}
\DeclareMathOperator{\Log}{Log}

\newcommand{\tightoverset}[2]{% for arrow vec
  \mathop{#2}\limits^{\vbox to -.5ex{\kern-0.75ex\hbox{$#1$}\vss}}}
\newcommand{\arrowvec}{\overrightarrow}
\renewcommand{\vec}{\mathbf}
\newcommand{\veci}{{\boldsymbol{\hat{\imath}}}}
\newcommand{\vecj}{{\boldsymbol{\hat{\jmath}}}}
\newcommand{\veck}{{\boldsymbol{\hat{k}}}}
\newcommand{\vecl}{\boldsymbol{\l}}
\newcommand{\utan}{\vec{\hat{t}}}
\newcommand{\unormal}{\vec{\hat{n}}}
\newcommand{\ubinormal}{\vec{\hat{b}}}

\newcommand{\dotp}{\bullet}
\newcommand{\cross}{\boldsymbol\times}
\newcommand{\grad}{\boldsymbol\nabla}
\newcommand{\divergence}{\grad\dotp}
\newcommand{\curl}{\grad\cross}
%% Simple horiz vectors
\renewcommand{\vector}[1]{\left\langle #1\right\rangle}


\outcome{Approximate function values using the tangent line}

\title{2.13 Linear Approximation}

\begin{document}

\begin{abstract}
In this lesson we will use the tangent line to approximate the value of a 
function near the point of tangency.
\end{abstract}

\maketitle


Given a function $y=f(x)$, the equation of the tangent line
at the point where $x = a$ is given by
\[y-f(a) = f'(a)(x - a)\]
or
\[y = f(a) + f'(a)(x - a).\]
The main idea of this section is that if we let 
\[L(x) = f(a) + f'(a)(x - a)\] then 
\[f(a) = L(a)\]
and
\[f(x) \approx L(x)\]
for values of $x$ close to $a$. The function $L(x)$ is called the \textbf{linearization} of $f(x)$ at $x = a$. 
The advantage of working with $L(x)$ is that values of a linear 
function are usually easy to compute.
In a typical linear approximation problem, we are trying to approximate a value of $f(x)$. We need to choose $a$ and create
$L(x)$. Once we have accomplished this, our solution is 
\[f(x) \approx L(x).\]
There are two keys to choosing $a$. First $a$ should be close to the $x$-value of interest and second, 
we must be able to compute the exact value of $f(a)$.



\begin{image}
\begin{tikzpicture}[scale=.7]
\draw[<->] (0,6)-- (0,0) --(8,0);
\draw[blue, thin]   plot[smooth,domain=0:8] (\x, {\x^2/10});
\draw[red, thin]   plot[smooth,domain=2.7:7.3] (\x, {(2.5)+(\x -5)});
%\draw [<->,thick, cyan] (0,0) to [out=90,in=199] (9,3);
\draw[mark=square] (5,2.5);
\node at (6, 2.5) {$(a,f(a)$};
%\draw[<->] (4, 0) --(7, 0);
%\node at (3.2, 0) {$x$};
%\draw (5.5,2) parabola (6.75,.25);
%\draw (5.5,2) parabola (4.25,.25);

%\draw (1.5, .1) -- (1.5, -.1) node[below] {$x_0$};
\node at (5,7) {Linear Approximation};
\node at (5,-1) {$f(x) \approx L(x)$ for $x$ near $a$};
\node at (11, 4.5) {$y=L(x)=f(a) + f'(a)(x-a)$};
%\draw[<->] (0, 0) --(3, 0);
\node at (9, 6) {$y = f(x)$};
\node at (5,-.35) {$a$};
\node at (-.7, 2.5) {$f(a)$};
\draw (5, .1) -- (5,-.1);
\draw (-0.1, 2.5)-- (.1,2.5);
\draw[blue, fill] (5,2.5) circle [radius=0.07];


%\draw (1.5,.25) parabola (0.25,2);
%\draw (1.5,.25) parabola (2.75,2);
%\draw (5.5, .1) -- (5.5, -.1) node[below] {$x_0$};
%\node at (5.5, -.9) {$f''(x_0) > 0$};
\end{tikzpicture}
\end{image}

 

\begin{example}[example 1]
Approximate $\sqrt{5}$ using a linear approximation.\\ 
Let $f(x) = \sqrt x$ and since 4 is near 5 and $\sqrt{4} = 2$, we let 
$a = 4$. To create $L(x)$ we also need to compute $f'(4)$. 
Since 
\[f'(x) = \frac{d}{dx} \sqrt x = \frac{d}{dx}(x^{1/2})= \frac{1}{2}x^{-1/2}   = \frac{1}{2\sqrt x},\] 
we have
\[f'(4) = \frac{1}{2\sqrt{4}} = \frac{1}{4}.\]

The linearization of $\sqrt x$ at $x=4$ is now given by
\begin{align*}
L(x) &= f(a) + f'(a)(x - a) \\
&= f(4) + f'(4)(x - 4) \\
&= 2 + \frac{1}{4}(x - 4).
\end{align*}
Finally, if $x$ is near 4, then $\sqrt x \approx L(x)$, so
\[\sqrt{5} \approx L(5) = 2 + \frac{1}{4}(5-4)\]
\[ = 2+ \frac{1}{4} = \frac{9}{4} = 2.25.\]
\end{example}


\begin{image}
\begin{tikzpicture}[scale=.7]
\draw[<->] (0,4)-- (0,0) --(10,0);
\draw[blue, thin]   plot[smooth, samples = 200, domain=0:10] (\x, {\x^(1/2)});
\draw[red, thin]   plot[smooth,domain=.1:9.7] (\x, {2+(1/4)*(\x -4)});
%\draw [<->,thick, cyan] (0,0) to [out=90,in=199] (9,3);
\draw[mark=square] (4,2);
\node at (4, 2.5) {$(4,2)$};
%\draw[<->] (4, 0) --(7, 0);
%\node at (3.2, 0) {$x$};
%\draw (5.5,2) parabola (6.75,.25);
%\draw (5.5,2) parabola (4.25,.25);

%\draw (1.5, .1) -- (1.5, -.1) node[below] {$x_0$};
\node at (4.5,5.5) {Linear Approximation};
\node at (4.5,-1.4) {$\sqrt x \approx 2 + \frac14(x-4)$ for $x$ near $4$};
\node at (9.5, 4) {$y=L(x)=2 + \frac14(x-4)$};
%\draw[<->] (0, 0) --(3, 0);
\node at (9.5, 2.5) {$y = \sqrt x$};
\node at (4,-.35) {$4$};
\node at (-.7, 2) {$2$};
\draw (4, .1) -- (4,-.1);
\draw (-0.1, 2)-- (.1,2);
\draw[blue, fill] (4,2) circle [radius=0.07];


%\draw (1.5,.25) parabola (0.25,2);
%\draw (1.5,.25) parabola (2.75,2);
%\draw (5.5, .1) -- (5.5, -.1) node[below] {$x_0$};
%\node at (5.5, -.9) {$f''(x_0) > 0$};
\end{tikzpicture}
\end{image}

 

\begin{example}[example 2]
Approximate $\sqrt{99}$ using linear approximation. \\
Let $f(x) = \sqrt x$ and since 100 is near 99 and $\sqrt{100} = 10$, we let 
$a = 100$. To create $L(x)$ we also need to compute $f'(100)$. 
Since 
\[f'(x) = \frac{d}{dx} \sqrt x = \frac{d}{dx}(x^{1/2})= \frac{1}{2}x^{-1/2}   = \frac{1}{2\sqrt x},\] 
we have
\[f'(100) = \frac{1}{2\sqrt{100}} = \frac{1}{20}.\]

The linearization of $\sqrt x$ at $x=100$ is now given by
\begin{align*}
L(x) &= f(a) + f'(a)(x - a) \\
&= f(100) + f'(100)(x - 100) \\
&= 10 + \frac{1}{20}(x - 100).
\end{align*}
Finally, if $x$ is near 100, then $\sqrt x \approx L(x)$, so
\[\sqrt{99} \approx L(99) = 10 + \frac{1}{20}(99 - 100)\]
\[ = 10 - \frac{1}{20} = \frac{199}{20} = 9.95.\]
\end{example}

\begin{problem}(problem 2a)
Find the linearization of $f(x) = \sqrt x$ at $x = 16$ and use it to approximate $\sqrt{18}$.

The linearization is  $L(x) = \answer{4 + .125 (x-16)}$.

The approximation is $\sqrt{18} \approx \answer{4.25}$.
\end{problem}

\begin{problem}(problem 2b)
Find the linearization of $f(x) = \sqrt x$ at $x = 36$ and use it to approximate $\sqrt{34}$.

The linearization is  $L(x) = \answer{6 + (1/12) (x-36)}$.

The approximation is $\sqrt{34} \approx \answer{35/6}$.
\end{problem}

\begin{example}[example 3]
Approximate $\sqrt[3]{10}$ using linear approximation.\\ 
Let $f(x) = \sqrt[3] x$ and since 10 is near 8 and $\sqrt[3]{8} = 2$, we let 
$a = 8$. To create $L(x)$ we also need to compute $f'(8)$. 
Since 
\[f'(x) = \frac{d}{dx} \sqrt[3] x = \frac{d}{dx}(x^{1/3})= \frac{1}{3}x^{-2/3}   = \frac{1}{3\sqrt[3] {x^2}},\] 
we have
\[f'(8) = \frac{1}{3\sqrt[3]{8^2}} =\frac{1}{3\sqrt[3]{64}}= \frac{1}{12}.\]

Next, 
\begin{align*}
L(x) &= f(a) + f'(a)(x - a) \\
&= f(8) + f'(8)(x - 8) \\
&= 2 + \frac{1}{12}(x - 8)
\end{align*}
and finally,
\[\sqrt[3]{10} \approx L(10) = 2 + \frac{1}{12}(10 - 8)\]
\[ = 2 + \frac{2}{12} = \frac{13}{6} = 2.1{\overline 6}.\]
\end{example}

\begin{problem}(problem 3)
Find the linearization of $f(x) = \sqrt[3] x$ at $x = 27$ and use it to approximate $\sqrt[3]{25}$.

The linearization is  $L(x) = \answer{3 + (1/27) (x-27)}$.

The approximation is $\sqrt[3]{25} \approx \answer{79/27}$.
\end{problem}

\begin{example}[example 4]
We can approximate $e^{-0.1}$ using linear approximation as follows. 
Let $f(x) = e^x$ and since 0 is near $-0.1$ and $e^0 = 1$, we let 
$a = 0$. To create $L(x)$ we also need to compute $f'(0)$. 
Since 
\[f'(x) = \frac{d}{dx} e^x = e^x,\] 
we have
\[f'(0) = e^0 = 1.\]

Next, 
\begin{align*}
L(x) &= f(a) + f'(a)(x - a) \\
&= f(0) + f'(0)(x - 0) \\
&= 1 + 1(x - 0)\\
& = 1 + x
\end{align*}
and finally,
\[
e^{-0.1} \approx L(-0.1) = 1 - 0.1 = 0.9.
\]

We conclude this example by computing \textbf{relative error}. Expressed as a percentage, the relative error is given by
\[
\text{relative error} = \Big|\frac{\text{actual} - \text{estimate}}{\text{actual}}\Big| \times 100\%.
\]
For the actual value, we will use the value of $e^{-0.1}$ from a calculator.
We have
\[
\text{relative error} = \Big|\frac{0.90483741803 - 0.9}{0.90483741803}\Big| \times 100\% = 0.534617373\%.
\]

\end{example}


\begin{image}
\begin{tikzpicture}[scale=.7]
\draw[<-] (0,4)-- (0,0);
\draw[<->] (-4,0) --(4,0);
\draw[blue, thin]   plot[smooth,domain=-3:1.3] (\x, {e^\x});
\draw[red, thin]   plot[smooth,domain=-1.4:1.4] (\x, {\x +1});
%\draw [<->,thick, cyan] (0,0) to [out=90,in=199] (9,3);
%\draw[mark=square] (4,2);
%\node at (4, 2.5) {$(4,2)$};
%\draw[<->] (4, 0) --(7, 0);
%\node at (3.2, 0) {$x$};
%\draw (5.5,2) parabola (6.75,.25);
%\draw (5.5,2) parabola (4.25,.25);

%\draw (1.5, .1) -- (1.5, -.1) node[below] {$x_0$};
\node at (0,4.5) {Linear Approximation};
\node at (0,-1.4) {$e^x \approx 1+x$ for $x$ near $0$};
\node at (3, 1.8) {$y=L(x)=1 + x$};
%\draw[<->] (0, 0) --(3, 0);
\node at (2, 3.1) {$y = e^x$};
\node at (0,-.35) {$0$};
\node at (-.7, 1) {$1$};
\draw (0, .1) -- (0,-.1);
\draw (-0.15, 1)-- (.15,1);
\draw[blue, fill] (0,1) circle [radius=0.07];


%\draw (1.5,.25) parabola (0.25,2);
%\draw (1.5,.25) parabola (2.75,2);
%\draw (5.5, .1) -- (5.5, -.1) node[below] {$x_0$};
%\node at (5.5, -.9) {$f''(x_0) > 0$};
\end{tikzpicture}
\end{image}

 
 
 

\begin{problem}(problem 4)
Find the linearization of $f(x) = e^x$ at $x = 0$ and use it to approximate $e^{0.2}$.
Also, find the relative error.

The linearization is  $L(x) = \answer{1 + x}$.

The approximation is $e^{0.2} \approx \answer{1.2}$.

The relative error in this approximation is $\answer{1.75230963}\%$.


\end{problem}

\begin{example}[example 5]
Approximate $\ln(1.1)$ using linear approximation.\\ 
Let $f(x) = \ln x$ and since 1 is near 1.1 and $\ln(1) = 0$, we let 
$a = 1$. To create $L(x)$ we also need to compute $f'(1)$. 
Since 
\[f'(x) = \frac{d}{dx} \ln x = \frac{1}{x},\] 
we have
\[f'(1) = \frac{1}{1} = 1.\]

Next, 
\begin{align*}
L(x) &= f(a) + f'(a)(x - a) \\
&= f(1) + f'(1)(x - 1) \\
&= 0 + 1(x - 1)\\
&=x-1
\end{align*}
and finally,
\[\ln(1.1) \approx L(1.1) = 1.1 - 1 = 0.1.\]


\end{example}

\begin{problem}(problem 5)
Find the linearization of $f(x) = \ln x$ at $x = 1$ and use it to approximate $\ln(0.8)$.

The linearization is  $L(x) = \answer{x-1}$.

The approximation is $\ln(0.8) \approx \answer{-0.2}$.
\end{problem}

\begin{example}[example 6]
Approximate $1.9^3$ using linear approximation.\\ 
Let $f(x) = x^3$ and since 2 is near 1.9 and $2^3 = 8$, we let 
$a = 2$. To create $L(x)$ we also need to compute $f'(2)$. 
Since 
\[f'(x) = \frac{d}{dx} (x^3) = 3x^2,\] 
we have
\[f'(2) = 3(2^2) = 12.\]

Next, 
\begin{align*}
L(x) &= f(a) + f'(a)(x - a) \\
&= f(2) + f'(2)(x - 2) \\
&= 8 + 12(x - 2)
\end{align*}
and finally,
\[1.9^3 \approx L(2) = 8 + 12(1.9 - 2) \]
\[= 8 - 12(.1) = 8-(1.2) = 6.8.\]
\end{example}

\begin{problem}(problem 6)
Find the linearization of $f(x) = x^4$ at $x = 1$ and use it to approximate $1.1^4$.

The linearization is  $L(x) = \answer{1 + 4 (x-1)}$.

The approximation is $1.1^4 \approx \answer{1.4}$.
\end{problem}

\begin{example}[example 7]
Approximate $\sin(1^{\circ})$ using linear approximation.\\
In the formula 
\[ \frac{d}{dx} \sin(x) = \cos(x) \]
it is understood that the angle is measured in radians. Therefore, in order to use our linear approximation 
formula we need to restate our problem in radians as:  
\[\text{approximate} \ \ \sin\big(\frac{\pi}{180}\big).\]

We let $f(x) = \sin x$ and since 0 is near $\frac{\pi}{180}$ and $\sin(0) = 0$, we let 
$a = 0$. To create $L(x)$ we also need to compute $f'(0)$. 
Since 
\[f'(x) = \frac{d}{dx} \sin(x) = \cos(x),\] 
we have
\[f'(0) = \cos(0) = 1.\]

Next, 
\begin{align*}
L(x) &= f(a) + f'(a)(x - a) \\
&= f(0) + f'(0)(x - 0) \\
&= 0 + 1(x - 0)\\
&= x
\end{align*}
and finally,
\[\sin(1^{\circ}) = \sin\big(\frac{\pi}{180}\big) \approx L\big(\frac{\pi}{180}\big) = \frac{\pi}{180}.\]
\end{example}


 

\begin{center}
\begin{foldable}
\unfoldable{Here is a detailed, lecture style video on linear approximation:}
\youtube{yKMlhij92i8}
\end{foldable}
\end{center}

\end{document}
































\begin{image}
\begin{tikzpicture}[scale = 1.2]
\begin{axis}[axis x line=bottom, axis y line= middle, ymin=0, ymax=4,
 xtick={4}, xticklabels={$a$}, ytick={1}, yticklabels={$f(a)$},
axis equal, title={Linear Approximation}]
\addplot[domain=0:10, thick, blue!65]{-1 + x^(1/2)};
\addplot[dashed, thick] coordinates{(4, 1)};
\addplot[domain=0:10, thick]{1+(1/4)*(x-4)};
\addplot[color=blue, mark=*] coordinates {(4,1)};
\end{axis}
\node at (8,4.5) {$y=L(x) = f(a) + f'(a)(x-a)$ };	
\node at (4,.5) {$f(x) \approx L(x)$ for $x$ near $a$ };	
\node at (7, 3) {$y = f(x)$};
\node at (3,3.5) {$(a,f(a)$};
\end{tikzpicture}
\end{image}
