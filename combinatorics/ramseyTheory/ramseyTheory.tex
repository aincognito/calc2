\documentclass[handout]{ximera}

%% You can put user macros here
%% However, you cannot make new environments



\newcommand{\ffrac}[2]{\frac{\text{\footnotesize $#1$}}{\text{\footnotesize $#2$}}}
\newcommand{\vasymptote}[2][]{
    \draw [densely dashed,#1] ({rel axis cs:0,0} -| {axis cs:#2,0}) -- ({rel axis cs:0,1} -| {axis cs:#2,0});
}


\graphicspath{{./}{firstExample/}}

\usepackage{amsmath}
\usepackage{amssymb}
\usepackage{array}
\usepackage[makeroom]{cancel} %% for strike outs
\usepackage{pgffor} %% required for integral for loops
\usepackage{tikz}
\usepackage{tikz-cd}
\usepackage{tkz-euclide}
\usetikzlibrary{shapes.multipart}


\usetkzobj{all}
\tikzstyle geometryDiagrams=[ultra thick,color=blue!50!black]


\usetikzlibrary{arrows}
\tikzset{>=stealth,commutative diagrams/.cd,
  arrow style=tikz,diagrams={>=stealth}} %% cool arrow head
\tikzset{shorten <>/.style={ shorten >=#1, shorten <=#1 } } %% allows shorter vectors

\usetikzlibrary{backgrounds} %% for boxes around graphs
\usetikzlibrary{shapes,positioning}  %% Clouds and stars
\usetikzlibrary{matrix} %% for matrix
\usepgfplotslibrary{polar} %% for polar plots
\usepgfplotslibrary{fillbetween} %% to shade area between curves in TikZ



%\usepackage[width=4.375in, height=7.0in, top=1.0in, papersize={5.5in,8.5in}]{geometry}
%\usepackage[pdftex]{graphicx}
%\usepackage{tipa}
%\usepackage{txfonts}
%\usepackage{textcomp}
%\usepackage{amsthm}
%\usepackage{xy}
%\usepackage{fancyhdr}
%\usepackage{xcolor}
%\usepackage{mathtools} %% for pretty underbrace % Breaks Ximera
%\usepackage{multicol}



\newcommand{\RR}{\mathbb R}
\newcommand{\R}{\mathbb R}
\newcommand{\C}{\mathbb C}
\newcommand{\N}{\mathbb N}
\newcommand{\Z}{\mathbb Z}
\newcommand{\dis}{\displaystyle}
%\renewcommand{\d}{\,d\!}
\renewcommand{\d}{\mathop{}\!d}
\newcommand{\dd}[2][]{\frac{\d #1}{\d #2}}
\newcommand{\pp}[2][]{\frac{\partial #1}{\partial #2}}
\renewcommand{\l}{\ell}
\newcommand{\ddx}{\frac{d}{\d x}}

\newcommand{\zeroOverZero}{\ensuremath{\boldsymbol{\tfrac{0}{0}}}}
\newcommand{\inftyOverInfty}{\ensuremath{\boldsymbol{\tfrac{\infty}{\infty}}}}
\newcommand{\zeroOverInfty}{\ensuremath{\boldsymbol{\tfrac{0}{\infty}}}}
\newcommand{\zeroTimesInfty}{\ensuremath{\small\boldsymbol{0\cdot \infty}}}
\newcommand{\inftyMinusInfty}{\ensuremath{\small\boldsymbol{\infty - \infty}}}
\newcommand{\oneToInfty}{\ensuremath{\boldsymbol{1^\infty}}}
\newcommand{\zeroToZero}{\ensuremath{\boldsymbol{0^0}}}
\newcommand{\inftyToZero}{\ensuremath{\boldsymbol{\infty^0}}}


\newcommand{\numOverZero}{\ensuremath{\boldsymbol{\tfrac{\#}{0}}}}
\newcommand{\dfn}{\textbf}
%\newcommand{\unit}{\,\mathrm}
\newcommand{\unit}{\mathop{}\!\mathrm}
%\newcommand{\eval}[1]{\bigg[ #1 \bigg]}
\newcommand{\eval}[1]{ #1 \bigg|}
\newcommand{\seq}[1]{\left( #1 \right)}
\renewcommand{\epsilon}{\varepsilon}
\renewcommand{\iff}{\Leftrightarrow}

\DeclareMathOperator{\arccot}{arccot}
\DeclareMathOperator{\arcsec}{arcsec}
\DeclareMathOperator{\arccsc}{arccsc}
\DeclareMathOperator{\si}{Si}
\DeclareMathOperator{\proj}{proj}
\DeclareMathOperator{\scal}{scal}
\DeclareMathOperator{\cis}{cis}
\DeclareMathOperator{\Arg}{Arg}
%\DeclareMathOperator{\arg}{arg}
\DeclareMathOperator{\Rep}{Re}
\DeclareMathOperator{\Imp}{Im}
\DeclareMathOperator{\sech}{sech}
\DeclareMathOperator{\csch}{csch}
\DeclareMathOperator{\Log}{Log}

\newcommand{\tightoverset}[2]{% for arrow vec
  \mathop{#2}\limits^{\vbox to -.5ex{\kern-0.75ex\hbox{$#1$}\vss}}}
\newcommand{\arrowvec}{\overrightarrow}
\renewcommand{\vec}{\mathbf}
\newcommand{\veci}{{\boldsymbol{\hat{\imath}}}}
\newcommand{\vecj}{{\boldsymbol{\hat{\jmath}}}}
\newcommand{\veck}{{\boldsymbol{\hat{k}}}}
\newcommand{\vecl}{\boldsymbol{\l}}
\newcommand{\utan}{\vec{\hat{t}}}
\newcommand{\unormal}{\vec{\hat{n}}}
\newcommand{\ubinormal}{\vec{\hat{b}}}

\newcommand{\dotp}{\bullet}
\newcommand{\cross}{\boldsymbol\times}
\newcommand{\grad}{\boldsymbol\nabla}
\newcommand{\divergence}{\grad\dotp}
\newcommand{\curl}{\grad\cross}
%% Simple horiz vectors
\renewcommand{\vector}[1]{\left\langle #1\right\rangle}


\pgfplotsset{compat=1.13}

\outcome{Compute Ramsey Numbers}

\title{4.2 Ramsey Theory}

\begin{document}

\begin{abstract}
We compute Ramsey numbers in small cases
\end{abstract}

\maketitle

\section{Ramsey Theory}

In this section we consider graph colorings. That is, each edge in a graph will be assigned a particular color.
The graph below is a $K_3$ colored with red and blue edges. The vertices represent people and the color represents whether
they know each other (red) or not (blue).

\begin{image}
\begin{tikzpicture}
\draw[fill] (0,0) circle (1pt) node[left]{A};
\draw[fill] (2,0) circle (1pt) node[right]{B};
\draw[fill] (1,1) circle (1pt)node[above]{C};
\draw[red] (0,0) -- (2,0);
\draw[blue] (0,0) -- (1,1);
\draw[blue] (2,0) -- (1,1);
\node at (1,-0.5) {A and B know each other,};
\node at (1, -1){B and C do not};
\end{tikzpicture}
\end{image}

The Pigeon Hole Principle suggests that in a $K_3$ with two colors, there must be two of the same color.
Hence, in a group of three people, there are at least 2 pairs of people that know each other or 
two pairs of people that do not.

\begin{image}
\begin{tikzpicture}
\draw[fill] (0,0) circle (1pt) node[left]{A};
\draw[fill] (2,0) circle (1pt) node[right]{B};
\draw[fill] (1,1) circle (1pt)node[above]{C};
\draw[red] (0,0) -- (2,0);
\draw[red] (0,0) -- (1,1);
\draw[red] (2,0) -- (1,1) ;
\node at (1,-0.5){Three people, each of whom knows the others};
\end{tikzpicture}
\end{image}

\begin{image}
\begin{tikzpicture}
\draw[fill] (0,0) circle (1pt) node[left]{A};
\draw[fill] (2,0) circle (1pt) node[right]{B};
\draw[fill] (1,1) circle (1pt)node[above]{C};
\draw[blue] (0,0) -- (2,0)node[midway,below = 6pt, black]{Three people, none of whom knows another};
\draw[blue] (0,0) -- (1,1);
\draw[blue] (2,0) -- (1,1) ;
\end{tikzpicture}
\end{image}

A fundamental question in Ramsey Theory is: how many people are needed in order to ensure that 
there is a subgroup of 3 people that are either mutually acquainted or mutually unacquainted?
In terms of graphs with edge colorings, what is the smallest value of $n$ such that $K_n$ with 2 colors 
must contain a monochromatic $K_3$? This value of $n$ is known as a Ramsey Number and denoted $R(3,3)$

\begin{image}
\begin{tikzpicture}
\draw[fill] (0,0) circle (1pt)node[ left]{A} ;
\draw[fill] (2,0) circle (1pt)node[right]{D} ;
\draw[fill] (0,2) circle (1pt)node[left]{B};
\draw[fill] (2,2) circle (1pt)node[right]{C};
\draw[blue] (0,0) -- (2,0)node[midway,below = 6pt, black]{$K_4$ with no monochromatic $K_3$};
\draw[red] (0,0) -- (2,2);
\draw[red] (0,0) -- (0,2) ;
\draw[blue] (2,0) -- (0,2) ;
\draw[red] (2,0) -- (2,2) ;
\draw[blue] (2,2) -- (0,2) ;
\end{tikzpicture}
\end{image}

The figure above shows that $R(3,3) > 4$. 

\begin{problem}(problem 1) 
Show that $R(3,3) > 5$ by exhibiting an edge coloring on $K_5$ that does not contain a monochromatic $K_3$
\end{problem}

We now prove that $R(3,3) = 6$. 

\begin{theorem}[$R(3,3) = 6$]
The smallest number of vertices $n$ for which a $K_n$ with edges colored red and blue must contain a 
monochromatic $K_3$ is 6, i.e., $R(3,3) = 6$.
% $K_6$ with blue and red edge colorings must contain a monochromatic $K_3$
%and $6$ is the smallest number of vertices for which this is true.
\end{theorem}
\begin{proof}
Consider a $K_6$ with edges colored red and blue. 
We must show that it contains a monochromatic $K_3$. Let $V$ be any vertex in the $K_6$. 
V has 5 edges emanating from it.  By the Pigeon Hole Principle, there must be at least 3 of the same color.

\begin{image}
\begin{tikzpicture}
\draw[fill] (0,0) circle (1pt)node[ left]{V} ;
\draw[fill] (2,2) circle (1pt)node[right]{A} ;
\draw[fill] (2,1) circle (1pt)node[right]{B};
\draw[fill] (2,0) circle (1pt)node[right]{C};
\draw[fill] (2,-1) circle (1pt)node[right]{D};
\draw[fill] (2,-2) circle (1pt)node[right]{E};
\draw[red] (0,0) -- (2,2);
\draw[red] (0,0) -- (2,1);
\draw[red] (0,0) -- (2,0) ;
\draw[blue] (0,0) -- (2,-1) ;
\draw[blue] (0,0) -- (2,-2) ;
\end{tikzpicture}
\end{image}

Without loss of generality, assume $V$ is connected to vertices $A, B$ and $C$ with red edges, as pictured above. 
Then consider the $K_3$ embedded in the $K_6$ consisting of vertices $A, B$ and $C$.
If this $K_3$ contains a red edge, say from $A$ to $B$, then the $K_3$ formed by $V, A$ and $B$ is monochromatic (red).
If the $K_3, (A, B, C)$ does not contain a red edge, then IT is a monochromatic $K_3$ (blue).
Either way, the $K_6$ must contain a monochromatic $K_3$. Finally, since $R(3,3) > 5$ we can conclude that $R(3,3) = 6$.

\end{proof}


We can use more than two colors. Suppose we use green, red and blue to color the edges of $K_n$. 
What is the minimum value of $n$ that guarantees a monochromatic $K_3$? This Ramsey Number is denoted $R(3,3,3)$.

\begin{theorem}[$R(3,3,3) = 17$]
Let $V$ be a vertex of $K_{17}$ with edges colored green, red and blue. There are 16 edges emanating from $V$.
By the Pigeon Hole Principle, there must be at least 6 of one color (otherwise, we would have at most 5 of each color
and 5+5+5 = 15 <16).  Which color this is is unimportant, so let's say it is green. Thus we can say that the vertex $V$ is 
joined to the vertices $A_1, A_2, A_3, A_4, A_5, A_6$ by green edges. Now, if any of the edges connecting an $A_i$
with an $A_j$ are green, then $V, A_i, A_j$ would be a green $K_3$ and hence the $K_{17}$ would have a monochromatic 
$K_3$ as claimed. So now we must consider the possibility that none of the edges connecting and $A_i$ to any other $A_j$
is colored green.  In this case, all of the edges among the 6 vertices $A_1, A_2, ..., A_6$ would be 
colored either red or blue.  Thus we would have a $K_6$ colored with two colors embedded in our $K_{17}$.  
By the previous theorem, $R(3,3) = 6$, so we are guaranteed that this embedded $K_6$ contains a monochromatic $K_3$.
We can now conclude that $R(3,3,3) \leq 17$. To see that it is actually 17, we must exhibit a coloring of $K_16$ with three 
colors that does not contain  a monochromatic $K_3$.  
The standard example is pictured below. The three identical monochromatic subgraphs are know as the Clebsch graph.

%\begin{image}
%\includegraphics{K16}
%\end{image}

\end{theorem}

\begin{problem}
Suppose $K_n$ is colored using 4 colors: yellow, green, red and blue. The smallest value of $n$ that guarantees
a monochromatic $K_3$ is denoted $R(3,3,3,3)$. Show that $R(3,3,3,3) \leq 66$
\begin{hint}
Use the fact that $R(3,3,3) = 17$
\end{hint}
How many edges must emanate from a vertex $V$ in $K_n$ to ensure that there are at least 17 of one color? $\answer{65}$
How many vertices must a graph have to be sure that this many edges emanate from each vertex? $\answer{66}$

\end{problem}

\end{document}


\begin{example}[example 1]
Pat has three distinct shirts and two distinct pairs of pants. Getting dressed requires one of each.  
In how many ways can Pat get dressed?\\
We make a tree diagram to determine the possible outfits Pat can create.
\begin{image}
\begin{tikzpicture}
\draw[mark=*,mark size=3pt,mark options={color=blue}] plot coordinates {(0,0)} node[left, blue]{Pat\; };
\draw (0,0) -- (3,4) ;
\draw[mark=*,mark size=3pt,mark options={color=blue}] plot coordinates {(3,4)} node[left, blue]{Shirt 1\, };
\draw (0,0) -- (3,0) ;
\draw[mark=*,mark size=3pt,mark options={color=blue}] plot coordinates {(3,0)} node[above left, blue]{Shirt 2 \,};
\draw (0,0) -- (3,-4) ;
\draw[mark=*,mark size=3pt,mark options={color=blue}] plot coordinates {(3,-4)} node[left, blue]{Shirt 3 \,};
\draw (3,4) -- (6,5) ;
\draw[mark=*,mark size=3pt,mark options={color=blue}] plot coordinates {(6,5)} node[right, blue]{ \,Pants 1};
\draw (3,4) -- (6,3) ;
\draw[mark=*,mark size=3pt,mark options={color=blue}] plot coordinates {(6,3)} node[right, blue]{\, Pants 2};
\draw (3,0) -- (6,1) ;
\draw[mark=*,mark size=3pt,mark options={color=blue}] plot coordinates {(6,1)} node[right, blue]{ \,Pants 1};
\draw (3,0) -- (6,-1) ;
\draw[mark=*,mark size=3pt,mark options={color=blue}] plot coordinates {(6,-1)} node[right, blue]{\, Pants 2};
\draw (3,-4) -- (6,-3) ;
\draw[mark=*,mark size=3pt,mark options={color=blue}] plot coordinates {(6,-3)} node[right, blue]{\, Pants 1};
\draw (3,-4) -- (6,-5) ;
\draw[mark=*,mark size=3pt,mark options={color=blue}] plot coordinates {(6,-5)} node[right, blue]{\,Pants 2};
\end{tikzpicture}
\end{image}
Each path from Pat to a node on the far right of the tree diagram represents a different way for Pat to dress. From top to bottom, 
these represent the following outfits: Shirt 1 with Pants 1, Shirt 1 with Pants 2, Shirt 2 with Pants 1, 
Shirt 2 with Pants 2, Shirt 3 with Pants 1 and Shirt 3 with Pants 2. 
Since there are 6 total paths, Pat can dress in 6 different ways.

\end{example}

\begin{problem}(problem 1a)
Sam is getting dressed and wants to include a belt and a hat.  
If Sam has three distinct belts and two distinct hats, how many ways
can Sam accessorize? Include a tree diagram that illustrates the possibilities.\\
Sam can accessorize in $\; \answer{6} \;$ different ways.
\end{problem}

\begin{problem}(problem 1b)
A couple wants to go on a date.  They decide to  eat at a restaurant, walk in a park and then see a show.
If there are two restaurants that the y both like, two nearby parks and two shows playing, 
how many possibilities are there for their date? Make a tree diagram illustrating the possibilities.\\
There are $\;\answer{8}\;$ different possibilities for their date.
\end{problem}


\begin{theorem}[Fundamental Principle of Counting]
Suppose that $n$ decisions are to be made and that the number of choices for the $k^{th}$ decision is $d_k$.
Then the total number of ways to make the $n$ decisions is
\[
\prod_{k=1}^n d_k = d_1 \cdot d_2 \cdot \ldots \cdot d_n
\]
\end{theorem}

\begin{example}
There are three types of Tesla Model 3: Standard Range, Long Range and Performance. 
Each type can be painted one of 5 colors. Furthermore, two interior color schemes are available. 
Given these parameters, how many choices are there for my Tesla Model 3 purchase?\\
In total, three decisions must be made: the type, the paint color and the interior color scheme.
The number of choices for each decision is 3, 5 and 2 respectively. 
Hence, by the Fundamental Principle of Counting, the total number of ways to configure my new 
Tesla Model 3 is $3 \cdot 5\cdot 2 = 30$.
\begin{image}
\begin{tikzpicture}

\draw (0,0) -- (2,0) node[blue, below, midway]{Type} node[above, midway]{3} node[right]{$\times$};
\draw (2.5,0) -- (4.5,0) node[blue, below, midway]{Paint} node[above, midway]{5} node[right]{$\times$};
\draw (5,0) -- (7,0) node[blue, below, midway]{Interior} node[above, midway]{2} node[right]{$=$};
\draw (7.5,0) -- (9.5,0) node[blue, below, midway]{Total} node[above, midway]{30} ;

\end{tikzpicture}
\end{image}
\end{example}


\begin{problem}(problem 2a)
A three course meal consists of an appetizer, an entree and a dessert. 
If a restaurant offers a dozen appetizers, a score of entrees and a bakers dozen desserts, 
how many possible three course meals can be ordered?\\
The total number of possible three course meals is $\answer{3120}$.
\end{problem}

\begin{problem}(problem 2b)
A sixth grade class of 20 students would like to select a President, Vice-President, Secretary and Treasurer.
If no student may serve in more than one post, in how many possible ways can the class officers be selected?\\
The total number of ways to select the class officers is $\answer{116280}$.
\end{problem}

\end{document}






















