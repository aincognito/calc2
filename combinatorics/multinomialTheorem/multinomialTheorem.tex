\documentclass[handout]{ximera}

%% You can put user macros here
%% However, you cannot make new environments



\newcommand{\ffrac}[2]{\frac{\text{\footnotesize $#1$}}{\text{\footnotesize $#2$}}}
\newcommand{\vasymptote}[2][]{
    \draw [densely dashed,#1] ({rel axis cs:0,0} -| {axis cs:#2,0}) -- ({rel axis cs:0,1} -| {axis cs:#2,0});
}


\graphicspath{{./}{firstExample/}}

\usepackage{amsmath}
\usepackage{amssymb}
\usepackage{array}
\usepackage[makeroom]{cancel} %% for strike outs
\usepackage{pgffor} %% required for integral for loops
\usepackage{tikz}
\usepackage{tikz-cd}
\usepackage{tkz-euclide}
\usetikzlibrary{shapes.multipart}


\usetkzobj{all}
\tikzstyle geometryDiagrams=[ultra thick,color=blue!50!black]


\usetikzlibrary{arrows}
\tikzset{>=stealth,commutative diagrams/.cd,
  arrow style=tikz,diagrams={>=stealth}} %% cool arrow head
\tikzset{shorten <>/.style={ shorten >=#1, shorten <=#1 } } %% allows shorter vectors

\usetikzlibrary{backgrounds} %% for boxes around graphs
\usetikzlibrary{shapes,positioning}  %% Clouds and stars
\usetikzlibrary{matrix} %% for matrix
\usepgfplotslibrary{polar} %% for polar plots
\usepgfplotslibrary{fillbetween} %% to shade area between curves in TikZ



%\usepackage[width=4.375in, height=7.0in, top=1.0in, papersize={5.5in,8.5in}]{geometry}
%\usepackage[pdftex]{graphicx}
%\usepackage{tipa}
%\usepackage{txfonts}
%\usepackage{textcomp}
%\usepackage{amsthm}
%\usepackage{xy}
%\usepackage{fancyhdr}
%\usepackage{xcolor}
%\usepackage{mathtools} %% for pretty underbrace % Breaks Ximera
%\usepackage{multicol}



\newcommand{\RR}{\mathbb R}
\newcommand{\R}{\mathbb R}
\newcommand{\C}{\mathbb C}
\newcommand{\N}{\mathbb N}
\newcommand{\Z}{\mathbb Z}
\newcommand{\dis}{\displaystyle}
%\renewcommand{\d}{\,d\!}
\renewcommand{\d}{\mathop{}\!d}
\newcommand{\dd}[2][]{\frac{\d #1}{\d #2}}
\newcommand{\pp}[2][]{\frac{\partial #1}{\partial #2}}
\renewcommand{\l}{\ell}
\newcommand{\ddx}{\frac{d}{\d x}}

\newcommand{\zeroOverZero}{\ensuremath{\boldsymbol{\tfrac{0}{0}}}}
\newcommand{\inftyOverInfty}{\ensuremath{\boldsymbol{\tfrac{\infty}{\infty}}}}
\newcommand{\zeroOverInfty}{\ensuremath{\boldsymbol{\tfrac{0}{\infty}}}}
\newcommand{\zeroTimesInfty}{\ensuremath{\small\boldsymbol{0\cdot \infty}}}
\newcommand{\inftyMinusInfty}{\ensuremath{\small\boldsymbol{\infty - \infty}}}
\newcommand{\oneToInfty}{\ensuremath{\boldsymbol{1^\infty}}}
\newcommand{\zeroToZero}{\ensuremath{\boldsymbol{0^0}}}
\newcommand{\inftyToZero}{\ensuremath{\boldsymbol{\infty^0}}}


\newcommand{\numOverZero}{\ensuremath{\boldsymbol{\tfrac{\#}{0}}}}
\newcommand{\dfn}{\textbf}
%\newcommand{\unit}{\,\mathrm}
\newcommand{\unit}{\mathop{}\!\mathrm}
%\newcommand{\eval}[1]{\bigg[ #1 \bigg]}
\newcommand{\eval}[1]{ #1 \bigg|}
\newcommand{\seq}[1]{\left( #1 \right)}
\renewcommand{\epsilon}{\varepsilon}
\renewcommand{\iff}{\Leftrightarrow}

\DeclareMathOperator{\arccot}{arccot}
\DeclareMathOperator{\arcsec}{arcsec}
\DeclareMathOperator{\arccsc}{arccsc}
\DeclareMathOperator{\si}{Si}
\DeclareMathOperator{\proj}{proj}
\DeclareMathOperator{\scal}{scal}
\DeclareMathOperator{\cis}{cis}
\DeclareMathOperator{\Arg}{Arg}
%\DeclareMathOperator{\arg}{arg}
\DeclareMathOperator{\Rep}{Re}
\DeclareMathOperator{\Imp}{Im}
\DeclareMathOperator{\sech}{sech}
\DeclareMathOperator{\csch}{csch}
\DeclareMathOperator{\Log}{Log}

\newcommand{\tightoverset}[2]{% for arrow vec
  \mathop{#2}\limits^{\vbox to -.5ex{\kern-0.75ex\hbox{$#1$}\vss}}}
\newcommand{\arrowvec}{\overrightarrow}
\renewcommand{\vec}{\mathbf}
\newcommand{\veci}{{\boldsymbol{\hat{\imath}}}}
\newcommand{\vecj}{{\boldsymbol{\hat{\jmath}}}}
\newcommand{\veck}{{\boldsymbol{\hat{k}}}}
\newcommand{\vecl}{\boldsymbol{\l}}
\newcommand{\utan}{\vec{\hat{t}}}
\newcommand{\unormal}{\vec{\hat{n}}}
\newcommand{\ubinormal}{\vec{\hat{b}}}

\newcommand{\dotp}{\bullet}
\newcommand{\cross}{\boldsymbol\times}
\newcommand{\grad}{\boldsymbol\nabla}
\newcommand{\divergence}{\grad\dotp}
\newcommand{\curl}{\grad\cross}
%% Simple horiz vectors
\renewcommand{\vector}[1]{\left\langle #1\right\rangle}


\pgfplotsset{compat=1.13}

\outcome{Explore the Multinomial Theorem}

\title{1.10 Multinomial Theorem}

\begin{document}

\begin{abstract}
We explore the Multinomial Theorem.
\end{abstract}

\maketitle

Consider the trinomial expansion of $(x+y+z)^6$.  
The terms will have the form $x^{n_1}y^{n_2}z^{n_3}$ where $n_1+n_2+n_3 = 6$, such as  $xy^3z^2$ and $x^4y^2$.  What are their coefficients?
The coefficient of the first of these is the number of permutations of the word $xyyyzz$, which is
\[
\frac{6!}{1!3!2!}
\]
and the coefficient of the second is 
\[
\frac{6!}{4!2!0!}
\]
These are multinomial coefficients and they are denoted
\[
\binom{6}{1\; 3\; 2} \;\; \text{and} \;\; \binom{6}{4\; 2\; 0}
\]
respectively.
Note that in this notation, ordinary binomial coefficients could be written as 
\[
\binom{n}{k} = \binom{n}{k\;\; n-k}
\]

The general multinomial coefficient is defined as
\[
\binom{n}{n_1\, n_2 \, \dots n_k} = \frac{n!}{n_1! n_2! \cdots n_k!}
\]
where $n, n_1, n_2, \dots, n_k$ are non-negative integers satisfying
\[
n_1 + n_2 + \cdots + n_k = n
\]


\begin{example} How many ways can a group of 10 people be broken into three subgroups consisting 
of 2, 3 and 5 people?\\
First, we can select the subgroup of 2 people in $C(10, 2)$ ways. Then we can select the subgroup of 3 people
in $C(8,3)$ ways. finally, we can select the subgroup of 5 people in $C(5,5)$ ways. 
By the FPC, the total number of ways to create the three subgroups is
\[
\binom{10}{2} \binom{8}{3} \binom{5}{5} = \frac{10!}{2! 8!} \cdot \frac{8!}{3! 5!}
 \cdot \frac{5!}{5! 0!} = \frac{10!}{2! 3! 5!} = \binom{10}{2\;3\;5}
\]
We can see this directly by lining up the 10 people and assigning each of them a 
2, 3 or a 5 to determine their subgroup.
The total number of such assignments is the number of 
permutations of the digits
$2233355555$ which is the indicated multinomial coefficient.
\end{example}

\begin{problem}(problem 1)
How many ways can a team of 15 players be split into three teams of 5 players each? $\; \answer{756756}$
\end{problem}

\begin{theorem}[Multinomial Theorem]
For a natural number $n$ and real numbers $x_1, x_2, \dots x_k$ we have
\[
(x_1 + x_2 + \cdots + x_k)^n = \sum \binom{n}{n_1 \, n_2 \, \dots n_k} x_1^{n_1}x_2^{n_2}\cdots x_k^{n_k}
\]
where the sum runs over all possible non-negative integer values of $n_1, n_2, \dots, n_k$ whose sum is $n$.
\end{theorem}

\begin{remark}
From the stars and bars method, the number of distinct terms in the 
multinomial expansion is $C(n+k-1, n)$.
\end{remark}


\begin{example}[example 2]
Find the coefficient of $x^2y^4z$ in the expansion of $(x+y+z)^7$.\\
According to the Multinomial Theorem, the desired coefficient is
\[
\binom{7}{2\;4\;1} = \frac{7!}{2!4!1!} = 105
\]
\end{example}

\begin{problem}(problem 2)
Find the coefficient of the given term of the multinomial expansion:\\
a) $x^2yz^2$ in $(x+y+z)^5: \; \answer{30}$\\
a) $x^2yz^2$ in $(2x-y+3z)^5: \; \answer{-1080}$\\
a) $xy^2z^2w^3$ in $(2x+y+z-2w)^8: \; \answer{-26880}$\\
a) $x^3z^2$ in $(x+y+z)^6: \; \answer{0}$\\
\end{problem}

\section{Multinomial Identities}

\begin{example}[example 3]
Find the sum of all multinomial coefficients of the form $\dis \binom{6}{n_1 n_2 n_3}$.\\
If we let $x = 1, y = 1$ and $z = 1$ in the expansion of $(x+y+z)^6$, the Multinomial theorem gives
\[
(1+1+1)^6 = \sum \binom{6}{n_1 \, n_2 \, n_3} \, 1^{n_1}\, 1^{n_2}\, 1^{n_3}
\]
where the sum runs over all possible non-negative integer values of $n_1, n_2$ and $n_3$ whose sum is 6.
Thus, the sum of all multinomial coefficients of the form $\dis \binom{6}{n_1 n_2 n_3}$ is $3^6 = 729$.
\end{example}


\begin{problem}(problem 3)
Find the indicated sum.\\
a) The sum of all multinomial coefficients of the form $\dis \binom{6}{n_1\, n_2\, n_3\, n_4}\; \answer{4096}$.\\[8pt]
b) The sum of all multinomial coefficients of the form $\dis \binom{7}{n_1\, n_2\, n_3} \; \answer{2143}$.\\

\end{problem}
 
\begin{proposition}[Sum of Multinomial Coefficients]
In general,
\[
\sum \binom{n}{n_1 \, n_2 \cdots \, n_k} = k^n
\]
where the sum runs over all non-negative values of $n_1, n_2, \dots, n_k$ whose sum is $n$.
\end{proposition}

\begin{proof}
The result follows from letting $x_1 = 1, x_2 = 1, \dots, x_k = 1$ in the multinomial expansion of 
$(x_1 + x_2 + \cdots + x_k)^n$.
\end{proof}

\begin{problem}(problem 4a)
Prove that
\[
\sum \binom{n}{n_1 \, n_2 \, n_3} (-1)^{n_3} = 1
\]
where the sum runs over all non-negative values of $n_1, n_2, n_3$ whose sum is $n$.
\end{problem}

\begin{problem}(problem 4b)
Prove that
\[
\sum \binom{n}{n_1 \, n_2 \, n_3 \, n_4} (-1)^{n_3 + n_4} = 0
\]
where the sum runs over all non-negative values of $n_1, n_2, n_3$ whose sum is $n$.
\end{problem}



\begin{proposition}[Pascal's Identity for Multinomial Coefficients]
\[
\binom{n}{n_1 \, n_2 \cdots \, n_k} = \binom{n-1}{n_1 -1 \, n_2 \cdots \, n_k} + \binom{n-1}{n_1 \, n_2 -1 \cdots \, n_k} + \cdots+ \binom{n-1}{n_1 \, n_2 \cdots \, n_k -1 }
\]
\end{proposition}

\begin{proof}
The result follows from a double counting argument for the number of ways to select 
subgroups of size $n_1, n_2, \dots, n_k$ from a group of size $n$ where $n_1 + n_2 + \cdots + n_k = n$. 
The left hand side of the identity gives this directly.  The right hand side is obtained by 
considering one of the $n$ people as special and partitioning the collection of all subgroups 
according to which subgroup the special person is a member of.
\end{proof}



%Pascal's identity and chaired comm identity


\end{document}

%C(n,k) number of ways to break n-group into k and n-k -subgroups
%C(n; n_1, n_2, ..., n_k) number of ways to break n-group into n_1, n_2, ...n_k-subgroups
%compute multinom coeff
%multnom theorem
%find coeff of multinom expansion
%sum of multinom coeff C(n; n_1, ..., n_k) = k^n

