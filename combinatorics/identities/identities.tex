\documentclass[handout]{ximera}

%% You can put user macros here
%% However, you cannot make new environments



\newcommand{\ffrac}[2]{\frac{\text{\footnotesize $#1$}}{\text{\footnotesize $#2$}}}
\newcommand{\vasymptote}[2][]{
    \draw [densely dashed,#1] ({rel axis cs:0,0} -| {axis cs:#2,0}) -- ({rel axis cs:0,1} -| {axis cs:#2,0});
}


\graphicspath{{./}{firstExample/}}

\usepackage{amsmath}
\usepackage{amssymb}
\usepackage{array}
\usepackage[makeroom]{cancel} %% for strike outs
\usepackage{pgffor} %% required for integral for loops
\usepackage{tikz}
\usepackage{tikz-cd}
\usepackage{tkz-euclide}
\usetikzlibrary{shapes.multipart}


\usetkzobj{all}
\tikzstyle geometryDiagrams=[ultra thick,color=blue!50!black]


\usetikzlibrary{arrows}
\tikzset{>=stealth,commutative diagrams/.cd,
  arrow style=tikz,diagrams={>=stealth}} %% cool arrow head
\tikzset{shorten <>/.style={ shorten >=#1, shorten <=#1 } } %% allows shorter vectors

\usetikzlibrary{backgrounds} %% for boxes around graphs
\usetikzlibrary{shapes,positioning}  %% Clouds and stars
\usetikzlibrary{matrix} %% for matrix
\usepgfplotslibrary{polar} %% for polar plots
\usepgfplotslibrary{fillbetween} %% to shade area between curves in TikZ



%\usepackage[width=4.375in, height=7.0in, top=1.0in, papersize={5.5in,8.5in}]{geometry}
%\usepackage[pdftex]{graphicx}
%\usepackage{tipa}
%\usepackage{txfonts}
%\usepackage{textcomp}
%\usepackage{amsthm}
%\usepackage{xy}
%\usepackage{fancyhdr}
%\usepackage{xcolor}
%\usepackage{mathtools} %% for pretty underbrace % Breaks Ximera
%\usepackage{multicol}



\newcommand{\RR}{\mathbb R}
\newcommand{\R}{\mathbb R}
\newcommand{\C}{\mathbb C}
\newcommand{\N}{\mathbb N}
\newcommand{\Z}{\mathbb Z}
\newcommand{\dis}{\displaystyle}
%\renewcommand{\d}{\,d\!}
\renewcommand{\d}{\mathop{}\!d}
\newcommand{\dd}[2][]{\frac{\d #1}{\d #2}}
\newcommand{\pp}[2][]{\frac{\partial #1}{\partial #2}}
\renewcommand{\l}{\ell}
\newcommand{\ddx}{\frac{d}{\d x}}

\newcommand{\zeroOverZero}{\ensuremath{\boldsymbol{\tfrac{0}{0}}}}
\newcommand{\inftyOverInfty}{\ensuremath{\boldsymbol{\tfrac{\infty}{\infty}}}}
\newcommand{\zeroOverInfty}{\ensuremath{\boldsymbol{\tfrac{0}{\infty}}}}
\newcommand{\zeroTimesInfty}{\ensuremath{\small\boldsymbol{0\cdot \infty}}}
\newcommand{\inftyMinusInfty}{\ensuremath{\small\boldsymbol{\infty - \infty}}}
\newcommand{\oneToInfty}{\ensuremath{\boldsymbol{1^\infty}}}
\newcommand{\zeroToZero}{\ensuremath{\boldsymbol{0^0}}}
\newcommand{\inftyToZero}{\ensuremath{\boldsymbol{\infty^0}}}


\newcommand{\numOverZero}{\ensuremath{\boldsymbol{\tfrac{\#}{0}}}}
\newcommand{\dfn}{\textbf}
%\newcommand{\unit}{\,\mathrm}
\newcommand{\unit}{\mathop{}\!\mathrm}
%\newcommand{\eval}[1]{\bigg[ #1 \bigg]}
\newcommand{\eval}[1]{ #1 \bigg|}
\newcommand{\seq}[1]{\left( #1 \right)}
\renewcommand{\epsilon}{\varepsilon}
\renewcommand{\iff}{\Leftrightarrow}

\DeclareMathOperator{\arccot}{arccot}
\DeclareMathOperator{\arcsec}{arcsec}
\DeclareMathOperator{\arccsc}{arccsc}
\DeclareMathOperator{\si}{Si}
\DeclareMathOperator{\proj}{proj}
\DeclareMathOperator{\scal}{scal}
\DeclareMathOperator{\cis}{cis}
\DeclareMathOperator{\Arg}{Arg}
%\DeclareMathOperator{\arg}{arg}
\DeclareMathOperator{\Rep}{Re}
\DeclareMathOperator{\Imp}{Im}
\DeclareMathOperator{\sech}{sech}
\DeclareMathOperator{\csch}{csch}
\DeclareMathOperator{\Log}{Log}

\newcommand{\tightoverset}[2]{% for arrow vec
  \mathop{#2}\limits^{\vbox to -.5ex{\kern-0.75ex\hbox{$#1$}\vss}}}
\newcommand{\arrowvec}{\overrightarrow}
\renewcommand{\vec}{\mathbf}
\newcommand{\veci}{{\boldsymbol{\hat{\imath}}}}
\newcommand{\vecj}{{\boldsymbol{\hat{\jmath}}}}
\newcommand{\veck}{{\boldsymbol{\hat{k}}}}
\newcommand{\vecl}{\boldsymbol{\l}}
\newcommand{\utan}{\vec{\hat{t}}}
\newcommand{\unormal}{\vec{\hat{n}}}
\newcommand{\ubinormal}{\vec{\hat{b}}}

\newcommand{\dotp}{\bullet}
\newcommand{\cross}{\boldsymbol\times}
\newcommand{\grad}{\boldsymbol\nabla}
\newcommand{\divergence}{\grad\dotp}
\newcommand{\curl}{\grad\cross}
%% Simple horiz vectors
\renewcommand{\vector}[1]{\left\langle #1\right\rangle}


\pgfplotsset{compat=1.13}

\outcome{Use combinatorial reasoning to establish identities}

\title{1.8 Combinatorial Identities}

\begin{document}

\begin{abstract}
We use combinatorial reasoning to prove identities.
\end{abstract}

\maketitle

\section{Combinatorial Identities}

\begin{example}[example 1]
Use combinatorial reasoning to establish the identity
\[
\binom{n}{k}  = \binom{n}{n-k}
\]
We will use bijective reasoning, i.e., we will show a one-to-one correspondence between objects to 
conclude that they must be equal in number.
Suppose a group of $n$ people is split into two groups. If the first group consists of $k$ people, 
then the second group must consist of $n-k$ people and vice versa.
 Hence groups of size $k$ and $n-k$ taken from a group of size $n$ must be equal in number. Thus
\[
\binom{n}{k}  = \binom{n}{n-k}
\]
\end{example}



\begin{example}[example 2]
Use combinatorial reasoning to establish Pascal's Identity:
\[
\binom{n}{k-1} + \binom{n}{k} = \binom{n+1}{k}
\]
This identity is the basis for creating Pascal's triangle.\\
To establish the identity we will use a double counting argument. 
That is we will pose a counting problem and reason its 
solution two different ways- one which yields the left hand side of the identity and the other which
 yields the right hand side. For the identity above, we begin with the right hand 
 side since it is simpler.  The quantity
 \[
 \binom{n+1}{k}
 \]
 counts the number of committees of $k$ people that can be selected from a group of $n+1$ people. 
 Hence, we must find another way of counting this same thing which yields the left hand side. 
 Reason as follows. Regard one of the $n+1$ people as special. 
 Then the committee either contains the special person or it does not.
 The number of committees which contain the special person is 
 \[
 \binom{n}{k-1}
 \]
 since $k-1$ other committee members must be selected from the other $n$ people.
 Then, the number of committees which do not contain the special person is 
 \[
 \binom{n}{k}
 \]
 since then entire $k$ person committee members must be selected from the other $n$ people.
 The total number of possible committees is thus
 \[
\binom{n}{k-1} + \binom{n}{k}
\]
Since we have two different forms for the solution to the same problem, they must be equal:
\[
\binom{n}{k-1} + \binom{n}{k} = \binom{n+1}{k}
\]
and the identity is established.
\end{example}

\begin{example}[example 3]
Use combinatorial reasoning to establish the identity
\[
k \binom{n}{k} = n\binom{n-1}{k-1}
\]
The left hand side represents the number of ways to select a committee of $k$ people from a group 
of $n$ people times the number of ways to select one member from that committee to be the chair. 
The right hand side represents the number of ways to select one person from the group of $n$ to be the 
chair of a committee times the number of ways to select $k-1$ other committee members from 
the remaining $n-1$ people.  In either case, we are counting the number of ways to selected 
a chaired committee of $k$ people from a group of $n$ people.
\end{example}

\begin{problem}(problem 3)
Use combinatorial reasoning to establish the identity
\[
\binom{n}{m} \binom{m}{k} = \binom{n}{k} \binom{n-m}{m-k}
\]
\begin{hint}
Use a double counting argument
\end{hint}
\begin{hint}
The left hand side represents the number of ways to select an $m$ member committee from a group of $n$ people and then 
select $k$ committee members to be on a sub-committee
\end{hint}

\begin{hint}
To get the right hand side, choose the sub-committee first
\end{hint}
\end{problem}

\begin{example}[example 4]
Use combinatorial reasoning to establish the identity
\[
\sum_{k=0}^n \binom{n}{k} = 2^n
\]
Let $S = \{1,2,3,...,n\}$. The number of subsets of $S$ can be counted in two different ways.
First, we can use the Fundamental Principle of Counting element by element. 
In a given subset, an element is either in it or not in it, giving 2 choices. 
Creating the subset requires making this decision for each of the elements in the set. 
The number of subsets that can be created (including the empty set) 
is $2 \cdot 2 \cdot \ldots \dot 2$ where there are $n$ factors of $2$ in the product.  
Hence the number of possible subsets is $2^n$.
On the other hand, we can create all of the possible subsets of $S$ by partitioning them 
into groups of the same size.
The number of subsets of $S$ of size $k$ is $\binom{n}{k}$ since we have to choose $k$ items from a 
set of $n$ items without regard to the order of selection. Thus the total number of possible 
subsets is obtained by summing over all possible values of $k$:
\[
\sum_{k=0}^n \binom{n}{k}
\]
This double counting argument establishes the identity
\[
\sum_{k=0}^n \binom{n}{k} = 2^n
\]

\end{example}

\begin{example}[example 5]
Use combinatorial reasoning to establish the Hockey Stick Identity:
\[
\sum_{k=r}^{n}\binom{k}{r}=\binom{n+1}{r+1} %\qquad {\text{ for }}n,r\in \mathbb {N} ,\quad n\geq r
\]
The right hand side counts the number of ways to form a committee of $k+1$ people from a group of $n+1$ people.
To establish this identity we will double count this by assigning each of the $n+1$ people a unique 
integer from $1$ to $n+1$ and then partitioning the committees according to the largest number 
assigned among its members.
Since there are $r+1$ members on the committee, the largest number assigned among the 
members in a particular committee ranges from $r+1$ to $n+1$.
The number of committees that can be formed having largest number $k+1$ among its members
is $\binom{k}{r}$ since we must choose
$r$ other members (in addition to the member numbered $k+1$) from those numbered $1$ to $k$. 
Since $k$ ranges 
from $r$ to $n$ we have that the total number of possible committees is
\[
\sum_{k=r}^{n}\binom{k}{r}
\]
By double counting, we have established the identity
\[
\sum_{k=r}^{n}\binom{k}{r}=\binom{n+1}{r+1}
\]
This is called the hockey stick identity due to the shape of the binomial 
coefficients involved when highlighted in Pascal's Triangle.
\end{example}


\begin{problem}(problem 1)
Use combinatorial reasoning to establish the identity
\[
P(n,k) = k!\binom{n}{k}
\]
\begin{hint}
Create the permutations by first creating a combination and then permuting the selected items
\end{hint}
\end{problem}



\begin{problem}(problem 2)
Use combinatorial reasoning to establish the identity
\[
\sum_{k=1}^n k\binom{n}{k} = n2^{n-1}
\]
\begin{hint}
Use a double counting argument
\end{hint}
\begin{hint}
The right hand side is the number of ways to select an element from a set of $n$ elements 
followed by any subset of the remaining $n-1$ elements
\end{hint}
\begin{hint}
Think of the element selected first as a committee chair
\end{hint}
\begin{hint}
The committee can be of any size, from $1$ to $n$
\end{hint}
\begin{hint}
Partition the committees according to the number of members (including the chair)
\end{hint}
\end{problem}


\begin{problem}(problem 3)
Use combinatorial reasoning to establish the identity
\[
\sum_{k=0}^n \binom{n}{k}^2 = \binom{2n}{n}
\]
\begin{hint}
Use a double counting argument
\end{hint}
\begin{hint}
The right hand side represents the number of committees of size $n$ that can be formed from $2n$ people
\end{hint}
\begin{hint}
Assume that half of the people are men and half are women
\end{hint}
\begin{hint}
Partition the possible committees according to how many women are on it
\end{hint}
\begin{hint}
Use the fact that $\dis \binom{n}{k} = \binom{n}{n-k}$
\end{hint}
\end{problem}

\begin{problem}(problem 4)
Use combinatorial reasoning to establish the identity
\[
\sum_{k=0}^n \binom{n_1}{k}\binom{n_2}{n-k} = \binom{n_1+n_2}{n}
\]
\begin{hint}
Modify the argument for problem 3
\end{hint}
\begin{hint}
Assume that there are $n_1$ men and $n_2$ women on the committee
\end{hint}
\end{problem}

\begin{problem}(problem 3)
Use combinatorial reasoning to establish the identity
\[
\sum_{k=0}^n k\binom{n}{k}^2 = n\binom{2n-1}{n-1}
\]
\begin{hint}
Modify the argument for problem 3
\end{hint}
\begin{hint}
Include a chairperson on the committee
\end{hint}
\end{problem}

\end{document}









\begin{problem}(problem 1a)
Sam is getting dressed and wants to include a belt and a hat.  
If Sam has three distinct belts and two distinct hats, how many ways
can Sam accessorize? Include a tree diagram that illustrates the possibilities.\\
Sam can accessorize in $\; \answer{6} \;$ different ways.
\end{problem}

\begin{problem}(problem 1b)
A couple wants to go on a date.  They decide to  eat at a restaurant, walk in a park and then see a show.
If there are two restaurants that the y both like, two nearby parks and two shows playing, 
how many possibilities are there for their date? Make a tree diagram illustrating the possibilities.\\
There are $\;\answer{8}\;$ different possibilities for their date.
\end{problem}


\begin{theorem}[Fundamental Principle of Counting]
Suppose that $n$ decisions are to be made and that the number of choices for the $k^{th}$ decision is $d_k$.
Then the total number of ways to make the $n$ decisions is
\[
\prod_{k=1}^n d_k = d_1 \cdot d_2 \cdot \ldots \cdot d_n
\]
\end{theorem}

\begin{example}
There are three types of Tesla Model 3: Standard Range, Long Range and Performance. 
Each type can be painted one of 5 colors. Furthermore, two interior color schemes are available. 
Given these parameters, how many choices are there for my Tesla Model 3 purchase?\\
In total, three decisions must be made: the type, the paint color and the interior color scheme.
The number of choices for each decision is 3, 5 and 2 respectively. 
Hence, by the Fundamental Principle of Counting, the total number of ways to configure my new 
Tesla Model 3 is $3 \cdot 5\cdot 2 = 30$.
\begin{image}
\begin{tikzpicture}

\draw (0,0) -- (2,0) node[blue, below, midway]{Type} node[above, midway]{3} node[right]{$\times$};
\draw (2.5,0) -- (4.5,0) node[blue, below, midway]{Paint} node[above, midway]{5} node[right]{$\times$};
\draw (5,0) -- (7,0) node[blue, below, midway]{Interior} node[above, midway]{2} node[right]{$=$};
\draw (7.5,0) -- (9.5,0) node[blue, below, midway]{Total} node[above, midway]{30} ;

\end{tikzpicture}
\end{image}
\end{example}


\begin{problem}(problem 2a)
A three course meal consists of an appetizer, an entree and a dessert. 
If a restaurant offers a dozen appetizers, a score of entrees and a bakers dozen desserts, 
how many possible three course meals can be ordered?\\
The total number of possible three course meals is $\answer{3120}$.
\end{problem}

\begin{problem}(problem 2b)
A sixth grade class of 20 students would like to select a President, Vice-President, Secretary and Treasurer.
If no student may serve in more than one post, in how many possible ways can the class officers be selected?\\
The total number of ways to select the class officers is $\answer{116280}$.
\end{problem}

\end{document}




















