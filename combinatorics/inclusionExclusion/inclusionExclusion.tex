\documentclass[handout]{ximera}

%% You can put user macros here
%% However, you cannot make new environments



\newcommand{\ffrac}[2]{\frac{\text{\footnotesize $#1$}}{\text{\footnotesize $#2$}}}
\newcommand{\vasymptote}[2][]{
    \draw [densely dashed,#1] ({rel axis cs:0,0} -| {axis cs:#2,0}) -- ({rel axis cs:0,1} -| {axis cs:#2,0});
}


\graphicspath{{./}{firstExample/}}

\usepackage{amsmath}
\usepackage{amssymb}
\usepackage{array}
\usepackage[makeroom]{cancel} %% for strike outs
\usepackage{pgffor} %% required for integral for loops
\usepackage{tikz}
\usepackage{tikz-cd}
\usepackage{tkz-euclide}
\usetikzlibrary{shapes.multipart}


\usetkzobj{all}
\tikzstyle geometryDiagrams=[ultra thick,color=blue!50!black]


\usetikzlibrary{arrows}
\tikzset{>=stealth,commutative diagrams/.cd,
  arrow style=tikz,diagrams={>=stealth}} %% cool arrow head
\tikzset{shorten <>/.style={ shorten >=#1, shorten <=#1 } } %% allows shorter vectors

\usetikzlibrary{backgrounds} %% for boxes around graphs
\usetikzlibrary{shapes,positioning}  %% Clouds and stars
\usetikzlibrary{matrix} %% for matrix
\usepgfplotslibrary{polar} %% for polar plots
\usepgfplotslibrary{fillbetween} %% to shade area between curves in TikZ



%\usepackage[width=4.375in, height=7.0in, top=1.0in, papersize={5.5in,8.5in}]{geometry}
%\usepackage[pdftex]{graphicx}
%\usepackage{tipa}
%\usepackage{txfonts}
%\usepackage{textcomp}
%\usepackage{amsthm}
%\usepackage{xy}
%\usepackage{fancyhdr}
%\usepackage{xcolor}
%\usepackage{mathtools} %% for pretty underbrace % Breaks Ximera
%\usepackage{multicol}



\newcommand{\RR}{\mathbb R}
\newcommand{\R}{\mathbb R}
\newcommand{\C}{\mathbb C}
\newcommand{\N}{\mathbb N}
\newcommand{\Z}{\mathbb Z}
\newcommand{\dis}{\displaystyle}
%\renewcommand{\d}{\,d\!}
\renewcommand{\d}{\mathop{}\!d}
\newcommand{\dd}[2][]{\frac{\d #1}{\d #2}}
\newcommand{\pp}[2][]{\frac{\partial #1}{\partial #2}}
\renewcommand{\l}{\ell}
\newcommand{\ddx}{\frac{d}{\d x}}

\newcommand{\zeroOverZero}{\ensuremath{\boldsymbol{\tfrac{0}{0}}}}
\newcommand{\inftyOverInfty}{\ensuremath{\boldsymbol{\tfrac{\infty}{\infty}}}}
\newcommand{\zeroOverInfty}{\ensuremath{\boldsymbol{\tfrac{0}{\infty}}}}
\newcommand{\zeroTimesInfty}{\ensuremath{\small\boldsymbol{0\cdot \infty}}}
\newcommand{\inftyMinusInfty}{\ensuremath{\small\boldsymbol{\infty - \infty}}}
\newcommand{\oneToInfty}{\ensuremath{\boldsymbol{1^\infty}}}
\newcommand{\zeroToZero}{\ensuremath{\boldsymbol{0^0}}}
\newcommand{\inftyToZero}{\ensuremath{\boldsymbol{\infty^0}}}


\newcommand{\numOverZero}{\ensuremath{\boldsymbol{\tfrac{\#}{0}}}}
\newcommand{\dfn}{\textbf}
%\newcommand{\unit}{\,\mathrm}
\newcommand{\unit}{\mathop{}\!\mathrm}
%\newcommand{\eval}[1]{\bigg[ #1 \bigg]}
\newcommand{\eval}[1]{ #1 \bigg|}
\newcommand{\seq}[1]{\left( #1 \right)}
\renewcommand{\epsilon}{\varepsilon}
\renewcommand{\iff}{\Leftrightarrow}

\DeclareMathOperator{\arccot}{arccot}
\DeclareMathOperator{\arcsec}{arcsec}
\DeclareMathOperator{\arccsc}{arccsc}
\DeclareMathOperator{\si}{Si}
\DeclareMathOperator{\proj}{proj}
\DeclareMathOperator{\scal}{scal}
\DeclareMathOperator{\cis}{cis}
\DeclareMathOperator{\Arg}{Arg}
%\DeclareMathOperator{\arg}{arg}
\DeclareMathOperator{\Rep}{Re}
\DeclareMathOperator{\Imp}{Im}
\DeclareMathOperator{\sech}{sech}
\DeclareMathOperator{\csch}{csch}
\DeclareMathOperator{\Log}{Log}

\newcommand{\tightoverset}[2]{% for arrow vec
  \mathop{#2}\limits^{\vbox to -.5ex{\kern-0.75ex\hbox{$#1$}\vss}}}
\newcommand{\arrowvec}{\overrightarrow}
\renewcommand{\vec}{\mathbf}
\newcommand{\veci}{{\boldsymbol{\hat{\imath}}}}
\newcommand{\vecj}{{\boldsymbol{\hat{\jmath}}}}
\newcommand{\veck}{{\boldsymbol{\hat{k}}}}
\newcommand{\vecl}{\boldsymbol{\l}}
\newcommand{\utan}{\vec{\hat{t}}}
\newcommand{\unormal}{\vec{\hat{n}}}
\newcommand{\ubinormal}{\vec{\hat{b}}}

\newcommand{\dotp}{\bullet}
\newcommand{\cross}{\boldsymbol\times}
\newcommand{\grad}{\boldsymbol\nabla}
\newcommand{\divergence}{\grad\dotp}
\newcommand{\curl}{\grad\cross}
%% Simple horiz vectors
\renewcommand{\vector}[1]{\left\langle #1\right\rangle}


\pgfplotsset{compat=1.13}

\outcome{Use the Inclusion-Exclusion Principle to count the number of elements in a set}

\title{2.2 Inclusion-Exclusion Principle}

\begin{document}



\begin{abstract}
We use the Inclusion-Exclusion Principle to enumerate sets.
\end{abstract}

\maketitle


Recall the Inclusion-Exclusion Principle for two sets:

\[
|A \cup B| = |A| + |B| - |A \cap B|
\]

In this section, we will generalize this to $n$ sets, but first, let's extend it from two sets to three sets.

\begin{proposition}
Given sets $A, B$ and $C$ with finite cardinalities $|A|, |B|$ and $|C|$, we have
\[
|A \cup B \cup C| = |A| + |B| + |C| - |A \cap B| - |A \cap C| - |B \cap C| + |A \cap B \cap C|
\]
\end{proposition}




\begin{proof}
Consider $A \cup B$ as one set and $C$ as the second set and apply the 
Inclusion-Exclusion Principle for two sets.  We have:
\[
|A \cup B \cup C| = |A \cup B| + |C| - | (A\cup B) \cap C|
\]
Next, use the Inclusion-Exclusion Principle for two sets on the first term, and 
distribute the intersection across the union in the third term to obtain:
\[
|A \cup B \cup C| = |A| + |B| - |A \cap B| + |C| - | (A \cap C) \cup (B\cap C) |
\]
Now, use the Inclusion Exclusion Principle for two sets on the fourth term to get:
\[
|A \cup B \cup C| = |A| + |B| - |A \cap B| + |C| - \bigg(| (A \cap C)| + |B\cap C| - |(A\cap B) \cap (B\cap C)| \bigg)
\]
Finally, the set in the last term is just $A \cap B \cap C$, so we have the final form of the 
Inclusion-Exclusion Principle for three sets:
\[
|A \cup B \cup C| = |A| + |B| + |C| - |A \cap B| - | (A \cap C)| - |B\cap C| +|A\cap B\cap C|
\]
\end{proof}

\begin{example}[example 1]
How many numbers in the set $S = \{1,2,3, \ldots, 1000\}$ are multiples of either 2, 5 or 9?\\
Let $A$ be the subset of $S$ consisting of the multiples of 2, $B$ be the subset of $S$ consisting of the multiples of 5,
and $C$ be the subset of $S$ consisting of multiples of 9.
We seek $|A \cup B \cup C|$. According to the proposition, 
\[
|A \cup B \cup C| = |A| + |B| +|C| - |A \cap B|- |A \cap C|- |B \cap C| + |A \cap B \cap C|
\]
We have $|A| =500,|B| = 200$ and $|C| = 111$. Elements in $A \cap B$ are multiples of 10, so $|A\cap B| = 100$.
Elements in $A \cap C$ are multiples of 18, so $|A\cap B| = 55$.
Elements in $B \cap C$ are multiples of 45, so $|B\cap C| = 22$.  
Lastly, elements in $A \cap B \cap C$ are multiples of 90, so $|A \cap B\cap C| = 11$. 
Using the Inclusion-Exclusion Principle (for three sets), we can conclude 
that the number of elements of $S$ that are either multiples of 2, 5 or 9 is
\[
|A \cup B \cup C| = 500 + 200 + 111 - 100 - 55 -22 + 11 = 645
\]
\end{example}

\begin{problem}(problem 1)
How many numbers from the given set $S= \{1,2,3, \ldots, 1000\}$ are multiples of the given numbers $a, b$ and $c$?\\
a)$\; a = 2, b = 3 , c = 5\;\; \answer{734}$\\
b)$\;  a = 4, b= 5, c = 6 \;\; \answer{466}$\\
c)$\; a = 6, b= 8, c = 10 \;\; \answer{300}$\\

\end{problem}


We now extend De Morgan's Law to three sets.
\begin{proposition}[De Morgan's Law (for three sets)]
Given sets $A, B$ and $C$ in a universal set $U$, De Morgan's Law states that
\[
\left(A \cup B \cup C\right)^c = A^c \cap B^c \cap C^c
\]
\end{proposition}

\begin{proof}
Using De Morgan's Law for two sets twice yields the result:
\begin{align*}
\left(A \cup B \cup C\right)^c &= \left[(A \cup B) \cup C\right]^c \\
        &= (A \cup B)^c \cap C^c \\
        &= A^c \cap B^c \cap C^c
\end{align*}
as desired.
\end{proof}



The next example uses De Morgan's Law (for three sets) in conjunction with the Rule for Complements and the 
Inclusion-Exclusion Principle (for three sets).

\begin{example}[example 2]
Computer passwords are eight characters long where a character can be either an upper case letter, 
lower case letter, digit 0-9, or one of 8 special symbols. 
How many passwords are possible if a password must contain at least one digit, at least one special symbol
and at least one capital letter?\\

Let $A$ be the set of passwords that do not contain any digits; let $B$ be the set of 
passwords that do not contain any special symbols; and let $C$ be the set of passwords 
that do not contain any capital letters. 
Since our passwords must contain a digit, a special symbol and a capital letter, we seek $|A^c \cap B^c \cap C^c|$.
By De Morgan's Law (for three sets), 
\[
A^c \cap B^c  \cap C^c= (A \cup B \cup C)^c
\]
Then according to the Rule for Complements,
\[
|(A \cup B \cup C)^c| = |S| - |A \cup B \cup C|,
\]
where $S$ is the set of all possible passwords with no restrictions. 
Combining the two equations above with the Inclusion-Exclusion Principle (for three sets), we have
\[
|A^c \cap B^c \cap C^c| = |S| - |A| - |B| - |C| + |A\cap B|+ |A\cap C|+ |B\cap C|- |A\cap B \cap C|
\]
Each of the terms on the right hand side can be computed using the Fundamental Principle of Counting. Since there are
70 different characters in total with 10 digits, 8 special symbols and 26 capital letters,
\[
|S| = 70^8, \; |A| = 60^8, \; |B| = 62^8, |C| = 44^8 \; \text{ and } 
\]
\[
 |A\cap B| = 52^8, |A\cap C| = 34^8, |B\cap C| = 36^8 \; \text{ and }
 \]
 \[
 |A\cap B\cap C| = 26^8
\]
In conclusion, the number of acceptable passwords is
\[
70^8 - 60^8 - 62^8 - 44^8 + 52^8 + 34^8 + 36^8 - 26^8 \approx 233987976069000
\]

\end{example}

\begin{problem}(problem 2)
Computer passwords are eight characters long where a character can be either an upper case letter, 
lower case letter, digit 0-9, or one of 8 special symbols. 
How many passwords are possible if a password must contain at least one digit, at least one lower case 
letter and at least one upper case letter? $\; \answer{70^8 - 60^8 - 44^8- 44^8 + 34^8 + 34^8 + 18^8 -8^8}$\\

\end{problem}









\begin{example}[example 3]
How many permutations are there of the letters in the phrase MATH IS FUN in which none of the
words MATH, IS or FUN appears?\\
Such permutations include MITFNSHAU and FASTHUMIN, but not IMATHUNSF, FMAISHNUT, or HTFUNISAM.
Let $A$ be the set of permutations which include the word MATH; let $B$ be the set of permutations which include the word
IS; and let $C$ be the set of permutations which include the word FUN.  We seek $|A^c \cap B^c \cap C^c|$. 
By De Morgan's Law (for three sets), this is the same as $|(A \cup B\cup C)^c|$. 
Let $S$ be the set of all permutations of the letters in the phrase. Then, applying the Rule for Complements,
and the Inclusion-Exclusion Principle (for three sets), we have
\begin{align*}
|A^c \cap B^c \cap C^c| &= |(A \cup B \cup C)^c|\\
               &= |S| - |A\cup B \cup C|\\
               &= |S| - \Big(|A| + |B| + |C| -|A\cap B|-|A\cap C|-|B\cap C|+|A\cap B \cap C|\Big)\\
               &= |S| -|A| - |B| - |C| +|A\cap B|+|A\cap C|+|B\cap C|-|A\cap B \cap C|
\end{align*}
The number of elements in $A$ is the number of permutations of the symbols MATH, I, S, F, U, N.  
Since this is 6 distinct symbols, $|A| = 6!$. 
The number of elements in $B$ is the number of 
permutations of the symbols M, A, T, H, IS, F, U, N.  
Since this is 8 distinct symbols, $|B| = 8!$. 
The number of elements in $C$ is the number of 
permutations of the symbols M, A, T, H, I, S, FUN.  
Since this is 7 distinct symbols, $|C| = 7!$.
The number of elements in $A \cap B$ is the number of permutations of the symbols MATH, IS, F, U, N.  
Since this is 5 distinct symbols, $|A \cap B| = 5!$.
The number of elements in $A \cap C$ is the number of permutations of the symbols MATH, I, S, FUN.  
Since this is 4 distinct symbols, $|A \cap C| = 4!$.
The number of elements in $B \cap C$ is the number of permutations of the symbols M, A, T, H, IS, FUN.  
Since this is 6 distinct symbols, $|B \cap C| = 6!$.
The number of elements in $A \cap B\cap C$ is the number of permutations of the symbols MATH, IS, FUN.  
Since this is 3 distinct symbols, $|A \cap B\cap C| = 3!$.
The number of elements in $S$
is $9!$. Hence, the total number of ways to permute the letters in the phrase MATH IS FUN which do not 
include any of the words MATH, IS or FUN is
\[
|A^c \cap B^c \cap C^c| = 9! - 6! - 8! - 7! + 5! + 4! + 6! -3! = 317658
\]
\end{example}


\begin{problem}(problem 3)
How many permutations are there of the letters in the phrase YOU HAD TIME in which do not include any of the  
 words YOU, HAD or TIME? $\; \answer{3344074}$
\end{problem}

\begin{example}[example 4]
How many permutations are there of the numbers 1-7 such that the 1 is not in the first position, 2 is not in the 
second position and 3 is not in the third position?\\
Let $A$ be the permutations with 1 in the first position, $B$ be the permutations with 2 in the 
second position, $C$ be the permutations with 3 in the third position and $S$ be the set of all permutations. 
We seek $|A^c \cap B^c \cap C^c|$. 
De Morgan's Law (for three sets), the Rule for Complements and the Inclusion-Exclusion Principle (for three sets) gives:
\begin{align*}
|A^c \cap B^c \cap C^c| &= |(A \cup B \cup C)^c|\\
               &= |S| - |A\cup B \cup C|\\
               &= |S| - \Big(|A| + |B| + |C| - |A\cap B|\ - |A\cap C| - |B\cap C| + |A \cap B \cap C| \Big)\\
               &= |S| - |A| - |B| - |C| + |A\cap B| + |A\cap C| + |B\cap C| - |A\cap B \cap C|\\
               &= 7! - 3 \times 6! + 3 \times 5! - 4!\\
               &= 3216
\end{align*}
Notice that in our computation we took advantage of the fact that certain sets had the same number of elements.
\end{example}

\begin{problem}(problem 4)
How many permutations are there of the letters in the word ABCDEF in which the letter A is not 
first and the letter F is not last? $\; \answer{426}$
\end{problem}





For four sets, the Inclusion-Exclusion Principle states:
\begin{align*}
|A \cup B \cup C \cup D | = &\;\;|A|  + |B| + |C| + |D| \\
&- |A \cap B| - | (A \cap C)| - |A \cap D| - |B\cap C| - | B \cap D| 
- |C \cap D| \\
& +|A\cap B\cap C| + |A\cap B\cap D|+|A\cap C\cap D|+| B\cap C\cap D| \\
&- |A\cap B\cap C \cap D|
\end{align*}
At this point, the number of terms is becoming quite large, so summation notation is to be preferred:
\[
|A_1 \cup A_2 \cup A_3 \cup A_4 | = \sum_{i = 1}^4 |A_i| - \sum_{1 = i < j}^4 |A_i \cap A_j| + 
\]
\[
\sum_{1 = i < j< k }^4 |A_i \cap A_j \cap A_k| - |A_1 \cap A_2 \cap A_3 \cap A_4 |
\]


\begin{example}[example 5]
How many non-negative integer solutions are there to the equation
\[
x_1 + x_2 + x_3 + x_4  = 10
\]
if $x_i \leq 3$ for $i = 1, 2, 3, 4$?\\
With no restrictions on the variables (except that they be non-negative integers), 
the total number of solutions is $|S| = C(13, 10)$ by the stars and bars method.
Let $A_i$ be the set of all solutions which have $x_i \geq 4$ for $i = 1, 2, 3, 4$.
We wish to find $|A_1^c \cap A_2^c \cap A_3^c \cap A_4^c|$.
De Morgan's Law (for four sets) and the Rule for Complements 
\begin{align*}
|A_1^c \cap A_2^c \cap A_3^c \cap A_4^c| &= |(A_1 \cup A_2 \cup A_3 \cup A_4)^c|\\
               &= |S| - |A_1 \cup A_2 \cup A_3 \cup A_4|
\end{align*}               
   
Now, the Inclusion-Exclusion Principle (for four sets) gives:
\[
|A_1^c \cap A_2^c \cap A_3^c \cap A_4^c| = |S| - \sum_{i = 1}^4 |A_i| + \sum_{1 = i < j}^4 |A_i \cap A_j|
\]
\[
-\sum_{1 = i < j< k }^4 |A_i \cap A_j \cap A_k| + |A_1 \cap A_2 \cap A_3 \cap A_4 |
\]
Since the conditions on the four variables is the same ($x_i \leq 3$), 
the number of elements in each intersection of a particular number of sets will be equal.
Thus, $|A_1| = |A_2| =|A_3| =|A_4| = C(9,6)$; $|A_i \cap A_j| = C(5, 2)$ for the six intersections of this form;
$|A_i \cap A_j \cap A_k| = 0$ for the four intersections of this form; and $|A_1 \cap A_2 \cap A_3 \cap A_4| = 0$.
Hence,
\[
|A_1^c \cap A_2^c \cap A_3^c \cap A_4^c| = \binom{13}{10} - 4\cdot \binom{9}{6} + 6 \cdot \binom{5}{2} - 4 \cdot 0 + 0 = 10 
\]






\end{example}

\begin{problem}(problem 5)
How many non-negative integer solutions are there to the equation
\[
x_1 + x_2 + x_3 + x_4 = 15
\]
if $x_i < 5$ for $i = 1, 2, 3, 4? \; \answer{576}$

\end{problem}



We are now ready to state and prove the general case of the Inclusion-Exclusion Principle.
\begin{proposition}[Inclusion-Exclusion Principle]
Given $n$ finite sets, $A_1, A_2, \ldots, A_n$ we have
\[
|A_1 \cup A_2 \cup \cdots \cup A_n| 
\]
\[
= \sum_{i =1}^n |A_i| - \sum_{1 =i<j}^n |A_i\cap A_j| +
\sum_{1 =i<j< k}^n |A_i\cap A_j \cap A_k| - \cdots + (-1)^{n-1} |A_1 \cap A_2 \cap \cdots \cap A_n|
\]
\end{proposition}

\begin{proof}
We will prove the proposition by induction on the number of sets, $n$. The base case, $n=2$ was proved in section 2.1.
For the induction hypothesis, we assume that the result is true for some number of sets $n$. 
We then wish to show that the result is true for $n+1$ sets. We will do this in a manner similar to the way we began
this section, obtaining the Inclusion-Exclusion Principle for three sets as a consequence of the result for two sets.
Thus, we consider the union of the first $n$ sets to be a single set and we obtain:
\[
|(A_1 \cup A_2 \cup \cdots \cup A_n) \cup A_{n+1}| 
\]
\[
= |A_1 \cup A_2 \cup \cdots \cup A_n| + |A_{n+1}| - 
|(A_1 \cup A_2 \cup \cdots \cup A_n) \cap A_{n+1}|
\]
In the last term, we can distribute the intersection over the unions to obtain a union of $n$ sets:
\[
(A_1 \cup A_2 \cup \cdots \cup A_n) \cap A_{n+1} = 
(A_1 \cap A_{n+1}) \cup (A_2 \cap A_{n+1}) \cup \cdots \cup (A_n \cap A_{n+1})
\]
We can apply the induction hypothesis the number of elements in this union, and noting 
that $(A_i \cap A_{n+1}) \cap (A_j \cap A_{n+1})  = (A_i \cap A_j \cap A_{n+1})$, we obtain
\[
|(A_1 \cap A_{n+1}) \cup (A_2 \cap A_{n+1}) \cup \cdots \cup (A_n \cap A_{n+1})|
\]
\[
= \sum_{i = 1}^n |A_i \cap A_{n+1}| - \sum_{1 = i < j}^n |A_i \cap A_j \cap A_{n+1}| + \cdots + (-1)^{n-1} |A_1 \cap A_2 \cap \cdots \cap A_{n+1}|
\]
Inserting this back into the first equation, and applying the induction hypothesis to the 
first term in that equation, we get
\begin{align*}
|A_1 \cup A_2 &\cup \cdots \cup A_n \cup A_{n+1}|   \\[6 pt]
& =|A_1 \cup A_2 \cup \cdots \cup A_n| + |A_{n+1}| - |(A_1 \cup A_2 \cup \cdots \cup A_n) \cap A_{n+1}| \\[7 pt]
 &= \bigg( \sum_{i =1}^n |A_i| - \sum_{1 =i<j}^n |A_i\cap A_j| \\
 &\quad +\sum_{1 =i<j< k}^n |A_i\cap A_j \cap A_k| - \cdots + (-1)^{n-1} |A_1 \cap A_2 \cap \cdots \cap A_n|\bigg) \\
  &\quad+ |A_{n+1}| - \bigg(\sum_{i = 1}^n |A_i \cap A_{n+1}| - \sum_{1 = i < j}^n |A_i \cap A_j \cap A_{n+1}| \\
  &\;\;+ \cdots + (-1)^{n-1} |A_1 \cap A_2 \cap \cdots \cap A_{n+1}|\bigg)\\
  &= \sum_{i =1}^{n+1} |A_i| - \sum_{1 =i<j}^{n+1} |A_i\cap A_j| \\
& \quad +\sum_{1 =i<j< k}^{n+1} |A_i\cap A_j \cap A_k| - \cdots + (-1)^{n} |A_1 \cap A_2 \cap \cdots \cap A_{n+1}| \\
\end{align*}
as required.

\end{proof}

\end{document}


