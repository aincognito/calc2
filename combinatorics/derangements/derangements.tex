\documentclass[handout]{ximera}

%% You can put user macros here
%% However, you cannot make new environments



\newcommand{\ffrac}[2]{\frac{\text{\footnotesize $#1$}}{\text{\footnotesize $#2$}}}
\newcommand{\vasymptote}[2][]{
    \draw [densely dashed,#1] ({rel axis cs:0,0} -| {axis cs:#2,0}) -- ({rel axis cs:0,1} -| {axis cs:#2,0});
}


\graphicspath{{./}{firstExample/}}

\usepackage{amsmath}
\usepackage{amssymb}
\usepackage{array}
\usepackage[makeroom]{cancel} %% for strike outs
\usepackage{pgffor} %% required for integral for loops
\usepackage{tikz}
\usepackage{tikz-cd}
\usepackage{tkz-euclide}
\usetikzlibrary{shapes.multipart}


\usetkzobj{all}
\tikzstyle geometryDiagrams=[ultra thick,color=blue!50!black]


\usetikzlibrary{arrows}
\tikzset{>=stealth,commutative diagrams/.cd,
  arrow style=tikz,diagrams={>=stealth}} %% cool arrow head
\tikzset{shorten <>/.style={ shorten >=#1, shorten <=#1 } } %% allows shorter vectors

\usetikzlibrary{backgrounds} %% for boxes around graphs
\usetikzlibrary{shapes,positioning}  %% Clouds and stars
\usetikzlibrary{matrix} %% for matrix
\usepgfplotslibrary{polar} %% for polar plots
\usepgfplotslibrary{fillbetween} %% to shade area between curves in TikZ



%\usepackage[width=4.375in, height=7.0in, top=1.0in, papersize={5.5in,8.5in}]{geometry}
%\usepackage[pdftex]{graphicx}
%\usepackage{tipa}
%\usepackage{txfonts}
%\usepackage{textcomp}
%\usepackage{amsthm}
%\usepackage{xy}
%\usepackage{fancyhdr}
%\usepackage{xcolor}
%\usepackage{mathtools} %% for pretty underbrace % Breaks Ximera
%\usepackage{multicol}



\newcommand{\RR}{\mathbb R}
\newcommand{\R}{\mathbb R}
\newcommand{\C}{\mathbb C}
\newcommand{\N}{\mathbb N}
\newcommand{\Z}{\mathbb Z}
\newcommand{\dis}{\displaystyle}
%\renewcommand{\d}{\,d\!}
\renewcommand{\d}{\mathop{}\!d}
\newcommand{\dd}[2][]{\frac{\d #1}{\d #2}}
\newcommand{\pp}[2][]{\frac{\partial #1}{\partial #2}}
\renewcommand{\l}{\ell}
\newcommand{\ddx}{\frac{d}{\d x}}

\newcommand{\zeroOverZero}{\ensuremath{\boldsymbol{\tfrac{0}{0}}}}
\newcommand{\inftyOverInfty}{\ensuremath{\boldsymbol{\tfrac{\infty}{\infty}}}}
\newcommand{\zeroOverInfty}{\ensuremath{\boldsymbol{\tfrac{0}{\infty}}}}
\newcommand{\zeroTimesInfty}{\ensuremath{\small\boldsymbol{0\cdot \infty}}}
\newcommand{\inftyMinusInfty}{\ensuremath{\small\boldsymbol{\infty - \infty}}}
\newcommand{\oneToInfty}{\ensuremath{\boldsymbol{1^\infty}}}
\newcommand{\zeroToZero}{\ensuremath{\boldsymbol{0^0}}}
\newcommand{\inftyToZero}{\ensuremath{\boldsymbol{\infty^0}}}


\newcommand{\numOverZero}{\ensuremath{\boldsymbol{\tfrac{\#}{0}}}}
\newcommand{\dfn}{\textbf}
%\newcommand{\unit}{\,\mathrm}
\newcommand{\unit}{\mathop{}\!\mathrm}
%\newcommand{\eval}[1]{\bigg[ #1 \bigg]}
\newcommand{\eval}[1]{ #1 \bigg|}
\newcommand{\seq}[1]{\left( #1 \right)}
\renewcommand{\epsilon}{\varepsilon}
\renewcommand{\iff}{\Leftrightarrow}

\DeclareMathOperator{\arccot}{arccot}
\DeclareMathOperator{\arcsec}{arcsec}
\DeclareMathOperator{\arccsc}{arccsc}
\DeclareMathOperator{\si}{Si}
\DeclareMathOperator{\proj}{proj}
\DeclareMathOperator{\scal}{scal}
\DeclareMathOperator{\cis}{cis}
\DeclareMathOperator{\Arg}{Arg}
%\DeclareMathOperator{\arg}{arg}
\DeclareMathOperator{\Rep}{Re}
\DeclareMathOperator{\Imp}{Im}
\DeclareMathOperator{\sech}{sech}
\DeclareMathOperator{\csch}{csch}
\DeclareMathOperator{\Log}{Log}

\newcommand{\tightoverset}[2]{% for arrow vec
  \mathop{#2}\limits^{\vbox to -.5ex{\kern-0.75ex\hbox{$#1$}\vss}}}
\newcommand{\arrowvec}{\overrightarrow}
\renewcommand{\vec}{\mathbf}
\newcommand{\veci}{{\boldsymbol{\hat{\imath}}}}
\newcommand{\vecj}{{\boldsymbol{\hat{\jmath}}}}
\newcommand{\veck}{{\boldsymbol{\hat{k}}}}
\newcommand{\vecl}{\boldsymbol{\l}}
\newcommand{\utan}{\vec{\hat{t}}}
\newcommand{\unormal}{\vec{\hat{n}}}
\newcommand{\ubinormal}{\vec{\hat{b}}}

\newcommand{\dotp}{\bullet}
\newcommand{\cross}{\boldsymbol\times}
\newcommand{\grad}{\boldsymbol\nabla}
\newcommand{\divergence}{\grad\dotp}
\newcommand{\curl}{\grad\cross}
%% Simple horiz vectors
\renewcommand{\vector}[1]{\left\langle #1\right\rangle}


\pgfplotsset{compat=1.13}

\outcome{Enumerate the number of derangements of a set}

\title{2.3 Derangements}

\begin{document}



\begin{abstract}
We use the Inclusion-Exclusion Principle to enumerate derangements.
\end{abstract}

\maketitle



\begin{definition}[Derangements]
A derangement is a permutation in which each of the permuted objects is forbidden from exactly one distinct position.
The number of derangements of a set of $n$ elements is denoted by $D_n$.
\end{definition}

\begin{remark}
In its simplest form, a derangement of the set $S = \{1, 2, ..., n\}$ is a permutation of $S$ in 
which the $1$ is not in the first position, the $2$ is not in the second position and so on.
\end{remark}

\begin{example}
How many derangements are there of the word $CAT$?\\
In this case, a derangement consists of a permutation of the letters in the word $CAT$ in 
which the $C$ is not in the first position, the $A$ is not in the second position 
and the $T$ is not in the third position. The list of such permutations is short 
enough (certainly less than $3! = 6$) that the count can be 
obtained by listing its members.  They are
\[
ATC \;\;\text{  and  } \;\;TCA
\]
Hence, there are $2$ derangements of the letters in the word $CAT$ and we conclude that $D_3 =2$.  
\end{example}

\begin{problem}(problem 1)
Show that $D_2 = 1$ and $D_4 = 9$ by explicitly listing all possible derangements of $AB$ and $ABCD$ respectively.
\end{problem}
%ABCD: BADC BCDA BDAC CADB CDAB CDBA DABC DCAB DCBA

We now use the Inclusion-Exclusion Principle to count the number of derangements of the set $S = \{1, 2, ..., n\}$
in which the number $j$ is not in the $j^{th}$ position for each $j$ from $1$ to $n$.

\begin{theorem}[Derangements]
The number of derangements of the set $S = \{1, 2, ..., n\}$
in which the number $j$ is not in the $j^{th}$ position for $j = 1, 2, ..., n$ is given by
\[
D_n = n!\sum_{k = 0}^n (-1)^k \frac{1}{k!}
\]
\end{theorem}



\begin{proof}
For each $j$ from $1$ to $n$, let $A_j$ be the set of permutations of $S$ that has $j$ in the $j^{th}$ position.
Then $|S| = n!$ and $|A_j| = (n-1)!$ for each $j$. Furthermore, $|A_i \cap A_j| = (n-2)!$ and $|A_i \cap A_j \cap A_k| = (n-3)!$.
We see that the number of elements in the intersection of $m$ of these sets will be $(n-m)!$.\\
We seek $\dis D_n = |A_1^c \cap A_2^c \cap \cdots \cap A_n^c|$
By De Morgan's Law (for $n$ sets), the Rule for Complements and the Inclusion-Exclusion Principle, we have
\begin{align*}
D_n &= |A_1^c \cap A_2^c \cap \cdots \cap A_n^c| = |\left(A_1 \cup A_2 \cup \cdots \cup A_n\right)^c|
   = |S| - |A_1 \cup A_2 \cup \cdots \cup A_n|\\
   &= n! - \bigg[\sum_{i =1}^n |A_i| - \sum_{1 =i<j}^n |A_i\cap A_j|+ 
   \sum_{1 =i<j< k}^n |A_i\cap A_j \cap A_k| \\
   &\hspace{2.5 in} - \cdots + (-1)^{n-1} |A_1 \cap A_2 \cap \cdots \cap A_n|\bigg]\\
   &= n! - \binom{n}{1}(n-1)! + \binom{n}{2}(n-2)! - \binom{n}{3}(n-3)! + \cdots + (-1)^n \binom{n}{n}(n-n)!\\
   &= n! - n! + \frac{n!}{2!} - \frac{n!}{3!} + \cdots + (-1)^n \frac{n!}{n!}\\
   &= n! \cdot \sum_{k=0}^n (-1)^n \frac{1}{k!}
\end{align*}


\end{proof}


\begin{remark}
Recall the Maclaurin series expansion for the exponential function:
\[
e^x = \sum_{k = 0}^\infty \frac{x^k}{k!}
\]
Letting $x = -1$ gives
\[
\frac{1}{e} = \sum_{k = 0}^\infty (-1)^k\frac{1}{k!}
\]
Thus, we see in $D_n$ the truncation of this series for $1/e$ at the $n^{th}$ term.
\end{remark}


\begin{example}[example 2]
Compute $D_8$.\\
\begin{align*}
D_8 &= 8! \sum_{k=0}^8 \frac{(-1)^k}{k!}\\
&= 8!\bigg(\frac{1}{0!} - \frac{1}{1!} + \frac{1}{2!} - \frac{1}{3!} + \frac{1}{4!}- \frac{1}{5!} + \frac{1}{6!} - \frac{1}{7!} + \frac{1}{8!}\bigg) \\
&=  \frac{8!}{2!} - \frac{8!}{3!} + \frac{8!}{4!} - \frac{8!}{5!} + \frac{8!}{6!} - \frac{8!}{7!} + \frac{8!}{8!} \\
&= 14833
\end{align*}
\end{example}

\begin{problem}(problem 2)
Use the formula for $D_n$ given in the theorem to compute $D_5$ and $D_6$.\\
\[
D_5 = \; \answer{265}
\]
\[
D_6 = \; \answer{1854}
\]
\end{problem}

 




It is also possible to use combinatorial reasoning to obtain a recursive formula for 
derangement numbers, $D_n$.

\begin{proposition}
Let $n \geq 2$. Then the following formula holds:
\[
D_n = (n-1)(D_{n-2} + D_{n-1})
\]
\end{proposition}

\begin{remark}
Since $D_1 = 0$ and $D_2 = 1$, We define $D_0 = 1$ so that the recursion formula 
holds for $n=2$, i.e., $D_2 = D_0 + D_1$.
\end{remark}

\begin{proof}
Let $S = \{1, 2, ..., n\}$ and consider the 
derangements of $123...n$ where the $j \neq 1$ is in the first position. 
These derangements can be partitioned into two sets- those permutations for which $1$ is in 
the $j^{th}$ position and those for which $1$ is not in the $j^{th}$ position.\\
Case 1) $j$ is in the first position and $1$ is in the $j^{th}$ position. 
The number of such derangements is simply $D_{n-2}$ since the remaining $n-2$ elements can be permuted among themselves 
in $D_{n-2}$ ways in order to obtain a derangement.\\
Case 2) $j$ is in the first position and $1$ is not in the $j^{th}$ position. 
The number of derangements in this case is the number of ways to permute $234...(j-1)1(j+1)...n$,
(the 1 is in the $j^{th}$ position) such that 2 is not first, 
3 is not second, ..., $1$ is not $j^{th}$, ..., and the $n$ is not last. 
This is the number of derangements of the $n-1$ symbols $234...(j-1)i(j+1)...n$ and hence it is $D_{n-1}$.\\
The result follows by noting that there are $n-1$ possibilities for the choice of $j$.
\end{proof}

\begin{example}[example 3]
Use the values $D_0 = 1$ and $D_1 = 0$ and the recursive formula for $D_n$ to compute $D_2, D_3, D_4$ and $D_5$.\\
We have
\begin{align*}
D_2 &= (2-1)(D_0 + D_1) = 1\\
D_3 &= (3-1)(D_1 + D_2) = 2(0+1) = 2\\
D_4 &= (4-1)(D_2 + D_3) = 3(1+2) = 9\\
D_5 &= (5-1)(D_4 + D_5) = 4(2+9) = 44\\
\end{align*}
\end{example}


\begin{problem}(problem 3)
Use the recursive formula to compute $D_6, D_7, D_8, D_9$ and $D_{10}$.
\end{problem}

\begin{example}[example 4] (The hat check problem)\\
$n$ people check their hats at an establishment. At closing time, the $n$ hats are returned at random to the $n$ people.
What is the probability that no one receives their own hat?\\
The total number of ways to return the $n$ hats to the $n$ people is $n!$ and since each of 
these $n!$ permutations is equally likely as the hats at returned at random, the probability of 
any particular permutation of the returned hats is $\frac{1}{n!}$. The probability that no one 
receives their own hat is thus $D_n*\frac{1}{n!} = \frac{D_n}{n!}$ since there are $D_n$ ways to return 
the $n$ hats in this manner.
\end{example}

\begin{problem}(problem 4)
Find the limit as $n \to \infty$ of the probability in the hat check problem. $\;\; \answer{1/e}$
\end{problem}

\begin{problem}(problem 5)
Seven people check their hats at an establishment. In how many ways can the hats be returned so that \\
a) no person receives their own hat? $\;\; \answer{1854}$\\
b) at least one person receives their own hat? $\;\; \answer{3186}$\\
c) at least two people receive their own hat? $\;\; \answer{1331}$\\
d) exactly six people receive their own hat? $\;\; \answer{0}$\\
\end{problem}

There is a second, slightly simpler recursion formula for the derangement numbers that can be derived from the recursion 
formula in the proposition above.

\begin{problem}(problem 6)
Verify that the recursion relation
\[
D_n = nD_{n-1} + (-1)^n
\]
holds for $n=1, 2$ and $3$. Then, use mathematical induction to prove that it holds for all $n \geq 1$.
\end{problem}

\begin{problem}(problem 7)
Use the recursion relation in problem 6 and your result for $D_{10}$ from problem 3 to compute $D_{11}, D_{12}$
and $D_{13}$.
\end{problem}


Permutations can be partitioned into the number of objects which are not moved from their original position.

\begin{problem}(problem 8)
Use a counting argument to establish the identity:
\[
n! = \binom{n}{0} D_n + \binom{n}{1} D_{n-1}  +\binom{n}{2} D_{n-2} + \cdots +\binom{n}{n-1} D_{1}+\binom{n}{n} D_0
\]
\end{problem}

\end{document}


\begin{proposition}
For $n \geq 1$ we have
\[
D_n = nD_{n-1} + (-1)^n
\]
\end{proposition}

\begin{proof}
For $n=1$, this formula says $D_1 = D_0 - 1 = 1-1 = 0$ which is true. For $n \geq 2$, we use the recursive formula from 
the previous proposition:
\[
D_n = (n-1)(D_{n-2} + D_{n-1})
\]
This equation can be re-written as
\[
D_n - nD_{n-1} = -[D_{n-1} - (n-1)D_{n-2}]
\]
Note that the expression in the brackets has the same form as the left hand side with $n$ replaced by $n-1$.
Hence the quantity inside the brackets can be re-written as 
\[
-[D_{n-2} - (n-2)D_{n-3}]
\]
which yields
\[
D_n - nD_{n-1} = (-1)^2[D_{n-2} - (n-2)D_{n-3}]
\]
We can continue reducing the right hand side until the subscripts are $1$ and $0$:
\begin{align*}
D_n - nD_{n-1} &= (-1)^3[D_{n-3} - (n-3)D_{n-4}]\\
               &= (-1)^4[D_{n-4} - (n-4)D_{n-5}]\\
               & = \cdots = (-1)^k [D_{n-k} - (n-k)D_{n-k-1}\\
               &= \cdots = (-1)^{n-1}[D_1 - (1)D_0]\\
               &= (-1)^{n-1}[0-1] = (-1)^n
\end{align*}
Hence, 
\[
D_n = nD_{n-1} +(-1)^n
\]
\end{proof}


\end{document}





