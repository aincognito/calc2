\documentclass[handout]{ximera}

%% You can put user macros here
%% However, you cannot make new environments



\newcommand{\ffrac}[2]{\frac{\text{\footnotesize $#1$}}{\text{\footnotesize $#2$}}}
\newcommand{\vasymptote}[2][]{
    \draw [densely dashed,#1] ({rel axis cs:0,0} -| {axis cs:#2,0}) -- ({rel axis cs:0,1} -| {axis cs:#2,0});
}


\graphicspath{{./}{firstExample/}}

\usepackage{amsmath}
\usepackage{amssymb}
\usepackage{array}
\usepackage[makeroom]{cancel} %% for strike outs
\usepackage{pgffor} %% required for integral for loops
\usepackage{tikz}
\usepackage{tikz-cd}
\usepackage{tkz-euclide}
\usetikzlibrary{shapes.multipart}


\usetkzobj{all}
\tikzstyle geometryDiagrams=[ultra thick,color=blue!50!black]


\usetikzlibrary{arrows}
\tikzset{>=stealth,commutative diagrams/.cd,
  arrow style=tikz,diagrams={>=stealth}} %% cool arrow head
\tikzset{shorten <>/.style={ shorten >=#1, shorten <=#1 } } %% allows shorter vectors

\usetikzlibrary{backgrounds} %% for boxes around graphs
\usetikzlibrary{shapes,positioning}  %% Clouds and stars
\usetikzlibrary{matrix} %% for matrix
\usepgfplotslibrary{polar} %% for polar plots
\usepgfplotslibrary{fillbetween} %% to shade area between curves in TikZ



%\usepackage[width=4.375in, height=7.0in, top=1.0in, papersize={5.5in,8.5in}]{geometry}
%\usepackage[pdftex]{graphicx}
%\usepackage{tipa}
%\usepackage{txfonts}
%\usepackage{textcomp}
%\usepackage{amsthm}
%\usepackage{xy}
%\usepackage{fancyhdr}
%\usepackage{xcolor}
%\usepackage{mathtools} %% for pretty underbrace % Breaks Ximera
%\usepackage{multicol}



\newcommand{\RR}{\mathbb R}
\newcommand{\R}{\mathbb R}
\newcommand{\C}{\mathbb C}
\newcommand{\N}{\mathbb N}
\newcommand{\Z}{\mathbb Z}
\newcommand{\dis}{\displaystyle}
%\renewcommand{\d}{\,d\!}
\renewcommand{\d}{\mathop{}\!d}
\newcommand{\dd}[2][]{\frac{\d #1}{\d #2}}
\newcommand{\pp}[2][]{\frac{\partial #1}{\partial #2}}
\renewcommand{\l}{\ell}
\newcommand{\ddx}{\frac{d}{\d x}}

\newcommand{\zeroOverZero}{\ensuremath{\boldsymbol{\tfrac{0}{0}}}}
\newcommand{\inftyOverInfty}{\ensuremath{\boldsymbol{\tfrac{\infty}{\infty}}}}
\newcommand{\zeroOverInfty}{\ensuremath{\boldsymbol{\tfrac{0}{\infty}}}}
\newcommand{\zeroTimesInfty}{\ensuremath{\small\boldsymbol{0\cdot \infty}}}
\newcommand{\inftyMinusInfty}{\ensuremath{\small\boldsymbol{\infty - \infty}}}
\newcommand{\oneToInfty}{\ensuremath{\boldsymbol{1^\infty}}}
\newcommand{\zeroToZero}{\ensuremath{\boldsymbol{0^0}}}
\newcommand{\inftyToZero}{\ensuremath{\boldsymbol{\infty^0}}}


\newcommand{\numOverZero}{\ensuremath{\boldsymbol{\tfrac{\#}{0}}}}
\newcommand{\dfn}{\textbf}
%\newcommand{\unit}{\,\mathrm}
\newcommand{\unit}{\mathop{}\!\mathrm}
%\newcommand{\eval}[1]{\bigg[ #1 \bigg]}
\newcommand{\eval}[1]{ #1 \bigg|}
\newcommand{\seq}[1]{\left( #1 \right)}
\renewcommand{\epsilon}{\varepsilon}
\renewcommand{\iff}{\Leftrightarrow}

\DeclareMathOperator{\arccot}{arccot}
\DeclareMathOperator{\arcsec}{arcsec}
\DeclareMathOperator{\arccsc}{arccsc}
\DeclareMathOperator{\si}{Si}
\DeclareMathOperator{\proj}{proj}
\DeclareMathOperator{\scal}{scal}
\DeclareMathOperator{\cis}{cis}
\DeclareMathOperator{\Arg}{Arg}
%\DeclareMathOperator{\arg}{arg}
\DeclareMathOperator{\Rep}{Re}
\DeclareMathOperator{\Imp}{Im}
\DeclareMathOperator{\sech}{sech}
\DeclareMathOperator{\csch}{csch}
\DeclareMathOperator{\Log}{Log}

\newcommand{\tightoverset}[2]{% for arrow vec
  \mathop{#2}\limits^{\vbox to -.5ex{\kern-0.75ex\hbox{$#1$}\vss}}}
\newcommand{\arrowvec}{\overrightarrow}
\renewcommand{\vec}{\mathbf}
\newcommand{\veci}{{\boldsymbol{\hat{\imath}}}}
\newcommand{\vecj}{{\boldsymbol{\hat{\jmath}}}}
\newcommand{\veck}{{\boldsymbol{\hat{k}}}}
\newcommand{\vecl}{\boldsymbol{\l}}
\newcommand{\utan}{\vec{\hat{t}}}
\newcommand{\unormal}{\vec{\hat{n}}}
\newcommand{\ubinormal}{\vec{\hat{b}}}

\newcommand{\dotp}{\bullet}
\newcommand{\cross}{\boldsymbol\times}
\newcommand{\grad}{\boldsymbol\nabla}
\newcommand{\divergence}{\grad\dotp}
\newcommand{\curl}{\grad\cross}
%% Simple horiz vectors
\renewcommand{\vector}[1]{\left\langle #1\right\rangle}


\pgfplotsset{compat=1.13}

\outcome{Explore Newton's Binomial Theorem}

\title{1.11 Newton's Binomial Theorem}

\begin{document}

\begin{abstract}
We explore Newton's Binomial Theorem.
\end{abstract}

\maketitle


In this section, we extend the definition of $\binom{n}{k}$ to allow $n$ to be any real number and $k$ to be negative.
First, we define $\binom{n}{k}$ to be zero if $k$ is negative. If $n$ is not a natural number, 
then we use $\alpha$ instead of $n$ and we write $\binom{\alpha}{k}$.  To define this, recall that 
\[
\binom{n}{k} = \frac{P(n,k)}{k!} = \frac{n(n-1)(n-2)\cdots(n-k+1)}{k!}
\]
The numerator of the last fraction contains $k$ factors, from $n$ counting down by one to $n-k+1$. 
The computation of this numerator in no way requires $n$ to be a natural number.
Hence, we define 
\[
\binom{\alpha}{k} = \frac{\alpha(\alpha - 1)(\alpha - 2)\cdots(\alpha - k+1)}{k!}
\]
where $\alpha$ is any real number and $k$ is any non-negative integer. If $k$ is a negative integer, we simply define 
$\binom{\alpha}{k}$ to be zero and if $k = 0$, then we define $\binom{\alpha}{k}$ to be one.

\begin{example}[example 1] 
Compute $\dis \binom{1/2}{3}$.\\
Using the definition of $\binom{\alpha}{k}$ with $\alpha = 1/2$ and $k = 3$, we have
\[
\binom{1/2}{3} = \frac{\frac12(\frac12 -1)(\frac12 -2)}{3!} = \frac{\left(\frac12\right) \left(-\frac12\right)\left(-\frac32\right)}{6} = \frac{1}{16}
\]
\end{example}

\begin{problem}(problem 1) Compute each of the following:\\
a) $\; \binom{3/4}{2} = \; \answer{-\frac{3}{32}}$\\
b) $\; \binom{5/2}{4} = \; \answer{-\frac{5}{128}}$\\
c) $\; \binom{-5}{3} = \; \answer{-35}$\\
d) $\; \binom{0}{8} = \; \answer{0}$\\
e) $\; \binom{5}{12} = \; \answer{0}$
\end{problem}

Newton's Binomial Theorem involves powers of a binomial which are not whole numbers, like $(1+x)^{1/2}$. Stating 
the theorem requires our new binomial coefficients, $\binom{\alpha}{k}$.

\begin{theorem}[Newton's Binomial Theorem]
If $|x| \leq 1$ and $\alpha \in \mathbb{R}$ then
\[
(1+x)^\alpha = \sum_{k = 0}^\infty \binom{\alpha}{k}x^k
\]
\end{theorem}

\begin{remark} If $k$ is a non-negative integer, Newton's Binomial Theorem agrees with the standard Binomial Theorem since 
\[
\binom{n}{k} = 0 \;\; \text{if} \;\; k > n
\]
and hence the infinite series becomes a finite sum in this case.
\end{remark}


\begin{example}[example 2]
Use Newton's Binomial Theorem to approximate $\sqrt 5$.\\
First, we write 
\[
\sqrt 5 = \sqrt{4+1} = \sqrt{4}\cdot \sqrt{1+\frac{1}{4}}
\]
Now we use Newton's Binomial Theorem with $\alpha = \frac12$ and $x = \frac14$.
We have
\[
\sqrt 5 = 2\left(1+\frac14\right)^{1/2} = 2\sum_{k=0}^\infty \binom{1/2}{k} \left(\frac14\right)^k
\]
To make an approximation, we truncate the sum at some index.  For the purpose of this example, we will truncate the series at $k = 3$. We have
\[
\sqrt 5 \approx 2\sum_{k=0}^3 \binom{1/2}{k} \left(\frac14\right)^k 
\]
\[
= 2\left[\binom{1/2}{0} \left(\frac14\right)^0 +\binom{1/2}{1} \left(\frac14\right)^1 + \binom{1/2}{2} \left(\frac14\right)^2 + \binom{1/2}{3} \left(\frac14\right)^3 \right]
\]
\[
= 2\left(1 + \frac18 - \frac{1}{128} + \frac{1}{1024}\right)
\]
\[
= \frac{1145}{512}
\]
Since the infinite series in our approximation begins to alternate after the second term, 
the error in our four term approximation is less than twice the absolute value of the next term (since our estimate involves multiplying the series by 2). 
This next term corresponds to $k= 4$ and its value is 
\[
\binom{1/2}{4}\left(\frac14\right)^4 = -\frac{\frac12\cdot \frac12\cdot \frac32 \cdot\frac52}{4!}\cdot \frac{1}{4^4}
 = -\frac{5}{32768}
\]
Hence, the error is at most 
\[
\dis 2 \cdot \frac{5}{32768} = \frac{5}{16384}
\]
Moreover, since the term corresponding to $k = 4$ is negative, our approximation of $\sqrt 5$ 
as $\dis \frac{1145}{512}$ is an overestimate.
\end{example}

\begin{problem}(problem 2) Use Newton's Binomial Theorem to approximate the given value. 
a) Approximate $\sqrt[3]{12}$ by writing it as $\dis 2 \sqrt[3]{1+\frac12}$. 
Truncate your infinite series at $k = 2$.\\
Approximation: $\; \answer{\frac{41}{18}}\;$  \\

b) Approximate: $\sqrt{10}$ by writing it as $\dis 3 \sqrt{1+\frac19}$.  
Truncate your infinite series at $k = 3$. \\ 
Approximation: $\; \answer{\frac{12255}{3888}}\;$ \\

c) Approximate $\sqrt{2}$ by writing it as $\dis  \sqrt{1+1}$.  Truncate your infinite series at $k = 4$. \\ 
Approximation: $\; \answer{\frac{179}{128}}$ \\


d) Approximate $\sqrt[4]{5}$ by writing it as $\dis 2 \sqrt[4]{1-\frac{11}{16}}$.  Truncate your infinite series at $k = 2$. \\ 
Approximation: $\; \answer{\frac{7321}{4096}}\;$ \\

\end{problem}

\begin{proposition}[Pascal's Identity]
For any number $\alpha$ and any integer $k$, Pascal's Identity holds, i.e.,
\[
\binom{\alpha +1}{k+1} = \binom{\alpha}{k+1}+\binom{\alpha}{k}
\]
\end{proposition}

\begin{proof}
We prove the case that $k > 0$ and leave for the reader to prove the cases $k=0, k = -1$ and $k \leq -2$.\\
We compute the right hand side:
\[
\binom{\alpha}{k+1}+\binom{\alpha}{k} = \frac{\alpha (\alpha - 1) \cdots (\alpha -k)}{(k+1)!} + 
\frac{\alpha (\alpha - 1) \cdots (\alpha -k+1)}{k!}
\]
\[
= \frac{\alpha (\alpha - 1) \cdots (\alpha -k)}{(k+1)!} + \frac{\alpha (\alpha - 1) \cdots (\alpha -k+1)}{k!} \cdot \frac{k+1}{k+1}
\]
\[
= \frac{\alpha (\alpha - 1) \cdots (\alpha -k + 1)[(\alpha -k) + (k+1)]}{(k+1)!}
\]
\[
= \frac{\alpha (\alpha - 1) \cdots (\alpha -k + 1)(\alpha +1)}{(k+1)!}
\]
\[
= \frac{(\alpha +1)(\alpha) (\alpha - 1) \cdots (\alpha -k + 1)}{(k+1)!}
\]
\[
=\binom{\alpha+1}{k+1}
\]
since $(\alpha + 1) - (k+1) +1 = \alpha -k +1$ which agrees with the last factor in the numerator.
\end{proof}

\end{document}





