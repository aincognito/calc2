\documentclass[handout]{ximera}

%% You can put user macros here
%% However, you cannot make new environments



\newcommand{\ffrac}[2]{\frac{\text{\footnotesize $#1$}}{\text{\footnotesize $#2$}}}
\newcommand{\vasymptote}[2][]{
    \draw [densely dashed,#1] ({rel axis cs:0,0} -| {axis cs:#2,0}) -- ({rel axis cs:0,1} -| {axis cs:#2,0});
}


\graphicspath{{./}{firstExample/}}

\usepackage{amsmath}
\usepackage{amssymb}
\usepackage{array}
\usepackage[makeroom]{cancel} %% for strike outs
\usepackage{pgffor} %% required for integral for loops
\usepackage{tikz}
\usepackage{tikz-cd}
\usepackage{tkz-euclide}
\usetikzlibrary{shapes.multipart}


\usetkzobj{all}
\tikzstyle geometryDiagrams=[ultra thick,color=blue!50!black]


\usetikzlibrary{arrows}
\tikzset{>=stealth,commutative diagrams/.cd,
  arrow style=tikz,diagrams={>=stealth}} %% cool arrow head
\tikzset{shorten <>/.style={ shorten >=#1, shorten <=#1 } } %% allows shorter vectors

\usetikzlibrary{backgrounds} %% for boxes around graphs
\usetikzlibrary{shapes,positioning}  %% Clouds and stars
\usetikzlibrary{matrix} %% for matrix
\usepgfplotslibrary{polar} %% for polar plots
\usepgfplotslibrary{fillbetween} %% to shade area between curves in TikZ



%\usepackage[width=4.375in, height=7.0in, top=1.0in, papersize={5.5in,8.5in}]{geometry}
%\usepackage[pdftex]{graphicx}
%\usepackage{tipa}
%\usepackage{txfonts}
%\usepackage{textcomp}
%\usepackage{amsthm}
%\usepackage{xy}
%\usepackage{fancyhdr}
%\usepackage{xcolor}
%\usepackage{mathtools} %% for pretty underbrace % Breaks Ximera
%\usepackage{multicol}



\newcommand{\RR}{\mathbb R}
\newcommand{\R}{\mathbb R}
\newcommand{\C}{\mathbb C}
\newcommand{\N}{\mathbb N}
\newcommand{\Z}{\mathbb Z}
\newcommand{\dis}{\displaystyle}
%\renewcommand{\d}{\,d\!}
\renewcommand{\d}{\mathop{}\!d}
\newcommand{\dd}[2][]{\frac{\d #1}{\d #2}}
\newcommand{\pp}[2][]{\frac{\partial #1}{\partial #2}}
\renewcommand{\l}{\ell}
\newcommand{\ddx}{\frac{d}{\d x}}

\newcommand{\zeroOverZero}{\ensuremath{\boldsymbol{\tfrac{0}{0}}}}
\newcommand{\inftyOverInfty}{\ensuremath{\boldsymbol{\tfrac{\infty}{\infty}}}}
\newcommand{\zeroOverInfty}{\ensuremath{\boldsymbol{\tfrac{0}{\infty}}}}
\newcommand{\zeroTimesInfty}{\ensuremath{\small\boldsymbol{0\cdot \infty}}}
\newcommand{\inftyMinusInfty}{\ensuremath{\small\boldsymbol{\infty - \infty}}}
\newcommand{\oneToInfty}{\ensuremath{\boldsymbol{1^\infty}}}
\newcommand{\zeroToZero}{\ensuremath{\boldsymbol{0^0}}}
\newcommand{\inftyToZero}{\ensuremath{\boldsymbol{\infty^0}}}


\newcommand{\numOverZero}{\ensuremath{\boldsymbol{\tfrac{\#}{0}}}}
\newcommand{\dfn}{\textbf}
%\newcommand{\unit}{\,\mathrm}
\newcommand{\unit}{\mathop{}\!\mathrm}
%\newcommand{\eval}[1]{\bigg[ #1 \bigg]}
\newcommand{\eval}[1]{ #1 \bigg|}
\newcommand{\seq}[1]{\left( #1 \right)}
\renewcommand{\epsilon}{\varepsilon}
\renewcommand{\iff}{\Leftrightarrow}

\DeclareMathOperator{\arccot}{arccot}
\DeclareMathOperator{\arcsec}{arcsec}
\DeclareMathOperator{\arccsc}{arccsc}
\DeclareMathOperator{\si}{Si}
\DeclareMathOperator{\proj}{proj}
\DeclareMathOperator{\scal}{scal}
\DeclareMathOperator{\cis}{cis}
\DeclareMathOperator{\Arg}{Arg}
%\DeclareMathOperator{\arg}{arg}
\DeclareMathOperator{\Rep}{Re}
\DeclareMathOperator{\Imp}{Im}
\DeclareMathOperator{\sech}{sech}
\DeclareMathOperator{\csch}{csch}
\DeclareMathOperator{\Log}{Log}

\newcommand{\tightoverset}[2]{% for arrow vec
  \mathop{#2}\limits^{\vbox to -.5ex{\kern-0.75ex\hbox{$#1$}\vss}}}
\newcommand{\arrowvec}{\overrightarrow}
\renewcommand{\vec}{\mathbf}
\newcommand{\veci}{{\boldsymbol{\hat{\imath}}}}
\newcommand{\vecj}{{\boldsymbol{\hat{\jmath}}}}
\newcommand{\veck}{{\boldsymbol{\hat{k}}}}
\newcommand{\vecl}{\boldsymbol{\l}}
\newcommand{\utan}{\vec{\hat{t}}}
\newcommand{\unormal}{\vec{\hat{n}}}
\newcommand{\ubinormal}{\vec{\hat{b}}}

\newcommand{\dotp}{\bullet}
\newcommand{\cross}{\boldsymbol\times}
\newcommand{\grad}{\boldsymbol\nabla}
\newcommand{\divergence}{\grad\dotp}
\newcommand{\curl}{\grad\cross}
%% Simple horiz vectors
\renewcommand{\vector}[1]{\left\langle #1\right\rangle}


\pgfplotsset{compat=1.13}

\outcome{Use the Fundamental Principle of Counting}

\title{1.1 Fundamental Principle of Counting}

\begin{document}

\begin{abstract}
We use the Fundamental Principle of Counting.
\end{abstract}

\maketitle

\section{Fundamental Principle of Counting}

\begin{example}[example 1]
Pat has three distinct shirts and two distinct pairs of pants. Getting dressed requires one of each.  
In how many ways can Pat get dressed?\\
We make a tree diagram to determine the possible outfits Pat can create.
\begin{image}
\begin{tikzpicture}
\draw[mark=*,mark size=3pt,mark options={color=blue}] plot coordinates {(0,0)} node[left, blue]{Pat\; };
\draw (0,0) -- (3,4) ;
\draw[mark=*,mark size=3pt,mark options={color=blue}] plot coordinates {(3,4)} node[left, blue]{Shirt 1\, };
\draw (0,0) -- (3,0) ;
\draw[mark=*,mark size=3pt,mark options={color=blue}] plot coordinates {(3,0)} node[above left, blue]{Shirt 2 \,};
\draw (0,0) -- (3,-4) ;
\draw[mark=*,mark size=3pt,mark options={color=blue}] plot coordinates {(3,-4)} node[left, blue]{Shirt 3 \,};
\draw (3,4) -- (6,5) ;
\draw[mark=*,mark size=3pt,mark options={color=blue}] plot coordinates {(6,5)} node[right, blue]{ \,Pants 1};
\draw (3,4) -- (6,3) ;
\draw[mark=*,mark size=3pt,mark options={color=blue}] plot coordinates {(6,3)} node[right, blue]{\, Pants 2};
\draw (3,0) -- (6,1) ;
\draw[mark=*,mark size=3pt,mark options={color=blue}] plot coordinates {(6,1)} node[right, blue]{ \,Pants 1};
\draw (3,0) -- (6,-1) ;
\draw[mark=*,mark size=3pt,mark options={color=blue}] plot coordinates {(6,-1)} node[right, blue]{\, Pants 2};
\draw (3,-4) -- (6,-3) ;
\draw[mark=*,mark size=3pt,mark options={color=blue}] plot coordinates {(6,-3)} node[right, blue]{\, Pants 1};
\draw (3,-4) -- (6,-5) ;
\draw[mark=*,mark size=3pt,mark options={color=blue}] plot coordinates {(6,-5)} node[right, blue]{\,Pants 2};
\end{tikzpicture}
\end{image}
Each path from Pat to a node on the far right of the tree diagram represents a different way for Pat to dress. From top to bottom, 
these represent the following outfits: Shirt 1 with Pants 1, Shirt 1 with Pants 2, Shirt 2 with Pants 1, 
Shirt 2 with Pants 2, Shirt 3 with Pants 1 and Shirt 3 with Pants 2. 
Since there are 6 total paths, Pat can dress in 6 different ways.

\end{example}

\begin{problem}(problem 1a)
Sam is getting dressed and wants to include a belt and a hat.  
If Sam has three distinct belts and two distinct hats, how many ways
can Sam accessorize? Include a tree diagram that illustrates the possibilities.\\
Sam can accessorize in $\; \answer{6} \;$ different ways.
\end{problem}

\begin{problem}(problem 1b)
A couple wants to go on a date.  They decide to  eat at a restaurant, walk in a park and then see a show.
If there are two restaurants that the y both like, two nearby parks and two shows playing, 
how many possibilities are there for their date? Make a tree diagram illustrating the possibilities.\\
There are $\;\answer{8}\;$ different possibilities for their date.
\end{problem}


\begin{theorem}[Fundamental Principle of Counting]
Suppose that $n$ decisions are to be made and that the number of choices for the $k^{th}$ decision is $d_k$.
Then the total number of ways to make the $n$ decisions is
\[
\prod_{k=1}^n d_k = d_1 \cdot d_2 \cdot \ldots \cdot d_n
\]
\end{theorem}

\begin{example}
There are three types of Tesla Model 3: Standard Range, Long Range and Performance. 
Each type can be painted one of 5 colors. Furthermore, two interior color schemes are available. 
Given these parameters, how many choices are there for my Tesla Model 3 purchase?\\
In total, three decisions must be made: the type, the paint color and the interior color scheme.
The number of choices for each decision is 3, 5 and 2 respectively. 
Hence, by the Fundamental Principle of Counting, the total number of ways to configure my new 
Tesla Model 3 is $3 \cdot 5\cdot 2 = 30$.
\begin{image}
\begin{tikzpicture}

\draw (0,0) -- (2,0) node[blue, below, midway]{Type} node[above, midway]{3} node[right]{$\times$};
\draw (2.5,0) -- (4.5,0) node[blue, below, midway]{Paint} node[above, midway]{5} node[right]{$\times$};
\draw (5,0) -- (7,0) node[blue, below, midway]{Interior} node[above, midway]{2} node[right]{$=$};
\draw (7.5,0) -- (9.5,0) node[blue, below, midway]{Total} node[above, midway]{30} ;

\end{tikzpicture}
\end{image}
\end{example}


\begin{problem}(problem 2a)
A three course meal consists of an appetizer, an entree and a dessert. 
If a restaurant offers a dozen appetizers, a score of entrees and a bakers dozen desserts, 
how many possible three course meals can be ordered?\\
The total number of possible three course meals is $\answer{3120}$.
\end{problem}

\begin{problem}(problem 2b)
A sixth grade class of 20 students would like to select a President, Vice-President, Secretary and Treasurer.
If no student may serve in more than one post, in how many possible ways can the class officers be selected?\\
The total number of ways to select the class officers is $\answer{116280}$.
\end{problem}


\begin{example}[example 3]
Suppose License plates consist of three letters followed by three digits. How many license plates are possible?\\
Altogether, there are six decisions to be made. By the Fundamental Principle of Counting, the answer is the
product of the number of choices for each decision.

\begin{image}
\begin{tikzpicture}[scale = .8]

\draw (0,0) -- (2,0) node[blue, below, midway]{$1^{st}$ letter} node[above, midway]{26} node[right]{$\times$};
\draw (2.5,0) -- (4.5,0) node[blue, below, midway]{$2^{nd}$ letter} node[above, midway]{26} node[right]{$\times$};
\draw (5,0) -- (7,0) node[blue, below, midway]{$3^{rd}$ letter} node[above, midway]{26} node[right]{$\times$};
\draw (7.5,0) -- (9.5,0) node[blue, below, midway]{$1^{st}$ digit} node[above, midway]{10} node[right]{$\times$};
\draw (10,0) -- (12,0) node[blue, below, midway]{$2^{nd}$ digit} node[above, midway]{10} node[right]{$\times$};
\draw (12.5,0) -- (14.5,0) node[blue, below, midway]{$3^{rd}$ digit} node[above, midway]{10} node[right]{$=$};
\draw (15,0) -- (17,0) node[blue, below, midway]{Total} node[above, midway]{$26^3 10^3$} ;
\end{tikzpicture}
\end{image}
Thus the total number of such license plates is $26 \cdot 26 \cdot 26 \cdot 10\cdot 10\cdot 10 = 17,576,000$.

\end{example}

\begin{problem}(problem 3)
Suppose telephone numbers consist of 7 digits, the first of which cannot be 0 or 1.  
How many such telephone numbers are possible?\\
$\answer{8000000}$

\end{problem}

\end{document}



\begin{example}[example ]

\end{example}


\begin{problem}(problem )

\end{problem}


















