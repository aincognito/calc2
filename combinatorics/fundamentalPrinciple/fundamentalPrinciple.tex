\documentclass[handout]{ximera}

%% You can put user macros here
%% However, you cannot make new environments



\newcommand{\ffrac}[2]{\frac{\text{\footnotesize $#1$}}{\text{\footnotesize $#2$}}}
\newcommand{\vasymptote}[2][]{
    \draw [densely dashed,#1] ({rel axis cs:0,0} -| {axis cs:#2,0}) -- ({rel axis cs:0,1} -| {axis cs:#2,0});
}


\graphicspath{{./}{firstExample/}}

\usepackage{amsmath}
\usepackage{amssymb}
\usepackage{array}
\usepackage[makeroom]{cancel} %% for strike outs
\usepackage{pgffor} %% required for integral for loops
\usepackage{tikz}
\usepackage{tikz-cd}
\usepackage{tkz-euclide}
\usetikzlibrary{shapes.multipart}


\usetkzobj{all}
\tikzstyle geometryDiagrams=[ultra thick,color=blue!50!black]


\usetikzlibrary{arrows}
\tikzset{>=stealth,commutative diagrams/.cd,
  arrow style=tikz,diagrams={>=stealth}} %% cool arrow head
\tikzset{shorten <>/.style={ shorten >=#1, shorten <=#1 } } %% allows shorter vectors

\usetikzlibrary{backgrounds} %% for boxes around graphs
\usetikzlibrary{shapes,positioning}  %% Clouds and stars
\usetikzlibrary{matrix} %% for matrix
\usepgfplotslibrary{polar} %% for polar plots
\usepgfplotslibrary{fillbetween} %% to shade area between curves in TikZ



%\usepackage[width=4.375in, height=7.0in, top=1.0in, papersize={5.5in,8.5in}]{geometry}
%\usepackage[pdftex]{graphicx}
%\usepackage{tipa}
%\usepackage{txfonts}
%\usepackage{textcomp}
%\usepackage{amsthm}
%\usepackage{xy}
%\usepackage{fancyhdr}
%\usepackage{xcolor}
%\usepackage{mathtools} %% for pretty underbrace % Breaks Ximera
%\usepackage{multicol}



\newcommand{\RR}{\mathbb R}
\newcommand{\R}{\mathbb R}
\newcommand{\C}{\mathbb C}
\newcommand{\N}{\mathbb N}
\newcommand{\Z}{\mathbb Z}
\newcommand{\dis}{\displaystyle}
%\renewcommand{\d}{\,d\!}
\renewcommand{\d}{\mathop{}\!d}
\newcommand{\dd}[2][]{\frac{\d #1}{\d #2}}
\newcommand{\pp}[2][]{\frac{\partial #1}{\partial #2}}
\renewcommand{\l}{\ell}
\newcommand{\ddx}{\frac{d}{\d x}}

\newcommand{\zeroOverZero}{\ensuremath{\boldsymbol{\tfrac{0}{0}}}}
\newcommand{\inftyOverInfty}{\ensuremath{\boldsymbol{\tfrac{\infty}{\infty}}}}
\newcommand{\zeroOverInfty}{\ensuremath{\boldsymbol{\tfrac{0}{\infty}}}}
\newcommand{\zeroTimesInfty}{\ensuremath{\small\boldsymbol{0\cdot \infty}}}
\newcommand{\inftyMinusInfty}{\ensuremath{\small\boldsymbol{\infty - \infty}}}
\newcommand{\oneToInfty}{\ensuremath{\boldsymbol{1^\infty}}}
\newcommand{\zeroToZero}{\ensuremath{\boldsymbol{0^0}}}
\newcommand{\inftyToZero}{\ensuremath{\boldsymbol{\infty^0}}}


\newcommand{\numOverZero}{\ensuremath{\boldsymbol{\tfrac{\#}{0}}}}
\newcommand{\dfn}{\textbf}
%\newcommand{\unit}{\,\mathrm}
\newcommand{\unit}{\mathop{}\!\mathrm}
%\newcommand{\eval}[1]{\bigg[ #1 \bigg]}
\newcommand{\eval}[1]{ #1 \bigg|}
\newcommand{\seq}[1]{\left( #1 \right)}
\renewcommand{\epsilon}{\varepsilon}
\renewcommand{\iff}{\Leftrightarrow}

\DeclareMathOperator{\arccot}{arccot}
\DeclareMathOperator{\arcsec}{arcsec}
\DeclareMathOperator{\arccsc}{arccsc}
\DeclareMathOperator{\si}{Si}
\DeclareMathOperator{\proj}{proj}
\DeclareMathOperator{\scal}{scal}
\DeclareMathOperator{\cis}{cis}
\DeclareMathOperator{\Arg}{Arg}
%\DeclareMathOperator{\arg}{arg}
\DeclareMathOperator{\Rep}{Re}
\DeclareMathOperator{\Imp}{Im}
\DeclareMathOperator{\sech}{sech}
\DeclareMathOperator{\csch}{csch}
\DeclareMathOperator{\Log}{Log}

\newcommand{\tightoverset}[2]{% for arrow vec
  \mathop{#2}\limits^{\vbox to -.5ex{\kern-0.75ex\hbox{$#1$}\vss}}}
\newcommand{\arrowvec}{\overrightarrow}
\renewcommand{\vec}{\mathbf}
\newcommand{\veci}{{\boldsymbol{\hat{\imath}}}}
\newcommand{\vecj}{{\boldsymbol{\hat{\jmath}}}}
\newcommand{\veck}{{\boldsymbol{\hat{k}}}}
\newcommand{\vecl}{\boldsymbol{\l}}
\newcommand{\utan}{\vec{\hat{t}}}
\newcommand{\unormal}{\vec{\hat{n}}}
\newcommand{\ubinormal}{\vec{\hat{b}}}

\newcommand{\dotp}{\bullet}
\newcommand{\cross}{\boldsymbol\times}
\newcommand{\grad}{\boldsymbol\nabla}
\newcommand{\divergence}{\grad\dotp}
\newcommand{\curl}{\grad\cross}
%% Simple horiz vectors
\renewcommand{\vector}[1]{\left\langle #1\right\rangle}


\pgfplotsset{compat=1.13}

\outcome{Use the Fundamental Principle of Counting}

\title{1.1 Fundamental Principle of Counting}

\begin{document}

\begin{abstract}
We use the Fundamental Principle of Counting.
\end{abstract}

\maketitle

\section{Fundamental Principle of Counting}

\begin{example}[example 1]
Pat has three distinct shirts and two distinct pairs of pants. Getting dressed requires one of each.  
In how many ways can Pat get dressed?\\
We make a tree diagram to determine the possible outfits Pat can create.
\begin{image}
\begin{tikzpicture}
\draw[mark=*,mark size=3pt,mark options={color=blue}] plot coordinates {(0,0)} node[left, blue]{Pat\; };
\draw (0,0) -- (3,4) ;
\draw[mark=*,mark size=3pt,mark options={color=blue}] plot coordinates {(3,4)} node[left, blue]{Shirt 1\, };
\draw (0,0) -- (3,0) ;
\draw[mark=*,mark size=3pt,mark options={color=blue}] plot coordinates {(3,0)} node[above left, blue]{Shirt 2 \,};
\draw (0,0) -- (3,-4) ;
\draw[mark=*,mark size=3pt,mark options={color=blue}] plot coordinates {(3,-4)} node[left, blue]{Shirt 3 \,};
\draw (3,4) -- (6,5) ;
\draw[mark=*,mark size=3pt,mark options={color=blue}] plot coordinates {(6,5)} node[right, blue]{ \,Pants 1};
\draw (3,4) -- (6,3) ;
\draw[mark=*,mark size=3pt,mark options={color=blue}] plot coordinates {(6,3)} node[right, blue]{\, Pants 2};
\draw (3,0) -- (6,1) ;
\draw[mark=*,mark size=3pt,mark options={color=blue}] plot coordinates {(6,1)} node[right, blue]{ \,Pants 1};
\draw (3,0) -- (6,-1) ;
\draw[mark=*,mark size=3pt,mark options={color=blue}] plot coordinates {(6,-1)} node[right, blue]{\, Pants 2};
\draw (3,-4) -- (6,-3) ;
\draw[mark=*,mark size=3pt,mark options={color=blue}] plot coordinates {(6,-3)} node[right, blue]{\, Pants 1};
\draw (3,-4) -- (6,-5) ;
\draw[mark=*,mark size=3pt,mark options={color=blue}] plot coordinates {(6,-5)} node[right, blue]{\,Pants 2};
\end{tikzpicture}
\end{image}
Each path from Pat to a node on the far right of the tree diagram represents a different way for Pat to dress. From top to bottom, 
these represent the following outfits: Shirt 1 with Pants 1, Shirt 1 with Pants 2, Shirt 2 with Pants 1, 
Shirt 2 with Pants 2, Shirt 3 with Pants 1 and Shirt 3 with Pants 2. 
Since there are 6 total paths, Pat can dress in 6 different ways.

\end{example}

\begin{problem}(problem 1a)
Sam is getting dressed and wants to include a belt and a hat.  
If Sam has three distinct belts and two distinct hats, how many ways
can Sam accessorize? Include a tree diagram that illustrates the possibilities.\\
Sam can accessorize is \; $\answer{6}$\; different ways.
\end{problem}

\begin{problem}(problem 1b)
A couple wants to go on a date.  They decide to  eat at a restaurant, walk in a park and then see a show.
If there are two restaurants that the y both like, two nearby parks and two shows playing, 
how many possibilities are there for their date? Make a tree diagram illustrating the possibilities.\\
There are $\;\answer{8}\;$ different possibilities for their date.
\end{problem}


\end{document}

























