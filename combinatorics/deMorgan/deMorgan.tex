\documentclass[handout]{ximera}

%% You can put user macros here
%% However, you cannot make new environments



\newcommand{\ffrac}[2]{\frac{\text{\footnotesize $#1$}}{\text{\footnotesize $#2$}}}
\newcommand{\vasymptote}[2][]{
    \draw [densely dashed,#1] ({rel axis cs:0,0} -| {axis cs:#2,0}) -- ({rel axis cs:0,1} -| {axis cs:#2,0});
}


\graphicspath{{./}{firstExample/}}

\usepackage{amsmath}
\usepackage{amssymb}
\usepackage{array}
\usepackage[makeroom]{cancel} %% for strike outs
\usepackage{pgffor} %% required for integral for loops
\usepackage{tikz}
\usepackage{tikz-cd}
\usepackage{tkz-euclide}
\usetikzlibrary{shapes.multipart}


\usetkzobj{all}
\tikzstyle geometryDiagrams=[ultra thick,color=blue!50!black]


\usetikzlibrary{arrows}
\tikzset{>=stealth,commutative diagrams/.cd,
  arrow style=tikz,diagrams={>=stealth}} %% cool arrow head
\tikzset{shorten <>/.style={ shorten >=#1, shorten <=#1 } } %% allows shorter vectors

\usetikzlibrary{backgrounds} %% for boxes around graphs
\usetikzlibrary{shapes,positioning}  %% Clouds and stars
\usetikzlibrary{matrix} %% for matrix
\usepgfplotslibrary{polar} %% for polar plots
\usepgfplotslibrary{fillbetween} %% to shade area between curves in TikZ



%\usepackage[width=4.375in, height=7.0in, top=1.0in, papersize={5.5in,8.5in}]{geometry}
%\usepackage[pdftex]{graphicx}
%\usepackage{tipa}
%\usepackage{txfonts}
%\usepackage{textcomp}
%\usepackage{amsthm}
%\usepackage{xy}
%\usepackage{fancyhdr}
%\usepackage{xcolor}
%\usepackage{mathtools} %% for pretty underbrace % Breaks Ximera
%\usepackage{multicol}



\newcommand{\RR}{\mathbb R}
\newcommand{\R}{\mathbb R}
\newcommand{\C}{\mathbb C}
\newcommand{\N}{\mathbb N}
\newcommand{\Z}{\mathbb Z}
\newcommand{\dis}{\displaystyle}
%\renewcommand{\d}{\,d\!}
\renewcommand{\d}{\mathop{}\!d}
\newcommand{\dd}[2][]{\frac{\d #1}{\d #2}}
\newcommand{\pp}[2][]{\frac{\partial #1}{\partial #2}}
\renewcommand{\l}{\ell}
\newcommand{\ddx}{\frac{d}{\d x}}

\newcommand{\zeroOverZero}{\ensuremath{\boldsymbol{\tfrac{0}{0}}}}
\newcommand{\inftyOverInfty}{\ensuremath{\boldsymbol{\tfrac{\infty}{\infty}}}}
\newcommand{\zeroOverInfty}{\ensuremath{\boldsymbol{\tfrac{0}{\infty}}}}
\newcommand{\zeroTimesInfty}{\ensuremath{\small\boldsymbol{0\cdot \infty}}}
\newcommand{\inftyMinusInfty}{\ensuremath{\small\boldsymbol{\infty - \infty}}}
\newcommand{\oneToInfty}{\ensuremath{\boldsymbol{1^\infty}}}
\newcommand{\zeroToZero}{\ensuremath{\boldsymbol{0^0}}}
\newcommand{\inftyToZero}{\ensuremath{\boldsymbol{\infty^0}}}


\newcommand{\numOverZero}{\ensuremath{\boldsymbol{\tfrac{\#}{0}}}}
\newcommand{\dfn}{\textbf}
%\newcommand{\unit}{\,\mathrm}
\newcommand{\unit}{\mathop{}\!\mathrm}
%\newcommand{\eval}[1]{\bigg[ #1 \bigg]}
\newcommand{\eval}[1]{ #1 \bigg|}
\newcommand{\seq}[1]{\left( #1 \right)}
\renewcommand{\epsilon}{\varepsilon}
\renewcommand{\iff}{\Leftrightarrow}

\DeclareMathOperator{\arccot}{arccot}
\DeclareMathOperator{\arcsec}{arcsec}
\DeclareMathOperator{\arccsc}{arccsc}
\DeclareMathOperator{\si}{Si}
\DeclareMathOperator{\proj}{proj}
\DeclareMathOperator{\scal}{scal}
\DeclareMathOperator{\cis}{cis}
\DeclareMathOperator{\Arg}{Arg}
%\DeclareMathOperator{\arg}{arg}
\DeclareMathOperator{\Rep}{Re}
\DeclareMathOperator{\Imp}{Im}
\DeclareMathOperator{\sech}{sech}
\DeclareMathOperator{\csch}{csch}
\DeclareMathOperator{\Log}{Log}

\newcommand{\tightoverset}[2]{% for arrow vec
  \mathop{#2}\limits^{\vbox to -.5ex{\kern-0.75ex\hbox{$#1$}\vss}}}
\newcommand{\arrowvec}{\overrightarrow}
\renewcommand{\vec}{\mathbf}
\newcommand{\veci}{{\boldsymbol{\hat{\imath}}}}
\newcommand{\vecj}{{\boldsymbol{\hat{\jmath}}}}
\newcommand{\veck}{{\boldsymbol{\hat{k}}}}
\newcommand{\vecl}{\boldsymbol{\l}}
\newcommand{\utan}{\vec{\hat{t}}}
\newcommand{\unormal}{\vec{\hat{n}}}
\newcommand{\ubinormal}{\vec{\hat{b}}}

\newcommand{\dotp}{\bullet}
\newcommand{\cross}{\boldsymbol\times}
\newcommand{\grad}{\boldsymbol\nabla}
\newcommand{\divergence}{\grad\dotp}
\newcommand{\curl}{\grad\cross}
%% Simple horiz vectors
\renewcommand{\vector}[1]{\left\langle #1\right\rangle}


\pgfplotsset{compat=1.13}

\outcome{Use De Morgan's Law to count the number of elements in a set}

\title{2.1 De Morgan's Law}

\begin{document}

\begin{abstract}
We use De Morgan's to enumerate sets.
\end{abstract}

\maketitle

\section{De Morgan's Law}

Recall set complements from section 1.2:

\begin{definition}[Complement]
Given a universal set $U$ and a subset, $a \subset U$, we define the complement of $A$, denoted, $A^c$ by
\[
A^c = \{x \in U | x \notin A \}
\]
\end{definition}


\begin{image}
\begin{tikzpicture}
\draw (0,0) -- (0,2) -- (3.2,2) -- (3.2, 0) -- (0,0);
\draw (1.6, 1) circle (0.7);
\node at (1.5,0.9) {$A$};
\node at (2.7,1.5) {$A^c$};
\node at (3.5,0.5) {$U$};
\node at (1.6,-0.5) {$A$ and $A^c$};
\end{tikzpicture}
\end{image}

\begin{definition}[Union]
Given a sets $A$ and $B$, their union is defined as
\[
A \cup B = \{x \,|\, x \in A \text{  or  } x \in B \}
\]
\end{definition}

\begin{image}
\begin{tikzpicture}
\draw (0,0) -- (0,2) -- (3.2,2) -- (3.2, 0) -- (0,0);
\draw[fill =blue!10] (1.2, 1) circle (0.7);
\draw[fill = blue!10] (2, 1) circle (0.7);
\draw (1.2, 1) circle (0.7);
\node at (0.3,0.9) {$A$};
\node at (2.9,0.9) {$B$};
\node at (1.6,-0.5) {$A \cup B$};
\end{tikzpicture}
\end{image}


\begin{definition}[Intersection]
Given a sets $A$ and $B$, their intersection is defined as
\[
A \cap B = \{x \,|\, x \in A \text{  and  } x \in B \}
\]
\end{definition}

\begin{image}
\begin{tikzpicture}
\draw (0,0) -- (0,2) -- (3.2,2) -- (3.2, 0) -- (0,0);
\draw (1.2, 1) circle (0.7);
\begin{scope}
      \clip
        (1.2, 1) circle (0.7);
      \draw[fill=blue!10] 
        (2, 1) circle (0.7);
    \end{scope}
    \draw (2,1) circle (0.7);
\node at (0.3,0.9) {$A$};
\node at (2.9,0.9) {$B$};
\node at (1.6,-0.5) {$A \cap B$};
\end{tikzpicture}
\end{image}


De Morgan's Law gives us a formula for the complement of a union in terms of the intersection of their complements.

\begin{proposition}[De Morgan's Law]
Given sets $A$ and $B$ in a universal set $U$ De Morgan's Law states that
\[
\left(A \cup B\right)^c = A^c \cap B^c
\]
\end{proposition}

\begin{proof}
First, let $x \in (A \cup B)^c$.  Then  $x \notin A \cup B$ which implies that $x \notin A$ and $x \notin B$. 
By definition of complement, this means that $x \in A^c$ and $x \in B^c$.   
Finally, by definition of intersection, $x \in A^c \cap B^c$.
Thus $(A \cup B)^c \subseteq A^c \cap B^c$.\\
Next, let $x \in A^c \cap B^c$. By definition of intersection, $x \in A^c$ and $x \in B^c$. 
Then, $x \notin A$ and $x \notin B$ which implies that $x \notin A \cup B$. Thus $x \in (A \cup B)^c$ and 
we have shown that  $A^c \cap B^c\subseteq (A \cup B)^c $.\\
By virtue of being subsets of each other, the two sets are equal and De Morgan's Law is proved.
\end{proof}

To take advantage of De Morgan's Law in counting, we need  to count the number of elements in the union of two sets.

\begin{proposition}[Inclusion-Exclusion Principle]
Given sets $A$ and $B$ with finite cardinalities $|A|$ and $|B|$, we have
\[
|A \cup B| = |A| + |B| - |A \cap B|
\]
\end{proposition}



\begin{proof}
Let $x \in A \cup B$.  Then $x$ is in $A, B$ or both.\\
Case 1. ($x$ is in both)
If $x$ is in both $A$ and $B$, then the right hand side of the formula accounts for $x$ twice, 
once as an element of $A$ and again as an element of $B$. But then, since $x$ is in 
both $A$ and $B$, $x$ is in their intersection, $A \cap B$.  
The coefficient of $|A \cap B|$ being $-1$, $x$ is ``de-accounted" for once in the 
right hand side of the formula.  In total, $x$ is accounted for twice and de-accounted for once, 
thereby accounting for it just once in total.\\
Case 2. ($x$ is in $A$ only)
If $x \in A$ but $x \notin B$, then the right hand side of the formula accounts 
for such $x$ exactly once, as a member of $A$.  Note that such an $x$ is not a 
member of $A \cap B$ and so it is not de-accounted for by the right hand side of the formula.\\
Case 3. ($x$ is in $B$ only)
This is the same as case 2 with the roles of $A$ and $B$ reversed.\\
Finally, we need to be sure that the right hand side of the formula does not count elements that are not in $A \cup B$.\\
Case 4. ($x$ is in neither $A$ nor $B$)
In this case, none of the terms on the right hand side of the formula will either account 
for or de-account for $x$ (since $x \notin A \cap B$ either).
\end{proof}


\begin{remark}
In our discussion of partitions, we did not have a subtraction term.  This is because for each pair of sets in a partition,
\[
S_i \cap S_j = \emptyset
\]
and hence $|S_i \cap S_j| = 0$.
\end{remark}


\begin{example}[example 1]
How many numbers in the set $S = \{1,2,3, \ldots, 1000\}$ are multiples of either 2 or 5?\\
Let $A$ be the subset of $S$ consisting of the multiples of 2, and $B$ be the subset of $S$ consisting of the multiples of 5.
We seek $|A \cup B|$. According to the proposition, 
\[
|A \cup B| = |A| + |B| - |A \cap B|
\]
We have $|A| =500$ and $|B| = 200$ and we still need $|A \cap B|$. The set $A \cap B$ consists of the multiples of 10, 
hence $|A \cap B| = 100$.  Now we can conclude that the number of elements of $S$ that are either multiples of 2 or 5 is
\[
|A \cup B| = 500 + 200 - 100 = 600
\]
\end{example}

\begin{problem}(problem 1)
How many nmumbers from the given set $S= \{1,2,3, \ldots, 1000\}$ are multiples of the given numbers $a$ and $b$?\\
a)$\; a = 4, b = 5 \;\; \answer{400}$\\
b)$\;  a = 7, b = 12 \;\; \answer{214}$\\
c)$\; a = 6, b = 10 \;\; \answer{233}$\\
d)$\;  a = 18, b = 30 \;\; \answer{77}$ \\
\end{problem}


\end{document}

\begin{example}[example 1]
Two 6-sided dice are thrown, one of which is red and the other green.  
In how many ways can at least one 6 be rolled.\\
We let $U$ be the set of all 36 possible outcomes of the form (red, green).
The set $S$ which we would like to enumerate is the subset of $U$ which contains a 6 as a component in at least one position.
The complement, $S^c$, is the set of outcomes which do not contain a 6 in either the red or the green component. It is easier to count the number of elements in $S^c$ than it is to count the number of elements in $S$, so we will use the rule for Complements.
\[
|S^c| = 5 \cdot 5 = 25
\]
since each component can be a number from 1 to 5. Now, the Rule for Complements gives
\[
|S| = |U| - |S^c| = 36 - 25 = 11
\]
Thus there are 11 outcomes where at least one of the die rolls is a 6.
\end{example}


\begin{problem}(problem 1a)
Two 6-sided dice are thrown, one of which is red and the other green.  
In how many ways can at least one 4 be rolled?\\
The number of outcomes with at least one 4 is $\answer{11}$.
\end{problem}

\begin{problem}(problem 1b)
Two 6-sided dice are thrown, one of which is red and the other green.  
In how many ways can their sum be at least 3?\\
The number of outcomes whose sum is at least 3 is $\answer{35}$.
\end{problem}


\begin{example}[example 2]
Two 6-sided dice are thrown, one of which is red and the other green.  
In how many ways can the sum be at most 10?\\
We let $U$ be the set of all 36 possible outcomes of the form (red, green).
The set $S$ which we would like to enumerate is the subset of $U$ whose components sum to at most 10. This means that their sum can be 2, 3, 4, ..., 10. This is most of the possible sums, so most likely, it will be easier to enumerate $S$ by examining its complement.
$S^c$ contains the outcomes whose sum is more than 10. That is, the sum is either 11 or 12. There are tow outcomes that give a sum of 11, namely $(5,6)$ and $(6,5)$ and there is one outcome that gives a sum of 12, $(6,6)$. Thus
\[
|S^c| = 2 + 1 = 3
\]
Note that $S^c$ was partitioned into two subsets here.
Now, by the Rule for Complements,
\[
|S| = |U| - |S^c| = 36 -3 = 33
\]
Thus there are 33 outcomes which have a sum of at most 10.
\end{example}


\begin{problem}(problem 2)
Two 6-sided dice are thrown, one of which is red and the other green.  
In how many ways can their sum be at least 5?\\
The number of outcomes whose sum is at least 5 is $\answer{30}$.
\end{problem}

\begin{example}[example 3]
Computer passwords are eight characters long where a character can be either an upper case letter, 
lower case letter, digit 0-9, or one of 8 special symbols. 
How many passwords are possible if a password must contain at least one special symbol?\\
With no restrictions on the symbols, we have seen that the total number of possibilities is 
\[
|U| = 70^8 = 576480100000000
\]
To enumerate the passwords with at least one special symbol, it will be helpful to consider the complement. 
That is, passwords with no special symbols. If we use only capital letters, 
lower case letters and digits only to construct our password, then there are
\[
|S^c| = 62^8 = 218340105584896
\]
possibilities. Thus, the Rule for Complements tells us that the number of passwords which contain at least one
special symbol is
\[
|S| = |U| - |S^c| = 70^8 - 62^8 = 358139994415104
\]
Thus there are over 358 trillion possible passwords if we insist on at least one special symbol.
\end{example}

\begin{problem}(problem 3a)
Computer passwords are eight characters long where a character can be either an upper case letter, 
lower case letter, digit 0-9, or one of 8 special symbols. 
How many passwords are possible if a password must contain at least one capital letter?\\
The number of passwords with at least one capital letter is  $\answer{70^8 - 44^8}$.
\end{problem}


\begin{example}[example 4]
How many $2 \times 2$ bit matrices contain at least one 1?\\
If we let $U$ be the set of all $2 \times 2$ bit matrices, then by the FPC, $|U| = 2^4 = 16$ We wish to find $|S|$,
where $S$ is the subset of $U$ consisting of $2 \times 2$ bit matrices with at least one 1. The complement, $S^c$
is the set of $2 \times 2$ bit matrices containing zero 1's and it is easy to enumerate.  In fact, $|S^c|= 1$, since the matrix must contain all 0's. Thus,
\[
|S| = |U| - |S^c| = 16 -1 = 15
\]
and there are 15 $2 \times 2$ bit matrices containing at least one 1.
\end{example}


\begin{problem}(problem 4)
How many $3 \times 2$ bit matrices contain at least one 0? \; $\answer{63}$
\end{problem}


\begin{example}[example 5]
A coin is flipped 5 times and the sequence of heads and tails is recorded.  
How many possible outcomes contain at least 2 heads?\\
By the FPC, the total number of outcomes for the 5 coin flips is $|U| = 2^5 = 32$. 
The condition at least two heads contains the possibilities of 2, 3, 4 or 5 heads.  
On the other hand, the complement consists of only the outcomes 0 or 1 heads, suggesting that we consider 
using the Rule for Complements. Since there is 1 way to get 0 heads (namely TTTTT), and there are 5 ways to get exactly 1 heads (the H can go in any one of 5 places, with the other 4 places being T's), we have
\[
|S^c| = |S_0| + |S_1| = 1 + 5 = 6
\]
By the Rule for Complements, 
\[
|S| = |U| - |S^c| = 32 -6 = 26
\]
Thus there are 26 outcomes consisting of 2 or more heads.
\end{example}

\begin{problem}(problem 5)
A coin is flipped 6 times and the sequence of heads and tails is recorded.\\
How many possible outcomes contain at least 2 tails? $\; \answer{57}$
\end{problem}

\end{document}



\begin{example}[example 5]
Doggy Daphne is having pups.  
Assuming that her litter consists of 5 pups, in how many ways can she have at least one male pup?\\
By the FPC, the total number of combinations of male and female pups is $|U| = 2^5 = 32$\\
If $S$ consists of all combinations with at least one male pup, then $|S^c| =1$, namely all female pups.
Thus, by the Rule for Complements, the number of possible 5 pup litters Daphne can have which consist of at least one male pup is
\[
|S| = |U| - |S^c| = 32 -1 = 31
\]
\end{example}

\begin{problem}(problem 5a)
Gracie the grizzly bear is about to give birth to 3 cubs. How many combinations of male and female cubs consisting of at most 2 male cubs are possible? $\; \answer{7}$.
\end{problem}

\begin{problem}(problem 5b)
Lioness Lucy is about to give birth to 4 cubs. How many combinations of male and female cubs consisting of at 
least 2 female cubs are possible? $\; \answer{11}$.
\end{problem}





\begin{example}[example 2] Steph has three different shirts, 
two different pairs of pants and 14 different hats. 
How many ways can Steph dress if the hat is optional?\\
If we denote by $S$ the set of all possible outfits Steph can wear, 
then we can partition $S$ into those outfits with a hat, $S_h$, 
and those without a hat, $S_{nh}$.
The number of outfits with a hat is $|S_h| = 3 \times 2 \times 14 = 84$ and 
the number of outfits with no hat is $|S_{nh}| = 3\times 2 = 6$. 
Thus, the total number of ways for Steph to dress is 
\[
|S| = |S_h| + |S_{nh}| = 84 + 6 = 90
\]
Note that we could have counted this slightly 
differently, by including ``no hat" as a $15^{th}$ option 
for the hat: $|S| = 3 \times 2 \times 15 = 90$.
\end{example}

\begin{problem}(problem 1c)
Two 6-sided dice are thrown, one of which is red and the other green.  
In how many ways can their sum be at least 5?\\
The number of outcomes whose sum is at least 5 is  $\answer{18}$.
\end{problem}


\begin{example}[example 3]
Sam wants to buy a slice of pizza.  
Sam enjoys three styles of pizza: Neapolitan, Chicago and Sicilian, 
and either eats the pizza plain or with pepperoni.  
How many different slices of pizza can Sam order?\\
If we denote by $S$ the set of all possible acceptable slices of pizza, 
then we can partition $S$ into the slices that have pepperoni, $S_p$ and 
those that do not, $S_{np}$. The number of slices of each is 3 and 
hence, the total number of possible slices Sam can order is
\[
|S| = |S_p| + |S_{np}| = 3+3 =6
\]
\end{example}

\begin{example}[example 4]
Alex and Charlie want to have 2 or fewer children.  
How many combinations of boys and girls are possible assuming 
that birth order of each gender is irrelevant?\\

If $S$ is the set of all possible genders of their children, then we can partition $S$ according to the number of children. Let $S_0$ be the number of ways for them to have 0 children. Then $|S_0| = 1$ because the only way to have 0 children is to have 0 boys and 0 girls. Let $S_1$ be the number of ways for them to have 1 child. Then $|S_0| = 2$ because they can have 1 boy and 0 girls or 1 girl and 0 boys.
 Let $S_2$ be the number of ways for them to have 2 children. Then $|S_2| = 3$ because they
 can have 2 boys and 0 girls, 2 girls and 0 boys or 1 of each.
 Thus the total number of combinations of boys and girls that they can have is
 \[
 |S| = |S_0|+|S_1|+|S_2|= 1 + 2+3 = 6
 \]
 \end{example}
 

\begin{problem}(problem 4)
Fluffy is about to give birth to either 5 or 6 kittens.  
How many combinations of males and females are possible assuming 
that birth order of each gender is irrelevant?\\
The total number of combinations of males and females that Fluffy can have is
$\answer{13}$
\end{problem}

\end{document}
