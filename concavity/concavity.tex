\documentclass{ximera}


%% You can put user macros here
%% However, you cannot make new environments



\newcommand{\ffrac}[2]{\frac{\text{\footnotesize $#1$}}{\text{\footnotesize $#2$}}}
\newcommand{\vasymptote}[2][]{
    \draw [densely dashed,#1] ({rel axis cs:0,0} -| {axis cs:#2,0}) -- ({rel axis cs:0,1} -| {axis cs:#2,0});
}


\graphicspath{{./}{firstExample/}}

\usepackage{amsmath}
\usepackage{amssymb}
\usepackage{array}
\usepackage[makeroom]{cancel} %% for strike outs
\usepackage{pgffor} %% required for integral for loops
\usepackage{tikz}
\usepackage{tikz-cd}
\usepackage{tkz-euclide}
\usetikzlibrary{shapes.multipart}


\usetkzobj{all}
\tikzstyle geometryDiagrams=[ultra thick,color=blue!50!black]


\usetikzlibrary{arrows}
\tikzset{>=stealth,commutative diagrams/.cd,
  arrow style=tikz,diagrams={>=stealth}} %% cool arrow head
\tikzset{shorten <>/.style={ shorten >=#1, shorten <=#1 } } %% allows shorter vectors

\usetikzlibrary{backgrounds} %% for boxes around graphs
\usetikzlibrary{shapes,positioning}  %% Clouds and stars
\usetikzlibrary{matrix} %% for matrix
\usepgfplotslibrary{polar} %% for polar plots
\usepgfplotslibrary{fillbetween} %% to shade area between curves in TikZ



%\usepackage[width=4.375in, height=7.0in, top=1.0in, papersize={5.5in,8.5in}]{geometry}
%\usepackage[pdftex]{graphicx}
%\usepackage{tipa}
%\usepackage{txfonts}
%\usepackage{textcomp}
%\usepackage{amsthm}
%\usepackage{xy}
%\usepackage{fancyhdr}
%\usepackage{xcolor}
%\usepackage{mathtools} %% for pretty underbrace % Breaks Ximera
%\usepackage{multicol}



\newcommand{\RR}{\mathbb R}
\newcommand{\R}{\mathbb R}
\newcommand{\C}{\mathbb C}
\newcommand{\N}{\mathbb N}
\newcommand{\Z}{\mathbb Z}
\newcommand{\dis}{\displaystyle}
%\renewcommand{\d}{\,d\!}
\renewcommand{\d}{\mathop{}\!d}
\newcommand{\dd}[2][]{\frac{\d #1}{\d #2}}
\newcommand{\pp}[2][]{\frac{\partial #1}{\partial #2}}
\renewcommand{\l}{\ell}
\newcommand{\ddx}{\frac{d}{\d x}}

\newcommand{\zeroOverZero}{\ensuremath{\boldsymbol{\tfrac{0}{0}}}}
\newcommand{\inftyOverInfty}{\ensuremath{\boldsymbol{\tfrac{\infty}{\infty}}}}
\newcommand{\zeroOverInfty}{\ensuremath{\boldsymbol{\tfrac{0}{\infty}}}}
\newcommand{\zeroTimesInfty}{\ensuremath{\small\boldsymbol{0\cdot \infty}}}
\newcommand{\inftyMinusInfty}{\ensuremath{\small\boldsymbol{\infty - \infty}}}
\newcommand{\oneToInfty}{\ensuremath{\boldsymbol{1^\infty}}}
\newcommand{\zeroToZero}{\ensuremath{\boldsymbol{0^0}}}
\newcommand{\inftyToZero}{\ensuremath{\boldsymbol{\infty^0}}}


\newcommand{\numOverZero}{\ensuremath{\boldsymbol{\tfrac{\#}{0}}}}
\newcommand{\dfn}{\textbf}
%\newcommand{\unit}{\,\mathrm}
\newcommand{\unit}{\mathop{}\!\mathrm}
%\newcommand{\eval}[1]{\bigg[ #1 \bigg]}
\newcommand{\eval}[1]{ #1 \bigg|}
\newcommand{\seq}[1]{\left( #1 \right)}
\renewcommand{\epsilon}{\varepsilon}
\renewcommand{\iff}{\Leftrightarrow}

\DeclareMathOperator{\arccot}{arccot}
\DeclareMathOperator{\arcsec}{arcsec}
\DeclareMathOperator{\arccsc}{arccsc}
\DeclareMathOperator{\si}{Si}
\DeclareMathOperator{\proj}{proj}
\DeclareMathOperator{\scal}{scal}
\DeclareMathOperator{\cis}{cis}
\DeclareMathOperator{\Arg}{Arg}
%\DeclareMathOperator{\arg}{arg}
\DeclareMathOperator{\Rep}{Re}
\DeclareMathOperator{\Imp}{Im}
\DeclareMathOperator{\sech}{sech}
\DeclareMathOperator{\csch}{csch}
\DeclareMathOperator{\Log}{Log}

\newcommand{\tightoverset}[2]{% for arrow vec
  \mathop{#2}\limits^{\vbox to -.5ex{\kern-0.75ex\hbox{$#1$}\vss}}}
\newcommand{\arrowvec}{\overrightarrow}
\renewcommand{\vec}{\mathbf}
\newcommand{\veci}{{\boldsymbol{\hat{\imath}}}}
\newcommand{\vecj}{{\boldsymbol{\hat{\jmath}}}}
\newcommand{\veck}{{\boldsymbol{\hat{k}}}}
\newcommand{\vecl}{\boldsymbol{\l}}
\newcommand{\utan}{\vec{\hat{t}}}
\newcommand{\unormal}{\vec{\hat{n}}}
\newcommand{\ubinormal}{\vec{\hat{b}}}

\newcommand{\dotp}{\bullet}
\newcommand{\cross}{\boldsymbol\times}
\newcommand{\grad}{\boldsymbol\nabla}
\newcommand{\divergence}{\grad\dotp}
\newcommand{\curl}{\grad\cross}
%% Simple horiz vectors
\renewcommand{\vector}[1]{\left\langle #1\right\rangle}


\outcome{Determine concavity}

\title{3.4 Concavity}

\begin{document}

\begin{abstract}
In this section we learn about the two types of curvature and determine the curvature of a function.
\end{abstract}

\maketitle

\section{Concavity}

In this section we will discuss the {\bf curvature} of the graph of a given function. 
There are two types of curvature: {\bf concave up} and {\bf concave down}.  
The main tool for discussing curvature is the second derivative, $f''(x)$.





\begin{definition}[Concavity]
Suppose $f(x)$ is differentiable on an open interval, $I$.
If $f'(x)$ is increasing on $I$, then $f(x)$ is \textbf{concave up} on $I$ and
if $f'(x)$ is decreasing on $I$, then $f(x)$ is \textbf{concave down} on $I$.
\end{definition} 


\begin{image}
\begin{tikzpicture}
\node at (5.5, 2.5) {concave down};
%\draw[<->] (4, 0) --(7, 0);
%\node at (3.2, 0) {$x$};
\draw[blue] (5.5,2) parabola (6.75,.25);
\draw[blue] (5.5,2) parabola (4.25,.25);

%\draw (1.5, .1) -- (1.5, -.1) node[below] {$x_0$};
%\node at (1.5, -.9) {$f''(x_0) < 0$};

\node at (1.5, 2.5) {concave up};
%\draw[<->] (0, 0) --(3, 0);
%\node at (3.2, 0) {$x$};

\draw[blue] (1.5,.25) parabola (0.25,2);
\draw[blue] (1.5,.25) parabola (2.75,2);
\draw[white] (0,0)--(0,0.1);
%\draw (5.5, .1) -- (5.5, -.1) node[below] {$x_0$};
%\node at (5.5, -.9) {$f''(x_0) > 0$};
\end{tikzpicture}
\end{image}


%Graphically, the above definitions correspond to the following.  
%If $f(x)$ is concave up on $I$, then the graph of $y=f(x)$
%lies above its tangent lines, and if $f(x)$ is concave down on $I$, 
%then the graph lies below its tangent lines.

%The simplest examples to illustrate this are $f(x) = x^2$, which is concave up everywhere and $f(x) = -x^2$
%which is concave down everywhere. 
%Make a rough sketch of each of these functions including a few tangent lines to see this.

The following theorem helps us to determine where a function is concave up and where it is concave down.\\

\begin{theorem}[Concavity]
If $f''(x) > 0$ on and interval $I$, then $f(x)$ is concave up on $I$ and
if $f''(x) < 0$ on an interval $I$, then $f(x)$ is concave down on $I$.
\end{theorem}



\begin{example}[example 1]
Consider the function $f(x) = x^2$.  It is concave up on the interval $(-\infty, \infty)$ since $f''(x) = 2 > 0$
for all $x$.
\end{example}

\begin{example}[example 2]
Consider the function $f(x) = e^x$.  It is concave up on the interval $(-\infty, \infty)$ since $f''(x) = e^x > 0$
for all $x$.
\end{example}


\begin{example}[example 3]
Consider the function $f(x) = x^3$.  Its second derivative is $f''(x) = 6x$ 
which has the same sign as $x$ itself. Thus $f(x) = x^3$ is concave up on the interval $(0, \infty)$
and concave down on the interval $(-\infty, 0)$.
\end{example}



The last example brings up a new concept.  The function $f(x) = x^3$ changes concavity at $x = 0$.  
We call this an {\bf inflection point} of the function.\\

\begin{definition}
If $f(x)$ is continuous at $x=a$ and $f(x)$ changes concavity at $x = a$, then
we say that $f(x)$ has an \textbf{inflection point} at $x = a$.
\end{definition} 

In the next two examples, we will discuss the curvature of the given functions and find their inflection points.


\begin{image}
\begin{tikzpicture}[scale=.7]
\draw[<->, thick] (-2, 3)-- (-2,.5)--(5,.5);
\draw[blue, thick]   plot[smooth,domain=-1.79:4.71] (\x, {2 + cos(deg(\x))});
\node at (1.57,4) {Inflection Point at $(a,f(a))$};
%\node at (.6,1.7) {$(a,f(a))$};
\node at (5.8, 2.2) {$y = f(x)$};
\draw[thin] (1.57, .4)--(1.57,.6);
\node at (1.57,.2) {$a$};
\draw[thin] (-2.1, 2)--(-1.9,2);
\node at (-2.7,2) {$f(a)$};
\draw[blue, fill] (1.57,2) circle [radius=0.07];

\end{tikzpicture}
\end{image}

 
 

\begin{example}[example 4]
Determine where the cubic polynomial $f(x) = x^3 + 12x^2 - 7x + 10$ is concave up, concave down and find the inflection points.\\
The second derivative of $f(x)$ is 
$f''(x) = 6x + 24$.  To determine where $f''(x)$ is positive and where it is negative, 
we will first determine where it is zero. Hence, we will solve the equation
\[f''(x) = 0\]
for $x$.

We have $6x + 24 = 0$ so $x = -4$.
This value breaks the real number line into two intervals, $(-\infty, -4)$ and $(-4, \infty)$.
The second derivative maintains the same sign throughout each of these intervals.  
To determine whether it is positive or negative, we choose a test point in each interval.
For the interval, $(-\infty, -4)$, we choose $x = -5$.  Plugging this into the second derivative, we get
$f''(-5) = 6(-5) + 24 = -6 < 0$. Next, we choose the test point 
$x = 0$ for the interval $(-4, \infty)$.  Plugging this into the second derivative gives $f''(0) = 6(0) + 24 = 24 > 0$.
We can use this information to create a sign chart for the second derivative, as shown below.

\begin{image}
\begin{tikzpicture}
\node at (.5,1.5) {Sign of $f''(x)$};

\node at (-5.5, .5) {- - - - -};
\node at (-2.5, .5) {+++};
\draw[thick, <->] (-7, 0) --(-1, 0);
\draw[thin] (-4, -.2) --(-4,.2);
%\draw[thin] (3,-.2) --(3,.2);
\node at (-4, -.5) {$-4$};
%\node at (3, -.5) {$3$};
\end{tikzpicture}
\end{image}
We can now use the concavity theorem to conclude that
the original function,
$f(x) = x^3 + 12x^2 - 7x +10$ is concave down on the interval $(-\infty, -4)$.
and concave up on the interval $(-4, \infty)$.
Finally, noting that the original polynomial is continuous everywhere and changes concavity at $x = -4$, we can say that
$f(x) = x^3 + 12x^2 - 7x +10$ has an inflection point at $x = -4$.

\begin{image}
\begin{tikzpicture}
\begin{axis}[axis lines = center,  axis y line=none, xmin=-10.5, xmax = 3.5,ymin = -50, ymax = 399, xtick={-4}, ytick={}, 
              title={Inflection point at $x = -4$}]
\addplot[smooth,blue, domain=-10:3, thick] {x^3 + 12*x^2 - 7*x + 10};
\addplot[blue, mark = *] coordinates{(-4,166)} node[right,blue] {inflection point};
\node[blue] at (axis cs:-2, 300){$y = x^3 + 12x^2 - 7x + 10$};
\node[blue] at (axis cs:-8, 250){\begin{tabular}{c} concave\\down \end{tabular}};
\node[blue] at (axis cs:0.2, 50){\begin{tabular}{c} concave\\up \end{tabular}};

\end{axis}
\end{tikzpicture}
\end{image}

\end{example}


\begin{problem}(problem 4a)
  Find the inflection point(s) of the function
  \[
     f(x) = x^3 - 6x^2 + 8x - 2.
  \]
 
    \begin{hint}
      Find the second derivative, $f''$
    \end{hint}
    \begin{hint}
      Where does $f''$ change sign?
    \end{hint}
    \begin{hint}
      Is $f$ continuous there?
    \end{hint}    
		The function $f(x) = x^3 - 6x^2 + 8x - 2$ has an inflection point at $x =$
		 $\answer[given]{2}$
	
\end{problem}


\begin{problem}(problem 4b)
  Find the inflection point(s) of the function
  \[
     f(x) = x^2 - x^3.
  \]
  
    \begin{hint}
      Find the second derivative, $f''$
    \end{hint}
    \begin{hint}
      Where does $f''$ change sign?
    \end{hint}
    \begin{hint}
      Is $f$ continuous there?
    \end{hint}    
		The function $f(x) = x^2 - x^3$ has an inflection point at $x =$
		 $\answer[given]{1/3}$
		
\end{problem}


\begin{example}[example 5]
Consider the cubic polynomial $f(x) = 2x^3 + 3x^2 - 36x +2$.  Its second derivative is 
$f''(x) = 12x + 6$.  To determine where $f''(x)$ is positive and where it is negative, 
we will first determine where it is zero. Hence, we will solve the equation
\[f''(x) = 0\]
for $x$.

We have $12x + 6 = 0$ so $x = -1/2$.
This value breaks the real number line into two intervals, $(-\infty, -1/2)$ and $(-1/2, \infty)$.
The second derivative maintains the same sign throughout each of these intervals.  
To determine whether it is positive or negative, we choose a test point in each interval.
For the interval, $(-\infty, -1/2)$, we choose $x = -1$.  Plugging this into the second derivative, we get
$f''(-1) = 12(-1) + 6 = -6 < 0$.  

Next, for the interval $(-1/2, \infty)$, we choose the test point 
$x = 0$.  Plugging this into the second derivative gives $f''(0) = 12(0) + 6 = 6 > 0$.

\begin{image}
\begin{tikzpicture}
\node at (.5,1.5) {Sign of $f''(x)$};
\draw[thick, <->] (-2.5, 0) --(1.5, 0);
\node at (-1.5, .5) {- - - - -};
\node at (0.5, .5) {+++};

\draw[thin] (-0.5, -.2) --(-0.5,.2);
%\draw[thin] (3,-.2) --(3,.2);
\node at (-0.5, -.5) {$-\frac12$};
%\node at (3, -.5) {$3$};
\end{tikzpicture}
\end{image}
We can now use the concavity theorem to conclude that
the original function,
$f(x) = 2x^3 + 3x^2 - 36x +2$ is concave down on the interval $(-\infty, -\frac12)$.
and concave up on the interval $(-\frac12, \infty)$.
Finally, noting that the original polynomial is continuous everywhere and changes concavity at $x = -\frac12$, 
we can say that
$f(x) = 2x^3 + 3x^2 - 36x +2$ has an inflection point at $x = -\frac12$.

\begin{image}
\begin{tikzpicture}
\begin{axis}[axis lines = center,  axis y line=none, xmin=-4.5, xmax = 3.5,ymin = -45, 
ymax = 90, xtick={-0.5}, xticklabels={$-\frac12$},ytick={}, 
              title={Inflection point at $x = -\frac12$}]
\addplot[smooth,blue, domain=-10:3, thick] {2*x^3 + 3*x^2 - 36*x +2};
\addplot[blue, mark = *] coordinates{(-1/2,20.5)} node[right,blue] {inflection point};
\node[blue] at (axis cs:0.2, 80){$y = 2x^3 + 3x^2 - 36x +2$};
\node[blue] at (axis cs:-3, 55){\begin{tabular}{c} concave\\down \end{tabular}};
\node[blue] at (axis cs:1.8, -20){\begin{tabular}{c} concave\\up \end{tabular}};

\end{axis}
\end{tikzpicture}
\end{image}

\end{example}

\begin{problem}(problem 5a)
  Find the inflection point(s) of the function
  \[
     f(x) = x^4 - 4x^3.
  \]
  
    \begin{hint}
      Find the second derivative, $f''$
    \end{hint}
    \begin{hint}
      Where does $f''$ change sign?
    \end{hint}
    \begin{hint}
      Is $f$ continuous there?
    \end{hint}  
		
		List multiple answers in ascending order.\\
		The function $f(x) = x^4 - 4x^3$ has inflection points at 
		$x =\answer{0}$ and $x = \answer{2}$.
		
	
\end{problem}


\begin{problem}(problem 5b)
  Find the inflection point(s) of the function
  \[
     f(x) = x^4 + 6x^2 - 5x + 2.
  \]
  
    \begin{hint}
      Find the second derivative, $f''$
    \end{hint}
    \begin{hint}
      Where does $f''$ change sign?
    \end{hint}
    \begin{hint}
      Is $f$ continuous there?
    \end{hint}  
		If there are no inflection points, type ``none".\\ 
		The function $f(x) = x^4 + 6x^2 - 5x + 2$ has inflection 
		point(s) at $x =\answer[given]{none}$
	
\end{problem}

\begin{problem}(problem 5c) Find the intervals of increase/decrease, local extremes, intervals of concavity and inflection points for the function
\[
f(x) = x^5 - 15x^3 + 40x + 11.
\]
\end{problem}


\begin{example}[example 6]
Consider the function $f(x) = x^2e^{3x}$.  To find its second derivative, 
we will need to use the product rule twice. First, 
\[f'(x) = x^2 \cdot 3e^{3x} + 2xe^{3x} = (3x^2 + 2x)e^{3x},\]
and second
\[f''(x) = (3x^2 + 2x)\cdot 3e^{3x} + (6x + 2)e^{3x} = (9x^2 + 12x + 2)e^{3x}.\]
Now that we have the second derivative, we set it equal to zero.  Solve
\[(9x^2 + 12x + 2)e^{3x}=0\]
for $x$.
Since the exponential $e^{3x}$ is never equal to zero, the only solutions come from setting the quadratic to zero:
\[9x^2 + 12x + 2=0.\]
This quadratic does not factor, so we need to use the quadratic formula.
The solutions are 
\[x = \frac{-12 \pm \sqrt{12^2 - 4(9)(2)}}{2(9)} = \frac{-2 \pm \sqrt{2}}{3}.\]
For simplicity, we will call these two roots $\alpha$ and $\beta$.
So 
\[\alpha = \frac{-2 - \sqrt{2}}{3} \approx -1.138\]
and
\[\beta = \frac{-2 + \sqrt{2}}{3} \approx -0.195\]

These two values break the real number line into three intervals: $(-\infty, \alpha), 
(\alpha, \beta)$ and $(\beta, \infty)$ with test points
$x = -2, x = -1$ and $x = 0$, respectively.
Plugging these into the second derivative gives
\[f''(-2) = (9(-2)^2 + 12(-2) + 2)e^{3(-2)} = 14e^{-6} > 0,\]
\[f''(-1) = (9(-1)^2 + 12(-1) + 2)e^{3(-1)} = -e^{-3} < 0,\]
and
\[f''(0) = (9(0)^2 + 12(0) + 2)e^{3(0)} = 2 > 0.\]

\begin{image}
\begin{tikzpicture}
\node at (-.75,1.5) {Sign of $f''(x)$};
\draw[thick, <->] (-2.5, 0) --(1, 0);
\node at (-1.8, .5) {+++};
\node at (-0.65, .5) {- - - -};
\node at (0.5, .5) {+++};
\draw[thin] (-1.138, -.2) --(-1.138,.2);

\node at (-1.138, -.5) {$\alpha$};
\draw[thin] (-0.195, -.2) --(-0.195,.2);

\node at (-0.195, -.5) {$\beta$};

\end{tikzpicture}
\end{image}
We can now use the concavity theorem to conclude that
the original function, $f(x) = x^2 e^{3x}$ is concave up on the intervals $(-\infty, \alpha)$ 
and $(\beta, \infty)$ and it is concave down on the interval 
$(\alpha, \beta)$.
Finally, we can infer from this that the continuous function $f(x) = x^2 e^{3x}$ has inflection points at 
\[
x = \alpha= \dfrac{-2 - \sqrt{2}}{3} \quad \text{and} \quad x = \beta = \dfrac{-2 + \sqrt{2}}{3}.
\]

\begin{image}
\begin{tikzpicture}
\begin{axis}[axis lines = center,  axis y line=none, xmin=-2.5, xmax = 0.5,ymin = 0, 
ymax = .2, xtick={-1.138, -0.195}, xticklabels={$\alpha$, $\beta$},ytick={}, 
              title={Inflection points at $x = \alpha$ and $x=\beta$}]
\addplot[smooth,blue, domain=-2:0.3, thick] {x^2*e^(3*x)};
\addplot[blue, mark = *] coordinates{(-1.138,0.0426)} ;
\addplot[blue, mark = *] coordinates{(-0.195,0.021224)} ;
\node[blue] at (axis cs:-0.1, .15){$y = x^2e^{3x}$};
\node[blue] at (axis cs:-1.8, .04){\begin{tabular}{c} concave\\up \end{tabular}};
\node[blue] at (axis cs:-0.7, .03){\begin{tabular}{c} concave\\down \end{tabular}};
\node[blue] at (axis cs:-0.04, .03){\begin{tabular}{c} c.\\up \end{tabular}};

\end{axis}
\end{tikzpicture}
\end{image}

\end{example}

\begin{problem}(problem 6a)
  Find the inflection point(s) of the function
  \[
     f(x) = xe^x
  \]
  
    \begin{hint}
      Find the second derivative, $f''$
    \end{hint}
    \begin{hint}
      Use the product rule to compute the derivatives
    \end{hint}
		\begin{hint}
      $f'(x) = (x+1)e^x$ and $f''(x)$ is similar
    \end{hint}
		\begin{hint}
      Where does $f''$ change sign?
    \end{hint}
    \begin{hint}
      Is $f$ continuous there?
    \end{hint}  
		The function $f(x) = xe^x$ has an inflection 
		point at $x =\answer[given]{-2}$
		
\end{problem}

\begin{problem}(problem 6b)
  Find the inflection point(s) of the function
  \[
     f(x) = x^4e^x
  \]
  
    \begin{hint}
      Find the second derivative, $f''$
    \end{hint}
    \begin{hint}
      Use the product rule to compute the derivatives
    \end{hint}
		\begin{hint}
      $f'(x) = (x^4+4x^3)e^x$ and $f''(x)$ is similar
    \end{hint}
		\begin{hint}
      Where does $f''$ change sign?
    \end{hint}
    \begin{hint}
      Is $f$ continuous there?
    \end{hint} 
		List multiple answers in ascending order.\\
		The function $f(x) = x^4e^x$ has inflection points at 
		$x =\answer[given]{-6}$ and $x =\answer[given]{-2}$
		
\end{problem}


\begin{problem}(problem 6c)
  Find the inflection point(s) of the function 
  \[
     f(x) = \cos(x)
  \]
	in the interval $[0, 2\pi].$\\
 
    \begin{hint}
      Find the second derivative, $f''$
    \end{hint}
		\begin{hint}
      Where does $f''$ change sign?
    \end{hint}
    \begin{hint}
      Is $f$ continuous there?
    \end{hint}  
		List multiple answers in ascending order.\\
	On the interval $[0, 2\pi]$, the function $f(x) = \cos(x)$ has inflection points at $x =\answer{\pi/2}$ and $x=\answer{3\pi/2}$.
		
	
\end{problem}



\begin{example}[example 7]
Consider the function $f(x) = e^{-x^2}.$ By the chain rule, its derivative is 
\[
f'(x) = -2xe^{-x^2}
\]
and by the product rule (and the chain rule again, as well), its second derivative is
\[
f''(x) = (-2x)(-2x)e^{-x^2} + (-2)e^{-x^2} = (4x^2 - 2)e^{-x^2}.
\]
Setting the second derivative equal to zero and noting that the exponential $e^{-x^2}$ is
always positive, we get
\[
4x^2 - 2 = 0.
\]
Solving for $x$ gives
\[
x^2 = \frac 12 \implies  x = \pm \frac{1}{\sqrt{2}} \approx \pm 0.707
\]

These two values of $x$ break the real number line into three intervals: $(-\infty, -\tfrac{1}{\sqrt 2}),
(-\tfrac{1}{\sqrt 2}, \tfrac{1}{\sqrt 2})$ and $(\tfrac{1}{\sqrt 2}, \infty)$. 
The sign of $f''(x)$ will remain the same on each of these intervals. 
To determine the sign on each interval, we use the test points
$x = -1, x = 0$ and $x = 1$ respectively.
Plugging the test points into the second derivative gives
\[
f''(-1) = (4(-1)^2 - 2) e^{-(-1)^2} = 2e^{-1} > 0
\]
\[
f''(0) = (4(0)^2 - 2) e^{-(0)^2} = -2 < 0
\]
and
\[
f''(1) = (4(1)^2 - 2) e^{-(1)^2} = 2e^{-1} > 0.
\]


\begin{image}
\begin{tikzpicture}
\node at (0,1.5) {Sign of $f''(x)$};
\draw[thick, <->] (-2.5, 0) --(2.5, 0);
\node at (-1.6, .5) {+++};
\node at (0, .5) {- - - -};
\node at (1.6, .5) {+++};
\draw[thin] (-0.707, -.2) --(-0.707,.2);

\node at (-0.707, -.7) {$-\frac{1}{\sqrt 2}$};
\draw[thin] (0.707, -.2) --(0.707,.2);

\node at (0.707, -.7) {$\frac{1}{\sqrt 2}$};

\end{tikzpicture}
\end{image}

We can now use the concavity theorem to conclude that
the original function, $f(x) = e^{-x^2}$
is concave up on the intervals $(-\infty, -\frac{1}{\sqrt 2})$ and $(\frac{1}{\sqrt 2}, \infty)$ and 
concave down on the interval $(-\frac{1}{\sqrt 2},\frac{1}{\sqrt 2})$. 
Finally, since the original function is continuous everywhere, we can
say that $x = \pm \frac{1}{\sqrt 2}$ are inflection points for $f(x) = e^{-x^2}$.

\begin{image}
\begin{tikzpicture}
\begin{axis}[axis lines = center,  axis y line=none, xmin=-3, xmax = 3,ymin = 0, 
ymax = 1.2, xtick={-.707, .707}, xticklabels={$-\frac{1}{\sqrt 2}$, $\frac{1}{\sqrt 2}$},ytick={}, 
              title={Inflection points at $x = \pm\frac{1}{\sqrt 2}$}]
\addplot[smooth,blue, domain=-2.5:2.5, thick] { e^(-x^2)};
\addplot[blue, mark = *] coordinates{(-.707,.607)} ;
\addplot[blue, mark = *] coordinates{(.707,.607)} ;
\node[blue] at (axis cs:1.2, 1){$y =  e^{-x^2}$};
\node[blue] at (axis cs:-1.8, .2){\begin{tabular}{c} concave\\up \end{tabular}};
\node[blue] at (axis cs:0, .75){\begin{tabular}{c} conc.\\down \end{tabular}};
\node[blue] at (axis cs:1.8, .2){\begin{tabular}{c} convave\\up \end{tabular}};

\end{axis}
\end{tikzpicture}
\end{image}

\end{example}

\begin{problem}(problem 7)
  Find the inflection point(s) of the function
  \[
     f(x) = \frac{1}{1+x^2}
  \]
 
    \begin{hint}
      Find the second derivative, $f''$
    \end{hint}
    \begin{hint}
      Use the Quotient Rule to compute the derivatives
    \end{hint}
		\begin{hint}
      $f'(x) = -\frac{2x}{1+2x^2+x^4}$
    \end{hint}
		\begin{hint}
      Where does $f'' = 0$?
    \end{hint}
		\begin{hint}
		 To solve the equation $3x^4 + 2x^2 - 1 = 0$, \\
		 let $u = x^2$ and solve for $u$ first
    \end{hint}
		\begin{hint}
      Is $f$ continuous there?
    \end{hint}  
		List multiple answers in ascending order.\\
		The function $f(x) = \frac{1}{1+x^2}$ has inflection 
		points at $x =\answer[given]{-1/\sqrt 3}$ and $x=\answer{1/\sqrt 3}$.
		
		%2+4x^2 +2x^4 - (8x^2 + 8x^4) 2 - 4x^2 - 6x^4}$  3u^2 + 2u - 1 = 0  (3u - 1)(u + 1)
		
\end{problem}



\begin{example}[example 8]
Consider the function $f(x) = \dfrac{1}{x}.$ Its derivative is 
\[
f'(x) = -\frac{1}{x^2}
\]
and its second derivative is
\[
f''(x) = \frac{2}{x^3}.
\]
Now we make the observation that $x^3$ has the same sign as $x$ and so $f''(x)$ will have the same sign as $x$
in this example. Thus $f''(x) > 0$ on the interval $(0, \infty)$ and $f''(x) < 0$ on the interval $(-\infty,0).$
We conclude that $f(x)$ is concave up on $(0, \infty)$ and concave down on the interval $(-\infty, 0).$ 
Since $f(x) = \frac{1}{x}$ is \textbf{not} continuous at $x=0$ (it has an infinite discontinuity there), 
there is no inflection point there.  

%\begin{image}
%\begin{tikzpicture}
%\begin{axis}[axis lines=middle,samples=200]
%\addplot[blue,domain=-2:-0.15] {1/x};
%\addplot[blue,domain=0.15:2] {1/x};
%\draw[red!20,dashed] (axis cs:2,-4) -- (axis cs:2,10); (draws vertical asymptote at x = 2)
%\node at (axis cs:1.25,2) {concave up};
%\node at (axis cs:-1.25,-3) {concave down};
%\end{axis}
%\end{tikzpicture}
%\caption{Concavity of y = 1/x}
%\end{image}
\end{example}


\begin{problem}(problem 8)
  Find the inflection point(s) of the function
  \[
     f(x) = \tan(x)
  \]
	on the interval $(-\pi/2, \pi/2)$.\\
  
    \begin{hint}
      Find the second derivative, $f''$
    \end{hint}
    \begin{hint}
      The derivative of $\tan(x)$ is $\sec^2(x)$
    \end{hint}
		\begin{hint}
      Use the Chain Rule to find the derivative of $\sec^2(x)$
    \end{hint}
		\begin{hint}
      Where does $f'' = 0$?
    \end{hint}
	On the interval $(-\pi/2, \pi/2)$, the function $f(x) = \tan(x)$ \\
	has an inflection point at $x =\answer[given]{0}$.
 
\end{problem}

\section{Using the graph of the derivative}
In this section, we are given the graph of the derivative, $f'(x)$, and we are asked to make conclusions about the original function, $f(x)$.

\begin{example}[example 9]
Use the graph of $f'(x)$, shown below, to answer the following questions about the graph of $f(x)$.
Where is $f(x)$ increasing/decreasing and where are its local extremes?
Where is $f(x)$ concave up/down and where are its inflection points?

\begin{image}
\begin{tikzpicture}[scale=.6]
\draw[<->, thick] (-4, 0)-- (4,0);
\draw[<->, thick] (0, -1.5)-- (0, 3.5);
\node at (0, 4.8) {The graph of the derivative, $y = f'(x)$};
\node at (4.3,0) {$x$};
\node at (0,3.8) {$y$};
\draw[blue, thick]   plot[smooth,domain=-4:4] (\x, {-1 + \x*\x/4});
\node at (-2.3,-0.65) {$-2$};
\node at (2.1,-0.65) {$2$};
\node at (4.9, 1.8) {$y = f'(x)$};
\draw[thin] (-2, -.3)--(-2,.3);

\draw[thin] (2, -0.3)--(2, 0.3);

%\draw[blue, fill] (1.57,2) circle [radius=0.07];

\end{tikzpicture}
\end{image}

To answer the first question, recall that when $f'(x) > 0$, then $f(x)$ is increasing,
and when $f'(x) < 0$, then $f(x)$ is decreasing.  From the graph, we see that $f'(x)$ is positive on the intervals 
$(-\infty, -2)$ and $(2, \infty)$.  Hence, the graph of $f(x)$ is increasing on these intervals. We can also see that 
$f'(x)$ is negative on the interval $(-2, 2)$ and therefore $f(x)$ is decreasing on this interval.
We can now use the first derivative test to determine the nature of the local extremes. Since $f'(-2) = 0$ and $f'(x)$
changes sign from positive to negative at $x = -2$, the function, $f(x)$, has a local maximum at $x = -2$.
Similarly, Since $f'(2) = 0$ and $f'(x)$
changes sign from negative to positive at $x = 2$, the function, $f(x)$, has a local minimum at $x = 2$.\\
To determine the concavity of $f(x)$,recall that $f(x)$ is concave up when $f'(x)$ is increasing and $f(x)$
is concave down when $f'(x)$ is decreasing.  From the graph, we see that $f'(x)$ is increasing on the interval $(0, \infty)$,
and decreasing on the interval $(-\infty, 0)$. Hence, the graph of $f(x)$ is concave up on $(0, \infty)$
and concave down on $(-\infty, 0)$.  Finally, $f(x)$ has an inflection point at $x = 0$ due to the change in concavity there.


\end{example}

\begin{problem}(problem 9)
Use the graph of the derivative, $f'(x)$, given below, to answer the following questions.

\begin{image}
\begin{tikzpicture}[scale=.6]
\draw[<->, thick] (-4, 0)-- (4,0);
\draw[<->, thick] (0, -1.5)-- (0, 3.5);
\node at (0, 4.8) {The graph of the derivative, $y = f'(x)$};
\node at (4.3,0) {$x$};
\node at (0,3.8) {$y$};
\draw[blue, thick]   plot[smooth,domain=-4:4] (\x, {2 - \x*\x/4});
\node at (-2.83,-0.65) {$-3$};
\node at (2.83,-0.65) {$3$};
\node[blue] at (4.9, 1.8) {$y = f'(x)$};
\draw[thin] (-2.83, -.3)--(-2.83,.3);

\draw[thin] (2.83, -0.3)--(2.83, 0.3);

%\draw[blue, fill] (1.57,2) circle [radius=0.07];

\end{tikzpicture}
\end{image}

Where is $f(x)$ increasing? Select all that apply.

\begin{selectAll}
\choice{$(0, \infty)$}
\choice{$(-\infty, 0)$}
\choice{$(3, \infty)$}
\choice[correct]{$(-3, 3)$}
\choice{$(-\infty, -3)$}
\end{selectAll}

Where is $f(x)$ decreasing? Select all that apply.

\begin{selectAll}
\choice{$(0, \infty)$}
\choice{$(-\infty, 0)$}
\choice[correct]{$(-\infty, -3)$}
\choice{$(-3,3)$}
\choice[correct]{$(3, \infty)$}
\end{selectAll}


Describe the local extremes of $f(x)$. Select all that apply.
\begin{selectAll}
\choice{local maximum at $x = -3$}
\choice[correct]{local maximum at $x = 3$}
\choice{local minimum at $x = 0$}
\choice[correct]{local minimum at $x = -3$}
\choice{local minimum at $x = 3$}
\end{selectAll}

Where is $f(x)$ concave up? Select all that apply.
\begin{selectAll}
\choice{$(0, \infty)$}
\choice[correct]{$(-\infty, 0)$}
\choice{$(-\infty, \infty)$}
\choice{$(-3, 3)$}
\choice{$(3, \infty)$}
\end{selectAll}


Where is $f(x)$ concave down? Select all that apply.
\begin{selectAll}
\choice[correct]{$(0, \infty)$}
\choice{$(-\infty, 0)$}
\choice{$(-\infty, \infty)$}
\choice{$(-3, 3)$}
\choice{$(3, \infty)$}
\end{selectAll}


Where are the inflection points of $f(x)$? Select all that apply.
\begin{selectAll}
\choice{inflection point at $x = -2$}
\choice[correct]{inflection point at $x = 0$}
\choice{inflection point at $x = 2$}
\choice{no inflection points}
\end{selectAll}
\end{problem}



\begin{example}[example 10]
Use the graph of $f'(x)$, shown below, to answer the following questions about the graph of $f(x)$.
Where is $f(x)$ increasing/decreasing and where are its local extremes?
Where is $f(x)$ concave up/down and where are its inflection points?

\begin{image}
\begin{tikzpicture}[scale=.6]
\draw[<->, thick] (-1.5, 0)-- (4,0);
\draw[<->, thick] (0, -3)-- (0, 3);
\node at (1.25, 4.4) {The graph of the derivative, $y = f'(x)$};
\node at (4.3,0) {$x$};
\node at (0,3.4) {$y$};
\draw[blue, thick]   plot[smooth,domain=-1.3:3.8] (\x, {\x*(\x - 2)*(\x-2)/4});
%\node at (-2.3,-0.65) {$-2$};

\node at (4.9, 1.8) {$y = f'(x)$};
%\draw[thin] (-2, -.3)--(-2,.3);
\node at (2.1,-0.65) {$2$};
\draw[thin] (2, -0.2)--(2, 0.2);
\node at (.67,-0.65) {$.7$};
\draw[thin] (.67, -0.2)--(.67, 0.2);

%\draw[blue, fill] (1.57,2) circle [radius=0.07];

\end{tikzpicture}
\end{image}

To answer the first question, recall that when $f'(x) > 0$, then $f(x)$ is increasing,
and when $f'(x) < 0$, then $f(x)$ is decreasing.  From the graph, we see that $f'(x)$ is positive on the intervals 
$(0, 2)$ and $(2, \infty)$.  Hence, the graph of $f(x)$ is increasing on these intervals. We can also see that 
$f'(x)$ is negative on the interval $(-\infty, 0)$ and therefore $f(x)$ is decreasing on this interval.
We can now use the first derivative test to determine the nature of the local extremes. Since $f'(0) = 0$ and $f'(x)$
changes sign from negative to positive at $x = 0$, the function, $f(x)$, has a local minimum at $x = 0$.
The situation at $x = 2$ is different. From the graph, we see that $f'(2) = 0$ but $f'(x)$ does not change sign at $x = 2$- 
it is positive on either side. Thus, $f(x)$ does not have a local extreme at the critical number, $x = 2$.\\
To determine the concavity of $f(x)$,recall that $f(x)$ is concave up when $f'(x)$ is increasing and $f(x)$
is concave down when $f'(x)$ is decreasing.  From the graph, we see that $f'(x)$ is increasing on the intervals $(-\infty, .7)$
and $(2, \infty)$ and decreasing on the interval $(0.7, 2)$. Hence, the graph of $f(x)$ is concave up on $(-\infty, .7)$ and $(2, \infty)$
and concave down on $(.7, 2)$.  Finally, $f(x)$ has an inflection points at $x = .7$ and $x = 2$ due to the changes in concavity at these points.


\end{example}

\begin{problem}(problem 10)
Use the graph of the derivative, $f'(x)$, given below, to answer the following questions.

\begin{image}
\begin{tikzpicture}[scale=.6]
\draw[<->, thick] (-1.5, 0)-- (5,0);
\draw[<->, thick] (0, -3)-- (0, 3.2);
\node at (1.75, 4.6) {The graph of the derivative, $y = f'(x)$};
\node at (5.3,0) {$x$};
\node at (0,3.6) {$y$};
\draw[blue, thick]   plot[smooth,domain=-1.1:4.8] (\x, {\x*(\x - 3)*(\x-3)/6});
%\node at (-2.3,-0.65) {$-2$};

\node at (5.9, 1.8) {$y = f'(x)$};
%\draw[thin] (-2, -.3)--(-2,.3);
\node at (3.1,-0.65) {$3$};
\draw[thin] (3, -0.2)--(3, 0.2);
\node at (1,-0.65) {$1$};
\draw[thin] (1, -0.2)--(1, 0.2);

%\draw[blue, fill] (1.57,2) circle [radius=0.07];

\end{tikzpicture}
\end{image}

Where is $f(x)$ increasing? Select all that apply.

\begin{selectAll}
\choice[correct]{$(0, 3)$}
\choice{$(-\infty, 1)$}
\choice[correct]{$(3, \infty)$}
\choice{$(-\infty, \infty)$}
\choice{$(-\infty, 0)$}
\end{selectAll}

Where is $f(x)$ decreasing? Select all that apply.

\begin{selectAll}
\choice{$(0, \infty)$}
\choice[correct]{$(-\infty, 0)$}
\choice{$(-\infty, \infty)$}
\choice{$(1, 3)$}
\choice{$(1, \infty)$}
\end{selectAll}


Describe the local extremes of $f(x)$. Select all that apply.
\begin{selectAll}
\choice{local maximum at $x = 1$}
\choice{local maximum at $x = 3$}
\choice[correct]{local minimum at $x = 0$}
\choice{local minimum at $x = 1$}
\choice{local minimum at $x = 3$}
\end{selectAll}

Where is $f(x)$ concave up? Select all that apply.
\begin{selectAll}
\choice{$(0, \infty)$}
\choice{$(-\infty, 0)$}
\choice[correct]{$(-\infty, 1)$}
\choice{$(-3, 3)$}
\choice[correct]{$(3, \infty)$}
\end{selectAll}


Where is $f(x)$ concave down? Select all that apply.
\begin{selectAll}
\choice{$(0, \infty)$}
\choice{$(0, 1)$}
\choice{$(-\infty, \infty)$}
\choice[correct]{$(1, 3)$}
\choice{$(3, \infty)$}
\end{selectAll}


Where are the inflection points of $f(x)$? Select all that apply.
\begin{selectAll}
\choice{inflection point at $x = 0$}
\choice[correct]{inflection point at $x = 1$}
\choice[correct]{inflection point at $x = 3$}
\choice{no inflection points}
\end{selectAll}
\end{problem}

%2\sec(x) \sec(x) \tan(x) 

\section{Video lessons}
\begin{center}
\begin{foldable}
\unfoldable{Here is a detailed, lecture style video on concavity:}
\youtube{zc6tD-g3ue8}
\end{foldable}
\end{center}


We next explore a special relationship between concavity and local extremes.

\begin{theorem}[Second Derivative Test]
Suppose $f'(a) = 0$.
If $f''(a) < 0$, then $f(x)$ has a local maximum at $x = a$, and
if $f''(a) > 0$, then $f(x)$ has a local minimum at $x = a$
\end{theorem}

The following figure should convince the reader of the validity of the Second Derivative Test.


\begin{image}
\begin{tikzpicture}
\node at (4, 3.5) {The Second Derivative Test};
\node at (1.5, 2.5) {Local Maximum at $x=a$};
\draw[<->] (0, 0) --(3, 0);
%\node at (3.2, 0) {$x$};
\draw[blue] (1.5,2) parabola (2.75,.25);
\draw[blue] (1.5,2) parabola (0.25,.25);
\draw[blue, fill] (1.5,2) circle [radius=0.07];
\draw[red] (0.5,2)--(2.5,2);

\draw (1.5, .1) -- (1.5, -.1) node[below] {$a$};
\node at (1.5, -.9) {$f'(a) = 0 \text{ and } f''(a) < 0$};

\node at (6.5, 2.5) {Local Minimum at $x=a$};
\draw[<->] (5, 0) --(8, 0);
%\node at (3.2, 0) {$x$};

\draw[blue] (6.5,.25) parabola (5.25,2);
\draw[blue] (6.5,.25) parabola (7.75,2);
\draw (6.5, .1) -- (6.5, -.1) node[below] {$a$};
\node at (6.5, -.9) {$f'(a) = 0 \text{ and } f''(a) > 0$};
\draw[blue, fill] (6.5,.25) circle [radius=0.07];
\draw[red] (5.5,0.25)--(7.5,0.25);
\end{tikzpicture}
\end{image}


\begin{example}[example 11]
Use the Second Derivative Test to find the local extremes of 
\[
f(x) = 2x^3 - 3x^2 - 36x +2.
\]
The critical numbers of $f(x)$ are $x = -2$ and $x = 3$ (verify).
Next, $f''(x) = 12x - 6$ (verify).
Plugging the critical numbers into the second derivative gives,
\[f''(-2) = -30 < 0 \; \text{and} \; f''(3) = 30 >0.\]
By the Second Derivative Test, $f(x) = 2x^3 - 3x^2 -36x +2$ has a local maximum at $x = -2$ and a local minimum at $x =3$.

\end{example}

\begin{problem}(problem 11)
Use the Second derivative Test to find the local extremes of
\[
f(x) = x^3 - 9x^2 + 24x - 5.
\]
The critical numbers are (list in ascending order) $x= \answer{2}$ and $\answer{4}$.\\
$f(x)$ has a local maximum at $x = \answer{2}$ and a local minimum at $x = \answer{4}.$
\end{problem}



\end{document}


































add second derivative test

add problems where the graph of the derivative is given

add multiple choice problems where the students select a graph with given properties


Here is a graph of the derivative $f'(x)$:
\begin{center}
$y=f'(x)$\\
\begin{tikzpicture}[scale=.5]
\draw [<->,thick] (-8,0) to (8,0);
\draw [<->,thick] (0,-4) to (0, 6);
\draw[dotted] (-8,-4) grid (8, 6);
\foreach \x in {-8,-6,-4,-2,0,1,3,5,7}
\draw (\x,1pt) -- (\x,-3pt)
			node[below] {\x};
    \foreach \y in {-4,-2,1,3,5}
     \draw (1pt,\y) -- (-3pt,\y)
     	node[anchor=east] {\y};
\draw [<->,thick] (-8,0) -- (8,0) node[right] {$x$};
\draw [<->,thick] (0,-4) to (0, 6);
\draw [<->,ultra thick] (-8,6) to [out=292,in=168] (-4,0) to [out=12,in=186] (0,4) to [out=352,in=95] (2,0) to [out=276,in=176](4,-3) to [out=10,in=250] (6,0) to (7.5,6) ;
\end{tikzpicture}
\end{center}

\begin{center}
\begin{tikzpicture}
\begin{axis}[axis x line=middle, axis y line= middle, xlabel={$x$}, ylabel={$y$}, axis equal, xtick={2},
xticklabels={$x_0$}, title={A local maximum}]
%\addplot[<->] coordinates{0.75, 0) (3.25, 0)};
\node at (3.25, 0.3) {$x$};
\addplot[domain=1:3]{3 - (x -2)^2};
%\addplot[dashed] coordinates{(0.75, 0)  (0.75, 1.682)};
%\addplot[dashed] coordinates{ (0.75, 1.682) (0, 1.682)};
%\addplot[dashed] coordinates{(1.5, 0) (1.5, 2.828)};
%\addplot[dashed] coordinates{(1.5, 2.828) (0, 2.828)};

\end{axis}
%\addplot[mark=*,fill=white] coordinates {(0,0)}
\node at (4, -1) {$f''(x_0) < 0$};				
\node at (6.5, 5) {$y = f(x)$};
 

\end{tikzpicture}
\hspace{1 in}
\begin{tikzpicture}
\begin{axis}[axis x line=none, axis y line= none, xlabel={$x$}, ylabel={$y$}, axis equal, xtick={2},
xticklabels={$x_0$}, title={A local minimum}]
%\addplot[<->] coordinates{0.75, 0) (3.25, 0)};
\addplot[domain=1:3]{1 + (x-2)^2};
%\addplot[dashed] coordinates{(0.5, 0)  (0.5, 0.707)};
%\addplot[dashed] coordinates{ (0.5, 0.707) (0, 0.707)};
%\addplot[dashed] coordinates{(1.5, 0) (1.5, 0.3536)};
%\addplot[dashed] coordinates{(1.5, 0.3536) (0, 0.3536)};

\end{axis}
%\addplot[mark=*,fill=white] coordinates {(0,0)}
\node at (4, -1) {$f''(x_0) > 0$};				
\node at (-0.7, 4.5) {$y = f(x)$};
 

\end{tikzpicture}

\end{center}
%align=flush ce
