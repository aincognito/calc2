\documentclass{ximera}

%% You can put user macros here
%% However, you cannot make new environments



\newcommand{\ffrac}[2]{\frac{\text{\footnotesize $#1$}}{\text{\footnotesize $#2$}}}
\newcommand{\vasymptote}[2][]{
    \draw [densely dashed,#1] ({rel axis cs:0,0} -| {axis cs:#2,0}) -- ({rel axis cs:0,1} -| {axis cs:#2,0});
}


\graphicspath{{./}{firstExample/}}

\usepackage{amsmath}
\usepackage{amssymb}
\usepackage{array}
\usepackage[makeroom]{cancel} %% for strike outs
\usepackage{pgffor} %% required for integral for loops
\usepackage{tikz}
\usepackage{tikz-cd}
\usepackage{tkz-euclide}
\usetikzlibrary{shapes.multipart}


\usetkzobj{all}
\tikzstyle geometryDiagrams=[ultra thick,color=blue!50!black]


\usetikzlibrary{arrows}
\tikzset{>=stealth,commutative diagrams/.cd,
  arrow style=tikz,diagrams={>=stealth}} %% cool arrow head
\tikzset{shorten <>/.style={ shorten >=#1, shorten <=#1 } } %% allows shorter vectors

\usetikzlibrary{backgrounds} %% for boxes around graphs
\usetikzlibrary{shapes,positioning}  %% Clouds and stars
\usetikzlibrary{matrix} %% for matrix
\usepgfplotslibrary{polar} %% for polar plots
\usepgfplotslibrary{fillbetween} %% to shade area between curves in TikZ



%\usepackage[width=4.375in, height=7.0in, top=1.0in, papersize={5.5in,8.5in}]{geometry}
%\usepackage[pdftex]{graphicx}
%\usepackage{tipa}
%\usepackage{txfonts}
%\usepackage{textcomp}
%\usepackage{amsthm}
%\usepackage{xy}
%\usepackage{fancyhdr}
%\usepackage{xcolor}
%\usepackage{mathtools} %% for pretty underbrace % Breaks Ximera
%\usepackage{multicol}



\newcommand{\RR}{\mathbb R}
\newcommand{\R}{\mathbb R}
\newcommand{\C}{\mathbb C}
\newcommand{\N}{\mathbb N}
\newcommand{\Z}{\mathbb Z}
\newcommand{\dis}{\displaystyle}
%\renewcommand{\d}{\,d\!}
\renewcommand{\d}{\mathop{}\!d}
\newcommand{\dd}[2][]{\frac{\d #1}{\d #2}}
\newcommand{\pp}[2][]{\frac{\partial #1}{\partial #2}}
\renewcommand{\l}{\ell}
\newcommand{\ddx}{\frac{d}{\d x}}

\newcommand{\zeroOverZero}{\ensuremath{\boldsymbol{\tfrac{0}{0}}}}
\newcommand{\inftyOverInfty}{\ensuremath{\boldsymbol{\tfrac{\infty}{\infty}}}}
\newcommand{\zeroOverInfty}{\ensuremath{\boldsymbol{\tfrac{0}{\infty}}}}
\newcommand{\zeroTimesInfty}{\ensuremath{\small\boldsymbol{0\cdot \infty}}}
\newcommand{\inftyMinusInfty}{\ensuremath{\small\boldsymbol{\infty - \infty}}}
\newcommand{\oneToInfty}{\ensuremath{\boldsymbol{1^\infty}}}
\newcommand{\zeroToZero}{\ensuremath{\boldsymbol{0^0}}}
\newcommand{\inftyToZero}{\ensuremath{\boldsymbol{\infty^0}}}


\newcommand{\numOverZero}{\ensuremath{\boldsymbol{\tfrac{\#}{0}}}}
\newcommand{\dfn}{\textbf}
%\newcommand{\unit}{\,\mathrm}
\newcommand{\unit}{\mathop{}\!\mathrm}
%\newcommand{\eval}[1]{\bigg[ #1 \bigg]}
\newcommand{\eval}[1]{ #1 \bigg|}
\newcommand{\seq}[1]{\left( #1 \right)}
\renewcommand{\epsilon}{\varepsilon}
\renewcommand{\iff}{\Leftrightarrow}

\DeclareMathOperator{\arccot}{arccot}
\DeclareMathOperator{\arcsec}{arcsec}
\DeclareMathOperator{\arccsc}{arccsc}
\DeclareMathOperator{\si}{Si}
\DeclareMathOperator{\proj}{proj}
\DeclareMathOperator{\scal}{scal}
\DeclareMathOperator{\cis}{cis}
\DeclareMathOperator{\Arg}{Arg}
%\DeclareMathOperator{\arg}{arg}
\DeclareMathOperator{\Rep}{Re}
\DeclareMathOperator{\Imp}{Im}
\DeclareMathOperator{\sech}{sech}
\DeclareMathOperator{\csch}{csch}
\DeclareMathOperator{\Log}{Log}

\newcommand{\tightoverset}[2]{% for arrow vec
  \mathop{#2}\limits^{\vbox to -.5ex{\kern-0.75ex\hbox{$#1$}\vss}}}
\newcommand{\arrowvec}{\overrightarrow}
\renewcommand{\vec}{\mathbf}
\newcommand{\veci}{{\boldsymbol{\hat{\imath}}}}
\newcommand{\vecj}{{\boldsymbol{\hat{\jmath}}}}
\newcommand{\veck}{{\boldsymbol{\hat{k}}}}
\newcommand{\vecl}{\boldsymbol{\l}}
\newcommand{\utan}{\vec{\hat{t}}}
\newcommand{\unormal}{\vec{\hat{n}}}
\newcommand{\ubinormal}{\vec{\hat{b}}}

\newcommand{\dotp}{\bullet}
\newcommand{\cross}{\boldsymbol\times}
\newcommand{\grad}{\boldsymbol\nabla}
\newcommand{\divergence}{\grad\dotp}
\newcommand{\curl}{\grad\cross}
%% Simple horiz vectors
\renewcommand{\vector}[1]{\left\langle #1\right\rangle}


\outcome{Find the volume of a solid of revolution}

\title{1.3 Solids of Revolution}

\begin{document}

\begin{abstract}
We use disks, washers and shells to find the volume of a solid of revolution.
\end{abstract}

\maketitle

A special type of solid whose cross-sections are familiar geometric shapes is the solid of revolution. 
We will use the idea of the last section, namely that volume can be found by integrating cross-sectional area, 
to find the volume of such a solid.  

\begin{definition}[Solid of revolution] A \textbf{solid of revolution} is a solid obtained by revolving a region in the
 $xy-$ plane about a given line.  The line is referred to as the \textbf{axis of revolution}.

\end{definition}
Cylinders, cones and spheres are examples of familiar solids which are solids of revolution. 
For simplicity, we will only consider horizontal and vertical lines for our axis of revolution.  
Furthermore, we will assume that our region in the plane is bounded by functions of $x$.  
In the event that we have a region bounded by functions of $y$, we can interchange the 
variables $x$ and $y$ throughout the problem to return to the more familiar stting of functions of $x$.

The main formulas we will need from geometry are the area of a disk, area of a washer and 
the lateral surface area of a cylinder:

\begin{image}
\begin{tikzpicture}
%\fill[gray,opacity=0.5] (0,0) circle (1.5cm);


\filldraw[fill=gray!10]
  (0,0) circle (1.5cm) node[below, yshift=-1.5cm]{$A = \pi r^2$};
  
\draw (0,0) -- (1.05, 1.05) node[midway,left]{$r$};
\filldraw[fill=black] (0,0) circle (0.03cm);

  
\filldraw[fill=gray!10]
  (4,0) circle (1.5cm) node[below, yshift=-1.5cm]{$A = \pi (R^2 - r^2)$};
\filldraw[fill=white]
  (4,0) circle (.75cm); 
\draw (4,0) -- (5.05, 1.05) node[left, yshift=-.2cm, xshift=-.2cm]{$R$};
\draw (4,0) -- (3.55, 0.6) node[midway,left]{$r$};
\filldraw[fill=black] (4,0) circle (0.03cm);
     
%\fill[gray,opacity=0.5] (4,0) circle (.75cm) -- (4,0) circle (1.5cm);
%%\draw[name=AA] (4,0) circle (.75cm);
%\draw[name=BB] (4,0) circle (1.5cm);
%\fillbetween[AA:BB,gray,opacity=0.5];

\draw (8,1.2) ellipse (1.25 and 0.3);
\draw (6.75,1.2) -- (6.75,-1.2);
\draw (6.75,-1.2) arc (180:360:1.25 and 0.3) node[below, xshift=-1.2cm, yshift = -.35cm]{$A = 2\pi rh$};
\draw [dashed] (6.75,-1.2) arc (180:360:1.25 and -0.3) ;
\draw (9.25,-1.2) -- (9.25,1.2) node[right,midway]{$h$};  
\fill [gray!30, opacity=0.5] (6.75,1.2) -- (6.75,-1.2) arc (180:360:1.25 and 0.3) -- (9.25,1.2) arc (0:180:1.25 and -0.3);
\filldraw[fill=black] (8,1.2) circle (0.03cm);
\draw (8,1.2) -- (9.25, 1.2) node[midway,above, xshift=-.1cm, yshift=-.1cm]{$r$};

\end{tikzpicture}
\end{image}
 
To see how these particular shapes come into play, imagine revolving a vertical segment about the axis of revolution. 
If the axis of revolution is horizontal and the vertical segment touches the axis at one endpoint, 
this will produce a disk (see the figure below).

\begin{image}
\begin{tikzpicture}
\draw[ <->, thick] (0,0) -- (3.9,0) ;

\node at (3, -1) {The radius of the disk is equal to the length of the vertical segment.};


\draw[thick, brown] (2,0) -- (2,2) ;



\filldraw[brown, thick] (6,1) circle(1);
\filldraw (6,1) circle(0.04);
\filldraw (2,0) circle(0.04);
%\filldraw[brown] (2,2) circle(0.04);


%\filldraw[brown] (1.7,3.4) circle(0.04);
%\filldraw (1.7,0) circle(0.04);


%\filldraw[brown] (7.5,3) circle(0.04) node at (6.65, 3.2) {$r$};




\draw [thin, ->] (0.6,0.4) to[out=180,in = 180] (.6,-.3)  to[out=0,in=-30] (0.7,.2);

\end{tikzpicture}
\end{image}



If the vertical segment does not touch the horizontal axis of revolution, then revolving it will create a washer.

\begin{image}
\begin{tikzpicture}
\draw[ <->, thick] (0,0) -- (3.9,0) ;
\draw[thick, brown] (2, 1) -- (2, 2.5);
\node at (3, -1) {\begin{tabular}{c} The radii depend on the distances from the endpoints \\ of the vertical 
segment to the axis of revolution.\end{tabular}};


  
\filldraw[brown]  (6,1) circle (1) ;
\filldraw[white]  (6,1) circle (.5cm); 

\filldraw[fill=black] (6,1) circle (0.03cm);
\filldraw[fill=black] (2,0) circle (0.03cm);
\filldraw[brown] (2,1) circle (0.03cm);
\filldraw[brown] (2,2.5) circle (0.03cm);

\draw [thin, ->] (0.6,0.4) to[out=180,in = 180] (.6,-.3)  to[out=0,in=-30] (0.7,.2);

\end{tikzpicture}
\end{image}

Finally, if the axis of revolution is vertical, then revolving the vertical segment about the axis of 
revolution will produce the lateral surface of a cylinder (see the figure below).


\begin{image}
\begin{tikzpicture}
\draw[thick, <->] (0,2.5) -- (0,-.5) ;



\draw[thick, brown] (2, 0) -- (2,2);
%\draw[scale = 2, thick, gray] (1, 0) -- (1,1);
%\draw[scale = 2, thin] (0.6, 0.05) -- (0.6, -0.05) node[below] {$x$};
%\draw[scale = 2, thin] (1, 0.05) -- (1, -0.05) node[below] {$1$};

\draw[brown] (6,2) ellipse (1.25 and 0.3);
%\draw (6.75,1.2) -- (6.75,-1.2);
\draw[brown] (4.75,0) arc (180:360:1.25 and 0.3);
\draw [brown, dashed] (4.75,0) arc (180:360:1.25 and -0.3) ;
%\draw (9.25,-1.2) -- (9.25,1.2) node[right,midway]{$h$};  
\fill [brown, opacity=0.8] (4.75,2) -- (4.75,0) arc (180:360:1.25 and 0.3) -- (7.25,2) arc (0:180:1.25 and -0.3);
%\filldraw[fill=black] (6,1.2) circle (0.03cm);
%\draw (8,1.2) -- (9.25, 1.2) node[midway,above, xshift=-.1cm, yshift=-.1cm]{$r$};




\draw [thin, ->] (0.35,0.27) to [out=250,in=-90] (-0.35,0.3) to[out=90,in=150] (0.3,0.4);
%\draw[thin, <->] (0.1,0.3) arc (330:15:0.1);

\draw[thin, <->] (6,2.75) -- (6,-.75) ;

\node at (3, -1.5) {\begin{tabular}{c} The height of the shell is length of the vertical segment and \\
the radius is the distance between the segment and the axis.\end{tabular}};


\end{tikzpicture}
\end{image}



\section{Disks}

%Consider a vertical line segment 2 units long that touches the $x$-axis at one end.  

%(SEE FIGURE)


%If we revolve this segment about the axis, it will make a circle of radius 2 whose area is $4\pi$.
%As in the previous section, we will associate a thickness, $dx$, to the infinitely narrow segment.
%Then after revolving the segment, we get a \textbf{disk} whose volume is 
%\[
%V = \pi r^2 h = \pi \cdot 2^2 \cdot dx = 4\pi dx.
%\]

%We now consider an example where we apply this idea to find the volume of a solid of revolution.


\begin{example}[example 1]
Find the volume of a right, circular cone with height 4, and base radius 2.\\
This cone can be obtained by revolving the the region between the line $y=x/2$ and the $x$-axis from $x = 0$ to $x = 4$
about the $x$-axis.\\
To begin, we sketch the region (see below) and we draw a vertical line segment in the region at $x$. 
Revolving this segment about the axis of revolution (the $x$-axis) creates a disk.  The area of this disk is
\[
A = \pi r^2,
\]
where the radius, $r$, is given by
\[
r = \frac{x}{2} - 0 = \frac{x}{2}.
\]
We find the volume by integrating area:
\[
V = \int_a^b \text{(area)} \; dx = \int_0^4 \pi r^2 \, dx = \int_0^4 \pi \left(\frac{x}{2}\right)^2 \, dx.
\]
Computing this integral gives
\[
V = \frac{\pi}{4} \int_0^4 x^2 \, dx = \frac{\pi}{4} \cdot \frac{x^3}{3} \bigg |_0^4 = \frac{\pi}{4}\cdot \frac{64}{3} = \frac{16\pi}{3}.
\]


\begin{image}
\begin{tikzpicture}
\draw[ <->, thick] (0,2.5) -- (0,0) -- (4.5,0);

\draw[thick, domain=0:4,smooth,variable=\x,brown] plot ({\x},{0.5*\x}) node at (2.5, 2) {$y = x/2$};

\draw[thick, brown] (4,0) -- (4,2) node at (4.5,1) {$x = 4$};
\draw[thick, teal] (8,1) -- (9,1);
\draw[thick, teal] (2.5,0) -- (2.5,1.25);
\draw[thin] (2.5, 0.05) -- (2.5, -0.05) node[below] {$x$};
\draw[thin] (4, 0.05) -- (4, -0.05) node[below] {$4$};

\draw[brown, thick] (8,1) circle(1);
\filldraw (8,1) circle(0.04);

\filldraw[brown] (2.5,1.25) circle(0.04);
\filldraw (2.5,0) circle(0.04);
%\draw[thin, brown] (4,3) -- (4.5, 3.1);
%\draw[thin, teal] (4,3) -- (3.5, 3.85);

\filldraw[brown] (9,1) circle(0.04) node[teal] at (8.65, 1.2) {$r$};

\node[teal] at (8, -0.5) {$r = x/2$};

%\draw [thin, ->] (0.5,0.25) to[out=225,in=-90] (0.35,0) to[out=-90,in=0] (0.5,-0.25) to[out=45,in=150] (0.65,0.15);
\draw [thin, ->] (1,0.3) to[out=180,in = 180] (1.1,-.3)  to[out=0,in=-30] (1.2,.3);
\end{tikzpicture}
\end{image}

\end{example}






\begin{problem}(problem 1a)
Find the formula for the volume of a right, circular cone with height $5$, and base radius $3$.\\

To create the cone, consider revolving a line through the origin about the $x$-axis.\\
The equation of this line is $y = \answer{\frac{3}{5} x}$.\\
We will integrate over the interval $\left(\answer{0},\answer{5}\right)$

The definite integral that gives the volume of the cone is:\\
\begin{multipleChoice}
\choice{$\displaystyle{V = \int_0^5 \pi \cdot \frac35 x^2\; dx}$}
\choice{$\displaystyle{V = \int_0^5 2\pi \left(\frac35 x\right)^2\; dx}$}
\choice[correct]{$\displaystyle{V = \int_0^5 \pi \left(\frac35 x\right)^2\; dx}$}
\end{multipleChoice}

The volume of the cone is $V = \answer{15 \pi }$.










\end{problem}



\begin{problem}(problem 1b)
Find the formula for the volume of a right, circular cone with height $h$, and base radius $r$.\\

To create the cone, consider revolving a line through the origin about the $x$-axis.\\
The equation of this line is $y = \answer{\frac{r}{h} x}$.\\
We will integrate over the interval $\left(\answer{0},\answer{h}\right)$

The definite integral that gives the volume of the cone is:\\
\begin{multipleChoice}
\choice[correct]{$\displaystyle{V = \int_0^h \pi \left(\frac{r}{h}x\right)^2\; dx}$}
\choice{$\displaystyle{V = \int_0^h 2\pi \left(\frac{r}{h}x\right)^2\; dx}$}
\choice{$\displaystyle{V = \int_0^h \pi \left(\frac{r}{h}\right)x^2\; dx}$}
\end{multipleChoice}

The formula for the volume of cone is $V = \answer{\pi r^2 h/3}$.

\end{problem}





\begin{example}[example 2]
Find the volume of a sphere of radius 2.\\
We can think of this sphere as the solid of revolution obtained by revolving the region between 
the semi-circle $y = \sqrt{4-x^2}$ and the $x$-axis about the $x$-axis.
To begin, we sketch the region (see below) and we draw a vertical line segment in the region at $x$. 
Revolving this segment about the axis of revolution (the $x$-axis) creates a disk.  The area of this disk is
\[
A = \pi r^2,
\]
where the radius, $r$, is given by
\[
r = \sqrt{4 - x^2} - 0 = \sqrt{4-x^2}.
\]
We find the volume by integrating area:
\[
V = \int_a^b \text{(area)} \; dx = \int_{-2}^2 \pi r^2 \, dx = \int_{-2}^2 \pi \left(\sqrt{4-x^2}\right)^2 \, dx.
\]
Computing this integral gives
\[
V = \pi \int_{-2}^2 (4 - x^2) \, dx = \pi \left(4x - \frac{x^3}{3}\right) \bigg |_{-2}^2 
\]
\[
= \pi \left[\left(8 - \frac83 \right) - \left(-8 + \frac83\right)\right] = \pi \left(16 - \frac{16}{3}\right)
= \frac{32 \pi}{3}
\]


\begin{image}
\begin{tikzpicture}
\draw[ <->, thick] (-2.5,0) -- (2.5,0);
\draw[->, thick] (0,0) -- (0, 2.5);

%\draw[thick, domain=-2:2,smooth,variable=\x,brown] plot ({\x},{sqrt{4 - \x*\x}}) ;
\draw[thick, brown] (2,0) arc (0:180:2) node at (1.8, 2.3) {$y = \sqrt{4 -x^2}$};

\draw[thick, teal] (5,1) -- (6,1);
\draw[thick, teal] (1,0) -- (1,1.732);
\draw[thin] (1, 0.05) -- (1, -0.05) node[below] {$x$};
\draw[thin] (2, 0.05) -- (2, -0.05) node[below] {$2$};
\draw[thin] (-2, 0.05) -- (-2, -0.05) node[below] {$-2$};

\draw[brown, thick] (5,1) circle(1);
\filldraw (5,1) circle(0.04);

\filldraw[brown] (1,1.732) circle(0.04);
\filldraw (2.5,0) circle(0.04);
%\draw[thin, brown] (4,3) -- (4.5, 3.1);
%\draw[thin, teal] (4,3) -- (3.5, 3.85);

\filldraw[brown] (6,1) circle(0.04) node[teal] at (5.65, 1.2) {$r$};

\node[teal] at (5, -0.5) {$r = \sqrt{4 -x^2}$};

%\draw [thin, ->] (0.5,0.25) to[out=225,in=-90] (0.35,0) to[out=-90,in=0] (0.5,-0.25) to[out=45,in=150] (0.65,0.15);
\draw [thin, ->] (-1,0.3) to[out=180,in = 180] (-0.9,-.3)  to[out=0,in=-30] (-0.8,.3);
\end{tikzpicture}
\end{image}




\end{example}


\begin{problem}(problem 2a)
Find the formula for the volume of a sphere of radius $3$.\\

To create the sphere, consider revolving a semi-circlular region centered at the origin about the $x$-axis.\\
The equation of this semi-circle is $y = \answer{\sqrt{9-x^2}}$.\\
We will integrate over the interval $\left(\answer{-3},\answer{3}\right)$

The definite integral that gives the volume of the sphere is:\\
\begin{multipleChoice}
\choice{$\displaystyle{V = \int_{-3}^3 \pi \sqrt{9 - x^2} \; dx}$}
\choice[correct]{$\displaystyle{V = \int_{-3}^3 \pi \left(\sqrt{9-x^2}\right)^2\; dx}$}
\choice{$\displaystyle{V = \int_{-3}^3  2\pi \left(\sqrt{9-x^2}\right)^2\; dx}$}
\end{multipleChoice}

The volume of the sphere is $V = \answer{36 \pi }$.


\end{problem}



\begin{problem}(problem 2b)
Find the formula for the volume of a sphere of radius $r$.\\

To create the sphere, consider revolving a semi-circlular region centered at the origin about the $x$-axis.\\
The equation of this semi-circle is $y = \answer{\sqrt{r^2-x^2}}$.\\
We will integrate over the interval $\left(\answer{-r},\answer{r}\right)$

The definite integral that gives the volume of the sphere is:\\
\begin{multipleChoice}
\choice[correct]{$\displaystyle{V = \int_{-r}^r \pi \left(\sqrt{r^2-x^2}\right)^2\; dx}$}
\choice{$\displaystyle{V = \int_{-r}^r \pi \sqrt{r^2 - x^2} \; dx}$}
\choice{$\displaystyle{V = \int_{-r}^r  2\pi \left(\sqrt{r^2-x^2}\right)^2\; dx}$}
\end{multipleChoice}

The formula for the volume of a sphere with radius $r$ is $V = \answer{4/3 \pi r^3 }$.
\end{problem}





\begin{example}[example 3] Find the volume of the solid obtained by revolving the region bounded by the graphs of $y=x^2, y=0, x=0$, and $x = 1$
about the $x$-axis.\\
We begin with a sketch of the region.

%(INSERT FIGURE HERE)

To compute the volume of the resulting solid, fix a value of $x$ between 0 and 1 and consider a vertical segment in the region at this $x$-value.

%(INSERT FIGURE HERE)

Revolving this individual segment about the $x$-axis yields a disk with radius $r = x^2$. 

%(INSERT FIGURE HERE)

The area of this disk is then
\[
V_x = \pi r^2  = \pi (x^2)^2  = \pi x^4.
\]
The volume of the solid can be otained by integrating $A_x$ as $x$ ranges from 0 to 1:
\[
V = \int_0^1 A_x \; dx  = \int_0^1 \pi x^4 \; dx = \pi \frac{x^5}{5}\bigg|_0^1 = \frac{\pi}{5}.
\]

%(INSERT GEOGEBRA APPLET HERE)

\end{example}


\begin{example}[example 4] Find the volume of the solid obtained by revolving the region bounded by the graphs of $y=\sqrt x, y=0, x=0$, and $x = 1$
about the $x$-axis.\\
We begin with a sketch of the region.

%(INSERT FIGURE HERE)

To compute the volume of the resulting solid, fix a value of $x$ between 0 and 1 and consider a vertical segment in the region at this $x$-value.

%(INSERT FIGURE HERE)

Revolving this individual segment about the $x$-axis yields a disk with radius $r = x^2$. 

%(INSERT FIGURE HERE)

The area of this disk is then
\[
A_x = \pi r^2  = \pi (\sqrt x)^2  = \pi x.
\]
The volume of the solid can be otained by integrating $A_x$ as $x$ ranges from 0 to 1:
\[
V = \int_0^1 A_x \; dx  = \int_0^1 \pi x \; dx = \pi \frac{x^2}{2}\bigg|_0^1 = \frac{\pi}{2}.
\]


%(INSERT GEOGEBRA APPLET HERE)

\end{example}



\section{Washers} In the next series of examples, the region will not border the axis of revolution. 
In this situation, the resulting cross-sections are washers rather than disks.
The area of a washer with inner radius, $r$, and outer radius, $R$, is given by 
\[
A = \pi(R^2 - r^2).
\]

%(INSERT FIGURE HERE)

We now use this to calculuate volumes of solids of revolutions about the $x$-axis in which the region does not border the axis.

\begin{example}[example 5] Find the volume of the solid obtained by revolving the region bounded by the graphs of 
$y = e^x, y = 3$, and $x = 0$ about the $x$-axis.\\
First, note that the curves $y = e^x$ and $y = 3$ intersect at $x = \ln(3)$.


Thus, to compute the volume of the resulting solid, fix a value of $x$ between $0$ and $\ln(3)$ and consider a vertical segment in the region at this $x$-value.


Revolving this individual segment about the $x$-axis yields a washer with inner radius, $r = e^x$, outer radius, $R = 3$. 


\begin{image}
\begin{tikzpicture}
\draw[scale= 2, <->] (0,3.5) -- (0,0) -- (1.3,0);
%\draw ‎[‎smooth,‎samples=100‎,domain=0:‎‎2‎‎] ‎plot (\x,‎‎‎‎‎‎‎‎\x);‎‎
\draw[scale = 2, thick, domain=0:1.1,smooth,variable=\x,brown] plot ({\x},{e^\x}) node at (0.7, 1.4) {$y = e^x$};
\draw[thick,teal]  (0, 6) -- (2.2,6) node at (-0.5, 6) {$y = 3$};
\draw[scale = 2, thick] (0.5, 1.649) -- (0.5,3);
\draw[scale = 2, thin] (0.5, 0.05) -- (0.5, -0.05) node[below] {$x$};
\draw[scale = 2, thin] (1.1, 0.05) -- (1.1, -0.05) node[below] {$\ln(3)$};
\draw[teal, thick] (4,3) circle(1);
\draw[brown, thick] (4,3) circle(0.5);
\filldraw (4,3) circle(0.04);
\filldraw[teal] (1,6) circle(0.04);
\filldraw[brown] (1,3.3) circle(0.04);
\filldraw (1,0) circle(0.04);
\draw[thin, brown] (4,3) -- (4.5, 3.1);
\draw[thin, teal] (4,3) -- (3.5, 3.85);
\filldraw[teal] (3.5,3.86) circle(0.04) node at (3.3, 4.1) {$R$};
\filldraw[brown] (4.48,3.1) circle(0.04) node at (4.65, 3.1) {$r$};
\node[teal] at (6, 3.3) {$R = 3$};
\node[brown] at (6, 2.8) {$r =  e^x$};


\end{tikzpicture}
\end{image}


The area of this washer is then
\[
A_x = \pi (R^2 - r^2)  = \pi \left[3^2 - \left(e^x\right)^2\right]  = \pi \left(9 - e^{2x}\right) .
\]
The volume of the solid can be otained by integrating $A_x$ as $x$ ranges from $0$ to $\ln 3$:
\[
V = \int_0^{\ln 3} A_x \; dx  = \int_0^{\ln 3} \pi \left(9 - e^{2x}\right) \; dx 
= \pi \left(9x - \frac{e^{2x}}{2}\right)\bigg|_0^{\ln 3} = \left(9\ln(3)-4\right)\pi.
\]



\end{example}


\begin{example}[example 6] Find the volume of the solid obtained by revolving the region bounded by the graphs of 
$y = \sec(x), y =  2, x = 0$, and $x = \frac{\pi}{4}$ about the $x$-axis.\\
We begin with a sketch of the region.



To compute the volume of the resulting solid, fix a value of $x$ between 0 and $\frac{\pi}{4}$ and consider a vertical segment in the region at this $x$-value.



Revolving this individual segment about the $x$-axis yields a washer with inner radius, $r = \sec(x)$ and outer radius, $R = 2$. 


\begin{image}
\begin{tikzpicture}
\draw[scale= 2, <->] (0,2.5) -- (0,0) -- (1,0);
\draw[scale = 2, thick, domain=0:pi/4,smooth,variable=\x,brown] plot ({\x},{sec(deg(\x))}) node at (0.7, .9) {$y = \sec x$};
\draw[thick,teal]  (0, 4) -- (pi/2,4) node at (-0.5, 4) {$y = 2$};
\draw[scale = 2, thick] (0.4, 1.08) -- (0.4,2);
\draw[scale = 2, thick, gray] (pi/4, sqrt 2) -- (pi/4, 2);
\draw[scale = 2, thin] (0.4, 0.05) -- (0.4, -0.05) node[below] {$x$};
\draw[scale = 2, thin] (pi/4, 0.05) -- (pi/4, -0.05) node[below] {$\frac{\pi}{4}$};
\draw[teal, thick] (4,2.5) circle(1);
\draw[brown, thick] (4,2.5) circle(0.5);
\filldraw (4,2.5) circle(0.04);
\filldraw[teal] (0.8,4) circle(0.04);
\filldraw[brown] (0.8,2.16) circle(0.04);
\filldraw (0.8,0) circle(0.04);
\draw[thin, brown] (4,2.5) -- (4.5, 2.6);
\draw[thin, teal] (4,2.5) -- (3.5, 3.35);
\filldraw[teal] (3.5,3.36) circle(0.04) node at (3.3, 3.5) {$R$};
\filldraw[brown] (4.48,2.6) circle(0.04) node at (4.65, 2.6) {$r$};
\node[teal] at (6, 2.8) {$R = 2$};
\node[brown] at (6, 2.3) {$r =  \sec x$};


\end{tikzpicture}
\end{image}


The area of this washer is then
\[
A_x = \pi (R^2 - r^2)  = \pi \left[2^2 - \sec^2(x)\right]  = \pi \left[4 - \sec^2(x)\right] .
\]
The volume of the solid can be otained by integrating $A_x$ as $x$ ranges from 0 to $\pi/4$:
\[
V = \int_0^{\pi/4} A_x \; dx  = \int_0^{\pi/4} \pi \left[4-\sec^2(x)\right] \; dx = \pi \left[4x - \tan(x)\right]\bigg|_0^{\pi/4} = \pi^2 - \pi.
\]



\end{example}



\section{Cylindrical shells}


We now find the volume of solids of revolution obtained by revolving a region about a vertical axis.

\begin{example}[example 7] Find the volume of the solid obtained by revolving the region bounded by the graphs of $y = x^2, y=0$, 
and $x = 1$ about the $y$-axis.\\
We begin with a sketch of the region.


To compute the volume of the resulting solid, fix a value of $x$ between 0 and 1 and consider a vertical segment in the region at this $x$-value.


Revolving this individual segment about the $y$-axis yields a cylindrical shell with radius, $r = x,$ height, $h = x^2$, and thickness, $dx$. 

\begin{image}
\begin{tikzpicture}
\draw[scale= 2, <->] (0,1.3) -- (0,0) -- (1.3,0);
\filldraw (0,0) circle(0.04);
\filldraw (0,0.72) circle(0.04);

\draw[scale = 2, thick, domain=0:1,smooth,variable=\x,teal] plot ({\x},{\x*\x}) node at (0.6, 0.9) {$y = x^2$};
%\draw[thick,teal]  (0, 4) -- (pi/2,4) node at (-0.5, 4) {$y = 2$};
\draw[scale = 2, thick, brown] (0.6, 0) -- (0.6,0.36);
\draw[scale = 2, thick, gray] (1, 0) -- (1,1);
\draw[scale = 2, thin] (0.6, 0.05) -- (0.6, -0.05) node[below] {$x$};
\draw[scale = 2, thin] (1, 0.05) -- (1, -0.05) node[below] {$1$};
\draw[teal] (4,0.72) ellipse (0.7 and 0.2);
\draw[black] (4,0) ellipse (0.7 and 0.2);
\draw[thin, ->] (4, 0) -- (4, 1.2);
\filldraw (4,0) circle(0.04);
\filldraw (4,0.72) circle(0.04);
\draw[thick, brown] (3.3,0) -- (3.3, 0.72);
\draw[thick, brown] (4.7,0) -- (4.7, 0.72) node at (4.9,0.45) {$h$};
\filldraw[teal] (1.2,0.72) circle(0.04);
\draw[thin, teal] (4,0.72) -- (4.5, 0.85);
\filldraw[teal] (4.48,0.85) circle(0.04) node at (4.65, 1) {$r$};
\node[teal] at (6, 0.75) {$r = x$};
\node[brown] at (6, 0.4) {$h =  x^2$};

\draw [thin, ->] (0.15,0.27) to[out=250,in=-90] (-0.15,0.3) to[out=90,in=150] (0.15,0.35);
%\draw[thin, <->] (0.1,0.3) arc (330:15:0.1);
\end{tikzpicture}
\end{image}


The surface area of this shell is:
\[
A_x = 2\pi rh  = 2\pi x(x^2)  = 2\pi x^3 .
\]
The volume of the solid can be otained by integrating $V_x$ as $x$ ranges from 0 to 1:
\[
V = \int_0^1 A_x \; dx = \int_0^1 2\pi x^3 \; dx = 2\pi \frac{x^4}{4}\bigg|_0^1 = \frac{\pi}{2}.
\]



\end{example}


\begin{example}[example 8] Find the volume of the solid obtained by revolving the region bounded by the graphs of $y = \sqrt x, y=2, x=0$, and $x = 1$ about the $y$-axis.\\
We begin with a sketch of the region.



To compute the volume of the resulting solid, fix a value of $x$ between 0 and 1 and consider a vertical segment in the region at this $x$-value.



Revolving this individual segment about the $y$-axis yields a cylindrical shell with 
radius, $r = x$ and height, $h = 2 - \sqrt x$. 

\begin{image}
\begin{tikzpicture}
\draw[scale= 2, <->] (0,2.2) -- (0,0) -- (1.3,0);
\filldraw (0,1.26) circle(0.04);
\filldraw (0,4) circle(0.04);

\draw[scale = 2, thick, domain=0:1,smooth,variable=\x,teal] plot ({\x*\x},\x) node at (0.75, 0.6) {$y = \sqrt x$};
\draw[thick,teal]  (0, 4) -- (2,4) node at (-0.5, 4) {$y = 2$};
\filldraw[teal] (0.8,4) circle(0.04);
\filldraw[teal] (0.8,1.26) circle(0.04);
\draw[scale = 2, thick, brown] (0.4, .63) -- (0.4,2);
\draw[scale = 2, thick, gray] (1, 1) -- (1,2);
\draw[scale = 2, thin] (0.4, 0.05) -- (0.4, -0.05) node[below] {$x$};
\draw[scale = 2, thin] (1, 0.05) -- (1, -0.05) node[below] {$1$};

\draw[teal] (4,4) ellipse (0.7 and 0.2);
\draw[teal] (4,1.26) ellipse (0.7 and 0.2);
\draw[thin, ->] (4, 1.26) -- (4, 4.6);

\filldraw (4,1.26) circle(0.04);
\filldraw (4,4) circle(0.04);

\draw[thick, brown] (3.3,1.26) -- (3.3, 4);
\draw[thick, brown] (4.7,1.26) -- (4.7, 4) node at (4.9,2.75) {$h$};

\draw[thin, teal] (4,4) -- (4.5, 4.15);
\filldraw[teal] (4.48,4.15) circle(0.04) node at (4.65, 4.3) {$r$};

\node[teal] at (6, 2.75) {$r = x$};
\node[brown] at (6.4, 2.4) {$h =  2-\sqrt x$};

\draw [thin, ->] (0.15,1.94) to[out=250,in=-90] (-0.15,2) to[out=90,in=150] (0.15,2.1);
%\draw[thin, <->] (0.1,0.3) arc (330:15:0.1);
\end{tikzpicture}
\end{image}


The surface area of the shell is:
\[
A_x = 2\pi rh  = 2\pi x(2 - \sqrt x)  = 2\pi \left(2x - x^{3/2}\right) .
\]
The volume of the solid can be otained by integrating $A_x$ as $x$ ranges from 0 to 1:
\[
V = \int_0^1 A_x \; dx = \int_0^1 2\pi \left(2x - x^{3/2}\right) \; dx = 2\pi \left(x^2 - \frac25 x^{5/2}\right)\bigg|_0^1 = \frac{6\pi}{5}.
\]



\end{example}




\section{Problems}


\begin{problem}(problem 1)
Find the volume of the solid obtained by revolving the region bounded by the graphs of $y = x$ and $y = \sqrt x$ about the line $x = -1$.\\
The method we will use is:
\begin{multipleChoice}
\choice{Disks}
\choice{Washers}
\choice[correct]{Shells}
\end{multipleChoice}

The formula associated with this method is:
\begin{multipleChoice}
\choice{$\displaystyle{V = \int_a^b \pi r^2 \, dx}$}
\choice{$\displaystyle{V = \int_a^b \pi (R^2 - r^2) \, dx}$}
\choice[correct]{$\displaystyle{V = \int_a^b 2\pi rh \, dx}$}
\end{multipleChoice}

The left and right endpoints of integration are:\\
$a = \answer{0}\;\;\;$ and $\;\;\; b = \answer{1}$.\\
The radius and height as functions of $x$ are:\\
$r = \answer{x+1} \;\;\;$ and $\; h = \answer{\sqrt{x} - x}$.\\

The definite integral that represents the volume is given by:\\
\begin{multipleChoice}
\choice[correct]{$\displaystyle{V = \int_0^1 2\pi \left(-x^2 + x^{3/2} - x + x^{1/2}\right)\; dx}$}
\choice{$\displaystyle{V = \int_0^1 2\pi \left(x^2 + x^{3/2} - x - x^{1/2}\right)\; dx}$}
\choice{$\displaystyle{V = \int_0^1 2\pi \left(-x^2 + x^{3/2} + x - x^{1/2}\right)\; dx}$}
\end{multipleChoice}

The volume of the solid is $V = \answer{7\pi/15}$.

\end{problem}





\begin{problem}(problem 2)
Find the volume of the solid obtained by revolving the region bounded by the graphs of $y = x$ and $y = \sqrt x$ about the line $y = 1$.\\
The method we will use is:
\begin{multipleChoice}
\choice{Disks}
\choice[correct]{Washers}
\choice{Shells}
\end{multipleChoice}

The formula associated with this method is:
\begin{multipleChoice}
\choice{$\displaystyle{V = \int_a^b \pi r^2 \, dx}$}
\choice[correct]{$\displaystyle{V = \int_a^b \pi (R^2 - r^2) \, dx}$}
\choice{$\displaystyle{V = \int_a^b 2\pi rh \, dx}$}
\end{multipleChoice}

The left and right endpoints of integration are:\\
$a = \answer{0}\;\;\;$ and $\;\;\; b = \answer{1}$.\\
The outer radius, $R$, and inner radius, $r$, as functions of $x$ are:\\
$R = \answer{1-x} \;\;\;$ and $\; r = \answer{1-\sqrt{x}}$.\\

The definite integral that represents the volume is given by:\\
\begin{multipleChoice}
\choice{$\displaystyle{V = \int_0^1 \pi \left(x^2 - x + 2\sqrt x\right)\; dx}$}
\choice[correct]{$\displaystyle{V = \int_0^1 \pi \left(x^2 - 3x + 2\sqrt x \right)\; dx}$}
\choice{$\displaystyle{V = \int_0^1 \pi \left(x^2 -3x - 2\sqrt x\right)\; dx}$}
\end{multipleChoice}

The volume of the solid is $V = \answer{\pi/6}$.

\end{problem}




\begin{problem}(problem 3)
Find the volume of the solid obtained by revolving the region bounded by the graphs of $y = 2/x, y = 0, x = 1$, and $x = 2$ about the $x$-axis.\\
The method we will use is:
\begin{multipleChoice}
\choice[correct]{Disks}
\choice{Washers}
\choice{Shells}
\end{multipleChoice}

The formula associated with this method is:
\begin{multipleChoice}
\choice[correct]{$\displaystyle{V = \int_a^b \pi r^2 \, dx}$}
\choice{$\displaystyle{V = \int_a^b \pi (R^2 - r^2) \, dx}$}
\choice{$\displaystyle{V = \int_a^b 2\pi rh \, dx}$}
\end{multipleChoice}

The left and right endpoints of integration are:\\
$a = \answer{1}\;\;\;$ and $\;\;\; b = \answer{2}$.\\
The radius, $r$, as a function of $x$ is:\\
$r = \answer{2/x}.$\\

The definite integral that represents the volume is given by:\\
\begin{multipleChoice}
\choice{$\displaystyle{V = \int_1^2 2\pi \frac{2}{x}\; dx}$}
\choice{$\displaystyle{V = \int_1^2 \pi \frac{2}{x^2}\; dx}$}
\choice[correct]{$\displaystyle{V = \int_1^2 \pi \frac{4}{x^2}\; dx}$}
\end{multipleChoice}

The volume of the solid is $V = \answer{2\pi}$.

\end{problem}




\begin{problem}(problem 4)
Find the volume of the solid obtained by revolving the region bounded by the graphs of $y = x^3, y = 0$, and $x = 1$ about the $x$-axis.\\
The method we will use is:
\begin{multipleChoice}
\choice[correct]{Disks}
\choice{Washers}
\choice{Shells}
\end{multipleChoice}

The formula associated with this method is:
\begin{multipleChoice}
\choice[correct]{$\displaystyle{V = \int_a^b \pi r^2 \, dx}$}
\choice{$\displaystyle{V = \int_a^b \pi (R^2 - r^2) \, dx}$}
\choice{$\displaystyle{V = \int_a^b 2\pi rh \, dx}$}
\end{multipleChoice}

The left and right endpoints of integration are:\\
$a = \answer{0}\;\;\;$ and $\;\;\; b = \answer{1}$.\\
The radius, $r$, as a function of $x$ is:\\
$r = \answer{x^3}.$\\

The definite integral that represents the volume is given by:\\
\begin{multipleChoice}
\choice{$\displaystyle{V = \int_0^1 \pi x^3\; dx}$}
\choice{$\displaystyle{V = \int_0^1 \pi x^5\; dx}$}
\choice[correct]{$\displaystyle{V = \int_0^1 \pi x^6\; dx}$}
\end{multipleChoice}

The volume of the solid is $V = \answer{\pi/7}$.

\end{problem}



\begin{problem}(problem 5)
Find the volume of the solid obtained by revolving the region bounded by the graphs of $y = x^2, y = 0, x = 0$ and $x = 2$ about the line $y = 4$.\\
The method we will use is:
\begin{multipleChoice}
\choice{Disks}
\choice[correct]{Washers}
\choice{Shells}
\end{multipleChoice}

The formula associated with this method is:
\begin{multipleChoice}
\choice{$\displaystyle{V = \int_a^b \pi r^2 \, dx}$}
\choice[correct]{$\displaystyle{V = \int_a^b \pi (R^2 - r^2) \, dx}$}
\choice{$\displaystyle{V = \int_a^b 2\pi rh \, dx}$}
\end{multipleChoice}

The left and right endpoints of integration are:\\
$a = \answer{0}\;\;\;$ and $\;\;\; b = \answer{2}$.\\
The outer radius, $R$, and inner radius, $r$, as functions of $x$ are:\\
$R = \answer{4} \;\;\;$ and $\; r = \answer{4-x^2}$.\\

The definite integral that represents the volume is given by:\\
\begin{multipleChoice}
\choice{$\displaystyle{V = \int_0^2 \pi \left(2x^2 - x^4\right)\; dx}$}
\choice{$\displaystyle{V = \int_0^2 \pi \left(x^4 - 8x^2 \right)\; dx}$}
\choice[correct]{$\displaystyle{V = \int_0^2 \pi \left(8x^2 - x^4\right)\; dx}$}
\end{multipleChoice}

The volume of the solid is $V = \answer{224\pi/15}$.

\end{problem}





\begin{problem}(problem 6)
Find the volume of the solid obtained by revolving the region bounded by the graphs of $y = 2e^x, y = 0, x = -1$, and $x = \ln 2$ about the $x$-axis.\\
The method we will use is:
\begin{multipleChoice}
\choice[correct]{Disks}
\choice{Washers}
\choice{Shells}
\end{multipleChoice}

The formula associated with this method is:
\begin{multipleChoice}
\choice[correct]{$\displaystyle{V = \int_a^b \pi r^2 \, dx}$}
\choice{$\displaystyle{V = \int_a^b \pi (R^2 - r^2) \, dx}$}
\choice{$\displaystyle{V = \int_a^b 2\pi rh \, dx}$}
\end{multipleChoice}

The left and right endpoints of integration are:\\
$a = \answer{-1}\;\;\;$ and $\;\;\; b = \answer{\ln 2}$.\\
The radius, $r$, as a function of $x$ is:\\
$r = \answer{2e^x}.$\\

The definite integral that represents the volume is given by:\\
\begin{multipleChoice}
\choice{$\displaystyle{V = \int_{-1}^{\ln 2} 4\pi e^x\; dx}$}
\choice{$\displaystyle{V = \int_{-1}^{\ln 2} 2\pi e^{2x}\; dx}$}
\choice[correct]{$\displaystyle{V = \int_{-1}^{\ln 2} 4\pi e^{2x}\; dx}$}
\end{multipleChoice}

The volume of the solid is $V = \answer{(8-2/{e^2})\pi}$.

\end{problem}





\begin{problem}(problem 7)
Find the volume of the solid obtained by revolving the region bounded by the graphs of $y = x^2$ and $y = x^3$ about the $y$-axis.\\
The method we will use is:
\begin{multipleChoice}
\choice{Disks}
\choice{Washers}
\choice[correct]{Shells}
\end{multipleChoice}

The formula associated with this method is:
\begin{multipleChoice}
\choice{$\displaystyle{V = \int_a^b \pi r^2 \, dx}$}
\choice{$\displaystyle{V = \int_a^b \pi (R^2 - r^2) \, dx}$}
\choice[correct]{$\displaystyle{V = \int_a^b 2\pi rh \, dx}$}
\end{multipleChoice}

The left and right endpoints of integration are:\\
$a = \answer{0}\;\;\;$ and $\;\;\; b = \answer{1}$.\\
The radius and height as functions of $x$ are:\\
$r = \answer{x} \;\;\;$ and $\; h = \answer{x^2 -x^3}$.\\

The definite integral that represents the volume is given by:\\
\begin{multipleChoice}
\choice[correct]{$\displaystyle{V = \int_0^1 2\pi \left(x^3 -x^4\right)\; dx}$}
\choice{$\displaystyle{V = \int_0^1 2\pi \left(x^4 -x^3\right)\; dx}$}
\choice{$\displaystyle{V = \int_0^1 2\pi \left(x^2 -x^3\right)\; dx}$}
\end{multipleChoice}

The volume of the solid is $V = \answer{\pi/10}$.

\end{problem}






\begin{problem}(problem 8)
Find the volume of the solid obtained by revolving the region bounded by the graphs of $y = x-1, y = 2- x$ and $x = 0$ about the line $y = -1$.\\
The method we will use is:
\begin{multipleChoice}
\choice{Disks}
\choice[correct]{Washers}
\choice{Shells}
\end{multipleChoice}

The formula associated with this method is:
\begin{multipleChoice}
\choice{$\displaystyle{V = \int_a^b \pi r^2 \, dx}$}
\choice[correct]{$\displaystyle{V = \int_a^b \pi (R^2 - r^2) \, dx}$}
\choice{$\displaystyle{V = \int_a^b 2\pi rh \, dx}$}
\end{multipleChoice}

The left and right endpoints of integration are:\\
$a = \answer{0}\;\;\;$ and $\;\;\; b = \answer{3/2}$.\\
The outer radius, $R$, and inner radius, $r$, as functions of $x$ are:\\
$R = \answer{3-x} \;\;\;$ and $\; r = \answer{x}$.\\

The definite integral that represents the volume is given by:\\
\begin{multipleChoice}
\choice{$\displaystyle{V = \int_0^{3/2} \pi \left(6x- 9\right)\; dx}$}
\choice[correct]{$\displaystyle{V = \int_0^{3/2} \pi \left(9 - 6x \right)\; dx}$}
\choice{$\displaystyle{V = \int_0^{3/2} \pi \left(x^2 -6x + 9\right)\; dx}$}
\end{multipleChoice}

The volume of the solid is $V = \answer{27\pi/4}$.

\end{problem}





\begin{problem}(problem 9)
Find the volume of the solid obtained by revolving the region bounded by the graphs of $y = 1/x, y = \sqrt[4] x$ and $x = 16$ about the line $x = 16$.\\
The method we will use is:
\begin{multipleChoice}
\choice{Disks}
\choice{Washers}
\choice[correct]{Shells}
\end{multipleChoice}

The formula associated with this method is:
\begin{multipleChoice}
\choice{$\displaystyle{V = \int_a^b \pi r^2 \, dx}$}
\choice{$\displaystyle{V = \int_a^b \pi (R^2 - r^2) \, dx}$}
\choice[correct]{$\displaystyle{V = \int_a^b 2\pi rh \, dx}$}
\end{multipleChoice}

The left and right endpoints of integration are:\\
$a = \answer{1}\;\;\;$ and $\;\;\; b = \answer{16}$.\\
The radius and height as functions of $x$ are:\\
$r = \answer{16-x} \;\;\;$ and $\; h = \answer{\sqrt[4]{x} - 1/x}$.\\

The definite integral that represents the volume is given by:\\
\begin{multipleChoice}
\choice{$\displaystyle{V = \int_1^{16} 2\pi \left(-x^2 + x^{3/2} - x + x^{1/2}\right)\; dx}$}
\choice[correct]{$\displaystyle{V = \int_1^{16} 2\pi \left(2x^{1/4} - \frac{2}{x} - x^{5/4} + 1\right)\; dx}$}
\choice{$\displaystyle{V = \int_1^{16} 2\pi \left(-x^2 + x^{3/2} + x - x^{1/2}\right)\; dx}$}
\end{multipleChoice}

The volume of the solid is $V = \answer{(16622/45 - 32\ln(16))\pi}$.

\end{problem}









\end{document}










\section{Video Lesson}




\begin{center}
\begin{foldable}
\unfoldable{Here is a video of Example 1}
%\youtube{} %vid of example 1
\end{foldable}
\end{center}


\begin{image}
\begin{tikzpicture}
\draw[scale= 2, <->] (0,1.2) -- (0,0) -- (1.2,0);
%\draw ‎[‎smooth,‎samples=100‎,domain=0:‎‎2‎‎] ‎plot (\x,‎‎‎‎‎‎‎‎\x);‎‎
\draw[scale = 2, thick, domain=0:1,smooth,variable=\x,blue] plot ({\x},{\x}) node at (0.7, 0.4) {$y = x$};
\draw[scale = 2, thick, domain=0:1,smooth,variable=\y,red]  plot ({\y*\y},{\y}) node at (0.5,1) {$y = \sqrt x$};
\draw[scale = 2, thick] (0.4, 0.4) -- (0.4,0.6325);
\draw[scale = 2, thin] (0.4, 0.05) -- (0.4, -0.05) node[below] {$x$};
\draw[scale = 2, thin] (1, 0.05) -- (1, -0.05) node[below] {$1$};
\draw[red] (4,1) circle(1);
\draw[blue] (4,1) circle(0.5);
\filldraw (4,1) circle(0.04);
\filldraw[red] (0.8,1.265) circle(0.04);
\filldraw[blue] (0.8,0.8) circle(0.04);
\filldraw (0.8,0) circle(0.04);
\draw[thin, blue] (4,1) -- (4.5, 1.1);
\draw[thin, red] (4,1) -- (3.5, 1.85);
\filldraw[red] (3.5,1.86) circle(0.04) node at (3.3, 2.1) {$R$};
\filldraw[blue] (4.48,1.1) circle(0.04) node at (4.65, 1.1) {$r$};
\node[red] at (6, 1.3) {$R = \sqrt x$};
\node[blue] at (6, 0.8) {$r =  x$};


\end{tikzpicture}
\end{image}


%multiple choice format; note the question environment; what is free response? what is verbatim?

%\begin{verbatim}
\begin{question}
What is your favorite color?
\begin{multipleChoice}
\choice[correct]{Rainbow}
\choice{Blue}
\choice{Green}
\choice{Red}
\end{multipleChoice}
\begin{freeResponse}
Hello
\end{freeResponse}
\end{question}
%\end{verbatim}


%note the indentation

\begin{question}
  Which one will you choose?
  \begin{multipleChoice}
    \choice[correct]{I'm correct.}
    \choice{I'm wrong.}
    \choice{I'm wrong too.}
  \end{multipleChoice}
\end{question}


%selectAll

\begin{question}
  Which one will you choose?
  \begin{selectAll}
    \choice[correct]{I'm correct.}
    \choice{I'm wrong.}
    \choice[correct]{I'm also correct.}
    \choice{I'm wrong too.}
  \end{selectAll}
\end{question}


\begin{freeResponse}
What is the chain rule used for?
\end{freeResponse}

Find the formula for the volume of a right, circular cone with height $h$, and base radius $r$.



To create the cone, consider revolving a line about the $x$-axis.
The equation of this line is the line $y = \frac{r}{h} x$ which goes through the origin and the point $(h,r)$.
Our cone is obtained by revolving the region between this line and the $x$-axis over the interval $[0,h]$



We imagine our cone as being constructed by revolving each of the vertical segments in our region individually.
For any value of $x$ between $x = 0$ and $x = h$, the length of the segment is given by the corresponding $y$-coordinate
on the line, $y = \frac{r}{h} x$. The area of the resulting disk is 
\[
A_x = \pi R(x)^2   = \pi \left(\frac{r}{h} x\right)^2  = \pi \frac{r^2}{h^2} x^2 .
\]


The volume of the solid of revolution, i.e., our cone, is then obtained by using a definite integral:
\[
V = \int_0^h A_x \; dx  = \int_0^h \pi R(x)^2 \; dx = \int_0^h \pi \frac{r^2}{h^2} x^2 \; dx 
\]
\[
= \pi \frac{r^2}{h^2}\int_0^h  x^2 \; dx = \pi \frac{r^2}{h^2} \left(\frac{x^3}{3}\right)\bigg|_0^h 
\]
\[
= \pi \frac{r^2}{h^2} \left(\frac{h^3}{3} - 0\right) = \pi \frac{r^2}{h^2} \cdot \frac{h^3}{3} = \frac13 \pi r^2 h.
\]

Thus, the volume of a cone of height $h$ and base radius $r$ is $V = \frac13 \pi r^2 h$, i.e., its volume is 1/3 of 
volume of the cylinder with the same height and radius.

%(INSERT FIGURE HERE)


\begin{example}[example 2]
In this example, we find the volume of a sphere with radius $r$.
To create the sphere, consider the semi-circle of radius $r$ centered at the origin: $y = \sqrt{r^2 - x^2}$.
Our sphere is obtained by revolving the region between this semi-circle and the $x$-axis over the interval $[-r,r]$

%(INSERT FIGURE HERE)

We imagine the sphere as being constructed by revolving each of the vertical segments in our region individually.
For any value of $x$ between $x = -r$ and $x = r$, the length of the segment is given by the corresponding $y$-coordinate
on the semi-circle, $y = \sqrt{r^2 - x^2}$. The area of the resulting disk is 
\[
A_x = \pi R(x)^2   = \pi \left(\sqrt{r^2 - x^2}\right)^2  = \pi (r^2 - x^2).
\]

%(INSERT FIGURE HERE)

The volume of the solid of revolution, i.e., our sphere, is then obtained by using a definite integral:
\[
V = \int_{-r}^r A_x \; dx  = \int_{-r}^r \pi R(x)^2 \; dx = \int_{-r}^r \pi (r^2 - x^2) \; dx 
\]
\[
= \pi \int_{-r}^r  (r^2 - x^2) \; dx = \pi  \left(r^2 x - \frac{x^3}{3}\right)\bigg|_{-r}^r 
\]
\[
= \pi \left[ \left(r^3 - \frac{r^3}{3}\right) - \left(-r^3 + \frac{r^3}{3}\right)\right] = \pi \left(\frac{2r^3}{3} + \frac{2r^3}{3}\right) = \frac43 \pi r^3.
\]

Thus, the volume of a sphere of radius $r$ is $V = \frac43 \pi r^3$.

%(INSERT FIGURE HERE)

\end{example}

