\documentclass{ximera}

%% You can put user macros here
%% However, you cannot make new environments



\newcommand{\ffrac}[2]{\frac{\text{\footnotesize $#1$}}{\text{\footnotesize $#2$}}}
\newcommand{\vasymptote}[2][]{
    \draw [densely dashed,#1] ({rel axis cs:0,0} -| {axis cs:#2,0}) -- ({rel axis cs:0,1} -| {axis cs:#2,0});
}


\graphicspath{{./}{firstExample/}}

\usepackage{amsmath}
\usepackage{amssymb}
\usepackage{array}
\usepackage[makeroom]{cancel} %% for strike outs
\usepackage{pgffor} %% required for integral for loops
\usepackage{tikz}
\usepackage{tikz-cd}
\usepackage{tkz-euclide}
\usetikzlibrary{shapes.multipart}


\usetkzobj{all}
\tikzstyle geometryDiagrams=[ultra thick,color=blue!50!black]


\usetikzlibrary{arrows}
\tikzset{>=stealth,commutative diagrams/.cd,
  arrow style=tikz,diagrams={>=stealth}} %% cool arrow head
\tikzset{shorten <>/.style={ shorten >=#1, shorten <=#1 } } %% allows shorter vectors

\usetikzlibrary{backgrounds} %% for boxes around graphs
\usetikzlibrary{shapes,positioning}  %% Clouds and stars
\usetikzlibrary{matrix} %% for matrix
\usepgfplotslibrary{polar} %% for polar plots
\usepgfplotslibrary{fillbetween} %% to shade area between curves in TikZ



%\usepackage[width=4.375in, height=7.0in, top=1.0in, papersize={5.5in,8.5in}]{geometry}
%\usepackage[pdftex]{graphicx}
%\usepackage{tipa}
%\usepackage{txfonts}
%\usepackage{textcomp}
%\usepackage{amsthm}
%\usepackage{xy}
%\usepackage{fancyhdr}
%\usepackage{xcolor}
%\usepackage{mathtools} %% for pretty underbrace % Breaks Ximera
%\usepackage{multicol}



\newcommand{\RR}{\mathbb R}
\newcommand{\R}{\mathbb R}
\newcommand{\C}{\mathbb C}
\newcommand{\N}{\mathbb N}
\newcommand{\Z}{\mathbb Z}
\newcommand{\dis}{\displaystyle}
%\renewcommand{\d}{\,d\!}
\renewcommand{\d}{\mathop{}\!d}
\newcommand{\dd}[2][]{\frac{\d #1}{\d #2}}
\newcommand{\pp}[2][]{\frac{\partial #1}{\partial #2}}
\renewcommand{\l}{\ell}
\newcommand{\ddx}{\frac{d}{\d x}}

\newcommand{\zeroOverZero}{\ensuremath{\boldsymbol{\tfrac{0}{0}}}}
\newcommand{\inftyOverInfty}{\ensuremath{\boldsymbol{\tfrac{\infty}{\infty}}}}
\newcommand{\zeroOverInfty}{\ensuremath{\boldsymbol{\tfrac{0}{\infty}}}}
\newcommand{\zeroTimesInfty}{\ensuremath{\small\boldsymbol{0\cdot \infty}}}
\newcommand{\inftyMinusInfty}{\ensuremath{\small\boldsymbol{\infty - \infty}}}
\newcommand{\oneToInfty}{\ensuremath{\boldsymbol{1^\infty}}}
\newcommand{\zeroToZero}{\ensuremath{\boldsymbol{0^0}}}
\newcommand{\inftyToZero}{\ensuremath{\boldsymbol{\infty^0}}}


\newcommand{\numOverZero}{\ensuremath{\boldsymbol{\tfrac{\#}{0}}}}
\newcommand{\dfn}{\textbf}
%\newcommand{\unit}{\,\mathrm}
\newcommand{\unit}{\mathop{}\!\mathrm}
%\newcommand{\eval}[1]{\bigg[ #1 \bigg]}
\newcommand{\eval}[1]{ #1 \bigg|}
\newcommand{\seq}[1]{\left( #1 \right)}
\renewcommand{\epsilon}{\varepsilon}
\renewcommand{\iff}{\Leftrightarrow}

\DeclareMathOperator{\arccot}{arccot}
\DeclareMathOperator{\arcsec}{arcsec}
\DeclareMathOperator{\arccsc}{arccsc}
\DeclareMathOperator{\si}{Si}
\DeclareMathOperator{\proj}{proj}
\DeclareMathOperator{\scal}{scal}
\DeclareMathOperator{\cis}{cis}
\DeclareMathOperator{\Arg}{Arg}
%\DeclareMathOperator{\arg}{arg}
\DeclareMathOperator{\Rep}{Re}
\DeclareMathOperator{\Imp}{Im}
\DeclareMathOperator{\sech}{sech}
\DeclareMathOperator{\csch}{csch}
\DeclareMathOperator{\Log}{Log}

\newcommand{\tightoverset}[2]{% for arrow vec
  \mathop{#2}\limits^{\vbox to -.5ex{\kern-0.75ex\hbox{$#1$}\vss}}}
\newcommand{\arrowvec}{\overrightarrow}
\renewcommand{\vec}{\mathbf}
\newcommand{\veci}{{\boldsymbol{\hat{\imath}}}}
\newcommand{\vecj}{{\boldsymbol{\hat{\jmath}}}}
\newcommand{\veck}{{\boldsymbol{\hat{k}}}}
\newcommand{\vecl}{\boldsymbol{\l}}
\newcommand{\utan}{\vec{\hat{t}}}
\newcommand{\unormal}{\vec{\hat{n}}}
\newcommand{\ubinormal}{\vec{\hat{b}}}

\newcommand{\dotp}{\bullet}
\newcommand{\cross}{\boldsymbol\times}
\newcommand{\grad}{\boldsymbol\nabla}
\newcommand{\divergence}{\grad\dotp}
\newcommand{\curl}{\grad\cross}
%% Simple horiz vectors
\renewcommand{\vector}[1]{\left\langle #1\right\rangle}


\outcome{Compute integrals involving powers and products of trigonometric functions}

\title{Trigonometric Integrals}

\begin{document}

\begin{abstract}
We compute integrals involving powers and products of trigonometric functions.
\end{abstract}

\maketitle

\section{Trigonometric Identities}

The following are the Pythagorean Trigonometric Identities (named for \link[Pythagoras of Samos]{https://en.wikipedia.org/wiki/Pythagoras}) which hold for all angles,
$\theta$, in the domains of the functions involved:
\[
\sin^2(\theta) + \cos^2(\theta) = 1,
\]
\[
1 + \tan^2(\theta) = \sec^2(\theta),
\]
and
\[
1 + \cot^2(\theta) = \csc^2(\theta).
\]
%You can learn more about these well know identities here (link).

Next, we have the double-angle formulas:

\[
\sin^2(\theta) = \frac{1-\cos(2\theta)}{2},
\]

and

\[
\cos^2(\theta) = \frac{1+\cos(2\theta)}{2}.
\]
We will find the double-angle formulas useful for integrating even powers of sine and cosine.

\section{U-substitution}
Let's compute some basic trig integrals using $u$-substitution.

\begin{example}
Compute
\[
\int \sin^3(x) \cos(x) \, dx.
\]
To find an anti-derivative, we will make the substitution $u = \sin(x)$, as it is the inside of a composition
and its differential, $du = \cos(x) \, dx$, is present as a factor of the integrand. 
This substitution transforms the original integral as follows:
\[
\int \sin^3(x) \cos(x) \, dx = \int u^3 \, du,
\]
which is an integral that can be computed from the power rule giving
\begin{align*}
\int \sin^3(x) \cos(x) \, dx &= \int u^3 \, du\\
                             &= \frac{u^4}{4} + C\\
                             &= \frac14 \sin^4(x) + C.
\end{align*}

\end{example}

\begin{example}
Compute
\[
\int \sec^5(x) \tan(x) \, dx.
\]
To find an anti-derivative, we will make the substitution $u = \sec(x)$, as it is the inside of a composition
and its differential, $du = \sec(x)\tan(x) \, dx$, is present as a factor. 
This substitution transforms the original integral as follows:
\[
\int \sec^5(x) \tan(x) \, dx = \int \sec^4(x) \sec(x) \tan(x) \, dx= \int u^4 \, du,
\]
which is an integral that can be computed from the power rule giving
\begin{align*}
\int \sec^5(x) \tan(x) \, dx &= \int \sec^4(x) \sec(x) \tan(x) \, dx\\
                             &= \int u^4 \, du\\
                             &= \frac{u^5}{5} + C\\
                             &= \frac15 \sec^5(x) + C.
\end{align*}

\end{example}

\begin{problem}
Make a $u$-substitution to compute each of the following indefinite integrals:
\begin{itemize}
\item $\displaystyle{\int 5\sin^{14}(x)\cos(x) \, dx}$
\item $\displaystyle{\int 12\cos^{3}(x)\sin(x) \, dx}$
\item $\displaystyle{\int 4\tan^{7}(x)\sec^2(x) \, dx}$
\item $\displaystyle{\int 10 \sec^{3}(x)\tan(x) \, dx}$
\item $\displaystyle{\int 16\csc^{16}(x)\cot(x) \, dx}$
\end{itemize}
\end{problem}

\section{Powers of Sine and Cosine}
We now consider integrals of the form 
\[
\int \sin^m(x) \cos^n(x) \; dx,
\]
where $m$ and $n$ are positive integers.
There are two main cases to be considered.
In the first case, $m$ and/or $n$ is odd.
The solution to problems of this type involves $u$-substitution,
as we saw above.
In the second case, $m$ and $n$ are both even. The double angle formulas above are used to 
compute integrals of this type. We will also need the basic formula:
\[
\int \cos(ax) \; dx = \frac{1}{a} \sin(ax) + C.
\]

\subsection{Case 1: $m$ or $n$ is odd}
\begin{example}
Compute the indefinite integral:
\[
\int \sin^4(x)\cos^3(x) \; dx.
\]
We will use the substitution $u = \sin(x)$ because of the even power.
Then $du = \cos(x) \, dx$ and so $dx = \frac{du}{\cos(x)}$ and the integral becomes
\begin{align*}
\int \sin^4(x)\cos^3(x) \; dx &= \int u^4 \cos^3(x) \frac{du}{\cos(x)}\\
   &= \int u^4 \cos^2(x) \; du &
   \text{(use a Pythagorean identity)}\\
   &= \int u^4 (1-\sin^2(x)) \; du\\
   &= \int u^4 (1-u^2) \; du\\
   &= \int (u^4 -u^6) \; du &
  \text{(now integrate)}\\
  &= \frac{u^5}{5} - \frac{u^7}{7} + C &  \text{(finally, substitute for $u$)}\\
  &= \frac15 \sin^5(x) - \frac17 \sin^7(x) + C.
\end{align*}
  
\end{example}

\begin{problem}
Compute the indefinite integrals:
\[
\int \cos^6(x)\sin^3(x) \; dx = \answer{\frac19 \cos^9(x) - \frac17 \cos^7(x)} + C.
\]
\[
\int \sin^4(x)\cos^5(x) \; dx = \answer{\frac15 \sin^5(x) - \frac27 \sin^7(x)+\frac19 \sin^9(x)} + C.
\]
\end{problem}

\subsection{Case 2: $m$ and $n$ are both even}
\begin{example}
Compute the indefinite integral:
\[
\int \sin^2(x)\cos^2(x) \; dx.
\]
We will use the double angle formulas
\[
\sin^2(\theta) = \frac{1-\cos(2\theta)}{2} \; \text{ and } \; \cos^2(\theta) = \frac{1+\cos(2\theta)}{2},
\]
to rewrite the integral as
\begin{align*}
\int \sin^2(x)\cos^2(x) \; dx &= \int \frac{1-\cos(2x)}{2}\cdot \frac{1+\cos(2x)}{2} \; dx\\
  &= \frac14 \int \left[1 - \cos^2(2x)\right] \; dx &
  \text{(double angle formula again)}\\
  &= \frac14 \int \left[ 1- \frac{1+\cos(4x)}{2}\right] \; dx\\
  &= \frac14 \int \left[1-\frac12 - \frac12\cos(4x)\right] \; dx\\
  &= \frac14 \int \left[\frac12 - \frac12\cos(4x)\right] \; dx & \text{(finally, integrate)}\\
  &= \frac14 \left[\frac12 x  - \frac18\sin(4x)\right] + C\\
  &= \frac{x}{8}  - \frac{1}{32}\sin(4x) + C.
\end{align*}
\end{example}





\begin{problem}
Compute the indefinite integrals:
\[
\int \sin^2(x) \; dx = \answer{\frac12 x - \frac14 \sin(2x)} + C.
\]
\[
\int \cos^4(x) \; dx = \answer{\frac{3x}{8} +\frac14 \sin(2x) + \frac{1}{32}\sin(4x)} + C.
\]
\end{problem}

\section{Powers of Secant and Tangent}
We consider integrals of the form
\[
\int \tan^m(x)\sec^n(x) \; dx,
\]
where either $m$ is odd or $n$ is even. In both cases, a $u$-substitution
can solve the problem. 

\subsection{Case 1: $m$ is odd}
\begin{example}
Compute the indefinite integral:
\[
\int \tan^5(x) \sec^3(x)\;dx.
\]
When $m$ is odd, we use the substitution $u = \sec(x)$,
which implies $du = \sec(x) \tan(x) \; dx$ or $dx = \frac{du}{\sec(x)\tan(x)}$.
We begin by just substituting for $dx$:
\begin{align*}
\int \tan^5(x) \sec^3(x)\;dx &= \int \tan^5(x) \sec^3(x) \frac{du}{\sec(x)\tan(x)}\\
&= \int \tan^4(x) \sec^2(x) \; du & \text{(cancelling)}\\
&= \int \tan^4(x) u^2 \; du & \text{(use Pythagorean identity)}\\
&= \int \left[\tan^2(x)\right]^2 u^2 \; du \\
&= \int \left[\sec^2(x) -1\right]^2 u^2 \; du \\
&= \int \left(u^2 -1\right)^2 u^2 \; du \\
&= \int \left(u^4 - 2u^2 +1\right) u^2 \; du & \text{(distribute $u^2$)}\\
&= \int \left(u^6 - 2u^4 +u^2\right) \; du & \text{(now integrate)}\\
&=  \frac{u^7}{7} - \frac{2u^5}{5} + \frac{u^3}{3} + C  \\
&= \frac17\sec^7(x) - \frac25 \sec^5(x) + \frac13 \sec^3(x) + C & \text{(back substituting)}
\end{align*}
\end{example}

\begin{problem}
Compute the indefinite integral:
\[
\int \tan^3(x) \sec^5(x)\;dx = \answer{\frac17 \sec^7(x) - \frac15 \sec^5(x)} + C.
\]
\end{problem}

\begin{problem}
Compute the indefinite integral:
\[
\int \tan^5(x) \sec^7(x)\;dx = \answer{\frac{1}{11} \sec^{11}(x) + \frac17 \sec^7(x) -\frac29 \sec^9(x)} + C.
\]
\end{problem}



\subsection{Case 2: $n$ is even}
\begin{example}
Compute the indefinite integral:
\[
\int \tan^4(x) \sec^4(x)\;dx.
\]
When $n$ is even, we use the substitution $u = \tan(x)$,
which implies $du = \sec^2(x) \; dx$ or $dx = \frac{du}{\sec^2(x)}$.
We begin by just substituting for $dx$:
\begin{align*}
\int \tan^4(x) \sec^4(x)\;dx &= \int \tan^4(x) \sec^4(x) \frac{du}{\sec^2(x)}\\
&= \int \tan^4(x) \sec^2(x) \; du & \text{(cancelling)}\\
&= \int u^4 \sec^2(x) \; du \\
&= \int u^4\left[1 + \tan^2(x)\right] \; du & \text{(use Pythagorean identity)}\\
&= \int u^4(1+u^2) \; du & \text{(distribute $u^4$)}\\
&= \int (u^4 + u^6) \; du & \text{(now integrate)}\\
&=  \frac{u^5}{5} + \frac{u^7}{7} + C  \\
&= \frac15\tan^5(x)  + \frac17 \tan^7(x) + C & \text{(substituting for $u$)}
\end{align*}
\end{example}

\begin{problem}
Compute the indefinite integral:
\[
\int \tan^6(x) \sec^4(x)\;dx = \answer{\frac17 \tan^7(x) + \frac19 \tan^9(x)} + C.
\]
\end{problem}

\begin{problem}
Compute the indefinite integral:
\[
\int \tan^{1/3}(x) \sec^4(x)\;dx = \answer{\frac34 \tan^{4/3}(x) + \frac{3}{10} \tan^{10/3}(x)} + C.
\]
\end{problem}

\begin{problem}
Compute the indefinite integral:
\[
\int  \sec^4(x)\;dx = \answer{\tan(x) + \frac13 \tan^3(x)} + C.
\]
\end{problem}


\section{Odd Powers of Secant}
\begin{example}
Compute the indefinite integral:
\[
\int \sec(x) \; dx.
\]
This integral is computed by using a $u$-substitution after rewriting the integrand
by multiplying by 1 in the following way:
\[
\int \sec(x) \; dx = \int \sec(x) \cdot \frac{\sec(x) + \tan(x)}{\sec(x) + \tan(x)} \; dx.
\]
After distributing $\sec(x)$ we get
\[
\int \sec(x) \; dx = \int \frac{\sec^2(x) + \sec(x)\tan(x)}{\sec(x) + \tan(x)} \; dx.
\]
Now we observe that the numerator is the derivative of the denominator which suggests the substitution
$u = \sec(x) + \tan(x)$. We have $du = [ \sec(x)\tan(x) + \sec^2(x)] \; dx$ and the integral becomes
\begin{align*}
\int \sec(x) \; dx &= \int \frac{\sec^2(x) + \sec(x)\tan(x)}{\sec(x) + \tan(x)} \; dx\\
&= \int \frac{1}{u} \; du\\
&= \ln|u| + C\\
&= \ln|\sec(x) + \tan(x)|.
\end{align*}

\end{example}

%The integral of $\sec^3(x)$ was handled in the Integration by parts section
%yielding
%\[
%\int \sec^3(x) \; dx = \frac12 \sec(x) \tan(x) + \frac12 \ln|\sec(x) + \tan(x)| + C.
%\]
Furthermore, IBP also gave a reduction formula for higher powers of $\sec(x)$:
\[
\int \sec^n(x) \; dx =  \frac{1}{n-1}\sec^{n-2}(x)\tan(x) + \frac{n-2}{n-1}\int \sec^{n-2}(x) \; dx. 
\]
This formula is particularly useful if
the power is odd, since if $n$ is odd, $n-2$ is also odd.  Therefore, if we repeat the formula enough times, 
we will eventually be left with $\int \sec(x) \, dx$
which we have just solved.

\begin{example}
Use the above reduction of powers formula twice to compute the indefinite integral
\[
\int \sec^5(x) \; dx.
\]
We have $n =5$, and the formula gives
\[
\int \sec^5(x) \; dx = \frac14 \sec^3(x)\tan(x) + \frac34 \int \sec^3(x) \; dx.
\]
We will use the reduction of powers formula once again, this time with $n=3$:
\begin{align*}
\int \sec^5(x) \; dx &= \frac14 \sec^3(x)\tan(x) + \frac34 \int \sec^3(x) \; dx\\
&= \frac14 \sec^3(x)\tan(x) + \frac34\left(\frac12\sec(x)\tan(x) + \frac12\int \sec(x) \; dx\right)\\
&=\frac14 \sec^3(x)\tan(x) + \frac38\sec(x)\tan(x) + \frac38 \int \sec(x) \; dx \\
&=\frac14 \sec^3(x)\tan(x) + \frac38\sec(x)\tan(x) + \frac38\ln|\sec(x) + \tan(x)| + C
\end{align*}
\end{example}



%\begin{center}
%\begin{foldable}
%\unfoldable{Here is a video of Example 1}
%\youtube{Yy6QXnFlnXs} %vid of example 1
%\end{foldable}
%\end{center}





\begin{center}
\begin{foldable}
\unfoldable{Here are two detailed, lecture style videos on trigonometric integrals:}
\youtube{qbRIlTVP_2E}
\youtube{G0R-dItF4EA}
\end{foldable}
\end{center}





\end{document}


\begin{example} %example #15
Find $h'(x)$ if $h(x) = x^{\sin(x)}$.\\
We will use the fact that the exponential and logarithm functions are inverses,
\[e^{\ln(x)} = x,\]
and the exponent property of logarithms, 
\[\ln(x^n) = n\ln(x),\]
to rewrite $h(x)$.  We have 
\[h(x) = x^{\sin(x)} = e^{\ln(x^{\sin(x)})} = e^{\sin(x)\ln(x)}\]
and we can now compute $h'(x)$ using a combination of the chain rule and product rule.
We can write $h(x)$ as a composition, $f(g(x))$ with 
\[g(x) = \sin(x)\ln(x) \quad \text{and} \quad f(x) = e^x.\]
Then to find $g'(x)$ we us the product rule and we get $g'(x) = \frac{\sin(x)}{x} + \cos(x)\ln(x)$.
Next $f'(x) = e^x$ and 
hence $f'(g(x)) = e^{g(x)} = e^{\sin(x)\ln(x)} = x^{\sin(x)}$.
We can then conclude $h'(x) = f'(g(x))g'(x) = x^{\sin(x)} \left[ \frac{\sin(x)}{x} + \cos(x)\ln(x)\right]$.
\end{example}

%more question formats below













%\begin{verbatim}
\begin{question}
What is your favorite color?
\begin{multipleChoice}
\choice[correct]{Rainbow}
\choice{Blue}
\choice{Green}
\choice{Red}
\end{multipleChoice}
\begin{freeResponse}
Hello
\end{freeResponse}
\end{question}
%\end{verbatim}





\begin{question}
  Which one will you choose?
  \begin{multipleChoice}
    \choice[correct]{I'm correct.}
    \choice{I'm wrong.}
    \choice{I'm wrong too.}
  \end{multipleChoice}
\end{question}


\begin{question}
  Which one will you choose?
  \begin{selectAll}
    \choice[correct]{I'm correct.}
    \choice{I'm wrong.}
    \choice[correct]{I'm also correct.}
    \choice{I'm wrong too.}
  \end{selectAll}
\end{question}


\begin{freeResponse}
What is the chain rule used for?
\end{freeResponse}
