\documentclass{ximera}

%% You can put user macros here
%% However, you cannot make new environments



\newcommand{\ffrac}[2]{\frac{\text{\footnotesize $#1$}}{\text{\footnotesize $#2$}}}
\newcommand{\vasymptote}[2][]{
    \draw [densely dashed,#1] ({rel axis cs:0,0} -| {axis cs:#2,0}) -- ({rel axis cs:0,1} -| {axis cs:#2,0});
}


\graphicspath{{./}{firstExample/}}

\usepackage{amsmath}
\usepackage{amssymb}
\usepackage{array}
\usepackage[makeroom]{cancel} %% for strike outs
\usepackage{pgffor} %% required for integral for loops
\usepackage{tikz}
\usepackage{tikz-cd}
\usepackage{tkz-euclide}
\usetikzlibrary{shapes.multipart}


\usetkzobj{all}
\tikzstyle geometryDiagrams=[ultra thick,color=blue!50!black]


\usetikzlibrary{arrows}
\tikzset{>=stealth,commutative diagrams/.cd,
  arrow style=tikz,diagrams={>=stealth}} %% cool arrow head
\tikzset{shorten <>/.style={ shorten >=#1, shorten <=#1 } } %% allows shorter vectors

\usetikzlibrary{backgrounds} %% for boxes around graphs
\usetikzlibrary{shapes,positioning}  %% Clouds and stars
\usetikzlibrary{matrix} %% for matrix
\usepgfplotslibrary{polar} %% for polar plots
\usepgfplotslibrary{fillbetween} %% to shade area between curves in TikZ



%\usepackage[width=4.375in, height=7.0in, top=1.0in, papersize={5.5in,8.5in}]{geometry}
%\usepackage[pdftex]{graphicx}
%\usepackage{tipa}
%\usepackage{txfonts}
%\usepackage{textcomp}
%\usepackage{amsthm}
%\usepackage{xy}
%\usepackage{fancyhdr}
%\usepackage{xcolor}
%\usepackage{mathtools} %% for pretty underbrace % Breaks Ximera
%\usepackage{multicol}



\newcommand{\RR}{\mathbb R}
\newcommand{\R}{\mathbb R}
\newcommand{\C}{\mathbb C}
\newcommand{\N}{\mathbb N}
\newcommand{\Z}{\mathbb Z}
\newcommand{\dis}{\displaystyle}
%\renewcommand{\d}{\,d\!}
\renewcommand{\d}{\mathop{}\!d}
\newcommand{\dd}[2][]{\frac{\d #1}{\d #2}}
\newcommand{\pp}[2][]{\frac{\partial #1}{\partial #2}}
\renewcommand{\l}{\ell}
\newcommand{\ddx}{\frac{d}{\d x}}

\newcommand{\zeroOverZero}{\ensuremath{\boldsymbol{\tfrac{0}{0}}}}
\newcommand{\inftyOverInfty}{\ensuremath{\boldsymbol{\tfrac{\infty}{\infty}}}}
\newcommand{\zeroOverInfty}{\ensuremath{\boldsymbol{\tfrac{0}{\infty}}}}
\newcommand{\zeroTimesInfty}{\ensuremath{\small\boldsymbol{0\cdot \infty}}}
\newcommand{\inftyMinusInfty}{\ensuremath{\small\boldsymbol{\infty - \infty}}}
\newcommand{\oneToInfty}{\ensuremath{\boldsymbol{1^\infty}}}
\newcommand{\zeroToZero}{\ensuremath{\boldsymbol{0^0}}}
\newcommand{\inftyToZero}{\ensuremath{\boldsymbol{\infty^0}}}


\newcommand{\numOverZero}{\ensuremath{\boldsymbol{\tfrac{\#}{0}}}}
\newcommand{\dfn}{\textbf}
%\newcommand{\unit}{\,\mathrm}
\newcommand{\unit}{\mathop{}\!\mathrm}
%\newcommand{\eval}[1]{\bigg[ #1 \bigg]}
\newcommand{\eval}[1]{ #1 \bigg|}
\newcommand{\seq}[1]{\left( #1 \right)}
\renewcommand{\epsilon}{\varepsilon}
\renewcommand{\iff}{\Leftrightarrow}

\DeclareMathOperator{\arccot}{arccot}
\DeclareMathOperator{\arcsec}{arcsec}
\DeclareMathOperator{\arccsc}{arccsc}
\DeclareMathOperator{\si}{Si}
\DeclareMathOperator{\proj}{proj}
\DeclareMathOperator{\scal}{scal}
\DeclareMathOperator{\cis}{cis}
\DeclareMathOperator{\Arg}{Arg}
%\DeclareMathOperator{\arg}{arg}
\DeclareMathOperator{\Rep}{Re}
\DeclareMathOperator{\Imp}{Im}
\DeclareMathOperator{\sech}{sech}
\DeclareMathOperator{\csch}{csch}
\DeclareMathOperator{\Log}{Log}

\newcommand{\tightoverset}[2]{% for arrow vec
  \mathop{#2}\limits^{\vbox to -.5ex{\kern-0.75ex\hbox{$#1$}\vss}}}
\newcommand{\arrowvec}{\overrightarrow}
\renewcommand{\vec}{\mathbf}
\newcommand{\veci}{{\boldsymbol{\hat{\imath}}}}
\newcommand{\vecj}{{\boldsymbol{\hat{\jmath}}}}
\newcommand{\veck}{{\boldsymbol{\hat{k}}}}
\newcommand{\vecl}{\boldsymbol{\l}}
\newcommand{\utan}{\vec{\hat{t}}}
\newcommand{\unormal}{\vec{\hat{n}}}
\newcommand{\ubinormal}{\vec{\hat{b}}}

\newcommand{\dotp}{\bullet}
\newcommand{\cross}{\boldsymbol\times}
\newcommand{\grad}{\boldsymbol\nabla}
\newcommand{\divergence}{\grad\dotp}
\newcommand{\curl}{\grad\cross}
%% Simple horiz vectors
\renewcommand{\vector}[1]{\left\langle #1\right\rangle}


\outcome{Analyze the position, velocity and acceleration of an object moving in a straight line.}

\title{2.14 Rectilinear Motion}



\begin{document}

\begin{abstract}
In this section we analyze the motion of a particle moving in a straight line. Our
analysis includes the position, velocity and acceleration of the particle.
\end{abstract}

\maketitle


\section{Rectilinear Motion}

In this section we analyze the motion of a particle moving in a straight line. Our
analysis includes the position, velocity and acceleration of the particle.
We assume that the line is a horizontal number line with the origin in a fixed position, although, in some situations, 
the line is more naturally placed vertically. We will consider the relationship between the position, 
velocity and acceleration as functions of time using a rate of change perspective. \\
Given a \textbf{position function}, $s(t)$, the \textbf{instantaneous velocity} of the particle at time $t$, 
denoted $v(t)$, is the rate of change of position 
with respect to time:
\[v(t) = s'(t).\]
Similarly, the \textbf{instantaneous acceleration} of the particle at time $t, a(t)$, is the rate of change of 
velocity with respect to time:
\[a(t) = v'(t).\]
The connection between position and acceleration is thus:
\[a(t) = s''(t).\]





\begin{example}[example 1]
Suppose that the position of a particle in motion along a straight line is given by:
\[s(t) = t^3 - 3t^2 + 2.\]
Find the initial position of the particle and the position at time $t = 2$ seconds.
Then, use these to find the \textbf{displacement} of the particle during the first two seconds.\\

The initial position is given by plugging $t=0$ into the position function: 
\[s(0) = 0^3 -3(0)^2 + 2 = 2.\]
Interpretation: assuming that the motion is taking place along a horizontal line,
with the positive direction to the right and the negative direction to the left, then the initial position 
of the particle is 2 units to the right of the origin.
At time $t=2$ the particle is at position
\[s(2) = 2^3 -3(2)^2 + 2 = 8 - 12 + 2 = -2,\]
so that at time $t = 2$, the particle is 2 units to the left of the origin.
\begin{center}
\begin{tikzpicture}
\draw[<->] (-3,0) -- (3,0);
\draw (-2,-.2) -- (-2,.2); %tick mark
\draw (0,-.2) -- (0,.2);
\draw (2,-.2) -- (2,.2);
\draw[fill] (-2,0) circle [radius=0.07];
\draw[fill] (2,0) circle [radius=0.07];
\node [above] at (-2,.3) {$s(2)$};
\node [above] at (2,.3) {$s(0)$};
\node[below] at (-2.1,-.3) {$-2$};
\node[below] at (0,-.3) {$0$};
\node[below] at (2,-.3) {$2$};
\node[below] at (0,-.8) {position axis};
\end{tikzpicture}
\end{center}

From this information, one might be tempted to conclude that the particle has traveled a distance  of 4 units
during the time interval [0,2].  However, we cannot be certain that the particle did not change direction 
during this time interval and so that conclusion might be false. 
One thing that we can conclude at this point is that the displacement from 
time $t = 0$ to time $t = 2$ is
\[s(2) - s(0) = -2 - 2 = -4,\]
meaning that the particle has been displaced 4 units to the left. To determine the distance traveled, we need to analyze the velocity function.  




\end{example}



\begin{problem}(problem 1)
A particle moving along a straight line has position function
\[s(t) = t^3 - 5t -3.\]
where $s$ is measured in feet and $t$ is measured in seconds.
Find the initial position, the position at time $t = 3$ 
and the displacement over the time interval $[0,3]$.

The initial position is $\answer{-3}$ feet.\\
The position at time $t=3$ is $\answer{9}$ feet.\\
The displacement over the interval $[0,3]$ is $\answer{12}$ feet.\\

\end{problem}

 
\begin{example}[example 2] Let's compute the velocity of the particle in the preceding example, where
\[s(t) = t^3 - 3t^2 + 2.\]
Since 
\[v(t) = s'(t),\]
we have
\[v(t) = 3t^2 - 6t.\]
From this we can note that
\[v(0) = 3(0)^2 - 6(0) = 0.\]
This means that the initial velocity of the particle was 0.  

It is interesting to solve the equation 
\[v(t) = 0\]
for $t$. The solutions will tell us all of the times that the particle was at rest. These times are significant, 
because assuming that the motion of the particle is continuous, 
these are the only times that the particle can change direction.
We have
\[v(t) = 0\]
\[3t^2 - 6t = 0\]
\[3t(t-2) = 0\]
\[3t = 0 \ \ \text{or} \ \ t-2 = 0\]
\[t = 0 \ \ \text{or} \ \ t = 2.\]

Since the particle can only change directions at $t=0$ and $t=2$, we can now conclude that the 
total distance traveled by the particle from time $t=0$ to time $t=2$ is 

\[|s(2) - s(0)| = |-2 - 2| = |-4| = 4 \ \text{units.}\]

When the velocity function is negative, that means the particle is moving to the left and when the 
velocity is positive that means that the particle is moving to the right. Since $v(t)$ can be written as
\[v(t) = 3t(t-2)\]
we can conclude that the velocity is negative when $0<t<2$ and positive when $t>2$.
Thus the particle begins at rest, then moves to the left between $t = 0$ and $t = 2$, comes to rest again at $t=2$
and moves to the right for $t>2$.

To find the total distance the particle traveled from time $t = 1$ to time $t = 3$, we would need to take into consideration
the fact that it changes direction at $t = 2$:
\[ \text{distance} \  = |s(2) - s(1)| + |s(3) -s(2)|\]
\[= |-2-0|+|2-(-2)| = 2 + 4 = 6 \ \text{units}.\]
Notice that the absolute value is used in computing distance but not when computing displacement.




\begin{center}
\begin{tikzpicture}
\draw[<->] (-3,0) -- (3,0);
\draw (-2,-.2) -- (-2,.2); %tick mark
\draw (0,-.2) -- (0,.2);
\draw (2,-.2) -- (2,.2);
\draw[fill] (-2,0) circle [radius=0.07];
\draw[fill] (0,0) circle [radius=0.07];
\draw[fill] (2,0) circle [radius=0.07];
\node [above] at (-2,.3) {$s(2)$};
\node [above] at (0,.3) {$s(1)$};
\node [above] at (2,.3) {$s(3)$};
\node[below] at (-2.1,-.3) {$-2$};
\node[below] at (0,-.3) {$0$};
\node[below] at (2,-.3) {$2$};
\draw[<-, thick] (-2,-1) -- (0,-1);
\draw[->, thick] (-2,-1.3) -- (2,-1.3);
\node[below] at (0,-1.6) {motion of the particle};
\end{tikzpicture}
\end{center}


\begin{center}
\geogebra{p6ttPn2n}{1200}{600}
\end{center}


\end{example}



\begin{problem}(problem 2)
A particle moving along a straight line has position function
\[s(t) = t^3 - 5t -3.\]
where $s$ is measured in feet and $t$ is measured in seconds.
Find the velocity function, initial velocity, and the velocity at $t = 3$ seconds.

The velocity function is $v(t) = \answer{3t^2 -5}$ (measured in ft/sec).\\
The initial velocity is $\answer{-5}$ ft/sec.\\
The velocity at time $t=3$ is $\answer{22}$ ft/sec.\\


\end{problem}

 

\begin{example}[example 3]
Next, we will analyze the acceleration of the particle from the first example where:
\[s(t) = t^3 - 3t^2 + 2.\]
We have
\[a(t) = s''(t) = 6t -6.\]
Plugging in $t=0$ gives,
\[a(0) = -6.\]
The negative sign tells us that the particle is accelerating to the left. 
This is compatible with our previous knowledge that the particle begins at rest and then starts moving to the left. 
A leftward acceleration is necessary in order for this to occur.
Next, when $t=1$ the acceleration is $a(1) = 6-6 = 0$, so the particle is not undergoing acceleration at this time.
Between $t=1$ and $t = 2$ the acceleration is positive, meaning to the right. 
This makes sense since the particle is moving to the left but coming to rest when $t = 2$.
In general, when the velocity and the acceleration are in the same direction (i.e. have the same sign), 
then the particle is speeding up and when
the velocity and the acceleration are in opposite directions (i.e. have opposite signs),  then the particle is slowing down.
\end{example}



\begin{problem}(problem 3)
A particle moving along a straight line has position function
\[s(t) = t^3 - 5t -3.\]
where $s$ is measured in feet and $t$ is measured in seconds.
Find the acceleration function, initial acceleration, and the acceleration at $t = 3$ seconds.\\

The acceleration function is a(t)$ = \answer{6t}$ (measured in ft/sec$^2$).\\
The initial acceleration is $\answer{0}$ ft/sec$^2$.\\
The acceleration at time $t=3$ is $\answer{18}$ ft/sec$^2$.


\end{problem}


%\item[strat] Re-organize projectile motion.  3 position functions. each example explores one facet. 
%Cycle through the three examples, then move on to next facet.
%facets: initial position, initial velocity, initial acceleration; how long in the air; max height; 
%total distance traveled; distance traveled in certain time interval;  velocity at a certain height.

\begin{example}[example 4]
A projectile is launched vertically with position function given by:
\[s(t) = 128t - 16t^2,\]
where $t$ is measured in seconds and $s(t)$ is measured in feet.
Determine the initial height and the initial velocity of the object.\\

To find the initial height, we plug $t = 0$ into $s(t)$:
\[s(0) = 128(0) - 16(0)^2 = 0.\]
Since the initial height is zero feet, the object was launched from the ground.

To compute the initial velocity, we first find the velocity function and then we plug in $t=0$.
We have
\[v(t) = s'(t) = 128 - 32t\]
and 
\[v(0) = 128 - 32(0) = 128.\]
Hence the initial velocity of the projectile was 128 ft/sec.

\end{example}




\begin{example}[example 5]
Determine its maximum height
of an object that is launched straight up into the air with position function:
\[s(t) = 12 + 20t - 16t^2,\]
where $t$ is measured in seconds and $s(t)$ is measured in feet. 
In this scenario, the positive direction is up and our position function measures the height of the object.\\
%The object is launched from a height of 12 ft. since 
%\[s(0) = 12 + 20(0) - 16(0)^2 = 12.\]
%The instantaneous velocity of the object at time $t$ is given by
%\[v(t) = s'(t) = 20-32t\]
%and the initial velocity is
%\[v(0) = 20 - 32(0) = 20 \ \text{ft/sec}.\]
%The only force acting on the object is gravity and
%\[a(t) = v'(t) = -32 \ \text{ft/sec$^2$}.\]
%The object is undergoing a constant acceleration in the downward direction.
To find the maximum height of the object we observe that at the maximum height the object is momentarily at rest. 
In other words, at the maximum height, the velocity is 0. We therefore begin by solving the equation
\[v(t) = 0  \  \ \text{for $t$.}\]
 We get 
\[20-32t = 0\]
\[t = \frac{20}{32} = \frac{5}{8} = 0.625 \ \text{seconds}.\]
The maximum height is then obtained by plugging this time into the position function:
\[s\Big(\frac{5}{8}\Big) = 12 + 20\Big(\frac{5}{8}\Big) - 16\Big(\frac{5}{8}\Big)^2 \]
\[= 12 + \frac{25}{2} - \frac{25}{4} = \frac{73}{4} = 18.25\  \text{feet}. \]
\end{example}


\begin{problem}(problem 5a)
Determine the maximum height
of an object that is launched straight up into the air with position function:
\[s(t) = 24 + 96t - 16t^2,\]
where $t$ is measured in seconds and $s(t)$ is measured in feet. 

\begin{hint}
The maximum height occurs when the velocity is zero
\end{hint}
\begin{hint}
The maximum height is obtained by plugging a time into the position function
\end{hint}

The maximum height is $\answer{168}$ feet.

\end{problem}


\begin{problem}(problem 5b)
Determine its maximum height
of an object that is launched straight up into the air on Planet X with position function:
\[s(t) = 6 + 36t - 12t^2,\]
where $t$ is measured in seconds and $s(t)$ is measured in feet. 

\begin{hint}
The maximum height occurs when the velocity is zero
\end{hint}
\begin{hint}
The maximum height is obtained by plugging a time into the position function
\end{hint}

The maximum height is $\answer{33}$ feet.

\end{problem}
 
 
\begin{example}[example 6]
Determine the total distance traveled by
an object that is sent straight up into the air with position function:
\[s(t) = 32 + 48t - 16t^2,\]
where $t$ is measured in seconds and $s(t)$ is measured in feet. \\
The object is launched from a height of 32 ft. since 
\[s(0) = 32 + 48(0) - 16(0)^2 = 32.\]
The instantaneous velocity of the object at time $t$ is given by
\[v(t) = s'(t) = 48-32t.\]
To determine the total distance traveled by the object, we need to find its maximum height, so
we set the velocity equal to zero and solve for $t$:
\[v(t) = 0  \] 
\[48-32t = 0\]
\[t = \frac{48}{32} = \frac{3}{2} \ \text{seconds}.\]
The maximum height is then obtained by plugging this time into the position function:
\[s\Big(\frac{3}{2}\Big) = 32 + 48\Big(\frac{3}{2}\Big) - 16\Big(\frac{3}{2}\Big)^2 \]
\[= 32 + 72 - 36 = 68 \  \text{feet}. \]

The total distance traveled is the distance up plus the distance down.  The distance down is the maximum height of the object,
$68$ feet.  The distance up is the maximum height minus the initial height, $68 - 32$ feet.
Thus the total distance traveled is  
\[68 + (68 - 32) = 68 + 36 = 104 \ \text{feet}.\]
\end{example}



\begin{problem}(problem 6)
Determine the total distance traveled by an object that is launched straight up into the air on Planet X with position function:
\[s(t) = 6 + 36t - 12t^2,\]
where $t$ is measured in seconds and $s(t)$ is measured in feet. 

\begin{hint}
The total distance traveled is the distance rising plus the distance falling
\end{hint}
\begin{hint}
Assume the object lands on the ground, at height zero feet.
\end{hint}

The distance traveled is $\answer{60}$ feet.

\end{problem}

\begin{example}[example 7]
Suppose the position of a vertical projectile is given by 
\[s(t) = 80 + 64t-16t^2, \]
where $t$ is measured in seconds and $s(t)$ is measured in feet. 
Find the total time that the object is in the air, and find its velocity when its height is 128 feet.\\
 
To find the time in the air we need to solve $s(t) = 0$ for $t$. We get:
\[80 + 64t-16t^2 = 0\]
\[5 + 4t-t^2 = 0\]
\[t^2 -4t -5  = 0\]
\[(t-5)(t+1)  = 0\]
and so, $t =  5$ or $-1$ seconds. Since $-1$ seconds does not make sense in this situation, we conclude that
the total time that the object was in the air was 5 seconds.
Next, to determine the instantaneous velocity when the object was 128 ft. up, 
we need to first figure out the times when this occurred and then plug them into the velocity function, 
\[v(t) = s'(t) = 64 - 32t.\]
We have
\[s(t) = 128 \]
\[80 + 64t -16t^2 = 128\]
\[  16t^2 - 64t + 48 = 0\]
\[  t^2 - 4t + 3 = 0\]
\[  (t-1)(t-3) = 0\]
and so $t= 1$ or $3$ seconds.
In words, the object is at a height of 128 feet after 1 second (while it is on the way up) and after 3 seconds (on the way down).
The instantaneous velocities at these times are
\[v(1) = 64 - 32(1) = 32 \ \text{ft/sec}\]
\begin{center}
  and
\end{center}
\[ \ v(3) = 64 - 32(3) = -32 \text{ft/sec.}\]
The positive velocity at $t= 1$ second indicates that the object was on the way up and the negative velocity 
at time $t=3$ seconds indicates 
that the object was on its way down.  The absolute value of the velocity tells us the objects' speed. 
Hence the object is going at the same speed of 32 ft/sec at the height of 128 feet when it is on the way up as it 
is on the way down.
This is a general phenomenon of projectiles. The object loses speed on the way up and 
regains it at exactly the same rate on its way down so that at a given height it is going at the same 
speed on the way up as it is on the way down.
\end{example}


\begin{problem}(problem 7)
Suppose the position of a vertical projectile on Planet X is given by 
\[s(t) = 36t - 12t^2, \]
where $t$ is measured in seconds and $s(t)$ is measured in feet. 
Find the total time that the object is in the air, and find its velocity when its height is 24 feet and falling.
\begin{hint}
When was the projectile at the specified height?
\end{hint}
\begin{hint}
Plug the appropriate time into the velocity function
\end{hint}

The time in the air is $\answer{3}$ seconds.\\
The velocity at 24 feet and falling is $\answer{-12}$ ft/sec.

\end{problem}


%Total time in air, max height, velocity at a certain height, 






\end{document}
