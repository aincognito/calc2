\documentclass{ximera}

%% You can put user macros here
%% However, you cannot make new environments



\newcommand{\ffrac}[2]{\frac{\text{\footnotesize $#1$}}{\text{\footnotesize $#2$}}}
\newcommand{\vasymptote}[2][]{
    \draw [densely dashed,#1] ({rel axis cs:0,0} -| {axis cs:#2,0}) -- ({rel axis cs:0,1} -| {axis cs:#2,0});
}


\graphicspath{{./}{firstExample/}}

\usepackage{amsmath}
\usepackage{amssymb}
\usepackage{array}
\usepackage[makeroom]{cancel} %% for strike outs
\usepackage{pgffor} %% required for integral for loops
\usepackage{tikz}
\usepackage{tikz-cd}
\usepackage{tkz-euclide}
\usetikzlibrary{shapes.multipart}


\usetkzobj{all}
\tikzstyle geometryDiagrams=[ultra thick,color=blue!50!black]


\usetikzlibrary{arrows}
\tikzset{>=stealth,commutative diagrams/.cd,
  arrow style=tikz,diagrams={>=stealth}} %% cool arrow head
\tikzset{shorten <>/.style={ shorten >=#1, shorten <=#1 } } %% allows shorter vectors

\usetikzlibrary{backgrounds} %% for boxes around graphs
\usetikzlibrary{shapes,positioning}  %% Clouds and stars
\usetikzlibrary{matrix} %% for matrix
\usepgfplotslibrary{polar} %% for polar plots
\usepgfplotslibrary{fillbetween} %% to shade area between curves in TikZ



%\usepackage[width=4.375in, height=7.0in, top=1.0in, papersize={5.5in,8.5in}]{geometry}
%\usepackage[pdftex]{graphicx}
%\usepackage{tipa}
%\usepackage{txfonts}
%\usepackage{textcomp}
%\usepackage{amsthm}
%\usepackage{xy}
%\usepackage{fancyhdr}
%\usepackage{xcolor}
%\usepackage{mathtools} %% for pretty underbrace % Breaks Ximera
%\usepackage{multicol}



\newcommand{\RR}{\mathbb R}
\newcommand{\R}{\mathbb R}
\newcommand{\C}{\mathbb C}
\newcommand{\N}{\mathbb N}
\newcommand{\Z}{\mathbb Z}
\newcommand{\dis}{\displaystyle}
%\renewcommand{\d}{\,d\!}
\renewcommand{\d}{\mathop{}\!d}
\newcommand{\dd}[2][]{\frac{\d #1}{\d #2}}
\newcommand{\pp}[2][]{\frac{\partial #1}{\partial #2}}
\renewcommand{\l}{\ell}
\newcommand{\ddx}{\frac{d}{\d x}}

\newcommand{\zeroOverZero}{\ensuremath{\boldsymbol{\tfrac{0}{0}}}}
\newcommand{\inftyOverInfty}{\ensuremath{\boldsymbol{\tfrac{\infty}{\infty}}}}
\newcommand{\zeroOverInfty}{\ensuremath{\boldsymbol{\tfrac{0}{\infty}}}}
\newcommand{\zeroTimesInfty}{\ensuremath{\small\boldsymbol{0\cdot \infty}}}
\newcommand{\inftyMinusInfty}{\ensuremath{\small\boldsymbol{\infty - \infty}}}
\newcommand{\oneToInfty}{\ensuremath{\boldsymbol{1^\infty}}}
\newcommand{\zeroToZero}{\ensuremath{\boldsymbol{0^0}}}
\newcommand{\inftyToZero}{\ensuremath{\boldsymbol{\infty^0}}}


\newcommand{\numOverZero}{\ensuremath{\boldsymbol{\tfrac{\#}{0}}}}
\newcommand{\dfn}{\textbf}
%\newcommand{\unit}{\,\mathrm}
\newcommand{\unit}{\mathop{}\!\mathrm}
%\newcommand{\eval}[1]{\bigg[ #1 \bigg]}
\newcommand{\eval}[1]{ #1 \bigg|}
\newcommand{\seq}[1]{\left( #1 \right)}
\renewcommand{\epsilon}{\varepsilon}
\renewcommand{\iff}{\Leftrightarrow}

\DeclareMathOperator{\arccot}{arccot}
\DeclareMathOperator{\arcsec}{arcsec}
\DeclareMathOperator{\arccsc}{arccsc}
\DeclareMathOperator{\si}{Si}
\DeclareMathOperator{\proj}{proj}
\DeclareMathOperator{\scal}{scal}
\DeclareMathOperator{\cis}{cis}
\DeclareMathOperator{\Arg}{Arg}
%\DeclareMathOperator{\arg}{arg}
\DeclareMathOperator{\Rep}{Re}
\DeclareMathOperator{\Imp}{Im}
\DeclareMathOperator{\sech}{sech}
\DeclareMathOperator{\csch}{csch}
\DeclareMathOperator{\Log}{Log}

\newcommand{\tightoverset}[2]{% for arrow vec
  \mathop{#2}\limits^{\vbox to -.5ex{\kern-0.75ex\hbox{$#1$}\vss}}}
\newcommand{\arrowvec}{\overrightarrow}
\renewcommand{\vec}{\mathbf}
\newcommand{\veci}{{\boldsymbol{\hat{\imath}}}}
\newcommand{\vecj}{{\boldsymbol{\hat{\jmath}}}}
\newcommand{\veck}{{\boldsymbol{\hat{k}}}}
\newcommand{\vecl}{\boldsymbol{\l}}
\newcommand{\utan}{\vec{\hat{t}}}
\newcommand{\unormal}{\vec{\hat{n}}}
\newcommand{\ubinormal}{\vec{\hat{b}}}

\newcommand{\dotp}{\bullet}
\newcommand{\cross}{\boldsymbol\times}
\newcommand{\grad}{\boldsymbol\nabla}
\newcommand{\divergence}{\grad\dotp}
\newcommand{\curl}{\grad\cross}
%% Simple horiz vectors
\renewcommand{\vector}[1]{\left\langle #1\right\rangle}


\outcome{Find the sum of a geometric series}

\title{3.1 Geometric Series}

\begin{document}

\begin{abstract}
We find the sum of a geometric series.
\end{abstract}

\maketitle

\section{Geometric Series}

\begin{definition}[Geometric Series] A \textbf{geometric series} is an infinite series of the form
\[
\sum_{n=0}^\infty ar^n = a + ar + ar^2 + ar^3 + \cdots.
\]
The parameter $r$ is called the \textbf{common ratio}.
\end{definition}

\begin{example}[example 1]
Consider the series:
\[
 \sum_{n=0}^\infty \frac{3}{2^n} 
 \]
 Write out the first four terms of the series. If the series is geometric, find the common ratio.
 
 The first four terms are \[
 3 + \frac32 + \frac34 + \frac38 + \cdots.
 \]

 The series is geometric and the common ratio is $\frac12$.
 \end{example}
 
 

\begin{problem}(problem 1a)
 Consider the series:
 \[
 \sum_{n=0}^\infty \frac{2}{3^n} 
 \]
 
 True or false: this is a geometric series \wordChoice{\choice[correct]{True}\choice{False}}
 
 The first term is $a = \answer{2}$\\
 The common ratio is $r = \answer{1/3}$
 \end{problem}
 
 
 \begin{problem}(problem 1b)
 Consider the series:
 \[
 \sum_{n=2}^\infty \frac{3}{2^n} 
 \]
 
 True or false: this is a geometric series \wordChoice{\choice[correct]{True}\choice{False}}
 
 The first term is $a = \answer{3/4}$\\
 The common ratio is $r = \answer{1/2}$
 \end{problem}
 
 
 \begin{example}[example 2]
 Consider the series:
\[
 \sum_{n=1}^\infty \frac{3}{n^2} 
 \]
 Write out the first four terms of the series. If the series is geometric, find the common ratio.
 
 The first four terms are 
 \[
 3 + \frac34 + \frac39 + \frac{3}{16} + \cdots.
 \]
 The ratio of the second term to the first is $\frac14$ but the ratio of the third term to the second is $\frac49$.
 The series is not geometric because it fails to have a \it{common} ratio.
 
 \end{example}
 

 
\begin{problem}(problem 2a)
 Consider the series:
 \[
 \sum_{n=1}^\infty \frac{1}{n} 
 \]
 
 True or false: this is a geometric series \wordChoice{\choice{True}\choice[correct]{False}}
 
 \end{problem}
 

   \begin{problem}(problem 2b)
 Consider the series:
 \[
 \sum_{n=0}^\infty \sqrt n 
 \]

True or false: this is a geometric series \wordChoice{\choice{True}\choice[correct]{False}}
 
 \end{problem}
 
 
\section{Convergence}

\begin{theorem}[Convergence of Geometric Series] 
The geometric series
\[
\sum_{n=0}^\infty ar^n
\]
converges if and only if the common ratio $r$ satisfies the condition $-1 < r < 1$.
This condition can also be written as $|r| < 1$.
\end{theorem}

\begin{example}[example 3]
Does the geometric series $\displaystyle{\sum_{n=0}^\infty \frac{3}{2^n}}$ converge or diverge?\\
The common ratio of this geometric series is $r = \frac12$. Since $-1 < \frac 12 < 1$,
the series converges.
\end{example}


\begin{problem}(problem 3a)
 Does the geometric series
 \[
 \sum_{n=2}^\infty \frac{3}{2^n} 
 \]
 converge or diverge?\\
 The common ratio is $r = \answer{1/2}$\\
 The series
 \begin{multipleChoice}
 \choice[correct]{converges}
 \choice{diverges}
 \end{multipleChoice}
 \end{problem}
 

\begin{problem}(problem 3b)
 Does the geometric series
 \[
 \sum_{n=1}^\infty (-1)^n\frac{2}{3^n} 
 \]
 converge or diverge?\\
 The common ratio is $r = \answer{-1/3}$\\
 The series
 \begin{multipleChoice}
 \choice[correct]{converges}
 \choice{diverges}
 \end{multipleChoice}
 \end{problem}
 
 \begin{problem}(problem 3c)
 Does the geometric series
 \[
 \sum_{n=0}^\infty (-2)^n 
 \]
 converge or diverge?\\
 The common ratio is $r = \answer{-2}$\\
 The series
 \begin{multipleChoice}
 \choice{converges}
 \choice[correct]{diverges}
 \end{multipleChoice}
 \end{problem}
 
 
 
 
\begin{example}[example 4]
Does the geometric series $\displaystyle{\sum_{n=0}^\infty \frac{3^n}{2^{n+1}}}$ converge or diverge?\\
The common ratio of this geometric series is $r = \frac32$. Since $\frac32$ is not between $-1$ and $1$,
the series does not converge, i.e., it diverges.
\end{example}

\begin{problem}(problem 4a)
 Does the geometric series
 \[
 \sum_{n=0}^\infty \frac{2^{n+1}}{3^n} 
 \]
 converge or diverge?\\
 The common ratio is $r = \answer{2/3}$\\
 The series
 \begin{multipleChoice}
 \choice[correct]{converges}
 \choice{diverges}
 \end{multipleChoice}
 \end{problem}
 
 
 \begin{problem}(problem 4b)
 Does the geometric series
 \[
 \sum_{n=0}^\infty \frac{2^{2n}}{3^n} 
 \]
 converge or diverge?\\
 The common ratio is $r = \answer{4/3}$\\
 The series
 \begin{multipleChoice}
 \choice{converges}
 \choice[correct]{diverges}
 \end{multipleChoice}
 \end{problem}
 

\begin{example}[example 5]
Does the geometric series $\displaystyle{\sum_{n=0}^\infty \left(-\frac23\right)^n }$ converge or diverge?\\
The common ratio of this geometric series is $r = -\frac23$. Since $-1 < -\frac23 < 1$,
the series converges.
\end{example}

\begin{problem}(problem 5)
 Does the geometric series
 \[
 \sum_{n=0}^\infty \left(-\frac56 \right)^n
 \]
 converge or diverge?\\
 The common ratio is $r = \answer{-5/6}$\\
 The series
 \begin{multipleChoice}
 \choice[correct]{converges}
 \choice{diverges}
 \end{multipleChoice}
 \end{problem}

\section{Sum of a Geometric Series}

Consider the geometric series $\sum_{n=0}^\infty ar^n$ where $|r| < 1$ (so that the series converges).
The $N^{th}$ partial sum of this series is given by
\[
S_N = \sum_{n=0}^N ar^n = a + ar + ar^2 + \cdots + ar^{N-1} + ar^N.
\]
Multiply both sides by $r$:
\[
rS_N =  ar + ar^2 + + ar^3 + \cdots  + ar^N + ar^{N+1}.
\] 
Now subtract $rS_N$ from $S_N$:
\begin{align*}
S_N - rS_N &= \left(a + ar + \cdots + ar^{N-1} + ar^N\right) - \left(ar + ar^2 + \cdots  + ar^N + ar^{N+1} \right)\\
           &= a - ar^{N+1}.
\end{align*}
We can factor out $S_N$ on the left side and then divide by $1-r$ to obtain
\[
S_N = \frac{a - ar^{N+1}}{1-r}.
\]
We can now compute the sum of the geometric series by taking the limit as $N \to \infty$:
\[
S = \sum_{n=0}^\infty ar^n = \lim_{N \to \infty} \sum_{n=0}^N ar^n = \lim_{N \to \infty} S_N = \lim_{N \to \infty} \frac{a - ar^{N+1}}{1-r} = \frac{a}{1-r}.
\]
We present this formula in the theorem below.

\begin{theorem}[Sum of a Geometric Series]
If $-1 < r < 1$, then the sum of the geometric series 
\[
\sum_{n=0}^\infty ar^n = a + ar + ar^2 + \cdots
\] is given by 
\[
S = \frac{a}{1-r}.
\]
That is, if $-1 < r < 1$, then
\[
\sum_{n=0}^\infty ar^n = \frac{a}{1-r}.
\]
\end{theorem}

\begin{example}[example 6]
Find the sum of the geometric series 
\[
\sum_{n=0}^\infty \frac{3}{2^n}.
\]
The first term is $a = 3$, and the common ratio is $r = \frac12$. The sum of the series is
\[
\sum_{n=0}^\infty \frac{3}{2^n} = \frac{3}{1-\frac12} = \frac{3}{\left(\frac12\right)} = 6.
\]
\end{example}


\begin{problem}(problem 6a)
Find the sum of the geometric series 
\[
\sum_{n=0}^\infty \frac{5}{3^n}.
\]
The first term is $a = \answer{5}$\\
The common ratio is $r = \answer{1/3}$\\
The sum is $S= \answer{15/2}$

\end{problem}

\begin{problem}(problem 6b)
Find the sum of the geometric series 
\[
\sum_{n=1}^\infty \frac{4}{2^n}.
\]
The first term is $a = \answer{2}$\\
The common ratio is $r = \answer{1/2}$\\
The sum is $S= \answer{4}$

\end{problem}

\begin{problem}(problem 6c)
Find the sum of the geometric series 
\[
\sum_{n=0}^\infty \frac{3^n}{2^{2n}}.
\]
The first term is $a = \answer{1}$\\
The common ratio is $r = \answer{3/4}$\\
The sum is $S= \answer{4}$

\end{problem}



\begin{example}[example 7]
Find the sum of the geometric series 
\[
\sum_{n=0}^\infty \left(-\frac23\right)^n.
\]
The first term is $a = 1$, and the common ratio is $r = -\frac23$. The sum of the series is
\[
\sum_{n=0}^\infty \left(-\frac23\right)^n = \frac{1}{1-\left(-\frac23\right)} = \frac{1}{1 + \frac23} = \frac{1}{\left(\frac53\right)} = \frac35.
\]
\end{example}

\begin{problem}(problem 7a)
Find the sum of the geometric series 
\[
\sum_{n=1}^\infty \left(-\frac45\right)^n
\]
The first term is $a = \answer{-4/5}$\\
The common ratio is $r = \answer{-4/5}$\\
The sum is $S= \answer{-4/9}$

\end{problem}


\begin{problem}(problem 7b)
Find the sum of the geometric series 
\[
\sum_{n=2}^\infty \left(-\frac{2}{\pi}\right)^n
\]
The first term is $a = \answer{4/\pi^2}$\\
The common ratio is $r = \answer{-2/\pi}$\\
The sum is $S= \answer{4/(\pi^2 + 2\pi)}$

\end{problem}

\begin{problem}(problem 7c)
Find the sum of the geometric series 
\[
\sum_{n=0}^\infty \left(\frac{2}{e}\right)^n
\]
The first term is $a = \answer{1}$\\
The common ratio is $r = \answer{2/e}$\\
The sum is $S= \answer{e/(e-2)}$

\end{problem}

\begin{example}[example 8]
Write the repeating decimal $0.\overline{123} = 0.123123123...$ as a ratio of integers.\\
First, note that we can write this repeating decimal as an infinite series:
\[
0.\overline{123} = 0.123 + 0.000123 + 0.000000123 + \dots
\]
Wring these decimals as fractions, we have
\[
0.\overline{123} =\frac{123}{10^3} + \frac{123}{10^6}+\frac{123}{10^9} + \dots
\]
This is a convergent geometric series with first term, $a = \frac{123}{1000}$ and common ratio $r = \frac{1}{1000}$.
The sum is 
\[
S = \frac{a}{1-r} = \frac{\left(\frac{123}{1000}\right)}{1-\frac{1}{1000}} = \frac{\left(\frac{123}{1000}\right)}{\left(\frac{999}{1000}\right)}
= \frac{123}{999} = \frac{41}{333}.
\]

\end{example}

\begin{problem}(problem 8)
Write the repeating decimal $0.\overline{45} = 0.454545...$ as a ratio of integers.\\
\begin{multipleChoice}
\choice{$\frac{9}{20}$}
\choice{$\frac{45}{98}$}
\choice[correct]{$\frac{5}{11}$}
\end{multipleChoice}
\end{problem}
\section{Video Lesson}


\begin{center}
\begin{foldable}
\unfoldable{Here is a detailed, lecture style video on Geometric Series:}
\youtube{oM28tybb_AE}
\end{foldable}
\end{center}




\end{document}

\section{More Problems}

\begin{example} %example #15
Find $h'(x)$ if $h(x) = x^{\sin(x)}$.\\
We will use the fact that the exponential and logarithm functions are inverses,
\[e^{\ln(x)} = x,\]
and the exponent property of logarithms, 
\[\ln(x^n) = n\ln(x),\]
to rewrite $h(x)$.  We have 
\[h(x) = x^{\sin(x)} = e^{\ln(x^{\sin(x)})} = e^{\sin(x)\ln(x)}\]
and we can now compute $h'(x)$ using a combination of the chain rule and product rule.
We can write $h(x)$ as a composition, $f(g(x))$ with 
\[g(x) = \sin(x)\ln(x) \quad \text{and} \quad f(x) = e^x.\]
Then to find $g'(x)$ we us the product rule and we get $g'(x) = \frac{\sin(x)}{x} + \cos(x)\ln(x)$.
Next $f'(x) = e^x$ and 
hence $f'(g(x)) = e^{g(x)} = e^{\sin(x)\ln(x)} = x^{\sin(x)}$.
We can then conclude $h'(x) = f'(g(x))g'(x) = x^{\sin(x)} \left[ \frac{\sin(x)}{x} + \cos(x)\ln(x)\right]$.
\end{example}

%more question formats below













%\begin{verbatim}
\begin{question}
What is your favorite color?
\begin{multipleChoice}
\choice[correct]{Rainbow}
\choice{Blue}
\choice{Green}
\choice{Red}
\end{multipleChoice}
\begin{freeResponse}
Hello
\end{freeResponse}
\end{question}
%\end{verbatim}





\begin{question}
  Which one will you choose?
  \begin{multipleChoice}
    \choice[correct]{I'm correct.}
    \choice{I'm wrong.}
    \choice{I'm wrong too.}
  \end{multipleChoice}
\end{question}


\begin{question}
  Which one will you choose?
  \begin{selectAll}
    \choice[correct]{I'm correct.}
    \choice{I'm wrong.}
    \choice[correct]{I'm also correct.}
    \choice{I'm wrong too.}
  \end{selectAll}
\end{question}


\begin{freeResponse}
What is the chain rule used for?
\end{freeResponse}
