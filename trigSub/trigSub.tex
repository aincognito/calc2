\documentclass{ximera}

%% You can put user macros here
%% However, you cannot make new environments



\newcommand{\ffrac}[2]{\frac{\text{\footnotesize $#1$}}{\text{\footnotesize $#2$}}}
\newcommand{\vasymptote}[2][]{
    \draw [densely dashed,#1] ({rel axis cs:0,0} -| {axis cs:#2,0}) -- ({rel axis cs:0,1} -| {axis cs:#2,0});
}


\graphicspath{{./}{firstExample/}}

\usepackage{amsmath}
\usepackage{amssymb}
\usepackage{array}
\usepackage[makeroom]{cancel} %% for strike outs
\usepackage{pgffor} %% required for integral for loops
\usepackage{tikz}
\usepackage{tikz-cd}
\usepackage{tkz-euclide}
\usetikzlibrary{shapes.multipart}


\usetkzobj{all}
\tikzstyle geometryDiagrams=[ultra thick,color=blue!50!black]


\usetikzlibrary{arrows}
\tikzset{>=stealth,commutative diagrams/.cd,
  arrow style=tikz,diagrams={>=stealth}} %% cool arrow head
\tikzset{shorten <>/.style={ shorten >=#1, shorten <=#1 } } %% allows shorter vectors

\usetikzlibrary{backgrounds} %% for boxes around graphs
\usetikzlibrary{shapes,positioning}  %% Clouds and stars
\usetikzlibrary{matrix} %% for matrix
\usepgfplotslibrary{polar} %% for polar plots
\usepgfplotslibrary{fillbetween} %% to shade area between curves in TikZ



%\usepackage[width=4.375in, height=7.0in, top=1.0in, papersize={5.5in,8.5in}]{geometry}
%\usepackage[pdftex]{graphicx}
%\usepackage{tipa}
%\usepackage{txfonts}
%\usepackage{textcomp}
%\usepackage{amsthm}
%\usepackage{xy}
%\usepackage{fancyhdr}
%\usepackage{xcolor}
%\usepackage{mathtools} %% for pretty underbrace % Breaks Ximera
%\usepackage{multicol}



\newcommand{\RR}{\mathbb R}
\newcommand{\R}{\mathbb R}
\newcommand{\C}{\mathbb C}
\newcommand{\N}{\mathbb N}
\newcommand{\Z}{\mathbb Z}
\newcommand{\dis}{\displaystyle}
%\renewcommand{\d}{\,d\!}
\renewcommand{\d}{\mathop{}\!d}
\newcommand{\dd}[2][]{\frac{\d #1}{\d #2}}
\newcommand{\pp}[2][]{\frac{\partial #1}{\partial #2}}
\renewcommand{\l}{\ell}
\newcommand{\ddx}{\frac{d}{\d x}}

\newcommand{\zeroOverZero}{\ensuremath{\boldsymbol{\tfrac{0}{0}}}}
\newcommand{\inftyOverInfty}{\ensuremath{\boldsymbol{\tfrac{\infty}{\infty}}}}
\newcommand{\zeroOverInfty}{\ensuremath{\boldsymbol{\tfrac{0}{\infty}}}}
\newcommand{\zeroTimesInfty}{\ensuremath{\small\boldsymbol{0\cdot \infty}}}
\newcommand{\inftyMinusInfty}{\ensuremath{\small\boldsymbol{\infty - \infty}}}
\newcommand{\oneToInfty}{\ensuremath{\boldsymbol{1^\infty}}}
\newcommand{\zeroToZero}{\ensuremath{\boldsymbol{0^0}}}
\newcommand{\inftyToZero}{\ensuremath{\boldsymbol{\infty^0}}}


\newcommand{\numOverZero}{\ensuremath{\boldsymbol{\tfrac{\#}{0}}}}
\newcommand{\dfn}{\textbf}
%\newcommand{\unit}{\,\mathrm}
\newcommand{\unit}{\mathop{}\!\mathrm}
%\newcommand{\eval}[1]{\bigg[ #1 \bigg]}
\newcommand{\eval}[1]{ #1 \bigg|}
\newcommand{\seq}[1]{\left( #1 \right)}
\renewcommand{\epsilon}{\varepsilon}
\renewcommand{\iff}{\Leftrightarrow}

\DeclareMathOperator{\arccot}{arccot}
\DeclareMathOperator{\arcsec}{arcsec}
\DeclareMathOperator{\arccsc}{arccsc}
\DeclareMathOperator{\si}{Si}
\DeclareMathOperator{\proj}{proj}
\DeclareMathOperator{\scal}{scal}
\DeclareMathOperator{\cis}{cis}
\DeclareMathOperator{\Arg}{Arg}
%\DeclareMathOperator{\arg}{arg}
\DeclareMathOperator{\Rep}{Re}
\DeclareMathOperator{\Imp}{Im}
\DeclareMathOperator{\sech}{sech}
\DeclareMathOperator{\csch}{csch}
\DeclareMathOperator{\Log}{Log}

\newcommand{\tightoverset}[2]{% for arrow vec
  \mathop{#2}\limits^{\vbox to -.5ex{\kern-0.75ex\hbox{$#1$}\vss}}}
\newcommand{\arrowvec}{\overrightarrow}
\renewcommand{\vec}{\mathbf}
\newcommand{\veci}{{\boldsymbol{\hat{\imath}}}}
\newcommand{\vecj}{{\boldsymbol{\hat{\jmath}}}}
\newcommand{\veck}{{\boldsymbol{\hat{k}}}}
\newcommand{\vecl}{\boldsymbol{\l}}
\newcommand{\utan}{\vec{\hat{t}}}
\newcommand{\unormal}{\vec{\hat{n}}}
\newcommand{\ubinormal}{\vec{\hat{b}}}

\newcommand{\dotp}{\bullet}
\newcommand{\cross}{\boldsymbol\times}
\newcommand{\grad}{\boldsymbol\nabla}
\newcommand{\divergence}{\grad\dotp}
\newcommand{\curl}{\grad\cross}
%% Simple horiz vectors
\renewcommand{\vector}[1]{\left\langle #1\right\rangle}


\outcome{Compute integrals involving square roots of sums and differences}

\title{Trigonometric Substitution}

\begin{document}

\begin{abstract}
We compute integrals involving square roots of sums and differences.
\end{abstract}

\maketitle

\section{Trig identities}

The following are the Pythagorean Trigonometric Identities 
(named for \link[Pythagoras of Samos]{https://en.wikipedia.org/wiki/Pythagoras}) 
which hold for all angles,
$\theta$, in the domains of the functions involved:
\[
\sin^2(\theta) + \cos^2(\theta) = 1,
\]
\[
1 + \tan^2(\theta) = \sec^2(\theta),
\]
and
\[
1 + \cot^2(\theta) = \csc^2(\theta).
\]
%You can learn more about these well know identities here (link).



\section{Integrands involving $\sqrt{a^2 - x^2}$}

For integrals involving $\sqrt[]{a^2-u^2}$,
\setlength{\parskip}{\baselineskip}

\begin{tikzpicture}[thick]
\coordinate (O) at (0,0);
\coordinate (A) at (4,0);
\coordinate (B) at (0,2);
\coordinate [label=right:$\theta$] (D) at (2.5,.3);
\draw (O)--(A)--(B)--cycle;

\tkzLabelSegment[below=2pt](O,A){\textit{$\sqrt[]{a^2-u^2}$}}
\tkzLabelSegment[left=2pt](O,B){\textit{u}}
\tkzLabelSegment[above right=2pt](A,B){\textit{a}}

\tkzMarkRightAngle(A,O,B)% square angle here

\end{tikzpicture}

Then $\sqrt[]{a^2-u^2}$=$acos\theta$, 
where $\frac{-\pi}{2}$ $\leq$ $\theta$ $\leq$ $\frac{\pi}{2}$.

\begin{example}
Compute the indefinite integral
\[
\int x^3 \sqrt{1-x^2} \; dx
\]
The key to the solution is to replace the variable $x$ with a trigonometric function 
in order to eliminate the square root.
Since $1-\cos^2(\theta)$ and the radicand, $1-x^2$,
have the same algebraic form, we will try the substitution $x = \cos(\theta)$.
In order to apply this to the integral, we also need to compute the differential, 
$dx = -\sin(\theta) \; d\theta$.
Substituting gives,
\[
\int x^3 \sqrt{1-x^2} \; dx = \int \cos^3(\theta) \sqrt{1-\cos^2(\theta)} [-\sin(\theta)] \; d\theta.
\]
Now, the point of this substitution is that the square root can be computed by
using a
Pythagorean trig identity. From there, we can compute the integral
using the skills developed in the section on trig integrals.
We have
\begin{align*}
\int x^3 \sqrt{1-x^2} \; dx &= \int \cos^3(\theta) \sqrt{1-\cos^2(\theta)} [-\sin(\theta)] \; d\theta\\
                           &= -\int \cos^3(\theta) \sqrt{\sin^2(\theta)} \sin(\theta) \; d\theta\\
                           &= -\int \cos^3(\theta) \sin(\theta) \sin(\theta) \; d\theta\\
                           &= -\int \cos^3(\theta) \sin^2(\theta) \; d\theta\\
                           \text{now, u-sub, with} \; & \; \text{  $u = \sin(\theta), \; du = \cos(\theta) \, d\theta$}\\
                           &=-\int \cos^3(\theta) u^2 \; \frac{du}{\cos(\theta)}\\
                           &=-\int \cos^2(\theta) u^2 \; du\\
                           &=-\int (1-\sin^2(\theta)) u^2 \; du\\
                           &=-\int (1-u^2)u^2 \; du\\
                           &=\int (u^2 -1)u^2 \; du\\
                           &=\int (u^4 - u^2) \; du\\
                           \text{now we can} & \text{ integrate (finally)}\\
                           &=\frac{u^5}{5} - \frac{u^3}{3} + C\\
                           &= \frac15 \sin^5(\theta) - \frac13 \sin^3(\theta) + C.
\end{align*}
At this point we have an answer in terms of the variable $\theta$, but we require the variable $x$.
We will use the triangle below to convert back to $x$.

%need triangle showing x = cos(theta)

From the triangle, we see that $\sin(\theta) = \sqrt{1 - x^2}$ and we back-substitute 
to get the final answer:
\[
\int x^3 \sqrt{1-x^2} \; dx = \frac15 \sin^5(\theta) - \frac13 \sin^3(\theta) + C = \frac15 (1-x^2)^{5/2} - \frac13 (1-x^2)^{3/2} + C.
\]

\end{example}

\begin{remark}
Technically, $\sqrt{\sin^2(\theta)} = |\sin(\theta)|$ but in the substitution
$x = \cos(\theta)$, we can choose the angle $\theta$ to be in the interval
$[0, \pi]$ so that $\sin(\theta) \geq 0$ and hence $|\sin(\theta)| = \sin(\theta)$.
\end{remark}



\begin{problem}
Compute the indefinite integral
\[
\int   x^3 \sqrt{16-x^2} \; dx.
\]
Use a trig sub with\\
\[
x = \answer{4\cos(\theta)}
\]
The differential is\\
\[
dx = \answer{-4\sin(\theta)}d\theta
\]
Substituting these and simplifying the integrand gives\\
\[
\int   x^3\sqrt{16 - x^2}\; dx = -16\int \answer{\cos^3(\theta)\sin^2(\theta)} d\theta
\]
Computing this integral gives
\[
\int \cos^3(\theta)\sin^2(\theta) \; d\theta = \answer{-(1/5)\sin^5(\theta) + (1/3)\sin^3(\theta)} + C.
\]
Based on the substitution $x = 4\cos(\theta)$, \\
\[
\sin(\theta) = \answer{\sqrt{16 - x^2}/4}
\]
The final answer is
\[
\int   x^3\sqrt{16 -  x^2}\; dx = 
\]
\begin{center}
\begin{multipleChoice}
\choice{$\frac{1}{80}(16-x^2)^{5/2} + \frac{1}{12}(16 -x^2)^{3/2} + C$}
\choice{$\frac{1}{80}(16-x^2)^{5/2} - \frac{1}{12}(16 -x^2)^{3/2} + C$}
\choice{$\frac{1}{320}(16-x^2)^{5/2} + \frac{1}{12}(16 -x^2)^{3/2} + C$}
\choice[correct]{$\frac{1}{320}(16-x^2)^{5/2} - \frac{1}{12}(16 -x^2)^{3/2} + C$}
\end{multipleChoice}
\end{center}


\end{problem}





























\begin{problem}

Compute the indefinite integral
\[
\int   x^5 \sqrt{4-x^2} \; dx.
\]
Use a trig sub with\\
\[
x = \answer{2\cos(\theta)}
\]
The differential is\\
\[
dx = \answer{-2\sin(\theta)}d\theta
\]
Substituting these and simplifying the integrand gives\\
\[
\int   x^5\sqrt{4 - x^2}\; dx = -128\int \answer{\cos^5(\theta)\sin^2(\theta)} d\theta
\]
Computing this integral gives
\[
\int \cos^5(\theta)\sin^2(\theta) \; d\theta = \answer{-(2/5)\sin^5(\theta) + (1/3)\sin^3(\theta) +\frac17 \sin^7(\theta)} + C.
\]
Based on the substitution $x = 2\cos(\theta)$, \\
\[
\sin(\theta) = \answer{\sqrt{4 - x^2}/2}
\]
The final answer is
\[
\int   x^5\sqrt{4 -  x^2}\; dx = 
\]
\begin{center}
\begin{multipleChoice}
\choice{$\frac17 (4-x^2)^{7/2} - \frac85(4-x^2)^{5/2} + \frac{16}{3} (4-x^2)^{3/2} + C$}
\choice[correct]{$-\frac17 (4-x^2)^{7/2} + \frac85(4-x^2)^{5/2} - \frac{16}{3} (4-x^2)^{3/2} + C$}
\choice{$-\frac17 (4-x^2)^{7/2} + \frac45(4-x^2)^{5/2} - \frac{16}{3} (4-x^2)^{3/2} + C$}
\choice{$\frac17 (4-x^2)^{7/2} - \frac45(4-x^2)^{5/2} + \frac{16}{3} (4-x^2)^{3/2} + C$}
\end{multipleChoice}
\end{center}

\end{problem}























\begin{example}
Compute the indefinite integral
\[
\int \frac{\sqrt{4-x^2}}{x^2} \; dx
\]
We choose the substitution $x = 2\cos(\theta)$ so that $dx = -2\sin(\theta) \, d\theta$.
The integral becomes
\begin{align*}
\int \frac{\sqrt{4-x^2}}{x} \; dx &= \int \frac{\sqrt{4-4\cos^2(\theta)}}{\cos^2(\theta)} [-2\sin(\theta)] \; d\theta\\
                                  &= -2\int \frac{\sqrt{4\sin^2(\theta)}}{\cos^2(\theta)}  \sin(\theta) \; d\theta\\
                                  &= -2\int \frac{2\sin(\theta)}{\cos^2(\theta)}  \sin(\theta) \; d\theta\\
 &= -4\int \frac{\sin^2(\theta)}{\cos^2(\theta)}  \; d\theta\\
 &= -4\int \tan^2(\theta)  \; d\theta\\
&= -4\int [\sec^2(\theta) - 1] \; d\theta\\
&= -4\tan(\theta) + 4\theta + C.
\end{align*}

%need triangle with x = 2 cos(theta)
From the triangle above, we can see that $\tan(\theta) = \frac{\sqrt{4-x^2}}{x}$ so that:
\[
\int \frac{\sqrt{4-x^2}}{x^2} \; dx = -4\tan(\theta) + 4\theta + C = -\frac{4\sqrt{4-x^2}}{x} + 4\cos^{-1}\left(\frac{x}{2}\right) + C. 
\]

\end{example}



\begin{problem}

Compute the indefinite integral
\[
\int \frac{\sqrt{1-9x^2}}{x^2} \; dx.
\]
Use a trig sub with\\
\[
x = \answer{\frac13\cos(\theta)}
\]
The differential is\\
\[
dx = \answer{-\frac13\sin(\theta)}d\theta
\]
Substituting these and simplifying the integrand gives\\
\[
\int \frac{\sqrt{1-9x^2}}{x^2} \; dx = -3\int \answer{\tan^2(\theta)} d\theta
\]
Computing this integral gives
\[
\int \tan^2(\theta) \; d\theta = \answer{\tan(\theta) - \theta} + C.
\]
Based on the substitution $x = \frac13 \cos(\theta)$, \\
\[
\tan(\theta) = \answer{\sqrt{1-9x^2}/(3x)}
\]
The final answer is
\[
\int  \frac{\sqrt{1-9x^2}}{x^2} \; dx = 
\]
\begin{center}
\begin{multipleChoice}
\choice{$-\frac{\sqrt{1-9x^2}}{3x} + \cos^{-1}(3x) + C$}
\choice{$\frac{\sqrt{1-9x^2}}{3x} - \cos^{-1}(3x) + C$}
\choice[correct]{$-\frac{\sqrt{1-9x^2}}{x} + 3\cos^{-1}(3x) + C$}
\choice{$\frac{\sqrt{1-9x^2}}{x} - 3\cos^{-1}(3x) + C$}
\end{multipleChoice}
\end{center}





\end{problem}







\begin{example}
Compute the indefinite integral
\[
\int \frac{1}{x\sqrt{9-4x^2}} \; dx
\]
We choose the substitution $x = \frac32\cos(\theta)$ so that $dx = -\frac32\sin(\theta) \, d\theta$.
The integral becomes
\begin{align*}
\int \frac{1}{x\sqrt{9-4x^2}}\; dx &= \int \frac{1}{\frac32\cos(\theta)\sqrt{9-9\cos^2(\theta)}} \left[-\frac32\sin(\theta)\right]\; d\theta\\
                                  &= -\int \frac{\sin(\theta)}{\cos(\theta)\sqrt{9\sin^2(\theta)}}   \; d\theta\\
                                 &= -\int \frac{\sin(\theta)}{\cos(\theta)\cdot 3\sin(\theta)}   \; d\theta\\
                                  &= -\frac13\int \frac{1}{\cos(\theta)}  \; d\theta\\
                                  &= -\frac13\int \sec(\theta) \; d\theta\\                                 
                                 &= -\frac13 \ln|\sec(\theta) + \tan(\theta)| + C.  
\end{align*}

%need triangle with x = 3/2 cos(theta)
From the triangle above, we can see that $\sec(\theta) = \frac{3}{2x}$ and $\tan(\theta) = \frac{\sqrt{9-4x^2}}{2x}$ so that:
\[
\int \frac{1}{x\sqrt{9-4x^2}} \; dx = -\frac13 \ln|\sec(\theta) + \tan(\theta)| + C = 
 - \frac13\ln\bigg|\frac{3 + \sqrt{9-4x^2}}{2x}\bigg| + C.  
\]

\end{example}




\begin{problem}

Compute the indefinite integral
\[
\int \frac{1}{x\sqrt{9-x^2}}  \; dx.
\]
Use a trig sub with\\
\[
x = \answer{3\cos(\theta)}
\]
The differential is\\
\[
dx = \answer{-3\sin(\theta)}\,d\theta
\]
Substituting these and simplifying the integrand gives\\
\[
\int \frac{1}{x\sqrt{9-x^2}}  \; dx = -\frac13\int \answer{\sec(\theta)} d\theta
\]
Computing this integral gives
\[
\int \sec(\theta) \; d\theta = \answer{\ln|\sec(\theta) + \tan(\theta)|} + C.
\]
Based on the substitution $x = 3 \cos(\theta)$, \\
\[
\sec(\theta) = \answer{\frac{3}{x}} \; \text{ and } \; \tan(\theta) = \answer{\sqrt{9-x^2}/(x)}
\]
The final answer is
\[
\int  \frac{1}{x\sqrt{9-x^2}}  \; dx = 
\]
\begin{center}
\begin{multipleChoice}
\choice[correct]{$-\frac13 \ln\bigg|\frac{3}{x} + \frac{\sqrt{9-x^2}}{x}\bigg| + C$}
\choice{$\frac13 \ln\bigg|\frac{3}{x} + \frac{\sqrt{9-x^2}}{x}\bigg| + C$}
\choice{$- \ln\bigg|\frac{3}{x} + \frac{\sqrt{9-x^2}}{x}\bigg| + C$}
\choice{$\ln\bigg|\frac{3}{x} + \frac{\sqrt{9-x^2}}{x}\bigg| + C$}
\end{multipleChoice}
\end{center}

\end{problem}






\begin{problem}

Compute the indefinite integral
\[
\int \frac{1}{x\sqrt{4-25x^2}}  \; dx.
\]
Use a trig sub with\\
\[
x = \answer{\frac25\cos(\theta)}
\]
The differential is\\
\[
dx = \answer{-\frac25\sin(\theta)}\,d\theta
\]
Substituting these and simplifying the integrand gives\\
\[
\int \frac{1}{x\sqrt{4-25x^2}}  \; dx = -\frac12\int \answer{\sec(\theta)} d\theta
\]
Computing this integral gives
\[
\int \sec(\theta) \; d\theta = \answer{\ln|\sec(\theta) + \tan(\theta)|} + C.
\]
Based on the substitution $x = \frac25 \cos(\theta)$, \\
\[
\sec(\theta) = \answer{\frac{2}{5x}} \; \text{ and } \; \tan(\theta) = \answer{\sqrt{4-25x^2}/(5x)}
\]
The final answer is
\[
\int  \frac{1}{x\sqrt{4-25x^2}} \; dx = 
\]
\begin{center}
\begin{multipleChoice}
\choice{$\frac12 \ln\bigg|\frac{2}{5x} + \frac{\sqrt{4-25x^2}}{5x}\bigg| + C$}
\choice[correct]{$-\frac12 \ln\bigg|\frac{2}{5x} + \frac{\sqrt{4-25x^2}}{5x}\bigg| + C$}
\choice{$- \ln\bigg|\frac{2}{5x} + \frac{\sqrt{4-25x^2}}{5x}\bigg| + C$}
\choice{$ \ln\bigg|\frac{2}{5x} + \frac{\sqrt{4-25x^2}}{5x}\bigg| + C$}
\end{multipleChoice}
\end{center}


\end{problem}

















\section{Integrands involving $\sqrt{x^2 - a^2}$}

\begin{example}
Compute the indefinite integral
\[
\int \frac{1}{\sqrt{x^2-1}} \; dx.
\]
We will use the substitution $x = \sec(\theta)$,
so that $dx = \sec(\theta)\tan(\theta) \, d\theta$.

The integral becomes
\begin{align*}
\int \frac{1}{\sqrt{x^2-1}}\; dx &= \int \frac{\sec(\theta)\tan(\theta)}{\sqrt{\sec^2(\theta)-1}}\; d\theta\\
                                 &= \int \frac{\sec(\theta)\tan(\theta)}{\sqrt{\tan^2(\theta)}}\; d\theta\\
                                 &= \int \frac{\sec(\theta)\tan(\theta)}{\tan(\theta)}\; d\theta\\
                                  &= \int \sec(\theta)\; d\theta\\
                                  &= \ln|\sec(\theta)+ \tan(\theta)| + C.
\end{align*}


%need triangle with x = sec(theta)
From the triangle above, we can see that $\sec(\theta) = x$ and $\tan(\theta) = \sqrt{x^2-1}$ so that:
\[
\int \frac{1}{\sqrt{x^2-1}} \; dx = \ln|\sec(\theta)+ \tan(\theta)| + C = \ln|x+\sqrt{x^2-1}| + C.
\]

\end{example}

\begin{remark}
Technically, $\sqrt{\tan^2(\theta)} = |\tan(\theta)|$, but in the substitution $x = \sec(\theta)$ we 
choose the angle $\theta$ to be in the interval 
$[0, \pi/2)$ or $[\pi, 3\pi/2)$ so that $\tan(\theta) \geq 0$ and hence $|\tan(\theta)| = \tan(\theta)$.
\end{remark}






\begin{example}
Compute the indefinite integral
\[
\int x^3\sqrt{x^2-4} \; dx.
\]
We will use the substitution $x = 2\sec(\theta)$,
so that $dx = 2\sec(\theta)\tan(\theta) \, d\theta$.

The integral becomes
\begin{align*}
\int x^3\sqrt{x^2-4}\; dx &= \int \sec^3(\theta)\sqrt{4\sec^2(\theta)-4}\cdot 2\sec(\theta)\tan(\theta)\, d\theta\\
                           &=  2\int \sec^4(\theta)\sqrt{4\tan^2(\theta)}\cdot \tan(\theta)\, d\theta\\   
                            &=  2\int \sec^4(\theta)\cdot2\tan(\theta)\cdot \tan(\theta)\, d\theta\\ 
                            &=  4\int \sec^4(\theta)\tan^2(\theta)\, d\theta\\ 
                            &=  4\int \sec^2(\theta)\cdot \sec^2(\theta) \cdot \tan^2(\theta)\, d\theta\\  
                            &=  4\int \sec^2(\theta)\cdot [1 + \tan^2(\theta)] \cdot \tan^2(\theta)\, d\theta\\  
                            \text{let } \; & u = \tan(\theta), \; du = \sec^2(\theta) \, d\theta\\
                            &=  4\int [1 + u^2]u^2\, du\\ 
                             &=  4\int (u^2 + u^4) \, du\\ 
                             &= \frac43 u^3 + \frac45 u^5 + C\\
                             &= \frac43 \tan^3(\theta) + \frac45 \tan^5(\theta) + C                            
\end{align*}


%need triangle with x = 2sec(theta)
From the triangle above, we can see that $\tan(\theta) = \frac{\sqrt{x^2-4}}{2}$ so that:
\[
\int x^3\sqrt{x^2-4} \; dx =  \frac43 \tan^3(\theta) + \frac45 \tan^5(\theta) + C =  \frac16 (x^2 - 4)^{3/2} + 
\frac{1}{40} (x^2 - 4)^{5/2} + C.
\]

\end{example}






\begin{example}
Compute the indefinite integral
\[
\int \frac{1}{x\sqrt{x^2-16}} \; dx.
\]
We will use the substitution $x = 4\sec(\theta)$,
so that $dx = 4\sec(\theta)\tan(\theta) \, d\theta$.

The integral becomes
\begin{align*}
\int \frac{1}{x\sqrt{x^2-16}}\; dx &= \int \frac{4\sec(\theta)\tan(\theta)}{4\sec(\theta)\sqrt{16\sec^2(\theta)-16}}\; d\theta\\
                                 &=  \int \frac{\tan(\theta)}{\sqrt{16\tan^2(\theta)}}\; d\theta\\
                                 &=  \int \frac{\tan(\theta)}{4\tan(\theta)}\; d\theta\\
                                  &=  \int \frac14 \; d\theta\\
                                  &= \frac14 \theta + C\\
                                  &= \frac14 \sec^{-1}\left(\frac{x}{4}\right)+C.
\end{align*}

\end{example}



\begin{example}
Compute the indefinite integral
\[
\int \frac{1}{x^2\sqrt{9x^2-4}} \; dx.
\]
We will use the substitution $x = \frac23\sec(\theta)$,
so that $dx = \frac23\sec(\theta)\tan(\theta) \, d\theta$.

The integral becomes
\begin{align*}
\int \frac{1}{x^2\sqrt{9x^2-4}}\; dx &= \int \frac{(2/3)\sec(\theta)\tan(\theta)}{(2/3)^2\sec^2(\theta)\sqrt{4\sec^2(\theta)-4}}\; d\theta\\
                                 &=  \frac32\int \frac{\tan(\theta)}{\sec(\theta)\sqrt{4\tan^2(\theta)}}\; d\theta\\
                                 &=  \frac32\int \frac{\tan(\theta)}{2\sec(\theta)\tan(\theta)}\; d\theta\\
                                  &=  \frac34 \int \frac{1}{\sec(\theta)} \; d\theta\\
                                  &= \frac34 \int \cos(\theta)  \; d\theta\\
                                  &= \frac34 \sin(\theta) + C.
\end{align*}
%need triangle with x = 2/3 sec(theta)
From the triangle above, we can see that $\sin(\theta) = \frac{\sqrt{9x^2-4}}{3x}$ so that:
\[
\int \frac{1}{x^2\sqrt{9x^2-4}} \; dx =  \frac34 \sin(\theta) + C =  \frac{\sqrt{9x^2 - 4}}{4x} + C.
\]

\end{example}


\begin{example}
Compute the indefinite integral
\[
\int \frac{1}{x^3\sqrt{x^2-25}} \; dx.
\]
We will use the substitution $x = 5\sec(\theta)$,
so that $dx = 5\sec(\theta)\tan(\theta) \, d\theta$.

The integral becomes
\begin{align*}
\int \frac{1}{x^3\sqrt{x^2-25}}\; dx &= \int \frac{5\sec(\theta)\tan(\theta)}{5^3\sec^3(\theta)\sqrt{25\sec^2(\theta)-25}}\; d\theta\\
                                 &=  \frac{1}{25}\int \frac{\tan(\theta)}{\sec^2(\theta)\sqrt{25\tan^2(\theta)}}\; d\theta\\
                                 &=  \frac{1}{25}\int \frac{\tan(\theta)}{5\sec^2(\theta)\tan(\theta)}\; d\theta\\
                                  &=  \frac{1}{125} \int \frac{1}{\sec^2(\theta)} \; d\theta\\
                                  &= \frac{1}{125} \int \cos^2(\theta)  \; d\theta\\
                                  &= \frac{1}{125} \int \frac{1+\cos(2\theta)}{2} \; d\theta\\
                                  &= \frac{1}{250}\left[\theta + \frac12 \sin(2\theta)\right] + C\\
                                  &= \frac{1}{250} \theta + \frac{1}{500} \cdot 2\sin(\theta)\cos(\theta) + C.
\end{align*}
%need triangle with x = 5 sec(theta)
From the triangle above, we can see that $\sin(\theta) = \frac{\sqrt{x^2-25}}{x}$ so that:
\begin{align*}
\int \frac{1}{x^3\sqrt{x^2-25}} \; dx &=  \frac{\theta}{250}  + \frac{\sin(\theta)\cos(\theta)}{250} + C \\
&= \frac{1}{250} \sec^{-1}\left(\frac{x}{5}\right) + \frac{1}{250} \cdot \frac{\sqrt{x^2 - 25}}{x}\cdot \frac{5}{x} + C\\
&= \frac{1}{250} \sec^{-1}\left(\frac{x}{5}\right) + \frac{\sqrt{x^2 - 25}}{50x^2}+ C.
\end{align*}

\end{example}




\begin{problem}
Compute the following indefinite integrals:
\begin{itemize}
\item $\displaystyle{\int   \frac{1}{\sqrt{x^2 - 4}} \; dx}$
\item $\displaystyle{\int   x^3 \sqrt{x^2 - 9} \; dx}$
\item $\displaystyle{\int \frac{1}{x\sqrt{4x^2 - 1}}\; dx}$
\item $\displaystyle{\int \frac{1}{x^2\sqrt{25x^2-16}} \; dx}$
\item $\displaystyle{\int \frac{1}{x^3\sqrt{x^2-1}} \; dx}$
\end{itemize}

\end{problem}





\section{Integrands involving $\sqrt{a^2 + x^2}$}

\begin{example}
Compute the indefinite integral:
\[
\int x^2\sqrt{1+x^2} \; dx.
\]
We will use the substitution $x = \tan(\theta)$.  Then the differential, 
$dx = \sec^2(\theta) \, d\theta$.  Substituting these gives
\begin{align*}
\int x^3\sqrt{1+x^2} \; dx &= \int \tan^3(\theta) \sqrt{1+\tan^2(\theta)} \cdot \sec^2(\theta) \; d\theta\\
                           &= \int \tan^3(\theta) \sqrt{\sec^2(\theta)} \cdot \sec^2(\theta) \; d\theta\\
                           &= \int \tan^3(\theta) \cdot \sec(\theta) \cdot \sec^2(\theta) \; d\theta\\
                           &= \int \tan^3(\theta) \sec^3(\theta) \; d\theta\\
                           \text{u-sub with } \; & u=\sec(\theta), \, du = \sec(\theta) \tan(\theta) d\theta\\
                           &= \int \tan^2(\theta) \sec^2(\theta) \cdot \sec(\theta) \tan(\theta)\; d\theta\\
                           &= \int \left(\sec^2(\theta)-1 \right) \sec^2(\theta) \cdot \sec(\theta) \tan(\theta)\; d\theta\\ 
                           &= \int (u^2 -1)u^2 \; du\\ 
                           &= \int (u^4 -u^2) \; du\\
                           &\text{now we can integrate}\\
                           &= \frac{u^5}{5} - \frac{u^3}{3} + C\\
                           &= \frac15 \sec^5(\theta)  - \frac13 \sec^3(\theta) + C.
\end{align*}

%need triangle with x = tan(theta)

From the triangle above, we see that $\sec(\theta)= \sqrt{1+x^2}$.  Back-substituting gives
\[
\int x^3\sqrt{1+x^2} \; dx = \frac15 \sec^5(\theta)  - \frac13 \sec^3(\theta) + C = \frac15 (1+x^2)^{5/2}  - \frac13 (1+x^2)^{3/2} + C
\]

\end{example}


\begin{remark}
Technically, $\sqrt{\sec^2(\theta)} = |\sec(\theta)|$, but in our substitution, $x = \tan(\theta)$,
we can specify that the angle $\theta$ be in the interval $(-\pi/2, \pi/2)$ so that 
$\sec(\theta) > 0$ and $\sqrt{\sec^2(\theta)} = |\sec(\theta)| = \sec(\theta)$.
\end{remark}




\begin{problem}
Compute the indefinite integral
\[
\int   x^3\sqrt{9 + x^2}\; dx.
\]
Use a trig sub with\\
\[
x = \answer{3\tan(\theta)}
\]
The differential is\\
\[
dx = \answer{3\sec^2(\theta)}d\theta
\]
Substituting these and simplifying the integrand gives\\
\[
\int   x^3\sqrt{9 + x^2}\; dx = 81\int \answer{\tan^3(\theta)\sec^3(\theta)} d\theta
\]
Computing this integral gives
\[
\int \tan^3(\theta)\sec^3(\theta) \; d\theta = \answer{(1/5)\sec^5(\theta) - (1/3)\sec^3(\theta)} + C.
\]
Based on the substitution $x = 3\tan(\theta)$, \\
\[
\sec(\theta) = \answer{\sqrt{9+x^2}/3}
\]
The final answer is
\[
\int   x^3\sqrt{9 + x^2}\; dx = 
\]
\begin{center}
\begin{multipleChoice}
\choice[correct]{$\frac{1}{15}(9+x^2)^{5/2} - (9+x^2)^{3/2} + C$}
\choice{$\frac{1}{15}(9+x^2)^{5/2} + (9+x^2)^{3/2} + C$}
\choice{$\frac{1}{15}(9+x^2)^{3/2} - (9+x^2)^{5/2} + C$}
\choice{$\frac{1}{15}(9+x^2)^{3/2} + (9+x^2)^{5/2} + C$}
\end{multipleChoice}
\end{center}
\end{problem}





\begin{example}
Compute the indefinite integral
\[
\int \frac{x^3}{\sqrt{9 + x^2}} \; dx.
\]
We will let $x = 3\tan(\theta)$. Then the differential
$dx = 3\sec^2(\theta) \, d\theta$.  Substituting these, we get
\begin{align*}
\int \frac{x^3}{\sqrt{9 + x^2}} \; dx &= \int \frac{27\tan^3(\theta)}{\sqrt{9 + 9\tan^2(\theta)}} \cdot 3\sec^2(\theta) \; d\theta\\
                                      &= \int \frac{27\tan^3(\theta)}{\sqrt{9\sec^2(\theta)}} \cdot 3\sec^2(\theta) \; d\theta\\
                                      &= \int \frac{27\tan^3(\theta)}{3\sec(\theta)} \cdot 3\sec^2(\theta) \; d\theta\\
                                      &= 27\int \tan^3(\theta) \sec(\theta) \; d\theta\\
                                      & \text{let} \; u = \sec(\theta), \, du = \sec(\theta) \tan(\theta) \, d\theta\\
                                      &= 27\int \tan^2(\theta) \cdot \sec(\theta) \tan(\theta) \; d\theta\\
                                      &= 27\int \left(\sec^2(\theta)-1\right) \cdot \sec(\theta) \tan(\theta) \; d\theta\\
                                      &= 27\int (u^2-1) \; du\\
                                      &= 9u^3 - \frac{u}{27} + C\\
                                      &= 9\sec^3(\theta) - \frac{1}{27} \sec(\theta) + C.
\end{align*}

%need triangle with x = 3 tan(theta)

From the triangle above, we can see that $\sec(\theta) = \frac{3}{\sqrt{9+x^2}}$ and so

\[
\int \frac{x^3}{\sqrt{9 + x^2}} \; dx = 9\sec^3(\theta) - \frac{1}{27} \sec(\theta) + C = \frac{243}{(9+x^2)^{3/2}} - \frac{1}{9(1+x^2)^{1/2}} + C.
\]
\end{example}



\begin{problem}
Compute the indefinite integral
\[
\int \frac{x^3}{\sqrt{4 + x^2}} \; dx.
\]
Use a trig sub with\\
\[
x = \answer{2\tan(\theta)}
\]
The differential is\\
\[
dx = \answer{2\sec^2(\theta)}d\theta
\]
Substituting these and simplifying the integrand gives\\
\[
\int \frac{x^3}{\sqrt{4 + x^2}} \; dx = 8\int \answer{\tan^3(\theta)\sec(\theta)} d\theta
\]
Computing this integral gives
\[
\int \tan^3(\theta)\sec(\theta) \; d\theta = \answer{(1/3)\sec^3(\theta) - \sec(\theta)} + C.
\]
Based on the substitution $x = 2\tan(\theta)$, \\
\[
\sec(\theta) = \answer{\sqrt{4+x^2}/2}
\]
The final answer is
\[
\int  \frac{x^3}{\sqrt{4 + x^2}} \; dx = 
\]
\begin{center}
\begin{multipleChoice}
\choice{$\frac{1}{3}(4+x^2)^{3/2} + 4(4+x^2)^{1/2} + C$}
\choice[correct]{$\frac{1}{3}(4+x^2)^{3/2} - 4(4+x^2)^{1/2} + C$}
\choice{$\frac{1}{3}(4+x^2)^{1/2} + 4(4+x^2)^{3/2} + C$}
\choice{$\frac{1}{3}(4+x^2)^{1/2} - 4(4+x^2)^{3/2} + C$}
\end{multipleChoice}
\end{center}
\end{problem}







\begin{example}
Compute the indefinite integral
\[
\int \sqrt{16 + 9x^2} \; dx.
\]
Let $x = \frac43 \tan(\theta)$. Then $dx = \frac43 \sec^2(\theta) \, d\theta$.
Substituting these, the integral becomes,
\begin{align*}
\int \sqrt{16 + 9x^2} \; dx &= \int \sqrt{16+16\tan^2(\theta)} \cdot \sec^2(\theta) \; d\theta\\
                            &= \int \sqrt{16\sec^2(\theta)} \cdot \sec^2(\theta) \; d\theta\\
                            &= \int 4\sec(\theta) \cdot \sec^2(\theta) \; d\theta\\
                            &= 4\int \sec^3(\theta) \; d\theta\\
                            &\;\;\;\;\;\text{(see the IBP section for this integral).}\\
                            &= 2 \sec(x) \tan(x) + 2 \ln|\sec(x) + \tan(x)| + C
\end{align*}
                            
%need triangle with x = 4/3 tan(theta)

From the triangle above, we can see that $\sec(\theta) = \frac{\sqrt{16+9x^2}}{4 }$. Back substitution gives
\begin{align*}
\int \sqrt{16 + 9x^2} \; dx &= 2 \sec(x) \tan(x) + 2 \ln|\sec(x) + \tan(x)| + C\\
                            &=  2 \cdot\frac{\sqrt{16+9x^2}}{4 } \cdot \frac{3x}{4}  + 2 \ln\bigg|\frac{\sqrt{16+9x^2}}{4 } + \frac{3x}{4}\bigg| + C\\
                            &=   \frac38 x\sqrt{16+9x^2}  + 2 \ln\left(3x +  \sqrt{16+9x^2}\right) + C\\
\end{align*}
Note that the absolute value bars were removed because $3x +  \sqrt{16+9x^2} > 0$ for any $x$. Also note that the $1/4$ was removed from inside the log 
because $\ln(u/v) = \ln(u) - \ln(v)$ and $2\ln(1/4)$ can become part of the constant of integration.
\end{example}




\begin{problem}
Compute the indefinite integral
\[
\int \sqrt{9 + 49x^2} \; dx.
\]
Use a trig sub with\\
\[
x = \answer{(3/7)\tan(\theta)}
\]
The differential is\\
\[
dx = \answer{(3/7)\sec^2(\theta)}d\theta
\]
Substituting these and simplifying the integrand gives\\
\[
\int \sqrt{9 + 49x^2} \; dx = \frac97 \int \answer{\sec^3(\theta)} d\theta
\]
Computing this integral gives
\[
\int \sec^3(\theta) \; d\theta = \answer{(1/2)\sec(\theta) \tan(\theta) + (1/2)\ln|\sec(\theta)+\tan(\theta)|} + C.
\]
Based on the substitution $x = \frac37 \tan(\theta)$, \\
\[
\sec(\theta) = \answer{\sqrt{9+49x^2}/3}
\]
The final answer is
\[
\int  \frac{x^3}{\sqrt{4 + x^2}} \; dx = 
\]
\begin{center}
\begin{multipleChoice}
\choice{$\frac12 x\sqrt{9+49x^2} + \frac{3}{14}\ln\left(7x + \sqrt{9+49x^2}\right) + C$}
\choice{$\frac12 x\sqrt{9+49x^2} - \frac{9}{14}\ln\left(7x + \sqrt{9+49x^2}\right) + C$}
\choice[correct]{$\frac12 x\sqrt{9+49x^2} + \frac{9}{14}\ln\left(7x+\sqrt{9+49x^2}\right)  + C$}
\choice{$\frac12 x\sqrt{9+49x^2} - \frac{3}{14}\ln|7x - \sqrt{9+49x^2}|  + C$}
\end{multipleChoice}
\end{center}
\end{problem}




\begin{problem}
Compute the following indefinite integrals:
\begin{itemize}
\item $\displaystyle{\int   x^3\sqrt{9 + x^2}\; dx}$
\item $\displaystyle{\int   \frac{x^3}{\sqrt{4 + x^2}} \; dx}$
\item $\displaystyle{\int \sqrt{9 + 49x^2} \; dx}$
%\item $\displaystyle{\int \frac{1}{x^2\sqrt{25x^2-16}} \; dx}$
%\item $\displaystyle{\int \frac{1}{x^3\sqrt{x^2-1}} \; dx}$
\end{itemize}

\end{problem}



\begin{example}
Compute the indefinite integral
\[
\int\frac{\sqrt{1+x^2}}{x}\; dx.
\]
Let $x = \tan(\theta)$, then $dx = \sec^2(\theta) \, d\theta$.
\end{example}

\begin{example}
Compute the indefinite integral
\[
\int\frac{1}{\sqrt{1+x^2}}\; dx.
\]
Let $x = \tan(\theta)$, then $dx = \sec^2(\theta) \, d\theta$.
\end{example}





\begin{center}
\begin{foldable}
\unfoldable{Here is a detailed, lecture style video on trigonometric substitution:}
\youtube{mVQ3ce08VzI&t}
\end{foldable}
\end{center}





\end{document}














\begin{remark}
In the previous example, we claimed that
\[
\sqrt{\sin^2(\theta)} = \sin(\theta).
\]
However, in general,
\[
\sqrt{\sin^2(\theta)} = |\sin(\theta)|.
\]
To resolve this difference, we go back to the original substitution
$x = \cos(\theta)$ and we add the additional stipulation
$0 \leq \theta \leq \pi$. This interval allows $\cos(\theta)$ to 
take on all of its values from $-1$ to $1$ with the additional benefit,
that for values of $\theta$ in this interval, $\sin(\theta) \geq 0$.
This last fact is what allows us to drop the absolute value bars and write
\[
\sqrt{\sin^2(\theta)} = \sin(\theta).
\]
\end{remark}






\begin{example} %example #15
Find $h'(x)$ if $h(x) = x^{\sin(x)}$.\\
We will use the fact that the exponential and logarithm functions are inverses,
\[e^{\ln(x)} = x,\]
and the exponent property of logarithms, 
\[\ln(x^n) = n\ln(x),\]
to rewrite $h(x)$.  We have 
\[h(x) = x^{\sin(x)} = e^{\ln(x^{\sin(x)})} = e^{\sin(x)\ln(x)}\]
and we can now compute $h'(x)$ using a combination of the chain rule and product rule.
We can write $h(x)$ as a composition, $f(g(x))$ with 
\[g(x) = \sin(x)\ln(x) \quad \text{and} \quad f(x) = e^x.\]
Then to find $g'(x)$ we us the product rule and we get $g'(x) = \frac{\sin(x)}{x} + \cos(x)\ln(x)$.
Next $f'(x) = e^x$ and 
hence $f'(g(x)) = e^{g(x)} = e^{\sin(x)\ln(x)} = x^{\sin(x)}$.
We can then conclude $h'(x) = f'(g(x))g'(x) = x^{\sin(x)} \left[ \frac{\sin(x)}{x} + \cos(x)\ln(x)\right]$.
\end{example}

%more question formats below













%\begin{verbatim}
\begin{question}
What is your favorite color?
\begin{multipleChoice}
\choice[correct]{Rainbow}
\choice{Blue}
\choice{Green}
\choice{Red}
\end{multipleChoice}
\begin{freeResponse}
Hello
\end{freeResponse}
\end{question}
%\end{verbatim}





\begin{question}
  Which one will you choose?
  \begin{multipleChoice}
    \choice[correct]{I'm correct.}
    \choice{I'm wrong.}
    \choice{I'm wrong too.}
  \end{multipleChoice}
\end{question}


\begin{question}
  Which one will you choose?
  \begin{selectAll}
    \choice[correct]{I'm correct.}
    \choice{I'm wrong.}
    \choice[correct]{I'm also correct.}
    \choice{I'm wrong too.}
  \end{selectAll}
\end{question}


\begin{freeResponse}
What is the chain rule used for?
\end{freeResponse}
