\documentclass{ximera}

%% You can put user macros here
%% However, you cannot make new environments



\newcommand{\ffrac}[2]{\frac{\text{\footnotesize $#1$}}{\text{\footnotesize $#2$}}}
\newcommand{\vasymptote}[2][]{
    \draw [densely dashed,#1] ({rel axis cs:0,0} -| {axis cs:#2,0}) -- ({rel axis cs:0,1} -| {axis cs:#2,0});
}


\graphicspath{{./}{firstExample/}}

\usepackage{amsmath}
\usepackage{amssymb}
\usepackage{array}
\usepackage[makeroom]{cancel} %% for strike outs
\usepackage{pgffor} %% required for integral for loops
\usepackage{tikz}
\usepackage{tikz-cd}
\usepackage{tkz-euclide}
\usetikzlibrary{shapes.multipart}


\usetkzobj{all}
\tikzstyle geometryDiagrams=[ultra thick,color=blue!50!black]


\usetikzlibrary{arrows}
\tikzset{>=stealth,commutative diagrams/.cd,
  arrow style=tikz,diagrams={>=stealth}} %% cool arrow head
\tikzset{shorten <>/.style={ shorten >=#1, shorten <=#1 } } %% allows shorter vectors

\usetikzlibrary{backgrounds} %% for boxes around graphs
\usetikzlibrary{shapes,positioning}  %% Clouds and stars
\usetikzlibrary{matrix} %% for matrix
\usepgfplotslibrary{polar} %% for polar plots
\usepgfplotslibrary{fillbetween} %% to shade area between curves in TikZ



%\usepackage[width=4.375in, height=7.0in, top=1.0in, papersize={5.5in,8.5in}]{geometry}
%\usepackage[pdftex]{graphicx}
%\usepackage{tipa}
%\usepackage{txfonts}
%\usepackage{textcomp}
%\usepackage{amsthm}
%\usepackage{xy}
%\usepackage{fancyhdr}
%\usepackage{xcolor}
%\usepackage{mathtools} %% for pretty underbrace % Breaks Ximera
%\usepackage{multicol}



\newcommand{\RR}{\mathbb R}
\newcommand{\R}{\mathbb R}
\newcommand{\C}{\mathbb C}
\newcommand{\N}{\mathbb N}
\newcommand{\Z}{\mathbb Z}
\newcommand{\dis}{\displaystyle}
%\renewcommand{\d}{\,d\!}
\renewcommand{\d}{\mathop{}\!d}
\newcommand{\dd}[2][]{\frac{\d #1}{\d #2}}
\newcommand{\pp}[2][]{\frac{\partial #1}{\partial #2}}
\renewcommand{\l}{\ell}
\newcommand{\ddx}{\frac{d}{\d x}}

\newcommand{\zeroOverZero}{\ensuremath{\boldsymbol{\tfrac{0}{0}}}}
\newcommand{\inftyOverInfty}{\ensuremath{\boldsymbol{\tfrac{\infty}{\infty}}}}
\newcommand{\zeroOverInfty}{\ensuremath{\boldsymbol{\tfrac{0}{\infty}}}}
\newcommand{\zeroTimesInfty}{\ensuremath{\small\boldsymbol{0\cdot \infty}}}
\newcommand{\inftyMinusInfty}{\ensuremath{\small\boldsymbol{\infty - \infty}}}
\newcommand{\oneToInfty}{\ensuremath{\boldsymbol{1^\infty}}}
\newcommand{\zeroToZero}{\ensuremath{\boldsymbol{0^0}}}
\newcommand{\inftyToZero}{\ensuremath{\boldsymbol{\infty^0}}}


\newcommand{\numOverZero}{\ensuremath{\boldsymbol{\tfrac{\#}{0}}}}
\newcommand{\dfn}{\textbf}
%\newcommand{\unit}{\,\mathrm}
\newcommand{\unit}{\mathop{}\!\mathrm}
%\newcommand{\eval}[1]{\bigg[ #1 \bigg]}
\newcommand{\eval}[1]{ #1 \bigg|}
\newcommand{\seq}[1]{\left( #1 \right)}
\renewcommand{\epsilon}{\varepsilon}
\renewcommand{\iff}{\Leftrightarrow}

\DeclareMathOperator{\arccot}{arccot}
\DeclareMathOperator{\arcsec}{arcsec}
\DeclareMathOperator{\arccsc}{arccsc}
\DeclareMathOperator{\si}{Si}
\DeclareMathOperator{\proj}{proj}
\DeclareMathOperator{\scal}{scal}
\DeclareMathOperator{\cis}{cis}
\DeclareMathOperator{\Arg}{Arg}
%\DeclareMathOperator{\arg}{arg}
\DeclareMathOperator{\Rep}{Re}
\DeclareMathOperator{\Imp}{Im}
\DeclareMathOperator{\sech}{sech}
\DeclareMathOperator{\csch}{csch}
\DeclareMathOperator{\Log}{Log}

\newcommand{\tightoverset}[2]{% for arrow vec
  \mathop{#2}\limits^{\vbox to -.5ex{\kern-0.75ex\hbox{$#1$}\vss}}}
\newcommand{\arrowvec}{\overrightarrow}
\renewcommand{\vec}{\mathbf}
\newcommand{\veci}{{\boldsymbol{\hat{\imath}}}}
\newcommand{\vecj}{{\boldsymbol{\hat{\jmath}}}}
\newcommand{\veck}{{\boldsymbol{\hat{k}}}}
\newcommand{\vecl}{\boldsymbol{\l}}
\newcommand{\utan}{\vec{\hat{t}}}
\newcommand{\unormal}{\vec{\hat{n}}}
\newcommand{\ubinormal}{\vec{\hat{b}}}

\newcommand{\dotp}{\bullet}
\newcommand{\cross}{\boldsymbol\times}
\newcommand{\grad}{\boldsymbol\nabla}
\newcommand{\divergence}{\grad\dotp}
\newcommand{\curl}{\grad\cross}
%% Simple horiz vectors
\renewcommand{\vector}[1]{\left\langle #1\right\rangle}


\outcome{Compute derivatives using the constant multiple, sum and difference rules}

\title{2.2 Basic Differentiation Rules}

\begin{document}

\begin{abstract}
The following rules allow us the find the derivative of multiples, sums and differences of functions 
whose derivatives are already known.
\end{abstract}


\maketitle

\youtube{mzOBlH32qdk}


\begin{center}
\bf{The Constant Multiple Rule}
\end{center}


\begin{theorem} If $f(x)$ is differentiable and $c$ is any constant, then
\[[cf(x)]' = c f'(x)\]
\end{theorem}

In words, the derivative of a constant times a function is the constant times the derivative of the function.\\



\begin{example}[example 1]
 If $f(x) = 5x^2$ then  we use the constant multiple rule with $c = 5$ and we get 
\[
f'(x) = 5(2x)  = 10x.
\]
\end{example}


\begin{problem}(problem 1a)
  Compute 
  \[
  \frac{d}{dx} \left(4x^2\right)
  \]
  
    \begin{hint}
      The derivative of $x^2$ is $2x$
    \end{hint}    
		The derivative of $4x^2$ with respect to $x$ is
		 $\answer[given]{8x}$
	
\end{problem}



\begin{problem}(problem 1b)
  Compute 
  \[
  \frac{d}{dx} \left(5x^3\right)
  \]
  
    \begin{hint}
      Use the Power Rule on $x^3$
    \end{hint}
    \begin{hint}
      The Power Rule says:
      \[
      \frac{d}{dx} x^n = nx^{n-1}
      \]
    \end{hint}
		\begin{hint}
		  Don't forget to multiply by $5$
		\end{hint}
		The derivative of $5x^3$ with respect to $x$ is
		 $\answer[given]{15x^2}$
	
\end{problem}



\begin{example}[example 2]
 If $f(x) = \frac{x^3}{9}$ then  we use the constant multiple rule with $c = \frac19$ and we get 
\[
f'(x) = \tfrac19 \cdot 3x^2 
= \frac{x^2}{3}.
\]
\end{example}

\begin{problem}(problem 2a)
  Compute 
  \[
  \frac{d}{dx} \left(\frac{x^2}{2}\right)
  \]
  
    \begin{hint}
      The derivative of $x^2$ is $2x$
    \end{hint}    
		The derivative of $\frac{x^2}{2}$ with respect to $x$ is
		 $\answer[given]{x}$
	
\end{problem}

\begin{problem}(problem 2b)
  Compute 
  \[
  \frac{d}{dx} \left(8x^{1/2}\right)
  \]
  
    \begin{hint}
      Use the Power Rule on $x^{1/2}$
    \end{hint}
    \begin{hint}
      The Power Rule says:
      \[
      \frac{d}{dx} x^n = nx^{n-1}
      \]
    \end{hint}
		\begin{hint}
		  Don't forget to multiply by $8$
		\end{hint}
		The derivative of $8x^{1/2}$ with respect to $x$ is
		 $\answer[given]{4x^{-1/2}}$
	
\end{problem}

\begin{problem}(problem 2c)
  Compute 
  \[
  \frac{d}{dx} \left(\frac{x^5}{10}\right)
  \]
  
    \begin{hint}
      Use the Power Rule on $x^5$
    \end{hint}
    \begin{hint}
      The Power Rule says:
      \[
      \frac{d}{dx} x^n = nx^{n-1}
      \]
    \end{hint}
		\begin{hint}
		  Don't forget to multiply by $1/10$
		\end{hint}
		
		The derivative of $\frac{x^5}{10}$ with respect to $x$ is
		 $\answer[given]{\frac{x^4}{2}}$
	
\end{problem}


\begin{example}[example 3]
 If $f(x) = \frac{4}{x^2}$ then we rewrite $f(x)$ as $4x^{-2}$  and we use the constant multiple rule with $c = 4$, 
giving 
\[f'(x) = 4 (-2x^{-3}) = -8x^{-3} = -\frac{8}{x^3}.
\]
\end{example}

\begin{problem}(problem 3a)
  Compute 
  \[
  \frac{d}{dx} \left(\frac{10}{x^5}\right)
  \]
  
    \begin{hint}
		  Rewrite using $\frac{1}{x^n} = x^{-n}$
		\end{hint}
		\begin{hint}
      Use the Power Rule on $x^{-5}$
    \end{hint}
    \begin{hint}
      The Power Rule says:
      \[
      \frac{d}{dx} x^n = nx^{n-1}
      \]
    \end{hint}
		\begin{hint}
		  Don't forget to multiply by 10
		\end{hint}
		
		The derivative of $\frac{10}{x^5}$ with respect to $x$ is
		 $\answer[given]{-50x^{-6}}$
	
\end{problem}


\begin{problem}(problem 3b)
  Compute 
  \[
  \frac{d}{dx} \left(6\sqrt x\right)
  \]
  
    \begin{hint}
		  Rewrite using $\sqrt[n] x = x^{1/n}$
		\end{hint}
		\begin{hint}
      Use the Power Rule on $x^{1/2}$
    \end{hint}
    \begin{hint}
      The Power Rule says:
      \[
      \frac{d}{dx} x^n = nx^{n-1}
      \]
    \end{hint}
		\begin{hint}
		  Don't forget to multiply by 6
		\end{hint}
		
		The derivative of $6\sqrt x$ with respect to $x$ is
		 $\answer[given]{3x^{-1/2}}$
	
\end{problem}


\begin{theorem} If $f(x)$ and $g(x)$ are differentiable functions then
\[[f(x) + g(x)]' =  f'(x) +  g'(x)\]
\end{theorem}

In words, the derivative of a sum is the sum of the derivatives.


\begin{example}[example 4]
If $f(x) = x^3 + x^2$ then $f'(x) = 3x^2 + 2x$.
\end{example}

\begin{problem}(problem 4a)
  Compute 
  \[
  \frac{d}{dx} (x^5 + x^4)
  \]
  
    \begin{hint}
      Use the Power Rule on each term
    \end{hint}
    \begin{hint}
      The Power Rule says:
      \[
      \frac{d}{dx} x^n = nx^{n-1}
      \]
    \end{hint}    
		The derivative of $x^5 + x^4$ with respect to $x$ is
		 $\answer[given]{5x^4 + 4x^3}$
	
\end{problem}

\begin{problem}(problem 4b)
  Compute 
  \[
  \frac{d}{dx} (x^{1000} + x^{100} + x^{10})
  \]
  
    \begin{hint}
		  The Sum Rule can be used on more than two terms
		\end{hint}
		\begin{hint}
      Use the Power Rule on each term
    \end{hint}
    \begin{hint}
      The Power Rule says:
      \[
      \frac{d}{dx} x^n = nx^{n-1}
      \]
    \end{hint}    
		The derivative of $x^{1000} + x^{100} + x^{10}$ with respect to $x$ is
		 $\answer[given]{1000x^{999} + 100x^{99} + 10x^9}$
	
\end{problem}

\begin{example}[example 5]
 If $f(x) = \frac{1}{x} + \frac{1}{x^2}$ then we rewrite $f(x)$ as $x^{-1} + x^{-2}$ and 
\[f'(x) = -x^{-2} -2x^{-3} = -\frac{1}{x^2} - \frac{2}{x^3} = - \frac{x + 2}{x^3}.\]
\end{example}

\begin{problem}(problem 5a)
  Compute 
  \[
  \frac{d}{dx} \left(\frac{1}{x^3} + \frac{1}{x^4}\right)
  \]
  
    \begin{hint}
		 $\frac{1}{x^n} = x^{-n}$
		\end{hint}
		\begin{hint}
      Use the Power Rule on each term
    \end{hint}
    \begin{hint}
      The Power Rule says:
      \[
      \frac{d}{dx} x^n = nx^{n-1}
      \]
    \end{hint}    
		The derivative of $\frac{1}{x^3} + \frac{1}{x^4}$ with respect to $x$ is
		 $\answer[given]{-3x^{-4} -4x^{-5}}$
	
\end{problem}


\begin{problem}(problem 5b)
  Compute 
  \[
  \frac{d}{dx} \left(\frac{1}{x^5} + \frac{1}{x^6} + \frac{1}{x^7}\right)
  \]
  
    \begin{hint}
		 $\frac{1}{x^n} = x^{-n}$
		\end{hint}
		\begin{hint}
		  The Sum Rule can be used on more than two terms
		\end{hint}
		\begin{hint}
      Use the Power Rule on each term
    \end{hint}
    \begin{hint}
      The Power Rule says:
      \[
      \frac{d}{dx} x^n = nx^{n-1}
      \]
    \end{hint}    
		The derivative of $\frac{1}{x^5} + \frac{1}{x^6} + \frac{1}{x^7}$ with respect to $x$ is
		 $\answer[given]{-5x^{-6} -6x^{-7} - 7x^{-8}}$
	\
\end{problem}



\begin{example}[example 6]
 If $f(x) = \sqrt x + \sqrt[3] x$ then we rewrite $f(x)$ as $x^{1/2} + x^{1/3}$ and
\begin{align*}
f'(x) &= \tfrac12 x^{-1/2} + \tfrac13 x^{-2/3} \\
&= \tfrac12 \cdot \frac{1}{x^{1/2}} + \tfrac13 \cdot \frac{1}{x^{2/3}}\\
&= \frac{1}{2\sqrt x} + \frac{1}{3\sqrt[3] {x^2}}.
\end{align*}
\end{example}


\begin{problem}(problem 6a)
  Compute 
  \[
  \frac{d}{dx} \left(\sqrt[4] x + \sqrt[5] x\right)
  \]
  
    \begin{hint}
		 $\sqrt[n] x = x^{1/n}$
		\end{hint}
		\begin{hint}
      Use the Power Rule on each term
    \end{hint}
    \begin{hint}
      The Power Rule says:
      \[
      \frac{d}{dx} x^n = nx^{n-1}
      \]
    \end{hint}    
		The derivative of $\sqrt[4] x + \sqrt[5] x$ with respect to $x$ is
		 $\answer[given]{\frac14 x^{-3/4} + \frac15 x^{-4/5}}$
	
\end{problem}


\begin{problem}(problem 6b)
  Compute 
  \[
  \frac{d}{dx} \left(\sqrt[6] x + \sqrt[7] x + \sqrt[8] x\right)
  \]
  
    \begin{hint}
		 $\sqrt[n] x = x^{1/n}$
		\end{hint}
		\begin{hint}
		  The Sum Rule can be used on more than two terms
		\end{hint}
		\begin{hint}
      Use the Power Rule on each term
    \end{hint}
    \begin{hint}
      The Power Rule says:
      \[
      \frac{d}{dx} x^n = nx^{n-1}
      \]
    \end{hint}    
		The derivative of $\sqrt[6] x + \sqrt[7] x + \sqrt[8] x$ with respect to $x$ is
		 $\answer[given]{\frac16 x^{-5/6} + \frac17 x^{-6/7} + \frac18 x^{-7/8}}$
	
\end{problem}




\begin{theorem} If $f(x)$ and $g(x)$ are differentiable functions, then
\[ [f(x) - g(x)]' =  f'(x) -  g'(x)\]
\end{theorem}

In words, the derivative of a difference is the difference of the derivatives.\\



\begin{example}[example 7]
 If $f(x) = x^5 - x$ then $f'(x) = 5x^4 - 1$.
\end{example}

\begin{problem}(problem 7a)
  Compute 
  \[
  \frac{d}{dx} \left(x^7 - x^4\right)
  \]
  
    \begin{hint}
      Use the Power Rule on each term
    \end{hint}
    \begin{hint}
      The Power Rule says:
      \[
      \frac{d}{dx} x^n = nx^{n-1}
      \]
    \end{hint}    
		The derivative of $x^7 - x^4$ with respect to $x$ is
		 $\answer[given]{7x^6 - 4x^3}$
	
\end{problem}

\begin{problem}(problem 7b)
  Compute 
  \[
  \frac{d}{dx} \left(x^{1000} - x^{100} - x^{10}\right)
  \]
  
    \begin{hint}
		  The Difference Rule can be used on more than two terms
		\end{hint}
		\begin{hint}
      Use the Power Rule on each term
    \end{hint}
    \begin{hint}
      The Power Rule says:
      \[
      \frac{d}{dx} x^n = nx^{n-1}
      \]
    \end{hint}    
		The derivative of $x^{1000} - x^{100} - x^{10}$ with respect to $x$ is
		 $\answer[given]{1000x^{999} - 100x^{99} - 10x^9}$
	
\end{problem}


\begin{problem}(problem 7c) %problem #1c
  Compute 
  \[
  \frac{d}{dx} \left(4x^4 - 5x^3 + 8x^2 - 15x - 4\right)
  \]
  
    \begin{hint}
		  The Difference Rule can be used on more than two terms
		\end{hint}
		\begin{hint}
		  The Sum, Difference and Constant Multiple Rules are often used together 
		\end{hint}
		\begin{hint}
      Use the Power Rule on each term
    \end{hint}
    \begin{hint}
      The Power Rule says:
      \[
      \frac{d}{dx} x^n = nx^{n-1}
      \]
    \end{hint}    
		The derivative of $4x^4 - 5x^3 + 8x^2 - 15x - 4$ with respect to $x$ is
		 $\answer[given]{16x^3 - 15x^2 + 16x - 15}$
	
\end{problem}


\begin{example}[example 8]
 If $f(x) = \frac{1}{x^3} - \frac{1}{x^4}$ then we rewrite $f(x)$ as $x^{-3} - x^{-4}$ and 
$f'(x) = -3x^{-4} + 4x^{-5} = -\frac{3}{x^4} + \frac{4}{x^5} = \frac{4}{x^5}-\frac{3}{x^4} = 
\frac{4x^4 - 3x^5}{x^9} = \frac{4 - 3x}{x^5}$.
\end{example}


\begin{problem}(problem 8a)
  Compute 
  \[
  \frac{d}{dx} \left(\frac{1}{x^5} - \frac{1}{x}\right)
  \]
  
    \begin{hint}
		 $\frac{1}{x^n} = x^{-n}$
		\end{hint}
		\begin{hint}
      Use the Power Rule on each term
    \end{hint}
    \begin{hint}
      The Power Rule says:
      \[
      \frac{d}{dx} x^n = nx^{n-1}
      \]
    \end{hint}    
		The derivative of $\frac{1}{x^5} - \frac{1}{x}$ with respect to $x$ is
		 $\answer[given]{-5x^{-6} + x^{-2}}$
	
\end{problem}


\begin{problem}(problem 8b)
  Compute 
  \[
  \frac{d}{dx} \left(\frac{1}{x^2} - \frac{1}{x^6} - \frac{1}{x^7}\right)
  \]
  
    \begin{hint}
		 $\frac{1}{x^n} = x^{-n}$
		\end{hint}
		\begin{hint}
		  The Difference Rule can be used on more than two terms
		\end{hint}
		\begin{hint}
      Use the Power Rule on each term
    \end{hint}
    \begin{hint}
      The Power Rule says:
      \[
      \frac{d}{dx} x^n = nx^{n-1}
      \]
    \end{hint}    
		The derivative of $\frac{1}{x^2} - \frac{1}{x^6} - \frac{1}{x^7}$ with respect to $x$ is
		 $\answer[given]{-2x^{-3} + 6x^{-7} + 7x^{-8}}$
	
\end{problem}


\begin{example}[example 9]
 If $f(x) = \sqrt x - \sqrt[4] {x^3}$ then we rewrite $f(x)$ as $x^{1/2} + x^{3/4}$ and
$f'(x) = \tfrac12 x^{-1/2} - \tfrac34 x^{-1/4} = \tfrac12 \cdot \frac{1}{x^{1/2}} - \tfrac34 \cdot \frac{1}{x^{1/4}} 
= \frac{1}{2\sqrt x} - \frac{3}{4\sqrt[4] x}$.
\end{example}


\begin{problem}(problem 9a)
  Compute 
  \[
  \frac{d}{dx} \left(\sqrt[3] x - \sqrt[5] x\right)
  \]
  
    \begin{hint}
		 $\sqrt[n] x = x^{1/n}$
		\end{hint}
		\begin{hint}
      Use the Power Rule on each term
    \end{hint}
    \begin{hint}
      The Power Rule says:
      \[
      \frac{d}{dx} x^n = nx^{n-1}
      \]
    \end{hint}    
		The derivative of $\sqrt[3] x - \sqrt[5] x$ with respect to $x$ is
		 $\answer[given]{\frac13 x^{-2/3} - \frac15 x^{-4/5}}$
	
\end{problem}


\begin{problem}(problem 9b)
  Compute 
  \[
  \frac{d}{dx} (\sqrt[6] {x^5} - \sqrt[7] {x^6} + \sqrt[8] {x^7})
  \]
  
    \begin{hint}
		 $\sqrt[n] x^m = x^{m/n}$
		\end{hint}
		\begin{hint}
		  The Sum and Difference Rules can be used on more than two terms
		\end{hint}
		\begin{hint}
      Use the Power Rule on each term
    \end{hint}
    \begin{hint}
      The Power Rule says:
      \[
      \frac{d}{dx} x^n = nx^{n-1}
      \]
    \end{hint}    
		The derivative of $\sqrt[6] {x^5} - \sqrt[7] {x^6} + \sqrt[8] {x^7}$ with respect to $x$ is
		 $\answer[given]{\frac56 x^{-1/6} - \frac67 x^{-1/7} + \frac78 x^{-1/8}}$
	
\end{problem}


\end{document}


