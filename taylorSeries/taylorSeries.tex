\documentclass{ximera}

%% You can put user macros here
%% However, you cannot make new environments



\newcommand{\ffrac}[2]{\frac{\text{\footnotesize $#1$}}{\text{\footnotesize $#2$}}}
\newcommand{\vasymptote}[2][]{
    \draw [densely dashed,#1] ({rel axis cs:0,0} -| {axis cs:#2,0}) -- ({rel axis cs:0,1} -| {axis cs:#2,0});
}


\graphicspath{{./}{firstExample/}}

\usepackage{amsmath}
\usepackage{amssymb}
\usepackage{array}
\usepackage[makeroom]{cancel} %% for strike outs
\usepackage{pgffor} %% required for integral for loops
\usepackage{tikz}
\usepackage{tikz-cd}
\usepackage{tkz-euclide}
\usetikzlibrary{shapes.multipart}


\usetkzobj{all}
\tikzstyle geometryDiagrams=[ultra thick,color=blue!50!black]


\usetikzlibrary{arrows}
\tikzset{>=stealth,commutative diagrams/.cd,
  arrow style=tikz,diagrams={>=stealth}} %% cool arrow head
\tikzset{shorten <>/.style={ shorten >=#1, shorten <=#1 } } %% allows shorter vectors

\usetikzlibrary{backgrounds} %% for boxes around graphs
\usetikzlibrary{shapes,positioning}  %% Clouds and stars
\usetikzlibrary{matrix} %% for matrix
\usepgfplotslibrary{polar} %% for polar plots
\usepgfplotslibrary{fillbetween} %% to shade area between curves in TikZ



%\usepackage[width=4.375in, height=7.0in, top=1.0in, papersize={5.5in,8.5in}]{geometry}
%\usepackage[pdftex]{graphicx}
%\usepackage{tipa}
%\usepackage{txfonts}
%\usepackage{textcomp}
%\usepackage{amsthm}
%\usepackage{xy}
%\usepackage{fancyhdr}
%\usepackage{xcolor}
%\usepackage{mathtools} %% for pretty underbrace % Breaks Ximera
%\usepackage{multicol}



\newcommand{\RR}{\mathbb R}
\newcommand{\R}{\mathbb R}
\newcommand{\C}{\mathbb C}
\newcommand{\N}{\mathbb N}
\newcommand{\Z}{\mathbb Z}
\newcommand{\dis}{\displaystyle}
%\renewcommand{\d}{\,d\!}
\renewcommand{\d}{\mathop{}\!d}
\newcommand{\dd}[2][]{\frac{\d #1}{\d #2}}
\newcommand{\pp}[2][]{\frac{\partial #1}{\partial #2}}
\renewcommand{\l}{\ell}
\newcommand{\ddx}{\frac{d}{\d x}}

\newcommand{\zeroOverZero}{\ensuremath{\boldsymbol{\tfrac{0}{0}}}}
\newcommand{\inftyOverInfty}{\ensuremath{\boldsymbol{\tfrac{\infty}{\infty}}}}
\newcommand{\zeroOverInfty}{\ensuremath{\boldsymbol{\tfrac{0}{\infty}}}}
\newcommand{\zeroTimesInfty}{\ensuremath{\small\boldsymbol{0\cdot \infty}}}
\newcommand{\inftyMinusInfty}{\ensuremath{\small\boldsymbol{\infty - \infty}}}
\newcommand{\oneToInfty}{\ensuremath{\boldsymbol{1^\infty}}}
\newcommand{\zeroToZero}{\ensuremath{\boldsymbol{0^0}}}
\newcommand{\inftyToZero}{\ensuremath{\boldsymbol{\infty^0}}}


\newcommand{\numOverZero}{\ensuremath{\boldsymbol{\tfrac{\#}{0}}}}
\newcommand{\dfn}{\textbf}
%\newcommand{\unit}{\,\mathrm}
\newcommand{\unit}{\mathop{}\!\mathrm}
%\newcommand{\eval}[1]{\bigg[ #1 \bigg]}
\newcommand{\eval}[1]{ #1 \bigg|}
\newcommand{\seq}[1]{\left( #1 \right)}
\renewcommand{\epsilon}{\varepsilon}
\renewcommand{\iff}{\Leftrightarrow}

\DeclareMathOperator{\arccot}{arccot}
\DeclareMathOperator{\arcsec}{arcsec}
\DeclareMathOperator{\arccsc}{arccsc}
\DeclareMathOperator{\si}{Si}
\DeclareMathOperator{\proj}{proj}
\DeclareMathOperator{\scal}{scal}
\DeclareMathOperator{\cis}{cis}
\DeclareMathOperator{\Arg}{Arg}
%\DeclareMathOperator{\arg}{arg}
\DeclareMathOperator{\Rep}{Re}
\DeclareMathOperator{\Imp}{Im}
\DeclareMathOperator{\sech}{sech}
\DeclareMathOperator{\csch}{csch}
\DeclareMathOperator{\Log}{Log}

\newcommand{\tightoverset}[2]{% for arrow vec
  \mathop{#2}\limits^{\vbox to -.5ex{\kern-0.75ex\hbox{$#1$}\vss}}}
\newcommand{\arrowvec}{\overrightarrow}
\renewcommand{\vec}{\mathbf}
\newcommand{\veci}{{\boldsymbol{\hat{\imath}}}}
\newcommand{\vecj}{{\boldsymbol{\hat{\jmath}}}}
\newcommand{\veck}{{\boldsymbol{\hat{k}}}}
\newcommand{\vecl}{\boldsymbol{\l}}
\newcommand{\utan}{\vec{\hat{t}}}
\newcommand{\unormal}{\vec{\hat{n}}}
\newcommand{\ubinormal}{\vec{\hat{b}}}

\newcommand{\dotp}{\bullet}
\newcommand{\cross}{\boldsymbol\times}
\newcommand{\grad}{\boldsymbol\nabla}
\newcommand{\divergence}{\grad\dotp}
\newcommand{\curl}{\grad\cross}
%% Simple horiz vectors
\renewcommand{\vector}[1]{\left\langle #1\right\rangle}


\outcome{Find the Taylor Series for a Function}

\title{Taylor Series}

\begin{document}

\begin{abstract}
We find the Taylor Series for a function.
\end{abstract}

\maketitle

\section{Taylor Series}

In this section, we will find a power series expansion centered at $x = a$ for a given infinitely differentiable function, $f(x)$.
Such a series is called a Taylor Series.  In the case that $a = 0$, we call the series a Maclaurin Series.

In other words, for a given function $f(x)$ and a given center, $x = a$, we wish to create a power series so that
\[
f(x) = \sum_{n=0}^\infty c_n (x-a)^n = c_0 + c_1(x-a) + c_2(x-a)^2 + \cdots,
\]
where the series converges in an interval of the form $(a-r, a+r)$ for some $r >0$. This is an interval centered at $x = a$ of radius $r$.
The key to creating the Taylor Series is to find the coefficients, $c_n$.
To find $c_0$, simply plug $x = a$ into both sides of the equation above. We get:
\[
f(a) = \sum_{n=0}^\infty c_n (a-a)^n = c_0 + c_1(a-a) + c_2(a-a)^2 + \cdots = c_0.
\]
Next, to find $c_1$, we differentiate both sides of the equation above, and then plug in $x = a$.
Differentiating gives
\[
f'(x) = c_1 + 2c_2(x-a) + 3c_3(x-a)^2 + \cdots,
\]
and plugging in $x = a$, this simplifies to 
\[
f'(a) = c_1 + 2c_2(a-a) + 3c_3(a-a)^2 + \cdots = c_1.
\]

Next, to find $c_2$, we differentiate both sides of the equation above, and then plug in $x = a$.
Differentiating gives
\[
f''(x) = 2c_2 + 3\cdot 2 c_3(x-a) + 4\cdot 3 c_4(x-a)^2 + \cdots,
\]
and plugging in $x = a$, this simplifies to 
\[
f''(a) = 2c_2 + 3\cdot 2 c_3(a-a) + 4\cdot 3 c_4(a-a)^2 + \cdots = 2c_2,
\]
so that $c_2 = \frac{f''(a)}{2}$.

Continuing in this fashion, we can find $c_3$:
\[
f'''(x) = 3 \cdot 2 \cdot 1 c_3 + 4\cdot 3 \cdot 2 c_4 (x-a) + 5 \cdot 4 \cdot 3 c_5 (x-a)^2 + \cdots,
\]
and plugging in $x = a$ gives
\[
f'''(a) = 3 \cdot 2 \cdot 1 c_3 + 4\cdot 3 \cdot 2 c_4 (a-a) + 5 \cdot 4 \cdot 3 c_5 (a-a)^2 + \cdots = 6c_3,
\]
so that $c_3 = \frac{f'''(a)}{6}$.
The pattern continues, so that $c_4 = \frac{f^{(4)}(a)}{4\cdot 3 \cdot 2 \cdot 1} = \frac{f^{(4)}(a)}{24}$
and $c_5 = \frac{f^{(5)}(a)}{5\cdot 4 \cdot 3 \cdot 2 \cdot 1} = \frac{f^{(5)}(a)}{120}$.
In general, we have $c_n = \frac{f^{(n)}(a)}{n\cdot (n-1) \cdot (n-2) \cdot \dots \cdot 1} = \frac{f^{(n)}(a)}{n!}$.
We present this results in the following theorem.

\begin{theorem}[Taylor Series]
Given a function $f(x)$ with derivatives of all orders at $x = a$, we can write $f(x)$ as a power series,
\[
f(x) = \sum_{n=0}^\infty \frac{f^{(n)}(a)}{n!} (x-a)^n.
\]
This is called the \textbf{Taylor Series} for $f(x)$ centered at $x = a$. In the case that $a = 0$, the series has the form
\[
 f(x) = \sum_{n=0}^\infty \frac{f^{(n)}(0)}{n!} x^n,
\]
and is called the \textbf{Maclaurin Series} for $f(x)$.
\end{theorem}

\begin{example} Find the first four terms of the Taylor Series $f(x) = \sqrt x$ centered at $x = 4$.\\
Since we are asked to find the first four terms, we must find the coefficients $c _0, c_1, c_2$ and $c_3$
and the answer will have the form 
\[
\sqrt x = c_0 + c_1(x-4) + c_2(x-4)^2 + c_3(x-4)^3 + \cdots .
\]
To find the coefficients, we create the following table:

\[
\begin{array}{|c|c|c|c|} 
\hline
\mathbf{n} & \mathbf{f^{(n)}(x)} & \mathbf{f^{(n)}(a), a = 4} & \bf{c_n} \\[1em] 
\hline
 0 & x^{1/2} & 4^{1/2} = 2 & 2 \\[1em]
\hline
1 & \frac12 x^{-1/2} & \frac12 \cdot 4^{-1/2} = \frac14 & \frac14 \\[1em]
\hline
 2 & -\frac14 x^{-3/2} & -\frac14 \cdot 4^{-3/2} 
= -\frac14 \cdot \frac18 = -\frac{1}{32} & -\frac{1}{64} \\[1em]
\hline
 3 & \frac38 x^{-5/2} & \frac38 \cdot 4^{-5/2} = \frac38 \cdot \frac{1}{32} = -\frac{3}{256} & -\frac{1}{512}\\[1em]
\hline
\end{array}
\]



We can now use the coefficients in the right-hand column of the table to write the final answer. 
The first four terms of the Taylor Series for the function $f(x) = \sqrt x$ centered at $x = 4$ are
\begin{align*}
\sqrt x &= c_0 + c_1(x-4) + c_2(x-4)^2 + c_3(x-4)^3 + \cdots \\
        &= 2 + \frac14(x-4) -\frac{1}{64}(x-4)^2 - \frac{1}{512}(x-4)^3 + \cdots
\end{align*}

\end{example}

\section{Video Lessons}

\begin{center}
\begin{foldable}
\unfoldable{Here is a detailed, lecture style video on Taylor Series:}
\youtube{YMTCxncDa4k}
\end{foldable}
\end{center}



\begin{center}
\begin{foldable}
\unfoldable{Here is a detailed, lecture style video on Maclaurin Series:}
\youtube{aa9rcfNJrfk}
\end{foldable}
\end{center}




\end{document}



\begin{table}[ht]
\caption{Finding the Coefficients, $c_n$}
\centering
\begin{tabular}{|c|c|c|c|}
\hline
\raisebox{7pt}{\phantom{i}}\textbf{$n$} & \textbf{$f^{(n)}(x)$} & \textbf{$f^{(n)}(a), a = 4$} & \textbf{$c_n$} \\
\hline
\raisebox{7pt}{\phantom{i}}$0$ & $x^{1/2}$ &\raisebox{8pt}{\phantom{M}} $4^{1/2} = 2$ & $2$ \\
\hline
\raisebox{7pt}{\phantom{i}}$1$ & $\frac12 x^{-1/2}$ & \raisebox{8pt}{\phantom{M}}$\displaystyle{\frac12 \cdot 4^{-1/2} = \frac14}$ & $\frac14$ \\
\hline
\raisebox{7pt}{\phantom{i}} $2$ & \raisebox{12pt}{\phantom{i}}$-\frac14 x^{-3/2}$ & $-\frac14 \cdot 4^{-3/2} 
= -\frac14 \cdot \frac18 = -\frac{1}{32}$ & $-\frac{1}{64}$ \\
\hline
\raisebox{7pt}{\phantom{i}} $3$ & $\frac38 x^{-5/2}$ & $\frac38 \cdot 4^{-5/2} = \frac38 \cdot \frac{1}{32} = -\frac{3}{256}$ & $-\frac{1}{512}$ \\
\hline
\end{tabular}
\end{table}




\begin{array}{|c|c|c|c|}  
\hline
\textbf{$n$} & \textbf{$f^{(n)}(x)$} & \textbf{$f^{(n)}(a), a = 4$} & \textbf{$c_n$} \\
\hline
$0$ & $x^{1/2}$ & $4^{1/2} = 2$ & $2$ \\
\hline
$1$ & $\frac12 x^{-1/2}$ & $\frac12 \cdot 4^{-1/2} = \frac14$ & $\frac14$ \\
\hline
 $2$ & $-\frac14 x^{-3/2}$ & $-\frac14 \cdot 4^{-3/2} 
= -\frac14 \cdot \frac18 = -\frac{1}{32}$ & $-\frac{1}{64}$ \\
\hline
 $3$ & $\frac38 x^{-5/2}$ & $\frac38 \cdot 4^{-5/2} = \frac38 \cdot \frac{1}{32} = -\frac{3}{256}$ & $-\frac{1}{512}$
\hline
\end{array}





\begin{example} %example #15
Find $h'(x)$ if $h(x) = x^{\sin(x)}$.\\
We will use the fact that the exponential and logarithm functions are inverses,
\[e^{\ln(x)} = x,\]
and the exponent property of logarithms, 
\[\ln(x^n) = n\ln(x),\]
to rewrite $h(x)$.  We have 
\[h(x) = x^{\sin(x)} = e^{\ln(x^{\sin(x)})} = e^{\sin(x)\ln(x)}\]
and we can now compute $h'(x)$ using a combination of the chain rule and product rule.
We can write $h(x)$ as a composition, $f(g(x))$ with 
\[g(x) = \sin(x)\ln(x) \quad \text{and} \quad f(x) = e^x.\]
Then to find $g'(x)$ we us the product rule and we get $g'(x) = \frac{\sin(x)}{x} + \cos(x)\ln(x)$.
Next $f'(x) = e^x$ and 
hence $f'(g(x)) = e^{g(x)} = e^{\sin(x)\ln(x)} = x^{\sin(x)}$.
We can then conclude $h'(x) = f'(g(x))g'(x) = x^{\sin(x)} \left[ \frac{\sin(x)}{x} + \cos(x)\ln(x)\right]$.
\end{example}

%more question formats below













%\begin{verbatim}
\begin{question}
What is your favorite color?
\begin{multipleChoice}
\choice[correct]{Rainbow}
\choice{Blue}
\choice{Green}
\choice{Red}
\end{multipleChoice}
\begin{freeResponse}
Hello
\end{freeResponse}
\end{question}
%\end{verbatim}





\begin{question}
  Which one will you choose?
  \begin{multipleChoice}
    \choice[correct]{I'm correct.}
    \choice{I'm wrong.}
    \choice{I'm wrong too.}
  \end{multipleChoice}
\end{question}


\begin{question}
  Which one will you choose?
  \begin{selectAll}
    \choice[correct]{I'm correct.}
    \choice{I'm wrong.}
    \choice[correct]{I'm also correct.}
    \choice{I'm wrong too.}
  \end{selectAll}
\end{question}


\begin{freeResponse}
What is the chain rule used for?
\end{freeResponse}
