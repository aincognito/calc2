\documentclass{ximera}
%\usepackage{tcolorbox}
%% You can put user macros here
%% However, you cannot make new environments



\newcommand{\ffrac}[2]{\frac{\text{\footnotesize $#1$}}{\text{\footnotesize $#2$}}}
\newcommand{\vasymptote}[2][]{
    \draw [densely dashed,#1] ({rel axis cs:0,0} -| {axis cs:#2,0}) -- ({rel axis cs:0,1} -| {axis cs:#2,0});
}


\graphicspath{{./}{firstExample/}}

\usepackage{amsmath}
\usepackage{amssymb}
\usepackage{array}
\usepackage[makeroom]{cancel} %% for strike outs
\usepackage{pgffor} %% required for integral for loops
\usepackage{tikz}
\usepackage{tikz-cd}
\usepackage{tkz-euclide}
\usetikzlibrary{shapes.multipart}


\usetkzobj{all}
\tikzstyle geometryDiagrams=[ultra thick,color=blue!50!black]


\usetikzlibrary{arrows}
\tikzset{>=stealth,commutative diagrams/.cd,
  arrow style=tikz,diagrams={>=stealth}} %% cool arrow head
\tikzset{shorten <>/.style={ shorten >=#1, shorten <=#1 } } %% allows shorter vectors

\usetikzlibrary{backgrounds} %% for boxes around graphs
\usetikzlibrary{shapes,positioning}  %% Clouds and stars
\usetikzlibrary{matrix} %% for matrix
\usepgfplotslibrary{polar} %% for polar plots
\usepgfplotslibrary{fillbetween} %% to shade area between curves in TikZ



%\usepackage[width=4.375in, height=7.0in, top=1.0in, papersize={5.5in,8.5in}]{geometry}
%\usepackage[pdftex]{graphicx}
%\usepackage{tipa}
%\usepackage{txfonts}
%\usepackage{textcomp}
%\usepackage{amsthm}
%\usepackage{xy}
%\usepackage{fancyhdr}
%\usepackage{xcolor}
%\usepackage{mathtools} %% for pretty underbrace % Breaks Ximera
%\usepackage{multicol}



\newcommand{\RR}{\mathbb R}
\newcommand{\R}{\mathbb R}
\newcommand{\C}{\mathbb C}
\newcommand{\N}{\mathbb N}
\newcommand{\Z}{\mathbb Z}
\newcommand{\dis}{\displaystyle}
%\renewcommand{\d}{\,d\!}
\renewcommand{\d}{\mathop{}\!d}
\newcommand{\dd}[2][]{\frac{\d #1}{\d #2}}
\newcommand{\pp}[2][]{\frac{\partial #1}{\partial #2}}
\renewcommand{\l}{\ell}
\newcommand{\ddx}{\frac{d}{\d x}}

\newcommand{\zeroOverZero}{\ensuremath{\boldsymbol{\tfrac{0}{0}}}}
\newcommand{\inftyOverInfty}{\ensuremath{\boldsymbol{\tfrac{\infty}{\infty}}}}
\newcommand{\zeroOverInfty}{\ensuremath{\boldsymbol{\tfrac{0}{\infty}}}}
\newcommand{\zeroTimesInfty}{\ensuremath{\small\boldsymbol{0\cdot \infty}}}
\newcommand{\inftyMinusInfty}{\ensuremath{\small\boldsymbol{\infty - \infty}}}
\newcommand{\oneToInfty}{\ensuremath{\boldsymbol{1^\infty}}}
\newcommand{\zeroToZero}{\ensuremath{\boldsymbol{0^0}}}
\newcommand{\inftyToZero}{\ensuremath{\boldsymbol{\infty^0}}}


\newcommand{\numOverZero}{\ensuremath{\boldsymbol{\tfrac{\#}{0}}}}
\newcommand{\dfn}{\textbf}
%\newcommand{\unit}{\,\mathrm}
\newcommand{\unit}{\mathop{}\!\mathrm}
%\newcommand{\eval}[1]{\bigg[ #1 \bigg]}
\newcommand{\eval}[1]{ #1 \bigg|}
\newcommand{\seq}[1]{\left( #1 \right)}
\renewcommand{\epsilon}{\varepsilon}
\renewcommand{\iff}{\Leftrightarrow}

\DeclareMathOperator{\arccot}{arccot}
\DeclareMathOperator{\arcsec}{arcsec}
\DeclareMathOperator{\arccsc}{arccsc}
\DeclareMathOperator{\si}{Si}
\DeclareMathOperator{\proj}{proj}
\DeclareMathOperator{\scal}{scal}
\DeclareMathOperator{\cis}{cis}
\DeclareMathOperator{\Arg}{Arg}
%\DeclareMathOperator{\arg}{arg}
\DeclareMathOperator{\Rep}{Re}
\DeclareMathOperator{\Imp}{Im}
\DeclareMathOperator{\sech}{sech}
\DeclareMathOperator{\csch}{csch}
\DeclareMathOperator{\Log}{Log}

\newcommand{\tightoverset}[2]{% for arrow vec
  \mathop{#2}\limits^{\vbox to -.5ex{\kern-0.75ex\hbox{$#1$}\vss}}}
\newcommand{\arrowvec}{\overrightarrow}
\renewcommand{\vec}{\mathbf}
\newcommand{\veci}{{\boldsymbol{\hat{\imath}}}}
\newcommand{\vecj}{{\boldsymbol{\hat{\jmath}}}}
\newcommand{\veck}{{\boldsymbol{\hat{k}}}}
\newcommand{\vecl}{\boldsymbol{\l}}
\newcommand{\utan}{\vec{\hat{t}}}
\newcommand{\unormal}{\vec{\hat{n}}}
\newcommand{\ubinormal}{\vec{\hat{b}}}

\newcommand{\dotp}{\bullet}
\newcommand{\cross}{\boldsymbol\times}
\newcommand{\grad}{\boldsymbol\nabla}
\newcommand{\divergence}{\grad\dotp}
\newcommand{\curl}{\grad\cross}
%% Simple horiz vectors
\renewcommand{\vector}[1]{\left\langle #1\right\rangle}


\outcome{Compute limits using L'Hopital's Rule}

\title{3.6 L'Hopital's Rule}


\begin{document}

\begin{abstract}
In this section we compute limits using L'Hopital's Rule which requires our knowledge of derivatives.
\end{abstract}

\maketitle

\section{L'Hopital's Rule}

L'Hopital's Rule uses the derivative to help us find limits involving indeterminate forms. 
The main indeterminate forms we will discuss are $\frac 00, \frac{\infty}{\infty}, 0\cdot \infty, 1^\infty$ and $0^0$. We begin with the fractional forms.

\begin{theorem}[L'Hopital's Rule]




If $\displaystyle{\lim_{x \to c} \frac{f(x)}{g(x)} = \frac{0}{0} \quad \text{or} \quad \frac{\infty}{\infty}}$\\
then $\displaystyle{\lim_{x \to c} \frac{f(x)}{g(x)} = \lim_{x \to c} \frac{f'(x)}{g'(x)}}$ provided the latter exists.\\
In the above statement, $\lim_{x \to c}$ can be replaced by a one-sided limit and $c$ can be $\pm \infty$. Also
the fraction $\frac{\infty}{\infty}$ is shorthand for $\frac{\pm \infty}{\pm \infty}$.
\end{theorem}
We begin with the $\frac 00$ form.


\section{The $0/0$ case}


\begin{example}[example 1]
Compute the limit: $\displaystyle{\lim_{x \to 0} \frac{\sin(x)}{x}}.$

Plugging in the terminal value, $x=0$, yields 
the indeterminate form $\frac00$, so L'Hopital's rule applies.

We have 

\[\lim_{x \to 0} \frac{\sin(x)}{x} = \lim_{x \to 0} \frac{\cos(x)}{1} = 1.\]
\end{example}

\begin{problem}(problem 1a)
  Compute
  \[
  \lim_{x \to 0} \frac{1 - \cos(x)}{x} = \answer{0}
  \]
  
    \begin{hint}
      Plug in $x=0$
    \end{hint}
    \begin{hint}
      If you got $\frac00$, use L'Hopital's Rule
    \end{hint}
    \begin{hint}
      Take the derivative of the numerator and denominator separately
    \end{hint}
    \begin{hint}
      Compute the limit of the new fraction
    \end{hint}
    
		
	
\end{problem}

\begin{problem}(problem 1b)
  Compute
  \[
  \lim_{x \to 0} \frac{\tan(x)}{3x} = \answer{1/3}
  \]
  
    \begin{hint}
      Plug in $x=0$
    \end{hint}
    \begin{hint}
      If you got $0/0$, use L'Hopital's Rule
    \end{hint}
    \begin{hint}
      Take the derivative of the numerator and denominator separately
    \end{hint}
		\begin{hint}
		  The derivative of $\tan(x)$ is $\sec^2(x)$
    \end{hint}
		\begin{hint}
      Compute the limit of the new fraction
    \end{hint}
	
\end{problem}


\begin{example}[example 2]
 Compute the limit: $\displaystyle{\lim_{x \to 0} \frac{e^x - 1}{x}}.$\\
Plugging in the terminal value, $x=0$, yields 
the indeterminate form $0/0$, so L'Hopital's rule applies.
We have 
\[\lim_{x \to 0} \frac{e^x - 1}{x} = \lim_{x \to 0} \frac{e^x}{1} = 1.\]
\end{example}


\begin{problem}(problem 2a)
  Compute
  \[
  \lim_{x \to 0} \frac{5x}{2 - 2e^x} = \answer{-5/2}
  \]
  
    \begin{hint}
      Plug in $x=0$
    \end{hint}
    \begin{hint}
      If you got $\frac00$, use L'Hopital's Rule
    \end{hint}
    \begin{hint}
      Take the derivative of the numerator and denominator separately
    \end{hint}
	  \begin{hint}
      Compute the limit of the new fraction
    \end{hint}
	
\end{problem}


\begin{problem}(problem 2b)
  Compute
  \[
  \lim_{x \to 0} \frac{e^{2x} -1}{7\sin(x)} = \answer{2/7}
  \]
  
    \begin{hint}
      Plug in $x=0$
    \end{hint}
    \begin{hint}
      If you got $0/0$, use L'Hopital's Rule
    \end{hint}
    \begin{hint}
      Take the derivative of the numerator and denominator separately
    \end{hint}
		\begin{hint}
		 The derivative of $e^{2x}$ is $2e^{2x}$, by the Chain Rule
	  \end{hint}
		\begin{hint}
      Compute the limit of the new fraction
    \end{hint}
    
\end{problem}


\begin{example}[example 3]
Compute the limit: $\displaystyle{\lim_{x \to 2} \frac{x^3 - 8}{x^5 - 32}}.$\\
Plugging in the terminal value, $x=2$, yields 
the indeterminate form $0/0$, so L'Hopital's rule applies.
We have 
\[\lim_{x \to 2} \frac{x^3 - 8}{x^5 - 32} = \lim_{x \to 2} \frac{3x^2}{5x^4} = \frac{12}{80} = \frac{3}{20}.\]
\end{example}

\begin{problem}(problem 3a)
  Compute
  \[
  \lim_{x \to 3} \frac{x^3 - 27}{x^4 - 81} = \answer{1/4}
  \]
  
    \begin{hint}
      Plug in $x=3$
    \end{hint}
    \begin{hint}
      If you got $0/0$, use L'Hopital's Rule
    \end{hint}
    \begin{hint}
      Take the derivative of the numerator and denominator separately
    \end{hint}
	  \begin{hint}
      Compute the limit of the new fraction
    \end{hint}
  
\end{problem}


\begin{problem}(problem 3b)
  Compute
  \[
  \lim_{x \to 4} \frac{x^2 - 3x - 4}{x^3 - 64} = \answer{5/48}
  \]
  
    \begin{hint}
      Plug in $x=4$
    \end{hint}
    \begin{hint}
      If you got $0/0$, use L'Hopital's Rule
    \end{hint}
    \begin{hint}
      Take the derivative of the numerator and denominator separately
    \end{hint}
	  \begin{hint}
      Compute the limit of the new fraction
    \end{hint}
  
\end{problem}


Sometimes we have to use L'Hopital's Rule more than once.

\begin{example}[example 4]
Compute the limit: $\displaystyle{\lim_{x \to 0} \frac{e^x - x - 1}{x^2}}.$

Plugging in the terminal value, $x=0$, yields 
the indeterminate form $0/0$, so L'Hopital's rule applies.
We have 
\[\lim_{x \to 0} \frac{e^x - x - 1}{x^2} = \lim_{x \to 0} \frac{e^x -1}{2x} = \frac{0}{0}.\]
Applying L'Hopital's Rule again gives
\[\lim_{x \to 0} \frac{e^x -1}{2x} = \frac{e^x}{2} = \frac{1}{2}.\]
Hence $\displaystyle{\lim_{x \to 0} \frac{e^x - x - 1}{x^2} = \frac 12}.$
\end{example}


\begin{problem}(problem 4a)
  Compute
  \[
  \lim_{x \to 0} \frac{1 - \cos(x)}{x^2} = \answer{1/2}
  \]
  
    \begin{hint}
      Plug in $x=0$
    \end{hint}
    \begin{hint}
      If you got $\frac00$, use L'Hopital's Rule
    \end{hint}
    \begin{hint}
      Take the derivative of the numerator and denominator separately
    \end{hint}
	  \begin{hint}
      Compute the limit of the new fraction
    \end{hint}
		\begin{hint}
		 If you get $\frac00$, use L'Hopital's Rule again
    \end{hint}

	
\end{problem}


\begin{problem}(problem 4b)
  Compute
  \[
  \lim_{x \to 0} \frac{\sec(x) - 1}{x^3 - x^2} = \answer{-1/2}
  \]
  
    \begin{hint}
      Plug in $x=0$
    \end{hint}
    \begin{hint}
      If you got $0/0$, use L'Hopital's Rule
    \end{hint}
    \begin{hint}
      Take the derivative of the numerator and denominator separately
    \end{hint}
		\begin{hint}
		  The derivative of $\sec(x)$ is $\sec(x)\tan(x)$
	  \end{hint}
		\begin{hint}
      Compute the limit of the new fraction
    \end{hint}
		\begin{hint}
		 If you get $\frac00$, use L'Hopital's Rule again
    \end{hint}
		\begin{hint}
		  Use the Product Rule to find the derivative of $\sec(x)\tan(x)$
		\end{hint}
		
\end{problem}


\begin{example}[example 5]
Compute the limit: $\displaystyle{\lim_{x \to 0} \frac{\tan(x)-x}{x^3}}.$\\
Plugging in the terminal value, $x=0$, yields 
the indeterminate form $\frac00$, so L'Hopital's rule applies.
We have 
\[\lim_{x \to 0} \frac{\tan(x)-x}{x^3} = \lim_{x \to 0} \frac{\sec^2(x)-1}{3x^2} = \frac{0}{0}.\]
Applying L'Hopital's Rule again gives
\[\lim_{x \to 0} \frac{\sec^2(x)-1}{3x^2} = \lim_{x \to 0} \frac{2\sec^2(x)\tan(x)}{6x} = \frac{0}{0}.\]	
We need to apply L'Hopital's Rule again, but first, the numerator is complicated and 								
\[\lim_{x \to 0} \sec(x) = 1 \neq 0\]
so we take a simplifying step before applying the rule.
\[\lim_{x \to 0} \frac{2\sec^2(x)\tan(x)}{6x} = 
\lim_{x \to 0} \sec^2(x) \cdot \lim_{x \to 0} \frac{2\tan(x)}{6x}=\]
 \[1\cdot \lim_{x \to 0} \frac{2\sec^2(x)}{6} =\frac{1}{3}.\]
\end{example}


We next consider problems of the form $\infty/\infty$.  These are handled the same way as the $0/0$ case above.



\section{ The $\dfrac{\infty}{\infty}$ case.}



\begin{example}[example 6]
Compute the limit: $\displaystyle{\lim_{x \to \infty} \frac{x^2}{e^x}}.$

As $x$ approaches $\infty$ we get the indeterminate form $\frac{\infty}{\infty}$ so L'Hopital's Rule applies.
We have 
\[\lim_{x\to \infty} \frac{x^2}{e^x} = \lim_{x \to \infty} \frac{2x}{e^x} = \frac{\infty}{\infty}.\]
Applying L'hopital again, we get
\[\lim_{x \to \infty} \frac{2x}{e^x} = \lim_{\infty} \frac{2}{e^x} = 0.\]
Hence $\lim_{x\to\infty}\frac{x^2}{e^x}=0$.  
This limit can be generalized as follows:
\[\lim_{x\to\infty}\frac{x^n}{e^x} =0\]
for any exponent $n = 1, 2, 3, \dots$.
This general result comes from using L'Hopital's Rule $n$ times, yielding
\[\lim_{x\to\infty} \frac{n!}{e^x} =0\]
where $n! = n(n-1)(n-2)\cdot \dots \cdot 3\cdot 2\cdot 1$.
The interpretation of this limit is that the exponential function $e^x$ grows faster than any power of $x$ as $x \to \infty$.
\end{example}




\begin{problem}(problem 6a)
  Compute
  \[
  \lim_{x \to \infty} \frac{3x+2}{2x + 3} = \answer{3/2}
  \]
  
    \begin{hint}
      ``Plug in'' $x=\infty$
    \end{hint}
    \begin{hint}
      If you got $\frac{\infty}{\infty}$, use L'Hopital's Rule
    \end{hint}
    \begin{hint}
      Take the derivative of the numerator and denominator separately
    \end{hint}
		\begin{hint}
      Compute the limit of the new fraction
    \end{hint}
	
\end{problem}


\begin{problem}(problem 6b)
  Compute
  \[
  \lim_{x \to \infty} \frac{5x^2 + 2x - 3}{4x^2 - x + 1} = \answer{5/4}
  \]
  
    \begin{hint}
      ``Plug in'' $x=\infty$
    \end{hint}
    \begin{hint}
      If you got $\frac{\infty}{\infty}$, use L'Hopital's Rule
    \end{hint}
    \begin{hint}
      Take the derivative of the numerator and denominator separately
    \end{hint}
		\begin{hint}
      Compute the limit of the new fraction
    \end{hint}
		\begin{hint}
		  If you got $\frac{\infty}{\infty}$, use L'Hopital's Rule again
	  \end{hint}
		
\end{problem}


\begin{example}[example 7]
Compute the limit: $\displaystyle{\lim_{x \to \infty} \frac{\sqrt{x}}{\ln(x)}}$.
As $x \to \infty$ we get $\infty/\infty$, so L'Hopital's Rule applies.
We have:
\[\lim_{x \to \infty} \frac{\sqrt x}{\ln(x)} = 
\lim_{x \to \infty}\frac{\left(\frac{1}{2\sqrt x}\right)}{\left(\frac{1}{x}\right)}\]
which simplifies to 
\[\lim_{x\to\infty} \frac{x}{2\sqrt x} = \lim_{x\to\infty}\frac{\sqrt x}{2} = \infty.\]
Hence, $\lim_{x\to\infty} \frac{\sqrt x}{\ln(x)} = \infty$.
The interpretation of this limit is that $\sqrt x$ goes to $\infty$ faster than $\ln(x)$ as $x\to\infty$.
\end{example}

\begin{problem}(problem 7a)
  Compute
  \[
  \lim_{x \to \infty} \frac{\sqrt x}{2x + 3} = \answer{0}
  \]
  
    \begin{hint}
      ``Plug in'' $x=\infty$
    \end{hint}
    \begin{hint}
      If you got $\frac{\infty}{\infty}$, use L'Hopital's Rule
    \end{hint}
    \begin{hint}
      Take the derivative of the numerator and denominator separately
    \end{hint}
		\begin{hint}
		  The derivative of $\sqrt x$ is $\frac{1}{2\sqrt x}$
	  \end{hint}
		\begin{hint}
      Compute the limit of the new fraction
    \end{hint}
	
\end{problem}

\begin{problem}(problem 7b)
  Compute
  \[
  \lim_{x \to \infty} \frac{\ln(x)}{1 + x^2} = \answer{0}
  \]
  
    \begin{hint}
      ``Plug in'' $x=\infty$
    \end{hint}
    \begin{hint}
      If you got $\frac{\infty}{\infty}$, use L'Hopital's Rule
    \end{hint}
    \begin{hint}
      Take the derivative of the numerator and denominator separately
    \end{hint}
		\begin{hint}
		  The derivative of $\ln(x)$ is $\frac{1}{x}$
	  \end{hint}
		\begin{hint}
      Simplify the new fraction and then compute the limit
    \end{hint}
	
\end{problem}


\section{The $0 \cdot \infty$ case}

L'Hopital's Rule requires a fractional indeterminate form such as $0/0$ or $\infty/\infty$, 
but we can use it to handle other indeterminate forms by rewriting expressions as fractions.

Examples of the $0\cdot\infty$ case.

\begin{example}[example 8]
Compute the limit: $\displaystyle{\lim_{x \to 0^+}x^2 \ln(x)}$.

As $x\to 0^+$ we get $0 \cdot (-\infty)$ which is an indeterminate form, but L'Hopital's Rule does not apply in this situation.
We must rewrite the problem as a fraction, in the following way:
\[\lim_{x \to 0^+}\frac{\ln(x)}{x^{-2}}.\]
Notice that this is equivalent to the original problem since 
\[x^2 = \frac{1}{x^{-2}}.\]
Also note that $x^{-2} =\frac{1}{x^2} \to \infty$ as $x\to 0^+$.
Now, we can use L'Hopital's Rule because 
\[\lim_{x \to 0+}\frac{\ln(x)}{x^{-2}}= \frac{-\infty}{\infty}.\]
We get
\[\lim_{x \to 0^+}\frac{\ln(x)}{x^{-2}}= \lim_{x\to 0^+} \frac{1/x}{-2x^{-3}}\]
which simplifies to
\[\lim_{x \to 0^+}-\frac{x^2}{2}= 0.\]
Hence, 
\[\lim_{x \to 0^+} x^2 \ln(x)=\lim_{x \to 0^+}\frac{ln(x)}{x^{-2}}= 0.\]
\end{example}


\begin{problem}(problem 8)
  Compute
  \[
  \lim_{x \to \infty} 3xe^{-x} = \answer{0}
  \]
  
    \begin{hint}
      ``Plug in'' $x=\infty$
    \end{hint}
    \begin{hint}
      If you got $\infty \cdot 0$, rewrite the expression as a fraction
    \end{hint}
    \begin{hint}
      Take advantage of negative exponents: $e^{-x} = \frac{1}{e^x}$
    \end{hint}
	
\end{problem}


\begin{center}
\begin{foldable}
\unfoldable{Here is a detailed, lecture style video on L'Hopital's Rule:}
\youtube{fo2rJzeYarw}
\end{foldable}
\end{center}



\end{document}
