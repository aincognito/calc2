\documentclass{ximera}

%% You can put user macros here
%% However, you cannot make new environments



\newcommand{\ffrac}[2]{\frac{\text{\footnotesize $#1$}}{\text{\footnotesize $#2$}}}
\newcommand{\vasymptote}[2][]{
    \draw [densely dashed,#1] ({rel axis cs:0,0} -| {axis cs:#2,0}) -- ({rel axis cs:0,1} -| {axis cs:#2,0});
}


\graphicspath{{./}{firstExample/}}

\usepackage{amsmath}
\usepackage{amssymb}
\usepackage{array}
\usepackage[makeroom]{cancel} %% for strike outs
\usepackage{pgffor} %% required for integral for loops
\usepackage{tikz}
\usepackage{tikz-cd}
\usepackage{tkz-euclide}
\usetikzlibrary{shapes.multipart}


\usetkzobj{all}
\tikzstyle geometryDiagrams=[ultra thick,color=blue!50!black]


\usetikzlibrary{arrows}
\tikzset{>=stealth,commutative diagrams/.cd,
  arrow style=tikz,diagrams={>=stealth}} %% cool arrow head
\tikzset{shorten <>/.style={ shorten >=#1, shorten <=#1 } } %% allows shorter vectors

\usetikzlibrary{backgrounds} %% for boxes around graphs
\usetikzlibrary{shapes,positioning}  %% Clouds and stars
\usetikzlibrary{matrix} %% for matrix
\usepgfplotslibrary{polar} %% for polar plots
\usepgfplotslibrary{fillbetween} %% to shade area between curves in TikZ



%\usepackage[width=4.375in, height=7.0in, top=1.0in, papersize={5.5in,8.5in}]{geometry}
%\usepackage[pdftex]{graphicx}
%\usepackage{tipa}
%\usepackage{txfonts}
%\usepackage{textcomp}
%\usepackage{amsthm}
%\usepackage{xy}
%\usepackage{fancyhdr}
%\usepackage{xcolor}
%\usepackage{mathtools} %% for pretty underbrace % Breaks Ximera
%\usepackage{multicol}



\newcommand{\RR}{\mathbb R}
\newcommand{\R}{\mathbb R}
\newcommand{\C}{\mathbb C}
\newcommand{\N}{\mathbb N}
\newcommand{\Z}{\mathbb Z}
\newcommand{\dis}{\displaystyle}
%\renewcommand{\d}{\,d\!}
\renewcommand{\d}{\mathop{}\!d}
\newcommand{\dd}[2][]{\frac{\d #1}{\d #2}}
\newcommand{\pp}[2][]{\frac{\partial #1}{\partial #2}}
\renewcommand{\l}{\ell}
\newcommand{\ddx}{\frac{d}{\d x}}

\newcommand{\zeroOverZero}{\ensuremath{\boldsymbol{\tfrac{0}{0}}}}
\newcommand{\inftyOverInfty}{\ensuremath{\boldsymbol{\tfrac{\infty}{\infty}}}}
\newcommand{\zeroOverInfty}{\ensuremath{\boldsymbol{\tfrac{0}{\infty}}}}
\newcommand{\zeroTimesInfty}{\ensuremath{\small\boldsymbol{0\cdot \infty}}}
\newcommand{\inftyMinusInfty}{\ensuremath{\small\boldsymbol{\infty - \infty}}}
\newcommand{\oneToInfty}{\ensuremath{\boldsymbol{1^\infty}}}
\newcommand{\zeroToZero}{\ensuremath{\boldsymbol{0^0}}}
\newcommand{\inftyToZero}{\ensuremath{\boldsymbol{\infty^0}}}


\newcommand{\numOverZero}{\ensuremath{\boldsymbol{\tfrac{\#}{0}}}}
\newcommand{\dfn}{\textbf}
%\newcommand{\unit}{\,\mathrm}
\newcommand{\unit}{\mathop{}\!\mathrm}
%\newcommand{\eval}[1]{\bigg[ #1 \bigg]}
\newcommand{\eval}[1]{ #1 \bigg|}
\newcommand{\seq}[1]{\left( #1 \right)}
\renewcommand{\epsilon}{\varepsilon}
\renewcommand{\iff}{\Leftrightarrow}

\DeclareMathOperator{\arccot}{arccot}
\DeclareMathOperator{\arcsec}{arcsec}
\DeclareMathOperator{\arccsc}{arccsc}
\DeclareMathOperator{\si}{Si}
\DeclareMathOperator{\proj}{proj}
\DeclareMathOperator{\scal}{scal}
\DeclareMathOperator{\cis}{cis}
\DeclareMathOperator{\Arg}{Arg}
%\DeclareMathOperator{\arg}{arg}
\DeclareMathOperator{\Rep}{Re}
\DeclareMathOperator{\Imp}{Im}
\DeclareMathOperator{\sech}{sech}
\DeclareMathOperator{\csch}{csch}
\DeclareMathOperator{\Log}{Log}

\newcommand{\tightoverset}[2]{% for arrow vec
  \mathop{#2}\limits^{\vbox to -.5ex{\kern-0.75ex\hbox{$#1$}\vss}}}
\newcommand{\arrowvec}{\overrightarrow}
\renewcommand{\vec}{\mathbf}
\newcommand{\veci}{{\boldsymbol{\hat{\imath}}}}
\newcommand{\vecj}{{\boldsymbol{\hat{\jmath}}}}
\newcommand{\veck}{{\boldsymbol{\hat{k}}}}
\newcommand{\vecl}{\boldsymbol{\l}}
\newcommand{\utan}{\vec{\hat{t}}}
\newcommand{\unormal}{\vec{\hat{n}}}
\newcommand{\ubinormal}{\vec{\hat{b}}}

\newcommand{\dotp}{\bullet}
\newcommand{\cross}{\boldsymbol\times}
\newcommand{\grad}{\boldsymbol\nabla}
\newcommand{\divergence}{\grad\dotp}
\newcommand{\curl}{\grad\cross}
%% Simple horiz vectors
\renewcommand{\vector}[1]{\left\langle #1\right\rangle}


\outcome{Find limits at infinity}

\title{1.6 End Behavior}


\begin{document}

\begin{abstract}
Find limits at infinity.
\end{abstract}

\maketitle

\section{End Behavior}
 



%\begin{center}
%{\bf End Behavior}
%\end{center}

In this section we consider limits as $x$ approaches either $\infty$ or $-\infty$.
There is a connection to the value of these limits and horizontal asymptotes.

\begin{theorem}
(1) If the limit
\[
\lim_{x \to \infty} f(x) = L_1,
\]
then the line $y=L_1$ is a horizontal asymptote for the graph of $y = f(x)$ on the right end.\\
(2) If the limit
\[
\lim_{x \to -\infty} f(x) = L_2,
\]
then the line $y=L_2$ is a horizontal asymptote for the graph of $y = f(x)$ on the left end.
\end{theorem}



\begin{example} %example 1
Find $\lim_{x \to \infty} \dfrac{1}{x}.$
  We argue as follows. If $x$ is a very large number, 
	then $1/x$ will be a very small number, near zero.  Furthermore, as $x$ increases, $1/x$ will decrease.
	
	Hence
	\[\lim_{x \to \infty} \frac{1}{x}= 0.\]
	This is evident from the graph of $y=1/x$ shown below.
	
	%\[\graph[panel, xAxisLabel=``x'', yAxisLabel=``y'', xmin=-3, xmax=75]{y = 1/x}\]
	
	The graph of the function $f(x) = \frac{1}{x}$ also provides evidence for this conclusion.  
	
	\[\graph{y=1/x}\]
	
	Lastly, the value of the limit corresponds to the horizontal asymptote of the graph, namely
	$y = 0$ in this example.
\end{example}




\begin{problem} %problem #1
  Compute the limit:
  \[
  \lim_{x \to \infty} \frac{7}{x^2}
  \]
  
    \begin{hint}
      As $x$ goes to infinity, so does $x^2$
    \end{hint}
    \begin{hint}
      As the denominator grows, the fraction shrinks
    \end{hint}
    \begin{hint}
      The fraction never shrinks below 0
    \end{hint}
		The value of the limit is
		 $\answer[given]{0}$
		
\end{problem}

An important generalization of the last example and problem, which follows from a similar analysis is 
\[\lim_{x \to \pm \infty} \frac{a}{x^n} = 0 \]
for any constant, $a$, and any positive exponent, $n$.

We will exploit this important fact in the next two examples.


\begin{example} %example 2
Compute 
\[
\lim_{x \to \infty} \frac{3x^2 + 5x + 2}{2x^2 -x- 4}.
\]

First note that as $x\to \pm \infty$, a polynomial, $p(x) \to \pm \infty$ according to it leading term.
In this example, since the lead terms $3x^2$ and $2x^2$ both go to $\infty$ as $x \to \infty$, 
our limit has the indeterminate form
$\frac{\infty}{\infty}$.
To resolve this issue, we will factor out of both the numerator and the denominator 
the highest power of $x$ seen in the denominator.  So, in this example, we will factor $x^2$ from both.
In the numerator, 
\[3x^2 + 5x + 2 = x^2(3 + \frac{5}{x} + \frac{2}{x^2})\]
and in the denominator,
\[2x^2 -x -4 = x^2(2- \frac{1}{x} - \frac{4}{x^2}).\]

With these factorizations, our limit becomes

\begin{align*}
\lim_{x \to \infty} \frac{3x^2 + 5x + 2}{2x^2 -x- 4} &= 
\lim_{x \to \infty} \frac{x^2(3 + \frac{5}{x} + \frac{2}{x^2})}{x^2(2- \frac{1}{x} - \frac{4}{x^2})} \\ \\
&=\lim_{x \to \infty} \frac{3 + \frac{5}{x} + \frac{2}{x^2}}{2- \frac{1}{x} - \frac{4}{x^2}} \\ \\
&=\frac{3 + 0 + 0}{2- 0 - 0} \\ \\
&= \frac32.
\end{align*}

The result of this limit means that the line $y = 3/2$ is a horizontal asymptote
for the graph of $y = \dfrac{3x^2 + 5x + 2}{2x^2 -x- 4}$ on the right end.
\end{example}


\begin{problem} %problem #2
  Compute the limit:
  \[
  \lim_{x \to -\infty} \frac{4x^2 + 3x - 2}{2x^2 - 2x + 5}
  \]
  
    \begin{hint}
      When you `plug in' $x = -\infty$, you get an $\frac{\infty}{\infty}$ form
    \end{hint}
    \begin{hint}
      Determine the highest power of $x$ in the denominator
    \end{hint}
    \begin{hint}
      Factor this power from both the numerator and denominator and cancel it
    \end{hint}
    \begin{hint}
      $\frac{a}{x^n} \to 0$ as $x \to \infty$.
    \end{hint}
		The value of the limit is
		 $\answer[given]{2}$
		
\end{problem}



\begin{problem} %problem #2
  Compute the limit:
  \[
  \lim_{x \to \infty} \frac{3 + 2x^2 - x^3}{5x^3 - 2x -4}
  \]
  
    \begin{hint}
      When you `plug in' $x = \infty$, you get an $\frac{\infty}{\infty}$ form
    \end{hint}
    \begin{hint}
      Determine the highest power of $x$ in the denominator
    \end{hint}
    \begin{hint}
      Factor this power from both the numerator and denominator and cancel it
    \end{hint}
    \begin{hint}
      $\frac{a}{x^n} \to 0$ as $x \to \infty$.
    \end{hint}
		The value of the limit is
		 $\answer[given]{-\frac15}$
		
\end{problem}




\begin{example} %example 3
Compute $\displaystyle{\lim_{x \to -\infty} \frac{4x^2 - 3x - 6}{x^3 +8x^2 -3x + 7}}.$

First note that as $x\to \pm \infty$, a polynomial, $p(x) \to \pm \infty$ according to it leading term.
In this example, the lead terms $4x^2$ and $x^3$  go to $\infty$ and $-\infty$ respectively, as $x \to -\infty$. 
Hence, our limit has the indeterminate form
$\frac{\infty}{-\infty}$.
To resolve this indeterminate form, we will factor out 
the highest power of $x$ in the denominator, namely $x^3$, from both the numerator and the denominator.  
In the numerator we get, 
\[4x^2 - 3x - 6 = x^3(\frac{4}{x} - \frac{3}{x^2} - \frac{6}{x^3})\]
and in the denominator we get,
\[x^3 +8x^2 -3x + 7 = x^3(1 +  \frac{8}{x} - \frac{3}{x^2} + \frac{7}{x^3}).\]

Factoring the $x^3$ and canceling, the limit can be resolved as follows:

\begin{align*}
\lim_{x \to -\infty}\frac{4x^2 - 3x - 6}{x^3 +8x^2 -3x + 7} &= 
\lim_{x \to -\infty} \frac{x^3(\frac{4}{x}-\frac{3}{x^2} -\frac{6}{x^3})}
{x^3(1 + \frac{8}{x}-\frac{3}{x^2}+\frac{7}{x^3})} \\ \\
&=\lim_{x \to -\infty} \frac{\frac{4}{x}-\frac{3}{x^2} -\frac{6}{x^3}}{1 + \frac{8}{x}-\frac{3}{x^2}+\frac{7}{x^3}} \\ \\
&=\frac{0 - 0 - 0}{1 +0- 0 + 0} \\ \\
&= \frac01 \\
&= 0.
\end{align*}

The result of this limit means that the line $y = 0$ (the $x$-axis) is a horizontal asymptote
for the graph of $y = \dfrac{4x^2 - 3x - 6}{x^3 +8x^2 -3x + 7}$ on the right end.
\end{example}


\begin{problem} %problem #3
  Compute the limit:
  \[
  \lim_{x \to -\infty} \frac{x^2 - 5x - 6}{2x^3 + x -5}
  \]
  
    \begin{hint}
      When you `plug in' $x = \infty$, you get an $\frac{\infty}{\infty}$ form
    \end{hint}
    \begin{hint}
      Determine the highest power of $x$ in the denominator
    \end{hint}
    \begin{hint}
      Factor this power from both the numerator and denominator
    \end{hint}
    \begin{hint}
      Cancel the power of $x$ that was factored out
    \end{hint}
    \begin{hint}
      $\frac{a}{x^n} \to 0$ as $x \to -\infty$.
    \end{hint}
		The value of the limit is
		 $\answer[given]{0}$
		
\end{problem}


\begin{problem} %problem #3
  Compute the limit:
  \[
  \lim_{x \to \infty} \frac{3x^3 - 5x + 6}{3-x+ 2x^3- x^4}
  \]
  
    \begin{hint}
      When you `plug in' $x = \infty$, you get an$\frac{\infty}{\infty}$ form
    \end{hint}
    \begin{hint}
      Determine the highest power of $x$ in the denominator
    \end{hint}
    \begin{hint}
      Factor this power from both the numerator and denominator
    \end{hint}
    \begin{hint}
      Cancel the power of $x$ that was factored out
    \end{hint}
    \begin{hint}
      $\frac{a}{x^n} \to 0$ as $x \to \infty$.
    \end{hint}
		The value of the limit is
		 $\answer[given]{0}$
		
\end{problem}


\begin{problem} %problem #3
  Compute the limit:
  \[
  \lim_{x \to -\infty} \frac{x^3 - x + 1}{2x^2- 3}
  \]
  
    \begin{hint}
      When you `plug in' $x = \infty$, you get an $\frac{\infty}{\infty}$ form
    \end{hint}
    \begin{hint}
      Determine the highest power of $x$ in the denominator
    \end{hint}
    \begin{hint}
      Factor this power from both the numerator and denominator
    \end{hint}
    \begin{hint}
      Cancel the power of $x$ that was factored out
    \end{hint}
    \begin{hint}
      $\frac{a}{x^n} \to 0$ as $x \to \infty$.
    \end{hint}
		The value of the limit is
		(type infinity for $\infty$, -infinity for $-\infty$ or DNE)
		 $\answer[given]{-\infty}$
		
\end{problem}


\begin{example}

Find 
\[
\lim_{x\to -\infty} e^x.
\]

Suppose $x$ is a large negative number.  Since $x$ is negative, $x = -|x|$ and so
\[
e^x = e^{-|x|} = \frac{1}{e^{|x|}},
\]
using the definition of negative exponents: $x^{-n} = \frac{1}{x^n}$.

Now note that as $x\to -\infty, |x| \to \infty$ and so 
\[
e^{|x|} \to \infty.
\]

Thus, 
\[
\frac{1}{e^{|x|}} \to 0,
\]
hence

\[
\lim_{x\to -\infty} e^x = 0.
\]

This means that the line $y=0$ is a horizontal asymptote of the graph of $y = e^x$,
as shown in the graph below.

\[
\graph{y=e^x}
\]

\end{example}

\begin{problem}
Compute 
\[ \lim_{x \to \infty} e^{-2x}.\]

\begin{hint}
$e^x \to 0$ as $x \to -\infty$
\end{hint}

The value of the limit is
		(type infinity for $\infty$, -infinity for $-\infty$ or DNE)
		 $\answer[given]{0}$
\end{problem}


\begin{example}
Discuss
\[
\lim_{x\to \infty} \sin(x).
\]

As $x$ increases, the sine function oscillates between $-1$ and $1$ without approaching any particular value.
Hence
\[
\lim_{x\to \infty} \sin(x) \ \text{DNE}
\]
due to oscillation.
This is illustrated in the graph of $y=\sin(x)$ shown below.

\[
\graph{y=\sin(x)}
\]

\end{example}

\begin{problem}
Compute 
\[ \lim_{x \to -\infty} \cos(x).\]

\begin{hint}
$\cos(x)$ oscillates between $-1$ and $1$ with period $2\pi$
\end{hint}

The value of the limit is
		(type infinity for $\infty$, -infinity for $-\infty$ or DNE)
		 $\answer[given]{DNE}$
\end{problem}

\begin{center}
\begin{foldable}
\unfoldable{Here is a detailed, lecture style video on finding limits at infinity:}
\youtube{FyhXNtLGr2o}
\end{foldable}
\end{center}



\end{document}


\begin{example} %example 9
Analyze the limit:
\[
\lim_{x\to -1} 
\frac{1}{x^2}
\]
Plugging $x = 0$ into the rational function
\[f(x) = \frac{1}{x^2}\]
gives the undefined expression $\frac{1}{0}$. From this information, we can conclude that the graph 
of the function has a vertical asymptote at $x = 0$. This means that the one-sided limits as $x$ approaches 0
will give either $\infty$ or $-\infty$, i.e., 
\[\lim_{x \to -1^-} \frac{1}{x^2}= \infty \ \text{or} \ -\infty.\]
and
\[\lim_{x \to -1^+} \frac{1}{x^2}= \infty \ \text{or} \ -\infty.\]
To determine which, we will do a sign analysis as follows. Because of the perfect square, 
we can consider both case simultaneously.
The numerator is 1, which is positive and the denominator is $x^2$ which is positive whether $x \to 0^-$
or $x \to 0^+$.  Hence 
\[\lim_{x \to 0^-} \frac{1}{x^2} =\frac{\text{pos}}{\text{pos}} = \infty,\]
and
\[\lim_{x \to 0^+} \frac{1}{x^2} =\frac{\text{pos}}{\text{pos}} = \infty,\]
since the choices were only $\infty$ and $-\infty$.
The graph of $f(x) = \frac{1}{x^2}$ near $x = 0$ looks like this:

\begin{center}
\begin{tikzpicture}
\begin{axis}[axis lines = center, xlabel = $x$,
    ylabel = {$f(x)$}]
\addplot[domain=-2:-0.1, 
    samples=100, color=blue]{1/x^2};
\addplot[domain=0.1:2, 
    samples=100, color=blue]{1/x^2};
\end{axis}
\end{tikzpicture}
\end{center}
\end{example}


Here is a cool truth table

\begin{problem}
Fill in the truth table below using your amazing logic skillz!

\begin{prompt}
\begin{center}
\[
\begin{array}{c|c|c|c}
		p & q & p \implies q & p \vee (p \implies q) \\
		\hline
		T & T & T & \answer{T} \\
		T & F & F & \answer{T} \\
		F & T & T & \answer{T} \\
		F & F & T & \answer{T}
	\end{array}
    \]
\end{center}
\end{prompt}
\end{problem}

\begin{center}
\begin{tikzpicture}
\begin{axis}[axis lines = center, xlabel = $x$,
    ylabel = {$f(x)$}]
\addplot[domain= 0.5:100, 
    samples=1000, color=blue]{1/x};

\end{axis}
\end{tikzpicture}
\end{center}
