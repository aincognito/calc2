\documentclass[handout]{ximera}
\usepgfplotslibrary{fillbetween}
%% You can put user macros here
%% However, you cannot make new environments



\newcommand{\ffrac}[2]{\frac{\text{\footnotesize $#1$}}{\text{\footnotesize $#2$}}}
\newcommand{\vasymptote}[2][]{
    \draw [densely dashed,#1] ({rel axis cs:0,0} -| {axis cs:#2,0}) -- ({rel axis cs:0,1} -| {axis cs:#2,0});
}


\graphicspath{{./}{firstExample/}}

\usepackage{amsmath}
\usepackage{amssymb}
\usepackage{array}
\usepackage[makeroom]{cancel} %% for strike outs
\usepackage{pgffor} %% required for integral for loops
\usepackage{tikz}
\usepackage{tikz-cd}
\usepackage{tkz-euclide}
\usetikzlibrary{shapes.multipart}


\usetkzobj{all}
\tikzstyle geometryDiagrams=[ultra thick,color=blue!50!black]


\usetikzlibrary{arrows}
\tikzset{>=stealth,commutative diagrams/.cd,
  arrow style=tikz,diagrams={>=stealth}} %% cool arrow head
\tikzset{shorten <>/.style={ shorten >=#1, shorten <=#1 } } %% allows shorter vectors

\usetikzlibrary{backgrounds} %% for boxes around graphs
\usetikzlibrary{shapes,positioning}  %% Clouds and stars
\usetikzlibrary{matrix} %% for matrix
\usepgfplotslibrary{polar} %% for polar plots
\usepgfplotslibrary{fillbetween} %% to shade area between curves in TikZ



%\usepackage[width=4.375in, height=7.0in, top=1.0in, papersize={5.5in,8.5in}]{geometry}
%\usepackage[pdftex]{graphicx}
%\usepackage{tipa}
%\usepackage{txfonts}
%\usepackage{textcomp}
%\usepackage{amsthm}
%\usepackage{xy}
%\usepackage{fancyhdr}
%\usepackage{xcolor}
%\usepackage{mathtools} %% for pretty underbrace % Breaks Ximera
%\usepackage{multicol}



\newcommand{\RR}{\mathbb R}
\newcommand{\R}{\mathbb R}
\newcommand{\C}{\mathbb C}
\newcommand{\N}{\mathbb N}
\newcommand{\Z}{\mathbb Z}
\newcommand{\dis}{\displaystyle}
%\renewcommand{\d}{\,d\!}
\renewcommand{\d}{\mathop{}\!d}
\newcommand{\dd}[2][]{\frac{\d #1}{\d #2}}
\newcommand{\pp}[2][]{\frac{\partial #1}{\partial #2}}
\renewcommand{\l}{\ell}
\newcommand{\ddx}{\frac{d}{\d x}}

\newcommand{\zeroOverZero}{\ensuremath{\boldsymbol{\tfrac{0}{0}}}}
\newcommand{\inftyOverInfty}{\ensuremath{\boldsymbol{\tfrac{\infty}{\infty}}}}
\newcommand{\zeroOverInfty}{\ensuremath{\boldsymbol{\tfrac{0}{\infty}}}}
\newcommand{\zeroTimesInfty}{\ensuremath{\small\boldsymbol{0\cdot \infty}}}
\newcommand{\inftyMinusInfty}{\ensuremath{\small\boldsymbol{\infty - \infty}}}
\newcommand{\oneToInfty}{\ensuremath{\boldsymbol{1^\infty}}}
\newcommand{\zeroToZero}{\ensuremath{\boldsymbol{0^0}}}
\newcommand{\inftyToZero}{\ensuremath{\boldsymbol{\infty^0}}}


\newcommand{\numOverZero}{\ensuremath{\boldsymbol{\tfrac{\#}{0}}}}
\newcommand{\dfn}{\textbf}
%\newcommand{\unit}{\,\mathrm}
\newcommand{\unit}{\mathop{}\!\mathrm}
%\newcommand{\eval}[1]{\bigg[ #1 \bigg]}
\newcommand{\eval}[1]{ #1 \bigg|}
\newcommand{\seq}[1]{\left( #1 \right)}
\renewcommand{\epsilon}{\varepsilon}
\renewcommand{\iff}{\Leftrightarrow}

\DeclareMathOperator{\arccot}{arccot}
\DeclareMathOperator{\arcsec}{arcsec}
\DeclareMathOperator{\arccsc}{arccsc}
\DeclareMathOperator{\si}{Si}
\DeclareMathOperator{\proj}{proj}
\DeclareMathOperator{\scal}{scal}
\DeclareMathOperator{\cis}{cis}
\DeclareMathOperator{\Arg}{Arg}
%\DeclareMathOperator{\arg}{arg}
\DeclareMathOperator{\Rep}{Re}
\DeclareMathOperator{\Imp}{Im}
\DeclareMathOperator{\sech}{sech}
\DeclareMathOperator{\csch}{csch}
\DeclareMathOperator{\Log}{Log}

\newcommand{\tightoverset}[2]{% for arrow vec
  \mathop{#2}\limits^{\vbox to -.5ex{\kern-0.75ex\hbox{$#1$}\vss}}}
\newcommand{\arrowvec}{\overrightarrow}
\renewcommand{\vec}{\mathbf}
\newcommand{\veci}{{\boldsymbol{\hat{\imath}}}}
\newcommand{\vecj}{{\boldsymbol{\hat{\jmath}}}}
\newcommand{\veck}{{\boldsymbol{\hat{k}}}}
\newcommand{\vecl}{\boldsymbol{\l}}
\newcommand{\utan}{\vec{\hat{t}}}
\newcommand{\unormal}{\vec{\hat{n}}}
\newcommand{\ubinormal}{\vec{\hat{b}}}

\newcommand{\dotp}{\bullet}
\newcommand{\cross}{\boldsymbol\times}
\newcommand{\grad}{\boldsymbol\nabla}
\newcommand{\divergence}{\grad\dotp}
\newcommand{\curl}{\grad\cross}
%% Simple horiz vectors
\renewcommand{\vector}[1]{\left\langle #1\right\rangle}


\outcome{Prepare for the final exam.}

\title{Final Exam Review}

%\newcommand{\ffrac}[2]{\frac{\mbox{\footnotesize $#1$}}{\mbox{\footnotesize $#2$}}}
%\newcommand{\vasymptote}[2][]{\draw [densely dashed,#1] 
%({rel axis cs:0,0} -| {axis cs:#2,0}) -- ({rel axis cs:0,1} -| {axis cs:#2,0});}


\begin{document}

\begin{abstract}
In this section we prepare for the final exam.
\end{abstract}

\maketitle



\section{Limits}


\begin{problem}
\[\lim_{x \to 2} \frac{x^2 - 4}{x^2 - 5x + 6} = \answer{-4}.\]

\[\lim_{x \to 3} \frac{x^2 -x -6}{x^2 - 9} = \answer{5/6}.\]
\end{problem}


\begin{problem}
\[\lim_{x \to 4} \frac{\sqrt x - 2}{x^2 - 5x + 4} = \answer{1/12}.\]

\[\lim_{x \to 9} \frac{\sqrt x - 3}{x^2 - 8x - 9} = \answer{1/60}.\]
\end{problem}

\begin{problem}
\[\lim_{x \to 2} \frac{\frac{1}{x} - \frac12 }{x- 2} = \answer{-1/4}.\]

\[\lim_{x \to {-3}} \frac{\frac{1}{x} + \frac13 }{x+ 3} = \answer{-1/9}.\]
\end{problem}

\begin{problem}
\[\lim_{x \to \infty} \frac{x^2 - 5x + 4}{2x^2 + 4x + 7} = \answer{1/2}.\]

\[\lim_{x \to -\infty} \frac{3x^2 + x - 2}{x^3 -3x + 1} = \answer{0}.\]
\end{problem}

\begin{problem}
\[\lim_{x \to 4^+} \frac{2x}{x-4} = \answer{\infty}.\]

\[\lim_{x \to 2^-} \frac{x+1}{x^2 - 4} = \answer{-\infty}.\]
\end{problem}



\begin{problem}
\[\lim_{x \to 0} \frac{\sin(3x)}{x} = \answer{3}.\]

\[\lim_{x \to 0} \frac{e^{2x} -1}{x} = \answer{2}.\]
\end{problem}

\begin{problem}
\[\lim_{x \to \infty} \frac{\ln(x)}{x} = \answer{0}.\]

\[\lim_{x \to \infty} \frac{e^x}{x^2} = \answer{\infty}.\]
\end{problem}

\section{Derivatives}

\begin{problem}
\[\frac{d}{dx}\left(5x^4 - 7x^2 + 3x -3  \right) = \answer{20x^3 - 14x + 3}.\]
\end{problem}


\begin{problem}
\[\frac{d}{dx}\left(x^3 + \frac{1}{x^3} + \sqrt[3]x  \right) = \answer{3x^2 -3x^{-4} +(1/3)x^{-2/3}}.\]
\end{problem}


\begin{problem}
\[\frac{d}{dx}\big[\tan(x) + \cot(x)  \big] = \answer{\sec^2(x) - \csc^2(x)}.\]

\[\frac{d}{dx}\big[\sec(x) + \csc(x)  \big] = \answer{\sec(x)\tan(x) - \csc(x)\cot(x)}.\]
\end{problem}

\begin{problem}
\[\frac{d}{dx}\left(2^x + e^x  \right) = \answer{2^x \ln(2) + e^x}.\]
\end{problem}

\begin{problem}
\[\frac{d}{dx}\big[\tan^{-1}(x) + \sin^{-1}(x)  \big] = 
\answer{\frac{1}{1+x^2} + \frac{1}{(1-x^2)^{1/2}}}.\]
\end{problem}

\begin{problem}
\[\frac{d}{dx}\left(x^2 \sin(x) + \sin(x^2) \right) = \answer{x^2\cos(x) + 2x\sin(x) + 2x\cos(x^2)}.\]
\end{problem}


\begin{problem}
\[\frac{d}{dx}(x^3 + 2x^2 - 5x - 4 )^6 = \answer{6(x^3 + 2x^2 - 5x - 4 )^5(3x^2 + 4x -5)}.\]
\end{problem}




\begin{problem}
\[\frac{d}{dx}\left(x^3 \ln(x) \right) = \answer{x^2 + 3x^2\ln(x)}.\]
\end{problem}

\begin{problem}
\[\frac{d}{dx}\left(e^{3x} \tan(2x)\right) = \answer{3e^{3x} \tan(2x) + 2e^{3x}\sec^2(2x)}.\]
\end{problem}

\begin{problem}
\[\frac{d}{dx}\left(\frac{2x+5}{3x - 4}\right) = \answer{\frac{-23}{(3x-4)^2}}.\]

\[\frac{d}{dx}\left(\frac{x}{x^2 + 1}\right) = \answer{\frac{1 - x^2}{(x^2 + 1)^2}}.\]
\end{problem}

\begin{problem}
Find the equation of the tangent line to the graph of $y=\sqrt{x}$ at $x = 16$, and use it to approximate $\sqrt{18}$.\\
The equation is $y = \answer{x/8 + 2}$ and $\sqrt{18} \approx \answer{4.25}$.


\end{problem}

\begin{problem}
Find the equation of the tangent line to the graph of $y=1/x$ at $x = 3$.\\
The equation is $y = \answer{-x/9 +2/3}$.
\end{problem}


\section{Integrals}

\begin{problem}
\[\int\left(x^2 - 5x + 3  \right) \ dx = \answer{x^3/3 - 5x^2 /2 + 3x} + C.\]
\end{problem}

\begin{problem}
\[\int\left(x^3 + \frac{1}{x^3} + \sqrt[3]x  \right) \ dx = \answer{x^4/4 +x^{-2}/{-2} + 
(3/4)x^{4/3}} + C.\]
\end{problem}

\begin{problem}
\[\int\big[2\sin(x) - 3\cos(x)  \big] \ dx = \answer{-2\cos(x) -3\sin(x)} + C.\]
\end{problem}

\begin{problem}
\[\int\big[4\sec^2(x) + 5 \csc^2(x) \big] \ dx = \answer{4\tan(x) - 5 \cot(x)} + C.\]
\end{problem}

\begin{problem}
\[\int x^2 \cos(x^3 + 1) \ dx = \answer{(1/3) \sin(x^3 +1)} + C.\]
\end{problem}

\begin{problem}
\[\int \frac{\ln^3(x)}{x} \ dx = \answer{\ln^4(x) /4} + C.\]
\end{problem}

\begin{problem}
\[\int_0^2  \left(x(x+1) + 1 \right) \ dx = \answer{20/3}.\]
\end{problem}

\begin{problem}
\[\int_0^\pi  \sin(2x) \ dx = \answer{0}.\]
\end{problem}

\begin{problem}
Find the area under the graph of $y = \sqrt x$ from $x=0$ to $x=16$.\\
The area is $\answer{128/3}$.
\end{problem}

\begin{problem}
Find the area under the graph of $y = 1/x$ from $x=1$ to $x=3$.\\
The area is $\answer{\ln(3)}$.
\end{problem}


\section{FTC, Part II}

\begin{problem}
\[\frac{d}{dx} \int_0^x \sin(2t^2) \ dt = \answer{\sin(2x^2)}.\]
\end{problem}

\begin{problem}
\[\frac{d}{dx} \int_x^0 \sin(2t^2) \ dt = \answer{-\sin(2x^2)}.\]
\end{problem}

\begin{problem}
\[\frac{d}{dx} \int_0^{3x} \sin(2t^2) \ dt = \answer{3\sin(18x^2)}.\]
\end{problem}

\begin{problem}
\[\frac{d}{dx} \int_0^{x^3} \sin(2t^2) \ dt = \answer{3x^2\sin(2x^6)}.\]
\end{problem}



\section{Implicit Differentiation}

\begin{problem}
Find $\dfrac{dy}{dx}$ if $x^3 + y^3 = 6xy$.
\[\frac{dy}{dx} = \answer{\frac{2y - x^2}{y^2 - 2x}}.\]
\end{problem}

\begin{problem}
Find $\dfrac{dy}{dx}$ if $ y^4 = x^4 - xy$.
\[\frac{dy}{dx} = \answer{\frac{4x^3 - y}{4y^3 +x}}.\]
\end{problem}

\section{Related Rates}
\begin{problem}
The radius of a circle is increasing at a rate of $2$ cm/min. 
How fast is the area increasing when the radius is $10$ cm?
\[\text{The area is increasing at a rate of} \ \answer{40\pi} \ \text{cm$^2$/min}.\]
\end{problem}

\begin{problem}
The radius of a sphere is increasing at a rate of $2$ cm/min. 
How fast is the volume increasing when the radius is $10$ cm? $(V= \frac43 \pi r^3)$
\[\text{The volume is increasing at a rate of} \ \answer{800\pi}  \text{cm$^3$/min}.\]
\end{problem}


\begin{problem}
The short leg of a right triangle is increasing at a rate of $2$ cm/min and 
the long leg is increasing at a rate of $3$ cm/min.
How fast is the hypotenuse increasing when the short 
leg is $12$ cm and the long leg is $16$ cm?
\[\text{The hypotenuse is increasing at a rate of} \ \answer{3.6} \text{cm/min}.\]
\end{problem}



\section{Optimization}

\begin{problem}
A farmer has 1200 feet of fence with which to fence in a rectangular pasture.  Moreover, he would like to partition the pasture into two plots with fence parallel to the bottom side (horizontal). What dimensions will maximize the total area of his enclosed pastures?\\
The optimal length (horizontal) is \ $\answer{200}$ feet and\\
the optimal width (vertical) is $\answer{300}$ feet.
\end{problem}

\begin{problem}
A fifth grader has $1200$ cm$^2$ of cardboard with which to construct a box with a square base and an open top. What is the maximum possible volume for his box?\\
The maximum possible volume is $\answer{4000} \  \text{cm}^3$.
\end{problem}

\section{Max/Min, Inc/Dec and Concavity}

\begin{problem}
Find the absolute maximum and the absolute minimum of the function $f(x) = x^3 - 3x+ 5$
on the interval $[0, 2]$.\\
The absolute maximum is $\answer{7}$ occurring at $x = \answer{2}$.\\
The absolute minimum is $\answer{3}$ occurring at $x = \answer{1}$.\\
\end{problem}

\begin{problem}
Find the absolute maximum and the absolute minimum of the function $f(x) = \sin(x) + \cos(x)$
on the interval $[0, \pi]$.\\
The absolute maximum is $\answer{\sqrt 2}$ occurring at $x = \answer{\pi/4}$.\\
The absolute minimum is $\answer{-1}$ occurring at $x = \answer{\pi}$.
\end{problem}


\begin{problem}
The function $f(x) = x^3 - 27x$ is increasing on the interval(s):
\begin{multipleChoice}
\choice[correct]{$(-\infty, -3)$ and $(3, \infty)$} 
\choice{$(-\infty, -3)$ only}
\choice{ $(3, \infty)$ only}
\choice{$(-3, 3)$}
\end{multipleChoice}
\end{problem}


\begin{problem}
The function $f(x) = x^3 - 27x$ is concave down on the interval(s):
\begin{multipleChoice}
\choice[correct]{$(-\infty, 0)$} 
\choice{$(-\infty, 6)$}
\choice{ $(0, \infty)$}
\choice{$(0, 6)$}
\end{multipleChoice}
\end{problem}


\section{Rectilinear Motion}

\begin{problem}
A projectile is launched vertically.  Its height (in feet) after $t$ seconds is given by
$s(t) = 12 + 64t -16t^2$.\\

The maximum height of the projectile is $\answer{76} \ \text{feet}$.\\
The velocity of the object when it is $60$ feet up and falling is $\answer{-32} \ \text{ft/sec}$.
\end{problem}

\begin{problem}
A projectile is launched vertically.  Its height (in feet) after $t$ seconds is given by
$s(t) = 8 + 48t -16t^2$.\\

The maximum height of the projectile is $\answer{44} \ \text{feet}$.\\
The velocity of the object when it is $40$ feet up and rising is $\answer{16} \ \text{ft/sec}$.
\end{problem}


\section{Mean Value Theorem}

\begin{problem}
Find the value of `$c$' from the Mean Value Theorem for the function $f(x) = 1/x^2$ on the interval 
$[1,2]$
\[c=\answer{(8/3)^{1/3}}.\]
\end{problem}

\begin{problem}
Show that there is no value of `$c$' from the Mean Value Theorem for the function 
$f(x) = 1/x^2$ on the interval $[-1, 1]$.  Why does this NOT contradict the statement of the theorem?
\end{problem}

\section{Area}


\begin{problem}
Find the exact area under the graph of $y = e^{2x}$ from $x = 0$ to $x = 1$. \\
\begin{hint}
Area under the curve $= \int_a^b f(x) \ dx$
\end{hint}

\[\text{area under the curve} = \answer{(e^2 -1)/2}.\]
\end{problem}


\begin{problem}
Find the exact area above the graph of $y = 1 - \sqrt x$ from $x = 1$ to $x = 4$. \\
\begin{hint}
Area above the curve $= -\int_a^b f(x) \ dx$
\end{hint}
\[\int_1^4 1 - \sqrt x \ dx = \answer{-5/3},\]
\[\text{area above the curve} = \answer{5/3}.\]
\end{problem}



\end{document}
