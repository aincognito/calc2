\documentclass{ximera}

%% You can put user macros here
%% However, you cannot make new environments



\newcommand{\ffrac}[2]{\frac{\text{\footnotesize $#1$}}{\text{\footnotesize $#2$}}}
\newcommand{\vasymptote}[2][]{
    \draw [densely dashed,#1] ({rel axis cs:0,0} -| {axis cs:#2,0}) -- ({rel axis cs:0,1} -| {axis cs:#2,0});
}


\graphicspath{{./}{firstExample/}}

\usepackage{amsmath}
\usepackage{amssymb}
\usepackage{array}
\usepackage[makeroom]{cancel} %% for strike outs
\usepackage{pgffor} %% required for integral for loops
\usepackage{tikz}
\usepackage{tikz-cd}
\usepackage{tkz-euclide}
\usetikzlibrary{shapes.multipart}


\usetkzobj{all}
\tikzstyle geometryDiagrams=[ultra thick,color=blue!50!black]


\usetikzlibrary{arrows}
\tikzset{>=stealth,commutative diagrams/.cd,
  arrow style=tikz,diagrams={>=stealth}} %% cool arrow head
\tikzset{shorten <>/.style={ shorten >=#1, shorten <=#1 } } %% allows shorter vectors

\usetikzlibrary{backgrounds} %% for boxes around graphs
\usetikzlibrary{shapes,positioning}  %% Clouds and stars
\usetikzlibrary{matrix} %% for matrix
\usepgfplotslibrary{polar} %% for polar plots
\usepgfplotslibrary{fillbetween} %% to shade area between curves in TikZ



%\usepackage[width=4.375in, height=7.0in, top=1.0in, papersize={5.5in,8.5in}]{geometry}
%\usepackage[pdftex]{graphicx}
%\usepackage{tipa}
%\usepackage{txfonts}
%\usepackage{textcomp}
%\usepackage{amsthm}
%\usepackage{xy}
%\usepackage{fancyhdr}
%\usepackage{xcolor}
%\usepackage{mathtools} %% for pretty underbrace % Breaks Ximera
%\usepackage{multicol}



\newcommand{\RR}{\mathbb R}
\newcommand{\R}{\mathbb R}
\newcommand{\C}{\mathbb C}
\newcommand{\N}{\mathbb N}
\newcommand{\Z}{\mathbb Z}
\newcommand{\dis}{\displaystyle}
%\renewcommand{\d}{\,d\!}
\renewcommand{\d}{\mathop{}\!d}
\newcommand{\dd}[2][]{\frac{\d #1}{\d #2}}
\newcommand{\pp}[2][]{\frac{\partial #1}{\partial #2}}
\renewcommand{\l}{\ell}
\newcommand{\ddx}{\frac{d}{\d x}}

\newcommand{\zeroOverZero}{\ensuremath{\boldsymbol{\tfrac{0}{0}}}}
\newcommand{\inftyOverInfty}{\ensuremath{\boldsymbol{\tfrac{\infty}{\infty}}}}
\newcommand{\zeroOverInfty}{\ensuremath{\boldsymbol{\tfrac{0}{\infty}}}}
\newcommand{\zeroTimesInfty}{\ensuremath{\small\boldsymbol{0\cdot \infty}}}
\newcommand{\inftyMinusInfty}{\ensuremath{\small\boldsymbol{\infty - \infty}}}
\newcommand{\oneToInfty}{\ensuremath{\boldsymbol{1^\infty}}}
\newcommand{\zeroToZero}{\ensuremath{\boldsymbol{0^0}}}
\newcommand{\inftyToZero}{\ensuremath{\boldsymbol{\infty^0}}}


\newcommand{\numOverZero}{\ensuremath{\boldsymbol{\tfrac{\#}{0}}}}
\newcommand{\dfn}{\textbf}
%\newcommand{\unit}{\,\mathrm}
\newcommand{\unit}{\mathop{}\!\mathrm}
%\newcommand{\eval}[1]{\bigg[ #1 \bigg]}
\newcommand{\eval}[1]{ #1 \bigg|}
\newcommand{\seq}[1]{\left( #1 \right)}
\renewcommand{\epsilon}{\varepsilon}
\renewcommand{\iff}{\Leftrightarrow}

\DeclareMathOperator{\arccot}{arccot}
\DeclareMathOperator{\arcsec}{arcsec}
\DeclareMathOperator{\arccsc}{arccsc}
\DeclareMathOperator{\si}{Si}
\DeclareMathOperator{\proj}{proj}
\DeclareMathOperator{\scal}{scal}
\DeclareMathOperator{\cis}{cis}
\DeclareMathOperator{\Arg}{Arg}
%\DeclareMathOperator{\arg}{arg}
\DeclareMathOperator{\Rep}{Re}
\DeclareMathOperator{\Imp}{Im}
\DeclareMathOperator{\sech}{sech}
\DeclareMathOperator{\csch}{csch}
\DeclareMathOperator{\Log}{Log}

\newcommand{\tightoverset}[2]{% for arrow vec
  \mathop{#2}\limits^{\vbox to -.5ex{\kern-0.75ex\hbox{$#1$}\vss}}}
\newcommand{\arrowvec}{\overrightarrow}
\renewcommand{\vec}{\mathbf}
\newcommand{\veci}{{\boldsymbol{\hat{\imath}}}}
\newcommand{\vecj}{{\boldsymbol{\hat{\jmath}}}}
\newcommand{\veck}{{\boldsymbol{\hat{k}}}}
\newcommand{\vecl}{\boldsymbol{\l}}
\newcommand{\utan}{\vec{\hat{t}}}
\newcommand{\unormal}{\vec{\hat{n}}}
\newcommand{\ubinormal}{\vec{\hat{b}}}

\newcommand{\dotp}{\bullet}
\newcommand{\cross}{\boldsymbol\times}
\newcommand{\grad}{\boldsymbol\nabla}
\newcommand{\divergence}{\grad\dotp}
\newcommand{\curl}{\grad\cross}
%% Simple horiz vectors
\renewcommand{\vector}[1]{\left\langle #1\right\rangle}


\outcome{Find limits using the Squeeze Theorem}

\title{1.7 Squeeze Theorem}


\begin{document}

\begin{abstract}
Find limits using the Squeeze Theorem.
\end{abstract}

\maketitle

\section{Squeeze Theorem}
 

In this section we find limits using the Squeeze Theorem.

\begin{theorem}[The Squeeze Theorem] 
Suppose that the compound inequality
\[
g_1(x) \leq f(x) \leq g_2(x)
\]
holds for all values of $x$ in some open interval about $x=a$, except possibly for $a$ itself.
If
\[
\lim_{x\to a} g_1(x) = L \quad \text{and} \quad \lim_{x\to a} g_2(x) = L,
\]
then we can conclude that
\[
\lim_{x\to a} f(x) = L
\]
as well.

\end{theorem}


\begin{remark}
If the limits
\[
\lim_{x\to a} g_1(x)  \quad \text{and} \quad \lim_{x\to a} g_2(x)
\]
are different (or DNE), then we can make no conclusion about the limit
\[
\lim_{x\to a} f(x).
\]
\end{remark}

\begin{example}[example 1]
Suppose 
\[
 2x-1 \leq f(x) \leq x^2
\]
for all $x$ except $x=1$.
Find
\[
\lim_{x\to 1} f(x).
\]

Since
\[
\lim_{x\to 1} (2x-1) = 1
\]
and
\[
\lim_{x\to 1} x^2 = 1
\]
we can use the Squeeze Theorem to conclude that
\[
\lim_{x\to 1} f(x) = 1
\]
as well.

\end{example}

\begin{problem}(problem 1)
Suppose that 
\[
 5-x^2\leq f(x) \leq 9-4x
\]
for all $x$ except $x=2$.
Find
\[
\lim_{x\to 2} f(x).
\]
First, we find the limits of the bounds:
\[
\lim_{x \to 2} 5 - x^2 = \answer{1} \quad \text{and} \quad  \lim_{x \to 2} 9-4x = \answer{1}.
\]
Since these answers are the same, the Squeeze Theorem allows us to conclude that 
\[
\lim_{x \to 2} f(x) = \answer{1} \quad \text{as well}.
\]

\end{problem}

\begin{example}[example 2]
Find
\[
\lim_{x\to 0} x^2 \sin\Big(\frac{1}{x}\Big).
\]
Since $\frac{1}{0}$ is undefined, plugging in $x=0$
does not give a definitive answer.
Using the fact that 
\[
-1 \leq \sin(\theta) \leq 1
\]
for all values of $\theta$, we can create a compound inequality for
the function
\[
f(x) = x^2 \sin\Big(\frac{1}{x}\Big)
\]
and find the limit using the Squeeze Theorem.
To begin, note that
\[
-1 \leq \sin\Big(\frac{1}{x}\Big) \leq 1
\]
for all values of $x$ except $x=0$.
Multiplying this compound inequality by the non-negative quantity, $x^2$,
we have
\[
-x^2 \leq x^2\sin\Big(\frac{1}{x}\Big) \leq x^2
\]
for all values of $x$ except $x=0$.
Next, note that
\[
\lim_{x\to 0} -x^2 = 0
\]
and
\[
\lim_{x\to 0} x^2 = 0.
\]
Finally, by the Squeeze Theorem, we can conclude that
\[
\lim_{x\to 0} x^2\sin\Big(\frac{1}{x}\Big) = 0
\]
as well. The graph below also shows that the limit is zero.
Zoom in on the origin to get the full effect.

\[
\graph{x^2, -x^2, x^2 \sin(1/x)}
\]

\end{example}

\begin{problem}(problem 2)
Find
\[
\lim_{x \to 0} x^4 \cos(2/x)
\]
using the Squeeze Theorem.\\
First, we need to find bounds. Since $-1 \leq \cos \theta \leq 1$ for all $\theta$, 
\[
\answer{-x^4} \; \leq x^4\cos(2/x) \leq \; \answer{x^4}
\]
for all $x$ except $x=0$.
Next, we need to find the limits of those bounds:
\[
\lim_{x \to 0}  - x^4 = \answer{0} \quad \text{and} \quad  \lim_{x \to 0} x^4 = \answer{0}.
\]
Since these answers are the same, the Squeeze Theorem allows us to conclude that 
\[
\lim_{x \to 0} x^4\cos(2/x) = \answer{0} \quad \text{as well}.
\]

\end{problem}

The Squeeze Theorem can also be used if $x \to \pm \infty$.

\begin{example}[example 3]
Find
\[
\lim_{x \to \infty} \frac{\sin x}{x}
\]
using the Squeeze Theorem.\\
First, we need to find bounds. Since $-1 \leq \sin \theta \leq 1$ for all $\theta$, 
\[
\answer{-1/x} \; \leq \frac{\sin x}{x} \leq \;  \answer{1/x}
\]
for all $x>0$.
Next, we need to find the limits of those bounds:
\[
\lim_{x \to \infty}  -\frac{1}{x} = \answer{0} \quad \text{and} \quad \lim_{x \to \infty} \frac{1}{x} = \answer{0}.
\]
Since these answers are the same, the Squeeze Theorem allows us to conclude that 
\[
\lim_{x \to \infty} \frac{\sin x}{x} = \answer{0} \text{as well}.
\]

\end{example}

We conclude this section by using the Squeeze Theorem to find a special limit, 
\[
\lim_{\theta \to 0} \frac{\sin(\theta)}{\theta}.
\]
Consider the figure below.  It consists of a small triangle, a sector of a circle of radius,  $r = 1$, and a large triangle.

\begin{image}
\begin{tikzpicture}
\draw[thin](2.828,0) -- (4,0) -- (4, 4) -- (2.828, 2.828);
\draw[thin]  (3.9, 0) -- (3.9, 0.1)-- (4, 0.1);
\draw[thin, blue] (0,0) -- (2.828, 0) -- (2.828, 2.828) -- cycle;
\draw[thin, blue]  (2.728, 0) -- (2.728, 0.1)-- (2.828, 0.1);
\draw[thin, red] (4,0) arc (0: 45:4);
\node[right] at (0.2,0.2) {$\theta$};
\node[below, blue] at (1.5,0) {$\cos(\theta)$};
\node[left, blue] at (2.8,1) {$\sin(\theta)$};
\node[right] at (4.1,2) {$\tan(\theta)$};
\node[above, blue] at (1.3,1.4) {$1$};
%\node[below] at (-.1,0) {$(0,0)$};
%\node[below] at (4,0) {$1$};
\end{tikzpicture}
\end{image}







The area of the small triangle is 
\[
A = \frac12 bh = \frac12 \cos(\theta) \sin(\theta).
\]
The area of the sector is 
\[
A = \frac12 \theta r^2 = \frac12 \theta.
\]
The area of the large triangle is 
\[
A = \frac12 bh = \frac12 \cdot 1 \cdot \tan(\theta) = \frac12 \tan(\theta).
\]

We can use the areas of these figures to create a compound inequality like the one found in the Squeeze theorem.
Since the area of the small triangle is less than the area of the sector which is less than the area of the large triangle, we have:
\[
\frac12 \cos(\theta) \sin(\theta) < \frac12 \theta < \frac12 \tan(\theta).
\]

Multiply through by 2:

\[
\cos(\theta) \sin(\theta) < \theta < \tan(\theta).
\]

Divide through by $\sin(\theta)$.  Note that if $\theta$ is a small positive angle, then $\sin(\theta) >0$ so the direction of the inequality symbols remains unchanged:

\[
\cos(\theta)  < \frac{ \theta}{\sin(\theta)} < \frac{1}{\cos(\theta)}.
\]

Next we take reciprocals (this will change the direction of the inequalitiy symbols):

\[
\frac{1}{\cos(\theta)}  > \frac{\sin(\theta)}{\theta} > \cos(\theta),
\]
which is equivalent to 

\[
\cos(\theta)  < \frac{\sin(\theta)}{\theta} < \frac{1}{\cos(\theta)}.
\]

We now compute the limits of the upper and lower bounds:
\[
\lim_{\theta \to 0^+} \cos(\theta) = \cos(0) = 1 \quad \text{and} \quad \lim_{\theta \to 0^+} \frac{1}{\cos(\theta)} = \frac{1}{\cos(0)} = 1. 
\]

Since the above limits are equal, by the Squeeze Theorem 
\[
\lim_{\theta \to 0^+} \frac{\sin(\theta)}{\theta} = 1 \;\; \text{  as well.}
\]

To compute the left-hand limit, we recall that $\sin(-\theta) = -\sin(\theta)$ for any angle $\theta$.
Therefore,
\[
\frac{\sin(-\theta)}{-\theta} = \frac{-\sin(\theta)}{-\theta} = \frac{\sin(\theta)}{\theta},
\]
which implies that the left-hand limit and the right-hand limit are equal:

\[
\lim_{\theta \to 0^-} \frac{\sin(\theta)}{\theta} =\lim_{\theta \to 0^+} \frac{\sin(\theta)}{\theta} = 1. 
\]

Thus the two-sided limit exists and

\[
\lim_{\theta \to 0} \frac{\sin(\theta)}{\theta} =1. 
\]

\end{document}



(1) If the limit
\[
\lim_{x \to \infty} f(x) = L_1,
\]
then the line $y=L_1$ is a horizontal asymptote for the graph of $y = f(x)$ on the right end.\\
(2) If the limit
\[
\lim_{x \to -\infty} f(x) = L_2,
\]
then the line $y=L_2$ is a horizontal asymptote for the graph of $y = f(x)$ on the left end.
\end{theorem}



\begin{example} %example 1
Find $\lim_{x \to \infty} \dfrac{1}{x}.$
  We argue as follows. If $x$ is a very large number, 
	then $1/x$ will be a very small number, near zero.  Furthermore, as $x$ increases, $1/x$ will decrease.
	
	Hence
	\[\lim_{x \to \infty} \frac{1}{x}= 0.\]
	This is evident from the graph of $y=1/x$ shown below.
	
	%\[\graph[panel, xAxisLabel=``x'', yAxisLabel=``y'', xmin=-3, xmax=75]{y = 1/x}\]
	
	The graph of the function $f(x) = \frac{1}{x}$ also provides evidence for this conclusion.  
	
	\[\graph{y=1/x}\]
	
	Lastly, the value of the limit corresponds to the horizontal asymptote of the graph, namely
	$y = 0$ in this example.
\end{example}




\begin{problem} %problem #1
  Compute the limit:
  \[
  \lim_{x \to \infty} \frac{7}{x^2}
  \]
  
    \begin{hint}
      As $x$ goes to infinity, so does $x^2$
    \end{hint}
    \begin{hint}
      As the denominator grows, the fraction shrinks
    \end{hint}
    \begin{hint}
      The fraction never shrinks below 0
    \end{hint}
		The value of the limit is
		 $\answer[given]{0}$
		
\end{problem}

An important generalization of the last example and problem, which follows from a similar analysis is 
\[\lim_{x \to \pm \infty} \frac{a}{x^n} = 0 \]
for any constant, $a$, and any positive exponent, $n$.

We will exploit this important fact in the next two examples.


\begin{example} %example 2
Compute 
\[
\lim_{x \to \infty} \frac{3x^2 + 5x + 2}{2x^2 -x- 4}.
\]

First note that as $x\to \pm \infty$, a polynomial, $p(x) \to \pm \infty$ according to it leading term.
In this example, since the lead terms $3x^2$ and $2x^2$ both go to $\infty$ as $x \to \infty$, 
our limit has the indeterminate form
$\frac{\infty}{\infty}$.
To resolve this issue, we will factor out of both the numerator and the denominator 
the highest power of $x$ seen in the denominator.  So, in this example, we will factor $x^2$ from both.
In the numerator, 
\[3x^2 + 5x + 2 = x^2(3 + \frac{5}{x} + \frac{2}{x^2})\]
and in the denominator,
\[2x^2 -x -4 = x^2(2- \frac{1}{x} - \frac{4}{x^2}).\]

With these factorizations, our limit becomes

\begin{align*}
\lim_{x \to \infty} \frac{3x^2 + 5x + 2}{2x^2 -x- 4} &= 
\lim_{x \to \infty} \frac{x^2(3 + \frac{5}{x} + \frac{2}{x^2})}{x^2(2- \frac{1}{x} - \frac{4}{x^2})} \\ \\
&=\lim_{x \to \infty} \frac{3 + \frac{5}{x} + \frac{2}{x^2}}{2- \frac{1}{x} - \frac{4}{x^2}} \\ \\
&=\frac{3 + 0 + 0}{2- 0 - 0} \\ \\
&= \frac32.
\end{align*}

The result of this limit means that the line $y = 3/2$ is a horizontal asymptote
for the graph of $y = \dfrac{3x^2 + 5x + 2}{2x^2 -x- 4}$ on the right end.
\end{example}

\begin{problem} %problem #2
  Compute the limit:
  \[
  \lim_{x \to \infty} \frac{3 + 2x^3 - x^3}{5x^3 - 2x -4}
  \]
  
    \begin{hint}
      When you `plug in' $x = \infty$, you get $\frac{-\infty}{\infty}$
    \end{hint}
    \begin{hint}
      Determine the highest power of $x$ in the denominator
    \end{hint}
    \begin{hint}
      Factor this power from both the numerator and denominator and cancel it
    \end{hint}
    \begin{hint}
      $\frac{a}{x^n} \to 0$ as $x \to \infty$.
    \end{hint}
		The value of the limit is
		 $\answer[given]{-\frac15}$
		
\end{problem}




\begin{example} %example 3
Compute $\displaystyle{\lim_{x \to -\infty} \frac{4x^2 - 3x - 6}{x^3 +8x^2 -3x + 7}}.$

First note that as $x\to \pm \infty$, a polynomial, $p(x) \to \pm \infty$ according to it leading term.
In this example, the lead terms $4x^2$ and $x^3$  go to $\infty$ and $-\infty$ respectively, as $x \to -\infty$. 
Hence, our limit has the indeterminate form
$\frac{\infty}{-\infty}$.
To resolve this indeterminate form, we will factor out 
the highest power of $x$ in the denominator, namely $x^3$, from both the numerator and the denominator.  
In the numerator we get, 
\[4x^2 - 3x - 6 = x^3(\frac{4}{x} - \frac{3}{x^2} - \frac{6}{x^3})\]
and in the denominator we get,
\[x^3 +8x^2 -3x + 7 = x^3(1 +  \frac{8}{x} - \frac{3}{x^2} + \frac{7}{x^3}).\]

Factoring the $x^3$ and canceling, the limit can be resolved as follows:

\begin{align*}
\lim_{x \to -\infty}\frac{4x^2 - 3x - 6}{x^3 +8x^2 -3x + 7} &= 
\lim_{x \to -\infty} \frac{x^3(\frac{4}{x}-\frac{3}{x^2} -\frac{6}{x^3})}
{x^3(1 + \frac{8}{x}-\frac{3}{x^2}+\frac{7}{x^3})} \\ \\
&=\lim_{x \to -\infty} \frac{\frac{4}{x}-\frac{3}{x^2} -\frac{6}{x^3}}{1 + \frac{8}{x}-\frac{3}{x^2}+\frac{7}{x^3}} \\ \\
&=\frac{0 - 0 - 0}{1 +0- 0 + 0} \\ \\
&= \frac01 \\
&= 0.
\end{align*}

The result of this limit means that the line $y = 0$ (the $x$-axis) is a horizontal asymptote
for the graph of $y = \dfrac{4x^2 - 3x - 6}{x^3 +8x^2 -3x + 7}$ on the right end.
\end{example}

\begin{problem} %problem #3
  Compute the limit:
  \[
  \lim_{x \to \infty} \frac{3x^3 - 5x + 6}{3-x+ 2x^3- x^4}
  \]
  
    \begin{hint}
      When you `plug in' $x = \infty$, you get $\frac{\infty}{-\infty}$
    \end{hint}
    \begin{hint}
      Determine the highest power of $x$ in the denominator
    \end{hint}
    \begin{hint}
      Factor this power from both the numerator and denominator
    \end{hint}
    \begin{hint}
      Cancel the power of $x$ that was factored out
    \end{hint}
    \begin{hint}
      $\frac{a}{x^n} \to 0$ as $x \to \infty$.
    \end{hint}
		The value of the limit is
		 $\answer[given]{0}$
		
\end{problem}

\begin{example}

Find 
\[
\lim_{x\to -\infty} e^x.
\]

Suppose $x$ is a large negative number.  Then $x = -|x|$ and so
\[
e^x = e^{-|x|} = \frac{1}{e^|x|},
\]
using the definition of negative exponents: $x^{-n} = \frac{1}{x^n}$.

Now note that as $x\to -\infty, |x| \to \infty$ and so 
\[
e^{|x|} \to \infty.
\]

Thus, 
\[
\frac{1}{e^|x|} \to 0,
\]
hence

\[
\lim_{x\to -\infty} e^x = 0.
\]

This means that the line $y=0$ is a horizontal asymptote of the graph of $y = e^x$,
as shown in the graph below.

\[
\graph{y=e^x}
\]

\end{example}

\begin{example}
Discuss
\[
\lim_{x\to \infty} \sin(x).
\]

As $x$ increases, the sine function oscillates between -1 and 1 without approaching any particular value.
Hence
\[
\lim_{x\to \infty} \sin(x) \ \text{DNE}
\]
due to oscillation.
This is illustrated in the graph of $y=\sin(x)$ shown below.

\[
\graph{y=\sin(x)}
\]

\end{example}



\begin{center}
\begin{foldable}
\unfoldable{Here is a detailed, lecture style video on finding limits at infinity:}
\youtube{FyhXNtLGr2o}
\end{foldable}
\end{center}



\end{document}


\begin{example} %example 9
Analyze the limit:
\[
\lim_{x\to -1} 
\frac{1}{x^2}
\]
Plugging $x = 0$ into the rational function
\[f(x) = \frac{1}{x^2}\]
gives the undefined expression $\frac{1}{0}$. From this information, we can conclude that the graph 
of the function has a vertical asymptote at $x = 0$. This means that the one-sided limits as $x$ approaches 0
will give either $\infty$ or $-\infty$, i.e., 
\[\lim_{x \to -1^-} \frac{1}{x^2}= \infty \ \text{or} \ -\infty.\]
and
\[\lim_{x \to -1^+} \frac{1}{x^2}= \infty \ \text{or} \ -\infty.\]
To determine which, we will do a sign analysis as follows. Because of the perfect square, 
we can consider both case simultaneously.
The numerator is 1, which is positive and the denominator is $x^2$ which is positive whether $x \to 0^-$
or $x \to 0^+$.  Hence 
\[\lim_{x \to 0^-} \frac{1}{x^2} =\frac{\text{pos}}{\text{pos}} = \infty,\]
and
\[\lim_{x \to 0^+} \frac{1}{x^2} =\frac{\text{pos}}{\text{pos}} = \infty,\]
since the choices were only $\infty$ and $-\infty$.
The graph of $f(x) = \frac{1}{x^2}$ near $x = 0$ looks like this:

\begin{center}
\begin{tikzpicture}
\begin{axis}[axis lines = center, xlabel = $x$,
    ylabel = {$f(x)$}]
\addplot[domain=-2:-0.1, 
    samples=100, color=blue]{1/x^2};
\addplot[domain=0.1:2, 
    samples=100, color=blue]{1/x^2};
\end{axis}
\end{tikzpicture}
\end{center}
\end{example}


Here is a cool truth table

\begin{problem}
Fill in the truth table below using your amazing logic skillz!

\begin{prompt}
\begin{center}
\[
\begin{array}{c|c|c|c}
		p & q & p \implies q & p \vee (p \implies q) \\
		\hline
		T & T & T & \answer{T} \\
		T & F & F & \answer{T} \\
		F & T & T & \answer{T} \\
		F & F & T & \answer{T}
	\end{array}
    \]
\end{center}
\end{prompt}
\end{problem}

\begin{center}
\begin{tikzpicture}
\begin{axis}[axis lines = center, xlabel = $x$,
    ylabel = {$f(x)$}]
\addplot[domain= 0.5:100, 
    samples=1000, color=blue]{1/x};

\end{axis}
\end{tikzpicture}
\end{center}
