\documentclass[handout]{ximera}

%% You can put user macros here
%% However, you cannot make new environments



\newcommand{\ffrac}[2]{\frac{\text{\footnotesize $#1$}}{\text{\footnotesize $#2$}}}
\newcommand{\vasymptote}[2][]{
    \draw [densely dashed,#1] ({rel axis cs:0,0} -| {axis cs:#2,0}) -- ({rel axis cs:0,1} -| {axis cs:#2,0});
}


\graphicspath{{./}{firstExample/}}

\usepackage{amsmath}
\usepackage{amssymb}
\usepackage{array}
\usepackage[makeroom]{cancel} %% for strike outs
\usepackage{pgffor} %% required for integral for loops
\usepackage{tikz}
\usepackage{tikz-cd}
\usepackage{tkz-euclide}
\usetikzlibrary{shapes.multipart}


\usetkzobj{all}
\tikzstyle geometryDiagrams=[ultra thick,color=blue!50!black]


\usetikzlibrary{arrows}
\tikzset{>=stealth,commutative diagrams/.cd,
  arrow style=tikz,diagrams={>=stealth}} %% cool arrow head
\tikzset{shorten <>/.style={ shorten >=#1, shorten <=#1 } } %% allows shorter vectors

\usetikzlibrary{backgrounds} %% for boxes around graphs
\usetikzlibrary{shapes,positioning}  %% Clouds and stars
\usetikzlibrary{matrix} %% for matrix
\usepgfplotslibrary{polar} %% for polar plots
\usepgfplotslibrary{fillbetween} %% to shade area between curves in TikZ



%\usepackage[width=4.375in, height=7.0in, top=1.0in, papersize={5.5in,8.5in}]{geometry}
%\usepackage[pdftex]{graphicx}
%\usepackage{tipa}
%\usepackage{txfonts}
%\usepackage{textcomp}
%\usepackage{amsthm}
%\usepackage{xy}
%\usepackage{fancyhdr}
%\usepackage{xcolor}
%\usepackage{mathtools} %% for pretty underbrace % Breaks Ximera
%\usepackage{multicol}



\newcommand{\RR}{\mathbb R}
\newcommand{\R}{\mathbb R}
\newcommand{\C}{\mathbb C}
\newcommand{\N}{\mathbb N}
\newcommand{\Z}{\mathbb Z}
\newcommand{\dis}{\displaystyle}
%\renewcommand{\d}{\,d\!}
\renewcommand{\d}{\mathop{}\!d}
\newcommand{\dd}[2][]{\frac{\d #1}{\d #2}}
\newcommand{\pp}[2][]{\frac{\partial #1}{\partial #2}}
\renewcommand{\l}{\ell}
\newcommand{\ddx}{\frac{d}{\d x}}

\newcommand{\zeroOverZero}{\ensuremath{\boldsymbol{\tfrac{0}{0}}}}
\newcommand{\inftyOverInfty}{\ensuremath{\boldsymbol{\tfrac{\infty}{\infty}}}}
\newcommand{\zeroOverInfty}{\ensuremath{\boldsymbol{\tfrac{0}{\infty}}}}
\newcommand{\zeroTimesInfty}{\ensuremath{\small\boldsymbol{0\cdot \infty}}}
\newcommand{\inftyMinusInfty}{\ensuremath{\small\boldsymbol{\infty - \infty}}}
\newcommand{\oneToInfty}{\ensuremath{\boldsymbol{1^\infty}}}
\newcommand{\zeroToZero}{\ensuremath{\boldsymbol{0^0}}}
\newcommand{\inftyToZero}{\ensuremath{\boldsymbol{\infty^0}}}


\newcommand{\numOverZero}{\ensuremath{\boldsymbol{\tfrac{\#}{0}}}}
\newcommand{\dfn}{\textbf}
%\newcommand{\unit}{\,\mathrm}
\newcommand{\unit}{\mathop{}\!\mathrm}
%\newcommand{\eval}[1]{\bigg[ #1 \bigg]}
\newcommand{\eval}[1]{ #1 \bigg|}
\newcommand{\seq}[1]{\left( #1 \right)}
\renewcommand{\epsilon}{\varepsilon}
\renewcommand{\iff}{\Leftrightarrow}

\DeclareMathOperator{\arccot}{arccot}
\DeclareMathOperator{\arcsec}{arcsec}
\DeclareMathOperator{\arccsc}{arccsc}
\DeclareMathOperator{\si}{Si}
\DeclareMathOperator{\proj}{proj}
\DeclareMathOperator{\scal}{scal}
\DeclareMathOperator{\cis}{cis}
\DeclareMathOperator{\Arg}{Arg}
%\DeclareMathOperator{\arg}{arg}
\DeclareMathOperator{\Rep}{Re}
\DeclareMathOperator{\Imp}{Im}
\DeclareMathOperator{\sech}{sech}
\DeclareMathOperator{\csch}{csch}
\DeclareMathOperator{\Log}{Log}

\newcommand{\tightoverset}[2]{% for arrow vec
  \mathop{#2}\limits^{\vbox to -.5ex{\kern-0.75ex\hbox{$#1$}\vss}}}
\newcommand{\arrowvec}{\overrightarrow}
\renewcommand{\vec}{\mathbf}
\newcommand{\veci}{{\boldsymbol{\hat{\imath}}}}
\newcommand{\vecj}{{\boldsymbol{\hat{\jmath}}}}
\newcommand{\veck}{{\boldsymbol{\hat{k}}}}
\newcommand{\vecl}{\boldsymbol{\l}}
\newcommand{\utan}{\vec{\hat{t}}}
\newcommand{\unormal}{\vec{\hat{n}}}
\newcommand{\ubinormal}{\vec{\hat{b}}}

\newcommand{\dotp}{\bullet}
\newcommand{\cross}{\boldsymbol\times}
\newcommand{\grad}{\boldsymbol\nabla}
\newcommand{\divergence}{\grad\dotp}
\newcommand{\curl}{\grad\cross}
%% Simple horiz vectors
\renewcommand{\vector}[1]{\left\langle #1\right\rangle}


\outcome{Apply the IVT.}

\title{1.9 Intermediate Value Theorem}



\begin{document}

\begin{abstract}
In this section we learn a theoretically important existence theorem called the Intermediate Value Theorem
and we investigate some applications.
\end{abstract}

\maketitle

\section{Intermediate Value Theorem}



In this section we discuss an important theorem related to continuous functions. Before we present the theorem, 
lets consider two real life situations and observe an important difference in their behavior. 
First, consider the ambient temperature and second, consider the amount of money in a bank account.

First, suppose that the temperature is $65^{\circ}$ at 8am and then suppose it is $75^{\circ}$ at noon. 
Because of the continuous nature of temperature variation, we can be sure that at some time 
between 8am and noon  the temperature was exactly $70^{\circ}$.
Can we make a similar claim about money in a bank account?  
Suppose the account has \$65  in it at 8am and then it has \$75 in it at noon.  
Did it have exactly \$70 in it at some time between 8 am and noon? 
We cannot answer that question with any certainty from the given information.  
On one hand, it is possible that a \$10 deposit was made at 11am and so the total in the 
bank would have jumped from \$65 dollars to \$75 without ever being exactly \$70. 
On the other hand, it is possible that the \$10 was added in \$5 increments. In this case, the account did have 
exactly \$70 in it at some time.
The fundamental reason why we can make certain conclusions in the first case but cannot in the second, 
is that temperature varies continuously, whereas money in a bank account does not 
(it will have jump discontinuities).  
When a quantity is known to vary continuously, then if the quantity is observed to have different 
values at different times then we can conclude that the quantity took on any given value between these 
two at some time between our two observations. Mathematically, this property is stated in the 
Intermediate Value Theorem.


\begin{theorem}[Intermediate Value Theorem]


If the function $f(x)$ is continuous on the closed interval $[a, b]$ 
and $I$ is a number between $f(a)$ and $f(b)$,
then the equation $f(x) = I$ has a solution in the open interval $(a,b)$. 
\end{theorem}


The value $I$ in the theorem is called an \textbf{intermediate value} for the function $f(x)$ 
on the interval $[a,b]$. Note that if a function is \textit{not continuous} on an interval, 
then the equation $f(x) = I$ may or may not have a solution on the interval. \\

\textbf{Remark:} saying that $f(x)$ has a solution in $(a,b)$ is equivalent to saying that 
there exists a number $x_0$ between $a$ and $b$ such that $f(x_0) = I$.


%Also note that saying that $I$ is a number between $f(a)$ and $f(b)$
%means that either
%\[
%f(a) < I < f(b) \text{  or}  f(b) < I < f(a).
%\]

The following figure illustrates the IVT.

\begin{image}
\begin{tikzpicture}[scale = 1.2]
\begin{axis}[axis x line=middle, axis y line= middle, axis equal, xtick={1, 2.1, 3},
xticklabels={$a$, $x_0$, $b$}, ytick={.914, 1.6, 2.328}, yticklabels={$f(a)$, $I$, $f(b)$}, title={Intermediate Value Theorem}]

\addplot[domain=-.5:3.5, thick, blue!65]{2^(x/2)-.5};
\addplot[dashed, thick] coordinates{(1, 0)  (1, .914)};
\addplot[dashed, thick] coordinates{ (1, .914) (0, .914)};
\addplot[dashed, thick] coordinates{(3, 0) (3, 2.328)};
\addplot[dashed, thick] coordinates{(3, 2.328) (0, 2.328)};
\addplot[dashed, red, thick] coordinates{(2.1, 0) (2.1, 1.6)};
\addplot[dashed, red, thick] coordinates{(0, 1.6) (2.1, 1.6)};
\end{axis}
%\addplot[mark=*,fill=white] coordinates {(0,0)}
\node at (3.5, -.5) {$f(x)$ is continuous on $[a,b]$ and $I$ is between $f(a)$ and $f(b)$, };	
\node at (3.5,-1) {so there is an $x_0$ between $a$ and $b$ such that $f(x_0) = I$  };	
\node at (7.5, 4.8) {$y = f(x)$};
\end{tikzpicture}
\end{image}






\begin{example}[example 1]
Show that the equation $x^3 = 10$ has a solution between $x = 2$ and $x = 3$.\\
First, the function $f(x) = x^3$  is continuous on the interval $[2,3]$ 
since is a polynomial. 
Second, observe that 
\[
f(2) = 2^3 = 8,
\]
and 
\[
f(3) = 3^3 = 27,
\] 
so that 10 is an intermediate value, i.e.,
\[f(2) < 10 < f(3).\]
Now we can apply the Intermediate Value Theorem to
conclude that the equation $x^3 = 10$ has a least one solution between $x=2$ and $x=3$. 
In this example, the number 10 is playing the role of $I$ in the statement of the theorem.
\end{example}


\begin{problem}(problem 1)
Determine whether the IVT can be used to show that the equation $x^4 = 50$ has a solution in the open interval $(2,3)$.

Is $f(x) = x^4$ continuous on the closed interval $[2,3]$? 
\begin{center}
\begin{multipleChoice}
  \choice[correct]{Yes}
  \choice{No}
\end{multipleChoice}
\end{center}


$f(2) = \answer{16}, f(3) = \answer{81}$ and $I = \answer{50}.$

Is $I$ an intermediate value? 
\begin{multipleChoice}
  \choice[correct]{Yes}
  \choice{No}
\end{multipleChoice}

Can we apply the IVT to conclude that the equation $x^4 = 50$ has a solution in the open interval $(2, 3)$?
\begin{multipleChoice}
  \choice[correct]{Yes}
  \choice{No}
\end{multipleChoice}
\end{problem}




\begin{example}[example 2]
Show that the equation $e^x =2$ has a solution between $x = 0$ and $x = 1$.\\

First, note that the function $f(x) = e^x$ is continuous on the 
interval $(-\infty, \infty)$ and hence it is continuous on the sub-interval, $[0, 1]$. 
Next, observe that 
\[
f(0) = e^0 = 1,
\]
and 
\[
f(1) = e^1 = e \approx 2.7,
\]
 so that 2 is an intermediate value, i.e.,
\[f(0) < 2 < f(1).\]
Finally, by the Intermediate Value Theorem we can 
conclude that the equation $e^x = 2$ has a solution on the open interval $(0,1)$.
In this example, the number 2 is playing the role of $I$ in the statement of the theorem.
\end{example}



\begin{problem}(problem 2)
Determine whether the IVT can be used to show that the equation $\frac{1}{x} = 0$ has a solution in the open interval $(-1, 1).$
Is $f(x) = \frac{1}{x}$ continuous on the closed interval $[-1,1]$? 
\begin{center}
\begin{multipleChoice}
  \choice{Yes}
  \choice[correct]{No}
\end{multipleChoice}
\end{center}

$f(-1) = \answer{-1}, f(1) = \answer{1}$ and $I = \answer{0}.$

Is $I$ an intermendiate value? 
\begin{multipleChoice}
  \choice[correct]{Yes}
  \choice{No}
\end{multipleChoice}

Does the IVT imply that the equation $\frac{1}{x} = 0$ has a solution in the open interval $(-1, 1)$?
\begin{multipleChoice}
  \choice{Yes}
  \choice[correct]{No}
\end{multipleChoice}
\end{problem}




\begin{example}[example 3]
Show that the function $f(x) = x^4 - 3x - 7$ has a root in the open interval $(-2, 1)$.\\

Recall that a root occurs when $f(x) = 0$. Since $f(x)$ is a polynomial, it is continuous on the interval 
$[-2, 1]$. Plugging in the endpoints shows that 0 is an intermediate value:
\[f(-2) = (-2)^4 - 3(-2) - 7 = 16 + 6 - 7 = 15 \]
and
\[f(1) = (1)^4 - 3(1) - 7 = 1 -3 - 7 = -9 \]
so 
\[f(1) < 0 < f(-2).\]
By the IVT, we can conclude that the equation $x^4 - 3x - 7 = 0$ has a solution (and hence $f(x)$ has a root) 
on the open interval $(-2, 1)$.
\end{example}


\begin{problem}(problem 3)
Determine whether the IVT can be used to show that the function $f(x) = e^x$ has a root in the open interval $(0,1)$.

Is $f(x) = e^x$ continuous on the closed interval $[0,1]$? 
\begin{center}
\begin{multipleChoice}
  \choice[correct]{Yes}
  \choice{No}
\end{multipleChoice}
\end{center}

$f(0) = \answer{1}, f(1) = \answer{e}$ and $I = \answer{0}.$

Is $I$ an intermendiate value? 
\begin{multipleChoice}
  \choice{Yes}
  \choice[correct]{No}
\end{multipleChoice}

Does the IVT imply that the function $f(x) = e^x$ has a root in the open interval $(0, 1)$?
\begin{multipleChoice}
  \choice{Yes}
  \choice[correct]{No}
\end{multipleChoice}
\end{problem}




\begin{example}[example 4]
 Show that the equation $\cos(x) = x$ has a solution 
between $x = 0$ and $x = \frac{\pi}{2}$. \\
Note that the equation $\cos(x) = x$ is equivalent to the equation $x - \cos(x) =0$.
The latter is the prefered form for using the IVT.
So let 
\[f(x) = x - \cos(x).\] 
 Since $f(x)$ is the difference between two continuous functions,
it is continuous on the closed interval $[0, \frac{\pi}{2}]$. 
Next, we compute $f(0)$ and $f(\frac{\pi}{2})$ and show that 0 is an intermediate value:
\[f(0) = 0 - \cos(0) = 0-1 = -1 \]
and
\[f\big(\frac{\pi}{2}\big) = \frac{\pi}{2} - \cos\big(\frac{\pi}{2}\big) = \frac{\pi}{2}-0 = \frac{\pi}{2} \]
and so,
\[f(0) < 0 < f(\frac{\pi}{2}).\]
By the IVT, the equation $ x - \cos(x) =0$ has a solution in the open 
interval $(0, \frac{\pi}{2})$. Hence the equivalent equation $\cos(x) = x$ 
has a solution on the same interval.
\end{example}




\begin{example}[example 5]
Use the IVT four times to approximate a root of the polynomial 
\[p(x) = x^3 - 3x^2 + 4x - 6.\]
First, $p(x)$ is a polynomial, so it is continuous on any closed interval.
Next, note that 
\[p(0) = -6 < 0 \ \text{and} \ p(8) = 346 > 0.\]
By the IVT $p(x)$ has a root in the interval $(0, 8)$. 
To use the IVT a second time, we now determine the midpoint of this interval: $(0+8)/2 = 4$ and we 
plug this in to $p(x)$:
\[p(4) =  14 > 0.\]
Combining this with
\[p(0) < 0,\]
we can use the IVT again to conclude that $p(x)$ has a root on the interval $(0, 4)$. 
This interval is half of the original interval- the original interval has been bisected. 
We now use the IVT a third time. The midpoint of the interval $(0, 4)$ is $x=2$. We plug this into $p(x)$:
\[p(2) =  -2 < 0.\]
Combining this with
\[p(4) > 0,\]
we can use the IVT to conclude that $p(x)$ has a root on the interval $(2,4)$. 
This interval is half of the previous interval- the previous interval has been bisected. 
We will do this a fourth and final time. Note that $x = 3$ is the midpoint of the interval $(2,4)$ and
\[p(3) = 6 > 0.\]
Combining this with
\[p(2) < 0,\]
we can use the IVT a fourth time to conclude that $p(x)$ has a root on the interval $(2, 3)$. 
At this stage, our approximation of the root is $x = 2.5$ which is the midpoint of this interval. 
Our error is then no more than $0.5$, which is half the width of the interval.
We will stop here, but the method could theoretically be continued indefinitely giving a better 
approximation to the root each time. This method of approximating roots is called the 
\textbf{Method of Continued Bisection}.
\end{example}




\begin{center}
\begin{foldable}
\unfoldable{Here is a detailed, lecture style video on the Intermediate Value Theorem:}
\youtube{961hYy7RjpE}
\end{foldable}
\end{center}



\end{document}


