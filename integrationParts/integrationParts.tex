\documentclass{ximera}

%% You can put user macros here
%% However, you cannot make new environments



\newcommand{\ffrac}[2]{\frac{\text{\footnotesize $#1$}}{\text{\footnotesize $#2$}}}
\newcommand{\vasymptote}[2][]{
    \draw [densely dashed,#1] ({rel axis cs:0,0} -| {axis cs:#2,0}) -- ({rel axis cs:0,1} -| {axis cs:#2,0});
}


\graphicspath{{./}{firstExample/}}

\usepackage{amsmath}
\usepackage{amssymb}
\usepackage{array}
\usepackage[makeroom]{cancel} %% for strike outs
\usepackage{pgffor} %% required for integral for loops
\usepackage{tikz}
\usepackage{tikz-cd}
\usepackage{tkz-euclide}
\usetikzlibrary{shapes.multipart}


\usetkzobj{all}
\tikzstyle geometryDiagrams=[ultra thick,color=blue!50!black]


\usetikzlibrary{arrows}
\tikzset{>=stealth,commutative diagrams/.cd,
  arrow style=tikz,diagrams={>=stealth}} %% cool arrow head
\tikzset{shorten <>/.style={ shorten >=#1, shorten <=#1 } } %% allows shorter vectors

\usetikzlibrary{backgrounds} %% for boxes around graphs
\usetikzlibrary{shapes,positioning}  %% Clouds and stars
\usetikzlibrary{matrix} %% for matrix
\usepgfplotslibrary{polar} %% for polar plots
\usepgfplotslibrary{fillbetween} %% to shade area between curves in TikZ



%\usepackage[width=4.375in, height=7.0in, top=1.0in, papersize={5.5in,8.5in}]{geometry}
%\usepackage[pdftex]{graphicx}
%\usepackage{tipa}
%\usepackage{txfonts}
%\usepackage{textcomp}
%\usepackage{amsthm}
%\usepackage{xy}
%\usepackage{fancyhdr}
%\usepackage{xcolor}
%\usepackage{mathtools} %% for pretty underbrace % Breaks Ximera
%\usepackage{multicol}



\newcommand{\RR}{\mathbb R}
\newcommand{\R}{\mathbb R}
\newcommand{\C}{\mathbb C}
\newcommand{\N}{\mathbb N}
\newcommand{\Z}{\mathbb Z}
\newcommand{\dis}{\displaystyle}
%\renewcommand{\d}{\,d\!}
\renewcommand{\d}{\mathop{}\!d}
\newcommand{\dd}[2][]{\frac{\d #1}{\d #2}}
\newcommand{\pp}[2][]{\frac{\partial #1}{\partial #2}}
\renewcommand{\l}{\ell}
\newcommand{\ddx}{\frac{d}{\d x}}

\newcommand{\zeroOverZero}{\ensuremath{\boldsymbol{\tfrac{0}{0}}}}
\newcommand{\inftyOverInfty}{\ensuremath{\boldsymbol{\tfrac{\infty}{\infty}}}}
\newcommand{\zeroOverInfty}{\ensuremath{\boldsymbol{\tfrac{0}{\infty}}}}
\newcommand{\zeroTimesInfty}{\ensuremath{\small\boldsymbol{0\cdot \infty}}}
\newcommand{\inftyMinusInfty}{\ensuremath{\small\boldsymbol{\infty - \infty}}}
\newcommand{\oneToInfty}{\ensuremath{\boldsymbol{1^\infty}}}
\newcommand{\zeroToZero}{\ensuremath{\boldsymbol{0^0}}}
\newcommand{\inftyToZero}{\ensuremath{\boldsymbol{\infty^0}}}


\newcommand{\numOverZero}{\ensuremath{\boldsymbol{\tfrac{\#}{0}}}}
\newcommand{\dfn}{\textbf}
%\newcommand{\unit}{\,\mathrm}
\newcommand{\unit}{\mathop{}\!\mathrm}
%\newcommand{\eval}[1]{\bigg[ #1 \bigg]}
\newcommand{\eval}[1]{ #1 \bigg|}
\newcommand{\seq}[1]{\left( #1 \right)}
\renewcommand{\epsilon}{\varepsilon}
\renewcommand{\iff}{\Leftrightarrow}

\DeclareMathOperator{\arccot}{arccot}
\DeclareMathOperator{\arcsec}{arcsec}
\DeclareMathOperator{\arccsc}{arccsc}
\DeclareMathOperator{\si}{Si}
\DeclareMathOperator{\proj}{proj}
\DeclareMathOperator{\scal}{scal}
\DeclareMathOperator{\cis}{cis}
\DeclareMathOperator{\Arg}{Arg}
%\DeclareMathOperator{\arg}{arg}
\DeclareMathOperator{\Rep}{Re}
\DeclareMathOperator{\Imp}{Im}
\DeclareMathOperator{\sech}{sech}
\DeclareMathOperator{\csch}{csch}
\DeclareMathOperator{\Log}{Log}

\newcommand{\tightoverset}[2]{% for arrow vec
  \mathop{#2}\limits^{\vbox to -.5ex{\kern-0.75ex\hbox{$#1$}\vss}}}
\newcommand{\arrowvec}{\overrightarrow}
\renewcommand{\vec}{\mathbf}
\newcommand{\veci}{{\boldsymbol{\hat{\imath}}}}
\newcommand{\vecj}{{\boldsymbol{\hat{\jmath}}}}
\newcommand{\veck}{{\boldsymbol{\hat{k}}}}
\newcommand{\vecl}{\boldsymbol{\l}}
\newcommand{\utan}{\vec{\hat{t}}}
\newcommand{\unormal}{\vec{\hat{n}}}
\newcommand{\ubinormal}{\vec{\hat{b}}}

\newcommand{\dotp}{\bullet}
\newcommand{\cross}{\boldsymbol\times}
\newcommand{\grad}{\boldsymbol\nabla}
\newcommand{\divergence}{\grad\dotp}
\newcommand{\curl}{\grad\cross}
%% Simple horiz vectors
\renewcommand{\vector}[1]{\left\langle #1\right\rangle}

\usetikzlibrary{matrix,positioning}

\outcome{Compute an integral using integration by parts}

\title{2.1 Integration by Parts}

\begin{document}

\begin{abstract}
We will compute integrals using the integration by parts technique.
\end{abstract}

\maketitle

\section{INTRODUCTION}
In this section, we are interested in integrating products.  
We have some experience with this from our exposure to $u$-substitution problems such as
\[
\int 2x \cos(1 + x^2) \; dx.
\]
In this scenario, the factor of $2x$ in the integrand is of great significance 
as it is the derivative of the inside of the composition, $1 + x^2$.
We will now turn to problems where the factors in the product making up the integrand
are unrelated in this or any other significant way, i.e., they are functionally independent of one another.
We will effectively create a product rule for integrals and interestingly, we will 
use the product rule for derivatives to help us. One stark difference between this product rule for
integrals and the product rule for derivatives, known as \textbf{integration by parts} (IBP), is that IBP does not give us a final answer. Instead, integration by parts
simply transforms our problem into another, hopefully easier one, which we then have to solve.
In this last respect, IBP is similar to $u$-substitution. The following integrals can be computed using IBP:

\[
\int 3x\sin(4x) \; dx, \int x^2 e^{-x} \; dx, \int \ln(x) \; dx,
\]
\[
\int e^{2x}\cos(3x) \; dx, \int x^{n-1} e^{-x} \; dx.
\]
The last integral is part of the famous Gamma function
\[
\Gamma(n) = \int_0^\infty e^{n-1} e^{-x} \; dx.
\]
This special function leads us to study the coming topic, Improper Integrals.
But, for now, let's discover the IBP formula! 


\section{IBP FORMULA}

Recall the product rule: $(uv)' = u'v+uv'$ (where $u$ and $v$ are functions of $x$).
Integrating both sides of this equation with respect to $x$ gives
\[
\int (uv)' = \int u'v + \int uv',
\]
where the $dx$ has intentionally been suppressed from the notation for simplicity.
Now the integral on the left hand side is simply $uv + C$ by the anti-differentiation
definition of the integral symbol. So,
\[
\int u'v + \int uv' = uv + C.
\]
Subtracting the second integral from both sides gives
\[
\int uv' = uv - \int u'v,
\]
which can be condensed as
\[
\int u \; dv = uv - \int v \; du,
\]
the famous integration by parts formula.


\begin{theorem}[Integration by Parts]
\[
\int u \; dv = uv - \int v \; du
\]
\end{theorem}

\begin{remark}
\begin{enumerate}
\item[1.] To use the IBP formula, the original problem must be written in the form $\int u \; dv$. This means that $u$ and $dv$
must be declared (similar to declaring $u$ in $u$-substitution), and then we compute $du$ by differentiating $u$, and $v$ by anti-differentiating $dv$.
\item[2.] The constant of integration does not appear on the right hand side of the IBP formula
(it mysteriously disappeared in the above derivation) because it was taken into account by the constant of integration implied in 
the integral on the right hand side, $\int v \,du$.
\item[3.] The IBP formula does not give a final answer to a problem. 
It replaces one integral with another another integral (which still has to be computed).
%The goal is for this new integral to be easier than the original one.
\end{enumerate}
\end{remark}

We will now use the integration by parts (IBP) formula to compute integrals involving products of unrelated factors, like the ones
shown above. 

\section{BASIC EXAMPLES}

\begin{example}[example 1] %example #1
Compute the integral using IBP:
  \[
  \int x\sin(x) \;dx.
  \]
  
  The integrand is the product of two factors, $x$ and $\sin(x)$.  We consider one factor to differentiate and the other factor to anti-differentiate.
  The factor we prefer to differentiate is $x$ since the derivative of $x$ is a constant and constants are easy to deal with in integrals.
  As for the $\sin(x)$ factor, whether we differentiate it or anti-differentiate it makes little difference.  In fact, only a change in sign occurs,
  as 
  \[\frac{d}{dx}{\sin(x)} = \cos(x) \;\;  \text{ and }\;\;   \int \sin(x) \;dx = -\cos(x) + C.\]
  Thus to apply the IBP formula, we let
  \[u=x \text{  and  }  dv = \sin(x) \;dx.\]
  Next, we compute
  \[du = dx  \;\;  \text{  and  }\;\;    v = -\cos(x).\]
  \begin{remark} 
  We do not need to add the constant of integration to v, as any anti-derivative will suffice 
  for our purpose and we choose the constant of integration to be zero.
  \end{remark}
  Now we use the Integration by Parts formula, $\int u\;dv = uv-\int v\; du$:
  \begin{align*}
  \int x\sin(x) &= \overbrace{x}^{u}\overbrace{[-\cos(x)]}^{v} - \int \overbrace{-\cos(x)}^{v} \;\overbrace{dx}^{du}\\
                &= -x\cos(x) + \int \cos(x) \;dx\\
                &= -x\cos(x) + \sin(x) + C.
 \end{align*}
 
  We have found an answer, but at this stage we should prove that it is the correct answer by differentiating it.  The result should be $x\sin(x)$.
  Note that the differentiation requires the product rule on the $uv$ term: $-x\cos(x)$.
  We have
  \[
  \frac{d}{dx}{\left[-x\cos(x) + \sin(x) + C\right]} = [-\cos(x) + x\sin(x)] + \cos(x) = x\sin(x)
  \]
  which proves that our answer was correct.
  
    
\end{example}

%\begin{center}
%\begin{foldable}
%\unfoldable{Here is a video of Example 1}
%\foldable{Here is a video of Example 1}
%\youtube{Yy6QXnFlnXs} %vid of example 1
%\end{foldable}
%\end{center}


\begin{problem}(problem 1a) %problem #1
  Compute the integral using IBP:
  \[
  \int xe^x \;dx.
  \]
  
  Let $u = \answer{x}$   and   $dv = \answer{e^x dx}$.\\
  Then $du = \answer{dx}$   and   $v = \answer{e^x}$.\\
  Thus 
  \[
  \int xe^x dx = xe^x - \int \answer{e^x} dx = \answer{xe^x - e^x} + C.
  \]
  


  %\begin{hint}
   %   $\int udv = uv-\int vdu$
  %\end{hint}
    

\end{problem}


\begin{problem}(problem 1b)
  Compute the integral using IBP:
  \[
  \int x\cos(x) \;dx.
  \]
  
  Let $u = \answer{x}$   and   $dv = \answer{\cos(x) \;dx}$.\\
  Then $du = \answer{dx}$   and   $v = \answer{\sin(x)}$.\\
  Thus 
  \[
  \int x\cos(x) dx = \answer{x\sin(x) + \cos(x)} + C.
  \]
  
  %x\sin(x) - \int \answer{\sin(x)} \;dx 
  


  %\begin{hint}
   %   $\int udv = uv-\int vdu$
  %\end{hint}
    

\end{problem}


\begin{example}[example 2]
Compute the integral using IBP:
  \[
  \int 3x\sin(4x) \;dx.
  \]

Let $u = 3x$ and $dv = \sin(4x) \,dx$.\\
Then $du = 3 \,dx$ and $v = -\frac14 \cos(4x)$.
Next, the IBP formula gives
\begin{align*}
  \int 3x\sin(4x) \;dx &= \overbrace{3x}^{u}\overbrace{\left[-\frac14 \cos(4x)\right]}^{v} - \int \overbrace{-\frac14 \cos(4x)}^{v}  \cdot \overbrace{3 \; dx}^{du} \\
                       &= -\frac34 x\cos(4x) + \frac34 \int \cos(4x) \; dx \\
                       &= -\frac34 x\cos(4x) + \frac34 \cdot \frac 14 \sin(4x) + C\\
                       &= -\frac34 x\cos(4x) + \frac{3}{16}\sin(4x) + C.
\end{align*}

\end{example}


\begin{problem}(problem 2a)
Compute the integral using IBP:
  \[
  \int 2x\sin(5x)\;dx.
  \]

Let $u = \answer{2x}$   and   $dv = \answer{\sin(5x) dx}$.\\
  Then $du = \answer{2 dx}$   and   $v = \answer{-\frac15 \cos(5x)}$.\\
  Thus 
  \[
  \int 2x\sin(5x) dx = -\frac25 x\cos(5x) + \int \answer{\frac25 \cos(5x)} dx = \answer{-\frac25 x\cos(5x) +\frac{2}{25} \sin(5x)} + C.
  \]
  
\end{problem}


\begin{problem}(problem 2b)
Compute the integral using IBP:
  \[
  \int 6xe^{-3x} \;dx.
  \]
Let $u = \answer{6x}$   and   $dv = \answer{e^{-3x} dx}$.\\
  Then $du = \answer{6 dx}$   and   $v = \answer{-\frac13 e^{-3x}}$.\\
  Thus 
  \[
  \int 6xe^{-3x} dx = -\frac12 xe^{-3x} + \int \answer{\frac12 e^{-3x}} dx = \answer{\frac12 xe^{-3x} - \frac16 e^{-3x}} + C.
  \]
  
\end{problem}


\begin{problem}(problem 2c)
Compute the integral using IBP:
  \[
  \int x\sin(x/2) \;dx.
  \]

Let $u = \answer{x}$   and   $dv = \answer{\sin(x/2) dx}$.\\
  Then $du = \answer{ dx}$   and   $v = \answer{-2 \cos(x/2)}$.\\
  Thus 
  \[
  \int x\sin(x/2) dx = -2x\cos(x/2) + \int \answer{2 \cos(x/2)} dx = \answer{-2x\cos(x/2) + 4 \sin(x/2)} + C.
  \]
  
\end{problem}


\begin{example}[example 3, IBP twice]
Compute the integral:
  \[
  \int x^2\sin(3x) \;dx.
  \]

\[
\text{Let} \;\; u = x^2 \;\; \text{and} \;\; dv = \sin(3x) \,dx,
\]
\[
\text{then} \;\; du = 2x \,dx \;\; \text{and} \;\; v = -\frac13 \cos(3x).
\]

Next, the IBP formula gives
\begin{align*}
\int x^2\sin(3x) \;dx &= x^2\left[-\frac13 \cos(3x)\right] - \int -\frac13 \cos(3x) \cdot 2x \; dx \\
                       &= -\frac13 x^2\cos(3x) + \frac13 \int 2x\cos(3x) \; dx\\                               
\text{we now use IBP} & \text{ again with } \;\; u = 2x \;\text{ and } \; dv = \cos(3x) \, dx\\
& \qquad \qquad \quad du = 2 \, dx \; \text{ and } \; v=\frac13 \sin(3x)\\
                       &= -\frac13 x^2\cos(3x) + \frac13\left[\frac23x\sin(3x) -  \int \frac13 \sin(3x)\cdot 2 \; dx\right]\\
                       &= -\frac13 x^2\cos(3x) + \frac29x\sin(3x) - \frac29 \int  \sin(3x) \; dx\\
                       &= -\frac13 x^2\cos(3x) + \frac29x\sin(3x) + \frac{2}{27} \cos(3x) + C.
\end{align*}

\end{example}


\begin{problem}(problem 3a)
Compute the integral:
  \[
  \int x^2\cos(2x) \;dx.
  \]

\end{problem}

\begin{problem}(problem 3b)
Compute the integral:
  \[
  \int x^2e^{-x} \;dx.
  \]

\end{problem}

\section{TABULAR INTEGRATION}

Integrals of the form $\int x^n f(x) \,dx$ require IBP $n$ times when $f(x)$ is a sine, cosine or exponential function.
A notational shortcut called \index{\textbf{tabular integration}} can be used to streamline the process of using IBP multiple times.
As an example, let's revisit the previous example:

\begin{example}[Tabular Integration]
Compute the integral using tabular integration:
  \[
  \int x^2\sin(3x) \;dx.
  \]

We will construct a table that lists the derivatives of $u$ and the anti-derivatives of $dv$.
We can then obtain the final answer by a series of multiplications and additions or subtractions, as indicated
by the arrows in the table.


\begin{image}[5cm]
\begin{tikzpicture}
 \matrix (m) [matrix of nodes, ampersand replacement=\&,
            column sep = 1.5cm]{
    u     \&                  dv \\\hline
    $x^2$ \& $\sin(3x)$ \\ [8pt]
     $2x$  \& $-\frac13\cos(3x)$ \\ [8pt]
      $2$ \& $-\frac19\sin(3x)$ \\ [8pt]
       $0$ \& $\frac{1}{27}\cos(3x)$ \\
 };
 \draw[-latex] (m-2-1) -- (m-3-2) node[midway,above]{$+$};  % <-- It works!
 \draw[-latex] (m-3-1) -- (m-4-2) node[midway,above]{$-$};  % <-- It works!
 \draw[-latex] (m-4-1) -- (m-5-2) node[midway,above]{$+$};  % <-- It works!
 %\node at (-.6, 1.5){$+$};
 %\node at (-.6, 0.3){$-$};
 %\node at (-.6, -0.9){$+$};
 %\draw (,) -- (,);
\end{tikzpicture}
\end{image}

Thus,
\[
  \int x^2\sin(3x) \;dx = -\frac13 x^2\cos(3x) + \frac29x\sin(3x) + \frac{2}{27} \cos(3x) + C.
  \]

\end{example}



\begin{example}[Tabular Integration]
Compute the integral using Tabular Integration:
  \[
  \int x^4 e^{-2x} \;dx.
  \]
We construct a table with the derivatives of $u = x^4$ and the anti-derivatives of $dv = e^{-2x} \, dx$.
We add arrows which indicate multiplication of an element of the $u$ column with elements in the $dv$ column one row lower.
We adorn the arrows with alternating $+$ and $-$ signs to indicate either addition or subtraction of terms.
\begin{image}[5cm]
\begin{tikzpicture}
 \matrix (m) [matrix of nodes, ampersand replacement=\&,
            column sep = 1.5cm]{
    u     \&                  dv \\\hline
    $x^4$ \& $e^{-2x}$ \\ [8pt]
     $4x^3$  \& $-\frac12 e^{-2x}$ \\ [8pt]
      $12x^2$ \& $\frac14 e^{-2x}$ \\ [8pt]
       $24x$ \& $-\frac18 e^{-2x}$ \\[8pt]
       $24$ \& $\frac{1}{16} e^{-2x}$ \\[8pt]
       $0$ \& $-\frac{1}{32} e^{-2x}$ \\
 };
 \draw[-latex] (m-2-1) -- (m-3-2) node[midway,above]{$+$};  % <-- It works!
 \draw[-latex] (m-3-1) -- (m-4-2) node[midway,above]{$-$};  % <-- It works!
 \draw[-latex] (m-4-1) -- (m-5-2) node[midway,above]{$+$};  % <-- It works!
 \draw[-latex] (m-5-1) -- (m-6-2) node[midway,above]{$-$};  % <-- It works!
 \draw[-latex] (m-6-1) -- (m-7-2) node[midway,above]{$+$};  % <-- It works!
 %\node at (-.6, 1.5){$+$};
 %\node at (-.6, 0.3){$-$};
 %\node at (-.6, -0.9){$+$};
 %\draw (,) -- (,);
\end{tikzpicture}
\end{image}

Thus,
\[
  \int x^4 e^{-2x} \;dx = -\frac12 x^4 e^{-2x} - \frac44x^3 e^{-2x} - \frac{12}{8} x^2 e^{-2x} - \frac{24}{16}x e^{-2x} - \frac{24}{32} e^{-2x} + C.
  \]
\[
= \left(-\frac12 x^4 - x^3  - \frac{3}{2} x^2  - \frac{3}{2}x  - \frac{3}{4}\right) e^{-2x} + C.
\]
\end{example}


\begin{problem}[Tabular Integration]
Compute the integral using Tabular Integration:
  \[
  \int x^3 \cos(5x) \;dx.
  \]
Complete the table and then enter the final answer.
\begin{center}
\[
\begin{array}{c|c}
		u & dv  \\\hline 
		x^3 & \cos(5x) \\ 
     \answer{3x^2} & \answer{\frac15 \sin(5x)} \\ 
      \answer{6x} & \answer{-\frac{1}{25} \cos(5x)} \\ 
       \answer{6} & \answer{-\frac{1}{125} \sin(5x)} \\
       0 & \answer{\frac{1}{625} \cos(5x)} 
	\end{array}
    \]
\end{center}



Thus,
\[
  \int x^3 \cos(5x) \;dx = \answer{\frac15 x^3\sin(5x) + \frac{3}{25}x^2 \cos(5x) - \frac{6}{125}\sin(5x) -\frac{6}{625}\cos(5x)} +C.
  \]
\end{problem}



\section{INVERSE FUNCTIONS}

We will now look at some examples involving inverse functions like $\ln(x), \tan^{-1}(x), \sin^{-1}(x)$ and $\cos^{-1}(x)$.
We do not (yet) know anti-derivatives for these functions, so we will have to let $u$ equal the inverse function.
 
\begin{example}
Compute the integral:
  \[
  \int x^5\ln(x) \;dx.
  \]
In our previous examples, we would have let $u = x^5$ and $dv = \ln(x) \, dx$.
But notice that,in this case, we do not know an anti-derivative of $\ln(x)$, so 
we would be stumped in finding $v$. Instead, we will let 
\[
u = \ln(x)  \text{  and  } dv = x^5 \, dx\]
then
\[
du = \frac{1}{x} \, dx \text{  and  } v = \frac{x^6}{6}.
\]
The power of $x$ has increased, rather than decreased, as in previous examples, but this 
is offset by the fact that the derivative of the natural log has a completely different form
which is easier to work with. Now, by IBP, we have,

\begin{align*}
  \int x^5\ln(x) \;dx &= \frac{x^6}{6}\ln(x) - \int \frac{x^6}{6}\cdot \frac{1}{x} \, dx\\
\text{Now, the integrand } & \text{ can be simplified by canceling $x$ allowing us to integrate}\\
&= \frac{x^6}{6}\ln(x) - \frac16\int x^5 \, dx\\
&= \frac{x^6}{6}\ln(x) -  \frac{x^6}{36} + C\\
\end{align*}

\end{example}



\begin{problem} 
  Compute the integral:
  \[
  \int x^8 \ln(x) \;dx.
  \]
  
  \begin{hint}
      include $dx$ where appropriate
  \end{hint}
  Let $u = \answer{\ln(x)}$   and   $dv = \answer{x^8 \;dx}$.\\
  Then $du = \answer{\frac{1}{x} dx}$   and   $v = \answer{x^9/9}$.\\
  Thus 
  \[
  \int x^8 \ln(x) dx = \frac{x^9}{9}\ln(x) - \int \answer{\frac{x^8}{9}} \;dx = \answer{\frac{x^9}{9}\ln(x) -\frac{x^9}{81}} + C.
  \]

\end{problem}

This method can be used to anti-differentiate $\ln(x)$ as well.


\begin{problem} 
  Compute the integral:
  \[
  \int \ln(x) \;dx.
  \]
  
  \begin{hint}
      include $dx$ where appropriate
  \end{hint}
  Let $u = \answer{\ln(x)}$   and   $dv = \answer{dx}$.\\
  Then $du = \answer{\frac{1}{x} dx}$   and   $v = \answer{x}$.\\
  Thus 
  \[
  \int  \ln(x) dx = x\ln(x) - \int \answer{1} \;dx = \answer{x\ln(x) -x} + C.
  \]

\end{problem}

\begin{problem} 
  Compute the integral:
  \[
  \int \frac{\ln(x)}{\sqrt x} \;dx.
  \]
  
  \begin{hint}
      include $dx$ where appropriate
  \end{hint}
  \begin{hint}
      rewrite $\frac{1}{\!\sqrt x}$ as a power of $x$
  \end{hint}
  Let $u = \answer{\ln(x)}$   and   $dv = \answer{x^{-1/2} \;dx}$.\\
  Then $du = \answer{\frac{1}{x} dx}$   and   $v = \answer{2x^{1/2}}$.\\
  Thus 
  \[
  \int \frac{\ln(x)}{\!\sqrt x} dx = 2\sqrt x \ln(x) - \int \answer{\frac{2}{\sqrt x}} \;dx = \answer{2\sqrt x \ln(x) -4\sqrt x} + C.
  \]

\end{problem}


\begin{example}
Compute the integral:
  \[
  \int \tan^{-1}(2x) \;dx.
  \]
Let $u = \tan^{-1}(2x)$   and   $dv = dx$.\\
  Then $du = \frac{2}{1+4x^2}\, dx$   and   $v = x$.\\
By IBP,
  \begin{align*}
  \int  \tan^{-1}(2x) dx &= x\tan^{-1}(2x) - \int \frac{2x}{1+4x^2} \;dx\\
  \text{we can use $u$-} & \text{substitution on this integral} \\
  \text{with $u =$} &  1+4x^2 \text{ and } du = 8x \, dx \\
  &= x\tan^{-1}(2x) - \frac14 \int \frac{1}{u} \; du\\
  &= x\tan^{-1}(2x) - \frac14 \ln|u| + C\\
  &= x\tan^{-1}(2x) - \frac14 \ln(1+4x^2) + C
  \end{align*}
  
\end{example}

\begin{problem}
Compute the integral:
  \[
  \int \tan^{-1}(x) \;dx.
  \]

\end{problem}

\begin{problem}
Compute the integral:
  \[
  \int \sin^{-1}(4x) \;dx.
  \]

\end{problem}

\begin{example}
Compute the integral:
  \[
  \int x^3\tan^{-1}(x) \;dx.
  \]
Let $u = \tan^{-1}(x)$   and   $dv = x^3 \, dx$.\\
  Then $du = \frac{1}{1+x^2}\, dx$   and   $v = \frac14 x^4$.\\
By IBP,
  \begin{align*}
  \int  \tan^{-1}(x) dx &= \frac14 x^4\tan^{-1}(x) - \frac14\int \frac{x^4}{1+x^2} \;dx\\
  & \text{now use polynomial long division to get} \\
  &= \frac14 x^4\tan^{-1}(x) - \frac14 \int x^2 - 1 + \frac{1}{1+x^2} \; dx\\
  &= \frac14 x^4\tan^{-1}(x) - \frac14\left(\frac{x^3}{3} - x + \tan^{-1}(x)\right) + C\\
  &= \frac14 x^4\tan^{-1}(x) - \frac{x^3}{12} - \frac{x}{4} + \frac14 \tan^{-1}(x) + C
  \end{align*}
  
\end{example}

\begin{problem}
Compute the integral:
  \[
  \int x^2\tan^{-1}(x) \;dx.
  \]
\begin{hint}
      include $dx$ where appropriate
  \end{hint}
  Let $u = \answer{\tan^{-1}(x)}$   and   $dv = \answer{x^2 \;dx}$.\\
  Then $du = \answer{\frac{1}{1+x^2} dx}$   and   $v = \answer{x^3/3}$.\\
  Thus 
  \[
  \int x^2 \tan^{-1}(x) \; dx = \frac13 x^3 \tan^{-1}(x) - \int \answer{\frac13 \frac{x^3}{1+x^2}} \;dx = 
  \answer{\frac13 x^3 \tan^{-1}(x) - \frac{x^2}{6} + \frac16\ln(1+x^2)} + C.
  \]
\end{problem}

\section{SPECIAL EXAMPLES}

\begin{example}
Compute the integral:
\[
\int e^{-6x}\cos(2x) \; dx.
\]
This classic problem requires IBP twice with the interesting twist that we will not get a final answer, but rather 
a full circle back to the original integral, which we will call $I$.  The final answer will then be obtained by solving an equation 
of the form
\[
I = f(x) + cI,
\]
where $c$ is a constant (not equal to 1).
We begin by letting 
\[
I = \int e^{-6x}\cos(2x) \; dx.
\]
We will now use IBP twice with $u$ being the trig function both times.
For the first application
\[
u = \cos(2x), \quad dv = e^{-6x} \; dx
\]
\[
du = -2\sin(2x) \; dx,  \quad v = -\frac16 e^{-6x},
\]
which gives
\[
I = \int e^{-2x}\cos(6x) \; dx = \overbrace{\cos(2x)}^{u} \cdot 
\overbrace{\left(-\frac16 e^{-6x}\right)}^{v} - \int \overbrace{-\frac16 e^{-6x}}^{v} \overbrace{[-2\sin(2x)] \; dx}^{du}
\]
\[
= -\frac16 e^{-6x}\cos(2x) - \frac13\int  e^{-6x}\sin(2x) \; dx.
\]
For the second application of IBP
\[
u = \sin(2x), \quad dv = e^{-6x} \; dx
\]
\[
du = 2\cos(2x) \; dx,  \quad v = -\frac16 e^{-6x},
\]
we continue with
\[
I = -\frac16 e^{-6x}\cos(2x) - \frac13\Big[ -\frac16 e^{-6x}\sin(2x) - \int  \overbrace{-\frac16 e^{-6x}}^{v} \overbrace{ 2\cos(2x)}^{du} \; dx    \Big]
\]
\[
= -\frac16 e^{-6x}\cos(2x) + \frac{1}{18} e^{-6x}\sin(2x) - \frac19 \int e^{-6x} \cos(2x) \; dx. 
\]
At this point in the computation we must observe that this last integral is the original integral, $I$ (with a coefficient of 1/9). 
Thus we can substitute in $I$ to get
\[
I = -\frac16 e^{-6x}\cos(2x) + \frac{1}{18} e^{-6x}\sin(2x) - \frac19 I,
\]
from which we can solve for $I$:
\[
I + \frac19 I = -\frac16 e^{-6x}\cos(2x) + \frac{1}{18} e^{-6x}\sin(2x) + C
\]
and since $1+\frac19 = \frac{10}{9}$, 
\[
I = \frac{9}{10} \left[-\frac16 e^{-6x}\cos(2x) + \frac{1}{18} e^{-6x}\sin(2x)\right] + C
\]
\[
= -\frac{3}{20} e^{-6x}\cos(2x) + \frac{1}{20} e^{-6x}\sin(2x) + C.
\]
We will now check our answer. It will be slightly easier if we factor out the common factors first:
\[
I = \int e^{-6x}\cos(2x) \; dx = \frac{1}{20} e^{-6x} \left[\sin(2x) - 3\cos(2x)\right] + C.
\]
The derivative of the RHS should yield the integrand:
\[
\frac{d}{dx}\left\{\frac{1}{20} e^{-6x} \left[\sin(2x) - 3\cos(2x)\right]\right\}    = \frac{1}{20} e^{-6x}\left[2\cos(2x) + 6\sin(2x)\right] 
- \frac{6}{20} e^{-6x} \left[\sin(2x) - 3\cos(2x)\right] 
\]
\[
= \frac{1}{20} e^{-6x} \cdot 20\cos(2x) = e^{-6x} \cos(2x),
\]
which is the integrand, so our answer is correct.
\end{example}

\begin{problem}
Compute the integral:
\[
\int e^x\sin(x) \; dx.
\]
\end{problem}


\begin{problem}
Compute the integral:
\[
\int e^{x/2}\sin(2x) \; dx.
\]
\end{problem}

Our next example involves the function $\sec^3(x)$ and arises in the calculation of arc length 
of the parabola $y = x^2$.

At the end of our computation, we will need to recall the following integral
\[
\int \sec(x) \; dx = \ln|\sec(x) + \tan(x)| \; dx.
\]

\begin{example}
Compute the integral:
\[
\int \sec^3(x) \; dx.
\]

We will call the original integral $I$ and use IBP, arriving back at the original integral (as in the previous example).
Let
\[
u = \sec(x), \quad dv = \sec^2(x) \; dx
\]
\[
du = \sec(x)\tan(x) \; dx, \quad v = \tan(x).
\]
Then,
\[
I = \int \sec^3(x) \; dx = \int \overbrace{\sec(x)}^u \cdot \overbrace{\sec^2(x) \; dx}^{dv} = \sec(x) \tan(x) - \int \sec(x) \tan^2(x) \; dx.
\]
Using the trig identity $1 + \tan^2(x) = \sec^2(x)$, gives
\[
I = \sec(x) \tan(x) - \int \sec(x) [\sec^2(x) -1] \; dx
\]
\[
=\sec(x) \tan(x) - \int  [\sec^3(x) -\sec(x)] \; dx 
\]
\[
= \sec(x) \tan(x) + \int \sec(x)\; dx -I,
\]
and so
\[
2I = \sec(x) \tan(x) + \ln|\sec(x) + \tan(x)| + C
\]
and finally,
\[
I = \int \sec^3(x) \; dx = \frac12 \sec(x) \tan(x) + \frac12 \ln|\sec(x) + \tan(x)| + C.
\]
\end{example}

\begin{problem}
Compute the integral
\[
\int 30\sec^3(5x) \; dx = 3\sec(5x) \tan(5x) +  \answer{3\ln|\sec(5x) + \tan(5x)|} + C.
\]
\end{problem}

%Other interesting examples: e^x cos(x), sec^3(x), gamma function, laplace transform

\section{REDUCTION OF POWERS}


\begin{example}
Create a reduction of powers formula for the integral:
\[
\int x^n e^x \; dx.
\]
Use IBP with
\[
u = x^n, \quad dv = e^x \; dx
\]
\[
du = nx^{n-1}\; dx,  \quad v = e^x.
\]
This gives
\[
\int x^n e^x \; dx = x^n e^x - n\int x^{n-1} e^x \; dx.
\]
\end{example}

\begin{problem}
Create a reduction of powers formula for the integral:
\[
\int x^{n-1}e^{-x} \; dx.
\]
\begin{remark}
This integral is the foundation of the Gamma function,
\[
\Gamma(n) = \int_0^\infty x^{n-1}e^{-x} \; dx,
\]
which will be studied in the section on Improper Integrals.
\end{remark}
To create the reduction formula, we will use IBP with
\[
u = \answer{x^{n-1}}, \quad dv = \answer{e^{-x}} \; dx
\]
\[
du = \answer{(n-1)x^{n-2}}\; dx,  \quad v = \answer{-e^{-x}}.
\]
This gives the reduction formula:
\[
\int x^{n-1}e^{-x} \; dx = -x^{n-1}e^{-x} + (n-1)\int \answer{x^{n-2}e^{-x}} \; dx
\]
\begin{remark}
We will see that $\Gamma(n) = (n-1)! = n(n-1)(n-2)\cdot \ldots \cdot 2 \cdot 1$.
\end{remark}

\end{problem}




\begin{example}
Create a reduction of powers formula for the integral:
\[
\int \sec^n(x) \; dx.
\]
Use IBP with
\[
u = \sec^{n-2}(x), \quad dv = \sec^2(x) \; dx
\]
\[
du = (n-2)\sec^{n-3}(x) \cdot \sec(x)\tan(x)\; dx,  \quad v = \tan(x).
\]
This gives
\[
\int \sec^n(x) \; dx = \sec^{n-2}(x)\tan(x) - (n-2)\int \sec^{n-2}(x)\tan^2(x) \; dx
\]
\[
= \sec^{n-2}(x)\tan(x) - (n-2)\int \sec^{n-2}(x)[\sec^2(x) - 1] \; dx
\]
\[
= \sec^{n-2}(x)\tan(x) - (n-2)\int [\sec^n(x)-\sec^{n-2}(x)]\; dx,
\]

and so,

\[
\int \sec^n(x) \; dx = \frac{1}{n-1}\sec^{n-2}(x)\tan(x) + \frac{n-2}{n-1}\int \sec^{n-2}(x) \; dx.
\]

\end{example}



\begin{problem}
Create a reduction of powers formula for the integral:
\[
\int \sin^n(x) \; dx.
\]
\end{problem}



\begin{example}
Create a reduction of powers formula for the integral
\[
\int x^n\sin(x) \; dx.
\]
Use IBP with
\[
u = x^n, \quad dv = \sin(x) \; dx
\]
\[
du = nx^{n-1}\; dx,  \quad v = -\cos(x).
\]
This gives
\[
\int x^n\sin(x) \; dx = -x^n\cos(x) + n\int x^{n-1}\cos(x) \; dx
\]
\end{example}

\begin{problem}
Create a reduction of powers formula for the integral
\[
\int x^n\cos(x) \; dx.
\]
Use IBP with
\[
u = \answer{x^n}, \quad dv = \answer{\cos(x)} \; dx
\]
\[
du = \answer{nx^{n-1}}\; dx,  \quad v = \answer{\sin(x)}.
\]
This gives the reduction formula:
\[
\int x^n\cos(x) \; dx = x^n\sin(x) - n\int \answer{x^{n-1}\sin(x)} \; dx
\]
\end{problem}



%reduction of powers formulas for gamma, powers of sec, x^n sine or cosine

\section{APPLICATIONS}

\begin{example}
Find the volume of the solid of revolution obtained by revolving the region between the curves $y = e^x, y = 0, x = 0$
and $x = 1$ about the $y$-axis.\\

%insert graphic

We use the method of shells and the formula
\[
V = \int_a^b 2\pi r(x)h(x) \; dx
\]
with
\[
a = 0,\; b = 1, \; r(x) = x \; \text{ and } \; h(x) = e^x.
\]
We get
\[
V = \int_0^1 2\pi xe^x \; dx,
\]
which we compute using integration by parts with
\[
u = 2\pi x, \; dv = e^x \; dx
\]
\[
du = 2\pi \;dx, \; v = e^x.
\]
This leads to 
\[
V = 2\pi xe^x \Big|_0^1 - \int_0^1 2\pi e^x \; dx
\] 
\[
= 2\pi e - 2\pi (e-1) = 2\pi.
\]
\end{example} 

\begin{problem}
Find the volume of the solid of revolution obtained by revolving the region between the curves $y = \cos(x), y = 0, x = 0$
and $x = \pi/2$ about the $y$-axis.\\

%insert graphic

We use the method of shells and the formula
\[
V = \int_a^b 2\pi r(x)h(x) \; dx
\]
with
\[
a = 0,\; b = \pi/2, \; r(x) = x \; \text{ and } \; h(x) = \cos(x).
\]
We get
\[
V = \int_0^{\pi/2} \answer{2\pi x \cos(x)} \; dx,
\]
which we compute using integration by parts with
\[
u = 2\pi x, \; dv = \cos(x) \; dx
\]
\[
du = 2\pi \;dx, \; v = \sin(x).
\]
This leads to 
\[
V = 2\pi x\sin(x) \Big|_0^{\pi/2} - \int_0^{\pi/2} \answer{2\pi \sin(x)} \; dx
\] 
\[
= \answer{\pi^2 - 2\pi}.
\]
\end{problem} 


%an area, volume and length each requiring IBP.

%\section{MORE PROBLEMS}

%\section{QUIZ}

\begin{center}
\begin{foldable}
\unfoldable{Here is a detailed, lecture style video on integration by parts:}
\youtube{2jSr68xaAfs}
\end{foldable}
\end{center}





\end{document}









































































































\begin{image}
\begin{tikzpicture}
 \matrix (m) [matrix of nodes, ampersand replacement=\&,
            column sep = 1.5cm]{
    u     \&                  dv \\\hline
    $x^3$ \& $\cos(5x)$ \\ [8pt]
     $\answer{3x^2}$  \& $\answer{\frac15 \sin(5x)}$ \\ [8pt]
      $\answer{6x}$ \& $\answer{-\frac{1}{25} \cos(5x)}$ \\ [8pt]
       $\answer{6}$ \& $\answer{-\frac{1}{125} \sin(5x)}$ \\[8pt]
       $0$ \& $\answer{\frac{1}{625} \cos(5x)}$\\     
 };
 \draw[-latex] (m-2-1) -- (m-3-2) node[midway,above]{$+$};  % <-- It works!
 \draw[-latex] (m-3-1) -- (m-4-2) node[midway,above]{$-$};  % <-- It works!
 \draw[-latex] (m-4-1) -- (m-5-2) node[midway,above]{$+$};  % <-- It works!
 \draw[-latex] (m-5-1) -- (m-6-2) node[midway,above]{$-$};  % <-- It works!
\end{tikzpicture}
\end{image}



\begin{example} %example #15
Find $h'(x)$ if $h(x) = x^{\sin(x)}$.\\
We will use the fact that the exponential and logarithm functions are inverses,
\[e^{\ln(x)} = x,\]
and the exponent property of logarithms, 
\[\ln(x^n) = n\ln(x),\]
to rewrite $h(x)$.  We have 
\[h(x) = x^{\sin(x)} = e^{\ln(x^{\sin(x)})} = e^{\sin(x)\ln(x)}\]
and we can now compute $h'(x)$ using a combination of the chain rule and product rule.
We can write $h(x)$ as a composition, $f(g(x))$ with 
\[g(x) = \sin(x)\ln(x) \quad \text{and} \quad f(x) = e^x.\]
Then to find $g'(x)$ we us the product rule and we get $g'(x) = \frac{\sin(x)}{x} + \cos(x)\ln(x)$.
Next $f'(x) = e^x$ and 
hence $f'(g(x)) = e^{g(x)} = e^{\sin(x)\ln(x)} = x^{\sin(x)}$.
We can then conclude $h'(x) = f'(g(x))g'(x) = x^{\sin(x)} \left[ \frac{\sin(x)}{x} + \cos(x)\ln(x)\right]$.
\end{example}

%more question formats below













%\begin{verbatim}
\begin{question}
What is your favorite color?
\begin{multipleChoice}
\choice[correct]{Rainbow}
\choice{Blue}
\choice{Green}
\choice{Red}
\end{multipleChoice}
\begin{freeResponse}
Hello
\end{freeResponse}
\end{question}
%\end{verbatim}





\begin{question}
  Which one will you choose?
  \begin{multipleChoice}
    \choice[correct]{I'm correct.}
    \choice{I'm wrong.}
    \choice{I'm wrong too.}
  \end{multipleChoice}
\end{question}


\begin{question}
  Which one will you choose?
  \begin{selectAll}
    \choice[correct]{I'm correct.}
    \choice{I'm wrong.}
    \choice[correct]{I'm also correct.}
    \choice{I'm wrong too.}
  \end{selectAll}
\end{question}


\begin{freeResponse}
What is the chain rule used for?
\end{freeResponse}
