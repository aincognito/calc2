\documentclass[handout]{ximera}

%% You can put user macros here
%% However, you cannot make new environments



\newcommand{\ffrac}[2]{\frac{\text{\footnotesize $#1$}}{\text{\footnotesize $#2$}}}
\newcommand{\vasymptote}[2][]{
    \draw [densely dashed,#1] ({rel axis cs:0,0} -| {axis cs:#2,0}) -- ({rel axis cs:0,1} -| {axis cs:#2,0});
}


\graphicspath{{./}{firstExample/}}

\usepackage{amsmath}
\usepackage{amssymb}
\usepackage{array}
\usepackage[makeroom]{cancel} %% for strike outs
\usepackage{pgffor} %% required for integral for loops
\usepackage{tikz}
\usepackage{tikz-cd}
\usepackage{tkz-euclide}
\usetikzlibrary{shapes.multipart}


\usetkzobj{all}
\tikzstyle geometryDiagrams=[ultra thick,color=blue!50!black]


\usetikzlibrary{arrows}
\tikzset{>=stealth,commutative diagrams/.cd,
  arrow style=tikz,diagrams={>=stealth}} %% cool arrow head
\tikzset{shorten <>/.style={ shorten >=#1, shorten <=#1 } } %% allows shorter vectors

\usetikzlibrary{backgrounds} %% for boxes around graphs
\usetikzlibrary{shapes,positioning}  %% Clouds and stars
\usetikzlibrary{matrix} %% for matrix
\usepgfplotslibrary{polar} %% for polar plots
\usepgfplotslibrary{fillbetween} %% to shade area between curves in TikZ



%\usepackage[width=4.375in, height=7.0in, top=1.0in, papersize={5.5in,8.5in}]{geometry}
%\usepackage[pdftex]{graphicx}
%\usepackage{tipa}
%\usepackage{txfonts}
%\usepackage{textcomp}
%\usepackage{amsthm}
%\usepackage{xy}
%\usepackage{fancyhdr}
%\usepackage{xcolor}
%\usepackage{mathtools} %% for pretty underbrace % Breaks Ximera
%\usepackage{multicol}



\newcommand{\RR}{\mathbb R}
\newcommand{\R}{\mathbb R}
\newcommand{\C}{\mathbb C}
\newcommand{\N}{\mathbb N}
\newcommand{\Z}{\mathbb Z}
\newcommand{\dis}{\displaystyle}
%\renewcommand{\d}{\,d\!}
\renewcommand{\d}{\mathop{}\!d}
\newcommand{\dd}[2][]{\frac{\d #1}{\d #2}}
\newcommand{\pp}[2][]{\frac{\partial #1}{\partial #2}}
\renewcommand{\l}{\ell}
\newcommand{\ddx}{\frac{d}{\d x}}

\newcommand{\zeroOverZero}{\ensuremath{\boldsymbol{\tfrac{0}{0}}}}
\newcommand{\inftyOverInfty}{\ensuremath{\boldsymbol{\tfrac{\infty}{\infty}}}}
\newcommand{\zeroOverInfty}{\ensuremath{\boldsymbol{\tfrac{0}{\infty}}}}
\newcommand{\zeroTimesInfty}{\ensuremath{\small\boldsymbol{0\cdot \infty}}}
\newcommand{\inftyMinusInfty}{\ensuremath{\small\boldsymbol{\infty - \infty}}}
\newcommand{\oneToInfty}{\ensuremath{\boldsymbol{1^\infty}}}
\newcommand{\zeroToZero}{\ensuremath{\boldsymbol{0^0}}}
\newcommand{\inftyToZero}{\ensuremath{\boldsymbol{\infty^0}}}


\newcommand{\numOverZero}{\ensuremath{\boldsymbol{\tfrac{\#}{0}}}}
\newcommand{\dfn}{\textbf}
%\newcommand{\unit}{\,\mathrm}
\newcommand{\unit}{\mathop{}\!\mathrm}
%\newcommand{\eval}[1]{\bigg[ #1 \bigg]}
\newcommand{\eval}[1]{ #1 \bigg|}
\newcommand{\seq}[1]{\left( #1 \right)}
\renewcommand{\epsilon}{\varepsilon}
\renewcommand{\iff}{\Leftrightarrow}

\DeclareMathOperator{\arccot}{arccot}
\DeclareMathOperator{\arcsec}{arcsec}
\DeclareMathOperator{\arccsc}{arccsc}
\DeclareMathOperator{\si}{Si}
\DeclareMathOperator{\proj}{proj}
\DeclareMathOperator{\scal}{scal}
\DeclareMathOperator{\cis}{cis}
\DeclareMathOperator{\Arg}{Arg}
%\DeclareMathOperator{\arg}{arg}
\DeclareMathOperator{\Rep}{Re}
\DeclareMathOperator{\Imp}{Im}
\DeclareMathOperator{\sech}{sech}
\DeclareMathOperator{\csch}{csch}
\DeclareMathOperator{\Log}{Log}

\newcommand{\tightoverset}[2]{% for arrow vec
  \mathop{#2}\limits^{\vbox to -.5ex{\kern-0.75ex\hbox{$#1$}\vss}}}
\newcommand{\arrowvec}{\overrightarrow}
\renewcommand{\vec}{\mathbf}
\newcommand{\veci}{{\boldsymbol{\hat{\imath}}}}
\newcommand{\vecj}{{\boldsymbol{\hat{\jmath}}}}
\newcommand{\veck}{{\boldsymbol{\hat{k}}}}
\newcommand{\vecl}{\boldsymbol{\l}}
\newcommand{\utan}{\vec{\hat{t}}}
\newcommand{\unormal}{\vec{\hat{n}}}
\newcommand{\ubinormal}{\vec{\hat{b}}}

\newcommand{\dotp}{\bullet}
\newcommand{\cross}{\boldsymbol\times}
\newcommand{\grad}{\boldsymbol\nabla}
\newcommand{\divergence}{\grad\dotp}
\newcommand{\curl}{\grad\cross}
%% Simple horiz vectors
\renewcommand{\vector}[1]{\left\langle #1\right\rangle}


\outcome{Differentiate a power series}

\title{3.12 Differentiation of Power Series}

\begin{document}

\begin{abstract}
We differentiate power series term by term.
\end{abstract}

\maketitle



Suppose that the power series 
\[
\sum_{n=0}^\infty a_n x^n
\]
converges for all $x$ in some open interval $I$. Then, on this interval, the power series represents a differentiable function and its derivative is given by 
\[
\frac{d}{dx} \left( \sum_{n=0}^\infty a_n x^n \right) = \sum_{n=0}^\infty  \frac{d}{dx}\big( a_n x^n \big)  
= \sum_{n=1}^\infty   n\cdot a_n x^{n-1}
\]
In other words, the derivative of a power series is a power series. and the derivative is computed term by term, as we would differentiate a polynomial.

\begin{example}[example 1]
Find the derivative of the power series
\[
\sum_{n=0}^\infty x^n
\]
This power series converges on the interval $(-1, 1)$.
On that interval, its derivative is given by
\[
\frac{d}{dx}\left( \sum_{n=0}^\infty x^n \right)= \sum_{n=0}^\infty \frac{d}{dx} \big(x^n\big) = \sum_{n=1}^\infty nx^{n-1}.
\]
We can re-index to rewrite this as
\[
\sum_{n=1}^\infty nx^{n-1} = \sum_{n=0}^\infty (n+1)x^n
\]
We have seen that the original power series above represents the function 
\[
f(x) = \frac{1}{1-x} = \sum_{n=0}^\infty x^n
\]
on the interval $(-1, 1)$. The derivative of this function is
\[
f'(x) = \frac{1}{(1-x)^2} \;\; \text{(verify)}
\]
Thus, we have a power series representation for this function:
\[
\frac{1}{(1-x)^2} = \sum_{n=0}^\infty (n+1)x^n
\]
\end{example}



\begin{problem}(problem 1)
Find the derivatives of the following power series:
\[
a) \; \sum_{n=1}^\infty \frac{x^n}{n},\qquad  b) \; \sum_{n=0}^\infty \frac{x^n}{2^n},
\]
\[
 c) \; \sum_{n=0}^\infty \frac{x^n}{n!},\qquad d) \; \sum_{n=0}^\infty x^{2n}
\]
\end{problem}

\end{document}








\begin{center}
\begin{foldable}
\unfoldable{Here is a detailed, lecture style video on average value:}
\youtube{gRJTe1i1muo}
\end{foldable}
\end{center}



\begin{tikzpicture}
\node[inner sep=0pt] (bird) at (0,0)
    {\includegraphics[width=.25\textwidth]{bird.jpg}};
\node[inner sep=0pt] (spiral) at (5,0)
    {\includegraphics[width=.25\textwidth]{spiral.jpg}};
\node[inner sep=0pt] (taylor) at (5,-5)
    {\includegraphics[width=1\textwidth]{taylor.png}};
\draw[->,thick] (bird) -- (spiral.north west)
    node[midway,fill=white] {bird};
\draw[->,thick] (spiral.south west) -- (bird.south east)
    node[midway,fill=white] {spiral};    
\end{tikzpicture}




\begin{example} %example #15
Find $h'(x)$ if $h(x) = x^{\sin(x)}$.\\
We will use the fact that the exponential and logarithm functions are inverses,
\[e^{\ln(x)} = x,\]
and the exponent property of logarithms, 
\[\ln(x^n) = n\ln(x),\]
to rewrite $h(x)$.  We have 
\[h(x) = x^{\sin(x)} = e^{\ln(x^{\sin(x)})} = e^{\sin(x)\ln(x)}\]
and we can now compute $h'(x)$ using a combination of the chain rule and product rule.
We can write $h(x)$ as a composition, $f(g(x))$ with 
\[g(x) = \sin(x)\ln(x) \quad \text{and} \quad f(x) = e^x.\]
Then to find $g'(x)$ we us the product rule and we get $g'(x) = \frac{\sin(x)}{x} + \cos(x)\ln(x)$.
Next $f'(x) = e^x$ and 
hence $f'(g(x)) = e^{g(x)} = e^{\sin(x)\ln(x)} = x^{\sin(x)}$.
We can then conclude $h'(x) = f'(g(x))g'(x) = x^{\sin(x)} \left[ \frac{\sin(x)}{x} + \cos(x)\ln(x)\right]$.
\end{example}

%more question formats below













%\begin{verbatim}
\begin{question}
What is your favorite color?
\begin{multipleChoice}
\choice[correct]{Rainbow}
\choice{Blue}
\choice{Green}
\choice{Red}
\end{multipleChoice}
\begin{freeResponse}
Hello
\end{freeResponse}
\end{question}
%\end{verbatim}





\begin{question}
  Which one will you choose?
  \begin{multipleChoice}
    \choice[correct]{I'm correct.}
    \choice{I'm wrong.}
    \choice{I'm wrong too.}
  \end{multipleChoice}
\end{question}


\begin{question}
  Which one will you choose?
  \begin{selectAll}
    \choice[correct]{I'm correct.}
    \choice{I'm wrong.}
    \choice[correct]{I'm also correct.}
    \choice{I'm wrong too.}
  \end{selectAll}
\end{question}


\begin{freeResponse}
What is the chain rule used for?
\end{freeResponse}
