\documentclass{ximera}

%% You can put user macros here
%% However, you cannot make new environments



\newcommand{\ffrac}[2]{\frac{\text{\footnotesize $#1$}}{\text{\footnotesize $#2$}}}
\newcommand{\vasymptote}[2][]{
    \draw [densely dashed,#1] ({rel axis cs:0,0} -| {axis cs:#2,0}) -- ({rel axis cs:0,1} -| {axis cs:#2,0});
}


\graphicspath{{./}{firstExample/}}

\usepackage{amsmath}
\usepackage{amssymb}
\usepackage{array}
\usepackage[makeroom]{cancel} %% for strike outs
\usepackage{pgffor} %% required for integral for loops
\usepackage{tikz}
\usepackage{tikz-cd}
\usepackage{tkz-euclide}
\usetikzlibrary{shapes.multipart}


\usetkzobj{all}
\tikzstyle geometryDiagrams=[ultra thick,color=blue!50!black]


\usetikzlibrary{arrows}
\tikzset{>=stealth,commutative diagrams/.cd,
  arrow style=tikz,diagrams={>=stealth}} %% cool arrow head
\tikzset{shorten <>/.style={ shorten >=#1, shorten <=#1 } } %% allows shorter vectors

\usetikzlibrary{backgrounds} %% for boxes around graphs
\usetikzlibrary{shapes,positioning}  %% Clouds and stars
\usetikzlibrary{matrix} %% for matrix
\usepgfplotslibrary{polar} %% for polar plots
\usepgfplotslibrary{fillbetween} %% to shade area between curves in TikZ



%\usepackage[width=4.375in, height=7.0in, top=1.0in, papersize={5.5in,8.5in}]{geometry}
%\usepackage[pdftex]{graphicx}
%\usepackage{tipa}
%\usepackage{txfonts}
%\usepackage{textcomp}
%\usepackage{amsthm}
%\usepackage{xy}
%\usepackage{fancyhdr}
%\usepackage{xcolor}
%\usepackage{mathtools} %% for pretty underbrace % Breaks Ximera
%\usepackage{multicol}



\newcommand{\RR}{\mathbb R}
\newcommand{\R}{\mathbb R}
\newcommand{\C}{\mathbb C}
\newcommand{\N}{\mathbb N}
\newcommand{\Z}{\mathbb Z}
\newcommand{\dis}{\displaystyle}
%\renewcommand{\d}{\,d\!}
\renewcommand{\d}{\mathop{}\!d}
\newcommand{\dd}[2][]{\frac{\d #1}{\d #2}}
\newcommand{\pp}[2][]{\frac{\partial #1}{\partial #2}}
\renewcommand{\l}{\ell}
\newcommand{\ddx}{\frac{d}{\d x}}

\newcommand{\zeroOverZero}{\ensuremath{\boldsymbol{\tfrac{0}{0}}}}
\newcommand{\inftyOverInfty}{\ensuremath{\boldsymbol{\tfrac{\infty}{\infty}}}}
\newcommand{\zeroOverInfty}{\ensuremath{\boldsymbol{\tfrac{0}{\infty}}}}
\newcommand{\zeroTimesInfty}{\ensuremath{\small\boldsymbol{0\cdot \infty}}}
\newcommand{\inftyMinusInfty}{\ensuremath{\small\boldsymbol{\infty - \infty}}}
\newcommand{\oneToInfty}{\ensuremath{\boldsymbol{1^\infty}}}
\newcommand{\zeroToZero}{\ensuremath{\boldsymbol{0^0}}}
\newcommand{\inftyToZero}{\ensuremath{\boldsymbol{\infty^0}}}


\newcommand{\numOverZero}{\ensuremath{\boldsymbol{\tfrac{\#}{0}}}}
\newcommand{\dfn}{\textbf}
%\newcommand{\unit}{\,\mathrm}
\newcommand{\unit}{\mathop{}\!\mathrm}
%\newcommand{\eval}[1]{\bigg[ #1 \bigg]}
\newcommand{\eval}[1]{ #1 \bigg|}
\newcommand{\seq}[1]{\left( #1 \right)}
\renewcommand{\epsilon}{\varepsilon}
\renewcommand{\iff}{\Leftrightarrow}

\DeclareMathOperator{\arccot}{arccot}
\DeclareMathOperator{\arcsec}{arcsec}
\DeclareMathOperator{\arccsc}{arccsc}
\DeclareMathOperator{\si}{Si}
\DeclareMathOperator{\proj}{proj}
\DeclareMathOperator{\scal}{scal}
\DeclareMathOperator{\cis}{cis}
\DeclareMathOperator{\Arg}{Arg}
%\DeclareMathOperator{\arg}{arg}
\DeclareMathOperator{\Rep}{Re}
\DeclareMathOperator{\Imp}{Im}
\DeclareMathOperator{\sech}{sech}
\DeclareMathOperator{\csch}{csch}
\DeclareMathOperator{\Log}{Log}

\newcommand{\tightoverset}[2]{% for arrow vec
  \mathop{#2}\limits^{\vbox to -.5ex{\kern-0.75ex\hbox{$#1$}\vss}}}
\newcommand{\arrowvec}{\overrightarrow}
\renewcommand{\vec}{\mathbf}
\newcommand{\veci}{{\boldsymbol{\hat{\imath}}}}
\newcommand{\vecj}{{\boldsymbol{\hat{\jmath}}}}
\newcommand{\veck}{{\boldsymbol{\hat{k}}}}
\newcommand{\vecl}{\boldsymbol{\l}}
\newcommand{\utan}{\vec{\hat{t}}}
\newcommand{\unormal}{\vec{\hat{n}}}
\newcommand{\ubinormal}{\vec{\hat{b}}}

\newcommand{\dotp}{\bullet}
\newcommand{\cross}{\boldsymbol\times}
\newcommand{\grad}{\boldsymbol\nabla}
\newcommand{\divergence}{\grad\dotp}
\newcommand{\curl}{\grad\cross}
%% Simple horiz vectors
\renewcommand{\vector}[1]{\left\langle #1\right\rangle}


\outcome{Find the Average Value of a Function}

\title{1.4 Average Value}

\begin{document}

\begin{abstract}
We find the average value of a function.
\end{abstract}

\maketitle

\section{Average value}

We will define and compute the average value of a function on an interval. 
%and find the value(s) of $x$ for which the function equals the average (MVT for Integrals).

\begin{definition}[Average Value] The average value of an integrable function, $f(x)$, on the interval $[a,b]$ is given by
\[
f_{ave} = \frac{1}{b-a} \int_a^b f(x) \, dx.
\]
\end{definition}

\begin{example}[example 1]
Find the average value of $f(x) = 3e^{4x}$ over the interval $[0, 2]$.\\
Using the definition of average value, we have
\[
f_{ave} = \frac{1}{2-0} \int_0^2 3e^{4x} \, dx
\]
\[
= \frac12 \cdot \frac34e^{4x} \bigg|_0^2
\]
\[
= \frac38 \left(e^8 - 1\right).
\]
\end{example}




\begin{problem}(problem 1a)
Find the average value of the function $f(x) = x^2$ over the interval $[0, 3]$.\\
The average value is $\answer{3}$.
\end{problem}





\begin{problem}(problem 1b)
Find the average value of the function $f(x) = 3e^{2x}$ over the interval $[0, 2]$.\\
The average value is $\answer{3/4 (e^4 - 1)}$.
\end{problem}




\begin{problem}(problem 1c)
Find the average value of the function $f(x) = \displaystyle{\frac{1}{2x+3}}$ over the interval $[-1, 2]$.\\
The average value is $\answer{1/6 \ln(7)}$.
\end{problem}




\begin{problem}(problem 1d)
Find the average value of the function $f(x) = \sin(2x)$ over the interval $[0, \pi/4]$.\\
The average value is $\answer{2/\pi}$.
\end{problem}



\begin{example}[example 2] Find the average value of $\cos^2\theta$ on the interval $[0, \pi/2]$.\\
According to the definition of average value, we have
\[
f_{ave} = \frac{1}{\frac{\pi}{2} - 0} \int_0^{\pi/2} \cos^2 \theta \, d\theta = \frac{2}{\pi} \int_0^{\pi/2} \cos^2 \theta \, d\theta.
\]
To integrate the function $\cos^2 \theta$, we need the trigonometric identity:
\[
\cos^2 \theta = \frac{1 + \cos 2\theta}{2}.
\]
Returning to our calculation,
\[
f_{ave} = \frac{2}{\pi} \int_0^{\pi/2} \frac{1 + \cos 2\theta}{2} \, d\theta
\]
\[
 = \frac{2}{\pi} \int_0^{\pi/2} \left(\frac12 + \frac12\cos 2\theta\right) \, d\theta
\]
\[
= \frac{2}{\pi} \left(\frac12 \theta + \frac14 \sin 2\theta \right) \bigg|_0^{\pi/2}
\]
\[
=\frac{2}{\pi} \cdot \frac{\pi}{4} = \frac12.
\]


\end{example}



\begin{problem}(problem 2)
Find the average value of $\sin^2 \theta$ over the interval $[0, \pi]$.\\

    \begin{hint}
      use the trigonometric identity $$\sin^2 \theta = \frac{1 - \cos 2\theta}{2}$$
    \end{hint}
    
		
The average value is $\answer[given]{1/2}$
\end{problem}

\begin{example}[example 3]
Find the average value of $\displaystyle{\frac{1}{x^2 + 4}}$ over the interval $[0, 2]$.\\
Using the definition of average value, we have
\[
f_{ave} = \frac{1}{2-0} \int_0^2 \frac{1}{x^2 + 4} \, dx
\]
\[
= \frac12 \int_0^2 \frac{1}{4\left(\frac{x^2}{4} + 1\right)} \, dx
\]
\[
= \frac18 \int_0^2 \frac{1}{\left(\frac{x}{2}\right)^2 + 1} \, dx.
\]
We now use a $u$-substitution, with $u = x/2$ and $du = \frac12 dx$, thus

\[
\int \frac{1}{\left(\frac{x}{2}\right)^2 + 1} \, dx = 2\int\frac{1}{u^2 + 1} \, du = 2\tan^{-1}(u) + C = 2\tan^{-1}\left(\frac{x}{2}\right) + C.
\]
Continuing our average value calculation, we have
\[
f_{ave} = \frac18 \cdot 2\tan^{-1}\left(\frac{x}{2}\right) \bigg|_0^2
\]
\[
= \frac14 \left[\tan^{-1}(1) - \tan^{-1}(0)\right]
\]
\[
= \frac14 \left(\frac{\pi}{4} - 0\right) = \frac{\pi}{16}.
\]



\end{example}

\begin{problem}(problem 3)
Find the average value of $\displaystyle{\frac{1}{x^2 + 9}}$ over the interval $[-3, 3]$.\\
First, use a u-substitution to compute the indefinte integral:
\[
\int \frac{1}{x^2 + 9} = \answer{1/3 \tan^{-1}(x/3)} + C.
\]
Now find the average value of the function:
\[
f_{ave} = \answer{\pi/36}.
\]


\end{problem}


\begin{theorem}[Mean Value Theorem for Integrals]
If $f(x)$ is continuous on the interval $[a, b]$ then there is a value $x = c$ betwwen $a$ and $b$ such that 
$f_{ave} = f(c)$, i.e.,
\[
f(c) = \frac{1}{b-a} \int_a^b f(x) \, dx,
\]
for some number $c$ in the interval $(a,b)$.
\end{theorem}

\begin{remark}
For a positive function, $f(x)$, the Mean Value theorem for Integrals implies that the area under the graph of 
$f(x)$ over the interval $[a,b]$ is equal to the area of a rectangle with base $b-a$ and height $f(c)$.


\end{remark}

 
\begin{image}
\begin{tikzpicture}
\begin{axis}[axis x line=  center, axis y line = none, xtick={-1,.1547, 1, 1.8}, 
xticklabels={$a$, $c$, $b$, $x$}, title={Mean Value Theorem for Integrals}] 

\addplot[name path = A, domain=-1:1, samples = 100, color=black, thick]{1 + 1/2*(x+1)^2};
\addplot[name path = B, domain=-1:1, samples=100, color=black]{0};
\addplot[name path = C, domain=-1:1, samples = 100, color=black, thick]{5/3};
\addplot[blue!10] fill between[of=A and B];
%\addplot[blue!20] fill between[of=C and B];
\addplot[thick, dashed] coordinates {(-1,0) (-1, 5/3)};
\addplot[thick, dashed] coordinates {(1,0) (1, 3)};
\addplot[thin] coordinates{(-1.55, 5/3) (-1.45, 5/3)};
%\addplot[white] coordinates {(1.5,-0.3) (1.8,-.3)};
\addplot[<->] coordinates {(-1.5,-1) (-1.5, 3)};
\addplot[<-] coordinates {(-1.85,0) (1.5, 0)};


\node at (axis cs: 0,-.9){ $ \int_a^b f(x) \; dx = f(c)(b-a)$};
\node at (axis cs: 0.2,2.35){$y = f(x)$};
\node at (axis cs: 1.45,-.15){$x$};
\node at (axis cs: -1.6,2.9){$y$};
\node at (axis cs: -1.7,5/3){$f(c)$};


\end{axis}
%\legend{$y = f(x)$, , secant line, tangent line, };
\end{tikzpicture}
\end{image}








\begin{center}
\begin{foldable}
\unfoldable{Here is a detailed, lecture style video on average value:}
\youtube{gRJTe1i1muo}
\end{foldable}
\end{center}





\end{document}


\begin{example} %example #15
Find $h'(x)$ if $h(x) = x^{\sin(x)}$.\\
We will use the fact that the exponential and logarithm functions are inverses,
\[e^{\ln(x)} = x,\]
and the exponent property of logarithms, 
\[\ln(x^n) = n\ln(x),\]
to rewrite $h(x)$.  We have 
\[h(x) = x^{\sin(x)} = e^{\ln(x^{\sin(x)})} = e^{\sin(x)\ln(x)}\]
and we can now compute $h'(x)$ using a combination of the chain rule and product rule.
We can write $h(x)$ as a composition, $f(g(x))$ with 
\[g(x) = \sin(x)\ln(x) \quad \text{and} \quad f(x) = e^x.\]
Then to find $g'(x)$ we us the product rule and we get $g'(x) = \frac{\sin(x)}{x} + \cos(x)\ln(x)$.
Next $f'(x) = e^x$ and 
hence $f'(g(x)) = e^{g(x)} = e^{\sin(x)\ln(x)} = x^{\sin(x)}$.
We can then conclude $h'(x) = f'(g(x))g'(x) = x^{\sin(x)} \left[ \frac{\sin(x)}{x} + \cos(x)\ln(x)\right]$.
\end{example}

%more question formats below













%\begin{verbatim}
\begin{question}
What is your favorite color?
\begin{multipleChoice}
\choice[correct]{Rainbow}
\choice{Blue}
\choice{Green}
\choice{Red}
\end{multipleChoice}
\begin{freeResponse}
Hello
\end{freeResponse}
\end{question}
%\end{verbatim}





\begin{question}
  Which one will you choose?
  \begin{multipleChoice}
    \choice[correct]{I'm correct.}
    \choice{I'm wrong.}
    \choice{I'm wrong too.}
  \end{multipleChoice}
\end{question}


\begin{question}
  Which one will you choose?
  \begin{selectAll}
    \choice[correct]{I'm correct.}
    \choice{I'm wrong.}
    \choice[correct]{I'm also correct.}
    \choice{I'm wrong too.}
  \end{selectAll}
\end{question}


\begin{freeResponse}
What is the chain rule used for?
\end{freeResponse}
