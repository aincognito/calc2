\documentclass[handout]{ximera}
\usepackage{tcolorbox}
%% You can put user macros here
%% However, you cannot make new environments



\newcommand{\ffrac}[2]{\frac{\text{\footnotesize $#1$}}{\text{\footnotesize $#2$}}}
\newcommand{\vasymptote}[2][]{
    \draw [densely dashed,#1] ({rel axis cs:0,0} -| {axis cs:#2,0}) -- ({rel axis cs:0,1} -| {axis cs:#2,0});
}


\graphicspath{{./}{firstExample/}}

\usepackage{amsmath}
\usepackage{amssymb}
\usepackage{array}
\usepackage[makeroom]{cancel} %% for strike outs
\usepackage{pgffor} %% required for integral for loops
\usepackage{tikz}
\usepackage{tikz-cd}
\usepackage{tkz-euclide}
\usetikzlibrary{shapes.multipart}


\usetkzobj{all}
\tikzstyle geometryDiagrams=[ultra thick,color=blue!50!black]


\usetikzlibrary{arrows}
\tikzset{>=stealth,commutative diagrams/.cd,
  arrow style=tikz,diagrams={>=stealth}} %% cool arrow head
\tikzset{shorten <>/.style={ shorten >=#1, shorten <=#1 } } %% allows shorter vectors

\usetikzlibrary{backgrounds} %% for boxes around graphs
\usetikzlibrary{shapes,positioning}  %% Clouds and stars
\usetikzlibrary{matrix} %% for matrix
\usepgfplotslibrary{polar} %% for polar plots
\usepgfplotslibrary{fillbetween} %% to shade area between curves in TikZ



%\usepackage[width=4.375in, height=7.0in, top=1.0in, papersize={5.5in,8.5in}]{geometry}
%\usepackage[pdftex]{graphicx}
%\usepackage{tipa}
%\usepackage{txfonts}
%\usepackage{textcomp}
%\usepackage{amsthm}
%\usepackage{xy}
%\usepackage{fancyhdr}
%\usepackage{xcolor}
%\usepackage{mathtools} %% for pretty underbrace % Breaks Ximera
%\usepackage{multicol}



\newcommand{\RR}{\mathbb R}
\newcommand{\R}{\mathbb R}
\newcommand{\C}{\mathbb C}
\newcommand{\N}{\mathbb N}
\newcommand{\Z}{\mathbb Z}
\newcommand{\dis}{\displaystyle}
%\renewcommand{\d}{\,d\!}
\renewcommand{\d}{\mathop{}\!d}
\newcommand{\dd}[2][]{\frac{\d #1}{\d #2}}
\newcommand{\pp}[2][]{\frac{\partial #1}{\partial #2}}
\renewcommand{\l}{\ell}
\newcommand{\ddx}{\frac{d}{\d x}}

\newcommand{\zeroOverZero}{\ensuremath{\boldsymbol{\tfrac{0}{0}}}}
\newcommand{\inftyOverInfty}{\ensuremath{\boldsymbol{\tfrac{\infty}{\infty}}}}
\newcommand{\zeroOverInfty}{\ensuremath{\boldsymbol{\tfrac{0}{\infty}}}}
\newcommand{\zeroTimesInfty}{\ensuremath{\small\boldsymbol{0\cdot \infty}}}
\newcommand{\inftyMinusInfty}{\ensuremath{\small\boldsymbol{\infty - \infty}}}
\newcommand{\oneToInfty}{\ensuremath{\boldsymbol{1^\infty}}}
\newcommand{\zeroToZero}{\ensuremath{\boldsymbol{0^0}}}
\newcommand{\inftyToZero}{\ensuremath{\boldsymbol{\infty^0}}}


\newcommand{\numOverZero}{\ensuremath{\boldsymbol{\tfrac{\#}{0}}}}
\newcommand{\dfn}{\textbf}
%\newcommand{\unit}{\,\mathrm}
\newcommand{\unit}{\mathop{}\!\mathrm}
%\newcommand{\eval}[1]{\bigg[ #1 \bigg]}
\newcommand{\eval}[1]{ #1 \bigg|}
\newcommand{\seq}[1]{\left( #1 \right)}
\renewcommand{\epsilon}{\varepsilon}
\renewcommand{\iff}{\Leftrightarrow}

\DeclareMathOperator{\arccot}{arccot}
\DeclareMathOperator{\arcsec}{arcsec}
\DeclareMathOperator{\arccsc}{arccsc}
\DeclareMathOperator{\si}{Si}
\DeclareMathOperator{\proj}{proj}
\DeclareMathOperator{\scal}{scal}
\DeclareMathOperator{\cis}{cis}
\DeclareMathOperator{\Arg}{Arg}
%\DeclareMathOperator{\arg}{arg}
\DeclareMathOperator{\Rep}{Re}
\DeclareMathOperator{\Imp}{Im}
\DeclareMathOperator{\sech}{sech}
\DeclareMathOperator{\csch}{csch}
\DeclareMathOperator{\Log}{Log}

\newcommand{\tightoverset}[2]{% for arrow vec
  \mathop{#2}\limits^{\vbox to -.5ex{\kern-0.75ex\hbox{$#1$}\vss}}}
\newcommand{\arrowvec}{\overrightarrow}
\renewcommand{\vec}{\mathbf}
\newcommand{\veci}{{\boldsymbol{\hat{\imath}}}}
\newcommand{\vecj}{{\boldsymbol{\hat{\jmath}}}}
\newcommand{\veck}{{\boldsymbol{\hat{k}}}}
\newcommand{\vecl}{\boldsymbol{\l}}
\newcommand{\utan}{\vec{\hat{t}}}
\newcommand{\unormal}{\vec{\hat{n}}}
\newcommand{\ubinormal}{\vec{\hat{b}}}

\newcommand{\dotp}{\bullet}
\newcommand{\cross}{\boldsymbol\times}
\newcommand{\grad}{\boldsymbol\nabla}
\newcommand{\divergence}{\grad\dotp}
\newcommand{\curl}{\grad\cross}
%% Simple horiz vectors
\renewcommand{\vector}[1]{\left\langle #1\right\rangle}


\outcome{Find the extreme values of a function.}

\title{3.2 Extreme Value Theorem}

%\newcommand{\ffrac}[2]{\frac{\mbox{\footnotesize $#1$}}{\mbox{\footnotesize $#2$}}}
%\newcommand{\vasymptote}[2][]{
 %   \draw [densely dashed,#1] ({rel axis cs:0,0} -| {axis cs:#2,0}) -- ({rel axis cs:0,1} -| {axis cs:#2,0});
%}


\begin{document}

\begin{abstract}
In this section we learn the Extreme Value Theorem and we find the extremes of a function.
\end{abstract}

\maketitle


\section{The Extreme Value Theorem}
In this section we will solve the problem of finding the maximum and minimum values of a 
continuous function on a closed interval.

\begin{theorem}[Extreme Value Theorem]
If $f(x)$ is continuous on the closed interval $[a,b]$, then $f(x)$ has both an absolute maximum and an absolute minimum 
on the interval.
\end{theorem}

It is important to note that the theorem contains two hypothesis.  
The first is that $f(x)$ is continuous and the second is that the interval is closed.  
If either of these conditions fails to hold, then $f(x)$ {\it might} fail to have either an 
absolute max or an absolute min (or both). It is also important to note that the theorem 
tells us that the max and the min occur in the interval, but it does not tell us how to find them.  

\begin{center}
\begin{tikzpicture}
\begin{axis}[ axis x line = bottom,  axis y line = left, xtick = {1,2}, ytick = {1,2},title={No absolute extremes on the open interval $(0,2)$}]
%\addplot[domain=0:0.98, color=blue]{x+1};
%\addplot[smooth,mark=*,blue] plot coordinates {(1,1.5)};
\addplot[domain=0.02:1.99,
    samples=100, color=blue, thick]{2^(x-1)};
\addplot[domain=0:0.1,
    samples=100, color=white, thick]{x+ 0.3};
\addplot[smooth,mark=o,blue, thick] coordinates {(0.0,0.5)};
\addplot[smooth,mark=o,blue, thick] coordinates {(2,2)};
\node[color=blue] at (axis cs: 1.5,1){$y = f(x)$};
\end{axis}
\end{tikzpicture}
\hspace{0.5in}
\begin{tikzpicture}
\begin{axis}[axis lines = center,xtick = {1,2}, ytick = {1,2}, title={No absolute extremes on $[0,2]$ due to discontinuity}]
\addplot[domain=0:.99, color=blue, thick]{2^x};
\addplot[smooth,mark=*,blue] plot coordinates {(0,1)};
\addplot[smooth,mark=o,blue, thick] plot coordinates {(1,2)};
\addplot[domain=1.02:2,
    color=blue, thick]{x-1};
\addplot[smooth,mark=o,blue, thick] plot coordinates {(1,0)};
\addplot[smooth,mark=*,blue] plot coordinates {(2,1)};
\addplot[smooth,mark=*,blue] plot coordinates {(1,1)};
\addplot[domain=2:2.02,
    color=white]{x-1};
\node[color=blue] at (axis cs: 1.3,1.3){$y = f(x)$};

\end{axis}
\end{tikzpicture}
\end{center}

The main idea is finding the location of the absolute max and absolute min of a continuous function on a closed 
interval is contained in the following theorem.
\begin{theorem}[Fermat's Theorem]
If $f(x)$ is defined on the open interval $(a,b)$ and that $f(x)$ has an absolute maximum or absolute minimum at 
a point $x = c$ in thie interval. Then $c$ is a critical number for $f(x)$.
\end{theorem}
As a result of Fermat's Theorem, we can conclude that the absolute extremes of
a continuous function on a closed interval must occur at the endpoints of the interval or at a critical numbers 
inside the interval.\\

\section{Closed Interval Method}
Finding the absolute extremes of a
continuous function, $f(x)$, on a closed interval $[a,b]$ is a three 
step process.
\begin{enumerate}
\item[1.] Find the critical numbers of $f(x)$ inside the interval $[a, b]$.
\item[2.] Compute the values of $f(x)$ at the critical numbers and 
at the endpoints.
\item[3.] The largest of the values from step 2 is the absolute 
maximum of $f(x)$ on the closed interval $[a,b]$, and the smallest 
of these values is the absolute minimum.
\end{enumerate}
%The largest of these function values is the absolute max and the smallest is the absolute min of $f(x)$ on $[a,b]$. 


\begin{example}[example 1] Find the absolute maximum and the absolute minimum values of $f(x) = x^2 - 4x + 5$ on the closed interval $[0, 3]$.

Solution:  The function is a polynomial, so it is continuous, and the interval is closed, 
so by the Extreme Value Theorem, we know that this function has an absolute maximum and an absolute minimum on the interval $[0,3]$.
First, we find the critical numbers of $f(x)$ in the interval $(0, 3)$. 
The derivative is 
\[
f'(x) = 2x - 4
\]
which exists for all values of $x$.
Solving the equation $f'(x) =0$ gives $x=2$ as the only critical number of the function.
Moreover, this critical number is in the interval $(0,3)$.
By the closed interval method, we know that the absolute extremes occur at either the endpoints, $x=0$ and $x = 3$, or the critical number $x = 2$.  
Plugging these values into the original function $f(x)$ yields:
\begin{eqnarray*}
f(0) &=& 0^2 - 4(0) + 5 =5\\
f(2) &=& 2^2 - 4(2) + 5 = 1\\
f(3) &=& 3^2 - 4(3) + 5 = 2.
\end{eqnarray*}
The largest of these values is 5, corresponding to $f(0)$, and the smallest is 1, corresponding to $f(2)$. 
Thus, we conclude that the absolute maximum of $f(x)$ on the closed interval $[0,3]$ is $5$ occurring 
at the left endpoint $x = 0$ and the absolute minimum of $f(x)$ in the interval is $1$
occurring at the critical number $x = 2$.
\begin{image}
\begin{tikzpicture}
\begin{axis}[ axis x line = bottom,  axis y line = left, xtick = {1,2, 3}, ytick={1, 2, 3, 4, 5, 6, 7}, title={Absolute min at $x=2$ and absolute max at $x = 0$}]
%\addplot[domain=0:0.98, color=blue]{x+1};
%\addplot[smooth,mark=*,blue] plot coordinates {(1,1.5)};
\addplot[domain=0:3,
    samples=100, color=blue, thick]{x^2 - 4*x + 5};
\addplot[smooth,mark=*,blue] coordinates {(0,5)};
\addplot[smooth,mark=*,blue] coordinates {(3,2)};
\addplot[smooth,mark=*,blue] coordinates {(2,1)};
\addplot[domain=0:1,
    samples=100, color=white, thick]{x};
\node[color=blue] at (axis cs: 2,2.5){$y = x^2 - 4x + 5$ on $[0,3]$};
\end{axis}
\end{tikzpicture}
\end{image}
\end{example}



\begin{problem}(problem 1)
Find the absolute maximum and the absolute minimum of $f(x) = x^2 - 4x + 2$
on the closed interval $[0, 3]$.

The absolute maximum is $\answer{2}$ and it occurs at $x = \answer{0}.$\\
The absolute minimum is $\answer{-2}$ and it occurs at $x = \answer{2}.$ 

\end{problem}



\begin{example}[example 2]
 Find the absolute max and the absolute min of $f(x) = x^3 - 6x^2 + 5$ on the closed interval $[\text{-}1, 6]$.

Solution:  First, we find the critical numbers of $f(x)$ in the interval $[\text{-}1, 6]$. 
The function is a polynomial, so it is differentiable everywhere.  
We solve the equation $f'(x) =0$.  This becomes $3x^2 -12x= 0$ which has two solutions $x=0$ and $x = 4$ (verify).
Hence $f(x)$ has two critical numbers in the interval.
The absolute extremes occur at either the endpoints, $x=\text{-}1, 6$ or the critical numbers $x = 0, 4$.  
Plugging these special values into the original function $f(x)$ yields:
\begin{eqnarray*}
f(\text{-}1) &=& (\text{-}1)^3 -6(\text{-}1)^2 + 5 = \text{-}2\\
f(0) &=& 0^3 - 6(0)^2 + 5 =5\\
f(4) &=& 4^3 - 6(4)^2 + 5 = \text{-}27\\
f(6) &=& 6^3 - 6(6)^2 + 5 = 5.
\end{eqnarray*}
From this data we conclude that the absolute maximum of $f(x)$ on the interval is $5$ occurring 
at both the critical number $x = 0$ and the right endpoint $x = 6$ and 
the absolute minimum of $f(x)$ in the interval is $-27$
occurring at the critical number $x = 4$.
\begin{image}
\begin{tikzpicture}
\begin{axis}[ axis x line = center,  axis y line = center, xtick = {-1,1,2, 3, 4, 5, 6}, ytick={-27, 5}, title={Absolute min at $x=4$ and absolute max at $x = 0$ and $x = 6$}]
%\addplot[domain=0:0.98, color=blue]{x+1};
%\addplot[smooth,mark=*,blue] plot coordinates {(1,1.5)};
\addplot[domain=-1:6,
    samples=100, color=blue, thick]{x^3 - 6*x^2 + 5};
\addplot[smooth,mark=*,blue] coordinates {(-1,-2)};
\addplot[smooth,mark=*,blue] coordinates {(0,5)};
\addplot[smooth,mark=*,blue] coordinates {(4,-27)};
\addplot[smooth,mark=*,blue] coordinates {(6,5)};
%\addplot[domain=0:1,    samples=100, color=white, thick]{x};
\node[color=blue] at (axis cs: 3.3,2.5){$y = x^3 - 6x^2 + 5$ on $[-1,6]$};
\addplot[domain=1:6,
    samples=100, color=white, thick]{x+6};
\end{axis}
\end{tikzpicture}
\end{image}
\end{example}



\begin{problem}(problem 2a)
Find the absolute maximum and the absolute minimum of $f(x) = x^3 - 12x + 1$
on the closed interval $[-3, 3]$.

The absolute maximum is $\answer{17}$ and it occurs at $x = \answer{-2}.$\\
The absolute minimum is $\answer{-15}$ and it occurs at $x = \answer{2}.$ 
\end{problem}

\begin{problem}(problem 2b)
Find the absolute maximum and the absolute minimum of $f(x) = x^3 + 3x^2$
on the closed interval $[-3, 1]$.

(If the max/min occurs in more than one place, list them in 
ascending order).\\
The absolute maximum is $\answer{4}$ and it occurs at $x = \answer{-2}$
and $x=\answer{1}.$
\\
The absolute minimum is $\answer{0}$ and it occurs at $x = \answer{-3}$
and $x=\answer{0}.$ 
\end{problem}


\begin{example}[example 3] Find the absolute max and the absolute min of $f(x) = \frac14 x^4 - 2x^3 + 4x^2 - 3$ on the 
closed interval $[\text{-}1, 3]$.


Solution:  First, we find the critical numbers of $f(x)$ in the interval $[\text{-}1, 3]$. 
The function is a polynomial, so it is differentiable everywhere.  
We solve the equation $f'(x) =0$.  This becomes $x^3 -6x^2 + 8x = 0$ and the solutions are $x=0, x=2$ and $x=4$ (verify). 
Noting that $x = 4$ is not in the interval $[\text{-}1,3]$ we see that 
$f(x)$ has two critical numbers in the interval, namely $x = 0$ and $x = 2$.
The absolute extremes occur at either the endpoints, $x=\text{-}1, 3$ or the critical numbers $x = 0,2$.  
Plugging these special values into the original function $f(x)$ yields:
\begin{eqnarray*}
f(\text{-}1) &=& \frac14(\text{-}1)^4 -2(\text{-}1)^3 + 4(\text{-}1)^2 - 3 = \frac14 + 2 + 4 - 3 = \frac{13}{4}\\
f(0) &=& \frac14 (0)^4 -2(0)^3 + 4(0)^2 - 3 = \text{-}3\\
f(2) &=& \frac14(2)^4 -2(2)^3 + 4(2)^2 - 3 = 4 -16 + 16 -3 = 1\\
f(3) &=& \frac14(3)^4 -2(3)^3 + 4(3)^2 - 3 = \frac{81}{4} - 54 + 36 - 3 = \text{-}\frac34.
\end{eqnarray*}


From this data we conclude that the absolute maximum of $f(x)$ on the interval is $3.25$ occurring 
at the endpoint $x = -1$ and the absolute minimum of $f(x)$ in the interval is $-3$
occurring at the critical number $x = 0$.
\begin{image}
\begin{tikzpicture}
\begin{axis}[ axis x line = center,  axis y line = center, xtick = {-1,2, 3}, ytick={-3,13/4}, title={Absolute min at $x=0$ and absolute max at $x = -1$}]
%\addplot[domain=0:0.98, color=blue]{x+1};
%\addplot[smooth,mark=*,blue] plot coordinates {(1,1.5)};
\addplot[domain=-1:3,
    samples=100, color=blue, thick]{0.25*x^4 - 2*x^3 + 4*x^2 - 3};
\addplot[smooth,mark=*,blue] coordinates {(-1,3.25)};
\addplot[smooth,mark=*,blue] coordinates {(0,-3)};
\addplot[smooth,mark=*,blue] coordinates {(2,1)};
\addplot[smooth,mark=*,blue] coordinates {(3,-0.75)};
%\addplot[domain=0:1,    samples=100, color=white, thick]{x};
\node[color=blue] at (axis cs: 1.3,2){$y=\frac14 x^4 - 2x^3 + 4x^2 - 3$ on $[-1,3]$};
%\addplot[domain=1:3,    samples=100, color=white, thick]{x+6};
\end{axis}
\end{tikzpicture}
\end{image}
\end{example}



\begin{problem}(problem 3)
Find the absolute maximum and the absolute minimum of $f(x) = x^4 - 4x^3 - 8x^2 - 16$  
on the closed interval $[-2, 1]$.

The absolute maximum is $\answer{0}$ and it occurs at $x = \answer{-2}.$
\\
The absolute minimum is $\answer{-27}$ and it occurs at $x = \answer{1}.$
\end{problem}



\begin{example}[example 4]
Find the absolute max and the absolute min of $f(x) = xe^{3x}$ on the closed interval $[-1, 0]$.

Solution:  First, we find the critical numbers of $f(x)$ in the 
interval $[-1, 0]$.   
We solve the equation $f'(x) =0$.  
Using the product rule and the chain rule, we have $f'(x) = 1\cdot e^{3x} + x\cdot e^{3x} \cdot 3$ which simplifies to $(3x+1)e^{3x}$. 
Thus we need to solve $(3x+1)e^{3x} = 0$ (verify) and the only solution is $x=\text{-}1/3$ (verify).
Hence $f(x)$ has one critical number in the interval and it occurs at $x = -1/3$.
The absolute extremes occur at either the endpoints, $x=-1, 2$ or the critical number $x = -1/3$.  
Plugging these special values into the original function $f(x)$ yields:
\begin{eqnarray*}
f(-1) &=& (-1)e^{-3} = -\frac{1}{e^3} \approx -0.0498\\
f(-1/3) &=& (-1/3)e^{-1} = -\frac{1}{3e} \approx -0.1226\\
f(0) &=& 0.
\end{eqnarray*}
From this data we conclude that the absolute maximum of $f(x)$ on the interval is $0$ occurring 
at the right endpoint $x = 0$ and the absolute minimum of $f(x)$ in the interval is $-\frac{1}{3e}$
occurring at the critical number $x = -1/3$.
\begin{image}
\begin{tikzpicture}
\begin{axis}[ axis x line = center,  axis y line = center, xtick = {-1,-1/3}, ytick={-0.0498,-0.1226},  
xticklabels = {-1, $-\frac13$},yticklabels = {}, 
title={Absolute min at $x=0$ and absolute max at $x = -1$}]
%\addplot[domain=0:0.98, color=blue]{x+1};
%\addplot[smooth,mark=*,blue] plot coordinates {(1,1.5)};
\addplot[domain=-1:0,
    samples=100, color=blue, thick]{x*e^(3*x)};
\addplot[smooth,mark=*,blue] coordinates {(-1,-0.0498)};
\addplot[smooth,mark=*,blue] coordinates {(-1/3,-0.1226)};
\addplot[smooth,mark=*,blue] coordinates {(0,0)};


\node[color=blue] at (axis cs: -0.5,-0.05){$y=xe^{3x}$ on $[-1,0]$};
\node at (axis cs: 0.1,-0.0498){$-\frac{1}{e^3}$};
\node at (axis cs: 0.1,-0.1226){$-\frac{1}{3e}$};
\addplot[domain=-1:0.4,
    samples=100, color=white, thick]{-0.15};
\addplot[domain=-1:0.4,
    samples=100, color=white, thick]{0.05};
    
\end{axis}
\end{tikzpicture}
\end{image}
\end{example}




\begin{problem}(problem 4a)
Find the absolute maximum and the absolute minimum of $f(x) = xe^{2x}$
on the closed interval $[-3, 2]$.
\begin{hint}
$f'(x) = (2x+1)e^{2x}$
\end{hint}

The absolute maximum is $\answer{2e^4}$ and it occurs at $x = \answer{2}.$\\
The absolute minimum is $\answer{-1/(2e)}$ and it occurs at $x = \answer{-1/2}.$ 
\end{problem}

\begin{problem}(problem 4b)
Find the absolute maximum and the absolute minimum of $f(x) = \frac{3x}{x^2+4}$
on the closed interval $[0, 4]$.

\begin{hint}
use the quotient rule to find $f'(x)$
\end{hint}
\begin{hint}
$f'(x) = \frac{12-3x^2}{(x^2+4)^2}$
\end{hint}

The absolute maximum is $\answer{3/4}$ and it occurs at $x = \answer{2}.$\\
The absolute minimum is $\answer{0}$ and it occurs at $x = \answer{0}.$\\
Note that the critical number $x= -2$ is not in the interval $[0, 4]$. 
\end{problem}


\begin{center}
\begin{foldable}
\unfoldable{Here is a detailed, lecture style video on the Extreme Value Theorem:}
\youtube{tBRx9Q02wxs}
\end{foldable}
\end{center}


\section{Proof of Fermat's Theorem}
Suppose that $f(x)$ is defined on the open interval $(a,b)$ and that $f(x)$ has an absolute max at $x=c$.
Thus $f(c) \geq f(x)$ for all $x$ in $(a,b)$. If $f'(c)$ is undefined then, $x=c$ is a critical number for $f(x)$.
If $f'(c)$ is defined, then from the definition of the derivative we have
\[
f'(c) = \lim_{h \to 0^-} \frac{f(c+h) -f(c)}{h} = \lim_{h \to 0^-} \frac{f(c+h) -f(c)}{h}
\]
The difference quotient in the left hand limit is positive (or zero) since the numerator is negative (or zero) and the denominator is negative.
Thus $f'(c) \geq 0$.  But the difference quotient in the right hand limit is negative (or zero) and so $f'(c) \leq 0$.
Hence $f'(c) = 0$ and the theorem is proved.

\end{document}















\section{Critical Numbers}

Critical numbers come in two main types, and keep in mind when considering the following definition 
that either type could describe a local extreme.  Here is the definition of {\bf critical number}.  

\begin{tcolorbox}
\begin{center}
{\bf Definition of Critical Number}
\end{center}
A critical number for the function $f(x)$ is a number 
$x_0$ in the domain of the function $f(x)$ (i.e. $f(x_0)$ is defined) such that either

\[f'(x_0) = 0\]

%\center{or}

%\[f'(x_0) \;\; \textbox{is undefined}.\]
\end{tcolorbox}


In other words, at a critical number $x_0$ we have  $f(x_0)$ exists and either 
$f'(x_0) = 0$ (think "horizontal tangent line") or $f(x)$ is not differentiable at $x_0$ (think "corner point").

An example of the first type is $x_0 = 0$ for the function $f(x) = x^2$ since $f(0)$ is defined (it equals 0)
and $f'(0) = 0$ since $f'(x) = 2x$.

An example of the second type is $x_0 = 0$ for the function $f(x) = |x|$ since $f(0)$ is defined (it equals 0)
and $f'(0)$ is undefined since $f(x) = |x|$ has a corner point at $x = 0$ and hence it is not differentiable there.

\begin{description}

\item[CN 1] Find the critical numbers of the function $f(x) = \dfrac{x^3}{3} - \dfrac{x^2}{2} - 6x - 3$.
We need to compute $f'(x)$.  We have
\[f'(x) = x^2 - x - 6.\]
Noting that $f'(x)$ exists for all values of $x$, there are no critical numbers of the second type.
To find critical numbers of the first type, we solve the equation
\[f'(x) = 0\]
for $x$. Geometrically, these are the points where the graph of $f(x)$ has horizontal tangent lines.
We get
\[ x^2 - x - 6 =0\]
\[ (x+2)(x-3) =0\]
\[x = -2 \;\;\textbox{or} \;\; x = 3.\]


\item[CN 2] Find the critical numbers of the function $f(x) = x^2e^{3x}$.
We need to compute $f'(x)$ using the product and chain rules.  We have
\[f'(x) = (3x^2 +2x)e^{3x} \;\; \textbox{verify}.\]
Noting that $f'(x)$ exists for all values of $x$, there are no critical numbers of the second type.
To find critical numbers of the first type, we solve the equation
\[f'(x) = 0\]
for $x$. Geometrically, these are the points where the graph of $f(x)$ has horizontal tangent lines.
We get
\[ (3x^2 +2x)e^{3x} =0\]
\[ x(3x+2)e^{3x} =0\]
\[x = 0 \;\;\textbox{or} \;\; x = -2/3.\]
Note that the equation $e^{3x} = 0$ has no solutions since an exponential function is always positive.


\item[CN 3] Find the critical numbers of the function $f(x) = \dfrac{x}{x^2 +1}$.
We need to compute $f'(x)$ using the quotient rule.  We have
\[f'(x) = \frac{1-x^2}{(x^2+1)^2} \;\; \textbox{verify}.\]
Noting that $f'(x)$ exists for all values of $x$ (since the denominator is never equal to 0), 
there are no critical numbers of the second type.
To find critical numbers of the first type, we solve the equation
\[f'(x) = 0\]
for $x$. Geometrically, these are the points where the graph of $f(x)$ has horizontal tangent lines.
We get
\[ \frac{1-x^2}{(x^2+1)^2} =0\]
\[ 1-x^2 =0\]
\[x = \pm 1.\]

Hence $f(x) = \dfrac{x}{x^2 +1}$ has two critical numbers, and they are both of the first type. 
They are $-1$ and $1$.


\item[CN 4] Find the critical numbers of the function $f(x) = \sqrt[3] x}$.
We need to compute $f'(x)$.  We have
\[f'(x) = \frac{1}{3\sqrt x^3} \;\; \textbox{verify}.\]
In this case, $f'(0)$ is undefined (division by zero). Hence $x=0$ is a critical number {\bf if}
$f(0)$ is defined.  We can easily check this: $f(0) = \sqrt[3] 0 = 0$, so it is defined and now we can conclude 
that $x=0$ is a critical number.
To find critical numbers of the first type, we solve the equation
\[f'(x) = 0\]
for $x$. Geometrically, these are the points where the graph of $f(x)$ has horizontal tangent lines.
We get
\[ \frac{1}{3\sqrt x^3 =0\]
\[ 1 =0.\]
So there are no solutions.  The function has no critical number of the first type.


Hence $f(x) = \sqrt[3] x$ has only one critical number, 0, and it is of the second type, 
where the function is not differentiable. 
Geometrically, the function $f(x) = \sqrt[3] x$ has a vertical tangent line at the critical number $x = 0$.




In this section, we use the derivative to determine intervals on which a given function is increasing or decreasing.  
We can then also determine the location of local extremes of the function.






A function $f(x)$ is called {\bf increasing} on an interval $I$ if given any two numbers, $x_1$ and $x_2$ in $I$ such that 
$x_1 < x_2$, we have $f(x_1) < f(x_2)$.  For example, the function $x^2$ is increasing on the interval $(0, \infty)$ since if $0 < x_1 < x_2$ then $(x_1)^2 < (x_2)^2$.

Similarly, a function $f(x)$ is called {\bf decreasing} on an interval $I$ if given any two numbers, $x_1$ and $x_2$ in $I$ such that 
$x_1 < x_2$, we have $f(x_1) > f(x_2)$.  For example, the function $x^2$ is increasing on the interval $(-\infty, 0)$ since if $ x_1 < x_2 < 0$ then $(x_1)^2 > (x_2)^2$.

Theorem:  If $f'(x) > 0$ on an interval $I$, then $f(x)$ is increasing on that interval.
Proof: Let $x_1$ and $x_2$ be in the interval $I$ with $x_1 < x_2$.  Then since $f$ is differentiable on the closed interval 
$[x_1, x_2]$, it is also continuous there.  Hence we can apply the MVT on to $f(x)$ on $[x_1, x_2]$ and conclude that
\[f(x_2) - f(x_1) = f'(c) (x_2 - x_1).\]
Both factors on the right hand side are positive, hence $f(x_2) - f(x_1)$ is positive and $f(x_2)> f(x_1)$.  Hence $x_1 < x_2$implies $f(x_1)< f(x_2)$ which means that $f(x)$ is increasing on $I$.

Theorem: If $f'(x) < 0$ on an interval $I$, then $f(x)$ is decreasing on that interval.
The proof is similar to the other case, above.

\begin{description}

\item[ID 1]  Determine intervals on which $f(x) = \dfrac{x^3}{3} - \dfrac{x^2}{2} - 6x + 2$ is increasing and decreasing.
To determine where $f'$ is either positive or negative, it is usually easier to consider places where $f'(x) = 0$ or 
$f'(x)$ is undefined. In this example, 
\[f'(x) = x^2 - x -6\] 
which exists for all $x$. We solve the equation
\[f'(x) = x^2 - x -6 = 0\] 
which yields 
\[(x+2)(x-3) = 0\] 
and hence, $x=-2$ or $x=3$.
These two $x$-values break the real number line into three open intervals: $(-\infty, -2), (-2, 3)$ and $(3, \infty)$.
On each of these intervals, $f'$ will be either strictly positive or strictly negative.  
To determine which, we will use test points. For the interval, $(-\infty, -2)$ consider $f'(-3) = (-3)^2 - (-3) -6 = 6 >0$.
This means that $f'(x) > 0$ for every $x$ value in the interval $(-\infty, -2)$ and by the theorem, 
$f(x)$ is increasing on this interval.

Next, for the interval $(-2, 3)$, consider $f'(0) = (0)^2 - (0) -6 = -6 <0$. 
This means that $f'(x) < 0$ for every $x$ value in the interval $(-2, 3)$ and by the theorem, 
$f(x)$ is decreasing on this interval.


Finally, for the interval $(3, \infty)$, consider $f'(4) = (4)^2 - (4) -6 = 10>0$. 
This means that $f'(x) > 0$ for every $x$ value in the interval $(3, \infty)$ and by the theorem, 
$f(x)$ is increasing on this interval.

\end{description}

The work that was done in the previous example can actually give us slightly more information about $f(x)$.  We can determine the {\bf local extremes} of $f(x)$.

Definition:  We say that $f(x)$ has a {\bf local maximum} at $x = x_0$ if there is an open interval $I$ containing $x_0$ 
such that $f(x) \leq f(x_0)$ for all values of $x$ in $I$.  Similarly, we say that $f(x)$ has a 
{\bf local minimum} at $x = x_0$ if there is an open interval $I$ containing $x_0$ 
such that $f(x) \geq f(x_0)$ for all values of $x$ in $I$.

First Derivative Test for Local Extremes:  If $f'(x)$ changes sign at a critical number $x_0$, then $f(x)$ has a local 
extreme at $x=x_0$.  More specifically, if the sign changes from positive to negative then the local extreme is a max and if the sign changes from negative to positive, then the local extreme is a min.

\begin{description}

\item[ID 2]
In the previous example, we determined that $-2$ and $3$ were critical numbers for the function 
$f(x) = \dfrac{x^3}{3} - \dfrac{x^2}{2} - 6x + 2$ and that $f(x)$
was increasing on the interval $(-\infty, -2)$, decreasing on $(-2, 3)$ and increasing on $(3, \infty)$.
By the First Derivative Test, we can conclude that $f(x)$ has a local maximum at $x = -2$ and a local minimum at $x = 3$.

\item[ID 3]  $f(x) = xe^{-2x}$.

\end{description}
