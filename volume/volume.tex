\documentclass{ximera}

%% You can put user macros here
%% However, you cannot make new environments



\newcommand{\ffrac}[2]{\frac{\text{\footnotesize $#1$}}{\text{\footnotesize $#2$}}}
\newcommand{\vasymptote}[2][]{
    \draw [densely dashed,#1] ({rel axis cs:0,0} -| {axis cs:#2,0}) -- ({rel axis cs:0,1} -| {axis cs:#2,0});
}


\graphicspath{{./}{firstExample/}}

\usepackage{amsmath}
\usepackage{amssymb}
\usepackage{array}
\usepackage[makeroom]{cancel} %% for strike outs
\usepackage{pgffor} %% required for integral for loops
\usepackage{tikz}
\usepackage{tikz-cd}
\usepackage{tkz-euclide}
\usetikzlibrary{shapes.multipart}


\usetkzobj{all}
\tikzstyle geometryDiagrams=[ultra thick,color=blue!50!black]


\usetikzlibrary{arrows}
\tikzset{>=stealth,commutative diagrams/.cd,
  arrow style=tikz,diagrams={>=stealth}} %% cool arrow head
\tikzset{shorten <>/.style={ shorten >=#1, shorten <=#1 } } %% allows shorter vectors

\usetikzlibrary{backgrounds} %% for boxes around graphs
\usetikzlibrary{shapes,positioning}  %% Clouds and stars
\usetikzlibrary{matrix} %% for matrix
\usepgfplotslibrary{polar} %% for polar plots
\usepgfplotslibrary{fillbetween} %% to shade area between curves in TikZ



%\usepackage[width=4.375in, height=7.0in, top=1.0in, papersize={5.5in,8.5in}]{geometry}
%\usepackage[pdftex]{graphicx}
%\usepackage{tipa}
%\usepackage{txfonts}
%\usepackage{textcomp}
%\usepackage{amsthm}
%\usepackage{xy}
%\usepackage{fancyhdr}
%\usepackage{xcolor}
%\usepackage{mathtools} %% for pretty underbrace % Breaks Ximera
%\usepackage{multicol}



\newcommand{\RR}{\mathbb R}
\newcommand{\R}{\mathbb R}
\newcommand{\C}{\mathbb C}
\newcommand{\N}{\mathbb N}
\newcommand{\Z}{\mathbb Z}
\newcommand{\dis}{\displaystyle}
%\renewcommand{\d}{\,d\!}
\renewcommand{\d}{\mathop{}\!d}
\newcommand{\dd}[2][]{\frac{\d #1}{\d #2}}
\newcommand{\pp}[2][]{\frac{\partial #1}{\partial #2}}
\renewcommand{\l}{\ell}
\newcommand{\ddx}{\frac{d}{\d x}}

\newcommand{\zeroOverZero}{\ensuremath{\boldsymbol{\tfrac{0}{0}}}}
\newcommand{\inftyOverInfty}{\ensuremath{\boldsymbol{\tfrac{\infty}{\infty}}}}
\newcommand{\zeroOverInfty}{\ensuremath{\boldsymbol{\tfrac{0}{\infty}}}}
\newcommand{\zeroTimesInfty}{\ensuremath{\small\boldsymbol{0\cdot \infty}}}
\newcommand{\inftyMinusInfty}{\ensuremath{\small\boldsymbol{\infty - \infty}}}
\newcommand{\oneToInfty}{\ensuremath{\boldsymbol{1^\infty}}}
\newcommand{\zeroToZero}{\ensuremath{\boldsymbol{0^0}}}
\newcommand{\inftyToZero}{\ensuremath{\boldsymbol{\infty^0}}}


\newcommand{\numOverZero}{\ensuremath{\boldsymbol{\tfrac{\#}{0}}}}
\newcommand{\dfn}{\textbf}
%\newcommand{\unit}{\,\mathrm}
\newcommand{\unit}{\mathop{}\!\mathrm}
%\newcommand{\eval}[1]{\bigg[ #1 \bigg]}
\newcommand{\eval}[1]{ #1 \bigg|}
\newcommand{\seq}[1]{\left( #1 \right)}
\renewcommand{\epsilon}{\varepsilon}
\renewcommand{\iff}{\Leftrightarrow}

\DeclareMathOperator{\arccot}{arccot}
\DeclareMathOperator{\arcsec}{arcsec}
\DeclareMathOperator{\arccsc}{arccsc}
\DeclareMathOperator{\si}{Si}
\DeclareMathOperator{\proj}{proj}
\DeclareMathOperator{\scal}{scal}
\DeclareMathOperator{\cis}{cis}
\DeclareMathOperator{\Arg}{Arg}
%\DeclareMathOperator{\arg}{arg}
\DeclareMathOperator{\Rep}{Re}
\DeclareMathOperator{\Imp}{Im}
\DeclareMathOperator{\sech}{sech}
\DeclareMathOperator{\csch}{csch}
\DeclareMathOperator{\Log}{Log}

\newcommand{\tightoverset}[2]{% for arrow vec
  \mathop{#2}\limits^{\vbox to -.5ex{\kern-0.75ex\hbox{$#1$}\vss}}}
\newcommand{\arrowvec}{\overrightarrow}
\renewcommand{\vec}{\mathbf}
\newcommand{\veci}{{\boldsymbol{\hat{\imath}}}}
\newcommand{\vecj}{{\boldsymbol{\hat{\jmath}}}}
\newcommand{\veck}{{\boldsymbol{\hat{k}}}}
\newcommand{\vecl}{\boldsymbol{\l}}
\newcommand{\utan}{\vec{\hat{t}}}
\newcommand{\unormal}{\vec{\hat{n}}}
\newcommand{\ubinormal}{\vec{\hat{b}}}

\newcommand{\dotp}{\bullet}
\newcommand{\cross}{\boldsymbol\times}
\newcommand{\grad}{\boldsymbol\nabla}
\newcommand{\divergence}{\grad\dotp}
\newcommand{\curl}{\grad\cross}
%% Simple horiz vectors
\renewcommand{\vector}[1]{\left\langle #1\right\rangle}


\outcome{Find the volume of a solid}

\title{Volume of Solids}

\begin{document}

\begin{abstract}
We use cross-sections to find the volume of a solid.
\end{abstract}

\maketitle


We have learned that the integral can be used to compute areas of regions in the plane.  
We will now learn that integrals can be used to compute volumes as well.
The integral notation suggests that we compute area in the plane by summing the areas of  infinitely thin line segments.
Of course this perspective is only the result of a limit of Riemann Sums of areas of rectangles.
The volume of a solid can be similarly conceptualized as resulting from pasting together infinitely many cross-sections,
each of which possesses area. Hence the integral can be used in a similar fashion to compute the volume of a solid.
In the case of volume, the defining Riemann Sums involve irregularly formed cylinders rather than rectangles
in our approximation process.

%replacing the line segments which when pasted together form a region,
%with cross-sectional areas which when themselves pasted together form a solid.

%visual aid here

Suppose that a solid be placed atop an axis, say the $x-$axis, as shown in the figure below.

%visual aid here

For each $x$-value between $x = a$ and $x = b$,  there is a cross-section of the solid whose area is $A(x)$.
Then our astonishing result is that the volume of the solid can be computed as

\[
V = \int_a^b A(x) \; dx.
\]

\begin{remark}
We can also re-orient our axis so that it would be more aptly, named `$y$', and with respect to this vertical axis, 
we could compute the volume with the exact same idea as
\[
V = \int_a^b A(y) \; dy.
\]

\end{remark}

%visual aid here with y-axis


\begin{explanation}
To be precise, we will next define the volume in terms of the limit of a Riemann Sums.  
Let us reconsider the solid above, from the perspective of the $x$-axis. Suppose that at
each $x$-value there is a cross-sectional area $A(x)$. Then to approximate the volume of the solid we 
can use a Riemann sum as follows. Subdivide $[0,1]$ into `$n$' equal sub-intervals each of length $\Delta x = 1/n$.
From the $i^\text{th}$ sub-interval,$\left[\frac{i-1}{n}, \frac{i}{n}\right]$, we choose a sample point, $x_i^*$, where 
the index, $i$, runs from $1$ to $n$. 
Corresponding to these $n$ values, $x_i^*$, there are cross-sections whose areas are given by $A(x_i^*)$.
From these cross-sections we can construct an irregular cylinder as shown below.

INSERT FIGURE OF IRREGULAR CYLINDER HERE

The volume of each of the $i^\text{th}$ cylinder is given by
\[
V_i = \text{area of base} \times \text{height} = A(x_i^*) \cdot \frac{1}{n} = A(x_i^*) \cdot \Delta x.
\]
Summing the volumes of these cylinders gives an approximation to the volume as a Riemann Sum
\[
V \approx \sum_{i=1}^n V_i = \sum_{i=1}^n A(x_i^*) \Delta x.
\]
The exact volume is then defined in terms of a limit these Riemann Sums:
\[
V = \lim_{n \to \infty} \sum_{i=1}^n A(x_i^*) \Delta x = \int_0^1 A(x) \; dx.
\]

\end{explanation}

We generalize this explanation in the definition below.

\begin{definition}[Volume of a Solid]
Let $A(x)$ be the cross-sectional area of a solid at an $x$-value between $x = a$ and $x=b$. Then, the exact volume of the solid is given by
\[
V = \int_a^b A(x) \; dx = \lim_{n \to \infty} \sum_{i=1}^n A(x_i^*) \Delta x.
\]
In this context, 
\[
\Delta x = \frac{b-a}{n}
\] 
and $x_i^*$ is a sample point in the $i^\text{th}$ sub-interval
\[
[a + (i-1)\Delta x, a + i\Delta x],
\]
where $i = 1, 2, \dots, n$.
\end{definition}


\begin{example} Compute the volume of a pyramid of height $h$ with a square base of side length $b$.\\
We will place the $y$-axis through the pyramid so that the apex passes through the axis at $h$ and the center of the base passes through 
the axis at $0$.

INSERT FIGURE HERE

The cross-sections of the pyramid are squares and the area of a square of side $s$ is $A = s^2$.
The side length depends on $y$, so we will denote it by $s(y)$ to emphasize this fact. For example, $s(0) = b$ since at the height $0$, the square is 
the base of the pyramid which has side length $b$. Also $s(h) = 0$ since the apex (a single point) is at the height $h$ on the $y$-axis.
Also, $s(y) = 0$ if $y>h$ or $y<0$ since the pyramid only exists for $y$ values between $0$ and $h$.
From the definition of volume given above, we can express the volume of the pyramid by the definite integral 
\[
V = \int_0^h s(y)^2 \; dy.
\]
In order to compute this integral, we need to find an explicit formula for $s(y)$.
The method for finding $s(y)$ involves using \textbf{similar triangles}. We will create two triangles- 
the first is a vertical cross-section of the pyramid through the apex and has height $h$ (the height of the pyramid) 
and base $b$ (the base of the pyramid). 

INSERT FIGURE HERE

The second triangle is positioned inside the first in such a way that it has height $h-y$, extending from $y$ up to $h$ on the $y$-axis,
and base $s(y)$ measured through the $y$-axis perpendicular to the side of the square cross-section at height $y$.

INSERT FIGURE HERE

These two triangles are similar because they have the same angles. Because the triangles are similar, the ratio of the base to 
the height of each triangle is the same, 
i.e.,
\[
\frac{s(y)}{h-y} = {b}{h}.
\]
 Solving for $s(y)$ gives
 \[
 s(y) = \frac{(h-y)b}{h} = \frac{h-y}{h} \cdot b = \left(1-\frac{y}{h}\right)b.
 \]
 We can now compute the volume of the pyramid by computing the definite integral given above:
 \begin{align*}
 V &= \int_0^h s(y)^2 \; dy\\
   &= \int_0^h \left(1-\frac{y}{h}\right)^2b^2 \; dy
   &= b^2 \int_0^h \left(1 - \frac{2y}{h} + \frac{y^2}{h^2}\right)\; dy
   &= b^2 \left(y - \frac{y^2}{h} + \frac{y^3}{3h^2}\right)\bigg|_0^h\\
   &= b^2 \left(h - \frac{h^2}{h} + \frac{h^3}{3h^2}\right)\\
   &= b^2 \left(h - h + \frac{h}{3}\right)\\
   &= \frac13 b^2h.
\end{align*}

\begin{center}
\geogebra{v8nntygp}{1000}{800}
\end{center}


\end{example}
 
 
 


\begin{example}  Find the volume of the intersection of two cylinders of radius $r$ show in the figure.

INSERT FIGURE HERE

Place the $x$-axis through the center of one of the cylinders, with the center of the other cylinder at $x = 0$.
For values of $x$ in the interval $[-r, r]$, cross-sections of the region taken perpendicular 
to the $x$-axis are squares with side length $s(x)$ (see the interactive diagram). 

The volume of the region is then given by the definite integral
\[
V = \int_{-r}^r s(x)^2 \; dx.
\]

To find the side length of the square cross-section, $s(x)$, consider the circle of radius $r$ centered at the origin, $x^2 + y^2 = r^2$. For a given value of $x$, the side 
length $s(x)$ is the difference between the corresponding $y$-coordinates on this circle (see the figure below). Solving for $y$, we see that those coordinates are 
$\sqrt{r^2 - x^2}$ and $-\sqrt{r^2 - x^2}$ so that the difference is 
\[
s(x) = \sqrt{r^2 - x^2}-(-\sqrt{r^2 - x^2}) = 2\sqrt{r^2 - x^2}.
\]
The volume of the region  is then
\begin{align*}
V &= \int_{-r}^r s(x)^2 \; dx\\
  &= \int_{-r}^r \left(2\sqrt{r^2 - x^2}\right)^2 \; dx\\
  &= \int_{-r}^r 4(r^2 - x^2) \; dx\\
  &= \int_{-r}^r (4r^2 - 4x^2) \; dx\\
  &= \left(4r^2x - \frac43x^3\right)\bigg|_{-r}^r\\
  &= \left(4r^3 - \frac43r^3\right) - \left(-4r^3 + \frac43 r^3\right)\\
  &= 8r^3 - \frac83 r^3\\
  &= \frac{16}{3} r^3.
\end{align*}

\end{example}



\section{Video Lesson}

\begin{center}
\begin{foldable}
\unfoldable{Here is a detailed, lecture style video on volumes of solids using cross-sections:}
\youtube{gRJTe1i1muo}
\end{foldable}
\end{center}





\end{document}


\begin{example} %example #15
Find $h'(x)$ if $h(x) = x^{\sin(x)}$.\\
We will use the fact that the exponential and logarithm functions are inverses,
\[e^{\ln(x)} = x,\]
and the exponent property of logarithms, 
\[\ln(x^n) = n\ln(x),\]
to rewrite $h(x)$.  We have 
\[h(x) = x^{\sin(x)} = e^{\ln(x^{\sin(x)})} = e^{\sin(x)\ln(x)}\]
and we can now compute $h'(x)$ using a combination of the chain rule and product rule.
We can write $h(x)$ as a composition, $f(g(x))$ with 
\[g(x) = \sin(x)\ln(x) \quad \text{and} \quad f(x) = e^x.\]
Then to find $g'(x)$ we us the product rule and we get $g'(x) = \frac{\sin(x)}{x} + \cos(x)\ln(x)$.
Next $f'(x) = e^x$ and 
hence $f'(g(x)) = e^{g(x)} = e^{\sin(x)\ln(x)} = x^{\sin(x)}$.
We can then conclude $h'(x) = f'(g(x))g'(x) = x^{\sin(x)} \left[ \frac{\sin(x)}{x} + \cos(x)\ln(x)\right]$.
\end{example}

%more question formats below













%\begin{verbatim}
\begin{question}
What is your favorite color?
\begin{multipleChoice}
\choice[correct]{Rainbow}
\choice{Blue}
\choice{Green}
\choice{Red}
\end{multipleChoice}
\begin{freeResponse}
Hello
\end{freeResponse}
\end{question}
%\end{verbatim}





\begin{question}
  Which one will you choose?
  \begin{multipleChoice}
    \choice[correct]{I'm correct.}
    \choice{I'm wrong.}
    \choice{I'm wrong too.}
  \end{multipleChoice}
\end{question}


\begin{question}
  Which one will you choose?
  \begin{selectAll}
    \choice[correct]{I'm correct.}
    \choice{I'm wrong.}
    \choice[correct]{I'm also correct.}
    \choice{I'm wrong too.}
  \end{selectAll}
\end{question}


\begin{freeResponse}
What is the chain rule used for?
\end{freeResponse}
