\documentclass{ximera}

%% You can put user macros here
%% However, you cannot make new environments



\newcommand{\ffrac}[2]{\frac{\text{\footnotesize $#1$}}{\text{\footnotesize $#2$}}}
\newcommand{\vasymptote}[2][]{
    \draw [densely dashed,#1] ({rel axis cs:0,0} -| {axis cs:#2,0}) -- ({rel axis cs:0,1} -| {axis cs:#2,0});
}


\graphicspath{{./}{firstExample/}}

\usepackage{amsmath}
\usepackage{amssymb}
\usepackage{array}
\usepackage[makeroom]{cancel} %% for strike outs
\usepackage{pgffor} %% required for integral for loops
\usepackage{tikz}
\usepackage{tikz-cd}
\usepackage{tkz-euclide}
\usetikzlibrary{shapes.multipart}


\usetkzobj{all}
\tikzstyle geometryDiagrams=[ultra thick,color=blue!50!black]


\usetikzlibrary{arrows}
\tikzset{>=stealth,commutative diagrams/.cd,
  arrow style=tikz,diagrams={>=stealth}} %% cool arrow head
\tikzset{shorten <>/.style={ shorten >=#1, shorten <=#1 } } %% allows shorter vectors

\usetikzlibrary{backgrounds} %% for boxes around graphs
\usetikzlibrary{shapes,positioning}  %% Clouds and stars
\usetikzlibrary{matrix} %% for matrix
\usepgfplotslibrary{polar} %% for polar plots
\usepgfplotslibrary{fillbetween} %% to shade area between curves in TikZ



%\usepackage[width=4.375in, height=7.0in, top=1.0in, papersize={5.5in,8.5in}]{geometry}
%\usepackage[pdftex]{graphicx}
%\usepackage{tipa}
%\usepackage{txfonts}
%\usepackage{textcomp}
%\usepackage{amsthm}
%\usepackage{xy}
%\usepackage{fancyhdr}
%\usepackage{xcolor}
%\usepackage{mathtools} %% for pretty underbrace % Breaks Ximera
%\usepackage{multicol}



\newcommand{\RR}{\mathbb R}
\newcommand{\R}{\mathbb R}
\newcommand{\C}{\mathbb C}
\newcommand{\N}{\mathbb N}
\newcommand{\Z}{\mathbb Z}
\newcommand{\dis}{\displaystyle}
%\renewcommand{\d}{\,d\!}
\renewcommand{\d}{\mathop{}\!d}
\newcommand{\dd}[2][]{\frac{\d #1}{\d #2}}
\newcommand{\pp}[2][]{\frac{\partial #1}{\partial #2}}
\renewcommand{\l}{\ell}
\newcommand{\ddx}{\frac{d}{\d x}}

\newcommand{\zeroOverZero}{\ensuremath{\boldsymbol{\tfrac{0}{0}}}}
\newcommand{\inftyOverInfty}{\ensuremath{\boldsymbol{\tfrac{\infty}{\infty}}}}
\newcommand{\zeroOverInfty}{\ensuremath{\boldsymbol{\tfrac{0}{\infty}}}}
\newcommand{\zeroTimesInfty}{\ensuremath{\small\boldsymbol{0\cdot \infty}}}
\newcommand{\inftyMinusInfty}{\ensuremath{\small\boldsymbol{\infty - \infty}}}
\newcommand{\oneToInfty}{\ensuremath{\boldsymbol{1^\infty}}}
\newcommand{\zeroToZero}{\ensuremath{\boldsymbol{0^0}}}
\newcommand{\inftyToZero}{\ensuremath{\boldsymbol{\infty^0}}}


\newcommand{\numOverZero}{\ensuremath{\boldsymbol{\tfrac{\#}{0}}}}
\newcommand{\dfn}{\textbf}
%\newcommand{\unit}{\,\mathrm}
\newcommand{\unit}{\mathop{}\!\mathrm}
%\newcommand{\eval}[1]{\bigg[ #1 \bigg]}
\newcommand{\eval}[1]{ #1 \bigg|}
\newcommand{\seq}[1]{\left( #1 \right)}
\renewcommand{\epsilon}{\varepsilon}
\renewcommand{\iff}{\Leftrightarrow}

\DeclareMathOperator{\arccot}{arccot}
\DeclareMathOperator{\arcsec}{arcsec}
\DeclareMathOperator{\arccsc}{arccsc}
\DeclareMathOperator{\si}{Si}
\DeclareMathOperator{\proj}{proj}
\DeclareMathOperator{\scal}{scal}
\DeclareMathOperator{\cis}{cis}
\DeclareMathOperator{\Arg}{Arg}
%\DeclareMathOperator{\arg}{arg}
\DeclareMathOperator{\Rep}{Re}
\DeclareMathOperator{\Imp}{Im}
\DeclareMathOperator{\sech}{sech}
\DeclareMathOperator{\csch}{csch}
\DeclareMathOperator{\Log}{Log}

\newcommand{\tightoverset}[2]{% for arrow vec
  \mathop{#2}\limits^{\vbox to -.5ex{\kern-0.75ex\hbox{$#1$}\vss}}}
\newcommand{\arrowvec}{\overrightarrow}
\renewcommand{\vec}{\mathbf}
\newcommand{\veci}{{\boldsymbol{\hat{\imath}}}}
\newcommand{\vecj}{{\boldsymbol{\hat{\jmath}}}}
\newcommand{\veck}{{\boldsymbol{\hat{k}}}}
\newcommand{\vecl}{\boldsymbol{\l}}
\newcommand{\utan}{\vec{\hat{t}}}
\newcommand{\unormal}{\vec{\hat{n}}}
\newcommand{\ubinormal}{\vec{\hat{b}}}

\newcommand{\dotp}{\bullet}
\newcommand{\cross}{\boldsymbol\times}
\newcommand{\grad}{\boldsymbol\nabla}
\newcommand{\divergence}{\grad\dotp}
\newcommand{\curl}{\grad\cross}
%% Simple horiz vectors
\renewcommand{\vector}[1]{\left\langle #1\right\rangle}


\outcome{Compute the area between two curves}

\title{Area Between Curves}

\begin{document}

\begin{abstract}
We compute the area between two curves.
\end{abstract}

\maketitle

\section{Area Between Curves}

In this section use definite integrals to find the area of a region 
in the $xy$-plane bounded by two or more curves. Recall that if $f(x)$ is a positive, continuous function 
over the interval $[a, b]$, then the area bounded below the curve, above the $x$-axis, and between the lines $x = a$
and $x = b$ is given by the definite integral, $\int_a^b f(x) \, dx$. If we modify this slightly, 
we can find the area of a region between two continuous functions $f(x)$ and $g(x)$. First, suppose $f(x)$ and $g(x)$
are positive, continuous functions on the interval $[a,b]$ with $f(x) \geq g(x)$.
Then the area between the curves is the difference between the area under $f(x)$ and the area under $g(x)$.


\begin{center}
\begin{tikzpicture}
\begin{axis}[axis x line=  center, axis y line = none, xtick={-1, 1}, 
xticklabels={$a$, $b$}, title={\textup{Area Between Curves}}] 

\addplot[name path=AA, domain=-1:1, samples = 100, color=black, thick]{x^3-x+2};
\addplot[name path=BB, domain=-1:1, samples = 100, color=black, thick]{x^2 +.5};
\addplot[name path = E, domain=-1:-.5, samples=100, color=black]{0};
\addplot[name path = F, domain=-.5:0, samples=100, color=black]{0};
\addplot[name path = G, domain=0:.5, samples=100, color=black]{0};


\addplot[blue!10] fill between[of=AA and BB];

%\addplot[white] coordinates {(1.51,0) (1.8, 0)}; %white space to put the x in the x-axis

\addplot[thin, dashed] coordinates {(-1,0) (-1, 2)};
\addplot[thin, dashed] coordinates {(1,0) (1, 2)};

\addplot[<->] coordinates {(-1.5,-.5) (-1.5, 2)};
\addplot[<->] coordinates {(-1.7,0) (1.5, 0)};


\node at (axis cs: .5,2){$y = f(x)$};
\node at (axis cs: -.5,.4){$y = g(x)$};
\node at (axis cs: -1.6,1.9){$y$};
\node at (axis cs: 1.45,-.15){$x$};


\end{axis}
\end{tikzpicture}
\end{center}


Hence
\[
\text{Area} = \int_a^b f(x) \; dx - \int_a^b g(x) \; dx.
\]
We can simplify this formula slightly by combining these two integrals into a single integral, as in the theorem below. 


\begin{theorem}
Suppose $f(x)$ and $g(x)$ are continuous on the interval $[a,b]$. If $f(x) \geq g(x)$ for all $x$ in the interval $[a,b]$, 
then the area of the region bounded by the graphs of $y = f(x), y=g(x), x=a,$ and $x=b$ is given by  
\[
\text{Area} = \int_a^b \left[f(x) -g(x)\right] \; dx.
\]
\end{theorem}


\begin{remark}
There is no need to assume that $f(x)$ and $g(x)$ are positive in the theorem.  It suffices that $f(x) \geq g(x)$ on $[a,b]$ since the area of the region between 
$f(x) + k$ and $g(x) + k$ is the same for any constant $k$ and if $k$ is large enough then both $f(x) + k$ and $g(x) + k$ are positive
on the closed interval $[a,b]$.
\end{remark}

\begin{center}
\geogebra{uqcrerff}{1000}{750}
\end{center}

\begin{example} %example #1
Find the area of the region bounded by curves $y = 5 - x^2$ and $y = x-4$ over the interval $[1,2]$.
Since $5 - x^2 > x-4$ over the interval $[1,2]$, the area is
\begin{align*}
\int_1^2 \left[(5-x^2) - (x-4)\right] \; dx &= \int_1^2 \left(9-x-x^2\right) \; dx \\
                                           &= \left(9x - \frac12 x^2 - \frac13 x^3 \right) \bigg|_1^2 \\
                                           &= \left(18 - 2 - \frac83 \right) - \left(9 - \frac12 - \frac13 \right)\\
                                           &= 7 - \frac73 + \frac12 \\
                                           & = \frac{31}{6}.
\end{align*}
                                           
\begin{center}
\begin{tikzpicture}
\begin{axis}[axis x line=  none, axis y line = none, xtick={.9, 2.1}, 
xticklabels={$1$, $2$}, title={\textup{Area between the parabola} $y=5-x^2$ and the line $y=x-4$ 
from $x = 1$ to $x=2$}]

\addplot[name path=AA, domain=1:2, samples = 100, color=black, thick]{5-x^2};
\addplot[name path=BB, domain=1:2, samples = 100, color=black, thick]{x-4};
\addplot[domain=.8:2.2, samples = 100, color=black, thick]{5-x^2};
\addplot[domain=.8:2.2, samples = 100, color=black, thick]{x-4};




\addplot[blue!10] fill between[of=AA and BB];
\addplot[white] coordinates {(2.5,0) (2.8,0)};
\addplot[white] coordinates {(0,0) (0,-5)};

\addplot[thick, blue] coordinates {(1,-3) (1, 4)};
\addplot[thick, blue] coordinates {(2,-2) (2, 1)};

\addplot[<->] coordinates {(0,-3) (0, 4)};
\addplot[<->] coordinates {(-.5,0) (2.5, 0)};


\node at (axis cs: 2.2,2.8){$f(x)=5-x^2$};
\node at (axis cs: 1.9,-3){$g(x)=x-4$};
%\node at (axis cs: 0.2,1.25){$y = f(x)$};
\node at (axis cs: 0,4.5){$y$};
\node at (axis cs: 2.6,0){$x$};
\node at (axis cs: .9,.-.4){$1$};
\node at (axis cs: 2.1,-.4){$2$};
\node at (axis cs: 1.4,-4.8){$\displaystyle{A(x)=\int_{a}^{b} \left[f(x)-g(x)\right]\; dx=\frac{31}{6}}$};

\end{axis}
\end{tikzpicture}
\end{center}

\end{example}


\begin{example}
Find the area of the region bounded by the curves $y = \cos(x)$ and $y = \sin(x)$ from $x = 0$ to $x = \frac{\pi}{4}$.
Since $\cos(x) \geq \sin(x)$ on the interval $\left[0, \frac{\pi}{4}\right]$ (see the figure below), the area of the region is given by
\begin{align*}
\int_0^{\pi/4} [\cos(x) - \sin(x)] \; dx &= [\sin(x) + \cos(x)]\Big|_0^{\pi/4} \\
                                         &= [\sin(\pi/4) + \cos(\pi/4)] - [\sin(0) + \cos(0)]\\
                                          &= \left(\tfrac{\sqrt 2}{2} + \tfrac{\sqrt 2}{2} \right) - (0 + 1) \\
                                         &= \sqrt 2 - 1.
\end{align*}

\begin{center}
\begin{tikzpicture}
\begin{axis}[axis x line=  none, axis y line = none, xtick={0, .8}, 
xticklabels={$1$, $2$}, title={Area between $y=\sin(x)$ and $y=\cos(x)$}] 

\addplot[name path=AA, domain=0:.785, samples = 100, color=black, thick]{sin(deg(x))};
\addplot[name path=BB, domain=0:.785, samples = 100, color=black, thick]{cos(deg(x))};
\addplot[domain=-.1:.9, samples = 100, color=black, thick]{sin(deg(x))};
\addplot[domain=-.1:.9, samples = 100, color=black, thick]{cos(deg(x))};




\addplot[blue!10] fill between[of=AA and BB];
\addplot[white] coordinates {(0,0) (0,-1)};
\addplot[thin] coordinates {(.785,-.05) (.785,.05)};

\addplot[<->] coordinates {(0,-.2) (0, 1.2)}; %y-axis
\addplot[<->] coordinates {(-.2,0) (1.2,0)}; %x-axis


\node at (axis cs: .6,1.1){$f(x)=\cos(x)$};
\node at (axis cs: .7,.3){$g(x)=\sin(x)$};
%\node at (axis cs: 0.2,1.25){$y = f(x)$};
\node at (axis cs: 0,1.3){$y$};
\node at (axis cs: 1.25,0){$x$};
\node at (axis cs: .05,.-.1){$0$};
\node at (axis cs: .785,-.2){$\frac{\pi}{4}$};
\node at (axis cs: .5,-.6){$\displaystyle{A(x)=\int_{a}^{b} \left[f(x)-g(x)\right]\; dx=\sqrt2-1}$};

\end{axis}
\end{tikzpicture}
\end{center}


\end{example}
                                         

In some cases, the left and right boundaries of the region are determined by points of intersection between the curves in question.

\begin{example}
Find the area between the parabola $y = x^2 + x - 3$ and the line $y = 2x - 1$. The left and right boundaries of the region between the line and the parabola are 
the points of intersection of the two curves. To find these points of intersection, we set them equal and solve for $x$:
\[
x^2 + x - 3 = 2x-1.
\]
Moving the terms to the left gives
\[
 x^2 -x -2 = 0.
 \]
 Then factor:
 \[
  (x+1)(x-2) = 0,
  \]
  which gives
  \[
   x = -1 \;\text{and}\; x= 2
   \]
   as the solutions.
Noting that the line is above the parabola (see the figure below), the area between the curves is 

\begin{align*}
\int_1^2 \left[(2x - 1) - (x^2 + x - 3)\right] \; dx &= \int_{-1}^2 \left(2 + x - x^2\right) \; dx \\
                                           &= \left(2x + \frac12 x^2 - \frac13 x^3 \right) \bigg|_{-1}^2 \\
                                           &= \left(4 + 2 - \frac83 \right) - \left(-2 + \frac12 + \frac13 \right)\\
                                           &= 8 - \frac93 - \frac12 \\
                                           & = \frac{9}{2}.
\end{align*}

\begin{center}
\begin{tikzpicture}
\begin{axis}[axis x line=  none, axis y line = none, xtick={-2,3}, 
xticklabels={$-2$, $-1$, $0$, $1$, $2$}, title={Area between $y=x^2+x-3$ and $y=2x-1$}] 

\addplot[name path=AA, domain=-1:2, samples = 100, color=black, thick]{x^2+x-3};
\addplot[name path=BB, domain=-1:2, samples = 100, color=black, thick]{(2*x-1};
\addplot[domain=-1.5:2.3, samples = 100, color=black, thick]{x^2+x-3};
\addplot[domain=-1.5:2.3, samples = 100, color=black, thick]{2*x-1};




\addplot[blue!10] fill between[of=AA and BB];
\addplot[white] coordinates {(0,6) (0,-7)};
\addplot[white] coordinates {(3,0) (3.7,0)};
\addplot[thin] coordinates {(-1,-.4)(-1,.4)};
\addplot[thin] coordinates {(2,-.4) (2,.4)};

\addplot[<->] coordinates {(0,-5) (0,5)}; %y-axis
\addplot[<->] coordinates {(-2,0) (3,0)}; %x-axis


\node at (axis cs: .3,2){$f(x)=2x-1$};
\node at (axis cs: 1.3,-3.5){$g(x)=x^2+x-3$};
%\node at (axis cs: 0.2,1.25){$y = f(x)$};
\node at (axis cs: .2,5){$y$};
\node at (axis cs: 3.2,0){$x$};
\node at (axis cs: -1.1,-.9){$-1$};
\node at (axis cs: 2,-.9){$2$};
\node at (axis cs: 0,-6.6){$\displaystyle{A(x)=\int_{a}^{b} \left[f(x)-g(x)\right]\; dx=\frac{9}{2}}$};

\end{axis}
\end{tikzpicture}
\end{center}

\end{example}

\begin{example}
Find the area of the region bounded by the curves $y = \sqrt x$ and $y = x^2$.
We find the points of intersection of the curves by solving the equation $\sqrt x = x^2$.
We can square both sides, move the terms to one side, and factor, but the elementary nature of this equation allows up 
to find the two solutions by inspection.  The curves intersect at $x = 0$ and at $x = 1$.
Noting that $\sqrt x \geq x^2$ on the interval $[0,1]$, we can calculate the area of the region as
\begin{align*}
\int_0^1 \left(\sqrt x - x^2 \right) \; dx &= \int_0^1 \left(x^{1/2} - x^2 \right) \; dx  \\
                                          &=  \left(\frac23 x^{3/2} - \frac13 x^3 \right) \bigg|_0^1 \\
                                          &=  \left( \tfrac23 - \tfrac13 \right) - (0-0) \\
                                          &= \frac13.
\end{align*}

\begin{center}
\begin{tikzpicture}
\begin{axis}[axis x line=  none, axis y line = none, xtick={-2,3}, 
xticklabels={$-2$, $-1$, $0$, $1$, $2$}, title={Area between $y=\sqrt[]{x}$ and $y=x^2$}] 

\addplot[name path=AA, domain=0:1, samples = 100, color=black, thick]{x^.5};
\addplot[name path=BB, domain=0:1, samples = 100, color=black, thick]{(x^2)};
\addplot[domain=-.5:1.2, samples = 100, color=black, thick]{x^.5};
\addplot[domain=-.5:1.2, samples = 100, color=black, thick]{x^2};




\addplot[blue!10] fill between[of=AA and BB];
\addplot[white] coordinates {(0,2) (0,-2)};
\addplot[white] coordinates {(2,0) (2.1,0)};

\addplot[thin] coordinates {(1,-.1) (1,.1)};

\addplot[<->] coordinates {(0,-1) (0,2)}; %y-axis
\addplot[<->] coordinates {(-1,0) (2,0)}; %x-axis


\node at (axis cs: .5,1.2){$f(x)=\sqrt[]{x}$};
\node at (axis cs: 1.3,.6){$g(x)=x^2$};
%\node at (axis cs: 0.2,1.25){$y = f(x)$};
\node at (axis cs: .1,1.8){$y$};
\node at (axis cs: 2.1,0){$x$};
\node at (axis cs: .1,-.2){$0$};
\node at (axis cs: 1.1,-.2){$1$};
\node at (axis cs: 0,-1.5){$\displaystyle{A(x)=\int_{a}^{b} \left[f(x)-g(x)\right]\; dx=\frac{1}{3}}$};

\end{axis}
\end{tikzpicture}
\end{center}


\end{example}

In the next example, we explore the area bounded between two curves which are intertwined.

\begin{example}
Find the area between the curves $y = \cos(x)$ and $y=\sin(x)$ over the interval $[0, \pi]$.
It is important to observe that the curves intersect at $\pi/4$  
($\sin(\pi/4) = \cos(\pi/4) = \sqrt2 / 2$) which is in the interval $[0, \pi]$. 
We must also observe that $\cos(x) \geq \sin(x)$ on the interval $[0, \pi/4]$ and
that the reverse is true on the interval $[\pi/4, \pi]$.
Thus, the area bounded between the two curves from $x = 0$ to $x = \pi$ is in the form of two separate regions and hence 
we will need two separate integrals to compute the total area:
\[
\text{Area} = \int_0^{\pi/4} \left[\cos(x) - \sin(x)\right] \; dx + \int_{\pi/4}^\pi \left[\sin(x) -\cos(x)\right] \; dx.
\]
We will compute these integrals separately and then add the results.
For the first integral,
\begin{align*}
\int_0^{\pi/4} \left[\cos(x) - \sin(x)\right] \; dx &= \left[\sin(x) + \cos(x)\right]\bigg|_0^{\pi/4}\\
                                                    &= \left[\sin(\tfrac{\pi}{4}) + \cos(\tfrac{\pi}{4})\right]-\left[\sin(0) + \cos(0)\right]\\
                                                    &= \tfrac{\sqrt 2}{2} + \tfrac{\sqrt 2}{2} - 1\\
                                                    &= \sqrt{2} - 1.
\end{align*}
For the second integral,
\begin{align*}
\int_{\pi/4}^\pi \left[\sin(x) -\cos(x)\right] \; dx &= \left[-\cos(x) - \sin(x) \right]\bigg|_{\pi/4}^{\pi} \\
                                                     &= \left[\cos(x) + \sin(x)\right]\bigg|_{\pi}^{\pi/4}\\
                                                     &= \left[\cos(\tfrac{\pi}{4}) + \sin(\tfrac{\pi}{4})\right]-\left[\cos(\pi) + \sin(\pi)\right]\\
                                                     &= \tfrac{\sqrt 2}{2} + \tfrac{\sqrt 2}{2} - (-1)\\
                                                     &= \sqrt{2} + 1.
\end{align*}                                             
To obtain the total area, we now add the values of these definite integrals:
\[
\text{Area} = 2\sqrt{2}.
\]

\begin{center}
\begin{tikzpicture}
\begin{axis}[axis x line=  none, axis y line = none, xtick={0, .8}, 
xticklabels={$1$, $2$}, title={Area between $y=\sin(x)$ and $y=\cos(x)$}] 

\addplot[name path=AA, domain=0:3.1415, samples = 100, color=black, thick]{sin(deg(x))};
\addplot[name path=BB, domain=0:3.1415, samples = 100, color=black, thick]{cos(deg(x))};
\addplot[domain=-.1:3.5, samples = 100, color=black, thick]{sin(deg(x))};
\addplot[domain=-.1:3.5, samples = 100, color=black, thick]{cos(deg(x))};
\addplot[domain=-0.3:0, samples = 100, color=black, thick]{sin(deg(x))};
\addplot[domain=-0.3:0, samples = 100, color=black, thick]{cos(deg(x))};



\addplot[blue!10] fill between[of=AA and BB];

\addplot[white] coordinates {(-3,0) (-2,0)};
\addplot[white] coordinates {(0,-3) (0,-2)};
\addplot[thin] coordinates {(.785,-.05) (.785,.05)};

\addplot[<->] coordinates {(0,-1.5) (0,1.5)}; %y-axis
\addplot[<->] coordinates {(-.5,0) (4,0)}; %x-axis
\addplot[dashed, thin] coordinates {(3.14, 0) (3.14, -1)};

\node at (axis cs: -1.4,1){$y=\cos(x)$};
%\node at (axis cs: 1.4,-1.2){$y=\cos(x)$};
\node at (axis cs: -1.4,-.3){$y=\sin(x)$};
%\node at (axis cs: 2,0.3){$f(x)=\sin(x)$};
%\node at (axis cs: 0.2,1.25){$y = f(x)$};
\node at (axis cs: 0,2.2){$y$};
\node at (axis cs: 4.2,0){$x$};
\node at (axis cs: .2,.-.2){$0$};
\node at (axis cs: .785,-.5){$\frac{\pi}{4}$};
\node at (axis cs: 3,-.2){$\pi$};
\node at (axis cs: 1.3,-2){$\displaystyle{Area=2\sqrt2}$};

\end{axis}
\end{tikzpicture}
\end{center}


\end{example}



\begin{problem}
Find the area bounded between the curves $y = x^3$ and $y = x$.
\begin{hint}
Two integrals are needed
\end{hint}
\begin{hint}
There are three points of intersection (one is negative)
\end{hint}
The area is $\answer{1/2}$

\end{problem}



In the next example, we find the area of a region bounded by three curves. In this situation, we will need to use two separate integrals to determine the total area.

\begin{example}
Find the area of the region bounded by the curves $y = 2-x^2, y = -2x-1,$ and $y = x$, as shown in the figure below.\\

\begin{center}
\begin{tikzpicture}
\begin{axis}[axis x line=  none, axis y line = none, title={Region between $y=2-x^2$, $y=-2x-1$ and $y=x$}] 

\addplot[name path=AA, domain=-1:1, samples = 100, color=black, thick]{2-x^2};
\addplot[name path=BB, domain=-1:-.33, samples = 100, color=black, thick]{(-2*x-1};
\addplot[name path=CC, domain=-.33:1, samples = 100, color=black, thick]{(x)};
\addplot[domain=-1.5:1.5, samples = 100, color=black, thick]{2-x^2};
\addplot[domain=-1.5:0.2, samples = 100, color=black, thick]{-2*x-1};
\addplot[domain=-.33:1.5, samples = 100, color=black, thick]{x};
\addplot[domain=-.8:-.33, samples = 100, color=black, thick]{(x)};


\addplot[blue!10] fill between[of=AA and BB];
\addplot[blue!10] fill between[of=AA and CC];
\addplot[white] coordinates {(0,3) (0,-3)};
\addplot[white] coordinates {(3,0) (3.7,0)};

\addplot[<->] coordinates {(0,-2) (0,3.5)}; %y-axis
\addplot[<->] coordinates {(-2.3,0) (2.3,0)}; %x-axis


\node at (axis cs: 1.1,2.3){$y=2-x^2$};
\node at (axis cs: -1.1,-1.1){$y=x$};
\node at (axis cs:-1.5,2.3){$y=-2x-1$};
%\node at (axis cs: 0.2,1.25){$y = f(x)$};
\node at (axis cs: .2,3.6){$y$};
\node at (axis cs: 2.4,0){$x$};


\end{axis}
\end{tikzpicture}
\end{center}



First, we need to find the points of intersection of the the curves. The intersection of the lines is at
\[
-2x-1 = x 
\]
\[
x = -\tfrac13.
\]
Now the intersection of the parabola and the first line:
\[
 2-x^2 = -2x-1
\]
\[
   0 = x^2 -2x -3 
\]
\[
   0 = (x+1)(x-3)
\]
\[
    x = -1, \; x = 3.
\]

And the intersection of the parabola and the second line:

\[
 2-x^2 = x
\]
\[
   0 =x^2 +x -2
\]
\[
   0 =(x+2)(x-1)
\]
\[
   x = -2, \; x = 1.
\]

We can now see that the region is bounded by $x = -1$ on the left, and  $x = 1$ on the right.
Furthermore, the parabola, $y = 2 - x^2$ is the top curve, while the line $y = -2x-1$ is the bottom curve 
from $x = -1$ to $x = -\frac13$, and the line $y = x$ is the bottom curve from $x = -\frac13$ to $x = 1$.
Hence the area is the sum of two integrals:
\[
\text{Area} = \int_{-1}^{-1/3} \left[(2 - x^2) - (-2x-1)\right] \; dx + \int_{-1/3}^1 \left[(2 - x^2) - (x)\right] \; dx.
\]
The value of the first integral is
\begin{align*}
\int_{-1}^{-1/3} \left[(2 - x^2) - (-2x-1)\right] \; dx &= \int_{-1}^{-1/3} \left((3 - x^2 +2x)\right) \; dx \\
                                                        &= \left(3x - \frac{x^3}{3} + x^2 \right)\bigg|_{-1}^{-1/3} \\
                                                        &= \left(-1 + \tfrac{1}{81} + \tfrac19 \right) - \left(-3 + \tfrac13 + 1 \right) \\
                                                        &= \tfrac{64}{81}
\end{align*}

And the value of the second integral is
\begin{align*}
\int_{-1/3}^{1} \left[(2 - x^2) - (x)\right] \; dx &= \int_{-1/3}^{1} \left(2 - x^2 - x\right) \; dx \\
                                                        &= \left(2x - \frac{x^3}{3} -\frac{x^2}{2} \right)\bigg|_{-1/3}^{1} \\
                                                        &= \left(2 - \tfrac{1}{3} - \tfrac12 \right) - \left(-\tfrac23 + \tfrac{1}{81} -\tfrac{1}{18} \right) \\
                                                        &= \tfrac{152}{81}.
\end{align*}
Thus, the total area of the region is the sum of these two integrals:
\[
\text{Area} = \frac{64}{81} + \frac{152}{81} = \frac{8}{3}.
\]

\begin{center}
\begin{tikzpicture}
\begin{axis}[axis x line=  none, axis y line = none, xtick={-2,3}, 
xticklabels={$-2$, $-1$, $0$, $1$, $2$}, title={Area between $y=2-x^2$, $y=-2x-1$ and $y=x$}] 

\addplot[name path=AA, domain=-1:1, samples = 100, color=black, thick]{2-x^2};
\addplot[name path=BB, domain=-1:-.33, samples = 100, color=black, thick]{(-2*x-1};
\addplot[name path=CC, domain=-.33:1, samples = 100, color=black, thick]{(x)};
\addplot[domain=-1.5:1.5, samples = 100, color=black, thick]{2-x^2};
\addplot[domain=-1.5:0, samples = 100, color=black, thick]{-2*x-1};
\addplot[domain=-.33:1.5, samples = 100, color=black, thick]{x};
\addplot[domain=-.8:-.33, samples = 100, color=black, thick]{(x)};


\addplot[blue!10] fill between[of=AA and BB];
\addplot[blue!10] fill between[of=AA and CC];
\addplot[white] coordinates {(0,3) (0,-3)};
\addplot[white] coordinates {(3,0) (3.7,0)};
\addplot[thin] coordinates {(-1,-.2)(-1,.2)};
\addplot[thin] coordinates {(1,-.2) (1,.2)};

\addplot[<->] coordinates {(0,-2) (0,3.5)}; %y-axis
\addplot[<->] coordinates {(-2.3,0) (2.3,0)}; %x-axis


\node at (axis cs: 1.1,2.3){$y=2-x^2$};
\node at (axis cs: -.8,-1){$y=x$};
\node at (axis cs:-1.5,2.5){$y=-2x-1$};
%\node at (axis cs: 0.2,1.25){$y = f(x)$};
\node at (axis cs: .2,3.6){$y$};
\node at (axis cs: 2.4,0){$x$};
\node at (axis cs: -1.1,-.4){$-1$};
\node at (axis cs: 1,-.4){$1$};
\node at (axis cs: 0,-2.7){$\displaystyle{Area=\frac{8}{3}}$};

\end{axis}
\end{tikzpicture}
\end{center}

\end{example}

In our next example, we consider curves of the form $x = f(y)$. In this situation, 
the formula for area is similar to the $y = f(x)$ situation.
If $x = f(y)$ and $x=g(y)$ are two curves on the interval from $y = c$ to $y = d$ with $f(y) \geq g(y)$
then the area between them is given by the formula
\[
\text{Area} = \int_c^d [f(y) - g(y)] \; dy.
\]
\begin{example}
Find the area between the curves $x = y^2 + y$ and $x = 2y + 6$.
First, we find the $y$-coordinates of the points of intersection of the line $x = 2y+6$ 
and the parabola $x=y^2 + y$ (which opens to the right), by setting these formulas equal and solving for $y$:
\begin{align*}
y^2 + y &= 2y + 6 \\
y^2 - y - 6 &= 0 \\
(y+2)(y-3) &= 0 \\
y = -2, & y = 3.
\end{align*}
The bigger function is the line $x = 2y+6$, as we can see by plugging in any value between $y = -2$ and $y=3$.
The area is given by
\begin{align*}
\text{Area} &= \int_{-2}^3 \left[(2y+6) - (y^2+y) \right] \; dy \\
            &= \int_{-2}^3 \left(y - y^2 + 6\right) \; dy \\
            &= \left(\frac{y^2}{2} - \frac{y^3}{3} + 6y \right) \bigg|_{-2}^{3} \\
            &= \left(\tfrac92 - 9 + 18\right) - \left(2 + \tfrac83 - 12\right)\\
            &= \tfrac{157}{6}         
\end{align*}


\begin{center}
\begin{tikzpicture}
\begin{axis}[axis x line=  none, axis y line = none, xtick={-2,3}, 
xticklabels={$-2$, $-1$, $0$, $1$, $2$}, title={Area between $x=y^2+y$ and $x=2y+6$}] 

\addplot[name path=AA, scale=.5, domain=-.25:4.7, samples = 100, color=black, thick]{((x+.25)^.5)-.5};
\addplot[name path=CC, scale=.5, domain=-.25:1.3, samples = 100, color=black, thick]{(-(x+.25)^.5)-.5};
\addplot[name path=BB, scale=.5, domain=0:5, samples = 100, color=black, thick]{.5(x)-3};


\node at (axis cs:.4,-2){$-2$};
\node at (axis cs:.4,2){$3$};
\node at (axis cs:1.7,1.8){$y^2+y$};
\node at (axis cs:2.6,-1.5){$2y+6$};
\addplot[thin] coordinates {(-.1,-2) (.1,-2)};
\addplot[thin] coordinates {(-.1,2) (.1,2)};
\addplot[white] coordinates {(0,3) (0,-5)};

\addplot[blue!10] fill between [of=AA and CC];

\addplot[<->] coordinates {(0,-3) (0,4)}; %y-axis
\addplot[<->] coordinates {(-1,0) (6,0)}; %x-axis

\node at (axis cs: 0,-4.5){$\displaystyle{A(x)=\int_{a}^{b} \left[f(x)-g(x)\right]\; dx=\frac{157}{6}}$};

\end{axis}
\end{tikzpicture}
\end{center}



\end{example}


%Need example with curves crossing, three curves, functions of y, velocity and distance.



\begin{center}
\begin{foldable}
\unfoldable{Here is a detailed, lecture style video on the area between curves:}
\youtube{OJUa7TynRoA}
\end{foldable}
\end{center}





\end{document}














\begin{center}
\begin{foldable}
\unfoldable{Here is a video of Example 1}
%\youtube{###} %vid of example 1
\end{foldable}
\end{center}

\begin{problem} %problem #1
  Find the area between...
  \[
  f(x) =  ...
  \]
    \begin{hint}
      Set up a definite integral
    \end{hint}
    \begin{hint}
      Determine the top and bottom curves
    \end{hint}
    
		
		The area between the curves is
		 $\answer[given]{number}$
\end{problem}







\begin{example} %example #15
Find $h'(x)$ if $h(x) = x^{\sin(x)}$.\\
We will use the fact that the exponential and logarithm functions are inverses,
\[e^{\ln(x)} = x,\]
and the exponent property of logarithms, 
\[\ln(x^n) = n\ln(x),\]
to rewrite $h(x)$.  We have 
\[h(x) = x^{\sin(x)} = e^{\ln(x^{\sin(x)})} = e^{\sin(x)\ln(x)}\]
and we can now compute $h'(x)$ using a combination of the chain rule and product rule.
We can write $h(x)$ as a composition, $f(g(x))$ with 
\[g(x) = \sin(x)\ln(x) \quad \text{and} \quad f(x) = e^x.\]
Then to find $g'(x)$ we us the product rule and we get $g'(x) = \frac{\sin(x)}{x} + \cos(x)\ln(x)$.
Next $f'(x) = e^x$ and 
hence $f'(g(x)) = e^{g(x)} = e^{\sin(x)\ln(x)} = x^{\sin(x)}$.
We can then conclude $h'(x) = f'(g(x))g'(x) = x^{\sin(x)} \left[ \frac{\sin(x)}{x} + \cos(x)\ln(x)\right]$.
\end{example}

%more question formats below













%\begin{verbatim}
\begin{question}
What is your favorite color?
\begin{multipleChoice}
\choice[correct]{Rainbow}
\choice{Blue}
\choice{Green}
\choice{Red}
\end{multipleChoice}
\begin{freeResponse}
Hello
\end{freeResponse}
\end{question}
%\end{verbatim}





\begin{question}
  Which one will you choose?
  \begin{multipleChoice}
    \choice[correct]{I'm correct.}
    \choice{I'm wrong.}
    \choice{I'm wrong too.}
  \end{multipleChoice}
\end{question}


\begin{question}
  Which one will you choose?
  \begin{selectAll}
    \choice[correct]{I'm correct.}
    \choice{I'm wrong.}
    \choice[correct]{I'm also correct.}
    \choice{I'm wrong too.}
  \end{selectAll}
\end{question}


\begin{freeResponse}
What is the chain rule used for?
\end{freeResponse}
