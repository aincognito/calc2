\documentclass{ximera}

%% You can put user macros here
%% However, you cannot make new environments



\newcommand{\ffrac}[2]{\frac{\text{\footnotesize $#1$}}{\text{\footnotesize $#2$}}}
\newcommand{\vasymptote}[2][]{
    \draw [densely dashed,#1] ({rel axis cs:0,0} -| {axis cs:#2,0}) -- ({rel axis cs:0,1} -| {axis cs:#2,0});
}


\graphicspath{{./}{firstExample/}}

\usepackage{amsmath}
\usepackage{amssymb}
\usepackage{array}
\usepackage[makeroom]{cancel} %% for strike outs
\usepackage{pgffor} %% required for integral for loops
\usepackage{tikz}
\usepackage{tikz-cd}
\usepackage{tkz-euclide}
\usetikzlibrary{shapes.multipart}


\usetkzobj{all}
\tikzstyle geometryDiagrams=[ultra thick,color=blue!50!black]


\usetikzlibrary{arrows}
\tikzset{>=stealth,commutative diagrams/.cd,
  arrow style=tikz,diagrams={>=stealth}} %% cool arrow head
\tikzset{shorten <>/.style={ shorten >=#1, shorten <=#1 } } %% allows shorter vectors

\usetikzlibrary{backgrounds} %% for boxes around graphs
\usetikzlibrary{shapes,positioning}  %% Clouds and stars
\usetikzlibrary{matrix} %% for matrix
\usepgfplotslibrary{polar} %% for polar plots
\usepgfplotslibrary{fillbetween} %% to shade area between curves in TikZ



%\usepackage[width=4.375in, height=7.0in, top=1.0in, papersize={5.5in,8.5in}]{geometry}
%\usepackage[pdftex]{graphicx}
%\usepackage{tipa}
%\usepackage{txfonts}
%\usepackage{textcomp}
%\usepackage{amsthm}
%\usepackage{xy}
%\usepackage{fancyhdr}
%\usepackage{xcolor}
%\usepackage{mathtools} %% for pretty underbrace % Breaks Ximera
%\usepackage{multicol}



\newcommand{\RR}{\mathbb R}
\newcommand{\R}{\mathbb R}
\newcommand{\C}{\mathbb C}
\newcommand{\N}{\mathbb N}
\newcommand{\Z}{\mathbb Z}
\newcommand{\dis}{\displaystyle}
%\renewcommand{\d}{\,d\!}
\renewcommand{\d}{\mathop{}\!d}
\newcommand{\dd}[2][]{\frac{\d #1}{\d #2}}
\newcommand{\pp}[2][]{\frac{\partial #1}{\partial #2}}
\renewcommand{\l}{\ell}
\newcommand{\ddx}{\frac{d}{\d x}}

\newcommand{\zeroOverZero}{\ensuremath{\boldsymbol{\tfrac{0}{0}}}}
\newcommand{\inftyOverInfty}{\ensuremath{\boldsymbol{\tfrac{\infty}{\infty}}}}
\newcommand{\zeroOverInfty}{\ensuremath{\boldsymbol{\tfrac{0}{\infty}}}}
\newcommand{\zeroTimesInfty}{\ensuremath{\small\boldsymbol{0\cdot \infty}}}
\newcommand{\inftyMinusInfty}{\ensuremath{\small\boldsymbol{\infty - \infty}}}
\newcommand{\oneToInfty}{\ensuremath{\boldsymbol{1^\infty}}}
\newcommand{\zeroToZero}{\ensuremath{\boldsymbol{0^0}}}
\newcommand{\inftyToZero}{\ensuremath{\boldsymbol{\infty^0}}}


\newcommand{\numOverZero}{\ensuremath{\boldsymbol{\tfrac{\#}{0}}}}
\newcommand{\dfn}{\textbf}
%\newcommand{\unit}{\,\mathrm}
\newcommand{\unit}{\mathop{}\!\mathrm}
%\newcommand{\eval}[1]{\bigg[ #1 \bigg]}
\newcommand{\eval}[1]{ #1 \bigg|}
\newcommand{\seq}[1]{\left( #1 \right)}
\renewcommand{\epsilon}{\varepsilon}
\renewcommand{\iff}{\Leftrightarrow}

\DeclareMathOperator{\arccot}{arccot}
\DeclareMathOperator{\arcsec}{arcsec}
\DeclareMathOperator{\arccsc}{arccsc}
\DeclareMathOperator{\si}{Si}
\DeclareMathOperator{\proj}{proj}
\DeclareMathOperator{\scal}{scal}
\DeclareMathOperator{\cis}{cis}
\DeclareMathOperator{\Arg}{Arg}
%\DeclareMathOperator{\arg}{arg}
\DeclareMathOperator{\Rep}{Re}
\DeclareMathOperator{\Imp}{Im}
\DeclareMathOperator{\sech}{sech}
\DeclareMathOperator{\csch}{csch}
\DeclareMathOperator{\Log}{Log}

\newcommand{\tightoverset}[2]{% for arrow vec
  \mathop{#2}\limits^{\vbox to -.5ex{\kern-0.75ex\hbox{$#1$}\vss}}}
\newcommand{\arrowvec}{\overrightarrow}
\renewcommand{\vec}{\mathbf}
\newcommand{\veci}{{\boldsymbol{\hat{\imath}}}}
\newcommand{\vecj}{{\boldsymbol{\hat{\jmath}}}}
\newcommand{\veck}{{\boldsymbol{\hat{k}}}}
\newcommand{\vecl}{\boldsymbol{\l}}
\newcommand{\utan}{\vec{\hat{t}}}
\newcommand{\unormal}{\vec{\hat{n}}}
\newcommand{\ubinormal}{\vec{\hat{b}}}

\newcommand{\dotp}{\bullet}
\newcommand{\cross}{\boldsymbol\times}
\newcommand{\grad}{\boldsymbol\nabla}
\newcommand{\divergence}{\grad\dotp}
\newcommand{\curl}{\grad\cross}
%% Simple horiz vectors
\renewcommand{\vector}[1]{\left\langle #1\right\rangle}


\outcome{Find the length of a curve}

\title{1.5 Arc Length}

\begin{document}

\begin{abstract}
We find the length of a curved segment.
\end{abstract}

\maketitle

\section{Arc Length}

\begin{theorem}[Arc Length] The length of the graph of the differentiable function, $y = f(x)$, from $x = a$ to $x = b$, is given by
\[
L = \int_a^b \sqrt{1+ f'(x)^2} \; dx.
\]
\end{theorem}

\begin{example}[example 1]
Find the length of the curve $y = 3x + 2$ from $x = 0$ to $x = 4$. \\
Since the graph of this function is a line, calculus is not required to find the length of the indicated segment.
Instead, we can use the distance formula, 
\[
d = \sqrt{(x_2 - x_1)^2 + (y_2-y_1)^2},
\]
 with the points $(0,2)$ and $(4, 14)$
to get 
\[
d = \sqrt{(4-0)^2 + (14-2)^2} = \sqrt{4^2 + 12^2} = \sqrt{160} = 4\sqrt{10}.
\]
We will also use the formula from the above theorem as a verification that the theorem is correct.
Since $f(x) = 3x+2$, we have $f'(x) = 3$ and the length of the segment is
\[
L = \int_0^4 \sqrt{1 + 3^2} \;dx = \int_0^4 \sqrt{10} \; dx = \sqrt{10}x\bigg|_0^4 = 4\sqrt{10},
\]
as expected.
\end{example}



\begin{problem}(problem 1a)
Find the length of the curve $ y = 2x - 5$ from $x = 2$ to $x = 5$ using the arc length formula.\\

$f'(x) = \answer{2}$\\

$\sqrt{1+f'(x)^2} = \answer{\sqrt{5}}$\\

Arc length = $\answer{3\sqrt{5}}$.
\end{problem}



\begin{problem}(problem 1b)
Find the length of the curve $y = 2x - 5$ from $x = 2$ to $x = 5$ using the distance formula.\\

The endpoints of the line segment are $(2, \answer{-1})$ and $(5, \answer{5})$\\

$d = \sqrt{(x_2 - x_1)^2 + (y_2 - y_1)^2} = \sqrt{(\answer{3})^2 + (\answer{6})^2}$\\

Arc length = $\answer{3\sqrt{5}}$.
\end{problem}




\begin{example}[example 2]
 Find the length of the graph of 
$y = 1+ x^{3/2}$ 
from  $x = 0$ to $x = 4$.\\
First, $f(x) = x^{3/2} + 1$ and so $f'(x) = \frac32 x^{1/2}$. 
Then the arc length is
\[
L = \int_0^4 \sqrt{1 + \left(\frac32 x^{1/2}\right)^2} \; dx = \int_0^4 \sqrt{1 + \frac94 x} \;dx.
\]
To compute this integral, we will use a $u$-substitution with $u = 1+\frac94 x$ and $du = \frac94 \; dx$.
Thus, the anti-derivative is,
\[
\int \sqrt{1+\frac94 x} \; dx = \int \frac49 \sqrt u \; du = \frac49 \cdot \frac23 u^{3/2} + C = \frac{8}{27} \left(1 + \frac94 x \right)^{3/2} + C.
\]

Returning to our computation of arc length, 
\[
L = \frac{8}{27} \left(1 + \frac94 x \right)^{3/2} \bigg|_0^4 = \frac{8}{27} (10^{3/2} - 1).
\]
\end{example}


\begin{problem}(problem 2a)
Find the length of the curve $y = 2 + 4x^{3/2}$ from $x = 0$ to $x = 1$\\

$f'(x) = \answer{6x^{1/2}}$\\

$\sqrt{1+f'(x)^2} = \answer{\sqrt{1+ 36x}}$\\

Arc length = $\answer{\frac{1}{54} (37^{3/2} -1)}$
\end{problem}



\begin{problem}(problem 2b)
Find the length of the curve $y = 3 + 2x^{3/2}$ from $x = 7$ to $x = 11$\\

$f'(x) = \answer{3x^{1/2}}$\\

$\sqrt{1+f'(x)^2} = \answer{\sqrt{1+ 9x}}$\\

Arc length = $\answer{\frac{2}{27} (488)}$
\end{problem}




\begin{example} Find the length of the graph of 
$y = \frac{x^3}{12} + \frac{1}{x}$ from $x = 1$ to $x = 2$.\\
In this example, computing a simplified form of the integrand is laborious, so we will proceed slowly.
First, $f(x) = \frac{x^3}{12} + x^{-1}$, so $f'(x) = \frac{x^2}{4} - \frac{1}{x^2}$ and
\[
f'(x)^2 = \left( \frac{x^2}{4} - \frac{1}{x^2} \right)^2 = \frac{x^4}{16} - \frac12 + \frac{1}{x^4}.
\]
Next, we add one:
\[
1 + f'(x)^2 = \frac{x^4}{16} + \frac12 + \frac{1}{x^4},
\]
and the key to this problem is that this expression is a perfect square:
\[
\frac{x^4}{16} + \frac12 + \frac{1}{x^4} = \left( \frac{x^2}{4} + \frac{1}{x^2}\right)^2.
\]
Now, we can compute the square root:
\[
\sqrt{1+f'(x)^2} = \sqrt{\left( \frac{x^2}{4} + \frac{1}{x^2}\right)^2 } = \frac{x^2}{4} + \frac{1}{x^2}.
\]
Finally, we can integrate to get the arc length:
\begin{align*}
L &= \int_1^2 \sqrt{1+f'(x)^2} \; dx \\
  &= \int_1^2 \left(\frac{x^2}{4} + \frac{1}{x^2} \right) \; dx  \\
  &= \int_1^2 \left(\frac{x^2}{4} + x^{-2} \right) \; dx \\
  &= \left(\frac{x^3}{12} + \frac{x^{-1}}{-1} \right) \bigg|_1^2 \\
  &= \left(\frac{x^3}{12} - \frac{1}{x} \right) \bigg|_1^2 \\
  &= \left(\frac23 - \frac12\right) - \left(\frac{1}{12} - 1\right) \\
  &= \frac{13}{12}.  
\end{align*}
\end{example}




\begin{problem}(problem 3a)
Find the length of the curve $\displaystyle{y = \frac{x^4}{4} + \frac{1}{8x^2}}$ from $x = 1$ to $x = 2$\\

$f'(x) = \answer{x^3 - 1/(4x^3)}$\\

$f'(x)^2 =$

\begin{multipleChoice}
\choice{$\displaystyle{x^6 + \frac{1}{16x^6}}$}
\choice{$\displaystyle{x^6 -\frac14 + \frac{1}{16x^2}}$}
\choice[correct]{$\displaystyle{x^6 -\frac12 + \frac{1}{16x^2}}$}
\end{multipleChoice}

$\sqrt{1+f'(x)^2} = \left(\answer{x^3 + 1/4x^3}\right)^2$\\

Arc length = $\answer{123/32}$
\end{problem}




\begin{problem}(problem 3b)
Find the length of the curve $\displaystyle{y = \frac{x^2}{8} - \ln(x)}$ from $x = 2$ to $x = 4$\\

$f'(x) = \answer{x/4 - 1/x}$\\

$f'(x)^2 =$

\begin{multipleChoice}
\choice{$\displaystyle{\frac{x^2}{16} + \frac{1}{x^2}}$}
\choice{$\displaystyle{\frac{x^2}{16} -\frac14 + \frac{1}{x^2}}$}
\choice[correct]{$\displaystyle{\frac{x^2}{16} -\frac12 + \frac{1}{x^2}}$}
\end{multipleChoice}

$\sqrt{1+f'(x)^2} = \left(\answer{x/4 + 1/x}\right)^2$\\

Arc length = $\answer{3/2 - \ln 2}$
\end{problem}


\begin{problem}(problem 3c)
Find the length of the curve $y = \ln(\sec(x))$ from $x = 0$ to $x = \pi/4$.

$f'(x) = \answer{\tan(x)}$\\

$1 + f'(x)^2 =$

\begin{multipleChoice}
\choice[correct]{$\sec^2(x)$}
\choice{$\tan^2(x)$}
\choice{$\cos^2(x)$}
\end{multipleChoice}


Arc length = $\answer{\ln(1 + \sqrt 2)}$
\end{problem}


\section{Video Lesson}


\begin{center}
\begin{foldable}
\unfoldable{Here is a detailed, lecture style video on arc length:}
\youtube{GFXmeteU-_M}
\end{foldable}
\end{center}

\section{Theoretical Justifications}
In this section, we derive the arc length formula.
Suppose $f(x)$ is a differentiable function on the open interval $(a,b)$ and continuous on the 
closed interval $[a,b]$. Let $\Delta x = \frac{b-a}{n}$ and let $x_i = a + i\cdot \Delta x$ for 
$i = 0, 1, 2, 3, \ldots, n$.  Note that $x_0 = a$ and $x_n = a + n \cdot \Delta x = b$.
Next, to approximate the arc length, connect the adjacent points $(x_{i-1}, f(x_{i-1}))$ and
$(x_i, f(x_i))$ with straight line segments for $i = 1, 2, \ldots n.$ From the distance formula,
the length of the $i^{th}$ segment is
\[
L_i = \sqrt{(x_i - x_{i-1})^2 + \left[f(x_i) -f(x_{i-1})\right]^2} 
\]
%which can be rewritten as
%\[
%L_i = (x_i- x_{i-1}) \sqrt{1 + \frac{\left[f(x_i) -f(x_{i-1})\right]^2}{(x_i - x_{i-1})^2}} = 
%\Delta x \cdot \sqrt{1 + \left[ \frac{f(x_i) -f(x_{i-1})}{x_i - x_{i-1}} \right]^2}
%\]
Now, we apply the Mean Value Theorem to $f(x)$ on the interval $[x_{i-1}, x_i]$ to conclude that
\[
f(x_i) -f(x_{i-1})  = f'(x_i^*)(x_i - x_{i-1}) = f'(x_i^*) \cdot \Delta x
\]
for some number $x_i^*$ in the interval $(x_{i -1}, x_i)$. Thus,

\[
L_i = \sqrt{ (\Delta x)^2 + f'(x_i^*)^2 \cdot (\Delta x)^2} = \sqrt{ 1 + f'(x_i^*)^2 }\cdot \Delta x, \; i = 1, 2, \ldots n.
\]

Returning to the arc length, $L$, of $ y = f(x)$ on $[a,b]$, we have
\[
L \approx \sum_{i=1}^n L_i = \sum_{i=1}^n   \sqrt{1 + f'(x_i^*)^2} \cdot \Delta x.
\]
Taking the limit as $n \to \infty$, we define the arc length as 
\[
L = \lim_{n \to \infty} \sum_{i=1}^n   \sqrt{1 + f'(x_i^*)^2} \cdot \Delta x = \int_a^b \sqrt{1 + f'(x)^2} \; dx.
\]



\end{document}









in ab 2pi x sqrt(1+m^2) = pi(b^2 - a^2) sqrt(1 + (R-r/b-a)^2) = pi (b+a) sqrt (b-a)^2 + (R-r)^2
\begin{example} %example #15
Find $h'(x)$ if $h(x) = x^{\sin(x)}$.\\
We will use the fact that the exponential and logarithm functions are inverses,
\[e^{\ln(x)} = x,\]
and the exponent property of logarithms, 
\[\ln(x^n) = n\ln(x),\]
to rewrite $h(x)$.  We have 
\[h(x) = x^{\sin(x)} = e^{\ln(x^{\sin(x)})} = e^{\sin(x)\ln(x)}\]
and we can now compute $h'(x)$ using a combination of the chain rule and product rule.
We can write $h(x)$ as a composition, $f(g(x))$ with 
\[g(x) = \sin(x)\ln(x) \quad \text{and} \quad f(x) = e^x.\]
Then to find $g'(x)$ we us the product rule and we get $g'(x) = \frac{\sin(x)}{x} + \cos(x)\ln(x)$.
Next $f'(x) = e^x$ and 
hence $f'(g(x)) = e^{g(x)} = e^{\sin(x)\ln(x)} = x^{\sin(x)}$.
We can then conclude $h'(x) = f'(g(x))g'(x) = x^{\sin(x)} \left[ \frac{\sin(x)}{x} + \cos(x)\ln(x)\right]$.
\end{example}

%more question formats below













%\begin{verbatim}
\begin{question}
What is your favorite color?
\begin{multipleChoice}
\choice[correct]{Rainbow}
\choice{Blue}
\choice{Green}
\choice{Red}
\end{multipleChoice}
\begin{freeResponse}
Hello
\end{freeResponse}
\end{question}
%\end{verbatim}





\begin{question}
  Which one will you choose?
  \begin{multipleChoice}
    \choice[correct]{I'm correct.}
    \choice{I'm wrong.}
    \choice{I'm wrong too.}
  \end{multipleChoice}
\end{question}


\begin{question}
  Which one will you choose?
  \begin{selectAll}
    \choice[correct]{I'm correct.}
    \choice{I'm wrong.}
    \choice[correct]{I'm also correct.}
    \choice{I'm wrong too.}
  \end{selectAll}
\end{question}


\begin{freeResponse}
What is the chain rule used for?
\end{freeResponse}
