\documentclass[handout]{ximera}

%% You can put user macros here
%% However, you cannot make new environments



\newcommand{\ffrac}[2]{\frac{\text{\footnotesize $#1$}}{\text{\footnotesize $#2$}}}
\newcommand{\vasymptote}[2][]{
    \draw [densely dashed,#1] ({rel axis cs:0,0} -| {axis cs:#2,0}) -- ({rel axis cs:0,1} -| {axis cs:#2,0});
}


\graphicspath{{./}{firstExample/}}

\usepackage{amsmath}
\usepackage{amssymb}
\usepackage{array}
\usepackage[makeroom]{cancel} %% for strike outs
\usepackage{pgffor} %% required for integral for loops
\usepackage{tikz}
\usepackage{tikz-cd}
\usepackage{tkz-euclide}
\usetikzlibrary{shapes.multipart}


\usetkzobj{all}
\tikzstyle geometryDiagrams=[ultra thick,color=blue!50!black]


\usetikzlibrary{arrows}
\tikzset{>=stealth,commutative diagrams/.cd,
  arrow style=tikz,diagrams={>=stealth}} %% cool arrow head
\tikzset{shorten <>/.style={ shorten >=#1, shorten <=#1 } } %% allows shorter vectors

\usetikzlibrary{backgrounds} %% for boxes around graphs
\usetikzlibrary{shapes,positioning}  %% Clouds and stars
\usetikzlibrary{matrix} %% for matrix
\usepgfplotslibrary{polar} %% for polar plots
\usepgfplotslibrary{fillbetween} %% to shade area between curves in TikZ



%\usepackage[width=4.375in, height=7.0in, top=1.0in, papersize={5.5in,8.5in}]{geometry}
%\usepackage[pdftex]{graphicx}
%\usepackage{tipa}
%\usepackage{txfonts}
%\usepackage{textcomp}
%\usepackage{amsthm}
%\usepackage{xy}
%\usepackage{fancyhdr}
%\usepackage{xcolor}
%\usepackage{mathtools} %% for pretty underbrace % Breaks Ximera
%\usepackage{multicol}



\newcommand{\RR}{\mathbb R}
\newcommand{\R}{\mathbb R}
\newcommand{\C}{\mathbb C}
\newcommand{\N}{\mathbb N}
\newcommand{\Z}{\mathbb Z}
\newcommand{\dis}{\displaystyle}
%\renewcommand{\d}{\,d\!}
\renewcommand{\d}{\mathop{}\!d}
\newcommand{\dd}[2][]{\frac{\d #1}{\d #2}}
\newcommand{\pp}[2][]{\frac{\partial #1}{\partial #2}}
\renewcommand{\l}{\ell}
\newcommand{\ddx}{\frac{d}{\d x}}

\newcommand{\zeroOverZero}{\ensuremath{\boldsymbol{\tfrac{0}{0}}}}
\newcommand{\inftyOverInfty}{\ensuremath{\boldsymbol{\tfrac{\infty}{\infty}}}}
\newcommand{\zeroOverInfty}{\ensuremath{\boldsymbol{\tfrac{0}{\infty}}}}
\newcommand{\zeroTimesInfty}{\ensuremath{\small\boldsymbol{0\cdot \infty}}}
\newcommand{\inftyMinusInfty}{\ensuremath{\small\boldsymbol{\infty - \infty}}}
\newcommand{\oneToInfty}{\ensuremath{\boldsymbol{1^\infty}}}
\newcommand{\zeroToZero}{\ensuremath{\boldsymbol{0^0}}}
\newcommand{\inftyToZero}{\ensuremath{\boldsymbol{\infty^0}}}


\newcommand{\numOverZero}{\ensuremath{\boldsymbol{\tfrac{\#}{0}}}}
\newcommand{\dfn}{\textbf}
%\newcommand{\unit}{\,\mathrm}
\newcommand{\unit}{\mathop{}\!\mathrm}
%\newcommand{\eval}[1]{\bigg[ #1 \bigg]}
\newcommand{\eval}[1]{ #1 \bigg|}
\newcommand{\seq}[1]{\left( #1 \right)}
\renewcommand{\epsilon}{\varepsilon}
\renewcommand{\iff}{\Leftrightarrow}

\DeclareMathOperator{\arccot}{arccot}
\DeclareMathOperator{\arcsec}{arcsec}
\DeclareMathOperator{\arccsc}{arccsc}
\DeclareMathOperator{\si}{Si}
\DeclareMathOperator{\proj}{proj}
\DeclareMathOperator{\scal}{scal}
\DeclareMathOperator{\cis}{cis}
\DeclareMathOperator{\Arg}{Arg}
%\DeclareMathOperator{\arg}{arg}
\DeclareMathOperator{\Rep}{Re}
\DeclareMathOperator{\Imp}{Im}
\DeclareMathOperator{\sech}{sech}
\DeclareMathOperator{\csch}{csch}
\DeclareMathOperator{\Log}{Log}

\newcommand{\tightoverset}[2]{% for arrow vec
  \mathop{#2}\limits^{\vbox to -.5ex{\kern-0.75ex\hbox{$#1$}\vss}}}
\newcommand{\arrowvec}{\overrightarrow}
\renewcommand{\vec}{\mathbf}
\newcommand{\veci}{{\boldsymbol{\hat{\imath}}}}
\newcommand{\vecj}{{\boldsymbol{\hat{\jmath}}}}
\newcommand{\veck}{{\boldsymbol{\hat{k}}}}
\newcommand{\vecl}{\boldsymbol{\l}}
\newcommand{\utan}{\vec{\hat{t}}}
\newcommand{\unormal}{\vec{\hat{n}}}
\newcommand{\ubinormal}{\vec{\hat{b}}}

\newcommand{\dotp}{\bullet}
\newcommand{\cross}{\boldsymbol\times}
\newcommand{\grad}{\boldsymbol\nabla}
\newcommand{\divergence}{\grad\dotp}
\newcommand{\curl}{\grad\cross}
%% Simple horiz vectors
\renewcommand{\vector}[1]{\left\langle #1\right\rangle}


\outcome{Use the DCT to determine the behavior of an infinite series}

\title{3.5 Direct Comparison Test}

\begin{document}

\begin{abstract}
We will use the DCT to determine if an infinite series converges or diverges.
\end{abstract}

\maketitle

\section{Direct Comparison Test}

In this section, we will determine whether a given series converges or diverges by comparing it to a series whose behavior is known.
Note that we will only be working with series of positive terms in this section. There are two ideas behind the Direct Comparison Test (DCT).
The first is that if the (positive) terms of one series are less than the corresponding terms of another, 
convergent series, then the original series must also converge. And conversely, if the terms of one series are greater than 
the terms of another, divergent series, then the original series must diverge also.
In short, the DCT can be summarized by saying ``less than a convergent series is convergent''
and ``greater than a divergent series is divergent.''
We state this formally in the theorem below, in which the original series is denoted by $\displaystyle{\sum_{n=0}^\infty a_n}$ 
and the second series whose behavior is known is denoted by $\displaystyle{\sum_{n=0}^\infty b_n}$.


\begin{theorem}[Direct Comparison Test]
 Suppose $a_n \geq 0$ and $b_n \geq 0$ for all values of $n$.\\
(Case 1: Convergence) If $\; \displaystyle{\sum_{n=0}^\infty b_n}$ converges, \textbf{and} if $a_n \leq b_n$ for all values of $n$,  
then the series $\displaystyle{\sum_{n=0}^\infty a_n}$ also converges.\\
(Case 2: Divergence) If $\; \displaystyle{\sum_{n=0}^\infty b_n}$ diverges, \textbf{and} if $a_n \geq b_n$ for all values of $n$,  
then the series $\displaystyle{\sum_{n=0}^\infty a_n}$ also diverges.
\end{theorem}


\begin{remark}
If \; $a_n \geq b_n \geq 0$ for all $n$ and $\displaystyle{\sum_{n=0}^\infty b_n}$ converges, then the DCT gives \textbf{no conclusion} about $\displaystyle{\sum_{n=0}^\infty a_n}$.
Also, if \; $0 \leq a_n \leq b_n$ for all $n$ and $\displaystyle{\sum_{n=0}^\infty b_n}$ diverges, then the DCT gives \textbf{no conclusion} about $\displaystyle{\sum_{n=0}^\infty a_n}$.
\end{remark}

\begin{remark}
The qualifying statement \textbf{for all $n$} associated with the
 conditions $a_n \leq b_n$ and $ a_n \geq b_n$ can be relaxed to
  \textbf{forall $n$ sufficiently large}
\end{remark}



\begin{example}[example 1] %example #1
Determine if the series 
\[
\sum_{n=1}^\infty \frac{1}{2n^2 + 3}
\]
converges or diverges.\\
We will use the DCT with the series 
\[
\sum_{n=1}^\infty \frac{1}{n^2}.
\] 
This is a $p$-series with $p=2 >1$ and so it converges.
To make the comparison, first note that 
\[
2n^2 + 3 > n^2 > 0
\]
 for all $n \geq 1$. Now we will take advantage of the fact that for positive numbers, $m$ and $n$ with $m>n$ we have 
 \[
 \frac{1}{m} < \frac{1}{n}.
 \]
 In other words, taking reciprocals reverses the sense of the inequality. In this case, taking reciprocals yields 
\[
\frac{1}{2n^2 + 3} < \frac{1}{n^2} \; \; \text{ for } \; \; n \geq 1,
\]
and so by the DCT (case 1), 
\[
\sum_{n=1}^\infty \frac{1}{2n^2 + 3}
\]
 also converges.
\end{example}


\begin{problem}(problem 1)
Consider the series $\displaystyle{\sum_{n=1}^\infty \frac{1}{n^3 + 1}}$.\\
Which series should we compare this to?
\begin{center}
\begin{multipleChoice}
\choice{\[\sum_{n=1}^\infty \frac{1}{n^2}\]}
\choice[correct]{\[\sum_{n=1}^\infty \frac{1}{n^3}\]}
\choice{\[\sum_{n=1}^\infty \frac{1}{3^n}\]}
\end{multipleChoice}
\end{center}

Which way does the comparison go?
\begin{center}
\begin{multipleChoice}
\choice[correct]{ $\displaystyle{\frac{1}{n^3 + 1} \leq \frac{1}{n^3}}$ for $n \geq 1$}
\choice{$\displaystyle{\frac{1}{n^3 + 1} \geq \frac{1}{n^3}}$ for $n \geq 1$}
\end{multipleChoice}
\end{center}

Describe the behavior of the series $\displaystyle{\sum_{n=1}^\infty \frac{1}{n^3 + 1}:}$
\begin{center}
\begin{multipleChoice}
\choice[correct]{Converges by DCT}
\choice{Diverges by DCT}
\choice{No Conclusion from DCT}
\end{multipleChoice}
\end{center}

\end{problem}


\begin{example}[example 2] %example #2
Determine if the series 
\[
\sum_{n=0}^\infty \frac{3^n}{4^n + 1}
\]
 converges or diverges.\\
We will use the DCT with the series 
\[
\sum_{n=0}^\infty \left(\frac34\right)^n.
\]
 This is a geometric series with common ratio, $r = 3/4$. Since $-1 < \frac 34 < 1$, this series converges.
To make the comparison, first note that $4^n + 1 > 4^n > 0$ for all $n \geq 0$. By taking the reciprocal of both sides,
we have 
\[
\frac{1}{4^n + 1} < \frac{1}{4^n} \; \; \text{ for } \;\; n \geq 0,
\]
and multiplying by $3^n$ we get
\[
\frac{3^n}{4^n + 1} < \frac{3^n}{4^n} = \left(\frac34\right)^n \;\; \text{ for } \;\; n \geq 0.
\]
Therefore, by the DCT (case 1), 
\[
\sum_{n=0}^\infty \frac{3^n}{4^n + 1}
\]
 also converges.
\end{example}


\begin{problem}(problem 2)
Consider the series 
\[
\sum_{n=0}^\infty \frac{1}{3^n + 1}.
\]

Which series should we compare this to?

\begin{multipleChoice}
\choice{$\sum_{n=0}^\infty \frac{1}{2^n}$}
\choice{$\sum_{n=1}^\infty \frac{1}{n^3}$}
\choice[correct]{$\sum_{n=0}^\infty \frac{1}{3^n}$}
\end{multipleChoice}

Which way does the comparison go?
\begin{multipleChoice}
\choice[correct]{ $\frac{1}{3^n + 1} \leq \frac{1}{3^n}$ for $n \geq 0$}
\choice{$\frac{1}{3^n + 1} \geq \frac{1}{3^n}$ for $n \geq 0$}
\end{multipleChoice}

Describe the behavior of the series 
\[
\sum_{n=0}^\infty \frac{1}{3^n + 1}:
\]

\begin{multipleChoice}
\choice[correct]{Converges by DCT}
\choice{Diverges by DCT}
\choice{No Conclusion from DCT}
\end{multipleChoice}

\end{problem}



\begin{example}[example 3] %example #3
Determine if the series 
\[
\sum_{n=1}^\infty \frac{\sin^2(n)}{n^3}
\]
 converges or diverges.\\
We will use the DCT with the series 
\[
\sum_{n=1}^\infty \frac{1}{n^3}.
\]
This is a $p$-series with $p=3 >1$ and so it converges.
To make the comparison, first note that $0 \leq \sin^2(n) \leq 1$ for all $n \geq 1$. Dividing by $n^3$,
we have 
\[
\frac{\sin^2(n)}{n^3} \leq \frac{1}{n^3} \; \; \text{ for } \; \; n \geq 1,
\]
and so by the DCT (case 1), 
\[
\sum_{n=1}^\infty \frac{\sin^2(n)}{n^3}
\]
 also converges.
\end{example}




\begin{problem}(problem 3)
Consider the series $\sum_{n=0}^\infty \frac{\cos^2(n)}{2^n + 1}$.
Which series should we compare this to?

\begin{multipleChoice}
\choice[correct]{$\sum_{n=0}^\infty \frac{1}{2^n}$}
\choice{$\sum_{n=0}^\infty \frac{\cos^2(n)}{2^n}$}
\choice{$\sum_{n=0}^\infty \frac{1}{3^n}$}
\end{multipleChoice}

Which way does the comparison go?
\begin{multipleChoice}
\choice[correct]{ $\frac{\cos^2(n)}{2^n + 1}\leq \frac{1}{2^n}$ for $n \geq 0$}
\choice{$\frac{\cos^2(n)}{2^n + 1} \geq \frac{1}{2^n}$ for $n \geq 0$}
\end{multipleChoice}

Describe the behavior of the series $\sum_{n=0}^\infty {\cos^2(n)}{2^n + 1}:$
\begin{multipleChoice}
\choice[correct]{Converges by DCT}
\choice{Diverges by DCT}
\choice{No Conclusion from DCT}
\end{multipleChoice}

\end{problem}




\begin{example}[example 4] %example #4
Determine if the series 
\[
\sum_{n=1}^\infty \frac{\cos^4(n)}{n}
\]
 converges or diverges.\\
We will use the DCT with the divergent harmonic series 
\[
\sum_{n=1}^\infty \frac{1}{n}.
\]
 
To make the comparison, first note that $0 \leq \cos^4(n) \leq 1$ for all $n \geq 1$. Dividing by $n$,
we have 
\[
\frac{\cos^4(n)}{n} \leq \frac{1}{n} \; \; \text{ for } \; \; n \geq 1.
\]
Noting that this comparison goes in the wrong direction (smaller than a divergent), the DCT gives \textbf{no conclusion}
about the behavior of the series $\sum_{n=1}^\infty \frac{\cos^4(n)}{n}$.
\end{example}




\begin{problem}(problem 4)
Consider the series $\displaystyle{\sum_{n=1}^\infty \frac{n^2}{n^3 + 1}}$.
Which series should we compare this to?

\begin{multipleChoice}
\choice[correct]{$\sum_{n=1}^\infty \frac{1}{n}$}
\choice{$\sum_{n=1}^\infty \frac{1}{n^2}$}
\choice{$\sum_{n=1}^\infty \frac{1}{n^3}$}
\end{multipleChoice}

Which way does the comparison go?
\begin{multipleChoice}
\choice[correct]{ $\frac{n^2}{n^3 + 1} \leq \frac{1}{n}$ for $n \geq 1$}
\choice{$\frac{n^2}{n^3 + 1} \geq \frac{1}{n}$ for $n \geq 1$}
\end{multipleChoice}

Describe the behavior of the series $\sum_{n=1}^\infty \frac{n^2}{n^3 + 1}:$
\begin{multipleChoice}
\choice{Converges by DCT}
\choice{Diverges by DCT}
\choice[correct]{No Conclusion from DCT}
\end{multipleChoice}

\end{problem}



\begin{example}[example 5]
Determine if the series 
\[
\sum_{n=2}^\infty \frac{1}{\sqrt{n^2 -1}}
\]
 converges or diverges.\\
We will use the DCT with the divergent harmonic series 
\[
\sum_{n=2}^\infty \frac{1}{n}.
\]

To make the comparison, first note that $0 < \sqrt{n^2 -1} < \sqrt{n^2} = n$ for all $n \geq 2$. By taking the reciprocal of both sides,
we have 
\[
\frac{1}{\sqrt{n^2 -1}} > \frac{1}{n},
\]
and so by the DCT (case 2), $\sum_{n=2}^\infty \frac{1}{\sqrt{n^2 -1}}$ also diverges.
\end{example}


\begin{problem}(problem 5)
Consider the series $\displaystyle{\sum_{n=1}^\infty \frac{n+3}{n^2}}$.
Which series should we compare this to?

\begin{multipleChoice}
\choice[correct]{$\sum_{n=1}^\infty \frac{1}{n}$}
\choice{$\sum_{n=1}^\infty \frac{1}{n^2}$}
\choice{$\sum_{n=1}^\infty \frac{3}{n^2}$}
\end{multipleChoice}

Which way does the comparison go?
\begin{multipleChoice}
\choice{ $\frac{n+3}{n^2} \leq \frac{1}{n}$ for $n \geq 1$}
\choice[correct]{$\frac{n+3}{n^2} \geq \frac{1}{n}$ for $n \geq 1$}
\end{multipleChoice}

Describe the behavior of the series $\sum_{n=1}^\infty \frac{n+3}{n^2}:$
\begin{multipleChoice}
\choice{Converges by DCT}
\choice[correct]{Diverges by DCT}
\choice{No Conclusion from DCT}
\end{multipleChoice}

\end{problem}



\begin{example}[example 6] %example #6
Determine if the series 
\[
\sum_{n=0}^\infty \frac{3^n + 5}{2^n}
\]
 converges or diverges.\\
We will use the DCT with the divergent geometric series 
\[
\sum_{n=0}^\infty \left(\frac32\right)^n.
\]

To make the comparison, first note that $3^n + 5 > 3^n > 0$ for all $n \geq 0$. Dividing by $2^n$, we have 
\[
\frac{3^n + 5}{2^n} > \frac{3^n}{2^n} = \left(\frac32\right)^n \;\; \text{ for } \;\; n \geq 0,
\]
and so by the DCT (case 2), $\sum_{n=0}^\infty \frac{2^n + 5}{3^n}$ also diverges.
\end{example}


\begin{problem}(problem 6)
Consider the series $\displaystyle{\sum_{n=0}^\infty \frac{5^n + 4}{3^n}}$.
Which series should we compare this to?

\begin{multipleChoice}
\choice{$\sum_{n=0}^\infty \left(\frac43\right)^n$}
\choice[correct]{$\sum_{n=0}^\infty \left(\frac53\right)^n$}
\choice{$\sum_{n=0}^\infty \left(\frac13\right)^n$}
\end{multipleChoice}

Which way does the comparison go?
\begin{multipleChoice}
\choice{ $\frac{5^n + 4}{3^n} \leq \left(\frac53\right)^n$ for $n \geq 0$}
\choice[correct]{$\frac{5^n + 4}{3^n}\geq \left(\frac53\right)^n$ for $n \geq 0$}
\end{multipleChoice}

Describe the behavior of the series $\sum_{n=0}^\infty \frac{5^n + 4}{3^n}:$
\begin{multipleChoice}
\choice{Converges by DCT}
\choice[correct]{Diverges by DCT}
\choice{No Conclusion from DCT}
\end{multipleChoice}

\end{problem}



\begin{example}[example 7] %example #7
Determine if the series 
\[
\sum_{n=2}^\infty \frac{1}{\ln(n)}
\]
converges or diverges.\\
We will use the DCT with the series 
\[
\sum_{n=2}^\infty \frac{1}{n\ln(n)}
\]
 which diverges (as shown in the section on the Integral Test).
To make the comparison, first note that $0 < \ln(n) < n\ln(n)$ for all $n \geq 2$. Taking reciprocals, we have 
\[
\frac{1}{\ln(n)} > \frac{1}{n\ln(n)}  \;\; \text{ for } \;\; n \geq 2,
\]
and so by the DCT (case 2), $\sum_{n=2}^\infty \frac{1}{\ln(n)}$ also diverges.
\end{example}


\begin{example}[example 8] %example #8
Determine if the series 
\[
\sum_{n=1}^\infty \frac{n+2}{n^3}
\]
 converges or diverges.\\
We will use the DCT with the convergent p-series 
\[
\sum_{n=1}^\infty \frac{1}{n^2}.
\]

To make the comparison, first note that $ n+2 > n$ for all $n \geq 1$. Dividing by $n^3$,
we have 
\[
\frac{n+2}{n^3} > \frac{n}{n^3} = \frac{1}{n^2} \; \; \text{ for } \; \; n \geq 1.
\]
Noting that this comparison goes in the wrong direction (larger than a convergent), the DCT gives \textbf{no conclusion}
about the behavior of the series $\sum_{n=1}^\infty \frac{n+2}{n^3}$. In the next section on the Limit Comparison Test, 
we will learn a technique that allows us to make a conclusion about this series by means of a slightly different type of comparison.
\end{example}



\begin{problem}(problem 8)
Consider the series $\displaystyle{\sum_{n=1}^\infty \frac{n^3 +3}{n^6}}$.
Which series should we compare this to?

\begin{multipleChoice}
\choice{$\sum_{n=1}^\infty \frac{1}{n^2}$}
\choice[correct]{$\sum_{n=1}^\infty \frac{1}{n^3}$}
\choice{$\sum_{n=1}^\infty \frac{3}{n^6}$}
\end{multipleChoice}

Which way does the comparison go?
\begin{multipleChoice}
\choice{ $\frac{n^3 +3}{n^6} \leq \frac{1}{n^3}$ for $n \geq 1$}
\choice[correct]{$\frac{n^3+3}{n^6} \geq \frac{1}{n^3}$ for $n \geq 1$}
\end{multipleChoice}

Describe the behavior of the series $\sum_{n=1}^\infty \frac{n^3+3}{n^6}:$
\begin{multipleChoice}
\choice{Converges by DCT}
\choice{Diverges by DCT}
\choice[correct]{No Conclusion from DCT}
\end{multipleChoice}

\end{problem}





\section{Video Lesson}

\begin{center}
\begin{foldable}
\unfoldable{Here is a detailed, lecture style video on the Direct Comparison Test:}
\youtube{2IQC8-9HP1w}
\end{foldable}
\end{center}


\section{More Problems}


\begin{problem}(problem 9)
Consider the series $\displaystyle{\sum_{n=0}^\infty \frac{4^n + 1}{2^n}}$.
Which series should we compare this to?

\begin{multipleChoice}
\choice[correct]{$\sum_{n=0}^\infty 2^n$}
\choice{$\sum_{n=0}^\infty \left(\frac12\right)^n$}
\choice{$\sum_{n=0}^\infty 4^n$}
\end{multipleChoice}

Which way does the comparison go?
\begin{multipleChoice}
\choice{ $\frac{4^n + 1}{2^n} \leq 2^n$ for $n \geq 0$}
\choice[correct]{$\frac{4^n + 1}{2^n}\geq 2^n$ for $n \geq 0$}
\end{multipleChoice}

Describe the behavior of the series $\sum_{n=0}^\infty \frac{4^n + 1}{2^n}:$
\begin{multipleChoice}
\choice{Converges by DCT}
\choice[correct]{Diverges by DCT}
\choice{No Conclusion from DCT}
\end{multipleChoice}

\end{problem}




\begin{problem}(problem 10)
Consider the series $\displaystyle{\sum_{n=0}^\infty \frac{\sin^8(n)}{4^n + 1}}$.
Which series should we compare this to?

\begin{multipleChoice}
\choice{$\sum_{n=0}^\infty \frac{1}{8^n}$}
\choice{$\sum_{n=0}^\infty \frac{\sin^8(n)}{4^n}$}
\choice[correct]{$\sum_{n=0}^\infty \frac{1}{4^n}$}
\end{multipleChoice}

Which way does the comparison go?
\begin{multipleChoice}
\choice[correct]{ $\frac{\sin^8(n)}{4^n + 1}\leq \frac{1}{4^n}$ for $n \geq 0$}
\choice{$\frac{\sin^8(n)}{4^n + 1} \geq \frac{1}{4^n}$ for $n \geq 0$}
\end{multipleChoice}

Describe the behavior of the series $\sum_{n=0}^\infty \frac{\sin^8(n)}{4^n + 1}:$
\begin{multipleChoice}
\choice[correct]{Converges by DCT}
\choice{Diverges by DCT}
\choice{No Conclusion from DCT}
\end{multipleChoice}

\end{problem}




\begin{problem}(problem 11)
Consider the series $\displaystyle{\sum_{n=0}^\infty \frac{1}{2^n + 3}}$.
Which series should we compare this to?

\begin{multipleChoice}
\choice[correct]{$\sum_{n=0}^\infty \frac{1}{2^n}$}
\choice{$\sum_{n=1}^\infty \frac{1}{n^3}$}
\choice{$\sum_{n=0}^\infty \frac{1}{3^n}$}
\end{multipleChoice}

Which way does the comparison go?
\begin{multipleChoice}
\choice[correct]{ $\frac{1}{2^n + 3} \leq \frac{1}{2^n}$ for $n \geq 0$}
\choice{$\frac{1}{2^n + 3} \geq \frac{1}{2^n}$ for $n \geq 0$}
\end{multipleChoice}

Describe the behavior of the series $\sum_{n=0}^\infty \frac{1}{2^n + 3}:$
\begin{multipleChoice}
\choice[correct]{Converges by DCT}
\choice{Diverges by DCT}
\choice{No Conclusion from DCT}
\end{multipleChoice}

\end{problem}





\begin{problem}(problem 12)
Consider the series $\displaystyle{\sum_{n=1}^\infty \frac{n^2 +1}{n^6}}$.
Which series should we compare this to?

\begin{multipleChoice}
\choice{$\sum_{n=1}^\infty \frac{1}{n^2}$}
\choice{$\sum_{n=1}^\infty \frac{1}{n^3}$}
\choice[correct]{$\sum_{n=1}^\infty \frac{1}{n^4}$}
\end{multipleChoice}

Which way does the comparison go?
\begin{multipleChoice}
\choice{ $\frac{n^2 +1}{n^6} \leq \frac{1}{n^4}$ for $n \geq 1$}
\choice[correct]{$\frac{n^2+1}{n^6} \geq \frac{1}{n^4}$ for $n \geq 1$}
\end{multipleChoice}

Describe the behavior of the series $\sum_{n=1}^\infty \frac{n^2+1}{n^6}:$
\begin{multipleChoice}
\choice{Converges by DCT}
\choice{Diverges by DCT}
\choice[correct]{No Conclusion from DCT}
\end{multipleChoice}

\end{problem}



\begin{problem}(problem 13)
Consider the series $\displaystyle{\sum_{n=1}^\infty \frac{n^3+4}{n^4}}$.
Which series should we compare this to?

\begin{multipleChoice}
\choice[correct]{$\sum_{n=1}^\infty \frac{1}{n}$}
\choice{$\sum_{n=1}^\infty \frac{4}{n^3}$}
\choice{$\sum_{n=1}^\infty \frac{1}{n^4}$}
\end{multipleChoice}

Which way does the comparison go?
\begin{multipleChoice}
\choice{ $\frac{n^ + 4}{n^4} \leq \frac{1}{n}$ for $n \geq 1$}
\choice[correct]{$\frac{n^3+4}{n^4} \geq \frac{1}{n}$ for $n \geq 1$}
\end{multipleChoice}

Describe the behavior of the series $\sum_{n=1}^\infty \frac{n^3 + 4}{n^4}:$
\begin{multipleChoice}
\choice{Converges by DCT}
\choice[correct]{Diverges by DCT}
\choice{No Conclusion from DCT}
\end{multipleChoice}

\end{problem}





\begin{problem}(problem 14)
Consider the series $\displaystyle{\sum_{n=1}^\infty \frac{n^5}{n^6 + 4}}$.
Which series should we compare this to?

\begin{multipleChoice}
\choice[correct]{$\sum_{n=1}^\infty \frac{1}{n}$}
\choice{$\sum_{n=1}^\infty \frac{1}{n^4}$}
\choice{$\sum_{n=1}^\infty \frac{1}{n^6}$}
\end{multipleChoice}

Which way does the comparison go?
\begin{multipleChoice}
\choice[correct]{ $\frac{n^5}{n^6 + 4} \leq \frac{1}{n}$ for $n \geq 1$}
\choice{$\frac{n^5}{n^6 + 4} \geq \frac{1}{n}$ for $n \geq 1$}
\end{multipleChoice}

Describe the behavior of the series $\sum_{n=1}^\infty \frac{n^5}{n^6 + 2}:$
\begin{multipleChoice}
\choice{Converges by DCT}
\choice{Diverges by DCT}
\choice[correct]{No Conclusion from DCT}
\end{multipleChoice}

\end{problem}






\begin{problem}(problem 15)
Consider the series $\displaystyle{\sum_{n=1}^\infty \frac{n}{n^3 + 1}}$.
Which series should we compare this to?

\begin{multipleChoice}
\choice[correct]{$\sum_{n=1}^\infty \frac{1}{n^2}$}
\choice{$\sum_{n=1}^\infty \frac{1}{n^3}$}
\choice{$\sum_{n=1}^\infty \frac{1}{3^n}$}
\end{multipleChoice}

Which way does the comparison go?
\begin{multipleChoice}
\choice[correct]{ $\frac{n}{n^3 + 1} \leq \frac{1}{n^2}$ for $n \geq 1$}
\choice{$\frac{n}{n^3 + 1} \geq \frac{1}{n^2}$ for $n \geq 1$}
\end{multipleChoice}

Describe the behavior of the series $\sum_{n=1}^\infty \frac{n}{n^3 + 1}:$
\begin{multipleChoice}
\choice[correct]{Converges by DCT}
\choice{Diverges by DCT}
\choice{No Conclusion from DCT}
\end{multipleChoice}

\end{problem}





\end{document}


\begin{example} %example #15
Find $h'(x)$ if $h(x) = x^{\sin(x)}$.\\
We will use the fact that the exponential and logarithm functions are inverses,
\[e^{\ln(x)} = x,\]
and the exponent property of logarithms, 
\[\ln(x^n) = n\ln(x),\]
to rewrite $h(x)$.  We have 
\[h(x) = x^{\sin(x)} = e^{\ln(x^{\sin(x)})} = e^{\sin(x)\ln(x)}\]
and we can now compute $h'(x)$ using a combination of the chain rule and product rule.
We can write $h(x)$ as a composition, $f(g(x))$ with 
\[g(x) = \sin(x)\ln(x) \quad \text{and} \quad f(x) = e^x.\]
Then to find $g'(x)$ we us the product rule and we get $g'(x) = \frac{\sin(x)}{x} + \cos(x)\ln(x)$.
Next $f'(x) = e^x$ and 
hence $f'(g(x)) = e^{g(x)} = e^{\sin(x)\ln(x)} = x^{\sin(x)}$.
We can then conclude $h'(x) = f'(g(x))g'(x) = x^{\sin(x)} \left[ \frac{\sin(x)}{x} + \cos(x)\ln(x)\right]$.
\end{example}

%more question formats below













%\begin{verbatim}
\begin{question}
What is your favorite color?
\begin{multipleChoice}
\choice[correct]{Rainbow}
\choice{Blue}
\choice{Green}
\choice{Red}
\end{multipleChoice}
\begin{freeResponse}
Hello
\end{freeResponse}
\end{question}
%\end{verbatim}





\begin{question}
  Which one will you choose?
  \begin{multipleChoice}
    \choice[correct]{I'm correct.}
    \choice{I'm wrong.}
    \choice{I'm wrong too.}
  \end{multipleChoice}
\end{question}


\begin{question}
  Which one will you choose?
  \begin{selectAll}
    \choice[correct]{I'm correct.}
    \choice{I'm wrong.}
    \choice[correct]{I'm also correct.}
    \choice{I'm wrong too.}
  \end{selectAll}
\end{question}


\begin{freeResponse}
What is the chain rule used for?
\end{freeResponse}
