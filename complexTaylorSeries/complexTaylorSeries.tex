\documentclass[handout]{ximera}

%% You can put user macros here
%% However, you cannot make new environments



\newcommand{\ffrac}[2]{\frac{\text{\footnotesize $#1$}}{\text{\footnotesize $#2$}}}
\newcommand{\vasymptote}[2][]{
    \draw [densely dashed,#1] ({rel axis cs:0,0} -| {axis cs:#2,0}) -- ({rel axis cs:0,1} -| {axis cs:#2,0});
}


\graphicspath{{./}{firstExample/}}

\usepackage{amsmath}
\usepackage{amssymb}
\usepackage{array}
\usepackage[makeroom]{cancel} %% for strike outs
\usepackage{pgffor} %% required for integral for loops
\usepackage{tikz}
\usepackage{tikz-cd}
\usepackage{tkz-euclide}
\usetikzlibrary{shapes.multipart}


\usetkzobj{all}
\tikzstyle geometryDiagrams=[ultra thick,color=blue!50!black]


\usetikzlibrary{arrows}
\tikzset{>=stealth,commutative diagrams/.cd,
  arrow style=tikz,diagrams={>=stealth}} %% cool arrow head
\tikzset{shorten <>/.style={ shorten >=#1, shorten <=#1 } } %% allows shorter vectors

\usetikzlibrary{backgrounds} %% for boxes around graphs
\usetikzlibrary{shapes,positioning}  %% Clouds and stars
\usetikzlibrary{matrix} %% for matrix
\usepgfplotslibrary{polar} %% for polar plots
\usepgfplotslibrary{fillbetween} %% to shade area between curves in TikZ



%\usepackage[width=4.375in, height=7.0in, top=1.0in, papersize={5.5in,8.5in}]{geometry}
%\usepackage[pdftex]{graphicx}
%\usepackage{tipa}
%\usepackage{txfonts}
%\usepackage{textcomp}
%\usepackage{amsthm}
%\usepackage{xy}
%\usepackage{fancyhdr}
%\usepackage{xcolor}
%\usepackage{mathtools} %% for pretty underbrace % Breaks Ximera
%\usepackage{multicol}



\newcommand{\RR}{\mathbb R}
\newcommand{\R}{\mathbb R}
\newcommand{\C}{\mathbb C}
\newcommand{\N}{\mathbb N}
\newcommand{\Z}{\mathbb Z}
\newcommand{\dis}{\displaystyle}
%\renewcommand{\d}{\,d\!}
\renewcommand{\d}{\mathop{}\!d}
\newcommand{\dd}[2][]{\frac{\d #1}{\d #2}}
\newcommand{\pp}[2][]{\frac{\partial #1}{\partial #2}}
\renewcommand{\l}{\ell}
\newcommand{\ddx}{\frac{d}{\d x}}

\newcommand{\zeroOverZero}{\ensuremath{\boldsymbol{\tfrac{0}{0}}}}
\newcommand{\inftyOverInfty}{\ensuremath{\boldsymbol{\tfrac{\infty}{\infty}}}}
\newcommand{\zeroOverInfty}{\ensuremath{\boldsymbol{\tfrac{0}{\infty}}}}
\newcommand{\zeroTimesInfty}{\ensuremath{\small\boldsymbol{0\cdot \infty}}}
\newcommand{\inftyMinusInfty}{\ensuremath{\small\boldsymbol{\infty - \infty}}}
\newcommand{\oneToInfty}{\ensuremath{\boldsymbol{1^\infty}}}
\newcommand{\zeroToZero}{\ensuremath{\boldsymbol{0^0}}}
\newcommand{\inftyToZero}{\ensuremath{\boldsymbol{\infty^0}}}


\newcommand{\numOverZero}{\ensuremath{\boldsymbol{\tfrac{\#}{0}}}}
\newcommand{\dfn}{\textbf}
%\newcommand{\unit}{\,\mathrm}
\newcommand{\unit}{\mathop{}\!\mathrm}
%\newcommand{\eval}[1]{\bigg[ #1 \bigg]}
\newcommand{\eval}[1]{ #1 \bigg|}
\newcommand{\seq}[1]{\left( #1 \right)}
\renewcommand{\epsilon}{\varepsilon}
\renewcommand{\iff}{\Leftrightarrow}

\DeclareMathOperator{\arccot}{arccot}
\DeclareMathOperator{\arcsec}{arcsec}
\DeclareMathOperator{\arccsc}{arccsc}
\DeclareMathOperator{\si}{Si}
\DeclareMathOperator{\proj}{proj}
\DeclareMathOperator{\scal}{scal}
\DeclareMathOperator{\cis}{cis}
\DeclareMathOperator{\Arg}{Arg}
%\DeclareMathOperator{\arg}{arg}
\DeclareMathOperator{\Rep}{Re}
\DeclareMathOperator{\Imp}{Im}
\DeclareMathOperator{\sech}{sech}
\DeclareMathOperator{\csch}{csch}
\DeclareMathOperator{\Log}{Log}

\newcommand{\tightoverset}[2]{% for arrow vec
  \mathop{#2}\limits^{\vbox to -.5ex{\kern-0.75ex\hbox{$#1$}\vss}}}
\newcommand{\arrowvec}{\overrightarrow}
\renewcommand{\vec}{\mathbf}
\newcommand{\veci}{{\boldsymbol{\hat{\imath}}}}
\newcommand{\vecj}{{\boldsymbol{\hat{\jmath}}}}
\newcommand{\veck}{{\boldsymbol{\hat{k}}}}
\newcommand{\vecl}{\boldsymbol{\l}}
\newcommand{\utan}{\vec{\hat{t}}}
\newcommand{\unormal}{\vec{\hat{n}}}
\newcommand{\ubinormal}{\vec{\hat{b}}}

\newcommand{\dotp}{\bullet}
\newcommand{\cross}{\boldsymbol\times}
\newcommand{\grad}{\boldsymbol\nabla}
\newcommand{\divergence}{\grad\dotp}
\newcommand{\curl}{\grad\cross}
%% Simple horiz vectors
\renewcommand{\vector}[1]{\left\langle #1\right\rangle}


\pgfplotsset{compat=1.13}

\outcome{Determine Taylor series of an analytic function}

\title{4.4 Complex Taylor Series}

\begin{document}

\begin{abstract}
We determine the Taylor series for a given analytic function.
\end{abstract}

\maketitle


\begin{definition} 
Suppose $f$ is analytic at $z_0$. Then the complex power series
\[
\sum_{k=0}^\infty \frac{f^{(k)}(z_0)}{k!}(z-z_0)^k
\]
is called the Taylor series representation for $f$ centered at $z_0$.
\end{definition}

\begin{theorem}
Suppose $f$ is analytic at $z_0$. Then there exists $\rho >0$ such that for all $z \in D(z_0, \rho), \,f(z)$ is equal to its Taylor series representation centered at $z_0$.
\end{theorem}

\begin{example}[example 1]
Find the Taylor series representation for $f(z) = \sin(z)$ centered at the origin.\\
The derivatives of $\sin z$ evaluated at the origin form the sequence $\{0, 1, 0, -1,...\}$,
where this cycle of length four repeats indefinitely.
Hence the series is 
\[
\sin z = z - \frac{z^3}{3!} + \frac{z^5}{5!} - \dots
\]
which can be written as
\[
\sin z = \sum_{k=0}^\infty (-1)^k \frac{z^{2k+1}}{(2k+1)!}
\]
The ratio test shows that this series converges for all $z$.
When the center of the Taylor series is the origin, it is referred to as the Maclaurin series for the function.
\end{example}

\begin{problem}(problem 1a)
Find the Maclaurin series for $f(z) = e^z$ and find its radius of convergence.
\end{problem}

\begin{problem}(problem 1b)
Find the Maclaurin series for $f(z) = \cos(z)$ and find its radius of convergence.
\end{problem}

\begin{problem}(problem 1c)
Find the Taylor series for $f(z) = e^z$ centered at $\pi i$ and find its radius of convergence.
\end{problem}

\begin{problem}(problem 1d)
Find the Taylor series for $f(z) = \Log(z)$ centered at $i$ and find its radius of convergence.
\end{problem}


\begin{example}[example 2]
Find the Maclaurin series for $f(z) = z\sin(\pi z)$.\\
First note that $f(z)$ is an entire function, so it has a Maclaurin series representation.
Using the Maclaurin series for $\sin z$ gives
\[
z\sin(\pi z) = z\sum_{k=0}^\infty (-1)^k \frac{(\pi z)^{2k+1}}{(2k+1)!} = \sum_{k=0}^\infty (-1)^k \frac{\pi^{2k+1} z^{2k+2}}{(2k+1)!}
\]
The ratio test shows that this series converges for all $z$.
\end{example}

\begin{problem}(problem 2a)
Find the Maclaurin series for $f(z) = e^{-z^2}$ and find its radius of convergence.
\end{problem}

\begin{problem}(problem 2b)
Find the Maclaurin series for $f(z) = z\cos(\pi z)$ and find its radius of convergence.
\end{problem}

\begin{example}[example 3]
Find the first few terms of the Maclaurin series for $\tan z$.\\
Since $\tan z = \frac{\sin z}{\cos z}$ and $\cos 0 \neq 0$, $\tan z$ is analytic at the origin 
and has a Maclaurin series expansion:
\[
\tan z = \sum_{k=0}^\infty c_k z^k = c_0 + c_1 z + c_2 z^2 + \cdots
\]
Using the Maclaurin series of $\sin z$ and $\cos z$ and writing $\sin z = \tan z \cdot \cos z$ we have
\[
z-\frac{z^3}{3!} + \frac{z^5}{5!} - \cdots = \left(c_0 + c_1 z + c_2 z^2 + \cdots\right) \cdot \left(1-\frac{z^2}{2!} + \frac{z^4}{4!} - \cdots \right)
\]
Multiplying on the RHS and comparing coefficients leads to a system of equations:
\begin{align*}
0 & = c_0 \\
1 & = c_1 \\
0 & = c_2 -\frac{c_0}{2!} \\
-\frac{1}{3!} & =  c_3- \frac{c_1}{2!} \\
0 & = c_4 - \frac{c_2}{2!} +  \frac{c_0}{4!} \\
 \frac{1}{5!} & = c_5- \frac{c_3}{2!} +  \frac{c_1}{4!}\\
 & \text{etc.}
\end{align*}
Solving this system yields $c_0 = c_2 = c_4 = 0$, etc., and  $c_1 = 1, c_3 = \frac13, c_5 = \frac{2}{15}$ etc.
Hence 
\[
\tan z = z + \frac13 z + \frac{2}{15}z^3 + \cdots
\]
\end{example}

\begin{problem}(problem 3)
Find the first few terms of the Maclaurin series for $\sec z$.
\end{problem}

\end{document}












