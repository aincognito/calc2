\documentclass{ximera}

%% You can put user macros here
%% However, you cannot make new environments



\newcommand{\ffrac}[2]{\frac{\text{\footnotesize $#1$}}{\text{\footnotesize $#2$}}}
\newcommand{\vasymptote}[2][]{
    \draw [densely dashed,#1] ({rel axis cs:0,0} -| {axis cs:#2,0}) -- ({rel axis cs:0,1} -| {axis cs:#2,0});
}


\graphicspath{{./}{firstExample/}}

\usepackage{amsmath}
\usepackage{amssymb}
\usepackage{array}
\usepackage[makeroom]{cancel} %% for strike outs
\usepackage{pgffor} %% required for integral for loops
\usepackage{tikz}
\usepackage{tikz-cd}
\usepackage{tkz-euclide}
\usetikzlibrary{shapes.multipart}


\usetkzobj{all}
\tikzstyle geometryDiagrams=[ultra thick,color=blue!50!black]


\usetikzlibrary{arrows}
\tikzset{>=stealth,commutative diagrams/.cd,
  arrow style=tikz,diagrams={>=stealth}} %% cool arrow head
\tikzset{shorten <>/.style={ shorten >=#1, shorten <=#1 } } %% allows shorter vectors

\usetikzlibrary{backgrounds} %% for boxes around graphs
\usetikzlibrary{shapes,positioning}  %% Clouds and stars
\usetikzlibrary{matrix} %% for matrix
\usepgfplotslibrary{polar} %% for polar plots
\usepgfplotslibrary{fillbetween} %% to shade area between curves in TikZ



%\usepackage[width=4.375in, height=7.0in, top=1.0in, papersize={5.5in,8.5in}]{geometry}
%\usepackage[pdftex]{graphicx}
%\usepackage{tipa}
%\usepackage{txfonts}
%\usepackage{textcomp}
%\usepackage{amsthm}
%\usepackage{xy}
%\usepackage{fancyhdr}
%\usepackage{xcolor}
%\usepackage{mathtools} %% for pretty underbrace % Breaks Ximera
%\usepackage{multicol}



\newcommand{\RR}{\mathbb R}
\newcommand{\R}{\mathbb R}
\newcommand{\C}{\mathbb C}
\newcommand{\N}{\mathbb N}
\newcommand{\Z}{\mathbb Z}
\newcommand{\dis}{\displaystyle}
%\renewcommand{\d}{\,d\!}
\renewcommand{\d}{\mathop{}\!d}
\newcommand{\dd}[2][]{\frac{\d #1}{\d #2}}
\newcommand{\pp}[2][]{\frac{\partial #1}{\partial #2}}
\renewcommand{\l}{\ell}
\newcommand{\ddx}{\frac{d}{\d x}}

\newcommand{\zeroOverZero}{\ensuremath{\boldsymbol{\tfrac{0}{0}}}}
\newcommand{\inftyOverInfty}{\ensuremath{\boldsymbol{\tfrac{\infty}{\infty}}}}
\newcommand{\zeroOverInfty}{\ensuremath{\boldsymbol{\tfrac{0}{\infty}}}}
\newcommand{\zeroTimesInfty}{\ensuremath{\small\boldsymbol{0\cdot \infty}}}
\newcommand{\inftyMinusInfty}{\ensuremath{\small\boldsymbol{\infty - \infty}}}
\newcommand{\oneToInfty}{\ensuremath{\boldsymbol{1^\infty}}}
\newcommand{\zeroToZero}{\ensuremath{\boldsymbol{0^0}}}
\newcommand{\inftyToZero}{\ensuremath{\boldsymbol{\infty^0}}}


\newcommand{\numOverZero}{\ensuremath{\boldsymbol{\tfrac{\#}{0}}}}
\newcommand{\dfn}{\textbf}
%\newcommand{\unit}{\,\mathrm}
\newcommand{\unit}{\mathop{}\!\mathrm}
%\newcommand{\eval}[1]{\bigg[ #1 \bigg]}
\newcommand{\eval}[1]{ #1 \bigg|}
\newcommand{\seq}[1]{\left( #1 \right)}
\renewcommand{\epsilon}{\varepsilon}
\renewcommand{\iff}{\Leftrightarrow}

\DeclareMathOperator{\arccot}{arccot}
\DeclareMathOperator{\arcsec}{arcsec}
\DeclareMathOperator{\arccsc}{arccsc}
\DeclareMathOperator{\si}{Si}
\DeclareMathOperator{\proj}{proj}
\DeclareMathOperator{\scal}{scal}
\DeclareMathOperator{\cis}{cis}
\DeclareMathOperator{\Arg}{Arg}
%\DeclareMathOperator{\arg}{arg}
\DeclareMathOperator{\Rep}{Re}
\DeclareMathOperator{\Imp}{Im}
\DeclareMathOperator{\sech}{sech}
\DeclareMathOperator{\csch}{csch}
\DeclareMathOperator{\Log}{Log}

\newcommand{\tightoverset}[2]{% for arrow vec
  \mathop{#2}\limits^{\vbox to -.5ex{\kern-0.75ex\hbox{$#1$}\vss}}}
\newcommand{\arrowvec}{\overrightarrow}
\renewcommand{\vec}{\mathbf}
\newcommand{\veci}{{\boldsymbol{\hat{\imath}}}}
\newcommand{\vecj}{{\boldsymbol{\hat{\jmath}}}}
\newcommand{\veck}{{\boldsymbol{\hat{k}}}}
\newcommand{\vecl}{\boldsymbol{\l}}
\newcommand{\utan}{\vec{\hat{t}}}
\newcommand{\unormal}{\vec{\hat{n}}}
\newcommand{\ubinormal}{\vec{\hat{b}}}

\newcommand{\dotp}{\bullet}
\newcommand{\cross}{\boldsymbol\times}
\newcommand{\grad}{\boldsymbol\nabla}
\newcommand{\divergence}{\grad\dotp}
\newcommand{\curl}{\grad\cross}
%% Simple horiz vectors
\renewcommand{\vector}[1]{\left\langle #1\right\rangle}


\outcome{Learn properties of indefinite integrals}

\title{4.2 Properties of Indefinite Integrals}


\begin{document}

\begin{abstract}
In this section we examine several properties of the indefinite integral.
\end{abstract}

\maketitle

\section{Properties of Indefinite Integrals}

\begin{theorem}[Constant Multiple Rule]
\[\int cf(x) \ dx = c\int f(x) \ dx\]
where c is a constant.
\end{theorem}


\begin{example} {\bf 1} %example CM1
\[
\int 5\cos(x) \ dx = 5 \int \cos(x) \ dx = 5\sin(x) +C.
\]
\end{example}


\begin{problem} %problem CM1a
\ Compute
%\[
%\int 8\cos(x) \ dx.
%\]

\begin{hint}
$\frac{d}{dx} \sin(x) = \cos(x)$
\end{hint}
\begin{hint}
\begin{center}
Do not add the +C to your answer
\end{center}
\end{hint}

\[
\int 8\cos(x) \ dx =
\answer[given]{8\sin(x)} \ +  C
\]
\end{problem}



\begin{problem} %problem CM1b
Compute
%\[
%\int 4\sin(x) \ dx.
%\]

\begin{hint}
$\frac{d}{dx} \cos(x) = -\sin(x)$
\end{hint}
\begin{hint}
\begin{center}
Do not add the +C to your answer
\end{center}
\end{hint}

\[
\int 4\sin(x) \ dx =
\answer[given]{-4\cos(x)} \ +  C
\]
\end{problem}



\begin{example} %example CM2
\[
\int 3\sec^2(x) \ dx = 3 \int \sec^2(x) \ dx = 3\tan(x) +C.
\]
\end{example}



\begin{problem} %problem CM2a
Compute
%\[
%\int 5\sec^2(x) \ dx.
%\]

\begin{hint}
$\frac{d}{dx} \tan(x) = \sec^2(x)$
\end{hint}
\begin{hint}
\begin{center}
Do not add the +C to your answer
\end{center}
\end{hint}

\[
\int 5\sec^2(x) \ dx =
\answer[given]{5\tan(x)} \ +  C
\]
\end{problem}


\begin{problem} %problem CM2b
Compute
%\[
%\int 2\csc^2(x) \ dx.
%\]

\begin{hint}
$\frac{d}{dx} \cot(x) = -\csc^2(x)$
\end{hint}
\begin{hint}
\begin{center}
Do not add the +C to your answer
\end{center}
\end{hint}

\[
\int 2\csc^2(x) \ dx =
\answer[given]{-2\cot(x)} \ +  C
\]
\end{problem}


\begin{example} %example CM3
\[
\int 2e^x \ dx = 2 \int e^x \ dx = 2e^x +C.
\]
\end{example}


\begin{problem} %problem CM3a
Compute
%\[
%\int 7e^x \ dx.
%\]

\begin{hint}
$\frac{d}{dx} e^x = e^x$
\end{hint}
\begin{hint}
\begin{center}
Do not add the +C to your answer
\end{center}
\end{hint}

\[
\int 7e^x \ dx =
\answer[given]{7e^x} \ +  C
\]
\end{problem}


\begin{problem} %problem CM3b
Compute
%\[
%\int \frac{e^x}{3} \ dx.
%\]

\begin{hint}
$\frac{d}{dx} e^x = e^x$
\end{hint}
\begin{hint}
\begin{center}
Do not add the +C to your answer
\end{center}
\end{hint}

\[
\int \frac{e^x}{3} \ dx =
\answer[given]{e^x /3} \ +  C
\]
\end{problem}


\begin{example} %example CM4
\[
\int \frac{4}{1+x^2} \ dx = 4 \int \frac{1}{1+x^2} \ dx = 4\tan^{-1}(x) +C.
\]
\end{example}



\begin{problem} %problem CM4a
Compute
%\[
%\int \frac{9}{1+x^2} \ dx.
%\]

\begin{hint}
$\frac{d}{dx} \tan^{-1}(x) = \frac{1}{1+x^2}$
\end{hint}
\begin{hint}
\begin{center}
Do not add the +C to your answer
\end{center}
\end{hint}

\[
\int \frac{9}{1+x^2} \ dx =
\answer[given]{9\tan^{-1}(x)} \ +  C
\]
\end{problem}

\begin{problem} %problem CM4b
Compute
%\[
%\int \frac{6}{\sqrt{1-x^2}} \ dx.
%\]

\begin{hint}
$\frac{d}{dx} \sin^{-1}(x) = \frac{1}{\sqrt{1-x^2}}$
\end{hint}
\begin{hint}
\begin{center}
Do not add the +C to your answer
\end{center}
\end{hint}

\[
\int \frac{6}{\sqrt{1-x^2}} \ dx =
\answer[given]{6\sin^{-1}(x)} \ +  C
\]
\end{problem}


\begin{example} %example CM5
\[
\int \frac{7}{12x} \ dx = \frac{7}{12} \int \frac{1}{x} \ dx = \frac{7}{12} \ln|x| +C.
\]
\end{example}


\begin{problem} %problem CM5a
Compute
%\[
%\int \frac{5}{x} \ dx.
%\]

\begin{hint}
$\frac{d}{dx} \ln|x| = \frac{1}{x}$
\end{hint}
\begin{hint}
\begin{center}
Do not add the +C to your answer
\end{center}
\end{hint}

\[
\int \frac{5}{x} \ dx =
\answer[given]{5\ln|x|} \ +  C
\]
\end{problem}

\begin{problem} %problem CM5b
Compute
%\[
%\int \frac{1}{2x} \ dx.
%\]

\begin{hint}
$\frac{d}{dx} \ln|x| = \frac{1}{x}$
\end{hint}
\begin{hint}
\begin{center}
Do not add the +C to your answer
\end{center}
\end{hint}

\[
\int \frac{1}{2x} \ dx =
\answer[given]{\ln|x| /2} \ +  C
\]
\end{problem}


\begin{example} %example CM6
\[
\int 4x^7 \ dx = 4 \int x^7 \ dx = 4\cdot \frac{x^8}{8} +C 
= \frac{x^8}{2}.
\]
\end{example}


\begin{problem} %problem 6a
Compute 
%\[
%\int 6x^4 \ dx.
%\]

\begin{hint}
Use the power rule with $n=4$
\end{hint}
\begin{hint}
The Power Rule says $\int x^n \ dx = \frac{x^{n+1}}{n+1} +$ C
\end{hint}
\begin{hint}
\begin{center}
Do not add the +C to your answer
\end{center}
\end{hint}

\[
\int 6x^4 \ dx =
\answer[given]{6x^5 /5} \ + C
\]
\end{problem}


\begin{problem} %problem 6b
Compute 
%\[
%\int 8x^3 \ dx.
%\]

\begin{hint}
Use the power rule with $n=3$
\end{hint}
\begin{hint}
The Power Rule says $\int x^n \ dx = \frac{x^{n+1}}{n+1} +$ C
\end{hint}
\begin{hint}
\begin{center}
Do not add the +C to your answer
\end{center}
\end{hint}

\[
\int 8x^3 \ dx =
\answer[given]{2x^4} \ + C
\]
\end{problem}


\begin{example} %example CM7
$\begin{aligned}[t]
\int \frac{3}{5x^2} \ dx &= \frac{3}{5} \int \frac{1}{x^2} \ dx \\[3pt]
&=\frac{3}{5} \int x^{-2} \ dx \\[3pt]
&= \frac{3}{5} \cdot \frac{x^{-1}}{-1} +C \\[5pt]
&= -\frac{3}{5} \cdot \frac{1}{x} +C \\[5pt]
&= -\frac{3}{5x} +C.
\end{aligned}$
\end{example}


\begin{problem} %problem 7a
Compute 
%\[
%\int \frac{3}{x^2} \ dx.
%\]

\begin{hint}
Negative exponents: $\frac{3}{x^2} = 3x^{-2}$
\end{hint}
\begin{hint}
Use the power rule with $n=-2$
\end{hint}
\begin{hint}
The Power Rule says $\int x^n \ dx = \frac{x^{n+1}}{n+1} +$ C
\end{hint}
\begin{hint}
\begin{center}
Do not add the +C to your answer
\end{center}
\end{hint}

\[
\int \frac{3}{x^2} \ dx =
\answer[given]{-3x^{-1}} \ + C
\]
\end{problem}


\begin{problem} %problem 7b
Compute 
%\[
%\int \frac{4}{5x^6} \ dx.
%\]

\begin{hint}
Negative exponents: $\frac{4}{5x^6} = \frac45 x^{-6}$
\end{hint}
\begin{hint}
Use the power rule with $n=-6$
\end{hint}
\begin{hint}
The Power Rule says $\int x^n \ dx = \frac{x^{n+1}}{n+1} +$ C
\end{hint}
\begin{hint}
\begin{center}
Do not add the +C to your answer
\end{center}
\end{hint}

\[
\int \frac{4}{5x^6} \ dx =
\answer[given]{-4/25 x^{-5}} \ + C
\]
\end{problem}


\begin{example} %example CM8
$\begin{aligned}[t]
\int 4\sqrt[3]x  \ dx &= 4 \int \sqrt[3] x \  dx  \\
&= 4 \int x^{1/3} \ dx \\
&= 4 \cdot \frac{x^{4/3}}{4/3} +C \\[3pt]
&= 4 \cdot \frac{3}{4} \cdot x^{4/3} +C \\[3pt]
&= 3x^{4/3} +C \\
&= 3\sqrt[3] {x^4} +C \\
&= 3x\sqrt[3] x +C.
\end{aligned}$
\end{example}


\begin{problem} %problem 8a
Compute 
%\[
%\int 3\sqrt x \ dx.
%\]

\begin{hint}
Rational exponents: $\sqrt x = x^{1/2}$
\end{hint}
\begin{hint}
Use the power rule with $n=1/2$
\end{hint}
\begin{hint}
The Power Rule says $\int x^n \ dx = \frac{x^{n+1}}{n+1} +$ C
\end{hint}
\begin{hint}
\begin{center}
Do not add the +C to your answer
\end{center}
\end{hint}

\[
\int 3\sqrt x \ dx =
\answer[given]{2x^{3/2}} \ + C
\]
\end{problem}


\begin{problem} %problem 8b
Compute 
\[
\int 5\sqrt[4]x \ dx.
\]

\begin{hint}
Rational exponents: $\sqrt[4]x = x^{1/4}$
\end{hint}
\begin{hint}
Use the power rule with $n=1/2$
\end{hint}
\begin{hint}
The Power Rule says $\int x^n \ dx = \frac{x^{n+1}}{n+1} +$ C
\end{hint}
\begin{hint}
\begin{center}
Do not add the +C to your answer
\end{center}
\end{hint}

\[
\int 5\sqrt[4]x \ dx =
\answer[given]{4x^{5/4}} \ + C
\]
\end{problem}



\begin{theorem}[Sum And Difference Rules]
The integral of a sum or difference is the sum or difference of the integrals:
\[
\int \left[f(x)\pm g(x)\right] \ dx = \int f(x) \; dx  \pm  \int g(x) \; dx.
\]
\end{theorem}



\begin{example} %example SR1
\[
\int \left(\cos(x) + e^x\right) \ dx = \sin(x) + e^x + C.
\]
\end{example}

\begin{problem} %problem SR1a
Compute
\[
\int \big(1 + \cos(x) \big) \ dx.
\]

\begin{hint}
$\frac{d}{dx} x = 1$
\end{hint}
\begin{hint}
$\frac{d}{dx} \sin(x) = \cos(x)$
\end{hint}
\begin{hint}
\begin{center}
Do not add the +C to your answer
\end{center}
\end{hint}

\[
\int \left[1 + \cos(x) \right] \ dx =
\answer[given]{x + \sin(x)} \ +  C
\]
\end{problem}


\begin{problem} %problem SR1b
Compute
\[
\int \left[e^x + \sin(x) \right] \ dx.
\]

\begin{hint}
$\frac{d}{dx} e^x = e^x$
\end{hint}
\begin{hint}
$\frac{d}{dx} \cos(x) = -\sin(x)$
\end{hint}
\begin{hint}
\begin{center}
Do not add the +C to your answer
\end{center}
\end{hint}

\[
\int \left(e^x + \sin(x) \right) \ dx =
\answer[given]{e^x -\cos(x)} \ +  C
\]
\end{problem}


\begin{example} %example DR1
\[
\int \left(1 - e^x\right) \ dx = x - e^x + C.
\]
\end{example}

\begin{problem}
Compute
\[
\int (3e^x - 4x) \; dx = \answer{3e^x - 2x^2} + C.
\]
\end{problem}



\begin{example} %example DR2
\[
\int \left(x^2 - x\right) \ dx = \dfrac{x^3}{3} - \dfrac{x^2}{2} + C.
\]
\end{example}


\begin{problem}
Compute
\[
\int (x^5 - x^3) \; dx = \answer{1/6 x^6 - 1/4 x^4} + C.
\]
\end{problem}




\begin{example} %example SR2
$\int \left(x + \sec^2(x)\right) \ dx = \frac{x^2}{2} + \tan(x) + C.$
\end{example}

\begin{problem} %problem SR2a
Compute
\[
\int \big(1 + \sec(x)\tan(x) \big) \ dx.
\]

\begin{hint}
$\frac{d}{dx} x = 1$
\end{hint}
\begin{hint}
$\frac{d}{dx} \sec(x) = \sec(x)\tan(x)$
\end{hint}
\begin{hint}
\begin{center}
Do not add the +C to your answer
\end{center}
\end{hint}

\[
\int \big(1 + \sec(x)\tan(x) \big) \ dx =
\answer[given]{x + \sec(x)} \ +  C
\]
\end{problem}


\begin{problem} %problem SR2b
Compute
\[
\int \left(x + \csc^2(x) \right) \ dx.
\]

\begin{hint}
$\frac{d}{dx} x^2 = 2x$
\end{hint}
\begin{hint}
$\frac{d}{dx} \cot(x) = -\csc^2(x)$
\end{hint}
\begin{hint}
\begin{center}
Do not add the +C to your answer
\end{center}
\end{hint}

\[
\int \left(x + \csc^2(x) \right) \ dx =
\answer[given]{x^2 /2 - \cot(x)} \ +  C
\]
\end{problem}

\begin{problem} %problem SR2c
Compute
\[
\int \sec(x)\big[\sec(x) + \tan(x) \big] \ dx.
\]

\begin{hint}
Distribute $\sec(x)$
\end{hint}
\begin{hint}
$\frac{d}{dx} \tan(x) = \sec^2(x)$
\end{hint}
\begin{hint}
$\frac{d}{dx} \sec(x) = \sec(x)\tan(x)$
\end{hint}
\begin{hint}
\begin{center}
Do not add the +C to your answer
\end{center}
\end{hint}

\[
\int \sec(x)\big[\sec(x) + \tan(x) \big] \ dx =
\answer[given]{\tan(x) + \sec(x)} \ +  C
\]
\end{problem}



\begin{example} %example DR3
\[
\int \left[3\cos(x) - 2\sin(x)\right] \ dx = 3\sin(x) + 2\cos(x) + C.
\]
\end{example}


\begin{problem}
Compute
\[
\int \left[4\sin(t) - 6\cos(t)\right] \; dt = \answer{-4\cos(t) - 6\sin(t)} + C.
\]
\end{problem}

\begin{example} %example DR5
\[
\int 4\csc(x)\big[\csc(x) - \cot(x)\big] \ dx = \int \big(4\csc^2(x)  - 4\csc(x)\cot(x)\big) \ dx
\]
\[
 = -4\cot(x) + 4\csc(x) + C
= 4\csc(x) - 4 \cot(x) + C.
\]
\end{example}

\begin{problem}
Compute
\[
\int 3\cot(x)\left[2\csc(x) - \tan(x)\right]\; dx = \answer{-6\csc(x) - 3x} + C.
\]
\end{problem}



\begin{example} %example SR3
$\int x^2 + \dfrac{1}{x^2} \ dx = \int x^2 + x^{-2} \ dx = \dfrac{x^3}{3} + \dfrac{x^{-1}}{-1} + C 
= \dfrac{x^3}{3} - \dfrac{1}{x} + C$
\end{example}


\begin{example} %example SR4
$\int \dfrac{x+2}{x} \ dx = \int \dfrac{x}{x} + \dfrac{2}{x} \ dx = \int 1 + \dfrac{2}{x} \ dx = x + 2\ln|x| +C.$
\end{example}


\begin{problem} %problem SR2b
Compute
\[
\int \dfrac{x^3 + 3x^2 - 4x + 5}{x^2} \ dx.
\]

\begin{hint}
Divide each term by $x^2$
\end{hint}
\begin{hint}
Use the power rule where appropriate ($n\neq -1$)
\end{hint}
\begin{hint}
\begin{center}
Do not add the +C to your answer
\end{center}
\end{hint}

\[
\int  \dfrac{x^3 + 3x^2 - 4x + 5}{x^2} \ dx =
\answer[given]{x^2 /2 + 3x - 4\ln|x| -5/x} \ +  C
\]
\end{problem}



\begin{example} %example DR4
\[
\int \left(3x^2 - \dfrac{3}{x} - \dfrac{1}{1+x^2}\right) \ dx = x^3 - 3\ln|x| - \tan^{-1}(x) + C.
\]
\end{example}


\begin{problem}
Compute
\[
\int \left(\frac{2}{x^3} - \frac{3}{x} \right) \; dx = \answer{-x^{-2} -3\ln|x|} + C.
\]
\end{problem}

\begin{example} %example SR5
$\begin{aligned}[t]
\int \big(\sqrt x + \sqrt[3] x \big) \ dx &= \int \big(x^{1/2} + x^{1/3}\big) \ dx \\
&= \frac{x^{3/2}}{3/2} + \frac{x^{4/3}}{4/3} + C \\
&= \tfrac{2}{3} x^{3/2} + \tfrac{3}{4} x^{4/3} + C \\
&= \tfrac{2}{3} \sqrt {x^3} + \tfrac{3}{4}  \sqrt[3] {x^4} + C \\
&= \tfrac{2}{3} x\sqrt x + \tfrac{3}{4} x \sqrt[3] x + C.
\end{aligned}$
\end{example}


\begin{problem} %problem SR2b
Compute
\[
\int x\big(\sqrt x + \sqrt[3] x \big) \ dx.
\]

\begin{hint}
Distribute the x (add exponents)
\end{hint}
\begin{hint}
Use the power rule where appropriate ($n\neq -1$)
\end{hint}
\begin{hint}
\begin{center}
Do not add the +C to your answer
\end{center}
\end{hint}

\[
\int  x\big(\sqrt x + \sqrt[3] x \big) \ dx =
\answer[given]{2/5 x^{5/2} + 3/7 x^{7/3}} \ +  C
\]
\end{problem}


\begin{example} %example SR6
$\int x^2(x^2 - 3x + 4) \ dx = \int (x^4 - 3x^3 + 4x^2) \ dx = \frac15 x^5 - \frac{3}{4}x^4 + \frac43 x^3 + C.$
\end{example}

\begin{problem}
Compute
\[
\int \left(x^4 - 2x^3 + 6x^2 - 8x + 7 \right) \; dx = \answer{ x^5/5 - x^4/2 + 2x^3 - 4x^2 + 7x} + C.
\]
\end{problem}

%\begin{theorem}(Difference Rule)
%If $\int f(x) \ dx = F(x) + C$ and $\int g(x) \ dx = G(x) + C$, then
%\[\int \big(f(x)-g(x)\big) \ dx = F(x) - G(x) + C\]
%\end{theorem}
%In words, the integral of a difference is the difference of the integrals.




\begin{center}
\textbf{Initial Value Problems}
\end{center}

One reason we might be interested in computing an indefinite integral is to solve a differential equation. Consider the following initial value problem:\\
Given, $f'(x) = h(x)$ and $f(x_0) = y_0$, find $f(x)$.\\
The condition, $f(x_0) = y_0$, is called an initial condition. The main step in slving this problem is to observe that the function $f(x)$ that we seek is an anti-derivative of the given function $h(x)$. Hence, to find $f(x)$ we will compute the definite integral, 

\[f(x) = \int f'(x) \ dx = \int h(x) \ dx.\]

Lets look at some examples.

\begin{example}
Solve the initial value problem:
\[f'(x) = x - 3, \ f(2) = 4.\]

First, 
\[f(x) = \int f'(x) \ dx = \int (x-3)  \ dx = \frac{x^2}{2} - 3x + C.\]
Then, we use the initial condition, $f(2) = 4$, to find the 
appropriate value of the constant $C$.
\[f(2) = \frac{2^2}{2} - 3(2) + C = -4 + C = 4,\]
which yields $C = 8$.
So, the final answer is  $f(x) = \frac{x^2}{2} -3x + 8.$
\end{example}

\begin{example}
Solve the initial value problem:
\[f'(x) = 2\cos(x) -3\sin(x), \ f(0) = 1.\]

First, 
\[f(x) = \int f'(x) \ dx = \int \big[2\cos(x) -3\sin(x)\big]  \ dx =
 2\sin(x) + 3\cos(x) + C.\]
Then, we use the initial condition, $f(0) = 1$, to find the 
appropriate value of the constant $C$.
\[f(0) = 2\sin(0) + 3\cos(0) + C = 0 + 3 + C = 1,\]
which yields $C = -2$.
So, the final answer is  $f(x) = 2\sin(x) + 3\cos(x)-2.$
\end{example}

\begin{problem}
Solve the initial value problem:
\[f'(x) = x-4, \ f(2) = 0.\]
\begin{hint}
anti-differentiate $f'(x)$
\end{hint}
\begin{hint}
Solve for the constant of integration, $C$
\end{hint}
\[f(x) = \answer{\frac{x^2}{2} - 4x + 6}.\]
\end{problem}

\begin{problem}
Solve the initial value problem:
\[f'(x) = 3\cos(x), \ f(0) = -2.\]
\begin{hint}
anti-differentiate $f'(x)$
\end{hint}
\begin{hint}
Solve for the constant of integration, $C$
\end{hint}
\[f(x) = \answer{3\sin(x) - 2}.\]
\end{problem}

\begin{problem}
Solve the initial value problem:
\[f'(x) = \frac{1}{x}, \ f(-1) = 5.\]
\begin{hint}
anti-differentiate $f'(x)$
\end{hint}
\begin{hint}
Solve for the constant of integration, $C$
\end{hint}
\[f(x) = \answer{\ln|x| + 5}.\]
\end{problem}

\begin{problem}
Solve the initial value problem:
\[f'(x) = e^{2x}, \ f(0) = 1.\]
\begin{hint}
Anti-differentiate $f'(x)$
\end{hint}
\begin{hint}
Solve for the constant of integration, $C$
\end{hint}
\[f(x) = \answer{\frac{e^{2x}}{2} + \frac12}.\]
\end{problem}


\end{document}
