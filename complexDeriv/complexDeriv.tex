\documentclass[handout]{ximera}

%% You can put user macros here
%% However, you cannot make new environments



\newcommand{\ffrac}[2]{\frac{\text{\footnotesize $#1$}}{\text{\footnotesize $#2$}}}
\newcommand{\vasymptote}[2][]{
    \draw [densely dashed,#1] ({rel axis cs:0,0} -| {axis cs:#2,0}) -- ({rel axis cs:0,1} -| {axis cs:#2,0});
}


\graphicspath{{./}{firstExample/}}

\usepackage{amsmath}
\usepackage{amssymb}
\usepackage{array}
\usepackage[makeroom]{cancel} %% for strike outs
\usepackage{pgffor} %% required for integral for loops
\usepackage{tikz}
\usepackage{tikz-cd}
\usepackage{tkz-euclide}
\usetikzlibrary{shapes.multipart}


\usetkzobj{all}
\tikzstyle geometryDiagrams=[ultra thick,color=blue!50!black]


\usetikzlibrary{arrows}
\tikzset{>=stealth,commutative diagrams/.cd,
  arrow style=tikz,diagrams={>=stealth}} %% cool arrow head
\tikzset{shorten <>/.style={ shorten >=#1, shorten <=#1 } } %% allows shorter vectors

\usetikzlibrary{backgrounds} %% for boxes around graphs
\usetikzlibrary{shapes,positioning}  %% Clouds and stars
\usetikzlibrary{matrix} %% for matrix
\usepgfplotslibrary{polar} %% for polar plots
\usepgfplotslibrary{fillbetween} %% to shade area between curves in TikZ



%\usepackage[width=4.375in, height=7.0in, top=1.0in, papersize={5.5in,8.5in}]{geometry}
%\usepackage[pdftex]{graphicx}
%\usepackage{tipa}
%\usepackage{txfonts}
%\usepackage{textcomp}
%\usepackage{amsthm}
%\usepackage{xy}
%\usepackage{fancyhdr}
%\usepackage{xcolor}
%\usepackage{mathtools} %% for pretty underbrace % Breaks Ximera
%\usepackage{multicol}



\newcommand{\RR}{\mathbb R}
\newcommand{\R}{\mathbb R}
\newcommand{\C}{\mathbb C}
\newcommand{\N}{\mathbb N}
\newcommand{\Z}{\mathbb Z}
\newcommand{\dis}{\displaystyle}
%\renewcommand{\d}{\,d\!}
\renewcommand{\d}{\mathop{}\!d}
\newcommand{\dd}[2][]{\frac{\d #1}{\d #2}}
\newcommand{\pp}[2][]{\frac{\partial #1}{\partial #2}}
\renewcommand{\l}{\ell}
\newcommand{\ddx}{\frac{d}{\d x}}

\newcommand{\zeroOverZero}{\ensuremath{\boldsymbol{\tfrac{0}{0}}}}
\newcommand{\inftyOverInfty}{\ensuremath{\boldsymbol{\tfrac{\infty}{\infty}}}}
\newcommand{\zeroOverInfty}{\ensuremath{\boldsymbol{\tfrac{0}{\infty}}}}
\newcommand{\zeroTimesInfty}{\ensuremath{\small\boldsymbol{0\cdot \infty}}}
\newcommand{\inftyMinusInfty}{\ensuremath{\small\boldsymbol{\infty - \infty}}}
\newcommand{\oneToInfty}{\ensuremath{\boldsymbol{1^\infty}}}
\newcommand{\zeroToZero}{\ensuremath{\boldsymbol{0^0}}}
\newcommand{\inftyToZero}{\ensuremath{\boldsymbol{\infty^0}}}


\newcommand{\numOverZero}{\ensuremath{\boldsymbol{\tfrac{\#}{0}}}}
\newcommand{\dfn}{\textbf}
%\newcommand{\unit}{\,\mathrm}
\newcommand{\unit}{\mathop{}\!\mathrm}
%\newcommand{\eval}[1]{\bigg[ #1 \bigg]}
\newcommand{\eval}[1]{ #1 \bigg|}
\newcommand{\seq}[1]{\left( #1 \right)}
\renewcommand{\epsilon}{\varepsilon}
\renewcommand{\iff}{\Leftrightarrow}

\DeclareMathOperator{\arccot}{arccot}
\DeclareMathOperator{\arcsec}{arcsec}
\DeclareMathOperator{\arccsc}{arccsc}
\DeclareMathOperator{\si}{Si}
\DeclareMathOperator{\proj}{proj}
\DeclareMathOperator{\scal}{scal}
\DeclareMathOperator{\cis}{cis}
\DeclareMathOperator{\Arg}{Arg}
%\DeclareMathOperator{\arg}{arg}
\DeclareMathOperator{\Rep}{Re}
\DeclareMathOperator{\Imp}{Im}
\DeclareMathOperator{\sech}{sech}
\DeclareMathOperator{\csch}{csch}
\DeclareMathOperator{\Log}{Log}

\newcommand{\tightoverset}[2]{% for arrow vec
  \mathop{#2}\limits^{\vbox to -.5ex{\kern-0.75ex\hbox{$#1$}\vss}}}
\newcommand{\arrowvec}{\overrightarrow}
\renewcommand{\vec}{\mathbf}
\newcommand{\veci}{{\boldsymbol{\hat{\imath}}}}
\newcommand{\vecj}{{\boldsymbol{\hat{\jmath}}}}
\newcommand{\veck}{{\boldsymbol{\hat{k}}}}
\newcommand{\vecl}{\boldsymbol{\l}}
\newcommand{\utan}{\vec{\hat{t}}}
\newcommand{\unormal}{\vec{\hat{n}}}
\newcommand{\ubinormal}{\vec{\hat{b}}}

\newcommand{\dotp}{\bullet}
\newcommand{\cross}{\boldsymbol\times}
\newcommand{\grad}{\boldsymbol\nabla}
\newcommand{\divergence}{\grad\dotp}
\newcommand{\curl}{\grad\cross}
%% Simple horiz vectors
\renewcommand{\vector}[1]{\left\langle #1\right\rangle}


\pgfplotsset{compat=1.13}

\outcome{Find derivatives of complex functions}

\title{3.2 Complex Derivatives}

\begin{document}

\begin{abstract}
We find derivatives of complex functions
\end{abstract}

\maketitle



We begin with the derivative of a complex function at $z = z_0$.

\begin{definition}
If $f(z)$ is defined in $D(z_0, r)$ for some $r>0$, then we define the derivative of $f$ at $z_0$ by
\[
f'(z_0) = \lim_{z \to z_0} \frac{f(z) - f(z_0)}{z-z_0}
\]
An equivalent alternative formulation is 
\[
f'(z_0) = \lim_{h \to 0} \frac{f(z_0 +h) -f(z_0)}{h}
\]
In the latter formulation, it is important to note that $h\in  \C$.
\end{definition}

\begin{example}
Let $z_0 \in \C$. Compute $f'(z_0)$ for $f(z) = z^4$.\\
Since $z^4 - z_0^4$ factors readily, we will use the first formulation of the derivative:
\[
f'(z_0) =\lim_{z \to z_0} \frac{z^4 - z_0^4}{z-z_0} =\lim_{z \to z_0} \frac{(z-z_0)(z+z_0)(z^2+z_0^2)}{z-z_0}
\]
\[
= \lim_{z \to z_0} (z+z_0)(z^2+z_0^2) = 2z_0\cdot 2z_0^2 = 4z_0^3
\]
Since this formula for $f'$ holds for any $z_0 \in \C$, we can write,
\[
f'(z) = 4z^3
\]
\end{example}


\begin{problem}
Compute $f'(z_0)$ for $f(z) = z^2$, $f(z) = z^3$ and $f(z) = \frac{1}{z}, (z_0 \neq 0)$ using the definition of the derivative.
\end{problem}

\begin{proposition}[Power Rule]
If $n$ is a positive integer, then $f(z) = z^n$ is differentiable on $\C$ and
\[
f'(z) = \frac{d}{dz} z^n = nz^{n-1}
\]
\end{proposition}

\begin{proof}
We will use the factorization
\[
z^n - z_0^n = (z-z_0)\left(z^{n-1} + z^{n-2}z_0 + \cdots + zz_0^{n-2} + z_0^{n-1} \right)
\]
We have
\begin{align*}
f'(z_0) &= \lim_{z \to z_0} \frac{z^n - z_0^n}{z-z_0} \\
        &=\lim_{z \to z_0} \frac{(z-z_0)\left(z^{n-1} + z^{n-2}z_0 + \cdots + zz_0^{n-2} + z_0^{n-1} \right)}{z-z_0} \\
        &= \lim_{z \to z_0} \left(z^{n-1} + z^{n-2}z_0 + \cdots + zz_0^{n-2} + z_0^{n-1} \right)\\
        & = z_0^{n-1} + z_0^{n-1} + \cdots z_0^{n-1}\\
        & = nz_0^{n-1}
\end{align*}
Since this holds for all $z_0 \in \C$, we can write $f'(z) = nz^{n-1}$.

\end{proof}


\begin{remark}
The power rule can also be proved using the alternate formulation of the derivative with the help of the Binomial Theorem:
\[
(z_0 + h)^n = \sum_{k = 0}^n \binom{n}{k} z_0^{n-k} h^k
\]
\end{remark}


\begin{problem}
Use the factorization 
\[
z^n - z_0^n = (z-z_0)\left(z^{n-1} + z^{n-2}z_0 + \cdots + zz_0^{n-2} + z_0^{n-1} \right)
\]
to show that if $\alpha \neq 1$ is an $n^{th}$ root of unity, i.e., $\alpha^n = 1$, then
\[
\alpha^{n-1} + \alpha^{n-2} + \cdots + \alpha + 1 = 0
\]
\end{problem}

\begin{problem}
Prove that for $z_0 \neq 0$, the power rule holds for negative integers.
\end{problem}

If a function is differentiable at $z= z_0$, then it must be continuous there.
\begin{theorem}
If $f$ is differentiable at $z = z_0$ then $f$ is continuous at $z = z_0$
\end{theorem}
\begin{proof}
\[
\lim_{z\to z_0} \left[f(z) - f(z_0)\right] = \lim_{z\to z_0} \frac{f(z) - f(z_0)}{z-z_0} \cdot (z-z_0)
\]
\[
= \lim_{z\to z_0} \frac{f(z) - f(z_0)}{z-z_0} \cdot \lim_{z\to z_0}(z-z_0) = f'(z_0) \cdot 0 = 0
\]
Thus
\[
 \lim_{z\to z_0} f(z) = f(z_0),
 \]
 and so $f(z)$ is continuous at $z = z_0$.
\end{proof}
We next state familiar rules for differentiation that extend from functions of a real variable to complex functions.

\begin{theorem}[Basic Rules for Differentiation]
If $f$ and $g$ are complex functions that are differentiable at $z = z_0$ then the following hold:
\begin{align*}
\mbox{Sum Rule} & \quad (f+g)'(z_0) = f'(z_0) + g'(z_0)\\[6 pt]
\mbox{Product Rule} & \quad (fg)'(z_0) = f'(z_0)g(z_0) + f(z_0)g'(z_0)\\[6pt]
\mbox{Quotient Rule} &\quad \left (\frac{f}{g}\right)'(z_0) = \frac{f'(z_0)g(z_0) - f(z_0)g'(z_0)}{g^2(z_0)}\quad (\mbox{if} \; g(z_0) \neq 0)
\end{align*}
\end{theorem}

\begin{proof}
We prove the quotient rule and leave the others as exercises. If $g(z_0) \neq 0$ then
\[
\left(\frac{f}{g}\right)'(z_0) = \lim_{z \to z_0} \frac{\frac{1}{g(z)} - \frac{1}{g(z_0)}}{z-z_0}
\]
\[
= \lim_{z \to z_0} \frac{g(z_0)- g(z)}{g(z)\cdot g(z_0) \cdot(z-z_0)}
\]
\[
= \lim_{z \to z_0} \frac{g(z_0)- g(z)}{z-z_0}   \cdot \lim_{z \to z_0} \frac{1}{g(z)\cdot g(z_0)}
\]
\[
= -g'(z_0) \cdot \frac{1}{g^2(z_0)} = -\frac{g'(z_0)}{g^2(z_0)}
\]
The quotient rule follows from combining this reciprocal rule with the product rule.

\end{proof}

\begin{theorem}[Chain Rule]
If $g(z)$ is differentiable at $z = z_0$ and $f(z)$ is differentiable at $z = g(z_0)$, then
$f \circ g$ is differentiable at $z_0$ and
\[
(f\circ g)'(z_0) = \frac{d}{dz} f(g(z)) \Big|_{z=z_0} = f'(g(z_0)) \cdot g'(z_0)
\]
\end{theorem}

\begin{proof}

Let
\[ \phi(z)  = \left\{
     \begin{array}{lr}
       \frac{f(g(z)) - f(g(z_0))}{g(z) -g(z_0)} & \ \text{for} \  g(z) \neq g(z_0) \\[12pt]
       f'(g(z_0)) & \ \text{for} \ g(z) = g(z_0)
     \end{array}
   \right.
\]
Then
\[
\lim_{z \to z_0} \phi(z) = f'(g(z_0))
\]
Now, 
\begin{align*}
(f\circ g)'(z_0) &= \lim_{z\to z_0} \frac{f(g(z)) - f(g(z_0))}{z-z_0} \\[10pt]
                 &= \lim_{z \to z_0} \phi(z_0) \cdot \frac{g(z) - g(z_0)}{z-z_0} \\[10pt]
                 &= f'(g(z_0)) \cdot g'(z_0)
\end{align*}

\end{proof}

\begin{example}
Find the derivative of $f(z) = (z^2 + iz + 1)^5$.\\
By the Chain Rule and the Power rule, we have
\[
f'(z) = 5(z^2 + z + 1)^4 \cdot (2z +i)
\]
\end{example}

\begin{problem}
Find the derivative of $f(z) = (z^3 - 2iz^2 + 3 + 2i)^4$.
\end{problem}
In the next example, we will see a function that is differentiable at $0$ but nowhere else.

\begin{example}
Let $f(z) = \overline{z}^2$.  Show that $f'(0) = 0$ but that $f$ is not differentiable for any $z_0 \neq 0$.\\
We begin with the derivative at $0$:
\[
f'(0) = \lim_{z \to  0} \frac{\overline{z}^2}{z} = \lim_{z \to  0} \frac{\overline{z}^3}{|z|^2}
\]
We will separate this into real and imaginary parts using 
\begin{align*}
\overline{z}^3 &= (x -iy)^3 = x^3 - 3x^2(iy) + 3x(iy)^2  - (iy)^3 
              &= (x^3 -3xy^2) + i(y^3 - 3x^2y)
\end{align*}
Doing so gives
\begin{align*}
f'(0) &= \lim_{z \to  0} \frac{\overline{z}^3}{|z|^2}\\
      &= \lim_{(x,y) \to (0,0)} \frac{x^3 -3xy^2}{x^2 +y^2} + i\cdot \lim_{(x,y) \to (0,0)} \frac{y^3 - 3x^2y}{x^2 +y^2}\\
      & = 0 + 0i \;\; \mbox{(verify)}
\end{align*}

Next, suppose $z_0 \neq 0$, then the alternative formulation of $f'(z_0)$ gives
\begin{align*}
f'(z_0) &= \lim_{h \to 0} \frac{\left(\overline{z_0 + h}\right)^2-\overline{z_0}^2 }{h}\\[6pt]
        &=\lim_{h \to 0} \frac{\overline{z_0}^2 +2\overline{z_0h}+ \overline{h}^2 -\overline{z_0}^2 }{h}\\[6pt]
        &=\lim_{h \to 0} \frac{\overline{h} \left(2\overline{z_0}+ \overline{h}\right)}{h}.
\end{align*}
This last limit is equal to $2\overline{z_0}$ if $h$ is real and it is equal to $-2z_0$ if $h$ is purely imaginary.
Since $2\overline{z_0} \neq -2\overline{z_0}$ if $z_0 \neq 0$, this limit does not exist and $f$ is not differentiable at $z_0 \neq 0$.\\
Hence $f(z) = \overline{z}^2$ is only differentiable at $0$ and $f'(0) = 0$

\end{example}


\begin{problem}
Show that the function $f(z) = \overline{z}$ is nowhere differentiable.
\end{problem}

\end{document}



