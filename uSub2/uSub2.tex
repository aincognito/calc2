\documentclass[handout]{ximera}
\usepackage{tcolorbox}
%% You can put user macros here
%% However, you cannot make new environments



\newcommand{\ffrac}[2]{\frac{\text{\footnotesize $#1$}}{\text{\footnotesize $#2$}}}
\newcommand{\vasymptote}[2][]{
    \draw [densely dashed,#1] ({rel axis cs:0,0} -| {axis cs:#2,0}) -- ({rel axis cs:0,1} -| {axis cs:#2,0});
}


\graphicspath{{./}{firstExample/}}

\usepackage{amsmath}
\usepackage{amssymb}
\usepackage{array}
\usepackage[makeroom]{cancel} %% for strike outs
\usepackage{pgffor} %% required for integral for loops
\usepackage{tikz}
\usepackage{tikz-cd}
\usepackage{tkz-euclide}
\usetikzlibrary{shapes.multipart}


\usetkzobj{all}
\tikzstyle geometryDiagrams=[ultra thick,color=blue!50!black]


\usetikzlibrary{arrows}
\tikzset{>=stealth,commutative diagrams/.cd,
  arrow style=tikz,diagrams={>=stealth}} %% cool arrow head
\tikzset{shorten <>/.style={ shorten >=#1, shorten <=#1 } } %% allows shorter vectors

\usetikzlibrary{backgrounds} %% for boxes around graphs
\usetikzlibrary{shapes,positioning}  %% Clouds and stars
\usetikzlibrary{matrix} %% for matrix
\usepgfplotslibrary{polar} %% for polar plots
\usepgfplotslibrary{fillbetween} %% to shade area between curves in TikZ



%\usepackage[width=4.375in, height=7.0in, top=1.0in, papersize={5.5in,8.5in}]{geometry}
%\usepackage[pdftex]{graphicx}
%\usepackage{tipa}
%\usepackage{txfonts}
%\usepackage{textcomp}
%\usepackage{amsthm}
%\usepackage{xy}
%\usepackage{fancyhdr}
%\usepackage{xcolor}
%\usepackage{mathtools} %% for pretty underbrace % Breaks Ximera
%\usepackage{multicol}



\newcommand{\RR}{\mathbb R}
\newcommand{\R}{\mathbb R}
\newcommand{\C}{\mathbb C}
\newcommand{\N}{\mathbb N}
\newcommand{\Z}{\mathbb Z}
\newcommand{\dis}{\displaystyle}
%\renewcommand{\d}{\,d\!}
\renewcommand{\d}{\mathop{}\!d}
\newcommand{\dd}[2][]{\frac{\d #1}{\d #2}}
\newcommand{\pp}[2][]{\frac{\partial #1}{\partial #2}}
\renewcommand{\l}{\ell}
\newcommand{\ddx}{\frac{d}{\d x}}

\newcommand{\zeroOverZero}{\ensuremath{\boldsymbol{\tfrac{0}{0}}}}
\newcommand{\inftyOverInfty}{\ensuremath{\boldsymbol{\tfrac{\infty}{\infty}}}}
\newcommand{\zeroOverInfty}{\ensuremath{\boldsymbol{\tfrac{0}{\infty}}}}
\newcommand{\zeroTimesInfty}{\ensuremath{\small\boldsymbol{0\cdot \infty}}}
\newcommand{\inftyMinusInfty}{\ensuremath{\small\boldsymbol{\infty - \infty}}}
\newcommand{\oneToInfty}{\ensuremath{\boldsymbol{1^\infty}}}
\newcommand{\zeroToZero}{\ensuremath{\boldsymbol{0^0}}}
\newcommand{\inftyToZero}{\ensuremath{\boldsymbol{\infty^0}}}


\newcommand{\numOverZero}{\ensuremath{\boldsymbol{\tfrac{\#}{0}}}}
\newcommand{\dfn}{\textbf}
%\newcommand{\unit}{\,\mathrm}
\newcommand{\unit}{\mathop{}\!\mathrm}
%\newcommand{\eval}[1]{\bigg[ #1 \bigg]}
\newcommand{\eval}[1]{ #1 \bigg|}
\newcommand{\seq}[1]{\left( #1 \right)}
\renewcommand{\epsilon}{\varepsilon}
\renewcommand{\iff}{\Leftrightarrow}

\DeclareMathOperator{\arccot}{arccot}
\DeclareMathOperator{\arcsec}{arcsec}
\DeclareMathOperator{\arccsc}{arccsc}
\DeclareMathOperator{\si}{Si}
\DeclareMathOperator{\proj}{proj}
\DeclareMathOperator{\scal}{scal}
\DeclareMathOperator{\cis}{cis}
\DeclareMathOperator{\Arg}{Arg}
%\DeclareMathOperator{\arg}{arg}
\DeclareMathOperator{\Rep}{Re}
\DeclareMathOperator{\Imp}{Im}
\DeclareMathOperator{\sech}{sech}
\DeclareMathOperator{\csch}{csch}
\DeclareMathOperator{\Log}{Log}

\newcommand{\tightoverset}[2]{% for arrow vec
  \mathop{#2}\limits^{\vbox to -.5ex{\kern-0.75ex\hbox{$#1$}\vss}}}
\newcommand{\arrowvec}{\overrightarrow}
\renewcommand{\vec}{\mathbf}
\newcommand{\veci}{{\boldsymbol{\hat{\imath}}}}
\newcommand{\vecj}{{\boldsymbol{\hat{\jmath}}}}
\newcommand{\veck}{{\boldsymbol{\hat{k}}}}
\newcommand{\vecl}{\boldsymbol{\l}}
\newcommand{\utan}{\vec{\hat{t}}}
\newcommand{\unormal}{\vec{\hat{n}}}
\newcommand{\ubinormal}{\vec{\hat{b}}}

\newcommand{\dotp}{\bullet}
\newcommand{\cross}{\boldsymbol\times}
\newcommand{\grad}{\boldsymbol\nabla}
\newcommand{\divergence}{\grad\dotp}
\newcommand{\curl}{\grad\cross}
%% Simple horiz vectors
\renewcommand{\vector}[1]{\left\langle #1\right\rangle}


\outcome{Compute an anti-derivative by reversing the chain rule.}

\title{0.3 Substitution}



\begin{document}

\begin{abstract}
In this section we learn to reverse the chain rule by making a substitution.
\end{abstract}

\maketitle

\section{$U$-Substitution}

We make a $u$-substitution to fill in the gaps in the following equation, which reverses the chain rule:

\[\int f(g(x))g'(x) \ dx = F(g(x)) + C. \]

If we let $u$ be the inside of the composition and we compute $du$:
 \begin{align*}
 &\text{Let} \quad \;\;u  = g(x) \\
 &\text{then} \quad du = g'(x) \ dx
 \end{align*}
 Substituting $u$ and $du$ into the original integral gives

\[\int f(g(x))g'(x) \ dx = \int f(u) \ du \]
which we can compute if we know an anti-derivative of $f(u)$:
\[=F(u) + C,\]
Finally, we can back-substitute to get the final answer:
\[=F(g(x)) + C. \]
Now we can see that
\[\int f(g(x))g'(x) \ dx = F(g(x)) + C. \]



\begin{example}[example 1]
Compute 
\[\int 2x\cos(x^2 + 1) \ dx.\]

Let $u = x^2 + 1$.  We have,
\[u = x^2 + 1\]
\[du = 2x \ dx\]

and the integral can be written as 
\[\int 2x\cos(x^2 + 1) \ dx = \int \cos(x^2 + 1) \ 2x\  dx = \int \cos(u) \ du.\]
The last integral can be computed as 
\[\int \cos(u) \ du = \sin(u) + C,\]
and by back-substituting, we have 
\[\sin(u) + C = \sin(x^2 + 1) + C.\]
Thus, using $u$-substitution we can conclude that
\[\int 2x\cos(x^2 + 1) \ dx =  \sin(x^2 + 1) + C.\]
\end{example}



\begin{problem}(problem 1)\\ Compute: $\displaystyle{\int 5x^4\sin(x^5 -7) \ dx}$.\\
Let $u = \answer{x^5 - 7}$. Then $du = \answer{5x^4 dx}$.\\
\begin{hint}
Don't forget the `dx' in your answer for `du'.
\end{hint}
Convert to an intergal in the variable $u$:
\[\int 5x^4\sin(x^5 -7) \ dx = \int\answer{\sin(u) du}\]
\begin{hint}
Don't forget the `du' in your integral.
\end{hint}
The final answer in terms of $x$ is:
\[\int 5x^4\sin(x^5 -7) \ dx = \answer{-\cos(x^5 -7)} +C.\]
\end{problem}


\begin{example}[example 2] Compute 
\[\int 2x(x^2 + 1)^4 \ dx.\]
Let $u = x^2 + 1$.  We have,
\[u = x^2 + 1\]
\[du = 2x \ dx\]

and the integral can be written as 
\[\int 2x(x^2 + 1)^4 \ dx = \int (x^2 + 1)^4 \ 2x \  dx = \int u^4 \ du.\]
The last integral can be computed as 
\[\int u^4 \ du = \tfrac{u^5}{5} + C,\]
and by back-substituting, we have 
\[\tfrac{u^5}{5} + C = \tfrac15(x^2 + 1)^5 + C .\]
Thus, using $u$-substitution we can conclude that
\[\int 2x(x^2 + 1)^4 \ dx  =  \tfrac15(x^2 + 1)^5 + C.\]
\end{example}

\begin{problem}(problem 2)\\
Compute $\displaystyle{\int 3x^2(x^3 + 2)^6 \ dx}$.\\
Let $u = \answer{x^3 + 2}$. Then $du = \answer{3x^2 dx}$.\\
\begin{hint}
Don't forget the `dx' in your answer for `du'.
\end{hint}
Convert to an intergal in the variable $u$:
\[\int 3x^2(x^3 + 2)^6 \ dx = \int\answer{u^6 du}\]
\begin{hint}
Don't forget the `du' in your integral.
\end{hint}
The final answer in terms of $x$ is:

\[\int 3x^2(x^3 + 2)^6 \ dx = \answer{(x^3 + 2)^7/7} +C.\]
\end{problem}



\begin{example}[example 3] Compute 
\[\int_0^2 2xe^{x^2 + 1} \ dx.\]
In this example, we have a definite integral.  When making a substitution in a definite integral, 
we must change the endpoints of integration in addition to the integrand and the differential.
As before, let $u = x^2 + 1$.  We have,
\[u = x^2 + 1\]
\[du = 2x \ dx\]
Now we prepare to change the endpoints. If $x = 0$, then $ u = 0^2 + 1 = 1$ and if $x = 2$, then $u = 2^2 +1 = 5$.
So, the endpoints in the original integral, 0 and 2, will be changed to 1 and 5 respectively. We get
\[\int_0^2 2xe^{x^2 + 1} \ dx  = \int_1^5  e^u \ du.\]
The last integral can be computed as 
\[\int_1^5 e^u \ du = e^u \Big{|}_1^5 = e^5 - e.\]

Thus, using $u$-substitution we can conclude that
\[\int_0^2 2xe^{x^2 + 1} \ dx =  e^5 - e.\]
\end{example}



\begin{problem}(problem 3)\\
Compute $\displaystyle{\int_1^2 -2xe^{-x^2} \ dx}$.\\
Let $u = \answer{-x^2}$. Then $du = \answer{-2x dx}$.\\
\begin{hint}
Don't forget the `dx' in your answer for `du'.
\end{hint}
If $x = 1$, then $u = \answer{-1}$ and \\
if $x = 2$, then $u = \answer{-4}$\\
Convert to a definite intergal in the variable $u$:
\[\int_1^2 -2xe^{-x^2} \ dx = \int_{\answer{-1}}^{\answer{-4}} \answer{e^u du}\]
\begin{hint}
Don't forget the `du' in your integral.
\end{hint}
The final answer is:
\[\int_1^2 -2xe^{-x^2} \ dx = \answer{e^{-4} - e^{-1}} \]
\end{problem}

\begin{example}[example 4] Compute 
\[\int 2x\sqrt{x^2 + 1} \ dx.\]
Let $u = x^2 + 1$.  We have,
\[u = x^2 + 1\]
\[du = 2x \ dx\]

\[\int 2x\sqrt{x^2 + 1} \ dx = \int \sqrt{x^2 + 1} \cdot 2x\  dx = \int \sqrt{u} \ du.\]
The last integral can be computed as 
\[\int \sqrt u  \ du = \int u^{1/2} \ du = \tfrac{u^{3/2}}{3/2} + C = \tfrac23 u^{3/2} + C,\]
and by back-substituting, we have 
\[\tfrac23 u^{3/2}  + C = \frac23 (x^2 + 1)^{3/2} + C.\]
Thus, using $u$-substitution we can conclude that
\[\int 2x\sqrt{x^2 + 1} \ dx =  \tfrac23 (x^2 + 1)^{3/2} + C.\]
\end{example}

\begin{problem}(problem 4) Compute $\displaystyle{\int 4x^3 \sqrt{x^4 +2} \ dx}$.\\
Let $u = \answer{x^4 + 2}$. Then $du = \answer{4x^3 dx}$.\\
\begin{hint}
Don't forget the `dx' in your answer for `du'.
\end{hint}
Convert to an intergal in the variable $u$:
\[\int 4x^3 \sqrt{x^4 +2} \ dx = \int\answer{\sqrt u du}\]
\begin{hint}
Don't forget the `du' in your integral.
\end{hint}

\[\int 4x^3 \sqrt{x^4 +2} \ dx = \answer{2/3(x^4 +2)^{3/2}} +C.\]
\end{problem}



\begin{example}[example 5] Compute 
\[\int \frac{2x}{x^2 + 1} \ dx.\]
Let $u = x^2 + 1$.  We have,
\[u = x^2 + 1\]
\[du = 2x \ dx\]

\[\int \frac{2x}{x^2 + 1} \ dx = \int \frac{1}{x^2 + 1} \cdot 2x\  dx = \int \frac{1}{u} \ du.\]
The last integral can be computed as 
\[\int \frac{1}{u} \ du = \ln|u| + C,\]
and by back-substituting, we have 
\[\ln|u| + C =  \ln|x^2 + 1| + C=\ln(x^2 + 1) + C.\]
Thus, using $u$-substitution we can conclude that
\[\int \frac{2x}{x^2 + 1} \ dx =  \ln(x^2 + 1) + C.\]
\end{example}


\begin{problem} (problem 5) Compute $\displaystyle{\int \frac{4x^3}{x^4 +2} \ dx}$.\\
Let $u = \answer{x^4 + 2}$. Then $du = \answer{4x^3 dx}$.\\
\begin{hint}
Don't forget the `dx' in your answer for `du'.
\end{hint}
Convert to an intergal in the variable $u$:
\[\int \frac{4x^3}{x^4 +2} \ dx = \int\answer{1/u du}\]
\begin{hint}
Don't forget the `du' in your integral.
\end{hint}

\[\int \frac{4x^3}{x^4 +2} \ dx = \answer{\ln|x^4 +2|} +C.\]
\end{problem}



\begin{example}[example 6] Compute 
\[\int e^x\sin(e^x) \ dx.\]
We let $u = e^x$.  We have,
\[u = e^x\]
\[du = e^x \ dx\]

and the integral can be written as 
\[\int e^x\sin(e^x) \ dx = \int \sin(e^x) \cdot e^x \  dx = \int \sin(u) \ du.\]
The last integral can be computed as 
\[\int \sin(u) \ du = -\cos(u) + C,\]
and by back-substituting, we have 
\[-\cos(u) + C = -\cos(e^x) + C.\]
Thus, using $u$-substitution we can conclude that
\[\int e^x\sin(e^x) \ dx =  -\cos(e^x) + C.\]
\end{example}




\begin{problem}(problem 6)\\
Compute $\displaystyle{\int \frac{\cos(\ln(x))}{x} \ dx}$.\\
Let $u = \answer{\ln(x)}$. Then $du = \answer{1/x dx}$.\\
\begin{hint}
Don't forget the `dx' in your answer for `du'.
\end{hint}
Convert to an intergal in the variable $u$:
\[\int \frac{\cos(\ln(x))}{x} \ dx = \int\answer{\cos(u) du}\]
\begin{hint}
Don't forget the `du' in your integral.
\end{hint}

\[\int \frac{\cos(\ln(x))}{x} \ dx = \answer{\sin(\ln(x))} +C.\]
\end{problem}


\begin{example}[example 7] Compute 
\[\int 3x^2\sec(x^3)\tan(x^3) \ dx.\]
Let $u = x^3$.  Then we have,
\[u = x^3\]
\[du = 3x^2 \ dx\]

and the integral can be written as 
\begin{align*}
\int 3x^2\sec(x^3)\tan(x^3) \ dx &= \int \sec(x^3)\tan(x^3) \cdot 3x^2 \ dx\\
&= \int \sec(u)\tan(u) \ du.
\end{align*}
The last integral can be computed as 
\[\int \sec(u)\tan(u) \ du = \sec(u) + C,\]
and by back-substituting, we have 
\[\sec(u) + C = \sec(x^3) + C.\]
Thus, using $u$-substitution we can conclude that
\[\int 3x^2\sec(x^3)\tan(x^3) \ dx =  \sec(x^3) + C.\]
\end{example}

\begin{problem}(problem 7)
\begin{hint}
Let $u = e^x$
\end{hint}
\begin{hint}
Compute $du$
\end{hint}
\[\int e^x\sec(e^x)\tan(e^x) \ dx = \answer{\sec(e^x)} +C.\]
\end{problem}


\begin{example}[example 8] Compute 
\[\int x^3\cos(x^4) \ dx.\]
Let $u = x^4$.  Then,
\[u = x^4\]
\[du = 4x^3 \ dx\]

and the integral can be written as 
\begin{align*}
\int x^3\cos(x^4) \ dx &= \int \cos(x^4) \cdot x^3\  dx \\
&=  \int \cos(x^4)\cdot \tfrac14 \cdot 4x^3\  dx\\
&=  \int \tfrac14\cos(u) \ du.
\end{align*}
The last integral can be computed as 
\[\int \tfrac14 \cos(u) \ du = \tfrac14 \sin(u) + C,\]
and by back-substituting, we have 
\[\tfrac14 \sin(u) + C = \tfrac14 \sin(x^4) + C.\]
Thus, using $u$-substitution we can conclude that
\[\int x^3\cos(x^4) \ dx =  \tfrac14 \sin(x^4) + C.\]
\end{example}

\begin{problem}(problem 8)
\begin{hint}
Let $u = x^2$
\end{hint}
\begin{hint}
Compute $du$
\end{hint}
\[\int x\cos(x^2) \ dx = \answer{\sin(x^2)/2} +C.\]
\end{problem}


\begin{example}[example 9] Compute 
\[\int xe^{-x^2} \ dx.\]
Let $u = -x^2 $.  Then we have,
\[u = -x^2\]
\[du = -2x \ dx\]
and the integral can be written as
\begin{align*}
\int xe^{-x^2} \ dx &= \int e^{-x^2} \cdot x\  dx \\
&=  \int e^{-x^2}( -\tfrac12)\ (-2x)\  dx \\
&=  \int -\tfrac12 e^u \ du.
\end{align*}
The last integral can be computed as 
\[ \int -\tfrac12 e^u \ du = -\tfrac12 e^u + C,\]
and by back-substituting, we have 
\[-\tfrac12 e^u + C = -\tfrac12 e^{-x^2} + C.\]
Thus, using $u$-substitution we can conclude that
\[\int xe^{-x^2} \ dx =  -\tfrac12 e^{-x^2} + C.\]
\end{example}


\begin{problem}(problem 9)
\begin{hint}
Let $u = x^3$
\end{hint}
\begin{hint}
Compute $du$
\end{hint}
\[\int x^2e^{x^3} \ dx = \answer{e^{x^3}/3} +C.\]
\end{problem}




\begin{example}[example 10] Compute 
\[\int \sin(3x) \ dx.\]
Let $u = 3x$.  Then we have,
\[u = 3x\]
\[du = 3 \ dx\]
and the integral can be written as 
\[\int\sin(3x) \ dx =  \int \sin(3x) \cdot \tfrac13\cdot 3 \   dx =   \int \tfrac13 \sin(u) \ du.\]
The last integral can be computed as 
\[ \int \tfrac13 \sin(u) \ du = -\tfrac13 \cos(u) + C,\]
and by back-substituting, we have 
\[-\tfrac13 \cos(u) + C = -\tfrac13 \cos(3x) + C.\]
Thus, using $u$-substitution we can conclude that
\[\int \sin(3x) \ dx =  -\tfrac13 \cos(3x) + C.\]
\end{example}

\begin{problem}(problem 10)
\begin{hint}
Let $u = 3x$
\end{hint}
\begin{hint}
Compute $du$
\end{hint}
\[\int \cos(3x) \ dx = \answer{1/3 \sin(3x)} +C.\]
\end{problem}






\begin{example}[example 11] Compute 
\[\int e^{\frac{x}{2}} \ dx.\]
Let $u = \frac{x}{2}$.  Then we have,
\[u = \frac{x}{2}\]
\[du = \tfrac12 \ dx\]
and the integral can be written as 
\[\int e^{\frac{x}{2}} \ dx =  \int e^{\frac{x}{2}} \cdot 2\cdot \tfrac{1}{2}  \   dx =   \int 2e^u \ du.\]
The last integral can be computed as 
\[\int 2e^u \ du = 2 e^u + C,\]
and by back-substituting, we have 
\[2e^u + C = 2e^{\frac{x}{2}}+ C.\]
Thus, using $u$-substitution we can conclude that
\[\int e^{x/2} \ dx = 2e^{x/2} + C.\]
\end{example}


\begin{problem}(problem 11)
\begin{hint}
Let $u = 5x$
\end{hint}
\begin{hint}
Compute $du$
\end{hint}
\[\int e^{5x} \ dx = \answer{1/5 e^{5x}} +C.\]
\end{problem}


\begin{example}[example 12] Compute 
\[\int (4x+3)^5 \ dx.\]
Let $u = 4x+3$. Then we have,
\[u = 4x+3\]
\[du = 4 \ dx\]
and the integral can be written as 
\[\int (4x+3)^5 \ dx =   \int (4x+3)^5 \cdot \tfrac14\cdot 4  \   dx =   \int \tfrac14 u^5 \ du.\]
The last integral can be computed as 
\[\tfrac14  \int u^5 \ du = \tfrac14  \cdot \tfrac{u^6}{6} + C = \tfrac{u^6}{24} + C,\]
and by back-substituting, we have 
\[\tfrac{u^6}{24} + C = \tfrac{(4x+3)^6}{24}+ C.\]
Thus, using $u$-substitution we can conclude that
\[\int (4x+3)^5 \ dx = \tfrac{1}{24}(4x+3)^6 + C.\]
\end{example}




\begin{problem}(problem 12)
\begin{hint}
Let $u = 2x - 5$
\end{hint}
\begin{hint}
Compute $du$
\end{hint}
\[\int (2x - 5)^4 \ dx = \answer{\frac{1}{10}(2x-5)^5} +C.\]
\end{problem}


\begin{example}[example 13]
 Compute $\dis \int \frac{1}{x\ln^2(x)} \ dx.$\\
Step 1) Transform the integral:
\begin{align*}
\int \frac{1}{x\ln^2(x)}& \ dx = \int \frac{1}{\ln^2(x)} \cdot \frac{1}{x} \ dx = \int \frac{1}{u^2} \ du\\
\text{let} \;\; u =& \ln(x)&  \\
\text{then} \;\; du =& \frac{1}{x} dx&
\end{align*}
Step 2) Anti-differentiate:
\[
\int \frac{1}{u^2} \, du = \int u^{-2} \, du = \frac{u^{-1}}{-1} + C = -\frac{1}{u} + C
\]
Step 3) Back-substitute:
\[
-\frac{1}{u} + C = -\frac{1}{\ln(x)} + C
\]
Hence,
\[
\int \frac{1}{x\ln^2(x)} \ dx = -\frac{1}{\ln(x)} + C
\]

Let $u = \ln(x)$. Then we have,
\[u = \ln(x)\]
\[du = \tfrac1x \ dx\]
and the integral can be written as 
\[\int \frac{1}{x\ln^2(x)} \ dx = \int \frac{1}{\ln^2(x)} \cdot \frac{1}{x}\  dx = \int \frac{1}{u^2} \ du.\]
The last integral can be computed as 
\[\int \frac{1}{u^2} \ du = \int u^{-2} \ du = \tfrac{u^{-1}}{-1} + C 
= -\frac{1}{u} + C,\]
and by back-substituting, we have 
\[-\frac{1}{u} + C =  -\frac{1}{\ln(x)} + C.\]
Thus, using $u$-substitution we can conclude that
\[\int \frac{1}{x\ln^2(x)} \ dx =  -\frac{1}{\ln(x)} + C.\]
\end{example}


\begin{problem}(problem 13)
\begin{hint}
Let $u = \ln(x)$
\end{hint}
\begin{hint}
Compute $du$
\end{hint}
\[\int \frac{1}{x\ln^3(x)} \ dx = \answer{- \ln^{-2}(x)/2} +C.\]
\end{problem}

\begin{example}[example 14] Compute 
\[\int \sin^4(x)\cos(x) \ dx.\]
Let $u = \sin(x)$. Then we have,
\[u = \sin(x)\]
\[du = \cos(x) \ dx\]
and the integral can be written as 
\[\int \sin^4(x)\cos(x) \ dx  = \int u^4 \ du.\]
The last integral can be computed as 
\[\int u^4 \ du = \tfrac{u^5}{5} + C,\]
and by back-substituting, we have 
\[\tfrac{u^5}{5} + C = \tfrac15 \sin^5(x) + C.\]
Thus, using $u$-substitution we can conclude that
\[\int \sin^4(x)\cos(x) \ dx =  \tfrac15 \sin^5(x) + C.\]
\end{example}



\begin{problem}(problem 14)
\begin{hint}
Let $u = \tan(x)$
\end{hint}
\begin{hint}
Compute $du$
\end{hint}
\[\int \tan^5(x)\sec^2(x) \ dx = \answer{\tan^6(x)/6} +C.\]
\end{problem}

\begin{example}[example 15] Compute 
\[\int \tan(x) \ dx.\]
First rewrite the integral:
\[\int \tan(x) \ dx =\int \frac{\sin(x)}{\cos(x)} \ dx.\]
Now, let $u = \cos(x)$; then $du = -\sin(x) \ dx$
and the integral can be written as
\begin{align*}
\int \frac{\sin(x)}{\cos(x)} \ dx &= \int \frac{1}{\cos(x)}\ \sin(x) \  dx \\
 &=  - \int \frac{1}{\cos(x)}\ \big(-\sin(x)\big) \  dx\\
&=-\int \frac{1}{u} \ du.
\end{align*}
The last integral can be computed as 
\[-\int \frac{1}{u} \ du = -\ln|u| + C,\]
and by back-substituting, we have 
\[-\ln|u| + C = -\ln|\cos(x)| + C = \ln|\sec(x)| +C.\]
Thus, using $u$-substitution we can conclude that
\[\int \tan(x) \ dx =  \ln|\sec(x)| + C.\]
\end{example}

\begin{problem}(problem 15)
\begin{hint}
Rewrite: $\cot(x) = \frac{\cos(x)}{\sin(x)}$
\end{hint}
\begin{hint}
Let $u = \sin(x)$
\end{hint}
\begin{hint}
Compute $du$
\end{hint}
\[\int \cot(x) \ dx = \answer{\ln|\sin(x)|} +C.\]


\end{problem}


\begin{example}[example 16] Compute 
\[\int \sec(x) \ dx.\]
First rewrite the integral:
\[\int \sec(x) \ dx =\int \sec(x)\frac{\sec(x)+\tan(x)}{\sec(x)+\tan(x)} \ dx.\]
Distributing in the numerator, we get
\[\int \sec(x)\frac{\sec(x)+\tan(x)}{\sec(x)+\tan(x)} \ dx = \int \frac{\sec^2(x)+\sec(x)\tan(x)}{\sec(x)+\tan(x)} \ dx.\]
Now, let $u = \sec(x) + \tan(x)$. Then $du = [\sec(x)\tan(x) + \sec^2(x)] \ dx$ and the integral can be rewritten as
\[ \int \frac{\sec^2(x)+\sec(x)\tan(x)}{\sec(x)+\tan(x)} \ dx = \int \frac{1}{u} \ du.\]
The last integral can be computed as
\[\int \frac{1}{u} \ du = \ln|u| + C,\]
and by back-substituting, we have 
\[\ln|u| + C = \ln|\sec(x) + \tan(x)| + C.\]
Thus, using $u$-substitution we can conclude that
\[\int \sec(x) \ dx =  \ln|\sec(x) + \tan(x)| + C.\]
\end{example}


\begin{center}
\begin{foldable}
\unfoldable{Here is a detailed, lecture style video on $u$-substitution:}
\youtube{5yb7_e9PCQU}
\end{foldable}
\end{center}

\end{document}





\begin{example}
Compute  $\displaystyle \int e^{3x} \ dx$\\
If we let $u = 3x$, the inside of the composition in the integrand, then we get $du = 3 dx$. We can rewrite 
this second equation as $dx = \frac13 du$.  Substituting $u$ for $3x$ and $\frac13 \ du$ for $dx$ in the original integral gives
\[
\int e^{3x} \ dx = \frac13 \int e^u \ du
\]
Note that the constant $\frac13$ has been moved to the front of the integral using the constant multiple rule.
We know an anti-derifvative of $e^u$ so we continue with
\[
\frac13 \int e^u \ du = \frac13 e^u + C
\]
Finally, back-substituting $3x$ for $u$ gives the final answer
\[
\int e^{3x} \ dx = \frac13 e^{3x} + C
\]
\end{example}

\begin{problem}
Compute the following integrals:
\begin{align*}
a) \;\; & \int e^{2x} \ dx = \; \answer{\frac12 e^2x+ C}  \\
b) \;\; & \int e^{-\frac{x}{2}} \ dx = \; \answer{-2 e^{-x/2}+ C}  \\
c) \;\; & \int 5\cos(x) \ dx = \; \answer{5\sin(x)+ C}  \\
d) \;\; & \int \cos(5x)\ dx = \; \answer{\frac15 \sin{5x}+ C}  \\
e) \;\; & \int 2\sin(\pi x) \ dx = \; \answer{-\frac{2}{\pi} \cos(\pi x)+ C} 
\end{align*}
\end{problem}

\begin{remark}
Integrands with $kx$ as an `inside' function arise frequently, so it is helpful if we can notice that
\[
\int f(kx) \ dx = \frac{1}{k} F(kx) + C
\]
where $F$ is an anti-derivative of $f$.
\end{remark}

\begin{example}
Compute  $\displaystyle \int \frac{3}{(2x+5)^2} \ dx$\\
If we let $u = 2x+5$, the inside of the composition in the integrand, then we get $du = 2 dx$. We can rewrite 
this second equation as $dx = \frac12 du$.  Substituting $u$ for $2x+5$ and $\frac12 \ du$ for $dx$ in the 
original integral gives
\[
\int \frac{3}{(2x+5)^2} \ dx = \frac32 \int \frac{1}{u^2} \ du
\]


Note that the constant $\frac32$ has been moved to the front of the integral using the constant multiple rule.
We use the power rule to find an anti-derivative of $\frac{1}{u^2} = u^{-2}$ so we continue with
\begin{align*}
\frac32 \int \frac{1}{u^2} \ du &= \frac32 \int u^{-2} \ du \\
                                &= \frac32 \left(\frac{u^{-1}}{-1}\right) + C\\
                                &= -\frac{3}{2u} + C
\end{align*}
Finally, back-substituting $2x+5$ for $u$ gives the final answer
\[
\int  \frac{3}{(2x+5)^2}  \ dx =-\frac{3}{4x+10} + C
\]

\end{example}
