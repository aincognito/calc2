\documentclass[handout]{ximera}


%% You can put user macros here
%% However, you cannot make new environments



\newcommand{\ffrac}[2]{\frac{\text{\footnotesize $#1$}}{\text{\footnotesize $#2$}}}
\newcommand{\vasymptote}[2][]{
    \draw [densely dashed,#1] ({rel axis cs:0,0} -| {axis cs:#2,0}) -- ({rel axis cs:0,1} -| {axis cs:#2,0});
}


\graphicspath{{./}{firstExample/}}

\usepackage{amsmath}
\usepackage{amssymb}
\usepackage{array}
\usepackage[makeroom]{cancel} %% for strike outs
\usepackage{pgffor} %% required for integral for loops
\usepackage{tikz}
\usepackage{tikz-cd}
\usepackage{tkz-euclide}
\usetikzlibrary{shapes.multipart}


\usetkzobj{all}
\tikzstyle geometryDiagrams=[ultra thick,color=blue!50!black]


\usetikzlibrary{arrows}
\tikzset{>=stealth,commutative diagrams/.cd,
  arrow style=tikz,diagrams={>=stealth}} %% cool arrow head
\tikzset{shorten <>/.style={ shorten >=#1, shorten <=#1 } } %% allows shorter vectors

\usetikzlibrary{backgrounds} %% for boxes around graphs
\usetikzlibrary{shapes,positioning}  %% Clouds and stars
\usetikzlibrary{matrix} %% for matrix
\usepgfplotslibrary{polar} %% for polar plots
\usepgfplotslibrary{fillbetween} %% to shade area between curves in TikZ



%\usepackage[width=4.375in, height=7.0in, top=1.0in, papersize={5.5in,8.5in}]{geometry}
%\usepackage[pdftex]{graphicx}
%\usepackage{tipa}
%\usepackage{txfonts}
%\usepackage{textcomp}
%\usepackage{amsthm}
%\usepackage{xy}
%\usepackage{fancyhdr}
%\usepackage{xcolor}
%\usepackage{mathtools} %% for pretty underbrace % Breaks Ximera
%\usepackage{multicol}



\newcommand{\RR}{\mathbb R}
\newcommand{\R}{\mathbb R}
\newcommand{\C}{\mathbb C}
\newcommand{\N}{\mathbb N}
\newcommand{\Z}{\mathbb Z}
\newcommand{\dis}{\displaystyle}
%\renewcommand{\d}{\,d\!}
\renewcommand{\d}{\mathop{}\!d}
\newcommand{\dd}[2][]{\frac{\d #1}{\d #2}}
\newcommand{\pp}[2][]{\frac{\partial #1}{\partial #2}}
\renewcommand{\l}{\ell}
\newcommand{\ddx}{\frac{d}{\d x}}

\newcommand{\zeroOverZero}{\ensuremath{\boldsymbol{\tfrac{0}{0}}}}
\newcommand{\inftyOverInfty}{\ensuremath{\boldsymbol{\tfrac{\infty}{\infty}}}}
\newcommand{\zeroOverInfty}{\ensuremath{\boldsymbol{\tfrac{0}{\infty}}}}
\newcommand{\zeroTimesInfty}{\ensuremath{\small\boldsymbol{0\cdot \infty}}}
\newcommand{\inftyMinusInfty}{\ensuremath{\small\boldsymbol{\infty - \infty}}}
\newcommand{\oneToInfty}{\ensuremath{\boldsymbol{1^\infty}}}
\newcommand{\zeroToZero}{\ensuremath{\boldsymbol{0^0}}}
\newcommand{\inftyToZero}{\ensuremath{\boldsymbol{\infty^0}}}


\newcommand{\numOverZero}{\ensuremath{\boldsymbol{\tfrac{\#}{0}}}}
\newcommand{\dfn}{\textbf}
%\newcommand{\unit}{\,\mathrm}
\newcommand{\unit}{\mathop{}\!\mathrm}
%\newcommand{\eval}[1]{\bigg[ #1 \bigg]}
\newcommand{\eval}[1]{ #1 \bigg|}
\newcommand{\seq}[1]{\left( #1 \right)}
\renewcommand{\epsilon}{\varepsilon}
\renewcommand{\iff}{\Leftrightarrow}

\DeclareMathOperator{\arccot}{arccot}
\DeclareMathOperator{\arcsec}{arcsec}
\DeclareMathOperator{\arccsc}{arccsc}
\DeclareMathOperator{\si}{Si}
\DeclareMathOperator{\proj}{proj}
\DeclareMathOperator{\scal}{scal}
\DeclareMathOperator{\cis}{cis}
\DeclareMathOperator{\Arg}{Arg}
%\DeclareMathOperator{\arg}{arg}
\DeclareMathOperator{\Rep}{Re}
\DeclareMathOperator{\Imp}{Im}
\DeclareMathOperator{\sech}{sech}
\DeclareMathOperator{\csch}{csch}
\DeclareMathOperator{\Log}{Log}

\newcommand{\tightoverset}[2]{% for arrow vec
  \mathop{#2}\limits^{\vbox to -.5ex{\kern-0.75ex\hbox{$#1$}\vss}}}
\newcommand{\arrowvec}{\overrightarrow}
\renewcommand{\vec}{\mathbf}
\newcommand{\veci}{{\boldsymbol{\hat{\imath}}}}
\newcommand{\vecj}{{\boldsymbol{\hat{\jmath}}}}
\newcommand{\veck}{{\boldsymbol{\hat{k}}}}
\newcommand{\vecl}{\boldsymbol{\l}}
\newcommand{\utan}{\vec{\hat{t}}}
\newcommand{\unormal}{\vec{\hat{n}}}
\newcommand{\ubinormal}{\vec{\hat{b}}}

\newcommand{\dotp}{\bullet}
\newcommand{\cross}{\boldsymbol\times}
\newcommand{\grad}{\boldsymbol\nabla}
\newcommand{\divergence}{\grad\dotp}
\newcommand{\curl}{\grad\cross}
%% Simple horiz vectors
\renewcommand{\vector}[1]{\left\langle #1\right\rangle}


\outcome{Estimate limits using numerical information}

\title{1.2 Numerical Limits}

\begin{document}

\begin{abstract}
We find limits using numerical information.
\end{abstract}

\maketitle


\section{Limit notation}
 


Limits are the backbone of calculus. A limit tells us the end of an infinite process. For example, consider the following infinite sequence of numbers:
\[ 0.9, 0.99, 0.999, 0.9999, 0.99999, 0.999999, ... \]
This infinite sequence of numbers is becoming arbitrarily close to the number 1, so we say the \textbf{limit} of the sequence is 1.
In calculus, we will be concerned with limits involving functions.
%A function is an input-output device involving numbers. 
The inputs of the function will undergo an infinite process which will then correspond to an infinite process for the outputs.  
Our goal will be to determine the limit of the outputs.
Suppose that $f$ represents a function of the input variable $x$.
We denote by $x \to c$ an infinite process where the inputs are becoming arbitrarily close to the value $c$ without ever reaching it.
Then, we denote by
\[ \lim_{x\to c} f(x) \]
the limit of the outputs of the function $f$ as $x$ approaches $c$.
The possibilities for the limit of the outputs of $f$ are: a numerical value, 
$\infty$, $-\infty$, or that the limit does not exist.
If the numerical value of the limit is $L$, then as the input variable $x$ approaches the value $c$, the outputs, $f(x)$ approach the value $L$, and we would write

\[ 
\lim_{x\to c} f(x) = L.
\]

If the value of the limit is $\infty$, then as $x \to c$, the outputs $f(x)$ are \textbf{increasing without bound},
and we write
\[
\lim_{x \to c} f(x) = \infty.
\]
  
Similarly, if the limit is $-\infty$, that means the outputs are decreasing without bound, and we write
\[
\lim_{x \to c} f(x) = -\infty.
\]


If none of the previous conclusions apply, we say that the limit \textbf{does not exist}, and we write
\[
\lim_{x \to c} f(x) \;\; \text{DNE}.
\] 

In addition to writing $x \to c$, we can also write the expression
$x \to c^-$ to indicate that $x$ is approaching $c$ from the left hand side and $x \to c^+$ to 
indicate that $x$ is approaching $c$ from the right hand side.

The expression $x \to c^-$ can be understood visually using the following diagram:




\begin{center}
\begin{tikzpicture}
\draw[<->, thick] (-3,0) -- (3,0);
\draw[thick] (-1,-.2) -- (-1,.2); %tick mark
\draw[thick] (1,-.2) -- (1,.2);
\node[below] at (-1,-.3) {$x$};
\node[below] at (1,-.3) {$c$};
\draw[->, thick] (-0.7,0.5) -- (0.7,0.5);
\draw[white](0, .6) -- (.1, .6);
\node[below] at (0,-0.8) {$x$ approaches $c$ from the left: $x \to c^-$};


\end{tikzpicture}

\end{center}

Note that $x \to c^-$ implies $x < c$.
Similarly, the expression $x \to c^+$ can be understood visually using the following diagram:



\begin{center}
\begin{tikzpicture}
\draw[<->, thick] (-3,0) -- (3,0);
\draw[thick] (-1,-.2) -- (-1,.2); %tick mark
\draw[thick] (1,-.2) -- (1,.2);
\node[below] at (-1,-.3) {$c$};
\node[below] at (1,-.3) {$x$};
\draw[<-, thick] (-0.7,0.5) -- (0.7,0.5);
\node[below] at (0,-0.8) {$x$ approaches $c$ from the right: $x \to c^+$};
\draw[white](0, .7) -- (.1, .7);
\end{tikzpicture}
\end{center}

Note that $x \to c^+$ implies $x > c$. We refer to the limits

\[\lim_{x \to c^-} f(x)  \quad \text{and} \quad \lim_{x \to c^+} f(x) \]

as \textbf{one-sided limits}, and we refer to 

\[\lim_{x \to c} f(x) \]

as a \textbf{two-sided limit}.


We can also let the inputs, $x$, either increase or decrease without bound, denoted by $x \to \infty$ and $x \to -\infty$ respectively.
These can be represented visually by the following diagrams:



\begin{center}
\begin{tikzpicture}[scale =0.7]
\draw[<->, thick] (-3,0) -- (3,0);
\draw[thick] (0,-.2) -- (0,.2); %tick mark
\node[below] at (0,-.3) {$x$};
\draw[->, thick] (0.3,0.5) -- (1.7,0.5);
\node[below] at (0,-1) {$x$ increases without bound: $x \to \infty$};
\draw[<->, thick] (-3,-3.5) -- (3,-3.5);
\draw[thick] (0,-3.3) -- (0,-3.7); %tick mark
\node[below] at (0,-3.7) {$x$};
\draw[<-, thick] (-1.7,-3) -- (-0.3,-3);
\node[below] at (0,-4.5) {$x$ decreases without bound: $x \to -\infty$};
\draw[white](0, .7) -- (.1, .7);
\end{tikzpicture}
\end{center}

We refer to the limits
\[\lim_{x \to \infty} f(x)  \quad \text{and} \quad \lim_{x \to -\infty} f(x) \]
as \textbf{limits at infinity}.


\section{Finding limits numerically}



In what follows, functions will be presented using formulas.  
We will determine the limit of a function by making an appropriate \textbf{table of values}.
 

\begin{example}[example 1]
Determine the following one-sided limit: 
\[\lim_{x \to 3^{+}} x^2 + 2.\]
First,  $x \to 3^{+}$ means that $x$ is approaching the number $3$ from the right, so $x > 3$. 
To represent this numerically, consider the following values for $x$: $3.1, 3.01, 3.001, 3.0001$.
These numbers are each greater than 3 and they are getting closer to 3.
 
Now, to determine the limit numerically,  we will plug these four values of $x$ into the function $x^2 + 2$ and  
look for a pattern in the outputs.

Here is the analysis:\\
If $x = 3.1$ then $x^2 + 2 = 11.61$.\\
If $x = 3.01$ then $x^2 + 2 = 11.0601$.\\
If $x = 3.001$ then $x^2 + 2 = 11.006001$.\\
If $x = 3.001$ then $x^2 + 2 = 11.00060001$.\\

We can summarize this information in a table:
  
\[
%\setlength{\extrarowheight}{5pt}
\begin{array}{ c | c | c | c | c }
  x & 3.1 & 3.01 & 3.001 & 3.0001 \\ 
	\hline
	x^2 + 2 & 11.61 & 11.0601 & 11.006001 & 11.00060001
\end{array}
\]


The outputs appear to be approaching the number 11, leading us to conclude that

\[\lim_{x \to 3^+} x^2 + 2 = 11.\] 
 
In this particular example, we could have arrived at the answer by simply plugging the number $x = 3$
into the function $x^2 + 2$ since $3^2 + 2 = 11$.  
\end{example}

In general, when \textbf{plugging in} the value that $x$ is approaching yields a number, 
that number is usually the correct answer.  



\begin{problem}(problem 1)
Compute the limit by plugging in the terminal value.
\[
\lim_{x \to -4} x^2 -3x -1.
\]

The limit is $\answer{27}$.
\end{problem}

We now turn our attention to examples where the shortcut of ``plugging in'' does not work.

\begin{example}[example 2]
Determine the following one-sided limit: 
\[\lim_{x \to 1^{+}} \frac{x^3 - 1}{x^2 -1}.\]
If we plug $x = 1$ into the function, we get the fraction  $0/0$, which is called an \textbf{indeterminate form}. 
This means that the answer can be anything and so a deeper analysis is required.

%In this particular example, we could not have arrived at the answer by simply plugging the number $x = 1$
%into the function $\frac{x^3 - 1}{x^2-1}$ since $\frac{1^3 - 1}{1^2-1} 
%= \frac00$, which is not a number. 

Since  $x \to 1^{+}$ we know that $x$ is approaching the number $1$ from the right side, so $x > 1$. 
To represent this numerically, consider the following values for $x$: $1.1, 1.01, 1.001, 1.0001$.
These numbers are each greater than 1 and they are getting closer to 1.
 
To determine the limit numerically,  we will plug these four values of $x$ into the function $\frac{x^3 - 1}{x^2-1}$ and  
look for a pattern in the outputs.

Here is the analysis:\\
If $x = 1.1$ then $\displaystyle{\frac{x^3 - 1}{x^2-1}= 1.57619}$.\\
If $x = 1.01$ then $\displaystyle{\frac{x^3 - 1}{x^2-1} = 1.507512}$.\\
If $x = 1.001$ then $\displaystyle{\frac{x^3 - 1}{x^2-1}= 1.500751}$.\\
If $x = 1.0001$ then $\displaystyle{\frac{x^3 - 1}{x^2-1}= 1.5000750}$.\\
If $x = 1.00001$ then $\displaystyle{\frac{x^3 - 1}{x^2-1}= 1.5000075}$.\\

We can summarize this information in a table:
  
\[
%\setlength{\extrarowheight}{5pt}
\begin{array}{ c | c | c | c | c | c}
  x & 1.1 & 1.01 & 1.001 & 1.0001 & 1.00001\\ 
	\hline
	\frac{x^3 - 1}{x^2-1} & 1.57619 & 1.507512 & 1.500751 & 1.500075 & 1.5000075
\end{array}
\]


The outputs appear to be approaching the number 1.5, leading us to conclude that

\[\lim_{x \to 1^+} \frac{x^3 - 1}{x^2-1} = 1.5.\] 
 
 
\end{example}


\begin{problem}(problem 2)
Fill in the table below with 5 decimal places of accuracy and use it to find the limit:
\[\lim_{x \to 2^+} \frac{x^3 - 8}{x^2 - 4}.\]

%\begin{prompt}
\begin{center}
\[
\begin{array}{ c | c | c }
  x & 2.1 & 2.01   \\ 
	\hline 
	 \frac{x^3 - 8}{x^2 - 4} & \answer{3.07561} & \answer{3.00751} 
\end{array}
\]
\[
\begin{array}{ c | c | c  }
  x  & 2.001 & 2.0001 \\ 
	\hline 
	 \frac{x^3 - 8}{x^2 - 4}  & \answer{3.00075} & \answer{3.00008}
\end{array}
\]
\end{center}
Now determine the limit:
\[
\lim_{x \to 2^+} \frac{x^3 - 8}{x^2 - 4} = \answer{3}
\]
%\end{prompt}
\end{problem}


The next example comes from solving a tangent line problem.



\begin{example}[example 3]
Compute the following limit:
\[\lim_{x \to 0^{+}} \frac{e^x-1}{x}.\]
In this example, $x$ is near $0$ and $x>0$.  The values of $x$ we will use to 
generate a table of values are: $0.1, 0.01, 0.001$ and $0.0001.$

\[
%\setlength{\extrarowheight}{5pt}
\begin{array}{ c | c | c | c | c }
  x & 0.1 & 0.01  & 0.001 & 0.0001 \\ 
	\hline
	 \frac{e^x - 1}{x} & 1.052 & 1.005 & 1.0005 & 1.00005
\end{array}
\]

Based on the numerical information provided in the table above, we conclude that 

\[\lim_{x \to 0^+} \frac{e^x - 1}{x} = 1.\]


Observe that plugging $x=0$ into the 
function $\displaystyle{f(x)= \frac{e^x -1}{x}}$ yields the indeterminate form $\frac{0}{0}$, which is not
the answer we sought.

\end{example}


\begin{problem}(problem 3a)
Fill in the table below with 5 decimal places of accuracy and use it to find the limit:
\[\lim_{x \to 0^-} \frac{e^x - 1}{x}.\]

\begin{prompt}
\begin{center}
\[
\begin{array}{ c | c | c }
  x & -0.1 & -0.01   \\ 
	\hline 
	 \frac{e^x - 1}{x} & \answer{0.95163} & \answer{0.99502} 
\end{array}
\]
\[
\begin{array}{ c | c | c  }
  x  & -0.001 & -0.0001 \\ 
	\hline 
	 \frac{e^x - 1}{x}  & \answer{0.99950} & \answer{0.99995}
\end{array}
\]
\end{center}
Now determine the limit:
\[
\lim_{x \to 0^-} \frac{e^x - 1}{x} = \answer{1}
\]
\end{prompt}

Based on the example above and the result of this problem, determine the two-sided limit:
\[
\lim_{x \to 0} \frac{e^x - 1}{x}.
\]

The limit is (DNE is a possibility) $\answer{1}$

\end{problem}


\begin{problem}(problem 3b)
Fill in the table below with 5 decimal places of accuracy and use it to find the limit:
\[\lim_{x \to 0^+} \frac{e^{2x} - 1}{x}.\]

\begin{prompt}
\begin{center}
\[
\begin{array}{ c | c | c }
  x & 0.1 & 0.01   \\ 
	\hline 
	 \frac{e^{2x} - 1}{x} & \answer{2.21403} & \answer{2.02013} 
\end{array}
\]
\[
\begin{array}{ c | c | c  }
  x  & 0.001 & 0.0001 \\ 
	\hline 
	 \frac{e^{2x} - 1}{x}  & \answer{2.00200} & \answer{2.00020}
\end{array}
\]
\end{center}
Now determine the limit:
\[
\lim_{x \to 0^+} \frac{e^{2x} - 1}{x} = \answer{2}
\]
\end{prompt}
\end{problem}



\begin{example}[example 4]
Compute the following limit: 
\[\lim_{x \to 0^+} \frac{\sin(x)}{x}.\]

%This is a two-sided limit.  We begin be computing each of the corresponding one-sided limits.
First, note that the shortcut of plugging $x=0$ into the 
function $\displaystyle{f(x)= \frac{\sin(x)}{x}}$ yields the indeterminate form $\frac{0}{0}$.
Hence, to compute this one-sided limit, we consider the following values for $x$: $0.1,\ 0.01,\ 0.001$ and $\ 0.0001$.
Plugging these values into the function, we generate the following table of values:

\[
%\setlength{\extrarowheight}{5pt}
\begin{array}{ c | c | c | c | c }
  x & 0.1 & 0.01  & 0.001 & 0.0001 \\ 
	\hline 
	 \frac{\sin(x)}{x} & 0.9983 & 0.99998 & 0.9999998 & 0.999999998
\end{array}
\]  

Based on this numerical evidence, 
it would be reasonable to guess that 
\[\lim_{x \to 0^{+}} \frac{\sin(x)}{x} = 1.\]

%To compute the left side limit, we can consider the 
%following values for $x$: $-0.1, -0.01, -0.001$ and $-0.0001$.

%However, we can also make the following observation about the sine function:
%\[\sin(-x) = -\sin(x),\]
%which is to say that the sine is an \textbf{odd function}.

%As a result, we have the following

%\[\frac{\sin(-x)}{-x} = \frac{-\sin(x)}{-x} = \frac{\sin(x)}{x},\]

%which means that the suggested negative values for $x$ in the left side limit 
%will produce exactly the same values 
%as their positive counterparts that were used in the right side limit.

%Thus,
%\[[\lim_{x \to 0^{-}} \frac{\sin(x)}{x} =\lim_{x \to 0^{+}} \frac{\sin(x)}{x} = 1.\]

%Moreover, since both one sided limits are the same, we can dispense with the direction and conclude

%\[\lim_{x \to 0} \frac{\sin(x)}{x} = 1.\]





\end{example}

	   

\begin{problem}(problem 4a)
Fill in the table below with 5 decimal places of accuracy and use it to find the limit:
\[\lim_{x \to 0^-} \frac{\sin(x)}{x}.\]

\begin{prompt}
\begin{center}
\[
\begin{array}{ c | c | c }
  x & -0.1 & -0.01   \\ 
	\hline 
	 \frac{\sin(x)}{x} & \answer{0.99833} & \answer{0.99998} 
\end{array}
\]
\[
\begin{array}{ c | c | c  }
  x  & -0.001 & -0.0001 \\ 
	\hline 
	 \frac{\sin(x)}{x}  & \answer{0.99999} & \answer{0.99999}
\end{array}
\]
\end{center}
Now determine the limit:
\[
\lim_{x \to 0^-} \frac{\sin(x)}{x} = \answer{1}
\]
\end{prompt}

Based on the example above and the result of this problem, determine the two-sided limit:
\[
\lim_{x \to 0} \frac{\sin(x)}{x}.
\]

The limit is (DNE is a possibility) $\answer{1}$
\end{problem}



\begin{problem}(problem 4b)
Fill in the table below with 5 decimal places of accuracy and use it to find the limit:
\[\lim_{x \to 0^+} \frac{\sin(3x)}{x}.\]

\begin{prompt}
\begin{center}
\[
\begin{array}{ c | c | c }
  x & 0.1 & 0.01   \\ 
	\hline 
	 \frac{\sin(3x)}{x} & \answer{2.95520} & \answer{2.99955} 
\end{array}
\]
\[
\begin{array}{ c | c | c  }
  x  & 0.001 & 0.0001 \\ 
	\hline 
	 \frac{\sin(3x)}{x}  & \answer{3} & \answer{3}
\end{array}
\]
\end{center}
Now determine the limit:
\[
\lim_{x \to 0^+} \frac{\sin(3x)}{x} = \answer{3}
\]
\end{prompt}
\end{problem}



\begin{example}[example 5]
Consider the limit:
\[\lim_{x \to -2^-} \frac{3}{x+2}. \]
First we observe that plugging in the value $x=-2$ gives $\frac{3}{0}$ which is undefined. 
Thus we are required to make a deeper analysis to determine the limit.
Since the values of $x$ are less than $-2$, we consider values for $x$ such as 
$-2.1, -2.01, -2.001,$ and $-2.0001$ to make the following table of values:\\

\[
%\setlength{\extrarowheight}{5pt}
\begin{array}{ c | c | c | c | c| c| c}
  x & -2.1 & -2.01 & -2.001 & -2.0001 & -2.00001 & -2.000001\\ 
	\hline  
	 \frac{3}{x+2} & -30 & -300 & -3,000 & -30,000 & -300,000 & -3,000,000
\end{array}
\] 

The numerical evidence suggests that as $x$ approaches $-2$ from the left, 
the values of $f(x) = \frac{3}{x+2}$ are decreasing without bound. We conclude that 

\[\lim_{x \to -2^-} \frac{3}{x+2} =  -\infty. \]

This result has geometric significance.  
It means that the line $x=-2$ is a vertical asymptote for the graph of the function $f(x) = \frac{3}{x+2}.$
\end{example}



\begin{problem}(problem 5)
Fill in the table below and use it to find the limit:
\[\lim_{x \to 3^+} \frac{x}{x-3}.\]

\begin{prompt}
\begin{center}
\[
\begin{array}{ c | c | c }
  x & 3.1 & 3.01   \\ 
	\hline 
	 \frac{x}{x-3} & \answer{31} & \answer{301} 
\end{array}
\]
\[
\begin{array}{ c | c | c  }
  x  & 3.001 & 3.0001 \\ 
	\hline 
	 \frac{x}{x-3}  & \answer{3001} & \answer{30001}
\end{array}
\]
\end{center}
Now determine the limit (type infinity for $\infty$ and -infinity for $-\infty$):
\[
\lim_{x \to 3^+} \frac{x}{x-3} = \answer{\infty}
\]
\end{prompt}
\end{problem}


In the following example, we discuss limits as the input $\infty$.
If $x \to \infty$ then $x$ is \textbf{increasing without bound} and we can use very large numbers for $x$ in our table.
%Similarly, for $x \to -\infty$.

\begin{example}[example 6]
Consider the limit:
\[\lim_{x \to \infty} \frac{3x -2}{x + 3}. \]

Since $x \to \infty$, we will use powers of ten to generate large values of $x$ when constructing 
our table: 
%$100,\; 1{,}000 ,\; 10{,}000,\; 100{,}000$ and $1{,}000{,}000$

\[
%\setlength{\extrarowheight}{5pt}
\begin{array}{ c | c | c | c | c|c}
  x & 100 & 1{,}000 & 10{,}000 & 100{,}000 & 1{,}000{,}000\\ 
	\hline
	 \frac{3x-2}{x+3} & 2.8932 & 2.9890 & 2.9989 & 2.99989 & 2.999989
	\end{array}
\] 

The numerical evidence suggests that as $x$ approaches $\infty$, that is, 
as $x$ increases without bound, the values of $\frac{3x -2}{x +3}$ are approaching the number $3$. Hence,



\[\lim_{x \to \infty} \frac{3x -2}{x +3} = 3. \]

This result has geometric significance.  It means that the line $y = 3$ is a horizontal 
asymptote for the graph of the function $f(x) = \frac{3x -2}{x +3}.$
\end{example}


\begin{problem}[problem 6]
Fill in the table below using \textbf{fractions} and use it to find the limit:
\[\lim_{x \to \infty} \frac{3x-2}{5x+6}.\]

\begin{prompt}
\begin{center}
\[
\begin{array}{ c | c | c }
  x & 10 & 100   \\ 
	\hline 
	 \frac{3x-2}{5x+6} & \answer{28/56} & \answer{298/506} 
\end{array}
\]
\[
\begin{array}{ c | c | c  }
  x  & 1000 & 10000 \\ 
	\hline 
	 \frac{3x-2}{5x+6}  & \answer{2998/5006} & \answer{29998/50006}
\end{array}
\]
\end{center}
Now determine the limit:
\[
\lim_{x \to \infty} \frac{3x-2}{5x+6} = \answer{3/5}
\]
\end{prompt}
\end{problem}


\section{Limits and the number ${\bf e}$}


We now consider a famous example involving compound interest and the number $e$.
The compound interest formula says 
\[ A = P\left(1+\frac{r}{n}\right)^{nt} \]
where $A$ represents the amount of money resulting from investing a principle $P$ at an annual rate $r$ with 
the interest compounded $n$ times per year for $t$ years.  A basic fact about compound interest is that the more frequently 
the interest is compounded, the faster the amount of money will grow. 
So, a natural question is ``what happens as $n \to \infty?$''
%To answer this question, we will simplify matters by letting $P = 1, r = 1$ and $t=1$.In this case, 
%\[A = \left(1+\frac{1}{n}\right)^n,\]

%and we will try to find
%\[\lim_{n \to \infty} \left(1+\frac{1}{n}\right)^n.\]


\begin{example}[example 7]
If we let $P=1, r= 1$ and $t=1$ in the compound interest formula above, we get:
\[
A = \left(1 + \frac{1}{n}\right)^n.
\]
 We will find the limit of $A$ as $n \to \infty$, i.e., 

\[
\lim_{n \to \infty} \left(1+\frac{1}{n}\right)^n.
\]

Let's construct a table of values with large values of $n$ and look for a pattern in the outputs:
%\setlength{\extrarowheight}{5pt}
\[
\begin{array}{ c | c | c | c | c|c}
  n & 100 & 1,000 & 10,000 & 100,000 & 1,000,000\\ 
	\hline
	A & 2.70481 & 2.71692 & 2.71815 & 2.71827 & 2.71828
	\end{array}
\] 

What we can see from this table is that even with the number of compounding periods being as large as 1,000,000 per year, 
the principle of \$1 will not even grow to \$3 by the end of the year.  So there appears to be a limit to the effect of increasing the 
number of compounding periods on the amount of money generated by compound interest at a rate of 100\%. 
This limit is the famous number $e$ that we see in the exponential function $e^x$ and as the base of the natural logarithm, $\ln x$.

In conclusion, we have discovered the famous limit

\[\lim_{n \to \infty} \left(1+\frac{1}{n}\right)^n = e,\]

which can be used to define the number $e$.

The renowned Swiss mathematician Leonhard Euler (1707-1783) was the first to refer to this number as $e$,
standing for "exponential".  $e$ is an irrational number, like $\pi$ and $\sqrt 2$ and to fifteen decimal places it is:

\[ e = 2.718281828459045... \]

\end{example}




\begin{problem}(problem 7)
Fill in the table below with 5 decimal places of accuracy and use it to find the limit:
\[\lim_{x \to -\infty} \left(1+\frac{1}{x}\right)^x.\]

\begin{prompt}
\begin{center}
\[
\begin{array}{ c | c | c }
  x & -100 & -1,000   \\ 
	\hline 
	 \left(1+\frac{1}{x}\right)^x & \answer{2.73200} & \answer{2.71964} 
\end{array}
\]
\[
\begin{array}{ c | c | c  }
  x  & -10,000 & -100,000 \\ 
	\hline 
	 (1+\frac{1}{x})^x  & \answer{2.71842} & \answer{2.71830}
\end{array}
\]
\end{center}
Now determine the limit (the answer is a well known number, denoted by a single letter):
\[
\lim_{x \to -\infty} (1+\frac{1}{x})^x = \answer{e}
\]
\end{prompt}
\end{problem}



\section{Limits and instantaneous velocity}


\textbf{Rectilinear motion} is motion along a straight line. We will consider an object to be in motion along a number line to keep track of its location.
We let the function $s(t)$ denote the \textbf{position} of the object at time $t$. Then the displacement of the object over a time interval,
$[t_1, t_2]$ is given by $s(t_2) - s(t_1)$.  If we divide the displacement by the duration of the time interval, we get the 
\textbf{average velocity} of the object over that time interval:
\[
\text{average velocity} = \frac{s(t_2) - s(t_1)}{t_2 -t_1}.
\]


Our goal is to determine the \textbf{instantaneous velocity} of the 
object at a given time.  To do this, we will consider time intervals of shorter and shorter duration.


\begin{example}[example 8]
Suppose the position of a falling object is given by $s(t) = 100-16t^2$ where $t$ is measured in seconds and $s(t)$ is measured in feet.
Find the instantaneous velocity of the object at time $t = 2$ seconds.\\
The average velocity of the object over the time interval $[t,2]$ is given by 
\[
\text{ave vel} = \frac{s(2)-s(t)}{2-t} = \frac{36-s(t)}{2-t},
\]
since $s(2) = 100 - 16(2^2) = 36$ feet.
To obtain the instantaneous velocity, we will look for a pattern in the average velocity as the time interval $[t, 2]$ gets shorter and shorter:
\[
\begin{array}{ c | c | c | c | c }
 t & 1.9 & 1.99  & 1.999 & 1.9999 \\ 
	\hline 
	 \text{ave vel} & -62.4 & -63.84 & -63.984 & -63.9984
\end{array}
\]  
It appears that as the average velocities are approaching the value $-64$ ft/sec, as the time intervals 
get shorter and shorter (negative velocity just means the object is falling).  Hence we conclude that the instantaneous velocity at time $t = 2$ seconds is $-64$ ft/sec.
Moreover, in terms of limits, the instantaneous velocity, $v(t)$, is a limit of average velocities:
\[
v(2) = \lim_{t\to 2^-} \frac{s(2) - s(t)}{2-t}.
\]
Technically, in this example we only considered the left hand limit, $t\to 2^-$.
In the next problem, we will verify that the right hand limit gives the same value. 
\end{example}

\begin{problem}(problem 8)
Suppose the position of a falling object is given by $s(t) = 100-16t^2$ where $t$ is measured in seconds and $s(t)$ is measured in feet.
Find the value of the limit
\[
\lim_{t\to 2^+} \frac{s(t) - s(2)}{t-2},
\]
by filling in the table below.

\begin{prompt}
\begin{center}
\[
\begin{array}{ c | c | c }
  t & 2.1 & 2.01   \\ 
	\hline 
	 \text{ave vel} & \answer{-65.6} & \answer{-64.16} 
\end{array}
\]
\[
\begin{array}{ c | c | c  }
  t  & 2.001 & 2.0001 \\ 
	\hline 
	 \text{ave vel} & \answer{-64.016} & \answer{-64.0016}
\end{array}
\]
\end{center}
Now determine the instantaneous velocity as a limit of the average velocities:
\[
v(2) = \lim_{t\to 2^+} \frac{s(t) - s(2)}{t-2} = \answer{-64} \text{ft/sec}.
\]
\end{prompt}



\end{problem}


\begin{center}
\begin{foldable}
\unfoldable{Here is a detailed, lecture style video on numerical limits:}
\youtube{YX-gSPKPCTU}
\end{foldable}
\end{center}







\end{document}






























\begin{example}[example 3]
Compute the following one-sided limit: 
\[\lim_{x \to 3^{-}} \frac{|x-3|}{x-3}.\]
In this limit, $x$ is approaching $3$ from the left, so $x<3$. 
We therefore consider the following numerical values for $x$: $2.9, 2.99, 2.999, 2.9999$.
These values are all less than three and they are getting closer to $3$.
Plugging these values into the function $|x-3|/(x-3)$ yields the
following table of values: 

  
\[
%\setlength{\extrarowheight}{5pt}
\begin{array}{ c | c | c | c | c }
  x & 2.9 & 2.99 & 2.999 & 2.9999 \\ 
	\hline
	 \frac{|x-3|}{x-3} & -1 & -1 & -1 & -1  
\end{array}
\]

The values of the function are are all equal to $-1$, 
and so we can conclude that 
\[\lim_{x \to 3^{-}} \frac{|x-3|}{x-3} = -1.\]
It should be noticed that the shortcut of plugging in $x=3$ yields the 
meaningless \textbf{indeterminate form} $\frac00$.\\
\end{example}



\begin{problem}[problem 3]
Fill in the table below with 5 decimal places of accuracy and use it to find the limit:
\[\lim_{x \to -3^-} \frac{2x+6}{|x+3|}.\]

\begin{prompt}
\begin{center}
\[
\begin{array}{ c | c | c }
  x & -3.1 & -3.01   \\ 
	\hline 
	 \frac{2x+6}{|x+3|} & \answer{-2} & \answer{-2} 
\end{array}
\]
\[
\begin{array}{ c | c | c  }
  x  & -3.001 & -3.0001 \\ 
	\hline 
	 \frac{2x+6}{|x+3|}  & \answer{-2} & \answer{-2}
\end{array}
\]
\end{center}
Now determine the limit:
\[
\lim_{x \to -3^-} \frac{2x+6}{|x+3|} = \answer[given]{-2}
\]
\end{prompt}
\end{problem}









\begin{example}[example 7]
Consider the limit:
\[\lim_{x \to \frac{\pi}{2}^-} \tan(x). \]
Since
\[\tan(x) = \frac{\sin(x)}{\cos(x)},\]
\[\sin\big(\frac{\pi}{2}\big) = 1 \ \text{and} \ \cos\big(\frac{\pi}{2}\big) = 0\]
we have
 \[\tan\big(\frac{\pi}{2}\big) = \frac{1}{0}\]
which is undefined, so a deeper analysis is required to determine the limit.
To understand the behavior of the tangent function for values of $x$ near $\frac{\pi}{2}$, 
use a calculator to make a table of values noting that $\frac{\pi}{2} \approx 1.570796327$.\\

\[
%\setlength{\extrarowheight}{5pt}
\begin{array}{ c | c | c | c | c | c}
  x & 1.5 & 1.57 & 1.5707  & 1.57079 & 1.570796 \\ 
	\hline
  \tan(x) & 14.101 & 1,255.77 & 10,381.33 & 158,057.91  & 3,060,023.31 
\end{array}
\] 

The numerical evidence suggests that as $x$ approaches $\frac{\pi}{2}$ from the left, 
the values of $f(x) = \tan(x)$ are 
increasing without bound.
Therefore, we are led to conclude that 

\[\lim_{x \to \frac{\pi}{2}^-} \tan(x) = \infty. \]

This result has geometric significance.  It means that the graph of the function $f(x) = \tan(x)$ has a 
vertical asymptote at $x=\frac{\pi}{2}.$
\end{example}



\begin{problem}(problem 7)
Fill in the table below with two decimal places of accuracy and use it to find the limit:
\[\lim_{x \to \frac{\pi}{2}^+} \tan(x).\]

\begin{prompt}
\begin{center}
\[
\begin{array}{ c | c | c }
  x & 1.6 & 1.58   \\ 
	\hline 
	 \tan(x) & \answer{-34.23} & \answer{-108.65} 
\end{array}
\]
\[
\begin{array}{ c | c | c  }
  x  & 1.571 & 1.5708 \\ 
	\hline 
	 \tan(x)  & \answer{-4909.83} & \answer{-272241.81}
\end{array}
\]
\end{center}
Now determine the limit (type infinity for $\infty$ and -infinity for $-\infty$):
\[
\lim_{x \to \frac{\pi}{2}^+} \tan(x) = \answer[given]{-\infty}
\]
\end{prompt}
\end{problem}




\begin{example}[example 9]
Consider the limit:
\[\lim_{x \to -\infty} \frac{2x + 1}{3x -4}. \]

Since $x \to -\infty$, we will use negative powers of ten when constructing 
our table: 
%$-100,\; -1{,}000,\; -10{,}000 , \; -100{,}000$ and $-1{,}000{,}000.$

\[
%\setlength{\extrarowheight}{5pt}
\begin{array}{ c | c | c | c | c |c}
  x & -100 & -1{,}000 & -10{,}000 & -100{,}000 & -1{,}000{,}000\\ 
	\hline
	 \frac{2x + 1}{3x -4} &  .6546 & .6654 & .6665 & .66665  & .666665
	\end{array}
\] 

The numerical evidence suggests that as $x$ approaches $-\infty$, that is, 
as $x$ decreases without bound, the values of $\frac{2x + 1}{3x -4}$ are approaching 
the decimal $0.6666...$ which we recognize as the fraction $2/3$.  Hence,



\[\lim_{x \to -\infty} \frac{2x + 1}{3x -4} = \frac{2}{3}. \]

This result has geometric significance.  It means that the line $y = 2/3$ is a horizontal 
asymptote for the graph of the function $f(x) = \frac{2x + 1}{3x -4}.$
\end{example}

















\begin{example}
Given that the position of an object moving along a straight line is given by $s(t) = t^2 + t$ (feet), 
find the instantaneous velocity of the object at time $t = 2$ (seconds). \\
To accomplish this, we will compute the average velocity of the object over the time intervals $[2, 2.1], [2, 2.01],[2, 2.001]$ and $[2, 2.0001]$.
If we use the variable $h$ to denote the duration of the time interval, then the values of $h$ are $0.1, 0.01, 0.001$ and $0.0001$ seconds, and we find the instantaneous velocity by letting $h \to 0$. In effect, we will be computing the following limit numerically:
\[
\lim_{h \to 0^+} \frac{s(2+h) - s(2)}{h}.
\]
As in the previous examples, we will set up a table of values. Note that $s(2) = 2^2 + 2 = 6$.

\[
%\setlength{\extrarowheight}{5pt}
\begin{array}{ c | c | c | c | c }
 h & 0.1 & 0.01  & 0.001 & 0.0001 \\ 
	\hline 
	 \frac{s(2+h) - s(2)}{h} & 5.1 & 5.01 & 5.001 & 5.0001
\end{array}
\]  

From this numerical information, it seems clear that

\[
\lim_{h \to 0^+} \frac{s(2+h) - s(2)}{h} = 5
\]

and hence the instantaneous velocity of the object at time $t = 2$ is 5 ft/sec. 
It should be noted that the instantaneous velocity could have also been obtained by using a left hand limit, i.e., with $h \to 0^-$.
\end{example}

\begin{problem}
Given that the position of an object moving along a straight line is given by $s(t) = t^2 + 2t$ (feet), 
find the instantaneous velocity of the object at time $t = 3$ (seconds). \\

Begin by filling in the following table of average velocities over time intervals of the form $[3, 3+h]$.
Note that $h$ represents the duration of the time interval.
\begin{prompt}
\begin{center}
\[
\begin{array}{ c | c | c }
  h & 0.1 & 0.01   \\ 
	\hline 
	 \frac{s(3+h) - s(3)}{h} & \answer{8.1} & \answer{8.01} 
\end{array}
\]
\[
\begin{array}{ c | c | c  }
  h  & 0.001 & 0.0001 \\ 
	\hline 
	 \frac{s(3+h) - s(3)}{h} & \answer{8.001} & \answer{8.0001}
\end{array}
\]
\end{center}
Now determine the instantaneous velocity as a limit of the average velocities:
\[
\lim_{h \to 0^+} \frac{s(3+h) - s(3)}{h} = \answer[given]{8} \text{ft/sec}
\]
\end{prompt}
\end{problem}

\begin{problem}
Fill in the truth table below using your amazing logic skillz!

\begin{prompt}
\begin{center}
\[
\begin{array}{c|c|c|c}
		p & q & p \implies q & p \vee (p \implies q) \\
		\hline
		T & T & T & \answer{T} \\
		T & F & F & \answer{T} \\
		F & T & T & \answer{T} \\
		F & F & T & \answer{T}
	\end{array}
    \]
\end{center}
\end{prompt}
\end{problem}





\begin{problem}[problem 11a]
Suppose the position of a falling object is given by $s(t) = 100-16t^2$ where $t$ is measured in seconds and $s(t)$ is measured in feet.
Find the average velocity of the object over the time interval $[1.5, 2]$.\\

\begin{hint}
A negative velocity indicates that the object is falling.
\end{hint}

The average velocity is $\answer{-56}$ ft/sec.
\end{problem}
