\documentclass{ximera}
\usepackage{tcolorbox}
%% You can put user macros here
%% However, you cannot make new environments



\newcommand{\ffrac}[2]{\frac{\text{\footnotesize $#1$}}{\text{\footnotesize $#2$}}}
\newcommand{\vasymptote}[2][]{
    \draw [densely dashed,#1] ({rel axis cs:0,0} -| {axis cs:#2,0}) -- ({rel axis cs:0,1} -| {axis cs:#2,0});
}


\graphicspath{{./}{firstExample/}}

\usepackage{amsmath}
\usepackage{amssymb}
\usepackage{array}
\usepackage[makeroom]{cancel} %% for strike outs
\usepackage{pgffor} %% required for integral for loops
\usepackage{tikz}
\usepackage{tikz-cd}
\usepackage{tkz-euclide}
\usetikzlibrary{shapes.multipart}


\usetkzobj{all}
\tikzstyle geometryDiagrams=[ultra thick,color=blue!50!black]


\usetikzlibrary{arrows}
\tikzset{>=stealth,commutative diagrams/.cd,
  arrow style=tikz,diagrams={>=stealth}} %% cool arrow head
\tikzset{shorten <>/.style={ shorten >=#1, shorten <=#1 } } %% allows shorter vectors

\usetikzlibrary{backgrounds} %% for boxes around graphs
\usetikzlibrary{shapes,positioning}  %% Clouds and stars
\usetikzlibrary{matrix} %% for matrix
\usepgfplotslibrary{polar} %% for polar plots
\usepgfplotslibrary{fillbetween} %% to shade area between curves in TikZ



%\usepackage[width=4.375in, height=7.0in, top=1.0in, papersize={5.5in,8.5in}]{geometry}
%\usepackage[pdftex]{graphicx}
%\usepackage{tipa}
%\usepackage{txfonts}
%\usepackage{textcomp}
%\usepackage{amsthm}
%\usepackage{xy}
%\usepackage{fancyhdr}
%\usepackage{xcolor}
%\usepackage{mathtools} %% for pretty underbrace % Breaks Ximera
%\usepackage{multicol}



\newcommand{\RR}{\mathbb R}
\newcommand{\R}{\mathbb R}
\newcommand{\C}{\mathbb C}
\newcommand{\N}{\mathbb N}
\newcommand{\Z}{\mathbb Z}
\newcommand{\dis}{\displaystyle}
%\renewcommand{\d}{\,d\!}
\renewcommand{\d}{\mathop{}\!d}
\newcommand{\dd}[2][]{\frac{\d #1}{\d #2}}
\newcommand{\pp}[2][]{\frac{\partial #1}{\partial #2}}
\renewcommand{\l}{\ell}
\newcommand{\ddx}{\frac{d}{\d x}}

\newcommand{\zeroOverZero}{\ensuremath{\boldsymbol{\tfrac{0}{0}}}}
\newcommand{\inftyOverInfty}{\ensuremath{\boldsymbol{\tfrac{\infty}{\infty}}}}
\newcommand{\zeroOverInfty}{\ensuremath{\boldsymbol{\tfrac{0}{\infty}}}}
\newcommand{\zeroTimesInfty}{\ensuremath{\small\boldsymbol{0\cdot \infty}}}
\newcommand{\inftyMinusInfty}{\ensuremath{\small\boldsymbol{\infty - \infty}}}
\newcommand{\oneToInfty}{\ensuremath{\boldsymbol{1^\infty}}}
\newcommand{\zeroToZero}{\ensuremath{\boldsymbol{0^0}}}
\newcommand{\inftyToZero}{\ensuremath{\boldsymbol{\infty^0}}}


\newcommand{\numOverZero}{\ensuremath{\boldsymbol{\tfrac{\#}{0}}}}
\newcommand{\dfn}{\textbf}
%\newcommand{\unit}{\,\mathrm}
\newcommand{\unit}{\mathop{}\!\mathrm}
%\newcommand{\eval}[1]{\bigg[ #1 \bigg]}
\newcommand{\eval}[1]{ #1 \bigg|}
\newcommand{\seq}[1]{\left( #1 \right)}
\renewcommand{\epsilon}{\varepsilon}
\renewcommand{\iff}{\Leftrightarrow}

\DeclareMathOperator{\arccot}{arccot}
\DeclareMathOperator{\arcsec}{arcsec}
\DeclareMathOperator{\arccsc}{arccsc}
\DeclareMathOperator{\si}{Si}
\DeclareMathOperator{\proj}{proj}
\DeclareMathOperator{\scal}{scal}
\DeclareMathOperator{\cis}{cis}
\DeclareMathOperator{\Arg}{Arg}
%\DeclareMathOperator{\arg}{arg}
\DeclareMathOperator{\Rep}{Re}
\DeclareMathOperator{\Imp}{Im}
\DeclareMathOperator{\sech}{sech}
\DeclareMathOperator{\csch}{csch}
\DeclareMathOperator{\Log}{Log}

\newcommand{\tightoverset}[2]{% for arrow vec
  \mathop{#2}\limits^{\vbox to -.5ex{\kern-0.75ex\hbox{$#1$}\vss}}}
\newcommand{\arrowvec}{\overrightarrow}
\renewcommand{\vec}{\mathbf}
\newcommand{\veci}{{\boldsymbol{\hat{\imath}}}}
\newcommand{\vecj}{{\boldsymbol{\hat{\jmath}}}}
\newcommand{\veck}{{\boldsymbol{\hat{k}}}}
\newcommand{\vecl}{\boldsymbol{\l}}
\newcommand{\utan}{\vec{\hat{t}}}
\newcommand{\unormal}{\vec{\hat{n}}}
\newcommand{\ubinormal}{\vec{\hat{b}}}

\newcommand{\dotp}{\bullet}
\newcommand{\cross}{\boldsymbol\times}
\newcommand{\grad}{\boldsymbol\nabla}
\newcommand{\divergence}{\grad\dotp}
\newcommand{\curl}{\grad\cross}
%% Simple horiz vectors
\renewcommand{\vector}[1]{\left\langle #1\right\rangle}


\outcome{Find the critical numbers of a function.}

\title{3.1 Critical Numbers}

%\newcommand{\ffrac}[2]{\frac{\mbox{\footnotesize $#1$}}{\mbox{\footnotesize $#2$}}}
%\newcommand{\vasymptote}[2][]{
 %   \draw [densely dashed,#1] ({rel axis cs:0,0} -| {axis cs:#2,0}) -- ({rel axis cs:0,1} -| {axis cs:#2,0});
%}


\begin{document}

\begin{abstract}
In this section we learn to find the critical numbers of a function.
\end{abstract}

\maketitle









\section{Critical Numbers}

In this section we will find the {\bf critical numbers} of a given function. 
Critical numbers come in two main types and the idea of the definition is to consider the possibilities at a local extreme. Here is the definition of  critical number.  \\

\begin{image}
\begin{tikzpicture}
\begin{axis}[axis x line =center, axis y line = center, xmin = -3.5, xmax = 3.5, ymin = -3.5, ymax = 3.5, xtick = {-1, 1}, xticklabels={$a$, $b$}, 
ytick={},yticklabels={}, title = {Local extremes at $x = a$ and $x = b$}]
\addplot[<-,blue,domain=-3.3:1,thick] {2-(x+1)^2};
\addplot[->,blue,domain=1:3.2,thick] {2.5*x-4.5};
\addplot[blue, mark = *] coordinates{(-1,2)};
\addplot[blue, mark = *] coordinates{(1,-2)};
\end{axis}
\end{tikzpicture}
\end{image}


%\begin{center}
%\textbf{Definition of Critical Number}
%\end{center}

\begin{definition}
A critical number for the function $f(x)$ is a number 
$x_0$ in the {\it domain} of the function $f(x)$ such that either
\[f'(x_0) = 0 \text{  (type 1)}\]
\[ \text{or} \]
\[f'(x_0)  \text{   is undefined  (type 2)}.\]
\end{definition}

In other words, at a critical number $x_0$ we have  $f(x_0)$ is defined and either $f(x)$ is not differentiable at $x_0$ 
 or $f'(x_0) = 0$.

The function $f(x) = x^2$ has a type 1 critical number at $x = 0$   
since $f'(0) = 0$. And the function $f(x) = |x|$ has a type 2 critical number at $x = 0$ 
since $|x|$ is not differentialble at $x=0$ due to a corner point. The figure below shows various situations that can occur at a critical number.



\begin{image}
\begin{tikzpicture}
\begin{axis}[axis x line =center, axis y line = center, xmin = -3.5, xmax = 3.5, ymin = -3.5, ymax = 3.5, xtick = {-1, 1}, xticklabels={}, 
ytick={},yticklabels={}, title = {Type 1 critical number at $x = 0$}]


\addplot[domain=-3:3, samples = 100, color=blue, thick]{2-1/2*x^2};
\addplot[blue,mark = *] coordinates {(0,2)};
\addplot[red, thick, domain=-1:1]{2};

\node[color=blue] at (axis cs:2.4,1.2) {$y = 2-x^2$};
\end{axis}
\end{tikzpicture}
\hspace{1 in}
\begin{tikzpicture}

\begin{axis}[axis x line =center, axis y line = center, xmin = -3.5, xmax = 3.5, ymin = -3.5, ymax = 3.5, xtick = {-1, 1}, xticklabels={}, 
ytick={},yticklabels={}, title = {Type 1 critical number at $x = 0$}]


\addplot[domain=-3:3, samples = 100, color=blue, thick]{1+x^3/9} ;
\addplot[blue,mark = *] coordinates {(0,1)};
\addplot[red, thick, domain=-1:1]{1};

\node[color=blue] at (axis cs:-1.7,-1.5) {$y = 1 + x^3$};

\end{axis}
\end{tikzpicture}

\end{image}

\begin{image}
\begin{tikzpicture}

\begin{axis}[axis x line =center, axis y line = center, xmin = -3.5, xmax = 3.5, ymin = -3.5, ymax = 3.5, xtick = {-1, 1}, xticklabels={}, 
ytick={},yticklabels={}, title = {Type 2 critical number at $x = 0$}]

\addplot[domain=0:3, samples = 100, color=blue, thick]{-1+1.6*x^(1/3)} ;
\addplot[domain=-3:0, samples = 100, color=blue, thick]{-1-1.6*(-x)^(1/3)};
\addplot[blue,mark = *] coordinates {(0,-1)};
\addplot[red, thick] coordinates {(0,-2)(0,0)};
\node[color=blue] at (axis cs:2,1.6) {$y = \sqrt[3]x - 1$};
\end{axis}
\end{tikzpicture}
\hspace{1 in}
\begin{tikzpicture}
\begin{axis}[axis x line =center, axis y line = center, xmin = -3.5, xmax = 3.5, ymin = -3.5, ymax = 3.5, xtick = {-4/3, 4/3}, xticklabels={}, 
ytick={-2, 2},yticklabels={}, title = {Type 2 critical number at $x = 0$}]

\addplot[domain=-3:0, samples = 100, color=blue, thick]{-3/2*x-2};
\addplot[domain=0:3, samples = 100, color=blue, thick]{3/2*x-2} ;

\addplot[blue,mark = *] coordinates {(0,-2)};
\node[color=blue] at (axis cs:1.5,-1.5) {$y = |x|-2$};
\end{axis}
\end{tikzpicture}

\end{image}

\section{Finding Critical Numbers}



\begin{example}[example 1] Find the critical numbers of the function 
\[f(x) = \dfrac{x^3}{3} - \dfrac{x^2}{2} - 6x +1.\]
Solution: We need to compute $f'(x)$.  We have
\[f'(x) = x^2 - x - 6.\]
Noting that $f'(x)$ is defined for all values of $x$, there are no type 2 critical numbers.
To find the type 1 critical numbers, we solve the equation
\[f'(x) = 0.\]
Geometrically, these are the points where the graph of $f(x)$ has horizontal tangent lines.
We get
\[ x^2 - x - 6 =0\]
\[ (x+2)(x-3) =0\]
\[x = -2 \mbox{  or  } x = 3.\]

Hence $f(x) = \frac{x^3}{3} - \frac{x^2}{2} - 6x - 3$ has two critical numbers: $x=-2$ and $x=3$. They are both of type 1,
correspoding to horizontal tangent lines.

\begin{image}
\begin{tikzpicture}
\begin{axis}[axis x line =center, axis y line = center, xmin = -3.5, xmax = 4.5, ymin = -14, ymax = 11, xtick = {-2, 3}, xticklabels={-2,3}, 
ytick={},yticklabels={}, title = {Type 1 critical numbers at $x = -2$ and $x = 3$}]

\addplot[domain=-3:4, samples = 100, color=blue, thick]{2*x^3 - 3*x^2 -36*x + 6)/6} ;
\addplot[blue, mark = *] coordinates{(-2,-8/3 -2 +12 +1)};
\addplot[blue, mark = *] coordinates{(3,9/2 - 17)};
\addplot[red, thick, domain=-3:-1]{-8/3 + 11};
\addplot[red, thick, domain=2:4]{9/2-17};

\addplot[thick, white] coordinates{(0,-8) (0,-14)};
\node[color=blue] at (axis cs:-0.4,-10) {$y = \frac{x^3}{3} - \frac{x^2}{2} - 6x +1$};
\end{axis}
\end{tikzpicture}
\end{image}

\end{example}



\begin{problem}(problem 1a)
  Find the critical numbers of the function $f(x) = x^2 - 8x + 11$.
		\begin{hint}
      Solve $f'(x) = 0$ for $x$
    \end{hint}
    \begin{hint}
      Are there any points where $f'(x)$ is undefined?
      If so, is $f(x)$ defined at these points?  
		\end{hint}
    
    
		The critical numbers of $f(x) = x^2 - 8x + 11$ are
		\begin{multipleChoice}
		\choice{$x = -4$}
		\choice[correct]{$x = 4$}
		\choice{no critical numbers}
		\end{multipleChoice}
		
\end{problem}


\begin{problem}(problem 1b)
  Find the critical numbers of the function $f(x) = 2x^3 - 3x^2 -36x + 7$
    
		\begin{hint}
      Solve $f'(x) = 0$ for $x$
    \end{hint}
    \begin{hint}
      Are there any points where $f'(x)$ is undefined?
      If so, is $f(x)$ defined at these points?  
		\end{hint}
    
    
		The critical numbers of $f(x) = 2x^3 - 3x^2 -36x + 7$ are
		\begin{multipleChoice}
		\choice{$x = -3$ and $x = 2$}
		\choice[correct]{$x = -2$ and $x = 3$}
		\choice{no critical numbers}
		\end{multipleChoice}
		\end{problem}

\begin{example}[example 2]
Find the critical numbers of the function $f(x) = x^4 - 4x^3 + 2$\\
Solution: We need to compute $f'(x)$.  We have
\[f'(x) = 4x^3 - 12x^3.\]
Noting that $f'(x)$ is defined for all values of $x$, there are no type 2 critical numbers.
To find the type 1 critical numbers, we solve the equation
\[f'(x) = 0.\]
Geometrically, these are the points where the graph of $f(x)$ has horizontal tangent lines.
We get
\[ 4x^3 - 12x^3 =0\]
\[ 4x^2(x-3) =0\]
\[x = 0 \mbox{  or  } x = 3.\]

Hence $f(x) = x^4 - 4x^3 + 2$ has two critical numbers, $0$ and $3$, and they are both type 1.
\begin{image}
\begin{tikzpicture}
\begin{axis}[axis x line=  center, axis y line = center, xmin=-2.5, xmax = 5, ymin = -30, ymax = 35,
xtick={3},title={Type 1 critical numbers at $x = 0$ and $x = 3$}];

\addplot[domain=-2:4.5, samples = 100, color=blue, thick]{x^4 - 4*x^3 + 2} ;
\addplot[blue, mark = *] coordinates{(0,2)};
\addplot[blue, mark = *] coordinates{(3,-25)};
\addplot[red, thick, domain=-1:1]{2};
\addplot[red, thick, domain=2:4]{-25};
\node[color=blue] at (axis cs:2,8) {$y = x^4 - 4x^3 + 2$};
\end{axis}
\end{tikzpicture}
\end{image}

\end{example}


\begin{problem}(problem 2)
  Find the critical numbers of the function $f(x) = x^4 - x^3 - 5$
    
		The critical numbers of $f(x) = x^4 - x^3 - 5$ are
		\begin{multipleChoice}
		\choice{$x = 3/4$}
		\choice[correct]{$x = 0$ and $x = 3/4$}
		\choice{no critical numbers}
		\end{multipleChoice} 
		
\end{problem}




\begin{example}[example 3] Find the critical numbers of the function 
\[f(x) = x^2e^{3x}.\]
We need to compute $f'(x)$ using the product and chain rules.  We have
\[f'(x) = (3x^2 +2x)e^{3x} \mbox{   (verify)}.\]
Noting that $f'(x)$ is defined for all values of $x$, there are no type 2 critical numbers.
To find the type 1 critical numbers, we solve the equation
\[f'(x) = 0.\]
Geometrically, these are the points where the graph of $f(x)$ has horizontal tangent lines.
We get
\[ (3x^2 +2x)e^{3x} =0\]
\[ x(3x+2)e^{3x} =0\]
\[x = 0 \mbox{   or   }  x = -2/3.\]
Note that the equation $e^{3x} = 0$ has no solutions since an exponential function is always positive.

Hence $f(x) = x^2e^{3x}$ has two critical numbers, $-2/3$ and $0$, and they are both type 1. 

\begin{image}
\begin{tikzpicture}
\begin{axis}[axis x line=  center, axis y line = center,xmin = -2.5, xmax = .75, ymin = -0.2, ymax =0.4,
xtick={-2/3},xticklabels={$-\frac23$}, ytick={1/3},yticklabels={$\frac13$},
title={Type 1 critical numbers at $x = -2/3$ and $x = 0$}];

\addplot[domain=-2:0.5, samples = 100, color=blue, thick]{x^2*e^(3*x)} ;
\addplot[blue, mark = *] coordinates{(0,0)};
\addplot[blue, mark = *] coordinates{(-2/3,4/66.5)};
\addplot[red, thick, domain=-1:-1/3]{4/66.5)};
\addplot[red, thick, domain=-1/3:1/3]{0};
\node[color=blue] at (axis cs:-1.5,0.1) {$y = x^2e^{3x}$};
\end{axis}
\end{tikzpicture}
\end{image}

\end{example}

\begin{problem}(problem 3a)
  Find the critical numbers of the function $f(x) = xe^{2x}$
  
  
    \begin{hint}
      Compute $f'(x)$ using the Product Rule
    \end{hint}    
    
		The critical numbers of $f(x) = xe^{2x}$ are
		\begin{multipleChoice}
		\choice[correct]{$x = -1/2$}
		\choice{$x = 0$ and $x = -1/2$}
		\choice{no critical numbers}
		\end{multipleChoice} 
		
\end{problem}


\begin{problem}(problem 3b)
  Find the critical numbers of the function $f(x) = 5xe^{-x/5}$
  
  
    \begin{hint}
      Compute $f'(x)$ using the Product Rule
    \end{hint}
 
		The critical numbers of $f(x) = 5xe^{-x/5}$ are
		 \begin{multipleChoice}
		\choice[correct]{$x = 5$}
		\choice{$x = 0$ and $x = 5$}
		\choice{no critical numbers}
		\end{multipleChoice} 
		
\end{problem}




\begin{example}[example 4] Find the critical numbers of the function 
\[f(x) = \dfrac{x}{x^2 +1}.\]
We need to compute $f'(x)$ using the quotient rule.  We have
\[f'(x) = \frac{1-x^2}{(x^2+1)^2} \mbox{   (verify)}.\]
Noting that $f'(x)$ is defined for all values of $x$ (since the denominator is never equal to 0), 
there are no type 2 critical numbers.
To find the type 1 critical numbers, we solve the equation
\[f'(x) = 0.\]
Geometrically, these are the points where the graph of $f(x)$ has horizontal tangent lines.
We get
\[ \frac{1-x^2}{(x^2+1)^2} =0\]
\[ 1-x^2 =0\]
\[x = \pm 1.\]

Hence $f(x) = \dfrac{x}{x^2 +1}$ has two critical numbers, $-1$ and $1$, and they are both type 1. 

\begin{image}
\begin{tikzpicture}
\begin{axis}[axis x line=  center, axis y line = center,xmin=-3.5, xmax=3.5, ymin =-0.7, ymax = 0.7,
ytick={-1/2, 1/2}, yticklabels={},xtick={-1,1},
title={Type 1 critical numbers at $x = -1$ and $x = 1$}];

\addplot[domain=-4:4, samples = 100, color=blue, thick]{x/(x^2 +1)} ;
\addplot[blue, mark = *] coordinates{(-1,-1/2)};
\addplot[blue, mark = *] coordinates{(1,1/2)};
\addplot[red, thick, domain=-1.6:-0.4]{-1/2};
\addplot[red, thick, domain=0.4:1.6]{1/2};
\node[color=blue] at (axis cs:1.5,-0.3) {$y = \frac{x}{x^2+1}$};
\end{axis}
\end{tikzpicture}
\end{image}

\end{example}

\begin{problem}(problem 4a)
  Find the critical numbers of the function $f(x) = \dfrac{1}{1 + x^2}$
  
  
    \begin{hint}
      Compute $f'(x)$ using the Quotient Rule
    \end{hint}    
    
		The critical numbers of $f(x) = \dfrac{1}{1 + x^2}$ are
		 \begin{multipleChoice}
		\choice[correct]{$x = 0$}
		\choice{$x = -1$ and $x = 0$}
		\choice{no critical numbers}
		\end{multipleChoice} 
\end{problem}

\begin{problem}(problem 4b)
  Find the critical numbers of the function $f(x) = \dfrac{3x}{x^2 +4}$
  
  
    \begin{hint}
      Compute $f'(x)$ using the Quotient Rule
    \end{hint}
    
    
		The critical numbers of $f(x) = \dfrac{3x}{x^2 + 4}$ are
		 \begin{multipleChoice}
		\choice{$x =2$}
		\choice[correct]{$x = -2$ and $x = 2$}
		\choice{no critical numbers}
		\end{multipleChoice} 
\end{problem}




\begin{example}[example 5] Find the critical numbers of the function 
\[f(x) = \sqrt[3] x.\]
We need to compute $f'(x)$.  We have
\[f'(x) = \frac{1}{3\sqrt[3]{x^2}} \text{   (verify)}.\]
In this case, $f'(0)$ is undefined (division by zero). Hence $x=0$ is a critical number \it{if}
$f(0)$ is defined.  We can easily check this: $f(0) = \sqrt[3] 0 = 0$, so it is defined and now we can conclude 
that $x=0$ is a type 2 critical number.
To find the type 1 critical numbers, we solve the equation
\[f'(x) = 0.\]
Geometrically, these are the points where the graph of $f(x)$ has horizontal tangent lines.
We get
\[ \frac{1}{3\sqrt[3]{x^2}} =0\]
\[ 1 =0.\]
So there are no solutions.  The function has no type 1 critical numbers.


Hence $f(x) = \sqrt[3] x$ has only one critical number, 0, and it type 2, 
where the function is not differentiable. 
Geometrically, the function $f(x) = \sqrt[3] x$ has a vertical tangent line at the critical number $x = 0$.


\begin{image}
\begin{tikzpicture}
\begin{axis}[axis x line=  center, axis y line = center,
title={Type 2 critical number at $x = 0$}]

\addplot[domain=0:5, samples = 100, color=blue, thick]{x^(1/3)} ;
\addplot[domain=-5:0, samples = 100, color=blue, thick]{-(-x)^(1/3)} ;
\addplot[blue,mark = *] coordinates {(0,0)};
\addplot[red, thick] coordinates {(0,2/3)(0,-2/3)} node[below,right,red]{vertical tangent line};

\node[color=blue] at (axis cs:1.5,0.7) {$y = \sqrt[3] x$};

\end{axis}
\end{tikzpicture}
\end{image}

\end{example}

\begin{problem}(problem 5a)
  Find the critical numbers of the function $f(x) = \sqrt[5] x$
  
    \begin{hint}
      The derivative can be written as $f'(x) = \dfrac{1}{5x^{4/5}}$
    \end{hint}
		
    
    
		The critical numbers of $f(x) = \sqrt[5] x$ are
		 \begin{multipleChoice}
		\choice[correct]{$x = 0$}
		\choice{$x = 1/5$}
		\choice{no critical numbers}
		\end{multipleChoice} 
\end{problem}



\begin{problem}(problem 5b)
  Find the critical numbers of the function $f(x) = x\sqrt[3]x$
  
    \begin{hint}
      Rewrite $f(x)$ as a power function by adding exponents
    \end{hint}
    
    
		The critical numbers of $f(x) = x\sqrt[3]x$ are
		\begin{multipleChoice}
		\choice{$x = -3/4$}
		\choice[correct]{$x = 0$}
		\choice{no critical numbers}
		\end{multipleChoice} 
\end{problem}



\begin{example}[example 6]  Find the critical numbers of the function 
\[f(x) = \dfrac{1}{x}.\]
We need to compute $f'(x)$.  We have
\[f'(x) = -\frac{1}{ x^2} \mbox{   (verify)}.\]
In this case, $f'(x)$ is undefined at $x = 0$ (division by zero). Hence,  {\bf if}
$f(0)$ is defined then $x=0$ would be a type 2 critical number.  However, we can easily see that $f(0) = \dfrac{1}{0}$, is undefined,  so 
that $x=0$ is a not in the domain of $f(x)$ and hence it is {\bf not} a critical number.
To find the type 1 critical numbers, we solve the equation
\[f'(x) = 0.\]
Geometrically, these are the points where the graph of $f(x)$ has horizontal tangent lines.
We get
\[ -\frac{1}{ x^2}  =0\]
\[ 1 =0.\]
So there are no solutions.  The function has no type 1 critical numbers.


Hence $f(x) = \dfrac{1}{x}$ has no critical numbers. 

\begin{image}
\begin{tikzpicture}
\begin{axis}[axis x line=  center, axis y line = center,
title={No critical number at $x = 0$}]

\addplot[domain=-4:-.25, samples = 100, color=blue, thick]{1/x} ;
\addplot[domain=.25:4, samples = 100, color=blue, thick]{1/x} ;
\node[color=blue] at (axis cs:1.5,1.5) {$y = \frac{1}{x}$};
\end{axis}
\end{tikzpicture}
\end{image}


\end{example}


\begin{problem}(problem 6a)
  Find the critical numbers of the function $f(x) = \dfrac{3}{x^2}$
  
  
    
    
    
		The critical numbers of $f(x) = 3/x^2$ are
		\begin{multipleChoice}
		\choice{$x = 0$}
		\choice{$x = 1/6$}
		\choice[correct]{no critical numbers}
		\end{multipleChoice} 
\end{problem}



\begin{problem}(problem 6b)
  Find the critical numbers of the function $f(x) = \tan(x)$ in the interval $[0, 2\pi)$
  
		\begin{hint}
      $\sec(x) = \frac{1}{\cos(x)}$
    \end{hint}
      
		The critical numbers of $f(x) = \tan(x)$ are
		 \begin{multipleChoice}
		\choice{$x = \pi/2$}
		\choice{$x = \pi/2$ and $x = 3\pi/2$}
		\choice[correct]{no critical numbers}
		\end{multipleChoice} 
\end{problem}



\begin{center}
\begin{foldable}
\unfoldable{Here is a detailed, lecture style video on critical numbers:}
\youtube{Bx1uaX9e-Qw}
\end{foldable}
\end{center}


\end{document}
