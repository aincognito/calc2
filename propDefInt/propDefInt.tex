\documentclass[handout]{ximera}
\usepgfplotslibrary{fillbetween}
%% You can put user macros here
%% However, you cannot make new environments



\newcommand{\ffrac}[2]{\frac{\text{\footnotesize $#1$}}{\text{\footnotesize $#2$}}}
\newcommand{\vasymptote}[2][]{
    \draw [densely dashed,#1] ({rel axis cs:0,0} -| {axis cs:#2,0}) -- ({rel axis cs:0,1} -| {axis cs:#2,0});
}


\graphicspath{{./}{firstExample/}}

\usepackage{amsmath}
\usepackage{amssymb}
\usepackage{array}
\usepackage[makeroom]{cancel} %% for strike outs
\usepackage{pgffor} %% required for integral for loops
\usepackage{tikz}
\usepackage{tikz-cd}
\usepackage{tkz-euclide}
\usetikzlibrary{shapes.multipart}


\usetkzobj{all}
\tikzstyle geometryDiagrams=[ultra thick,color=blue!50!black]


\usetikzlibrary{arrows}
\tikzset{>=stealth,commutative diagrams/.cd,
  arrow style=tikz,diagrams={>=stealth}} %% cool arrow head
\tikzset{shorten <>/.style={ shorten >=#1, shorten <=#1 } } %% allows shorter vectors

\usetikzlibrary{backgrounds} %% for boxes around graphs
\usetikzlibrary{shapes,positioning}  %% Clouds and stars
\usetikzlibrary{matrix} %% for matrix
\usepgfplotslibrary{polar} %% for polar plots
\usepgfplotslibrary{fillbetween} %% to shade area between curves in TikZ



%\usepackage[width=4.375in, height=7.0in, top=1.0in, papersize={5.5in,8.5in}]{geometry}
%\usepackage[pdftex]{graphicx}
%\usepackage{tipa}
%\usepackage{txfonts}
%\usepackage{textcomp}
%\usepackage{amsthm}
%\usepackage{xy}
%\usepackage{fancyhdr}
%\usepackage{xcolor}
%\usepackage{mathtools} %% for pretty underbrace % Breaks Ximera
%\usepackage{multicol}



\newcommand{\RR}{\mathbb R}
\newcommand{\R}{\mathbb R}
\newcommand{\C}{\mathbb C}
\newcommand{\N}{\mathbb N}
\newcommand{\Z}{\mathbb Z}
\newcommand{\dis}{\displaystyle}
%\renewcommand{\d}{\,d\!}
\renewcommand{\d}{\mathop{}\!d}
\newcommand{\dd}[2][]{\frac{\d #1}{\d #2}}
\newcommand{\pp}[2][]{\frac{\partial #1}{\partial #2}}
\renewcommand{\l}{\ell}
\newcommand{\ddx}{\frac{d}{\d x}}

\newcommand{\zeroOverZero}{\ensuremath{\boldsymbol{\tfrac{0}{0}}}}
\newcommand{\inftyOverInfty}{\ensuremath{\boldsymbol{\tfrac{\infty}{\infty}}}}
\newcommand{\zeroOverInfty}{\ensuremath{\boldsymbol{\tfrac{0}{\infty}}}}
\newcommand{\zeroTimesInfty}{\ensuremath{\small\boldsymbol{0\cdot \infty}}}
\newcommand{\inftyMinusInfty}{\ensuremath{\small\boldsymbol{\infty - \infty}}}
\newcommand{\oneToInfty}{\ensuremath{\boldsymbol{1^\infty}}}
\newcommand{\zeroToZero}{\ensuremath{\boldsymbol{0^0}}}
\newcommand{\inftyToZero}{\ensuremath{\boldsymbol{\infty^0}}}


\newcommand{\numOverZero}{\ensuremath{\boldsymbol{\tfrac{\#}{0}}}}
\newcommand{\dfn}{\textbf}
%\newcommand{\unit}{\,\mathrm}
\newcommand{\unit}{\mathop{}\!\mathrm}
%\newcommand{\eval}[1]{\bigg[ #1 \bigg]}
\newcommand{\eval}[1]{ #1 \bigg|}
\newcommand{\seq}[1]{\left( #1 \right)}
\renewcommand{\epsilon}{\varepsilon}
\renewcommand{\iff}{\Leftrightarrow}

\DeclareMathOperator{\arccot}{arccot}
\DeclareMathOperator{\arcsec}{arcsec}
\DeclareMathOperator{\arccsc}{arccsc}
\DeclareMathOperator{\si}{Si}
\DeclareMathOperator{\proj}{proj}
\DeclareMathOperator{\scal}{scal}
\DeclareMathOperator{\cis}{cis}
\DeclareMathOperator{\Arg}{Arg}
%\DeclareMathOperator{\arg}{arg}
\DeclareMathOperator{\Rep}{Re}
\DeclareMathOperator{\Imp}{Im}
\DeclareMathOperator{\sech}{sech}
\DeclareMathOperator{\csch}{csch}
\DeclareMathOperator{\Log}{Log}

\newcommand{\tightoverset}[2]{% for arrow vec
  \mathop{#2}\limits^{\vbox to -.5ex{\kern-0.75ex\hbox{$#1$}\vss}}}
\newcommand{\arrowvec}{\overrightarrow}
\renewcommand{\vec}{\mathbf}
\newcommand{\veci}{{\boldsymbol{\hat{\imath}}}}
\newcommand{\vecj}{{\boldsymbol{\hat{\jmath}}}}
\newcommand{\veck}{{\boldsymbol{\hat{k}}}}
\newcommand{\vecl}{\boldsymbol{\l}}
\newcommand{\utan}{\vec{\hat{t}}}
\newcommand{\unormal}{\vec{\hat{n}}}
\newcommand{\ubinormal}{\vec{\hat{b}}}

\newcommand{\dotp}{\bullet}
\newcommand{\cross}{\boldsymbol\times}
\newcommand{\grad}{\boldsymbol\nabla}
\newcommand{\divergence}{\grad\dotp}
\newcommand{\curl}{\grad\cross}
%% Simple horiz vectors
\renewcommand{\vector}[1]{\left\langle #1\right\rangle}


\outcome{Compute and interpret definite integrals}

\title{4.7 Properties of Definite Integrals}

%\newcommand{\ffrac}[2]{\frac{\mbox{\footnotesize $#1$}}{\mbox{\footnotesize $#2$}}}
%\newcommand{\vasymptote}[2][]{\draw [densely dashed,#1] 
%({rel axis cs:0,0} -| {axis cs:#2,0}) -- ({rel axis cs:0,1} -| {axis cs:#2,0});}


\begin{document}

\begin{abstract}
In this section we use properties of definite integrals to compute and interpret them.
\end{abstract}

\maketitle



\section{Properties of Definite Integrals}

We begin with some basic properties of definite integrals- many of which are familiar from our study of derivatives and their basic properties.

\begin{proposition}
If $f(x)$ and $g(x)$ are continuous on the interval $[a,b]$ then 
\begin{enumerate}
\item $\displaystyle{\int_a^b \Big[f(x) \pm g(x) \Big] \ dx = \int_a^b f(x) \ dx \pm \int_a^b g(x) \ dx}$
\item $\displaystyle{\int_a^b cf(x) \ dx = c\int_a^b f(x) \ dx}$
\item $\displaystyle{\int_a^a f(x) \ dx = 0}$
\item $\displaystyle{\int_b^a f(x) \ dx = -\int_a^b f(x) \ dx}$\\
\end{enumerate}
and if $f(x) \leq g(x)$ on $[a,b]$, then
\begin{enumerate}[resume]
\item $\displaystyle{\int_a^b f(x) \ dx \leq \int_a^b g(x) \ dx.}$
\end{enumerate}
\end{proposition}

Recall that if $f(x) \geq 0$ on an interval $[a, b]$ then the definite integral, $\int_a^b f(x) \ dx$,
gives the area under the curve and if $f(x) \leq 0$ on an interval $[a, b]$ then 
the definite integral, $\int_a^b f(x) \ dx$, gives -1 times the area above the curve. 
We now consider the situation where the integrand $f(x)$ changes sign on the interval $[a,b]$.
The key to handling this situation is to use the following property.
\begin{proposition}[Additivity]
If $f(x)$ is continuous on the interval $[a,b]$ and $c$ is a number between $a$ and $b$, i.e., $a<c<b$ then
\[\int_a^b f(x) \ dx = \int_a^c f(x) \ dx  + \int_c^b f(x) \ dx.\]
\end{proposition}

We apply this proposition in the following examples.

\begin{example}[example 1]
Use the figure below to determine the value of the definite integral $\displaystyle{\int_a^b f(x) \ dx}$.
\\
%title={The Area Above $f(x)=-\sin(x)$ on $[0,\pi]$}]
\begin{image}
\begin{tikzpicture}
\begin{axis}[axis x line=  center, axis y line = none, xtick={0, 0.473, 1.549, 2.811}, 
xticklabels={0, $a$, $c$, $b$}] 
\addplot[ domain=.3:0.473, 
    samples=100, color=black]{(x-1.5)^3 -(x+.5)^2 /3 + 1.4};
\addplot[name path = B, domain=0.473:1.549, 
    samples=100, color=black]{(x-1.5)^3 -(x+.5)^2 /3 + 1.4};
\addplot[name path = D, domain=1.549:2.811, 
    samples=100, color=black]{(x-1.5)^3 -(x+.5)^2 /3 + 1.4};
\addplot[domain=2.811:3, 
    samples=100, color=black]{(x-1.5)^3 -(x+.5)^2 /3 + 1.4};
		
\addplot[name path = A, domain=0.473:1.549, samples=100, color=black]{0};
\addplot[name path = C, domain=1.549:2.811, samples=100, color=black]{0};
\addplot[blue!10] fill between[of=A and B];
\addplot[red!10] fill between[of=C and D];
%\addplot[<->] coordinates {(0.3,-1) (0.3, 1)};
%\addplot [domain= .1:.2, samples=10, color=black]{-x + .2};
%\addplot [domain= .2:.24, samples=10, color=black]{.04-(x- .2)};
%\addplot [domain= .3:.39, samples=10, color=black]{.09-(x- .3)};
%\addplot [domain= .4:.56, samples=10, color=black]{.16-(x- .4)};
%\addplot [domain= .5:.75, samples=10, color=black]{.25-(x- .5)};
%\addplot [domain= .6:.96, samples=10, color=black]{.36-(x- .6)};
\node at (axis cs: .95,.15){Area = 4};
\node at (axis cs: 2.2,-.25){Area = 6};
\node at (axis cs: 1.5,.5){$y=f(x)$};
%\addplot [domain= .7:1, samples=10, color=black]{.49-(x- .7)};
%\addplot [domain= .8:1, samples=10, color=black]{.64-(x- .8)};
%\addplot [domain= .9:1, samples=10, color=black]{.81-(x- .9)};

%\addplot [domain= .2:.4, samples=10, color=black]{-x + .05};
%
%\addplot[thin, samples = 100, color=black] coordinates {(0,0) (0,1)};

\end{axis}
%\legend{$y = f(x)$, , secant line, tangent line, };
\end{tikzpicture}
\end{image}

By additivity, we can write
\[\int_a^b f(x) \ dx = \int_a^c f(x) \ dx  + \int_c^b f(x) \ dx.\]
Next, we can determine the definite integrals on the right hand side of the above equation
by relating them to area.
On the interval $[a,c], f(x) \geq 0$ and so 
\[\int_a^c f(x) \ dx = \text{area under the curve} = 4.\]
On the interval $[c,b], f(x) \leq 0$ and so 
\[\int_c^b f(x) \ dx = (-1)\cdot \text{area above the curve} = -6.\]
Now, by the additivity of the definite integral,
\[\int_a^b f(x) \ dx = 4 + (-6) = -2.\]


\end{example}


\begin{problem}(problem 1a)
Use the figure below to determine the definite integrals.
\begin{image}
\begin{tikzpicture}
\begin{axis}[axis x line=  center, axis y line = none, xtick={0, 3.14, 6.28}, xticklabels={$a$, $c$, $b$}] 
\addplot[name path = B, domain=0:3.14, 
   samples=100, color=black]{2*sin(deg(x))};
\addplot[name path = D, domain=3.14:6.28, 
    samples=100, color=black]{1.5*sin(deg(x))};		
\addplot[name path = A, domain=0:3.14, samples=100, color=black]{0};
\addplot[name path = C, domain=3.14:6.28, samples=100, color=black]{0};
\addplot[blue!10] fill between[of=A and B];
\addplot[red!10] fill between[of=C and D];
\node at (axis cs: 1.57,.6){Area = 5};
\node at (axis cs: 4.71,-.5){Area = 3};
\node at (axis cs: 3,1.6){$y=f(x)$};


\end{axis}
\end{tikzpicture}
\end{image}


\[\int_a^c f(x) = \answer{5}.\]
\[\int_c^b f(x) = \answer{-3}.\]
\[\int_a^b f(x) = \answer{2}.\]
\end{problem}


\begin{problem}(problem 1b)
Use the figure below to determine the definite integrals.
\begin{image}
\begin{tikzpicture}
\begin{axis}[axis x line=  center, axis y line = none, xtick={0, 3.14, 6.28, 9.42}, 
xticklabels={$a$, $c$, $d$, $b$}] 
\addplot[name path = B, domain=0:3.14, 
   samples=100, color=black]{2.5*sin(deg(x))};
\addplot[name path = D, domain=3.14:6.28, 
    samples=100, color=black]{2*sin(deg(x))};
\addplot[name path = F, domain=6.28:9.42, 
    samples=100, color=black]{1.5*sin(deg(x))};		
\addplot[name path = A, domain=0:3.14, samples=100, color=black]{0};
\addplot[name path = C, domain=3.14:6.28, samples=100, color=black]{0};
\addplot[name path = E, domain=6.28:9.42, samples=100, color=black]{0};
\addplot[blue!10] fill between[of=A and B];
\addplot[red!10] fill between[of=C and D];
\addplot[blue!10] fill between[of=E and F];
\node at (axis cs: 1.57,.6){Area = 6};
\node at (axis cs: 4.71,-.6){Area = 5};
\node at (axis cs: 7.85,.4){Area = 4};
\node at (axis cs: 3.4,1.9){$y=f(x)$};


\end{axis}
\end{tikzpicture}
\end{image}

\[\int_a^c f(x) = \answer{6}.\]
\[\int_c^d f(x) = \answer{-5}.\]
\[\int_d^b f(x) = \answer{4}.\]
\[\int_a^d f(x) = \answer{1}.\]
\[\int_c^b f(x) = \answer{-1}.\]
\[\int_a^b f(x) = \answer{5}.\]
\end{problem}


\begin{example}[example 2]
Use geometry and additivity of the definite integral to compute  
$\displaystyle{\int_0^5  (3-x)  \ dx}$.
The graph of $y = 3-x$ is a line with slope $-1$ passing through the points $(0,3)$ and $(5,-2)$. To compute the integral,
we need to determine where $3-x$ is positive and where it is negative. We accomplish this by first finding the $x$-intercept(s)
by solving $3-x = 0$.  The solution is clearly, $x = 3$.  Now, if $x < 3$, then $3 - x >0$ and if
$x >3$, then $3-x <0$.
We now use the additivity property:
\[\int_0^5 (3-x) \ dx = \int_0^3 (3-x) \ dx  + \int_3^5 (3-x) \ dx.\]
We can now take an area approach to evaluate the two integrals on the right-hand side.
\[\int_0^3 (3-x) \ dx = \text{area of triangle under the curve} = \tfrac12bh = \tfrac12(3)(3) = \tfrac92,\]
and
\[\int_3^5 (3-x) \ dx = (-1)\cdot\text{area of triangle above the curve} = -\tfrac12(2)(2) = -2\]
Thus
\[\int_0^5 (3-x) \ dx = \tfrac92  + (-2) = \tfrac52.\]


\begin{image}
\begin{tikzpicture}
\begin{axis}[axis x line=  center, axis y line = left, xtick={0, 1, 2, 3, 4, 5}] 
\addplot[name path = B, domain=0:3, 
    samples=100, color=black]{3-x};
\addplot[name path = D, domain=3:5, 
    samples=100, color=black]{3-x};
		
\addplot[name path = A, domain=0:3, samples=100, color=black]{0};
\addplot[name path = C, domain=3:5, samples=100, color=black]{0};
\addplot[blue!10] fill between[of=A and B];
\addplot[red!10] fill between[of=C and D];

\node at (axis cs: 1,.7){Area = 9/2};
\node at (axis cs: 4.4,-.7){Area = 2};
\node at (axis cs: 1.75,2){$y=3-x$};


\end{axis}
%\legend{$y = f(x)$, , secant line, tangent line, };
\end{tikzpicture}
\end{image}

\end{example}


\begin{problem}(problem 2)
Use geometry and additivity of the definite integral to compute the definite integral.  
\[\int_0^3  (x-2)  \ dx= \answer{-3/2}.\]
\begin{hint}
$y=x-2$ is a line with slope $1$ passing thru $(0,-2), (2,0)$ and $(3,1)$
\end{hint}
\begin{hint}
If $f(x) \geq 0$ then the integral gives the area under the curve
\end{hint}
\begin{hint}
If $f(x) \leq 0$ then the integral gives $(-1)$ times the area above the curve
\end{hint}

\end{problem}

Since the definite integral is related to area, there are situations where 
computing these integrals can be facilitated by observing a pattern of symmetry 
in a region. To aid us in our discussion, we will have to recall the idea of 
even and odd functions.

\begin{definition}[Even and Odd Functions]
$f(x)$ is called \textbf{even} if $f(-x) = f(x)$ and\\
$f(x)$ is called \textbf{odd} if $f(-x) = -f(x)$.
\end{definition}
The terms even and odd come from the observation that
functions of the form $f(x) = x^n$ are even functions if $n$ is even and they are
odd functions if $n$ is odd. Furthermore, $\cos(x)$ is an even function, while $\sin(x)$ is an odd function.
The graph of an 
even function is symmetric about the $y$-axis and the graph of an odd function is symmetric about the origin.
The following proposition describes how these symmetries affect the definite integral.

\begin{proposition}
Let $a>0$ and suppose $f(x)$ is continuous on the interval $[-a, a]$.
If $f(x)$ is an even function then
\[\int_{-a}^a f(x) \ dx = 2 \int_0^a f(x) \ dx,\]
and if $f(x)$ is an odd function, then 
\[\int_{-a}^a f(x) \ dx = 0.\]
\end{proposition}  

%xtick={0, 0.473, 1.549, 2.811}, 
%xticklabels={0, $a$, $c$, $b$}
%

\begin{image}
\begin{tikzpicture}
\begin{axis}[title={The Graph of an Even Function},axis x line=  center, axis y line = center,
xticklabels=none, yticklabels=none]; 
\addplot[name path=B, domain=-1:0, 
    samples=100, color=black]{.5 + x^2-x^4};
\addplot[name path = D, domain=0:1, 
    samples=100, color=black]{.5+x^2-x^4};

		
\addplot[name path = A, domain=-1:0, samples=100, color=black]{0};
\addplot[name path = C, domain=0:1, samples=100, color=black]{0};
\addplot[blue!20] fill between[of=A and B];
\addplot[blue!10] fill between[of=C and D];
\addplot[white] coordinates {(0,-.21) (0, -.39)};

\node at (axis cs: 0,-.3){The two blue regions have the same area};
%\node at (axis cs: 2.2,-.25){Area = 6};
%\node at (axis cs: 1.5,.5){$y=f(x)$};

%\addplot[thin, samples = 100, color=black] coordinates {(0,0) (0,1)};

\end{axis}
%\legend{$y = f(x)$, , secant line, tangent line, };
\end{tikzpicture}
\end{image}



\begin{image}
\begin{tikzpicture}
\begin{axis}[title={The Graph of an Odd Function},axis x line=  center, axis y line = center,
xticklabels=none, yticklabels=none]; 
\addplot[name path=B, domain=-1:0, 
    samples=100, color=black]{x-x^3};
\addplot[name path = D, domain=0:1, 
    samples=100, color=black]{x-x^3};

		
\addplot[name path = A, domain=-1:0, samples=100, color=black]{0};
\addplot[name path = C, domain=0:1, samples=100, color=black]{0};
\addplot[red!10] fill between[of=A and B];
\addplot[blue!10] fill between[of=C and D];
\addplot[white] coordinates {(0,-.41) (0, -.59)};

\node at (axis cs: 0,-.5){The red and blue regions have the same area};
%\node at (axis cs: 2.2,-.25){Area = 6};
%\node at (axis cs: 1.5,.5){$y=f(x)$};

%\addplot[thin, samples = 100, color=black] coordinates {(0,0) (0,1)};

\end{axis}
%\legend{$y = f(x)$, , secant line, tangent line, };
\end{tikzpicture}
\end{image}


This result is especially important for odd functions, since the calculation of definite integrals in this case
becomes trivial. For even functions the advantage of this result less significant.  It allows us to work with the number
$0$ instead of the negative number $-a$, thereby simplifying any arithmetic associated with the 
Fundamental Theorem of Calculus.

\begin{example}[example 3]
Since $x^2$ is an even function, 
\[\int_{-3}^3 x^2 \ dx = 2\int_0^3 x^2 \ dx = \tfrac{2}{3}x^3\Bigg|_0^3 = 18.\]
\end{example}

\begin{example}[example 4]
Since $\cos(x)$ is an even function, 
\[\int_{-\pi/2}^{\pi/2} \cos(x) \ dx = 2\int_0^{\pi/2} \cos(x) \ dx = 2\sin(x)\Bigg|_0^{\pi/2} = 2.\]
\end{example}

\begin{problem}(problem 4)
Use the fact that the integrand is even to compute the definite integral.
\[\int_{-1}^1 \left(x^4 + 5x^2 + 3 \right) \ dx =\answer {146/15}.\]
\end{problem}



Note that in the last two examples, plugging in the value zero gave zero.  This is not a coincidence.
An even function always has an odd anti-derivative and odd functions must pass thru the origin.\\
Now we look at some examples of odd functions.

\begin{example}[example 5]
The following definite integrals are all equal to zero, since in each case, the integrand is an odd function and the 
interval of integration has the form $[-a, a]$:

\[\int_{-2}^2 \left(2x^5 - 4x^3 + 6x \right)\ dx, \int_{-1}^1 \left(\sqrt[3] x + \sqrt[5]x\right) \ dx  \; \text{and} \; \int_{-\pi}^\pi \Big[\sin(x) + \sin^3(x)\Big] \ dx.\]
\end{example}

\begin{problem}(problem 5a)
\[\int_{-5}^5 \left(x^5 - 8x^3 +2x \right) \ dx =\answer {0}.\]
\begin{hint}
The integrand is odd
\end{hint}
\end{problem}

\begin{problem}(problem 5b)
\[\int_{-1}^1 \Big[x\cos(x) - x^2\sin(x) \Big] \ dx =\answer {0}.\]
\begin{hint}
An odd function times an even function is an odd function
\end{hint}
\end{problem}

\begin{problem}(problem 5c)
\[\int_{-\pi}^\pi \Big[\sin^3(x) + \sin(x^3)\Big] \ dx =\answer {0}.\]
\begin{hint}
The composition of two odd functions is odd
\end{hint}
\end{problem}


In the next example, we combine an even function and an odd function.

\begin{example}[example 6]
The function $x^4$ is even and the function $x^5$ is odd.  Therefore,
\[\int_{-2}^2 (x^4 + x^5) \ dx = \int_{-2}^2 x^4 \ dx + \int_{-2}^2 x^5 \ dx = 2\int_0^2 x^4 \ dx + 0 = \tfrac{2}{5}x^5\Bigg|_0^2 = \frac{64}{5}.\]
\end{example}
%Furthermore,the sum, product and composition of even functions is even. 

\begin{problem}(problem 6a)
\[\int_{-3}^3 \Big[x^2 - \sqrt[3]x +2\sin(x) \Big] \ dx =\answer {18}.\]
\begin{hint}
$x^2$ is even and $\sqrt[3]x$ and $2\sin(x)$ are odd
\end{hint}
\end{problem}

\begin{problem}(problem 6b)
\[\int_{-1}^{1} x^4\Big[1 + \sin(x)\Big] \ dx =\answer {2/5}.\]
\begin{hint}
$x^4$ is even and $x^4\sin(x)$ is odd
\end{hint}
\end{problem}


\end{document}


What if f changes sign?
int ab = int ac + int cb
\int_a^a = 0
\int_b^a = -\int_a^b
f<g implies \int f < \int g
if f is even on [-a, a] then \int f = 2 int f_0^a
if f is odd...
