\documentclass{ximera}

\usetikzlibrary{calc,patterns,angles,quotes}


%% You can put user macros here
%% However, you cannot make new environments



\newcommand{\ffrac}[2]{\frac{\text{\footnotesize $#1$}}{\text{\footnotesize $#2$}}}
\newcommand{\vasymptote}[2][]{
    \draw [densely dashed,#1] ({rel axis cs:0,0} -| {axis cs:#2,0}) -- ({rel axis cs:0,1} -| {axis cs:#2,0});
}


\graphicspath{{./}{firstExample/}}

\usepackage{amsmath}
\usepackage{amssymb}
\usepackage{array}
\usepackage[makeroom]{cancel} %% for strike outs
\usepackage{pgffor} %% required for integral for loops
\usepackage{tikz}
\usepackage{tikz-cd}
\usepackage{tkz-euclide}
\usetikzlibrary{shapes.multipart}


\usetkzobj{all}
\tikzstyle geometryDiagrams=[ultra thick,color=blue!50!black]


\usetikzlibrary{arrows}
\tikzset{>=stealth,commutative diagrams/.cd,
  arrow style=tikz,diagrams={>=stealth}} %% cool arrow head
\tikzset{shorten <>/.style={ shorten >=#1, shorten <=#1 } } %% allows shorter vectors

\usetikzlibrary{backgrounds} %% for boxes around graphs
\usetikzlibrary{shapes,positioning}  %% Clouds and stars
\usetikzlibrary{matrix} %% for matrix
\usepgfplotslibrary{polar} %% for polar plots
\usepgfplotslibrary{fillbetween} %% to shade area between curves in TikZ



%\usepackage[width=4.375in, height=7.0in, top=1.0in, papersize={5.5in,8.5in}]{geometry}
%\usepackage[pdftex]{graphicx}
%\usepackage{tipa}
%\usepackage{txfonts}
%\usepackage{textcomp}
%\usepackage{amsthm}
%\usepackage{xy}
%\usepackage{fancyhdr}
%\usepackage{xcolor}
%\usepackage{mathtools} %% for pretty underbrace % Breaks Ximera
%\usepackage{multicol}



\newcommand{\RR}{\mathbb R}
\newcommand{\R}{\mathbb R}
\newcommand{\C}{\mathbb C}
\newcommand{\N}{\mathbb N}
\newcommand{\Z}{\mathbb Z}
\newcommand{\dis}{\displaystyle}
%\renewcommand{\d}{\,d\!}
\renewcommand{\d}{\mathop{}\!d}
\newcommand{\dd}[2][]{\frac{\d #1}{\d #2}}
\newcommand{\pp}[2][]{\frac{\partial #1}{\partial #2}}
\renewcommand{\l}{\ell}
\newcommand{\ddx}{\frac{d}{\d x}}

\newcommand{\zeroOverZero}{\ensuremath{\boldsymbol{\tfrac{0}{0}}}}
\newcommand{\inftyOverInfty}{\ensuremath{\boldsymbol{\tfrac{\infty}{\infty}}}}
\newcommand{\zeroOverInfty}{\ensuremath{\boldsymbol{\tfrac{0}{\infty}}}}
\newcommand{\zeroTimesInfty}{\ensuremath{\small\boldsymbol{0\cdot \infty}}}
\newcommand{\inftyMinusInfty}{\ensuremath{\small\boldsymbol{\infty - \infty}}}
\newcommand{\oneToInfty}{\ensuremath{\boldsymbol{1^\infty}}}
\newcommand{\zeroToZero}{\ensuremath{\boldsymbol{0^0}}}
\newcommand{\inftyToZero}{\ensuremath{\boldsymbol{\infty^0}}}


\newcommand{\numOverZero}{\ensuremath{\boldsymbol{\tfrac{\#}{0}}}}
\newcommand{\dfn}{\textbf}
%\newcommand{\unit}{\,\mathrm}
\newcommand{\unit}{\mathop{}\!\mathrm}
%\newcommand{\eval}[1]{\bigg[ #1 \bigg]}
\newcommand{\eval}[1]{ #1 \bigg|}
\newcommand{\seq}[1]{\left( #1 \right)}
\renewcommand{\epsilon}{\varepsilon}
\renewcommand{\iff}{\Leftrightarrow}

\DeclareMathOperator{\arccot}{arccot}
\DeclareMathOperator{\arcsec}{arcsec}
\DeclareMathOperator{\arccsc}{arccsc}
\DeclareMathOperator{\si}{Si}
\DeclareMathOperator{\proj}{proj}
\DeclareMathOperator{\scal}{scal}
\DeclareMathOperator{\cis}{cis}
\DeclareMathOperator{\Arg}{Arg}
%\DeclareMathOperator{\arg}{arg}
\DeclareMathOperator{\Rep}{Re}
\DeclareMathOperator{\Imp}{Im}
\DeclareMathOperator{\sech}{sech}
\DeclareMathOperator{\csch}{csch}
\DeclareMathOperator{\Log}{Log}

\newcommand{\tightoverset}[2]{% for arrow vec
  \mathop{#2}\limits^{\vbox to -.5ex{\kern-0.75ex\hbox{$#1$}\vss}}}
\newcommand{\arrowvec}{\overrightarrow}
\renewcommand{\vec}{\mathbf}
\newcommand{\veci}{{\boldsymbol{\hat{\imath}}}}
\newcommand{\vecj}{{\boldsymbol{\hat{\jmath}}}}
\newcommand{\veck}{{\boldsymbol{\hat{k}}}}
\newcommand{\vecl}{\boldsymbol{\l}}
\newcommand{\utan}{\vec{\hat{t}}}
\newcommand{\unormal}{\vec{\hat{n}}}
\newcommand{\ubinormal}{\vec{\hat{b}}}

\newcommand{\dotp}{\bullet}
\newcommand{\cross}{\boldsymbol\times}
\newcommand{\grad}{\boldsymbol\nabla}
\newcommand{\divergence}{\grad\dotp}
\newcommand{\curl}{\grad\cross}
%% Simple horiz vectors
\renewcommand{\vector}[1]{\left\langle #1\right\rangle}


\outcome{Compute angles}

\title{1.1 Angles}

\begin{document}

\begin{abstract}
We will compute angles
\end{abstract}

\maketitle

\begin{definition}


\end{definition}


\section{Angles}

 \begin{tikzpicture}[thick]
  \draw (0,0) node[left]{$A$}  -- (4,0) node[midway,below]{$f$} -- (4,2) node[midway,right]{$\frac{x}{2}$}-- cycle;
\end{tikzpicture}





  \begin{tikzpicture}
    \coordinate (origo) at (0,0);
    \coordinate (pivot) at (1,5);

    % draw axes
    \fill[black] (origo) circle (0.05);
    \draw[thick,gray,->] (origo) -- ++(4,0) node[black,right] {$x$};
    \draw[thick,gray,->] (origo) -- ++(0,-4) node (mary) [black,below] {$y$};

    % draw roof
    \fill[pattern = north east lines] ($ (origo) + (-1,0) $) rectangle ($ (origo) + (1,0.5) $);
    \draw[thick] ($ (origo) + (-1,0) $) -- ($ (origo) + (1,0) $);

    \draw[thick] (origo) -- ++(300:3) coordinate (bob);
    \fill (bob) circle (0.2);

    \pic [draw, ->, "$\theta$", angle eccentricity=1.5] {angle = mary--origo--bob};
  \end{tikzpicture}


\begin{tikzpicture}
\coordinate[label=below left:$A$] (A) at (0,0);
\coordinate[label=below right:$x$] (X) at (6,1);
\coordinate[label=above left:$y$] (Y) at (3,5);

\draw[thick] (X) -- (A) -- (Y);

% Mark the angle XAY
\begin{scope}
\path[clip] (A) -- (X) -- (Y);
\fill[red, opacity=0.5, draw=black] (A) circle (5mm);
\node at ($(A)+(30:7mm)$) {$\theta$};
\end{scope}

\end{tikzpicture}



\begin{tikzpicture}
 \draw  (3,-1) coordinate (a) node[right] {a}   -- (0,0) coordinate (b) node[left] {b}
  -- (2,2) coordinate (c) node[above right] {c}
  pic["$\alpha$", draw=orange, <->, angle eccentricity=1.2, angle radius=1cm]
  {angle=a--b--c};
    
  \pic [draw=blue, text=blue, ->, "$\theta$", angle eccentricity=1.5] {angle = a--b--c};
\end{tikzpicture}




\begin{enumerate}


\item[1.] 

\begin{problem}
Angle angle whose measure is $18^\circ$ is called \wordChoice{\choice[correct]{acute} 
\choice{obtuse }\choice{right}\choice{straight}}
\end{problem}

\begin{enumerate}

\item[a.] $\displaystyle $
\item[b.] $\displaystyle $

 
\end{enumerate}


\item[5.] Natural exponential function:
\[
\int e^x \; dx = e^x + C.
\]


\end{enumerate}








\section{Problems}






\begin{problem}(problem 1)

\begin{hint}



\begin{center}

\end{center}

\end{hint}

\[
\answer[given]{10^\circ} 
\]

\end{problem}




\end{document}

















\section{Combining functions}

The \textbf{constant multiple rule}:
\[
\int af(x) \; dx = a\int f(x) \; dx,
\]
where $a$ is any constant.\\

The \textbf{sum and difference rules}:
\[
\int \left[f(x) \pm g(x) \right] \; dx = \int f(x) \; dx \pm \int g(x) \; dx.
\]



\item[initial value] 
given f' and f(0), find f

\item[second derivative]
given f'' and f'(0), f(0), find f

\item[Indefinite integrals]
\[
\int f'(x) \; dx = f(x) + C \;\; \text{or} \;\; \int f(x) \; dx = F(x) + C.
\]

\item[Interaction with Constants]
\[
\int [af'(x) + bg'(x)] \; dx = a \int f'(x) \; dx + b \int g'(x) \; dx.
\]

\item[Trig Functions]
\[
\int \cos(x) \; dx = \sin(x) + C \:\:  \text{and} \;\; \int \sec^2(x) \; dx = \tan(x) + C.
\]

\item[Power Functions]
\[
\int x^2 \; dx = \frac{x^3}{3} + C \;\; \text{and} \;\; \int \frac{1}{x} \; dx = \ln|x| + C.
\]

\item[Power Rule for Indefinite Integrals]
\[
\int x^n \; dx = \frac{x^{n+1}}{n+1} + C, n\neq -1.
\]

\item[examples for the power rule]
\[
\int \frac{1}{2\sqrt x} \; dx = \sqrt x + C \;\; \text{and} \;\; \int x^{-3} \; dx = -\frac{x^{-2}}{2} + C = -\frac{1}{2x^2} + C.
\]

\item[exponential functions]
\[
\int be^{ax} \; dx = \frac{b}{a} e^{ax} + C
\]

\item[other examples]
\[
\int \frac{1}{x^2 + a^2} \; dx = \frac{1}{a} \tan^{-1}(\frac{x}{a}) + C \;\; \text{and} \;\; \int f'(ax) \; dx = \frac{1}{a} f(ax) + C.
\]

\item[compositions]
\[
\int f(g(x))g'(x) \; dx = F(g(x)) + C.
\]

\item[products]
\[
\int f(x)g'(x) \; dx = f(x)g(x) + \int f'(x)g(x) \; dx \;\; \text{or} \;\; \int u\; dv = uv - \int v \; du.
\]

\item[rational functions]
\[
\int \frac{p(x)}{(x-a)^n(x^2 + b^2)} \; dx = \int \frac{A_1}{x-a} + \frac{A_2}{(x-a)^2} + \cdots + \frac{A_n}{(x-a)^n} + \frac{Bx+C}{x^2 + b^2} \; dx.
\]

\item[trig integrals]
\[
\int \sin^m(x) \cos^n(x) \; dx \;\; \text{and} \int\frac{1}{x^2\sqrt{x^2-a^2}} \; dx = \frac{1}{a^2} \int \cos \theta \; d\theta.
\]


\end{enumerate}

\end{example}


\end{document}

Integration by parts in economics

Present value of a flow
Suppose P(t) is the annual rate at which income is flowing at time
t. Given r the interest rate. Consider a period between a and b, if
we partition the time interval with sub-intervals [t i−1 , t i ], for
i = 1, .., n with t 0 = a and t n = b we know that we can
approximate the income in any sub-interval as P(t i )(t i − t i−1 ). The
discount factor will be e −rt i .
Using integral to get the present value of the flow on the whole
period
Z b
e −rt P(t)dt



An example of improper integral in economics
Denoting by c(t) consumption at time t, by U the instantaneous
utility function and r a discount rate. An integral
Z ∞
U(c(t))e −rt dt
t 0
represents the present value of future cumulative utility coming
from consumption.

Applications in mathematics
Laplace transform
gamma function
prob density function (doubly improper integral) int 1/(1+x^2)
computer science int of download speed = amount downloaded
