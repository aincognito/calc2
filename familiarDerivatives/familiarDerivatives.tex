\documentclass{ximera}
%\usepackage{tcolorbox}
%% You can put user macros here
%% However, you cannot make new environments



\newcommand{\ffrac}[2]{\frac{\text{\footnotesize $#1$}}{\text{\footnotesize $#2$}}}
\newcommand{\vasymptote}[2][]{
    \draw [densely dashed,#1] ({rel axis cs:0,0} -| {axis cs:#2,0}) -- ({rel axis cs:0,1} -| {axis cs:#2,0});
}


\graphicspath{{./}{firstExample/}}

\usepackage{amsmath}
\usepackage{amssymb}
\usepackage{array}
\usepackage[makeroom]{cancel} %% for strike outs
\usepackage{pgffor} %% required for integral for loops
\usepackage{tikz}
\usepackage{tikz-cd}
\usepackage{tkz-euclide}
\usetikzlibrary{shapes.multipart}


\usetkzobj{all}
\tikzstyle geometryDiagrams=[ultra thick,color=blue!50!black]


\usetikzlibrary{arrows}
\tikzset{>=stealth,commutative diagrams/.cd,
  arrow style=tikz,diagrams={>=stealth}} %% cool arrow head
\tikzset{shorten <>/.style={ shorten >=#1, shorten <=#1 } } %% allows shorter vectors

\usetikzlibrary{backgrounds} %% for boxes around graphs
\usetikzlibrary{shapes,positioning}  %% Clouds and stars
\usetikzlibrary{matrix} %% for matrix
\usepgfplotslibrary{polar} %% for polar plots
\usepgfplotslibrary{fillbetween} %% to shade area between curves in TikZ



%\usepackage[width=4.375in, height=7.0in, top=1.0in, papersize={5.5in,8.5in}]{geometry}
%\usepackage[pdftex]{graphicx}
%\usepackage{tipa}
%\usepackage{txfonts}
%\usepackage{textcomp}
%\usepackage{amsthm}
%\usepackage{xy}
%\usepackage{fancyhdr}
%\usepackage{xcolor}
%\usepackage{mathtools} %% for pretty underbrace % Breaks Ximera
%\usepackage{multicol}



\newcommand{\RR}{\mathbb R}
\newcommand{\R}{\mathbb R}
\newcommand{\C}{\mathbb C}
\newcommand{\N}{\mathbb N}
\newcommand{\Z}{\mathbb Z}
\newcommand{\dis}{\displaystyle}
%\renewcommand{\d}{\,d\!}
\renewcommand{\d}{\mathop{}\!d}
\newcommand{\dd}[2][]{\frac{\d #1}{\d #2}}
\newcommand{\pp}[2][]{\frac{\partial #1}{\partial #2}}
\renewcommand{\l}{\ell}
\newcommand{\ddx}{\frac{d}{\d x}}

\newcommand{\zeroOverZero}{\ensuremath{\boldsymbol{\tfrac{0}{0}}}}
\newcommand{\inftyOverInfty}{\ensuremath{\boldsymbol{\tfrac{\infty}{\infty}}}}
\newcommand{\zeroOverInfty}{\ensuremath{\boldsymbol{\tfrac{0}{\infty}}}}
\newcommand{\zeroTimesInfty}{\ensuremath{\small\boldsymbol{0\cdot \infty}}}
\newcommand{\inftyMinusInfty}{\ensuremath{\small\boldsymbol{\infty - \infty}}}
\newcommand{\oneToInfty}{\ensuremath{\boldsymbol{1^\infty}}}
\newcommand{\zeroToZero}{\ensuremath{\boldsymbol{0^0}}}
\newcommand{\inftyToZero}{\ensuremath{\boldsymbol{\infty^0}}}


\newcommand{\numOverZero}{\ensuremath{\boldsymbol{\tfrac{\#}{0}}}}
\newcommand{\dfn}{\textbf}
%\newcommand{\unit}{\,\mathrm}
\newcommand{\unit}{\mathop{}\!\mathrm}
%\newcommand{\eval}[1]{\bigg[ #1 \bigg]}
\newcommand{\eval}[1]{ #1 \bigg|}
\newcommand{\seq}[1]{\left( #1 \right)}
\renewcommand{\epsilon}{\varepsilon}
\renewcommand{\iff}{\Leftrightarrow}

\DeclareMathOperator{\arccot}{arccot}
\DeclareMathOperator{\arcsec}{arcsec}
\DeclareMathOperator{\arccsc}{arccsc}
\DeclareMathOperator{\si}{Si}
\DeclareMathOperator{\proj}{proj}
\DeclareMathOperator{\scal}{scal}
\DeclareMathOperator{\cis}{cis}
\DeclareMathOperator{\Arg}{Arg}
%\DeclareMathOperator{\arg}{arg}
\DeclareMathOperator{\Rep}{Re}
\DeclareMathOperator{\Imp}{Im}
\DeclareMathOperator{\sech}{sech}
\DeclareMathOperator{\csch}{csch}
\DeclareMathOperator{\Log}{Log}

\newcommand{\tightoverset}[2]{% for arrow vec
  \mathop{#2}\limits^{\vbox to -.5ex{\kern-0.75ex\hbox{$#1$}\vss}}}
\newcommand{\arrowvec}{\overrightarrow}
\renewcommand{\vec}{\mathbf}
\newcommand{\veci}{{\boldsymbol{\hat{\imath}}}}
\newcommand{\vecj}{{\boldsymbol{\hat{\jmath}}}}
\newcommand{\veck}{{\boldsymbol{\hat{k}}}}
\newcommand{\vecl}{\boldsymbol{\l}}
\newcommand{\utan}{\vec{\hat{t}}}
\newcommand{\unormal}{\vec{\hat{n}}}
\newcommand{\ubinormal}{\vec{\hat{b}}}

\newcommand{\dotp}{\bullet}
\newcommand{\cross}{\boldsymbol\times}
\newcommand{\grad}{\boldsymbol\nabla}
\newcommand{\divergence}{\grad\dotp}
\newcommand{\curl}{\grad\cross}
%% Simple horiz vectors
\renewcommand{\vector}[1]{\left\langle #1\right\rangle}


\outcome{Learn the derivatives of familiar functions}

\title{2.10 Table of Derivatives}

%\newcommand{\ffrac}[2]{\frac{\mbox{\footnotesize $#1$}}{\mbox{\footnotesize $#2$}}}
%\newcommand{\vasymptote}[2][]{
%    \draw [densely dashed,#1] ({rel axis cs:0,0} -| {axis cs:#2,0}) -- ({rel axis cs:0,1} -| {axis cs:#2,0});
%}


\begin{document}

\begin{abstract}
We learn the derivatives of many familiar functions.
\end{abstract}

\maketitle


\section{Table of Derivatives}

In this section we present the derivatives of the functions we have seen previously.  

\begin{center}
\[
\begin{array}{c|c|c}
		Type & f(x)= & f'(x)=  \\
		\hline
		\text{Constant} & c & 0 \\
		\text{Linear} & mx & m  \\
		\text{Power} & x^n & nx^{n-1}  \\
		\hline
		\text{Trig} & \sin(x) & \cos(x) \\
		\text{Trig} & \cos(x) & -\sin(x) \\
		\hline
		\text{Trig} & \tan(x) & \sec^2(x) \\
		\text{Trig} & \sec(x) & \sec(x)\tan(x) \\
		\hline
		\text{Trig} & \cot(x) & -\csc^2(x)\ \\
		\text{Trig} & \csc(x) & -\csc(x)\cot(x) \\
		\hline
		\text{Inv Trig} & \sin^{-1}(x) & \frac{1}{\sqrt{1-x^2}} \\
		\text{Inv Trig} & \cos^{-1}(x) & -\frac{1}{\sqrt{1-x^2}} \\
		\text{Inv Trig} & \tan^{-1}(x) & \frac{1}{1+x^2} \\
		\hline
		\text{Exponential} & e^x & e^x \\
		\text{Exponential} & a^x & a^x\ln(a) \\
		\hline
		\text{Log} & \ln(x) & \frac{1}{x} \\
		\text{Log} & \log_a(x) & \frac{1}{x\ln(a)}
	\end{array}
    \]
\end{center}



The following table consists of the chain rule versions of the basic derivative formulas above.

\begin{center}
\[
\begin{array}{c|c|c}
		Type & f(x)= & f'(x)=  \\
		\hline
		\text{Power} & g^n(x) & ng^{n-1}(x)\cdot g'(x)  \\[6pt]
		\hline
		\text{Trig} & \sin\left(g(x)\right) & \cos\left(g(x)\right) \cdot g'(x) \\[6pt]
		\text{Trig} & \cos\left(g(x)\right) & -\sin\left(g(x)\right)\cdot g'(x)  \\[6pt]
		\hline
		\text{Trig} & \tan\left(g(x)\right) & \sec^2\left(g(x)\right) \cdot g'(x) \\[6pt]
		\text{Trig} & \sec\left(g(x)\right) & \sec\left(g(x)\right)\tan\left(g(x)\right) \cdot g'(x) \\[6pt]
		\hline
		\text{Trig} & \cot\left(g(x)\right) & -\csc^2\left(g(x)\right) \cdot g'(x)  \\[6pt]
		\text{Trig} & \csc\left(g(x)\right) & -\csc\left(g(x)\right)\cot\left(g(x)\right) \cdot g'(x)  \\[6pt]
		\hline
		\text{Inv Trig} & \sin^{-1}\left(g(x)\right) & \displaystyle{\frac{1}{\sqrt{1-g^2(x)}}} \cdot g'(x) = \frac{g'(x)}{\sqrt{1-g^2(x)}}  \\[6pt]
		\text{Inv Trig} & \cos^{-1}\left(g(x)\right) & \displaystyle{-\frac{1}{\sqrt{1-g^2(x)}}} \cdot g'(x) = -\frac{g'(x)}{\sqrt{1-g^2(x)}}\\[6pt]
		\text{Inv Trig} & \tan^{-1}\left(g(x)\right) & \displaystyle{\frac{1}{1+g^2(x)}} \cdot g'(x) = \frac{g'(x)}{1+g^2(x)} \\
		\hline
		\text{Exponential} & e^{g(x)} & e^{g(x)} \cdot g'(x) \\[6pt]
		\text{Exponential} & a^{g(x)} & a^{g(x)}\ln(a) \cdot g'(x) \\[6pt]
		\hline
		\text{Log} & \ln\left(g(x)\right) & \displaystyle{\frac{1}{g(x)}}\cdot g'(x) = \frac{g'(x)}{g(x)} \\[6pt]
		\text{Log} & \log_a\left(g(x)\right) & \displaystyle{\frac{1}{g(x)\ln(a)}}\cdot g'(x) = \frac{g'(x)}{g(x)\ln(a)}
	\end{array}
    \]
\end{center}








\end{document}

























\begin{onlineOnly}
  \begin{center}
    \desmos{hw5vyrfojt}{800}{600}
  \end{center}
\end{onlineOnly}


\section{Leibniz Notation}

In this section we introduce another notation for the derivative. The slope of a line segment is given by
\[m = \frac{\text{rise}}{\text{run}} = \frac{\Delta y}{\Delta x}\]
and since the derivative represents a formula for the slope of a tangent line to a curve $y = f(x)$,
a natural notation for the derivative is Leibniz' notation:
\[ f'(x) = \frac{dy}{dx},\]
which closely resembles the delta notation for slope.





\begin{center}

\bf{Examples of Leibniz Notation}

\end{center}


\begin{example} %example 21
If $y = x^2$ then $\displaystyle{\frac{dy}{dx} = 2x.}$
\end{example}

\begin{example} %example 22
If $y = \sqrt x$ then $\displaystyle{\frac{dy}{dx} = \frac{1}{2\sqrt x}.}$
\end{example}

\begin{example} %example 23
If $y = \frac{1}{x}$ then $\displaystyle{\frac{dy}{dx} = -\frac{1}{x^2}.}$
\end{example}

\begin{example} %example 24
If $y = \sin(x)$ then $\displaystyle{\frac{dy}{dx} = \cos(x).}$
\end{example}

\begin{example} %example 25
If $y = \cos(x)$ then $\displaystyle{\frac{dy}{dx} = -\sin(x).}$
\end{example}

There is also a way to describe the derivative using Leibniz style notation without explicitly stating the function first:

\[f'(x) = \frac{d}{dx}[f(x)].\]

Some examples in this format:

\begin{example} %example 26
$\displaystyle{\frac{d}{dx}\big( \tan(x) \big) = \frac{d}{dx} \tan(x) = \sec^2(x).}$
\end{example}

\begin{example} %example 27
$\displaystyle{\frac{d}{dx}\big( \sec(x) \big) = \frac{d}{dx} \sec(x) =\sec(x)\tan(x).}$
\end{example}

\begin{example} %example 28
$\displaystyle{\frac{d}{dx}\big( e^x \big) = e^x.}$
\end{example}

\begin{example} %example 29
$\displaystyle{\frac{d}{dx}\big( \ln(x) \big) = \frac{d}{dx} \ln(x)= \frac{1}{x}.}$
\end{example}

\begin{example} %example 30
$\displaystyle{\frac{d}{dx}\big( \tan^{-1}(x) \big) = \frac{d}{dx} \tan^{-1}(x) = \frac{1}{1+x^2}.}$
\end{example}




















\begin{description}
\item[Constant:] If $f(x) = c$ then $f'(x) = 0$, where $c$ is any constant.
\item[Linear:] If $f(x) = ax$ then $f'(x) = a$, where $a$ is any nonzero constant.
\item[Power Rule:] If $f(x) = x^n$ then $f'(x) = nx^{n-1}$ where $n$ is a nonzero constant.
\item[Trig:] If $f(x) = \sin(x)$ then $f'(x) = \cos(x)$.
\item[Trig:] If $f(x) = \cos(x)$ then $f'(x) = -\sin(x)$.
\item[Trig:] If $f(x) = \tan(x)$ then $f'(x) = \sec^2(x)$.
\item[Trig:] If $f(x) = \sec(x)$ then $f'(x) = \sec(x)\tan(x)$.
\item[Trig:] If $f(x) = \cot(x)$ then $f'(x) = -\csc^2(x)$.
\item[Trig:] If $f(x) = \csc(x)$ then $f'(x) = -\csc(x)\cot(x)$.
\item[Inverse Trig:] If $f(x)  = \sin^{-1}(x)$ then $f'(x) = \frac{1}{\sqrt{1-x^2}}$.
\item[Inverse Trig:] If $f(x)  = \cos^{-1}(x)$ then $f'(x) = -\frac{1}{\sqrt{1-x^2}}$.
\item[Inverse Trig:] If $f(x)  = \tan^{-1}(x)$ then $f'(x) = \frac{1}{1+x^2}$.
\item[Exponential:] If $f(x) = e^x$ then $f'(x) = e^x$.
\item[Exponential:] If $f(x) = a^x$ then $f'(x) = a^x \cdot \ln(a)$.
%where $a>0$ and $a \neq 1$
\item[Log:] If $f(x) = \ln(x)$ then $f'(x) = \frac{1}{x}.$
\item[Log:] If $f(x) = \log_a(x)$ then $f'(x) = \frac{1}{x \ln(a)}.$
\end{description}
