\documentclass[handout]{ximera}

%% You can put user macros here
%% However, you cannot make new environments



\newcommand{\ffrac}[2]{\frac{\text{\footnotesize $#1$}}{\text{\footnotesize $#2$}}}
\newcommand{\vasymptote}[2][]{
    \draw [densely dashed,#1] ({rel axis cs:0,0} -| {axis cs:#2,0}) -- ({rel axis cs:0,1} -| {axis cs:#2,0});
}


\graphicspath{{./}{firstExample/}}

\usepackage{amsmath}
\usepackage{amssymb}
\usepackage{array}
\usepackage[makeroom]{cancel} %% for strike outs
\usepackage{pgffor} %% required for integral for loops
\usepackage{tikz}
\usepackage{tikz-cd}
\usepackage{tkz-euclide}
\usetikzlibrary{shapes.multipart}


\usetkzobj{all}
\tikzstyle geometryDiagrams=[ultra thick,color=blue!50!black]


\usetikzlibrary{arrows}
\tikzset{>=stealth,commutative diagrams/.cd,
  arrow style=tikz,diagrams={>=stealth}} %% cool arrow head
\tikzset{shorten <>/.style={ shorten >=#1, shorten <=#1 } } %% allows shorter vectors

\usetikzlibrary{backgrounds} %% for boxes around graphs
\usetikzlibrary{shapes,positioning}  %% Clouds and stars
\usetikzlibrary{matrix} %% for matrix
\usepgfplotslibrary{polar} %% for polar plots
\usepgfplotslibrary{fillbetween} %% to shade area between curves in TikZ



%\usepackage[width=4.375in, height=7.0in, top=1.0in, papersize={5.5in,8.5in}]{geometry}
%\usepackage[pdftex]{graphicx}
%\usepackage{tipa}
%\usepackage{txfonts}
%\usepackage{textcomp}
%\usepackage{amsthm}
%\usepackage{xy}
%\usepackage{fancyhdr}
%\usepackage{xcolor}
%\usepackage{mathtools} %% for pretty underbrace % Breaks Ximera
%\usepackage{multicol}



\newcommand{\RR}{\mathbb R}
\newcommand{\R}{\mathbb R}
\newcommand{\C}{\mathbb C}
\newcommand{\N}{\mathbb N}
\newcommand{\Z}{\mathbb Z}
\newcommand{\dis}{\displaystyle}
%\renewcommand{\d}{\,d\!}
\renewcommand{\d}{\mathop{}\!d}
\newcommand{\dd}[2][]{\frac{\d #1}{\d #2}}
\newcommand{\pp}[2][]{\frac{\partial #1}{\partial #2}}
\renewcommand{\l}{\ell}
\newcommand{\ddx}{\frac{d}{\d x}}

\newcommand{\zeroOverZero}{\ensuremath{\boldsymbol{\tfrac{0}{0}}}}
\newcommand{\inftyOverInfty}{\ensuremath{\boldsymbol{\tfrac{\infty}{\infty}}}}
\newcommand{\zeroOverInfty}{\ensuremath{\boldsymbol{\tfrac{0}{\infty}}}}
\newcommand{\zeroTimesInfty}{\ensuremath{\small\boldsymbol{0\cdot \infty}}}
\newcommand{\inftyMinusInfty}{\ensuremath{\small\boldsymbol{\infty - \infty}}}
\newcommand{\oneToInfty}{\ensuremath{\boldsymbol{1^\infty}}}
\newcommand{\zeroToZero}{\ensuremath{\boldsymbol{0^0}}}
\newcommand{\inftyToZero}{\ensuremath{\boldsymbol{\infty^0}}}


\newcommand{\numOverZero}{\ensuremath{\boldsymbol{\tfrac{\#}{0}}}}
\newcommand{\dfn}{\textbf}
%\newcommand{\unit}{\,\mathrm}
\newcommand{\unit}{\mathop{}\!\mathrm}
%\newcommand{\eval}[1]{\bigg[ #1 \bigg]}
\newcommand{\eval}[1]{ #1 \bigg|}
\newcommand{\seq}[1]{\left( #1 \right)}
\renewcommand{\epsilon}{\varepsilon}
\renewcommand{\iff}{\Leftrightarrow}

\DeclareMathOperator{\arccot}{arccot}
\DeclareMathOperator{\arcsec}{arcsec}
\DeclareMathOperator{\arccsc}{arccsc}
\DeclareMathOperator{\si}{Si}
\DeclareMathOperator{\proj}{proj}
\DeclareMathOperator{\scal}{scal}
\DeclareMathOperator{\cis}{cis}
\DeclareMathOperator{\Arg}{Arg}
%\DeclareMathOperator{\arg}{arg}
\DeclareMathOperator{\Rep}{Re}
\DeclareMathOperator{\Imp}{Im}
\DeclareMathOperator{\sech}{sech}
\DeclareMathOperator{\csch}{csch}
\DeclareMathOperator{\Log}{Log}

\newcommand{\tightoverset}[2]{% for arrow vec
  \mathop{#2}\limits^{\vbox to -.5ex{\kern-0.75ex\hbox{$#1$}\vss}}}
\newcommand{\arrowvec}{\overrightarrow}
\renewcommand{\vec}{\mathbf}
\newcommand{\veci}{{\boldsymbol{\hat{\imath}}}}
\newcommand{\vecj}{{\boldsymbol{\hat{\jmath}}}}
\newcommand{\veck}{{\boldsymbol{\hat{k}}}}
\newcommand{\vecl}{\boldsymbol{\l}}
\newcommand{\utan}{\vec{\hat{t}}}
\newcommand{\unormal}{\vec{\hat{n}}}
\newcommand{\ubinormal}{\vec{\hat{b}}}

\newcommand{\dotp}{\bullet}
\newcommand{\cross}{\boldsymbol\times}
\newcommand{\grad}{\boldsymbol\nabla}
\newcommand{\divergence}{\grad\dotp}
\newcommand{\curl}{\grad\cross}
%% Simple horiz vectors
\renewcommand{\vector}[1]{\left\langle #1\right\rangle}


\pgfplotsset{compat=1.13}

\outcome{Learn the arithmetic of complex numbers}

\title{1.7 Hyperbolic Functions}

\begin{document}

\begin{abstract}
We learn the basic properties of the hyperbolic functions.
\end{abstract}

\maketitle

The hyperbolc sine and hyperbolic cosine functions are defined as
\[
\sinh x = \frac{e^x - e^{-x}}{2} \;\; \mbox{and} \;\; \cosh x = \frac{e^x + e^{-x}}{2}
\]



\begin{example}[example 1]
Solve the equation $\cosh x = c$ for $x$.\\
We use the definition of the hyperbolic cosine:
\[
\cosh x = \frac{e^x + e^{-x}}{2} = 6
\]
Ergo
\[
e^x +e^{-x} = 12
\]
Multiplying by $e^x$, we get
\[
e^{2x} - 12e^x +1 = 0 \;\; \mbox{or} \;\; u^2 - 12u + 1 = 0
\]
where $u = e^x$.  Now the quadratic formula gives
\[
u = \frac{12 \pm \sqrt{144 - 4}}{2} = 6 \pm \sqrt{35}
\]
Finally, we get the two solutions by taking the natural logarithm of $u$:
\[
x = \ln\left(6 \pm \sqrt{35}\right)
\]
These answers are negatives of one another (verify).
\end{example}

\begin{problem}[problem 1a]
Solve the following equation $\cosh x = 3$\\
The answers are (in increasing order):\\
$x = \answer{\ln(3-2\sqrt 2)}$ and $x = \answer{\ln(3+2\sqrt 2)}$
\end{problem}


\begin{problem}[problem 1b]
Solve the following equation $\sinh x = 3$\\
The answer is:\\
$x = \answer{\ln(3+\sqrt{10} 2)}$
\end{problem}

\begin{problem}(problem 1c) 
Find inverses for the hyperbolic sine and hyperbolic cosine functions.
\end{problem}

The hyperbolic sine function is an odd function:
\[
\sinh(-x) = -\sinh(x)
\]
and the hyperbolic cosine function is even:
\[
\cosh(-x) = -\cosh(x)
\]

The hyperbolic sine and cosine satisfy the fundamental identity
\[
\cosh^2 x - \sinh^2 x =1
\]
which means that the point $(\cosh t, \sinh t)$ lies on the (right branch of) the hyperbola
\[
x^2 -y^2 = 1
\]
This is why the functions are refered to as the hyperbolic functions.

The other four hyperbolic functions can be created from the hyperbolic sine and hyperbolic cosine functions:
\[
\tanh x = \frac{\sinh x}{\cosh x},\;\; \coth x = \frac{\cosh x}{\sinh x},
\]
\[
\sech x = \frac{1}{\cosh x}, \;\; \mbox{and}\;\; \csch x = \frac{1}{\sinh x}
\]

\begin{example}[example 2]
Find the derivative of the hyperbolic sine function.\\
The derivative of the hyperbolic sine function is given by
\[
\frac{d}{dx} \sinh(x) = \frac{d}{dx} \left( \frac{e^x -e^{-x}}{2} \right) = \frac{e^x + e^{-x}}{2} = \cosh x
\]
\end{example}

\begin{problem}(problem 2)
Find the derivatives of the other 5 hyperbolic functions.\\
$\frac{d}{dx} \cosh x = \answer{\sinh x}$\\
$\frac{d}{dx} \tanh x = \answer{\sech^2 x}$\\
$\frac{d}{dx} \sech x = \answer{-\sech x \tanh x}$\\
$\frac{d}{dx} \coth x = \answer{-\csch^2 x}$\\
$\frac{d}{dx} \csch x = \answer{-\csch x \coth x}$
\end{problem}


\end{document}


\section{Plotting a Complex Number}

Just as the real numbers can be represented visually, or {\bf geometrically}, by the real number line, 
complex numbers can be represented by the complex plane.  
Each complex number will correspond to a point in the plane and visa-versa. The plane is analogous to the $xy$-plane.  
The difference is that the vertical axis is labeled as $i, 2i, 3i, ...$ instead of $1, 2, 3, ...$.
We plot the complex number $a+bi$ in the complex plane just as we would plot the point $(a,b)$ in the $xy$-plane.


\begin{example}[Example 1]
Plot the complex number $1+2i$ in the complex plane.

\begin{image}
\begin{tikzpicture}
\draw[<->, thick] (-3,0)--(3,0);
 \node[blue] at (2.9,.5){Real Axis};
\draw[<->, thick] (0, 3)--(0,-3) node[ below=10pt, blue]{\large The Complex Plane, $\mathbb{C}$};
\node[blue] at (-1.5, 2.9){Imaginary Axis};

\draw[thin] (1,.2)--(1,-.2) node[below]{$1$};
\draw[thin] (-1,.2)--(-1,-.2) node[below]{$-1$};
\draw[thin] (2,.2)--(2,-.2) node[below]{$2$};
\draw[thin] (-2,.2)--(-2,-.2) node[below]{$-2$};

\draw[thin] (.2,1)--(-.2,1) node[left]{$i$};
\draw[thin] (.2,-1)--(-.2,-1) node[left]{$-i$};
\draw[thin] (.2,2)--(-.2,2) node[left]{$2i$};
\draw[thin] (.2,-2)--(-.2,-2) node[left]{$-2i$};

\draw[mark=*,mark size=1pt,mark options={color=blue}] plot coordinates {(1,2)} node[above right, blue]{$1+2i$};

\draw[ultra thin, dashed, blue] (1,0.3) -- (1,1.9);
\draw[ultra thin, dashed, blue] (.9,2) --(0.2,2);

\end{tikzpicture}
\end{image}


\end{example}

\begin{problem}(Problem 1)
Plot the complex numbers in the complex plane:
\begin{flalign*}
\quad (i)& \quad 3+i & \\
\quad (ii)& \quad -2-2i \\
\quad (iii)& \quad 4i \\
\quad (iv)& \quad -5
\end{flalign*}

\end{problem}



\section{The Real and Imaginary Parts}
Geometrically, the real part of a complex number $a+bi$ its horizontal component and the imaginary part is its vertical component.  
It is important to note
that the imaginary part is a real number.

% 

\begin{image}
\begin{tikzpicture}
\draw[->, thick] (-0.4,0)--(3.8,0) node[midway, below=15pt,blue] {Real and Imaginary Parts};
%
%\node[blue] at (2.9,.5){Real Axis};
\draw[<-, thick] (0, 3)--(0,-0.4) ;
%\node[blue] at (-1.5, 2.9){Imaginary Axis};

\draw[thin] (2.6,.02)--(2.6,-.02) node[below]{$a$};


\draw[thin] (.02,1.8)--(-.02,1.8) node[left]{$bi$};


\draw[mark=*,mark size=1pt] plot coordinates {(2.6,1.8)} node[above right]{$a+bi$};

\draw[dashed,ultra thin, blue] (2.6,-0.05) -- (2.6,1.8) node[midway, right , blue]{$\Imp(a+bi)$};

%\draw[snake=brace, blue] (-.2, 0.3) -- (-.2, 1.5)  ;
\draw[dashed,ultra thin, blue] (-0.05,1.8) -- (2.6,1.8) node[midway, above, blue]{$\Rep(a+bi)$};

%\draw[snake=brace, mirror snake, blue] (0.01,-0.2) -- (2.59,-0.2) node[midway, below, blue]{Re($a+bi$)};

\end{tikzpicture}
\end{image}



\section{The Modulus of a Complex Number}
Just as a point in the $xy$-plane can be thought of as a vector emanating from the origin, a complex number can be interpreted in the same manner.
The magnitude of a complex number, considering it as a vector, is called the {\bf modulus}, denoted $|a+bi|$.
From the Pythagorean Theorem, we can see that
\[
|a+bi| = \sqrt{a^2+b^2}
\]

\begin{remark}
The modulus of a real number is just its absolute value.
\end{remark}

\begin{image}
\begin{tikzpicture}
\draw[->, thick] (-0.4,0)--(3.8,0) node[midway, below=15pt, blue]{ Modulus, $|a+bi|$};
%\node[blue] at (2.9,.5){Real Axis};
\draw[<-, thick] (0, 3)--(0,-0.4) ;
%\node[blue] at (-1.5, 2.9){Imaginary Axis};

\draw[ultra thin] (2.6,.1)--(2.6,-.1) node[below]{$a$};


\draw[ultra thin] (.1,1.8)--(-.1,1.8) node[left]{$bi$};


\draw[mark=*,mark size=1pt] plot coordinates {(2.6,1.8)} node[above right]{$a+bi$};

%\draw[ultra thin, dashed, blue] (2.6,0.3) -- (2.6,1.6) ;
%\draw[snake=brace, blue, mirror snake] (2.7, 0.1) -- (2.7, 1.7) node[midway, right , blue]{Im($a+bi$)};
%\draw[ultra thin, dashed, blue] (0.2,1.8) -- (2.4,1.8) node[midway, above, blue]{Re($a+bi$)};
%\draw[snake=brace, blue] (0.1,0.1) -- (2.5,1.9) node[midway, above, blue, sloped]{$|a+bi|$};
\draw[->, blue] (0,0) -- (2.58,1.78) node[midway, above, blue, sloped]{$|a+bi|$};

\end{tikzpicture}
\end{image}

\begin{example}[Example 2]
Find the modulus of the complex number $-5+6i$.\\
Using the formula $|a+bi| =\sqrt{a^2+b^2}$, we have
\[
|-5+6i| = \sqrt{(-5)^2 + 6^2} = \sqrt{61}
\]
\end{example}

\begin{problem}(Problem 2a)

Find the modulus of the complex number:
\begin{flalign*}
\quad (i)& \quad |-3i| = \answer{3} &\\
\quad (ii)& \quad |-12| = \answer{12} \\
\quad (iii)& \quad |6+8i| = \answer{10} \\
\quad (iv)& \quad |\overline{2+3i}| = \answer{\sqrt{13}}
\end{flalign*}

\end{problem}

\begin{problem}(Problem 2b)
Select all of the statements that are true for any complex number, $a+bi$:
\begin{selectAll}
\choice[correct] {$|\Rep(a+bi)| \leq |a+bi|$}
\choice[correct] {$|\Imp(a+bi)| \leq |a+bi|$}
\choice[correct] {$ |a+bi| \leq |\Rep(a+bi)| + |\Imp(a+bi)|$}
\choice[correct] {$\left|\overline{a+bi}\right| = |a+bi|$}
\end{selectAll}

\end{problem}

\begin{problem}(Problem 2c)
Prove each of the true statements from problem 2b, above.
\end{problem}


\section{The Polar Form of a Complex Number}
We can describe the position of a complex number in polar form using a magnitude, or radius, 
and a direction angle rather than using the rectangular form involving horizontal and vertical components. 
With the standard polar notations $r$ and $\theta$ for the radius and the direction angle, we can represent 
a complex number in polar form  
\[
r\cos \theta + ir\sin \theta = r\left(\cos \theta + i\sin \theta\right) = r \cis \theta
\]

\begin{image}
\begin{tikzpicture}
\draw[->, thick] (-0.4,0)--(3.8,0) node[midway, below=25pt, blue]{ Polar form, $r \cis \theta$};
%\node[blue] at (2.9,.5){Real Axis};
\draw[<-, thick] (0, 3)--(0,-0.4) ;
%\node[blue] at (-1.5, 2.9){Imaginary Axis};

\draw[ultra thin] (2.6,.1)--(2.6,-.1) node[below]{$a$};


\draw[ultra thin] (.1,1.8)--(-.1,1.8) node[left]{$bi$};

\draw[blue, dashed] (2.6, 0) -- (2.6, 1.8) node[midway, right , blue]{$r \sin \theta$};
\node[blue] at (1.3, -0.2){$r \cos \theta$};
%[snake=brace, blue, mirror snake]

\draw[mark=*,mark size=1pt] plot coordinates {(2.6,1.8)} node[above right]{$a+bi$};

%\draw[ultra thin, dashed, blue] (2.6,0.3) -- (2.6,1.6) ;
%\draw[snake=brace, blue, mirror snake] (2.7, 0.1) -- (2.7, 1.7) node[midway, right , blue]{Im($a+bi$)};
%\draw[ultra thin, dashed, blue] (0.2,1.8) -- (2.4,1.8) node[midway, above, blue]{Re($a+bi$)};
%\draw[snake=brace, blue] (0.1,0.1) -- (2.5,1.9) node[midway, above, blue, sloped]{$|a+bi|$};
\draw[ ->, blue] (0,0) -- (2.58,1.78) node[midway, above, blue, sloped]{$r$};
\node[blue] at (0.6, 0.2){$\theta$};
\end{tikzpicture}
\end{image}

\begin{remark}
The polar radius of a complex number is just its modulus:
\[
r = |a+bi|
\]
\end{remark}

\begin{remark}
Later, we will discover that $e^{i\theta} = \cis \theta$. This is known as Euler's Formula.
\end{remark}

\begin{example}[Example 3]
Find the polar form of the complex number $-\sqrt 3 + i$.\\
The polar radius is
\[
r = |-\sqrt 3 + i | =2
\]
To find the polar angle, $\theta$, we will use the equation
\[
\tan \theta = -\frac{\ 1}{\sqrt 3}
\]
We might erroneously conclude that
\[
\theta = \tan^{-1} \left(-\frac{\ 1}{\sqrt 3}\right) = -\frac{\pi}{6}
\]
However, the complex number  $-\sqrt 3 + i$ lies in the second quadrant so we add $\pi$:
\[
\theta = -\frac{\pi}{6} + \pi = \frac{5\pi}{6}
\]
We can now write the polar form:
\[
-\sqrt 3 +i = 2\cis \left(\frac{5\pi}{6}\right)
\]

\begin{image}
\begin{tikzpicture}
\draw[<->, thick] (-2,0)--(2,0);

\draw[<-, thick] (0, 2)--(0,-.3);

\draw[ultra thin] (-1.73,.1)--(-1.73,-.1) node[below]{$-\sqrt 3$};


\draw[ultra thin] (.1,1)--(-.1,1) node[left]{$i$};

\draw[mark=*,mark size=1pt] plot coordinates {(-1.73,1)} node[above left]{$-\sqrt 3 + i$};

\draw[blue, ->] (0,0) -- (-1.73,1) node[left, xshift = -1mm, yshift = -1mm, midway, blue]{$2$};

\draw[blue, ->] (3mm,0mm) arc (0:150:3mm) ;
\node[blue] at (.4,.4) {$\tfrac{5\pi}{6}$};
\end{tikzpicture}
\end{image}


\end{example}

\begin{problem}(Problem 3)

Express the complex number in polar form, $r\cis \theta$ where $0 \leq \theta < 2\pi$:
\begin{flalign*}
\quad (i)& \quad 2i = \answer{2cis(\pi/2)} &\\
\quad (ii)& \quad -3 = \answer{3cis(\pi)} \\
\quad (iii)& \quad 1+i = \answer{\sqrt 2 cis(\pi/4)} \\
\quad (iv)& \quad -2-2\sqrt 3i = \answer{4cis(4\pi/3)}
\end{flalign*}

\end{problem}

\end{document}





%node distance = 2 cm,





\begin{center}
\begin{foldable}
\unfoldable{Here is a video of Example 1}
%\youtube{###} %vid of example 1
\end{foldable}
\end{center}

\begin{problem} %problem #1
  Find the area between...
  \[
  f(x) =  ...
  \]
    \begin{hint}
      Set up a definite integral
    \end{hint}
    \begin{hint}
      Determine the top and bottom curves
    \end{hint}
    
		
		The area between the curves is
		 $\answer[given]{number}$
\end{problem}







\begin{example} %example #15
Find $h'(x)$ if $h(x) = x^{\sin(x)}$.\\
We will use the fact that the exponential and logarithm functions are inverses,
\[e^{\ln(x)} = x,\]
and the exponent property of logarithms, 
\[\ln(x^n) = n\ln(x),\]
to rewrite $h(x)$.  We have 
\[h(x) = x^{\sin(x)} = e^{\ln(x^{\sin(x)})} = e^{\sin(x)\ln(x)}\]
and we can now compute $h'(x)$ using a combination of the chain rule and product rule.
We can write $h(x)$ as a composition, $f(g(x))$ with 
\[g(x) = \sin(x)\ln(x) \quad \text{and} \quad f(x) = e^x.\]
Then to find $g'(x)$ we us the product rule and we get $g'(x) = \frac{\sin(x)}{x} + \cos(x)\ln(x)$.
Next $f'(x) = e^x$ and 
hence $f'(g(x)) = e^{g(x)} = e^{\sin(x)\ln(x)} = x^{\sin(x)}$.
We can then conclude $h'(x) = f'(g(x))g'(x) = x^{\sin(x)} \left[ \frac{\sin(x)}{x} + \cos(x)\ln(x)\right]$.
\end{example}

%more question formats below













%\begin{verbatim}
\begin{question}
What is your favorite color?
\begin{multipleChoice}
\choice[correct]{Rainbow}
\choice{Blue}
\choice{Green}
\choice{Red}
\end{multipleChoice}
\begin{freeResponse}
Hello
\end{freeResponse}
\end{question}
%\end{verbatim}





\begin{question}
  Which one will you choose?
  \begin{multipleChoice}
    \choice[correct]{I'm correct.}
    \choice{I'm wrong.}
    \choice{I'm wrong too.}
  \end{multipleChoice}
\end{question}


\begin{question}
  Which one will you choose?
  \begin{selectAll}
    \choice[correct]{I'm correct.}
    \choice{I'm wrong.}
    \choice[correct]{I'm also correct.}
    \choice{I'm wrong too.}
  \end{selectAll}
\end{question}


\begin{freeResponse}
What is the chain rule used for?
\end{freeResponse}
