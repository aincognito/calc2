\documentclass[handout]{ximera}

%% You can put user macros here
%% However, you cannot make new environments



\newcommand{\ffrac}[2]{\frac{\text{\footnotesize $#1$}}{\text{\footnotesize $#2$}}}
\newcommand{\vasymptote}[2][]{
    \draw [densely dashed,#1] ({rel axis cs:0,0} -| {axis cs:#2,0}) -- ({rel axis cs:0,1} -| {axis cs:#2,0});
}


\graphicspath{{./}{firstExample/}}

\usepackage{amsmath}
\usepackage{amssymb}
\usepackage{array}
\usepackage[makeroom]{cancel} %% for strike outs
\usepackage{pgffor} %% required for integral for loops
\usepackage{tikz}
\usepackage{tikz-cd}
\usepackage{tkz-euclide}
\usetikzlibrary{shapes.multipart}


\usetkzobj{all}
\tikzstyle geometryDiagrams=[ultra thick,color=blue!50!black]


\usetikzlibrary{arrows}
\tikzset{>=stealth,commutative diagrams/.cd,
  arrow style=tikz,diagrams={>=stealth}} %% cool arrow head
\tikzset{shorten <>/.style={ shorten >=#1, shorten <=#1 } } %% allows shorter vectors

\usetikzlibrary{backgrounds} %% for boxes around graphs
\usetikzlibrary{shapes,positioning}  %% Clouds and stars
\usetikzlibrary{matrix} %% for matrix
\usepgfplotslibrary{polar} %% for polar plots
\usepgfplotslibrary{fillbetween} %% to shade area between curves in TikZ



%\usepackage[width=4.375in, height=7.0in, top=1.0in, papersize={5.5in,8.5in}]{geometry}
%\usepackage[pdftex]{graphicx}
%\usepackage{tipa}
%\usepackage{txfonts}
%\usepackage{textcomp}
%\usepackage{amsthm}
%\usepackage{xy}
%\usepackage{fancyhdr}
%\usepackage{xcolor}
%\usepackage{mathtools} %% for pretty underbrace % Breaks Ximera
%\usepackage{multicol}



\newcommand{\RR}{\mathbb R}
\newcommand{\R}{\mathbb R}
\newcommand{\C}{\mathbb C}
\newcommand{\N}{\mathbb N}
\newcommand{\Z}{\mathbb Z}
\newcommand{\dis}{\displaystyle}
%\renewcommand{\d}{\,d\!}
\renewcommand{\d}{\mathop{}\!d}
\newcommand{\dd}[2][]{\frac{\d #1}{\d #2}}
\newcommand{\pp}[2][]{\frac{\partial #1}{\partial #2}}
\renewcommand{\l}{\ell}
\newcommand{\ddx}{\frac{d}{\d x}}

\newcommand{\zeroOverZero}{\ensuremath{\boldsymbol{\tfrac{0}{0}}}}
\newcommand{\inftyOverInfty}{\ensuremath{\boldsymbol{\tfrac{\infty}{\infty}}}}
\newcommand{\zeroOverInfty}{\ensuremath{\boldsymbol{\tfrac{0}{\infty}}}}
\newcommand{\zeroTimesInfty}{\ensuremath{\small\boldsymbol{0\cdot \infty}}}
\newcommand{\inftyMinusInfty}{\ensuremath{\small\boldsymbol{\infty - \infty}}}
\newcommand{\oneToInfty}{\ensuremath{\boldsymbol{1^\infty}}}
\newcommand{\zeroToZero}{\ensuremath{\boldsymbol{0^0}}}
\newcommand{\inftyToZero}{\ensuremath{\boldsymbol{\infty^0}}}


\newcommand{\numOverZero}{\ensuremath{\boldsymbol{\tfrac{\#}{0}}}}
\newcommand{\dfn}{\textbf}
%\newcommand{\unit}{\,\mathrm}
\newcommand{\unit}{\mathop{}\!\mathrm}
%\newcommand{\eval}[1]{\bigg[ #1 \bigg]}
\newcommand{\eval}[1]{ #1 \bigg|}
\newcommand{\seq}[1]{\left( #1 \right)}
\renewcommand{\epsilon}{\varepsilon}
\renewcommand{\iff}{\Leftrightarrow}

\DeclareMathOperator{\arccot}{arccot}
\DeclareMathOperator{\arcsec}{arcsec}
\DeclareMathOperator{\arccsc}{arccsc}
\DeclareMathOperator{\si}{Si}
\DeclareMathOperator{\proj}{proj}
\DeclareMathOperator{\scal}{scal}
\DeclareMathOperator{\cis}{cis}
\DeclareMathOperator{\Arg}{Arg}
%\DeclareMathOperator{\arg}{arg}
\DeclareMathOperator{\Rep}{Re}
\DeclareMathOperator{\Imp}{Im}
\DeclareMathOperator{\sech}{sech}
\DeclareMathOperator{\csch}{csch}
\DeclareMathOperator{\Log}{Log}

\newcommand{\tightoverset}[2]{% for arrow vec
  \mathop{#2}\limits^{\vbox to -.5ex{\kern-0.75ex\hbox{$#1$}\vss}}}
\newcommand{\arrowvec}{\overrightarrow}
\renewcommand{\vec}{\mathbf}
\newcommand{\veci}{{\boldsymbol{\hat{\imath}}}}
\newcommand{\vecj}{{\boldsymbol{\hat{\jmath}}}}
\newcommand{\veck}{{\boldsymbol{\hat{k}}}}
\newcommand{\vecl}{\boldsymbol{\l}}
\newcommand{\utan}{\vec{\hat{t}}}
\newcommand{\unormal}{\vec{\hat{n}}}
\newcommand{\ubinormal}{\vec{\hat{b}}}

\newcommand{\dotp}{\bullet}
\newcommand{\cross}{\boldsymbol\times}
\newcommand{\grad}{\boldsymbol\nabla}
\newcommand{\divergence}{\grad\dotp}
\newcommand{\curl}{\grad\cross}
%% Simple horiz vectors
\renewcommand{\vector}[1]{\left\langle #1\right\rangle}


\outcome{Determine if an infinite series diverges}

\title{3.4 Test for Divergence}

\begin{document}

\begin{abstract}
We determine if an infinite series diverges.
\end{abstract}

\maketitle

\section{Definition of Infinite Series}

\begin{definition}[Infinite Series]
We define
\[\sum_{n=1}^\infty a_n = \lim_{N \to \infty} \sum_{n=1}^N a_n.\]
If this limit is not finite or does not exist, then we say that the infinite series \textbf{diverges}.
If this limit is a finite number, then we say that the infinite series \textbf{converges}.
\end{definition}

\begin{remark}
The sum, $\displaystyle{\sum_{n=1}^N a_n}$, is called the $N^{th}$ partial sum of the infinite series
$\displaystyle{\sum_{n=1}^\infty a_n}$. Thus, in words, we define an infinite series to be the limit of its partial sums.
This is reminiscent of an improper integral which is also defined in terms of a limit.
\end{remark}

\section{Test for Divergence}

In this section, we will learn a simple criterion for the divergence of an infinite series.
The main idea is that in order for an infinite series to converge to a finite value, the terms in the series
must approach zero. We now state this fact in its equivalent \textbf{contrapositive} form.


\begin{theorem}[Test for Divergence]


\[
\text{If}  \; \lim_{n\to \infty} a_n \neq 0,
\; \text{then the series} \;
\sum_{n=1}^\infty a_n \;
\text{diverges.}
\]

\end{theorem}

\begin{remark}
It is important to recognize that if $\displaystyle{\lim_{n \to \infty} a_n = 0}$ then the series, $\displaystyle{\sum_{n=1}^\infty a_n}$, may either converge or diverge.
\end{remark}

\begin{example}[example 1]
Consider the infinite series
\[
\sum_{n=1}^\infty \frac{n+1}{2n+3}.
\]
The degree of the numerator and denominator of $a_n$ are equal (they are both one), so we can use the ratio of the lead coefficients to determine that
\[
\lim_{n\to \infty} \frac{n+1}{2n+3} = \frac12.
\]
Since this limit is not zero, we can conclude that the series $\displaystyle{\sum_{n=1}^\infty \frac{n+1}{2n+3}}$ diverges by the Test for Divergence. 
\end{example}

\begin{problem}(problem 1a)
Consider the infinite series

\[
\sum_{n=1}^\infty \frac{3n+2}{5n+1}
\]
The limit of its terms is
\[
\lim_{n\to \infty} \frac{3n+2}{5n+1} = \answer{3/5}
\]
By the test for divergence, the series
\begin{multipleChoice}
\choice{converges}
\choice[correct]{diverges}
\choice{no conclusion}
\end{multipleChoice}

\end{problem}



\begin{problem}(problem 1b)
Consider the infinite series

\[
\sum_{n=2}^\infty  \ln(n).
\]
The limit of its terms is
\[
\lim_{n\to \infty} \ln(n) = \answer{\infty}
\]
By the test for divergence, the series
\begin{multipleChoice}
\choice{converges}
\choice[correct]{diverges}
\choice{no conclusion}
\end{multipleChoice}

\end{problem}


\begin{problem}(problem 1c)
Consider the infinite series

\[
 \sum_{n=1}^\infty \sin(n).
\]
The limit of its terms is
\begin{hint}
type DNE if the limit does not exist
\end{hint}
\[
\lim_{n\to \infty} \sin(n) = \answer{DNE}
\]
By the test for divergence, the series
\begin{multipleChoice}
\choice{converges}
\choice[correct]{diverges}
\choice{no conclusion}
\end{multipleChoice}

\end{problem}


\begin{problem}(problem 1d)
Consider the infinite series

\[
\sum_{n=1}^\infty \frac{n^2 + 1}{2n^2 + 3n + 5}.
\]
The limit of its terms is
\[
\lim_{n\to \infty} \frac{n^2 + 1}{2n^2 + 3n + 5} = \answer{1/2}
\]
By the test for divergence, the series
\begin{multipleChoice}
\choice{converges}
\choice[correct]{diverges}
\choice{no conclusion}
\end{multipleChoice}

\end{problem}


\begin{problem}(problem 1e)
Consider the infinite series

\[
\sum_{n=1}^\infty \sqrt[n] 2.
\]
The limit of its terms is
\[
\lim_{n\to \infty} \sqrt[n] 2 = \answer{1}
\]
By the test for divergence, the series
\begin{multipleChoice}
\choice{converges}
\choice[correct]{diverges}
\choice{no conclusion}
\end{multipleChoice}

\end{problem}



\begin{example}[example 2]
Consider the infinite series
\[
\sum_{n=1}^\infty \frac{n+1}{2n^2+3}.
\]
The degree of the numerator (one) is less than the degree of the denominator (two) so
\[
\lim_{n\to \infty} \frac{n+1}{2n^2+3} = 0,
\]
Since this limit is zero, the Test for Divergence \textbf{does not apply} to the series, $\displaystyle{\sum_{n=1}^\infty \frac{n+1}{2n^2+3}}$ and 
we cannot determine if the series converges or diverges. 
The Test for Divergence is \textbf{inconclusive} in this case.
 
\end{example}


\begin{problem}(problem 2a)
Consider the infinite series

\[
\sum_{n=1}^\infty a_n = \sum_{n=1}^\infty \frac{3n+2}{5n^2+1}.
\]
The limit of its terms is
\[
\lim_{n\to \infty} \frac{3n+2}{5n^2+1} = \answer{0}
\]
By the test for divergence, the series
\begin{multipleChoice}
\choice{converges}
\choice{diverges}
\choice[correct]{no conclusion}
\end{multipleChoice}

\end{problem}


\begin{problem}(problem 2b)
Consider the infinite series

\[
\sum_{n=2}^\infty  \frac{1}{\ln(n)}.
\]
The limit of its terms is
\[
\lim_{n\to \infty} \frac{1}{\ln(n)} = \answer{0}
\]
By the test for divergence, the series
\begin{multipleChoice}
\choice{converges}
\choice{diverges}
\choice[correct]{no conclusion}
\end{multipleChoice}

\end{problem}




\begin{problem}(problem 2c)
Consider the infinite series

\[
\sum_{n=1}^\infty \frac{n^2 + 1}{2n^4 + 3n^2 + 5}.
\]
The limit of its terms is
\[
\lim_{n\to \infty} \frac{n^2 + 1}{2n^4 + 3n^2 + 5} = \answer{0}
\]
By the test for divergence, the series
\begin{multipleChoice}
\choice{converges}
\choice{diverges}
\choice[correct]{no conclusion}
\end{multipleChoice}

\end{problem}



\begin{problem}(problem 2d)
Consider the infinite series

\[
\sum_{n=1}^\infty \frac{1}{n!}.
\]
The limit of its terms is
\[
\lim_{n\to \infty} \frac{1}{n!} = \answer{0}
\]
By the test for divergence, the series
\begin{multipleChoice}
\choice{converges}
\choice{diverges}
\choice[correct]{no conclusion}
\end{multipleChoice}

\end{problem}


\begin{remark}
Note that the test for divergence can NEVER be used to conclude that an infinite series converges.
\end{remark}


\section{Proof of the Test for Divergence}
To prove the test for divergence, we will show that if $\displaystyle{\sum_{n=1}^\infty a_n}$ converges, then the limit, 
$\displaystyle{\lim_{n \to \infty} a_n}$, must equal zero. The logic is then that if this limit is not zero, 
the associated series cannot converge, and it therefore must diverge. We begin by considering the partial sums of the series, $S_N$.
Since the series converges, the limit of the partial sums exists and equals a finite value, which we will call $L$.
In mathematical symbols, 
\[
\lim_{N \to \infty} S_N = L, 
\]
where
\[
S_N = \sum_{n=1}^N a_n.
\]
To proceed, we make the following two observations:
\[
(1)\qquad \lim_{N \to \infty} S_{N-1} = L
\]
and 
\[
(2) \qquad a_N = S_N - S_{N-1}.
\]
Putting these facts together, we can conclude that
\[
\lim_{N \to \infty} a_N = \lim_{N \to \infty} (S_N - S_{N-1}) = L-L = 0.
\]
In the above limit, there is nothing special about the index $N$ (it was only helpful in dealing with the partial sums, $S_N$) 
and we can change it to $n$ to match statement in the theorem. We have therefore proved the theorem, since if 
\[
\sum_{n=1}^\infty a_n = L \; (\text{i.e., the series converges}),
\]
then 
\[
\lim_{n \to \infty} a_n = 0.
\]
We will wrap up this section by examining the converse of the Test for Divergence.
It was mentioned in the remark following the statement of the theorem above, that if the limit of the terms is zero,
the associated series could either converge or diverge.
We now present two classic series which exemplify this point.
The two series are
\[
(1)  \qquad 1 + \frac12 + \frac14 + \frac18 + \cdots,
\]
and 
\[
(2) \qquad 1 + \frac12 +  \frac13 + \frac14 + \cdots.
\]
In both cases, the limit of the terms is zero, but the first series converges while the second series diverges.
We will explore both of these series in future sections 
(the first in the section on Geometric Series and the second in the section on the Integral Test).





\begin{center}
\begin{foldable}
\unfoldable{Here is a detailed, lecture style video on the Test for Divergence:}
\youtube{Ttq3S0GAtEA}
\end{foldable}
\end{center}





\end{document}


\begin{example} %example #15
Find $h'(x)$ if $h(x) = x^{\sin(x)}$.\\
We will use the fact that the exponential and logarithm functions are inverses,
\[e^{\ln(x)} = x,\]
and the exponent property of logarithms, 
\[\ln(x^n) = n\ln(x),\]
to rewrite $h(x)$.  We have 
\[h(x) = x^{\sin(x)} = e^{\ln(x^{\sin(x)})} = e^{\sin(x)\ln(x)}\]
and we can now compute $h'(x)$ using a combination of the chain rule and product rule.
We can write $h(x)$ as a composition, $f(g(x))$ with 
\[g(x) = \sin(x)\ln(x) \quad \text{and} \quad f(x) = e^x.\]
Then to find $g'(x)$ we us the product rule and we get $g'(x) = \frac{\sin(x)}{x} + \cos(x)\ln(x)$.
Next $f'(x) = e^x$ and 
hence $f'(g(x)) = e^{g(x)} = e^{\sin(x)\ln(x)} = x^{\sin(x)}$.
We can then conclude $h'(x) = f'(g(x))g'(x) = x^{\sin(x)} \left[ \frac{\sin(x)}{x} + \cos(x)\ln(x)\right]$.
\end{example}

%more question formats below













%\begin{verbatim}
\begin{question}
What is your favorite color?
\begin{multipleChoice}
\choice[correct]{Rainbow}
\choice{Blue}
\choice{Green}
\choice{Red}
\end{multipleChoice}
\begin{freeResponse}
Hello
\end{freeResponse}
\end{question}
%\end{verbatim}





\begin{question}
  Which one will you choose?
  \begin{multipleChoice}
    \choice[correct]{I'm correct.}
    \choice{I'm wrong.}
    \choice{I'm wrong too.}
  \end{multipleChoice}
\end{question}


\begin{question}
  Which one will you choose?
  \begin{selectAll}
    \choice[correct]{I'm correct.}
    \choice{I'm wrong.}
    \choice[correct]{I'm also correct.}
    \choice{I'm wrong too.}
  \end{selectAll}
\end{question}


\begin{freeResponse}
What is the chain rule used for?
\end{freeResponse}
